\documentclass[10pt]{amsart}
\usepackage{amsmath,amsthm,amssymb,amsfonts,mymath}
\openup 5pt
\author{Blake Farman}
\title{Math-252:\\Exam 2}
\date{April 18, 2011}
\pdfpagewidth 8.5in
\pdfpageheight 11in
\usepackage[margin=1in]{geometry}
\begin{document}
\maketitle

\newtheorem{thm}{}

\begin{thm}
  \label{Ex1}
  The matrices 
  $$
  A = \begin{pmatrix}
    8 & -17 & -25\\
    -4 & 9 & 16\\
    5 & -11 & -18
  \end{pmatrix}
  \quad \text{and} \quad
  B= \begin{pmatrix}
    3 & -2 & 2\\
    0 & -1 & 0\\
    -4 & 2 & -3
  \end{pmatrix}
  $$
  have the same characteristic polynomial $(x-1)(x+1)^2$.
  Determine the rational and JOrdan canonical forms for both $A$ and $B$, whether $A$ and $B$ are similar, and whether the matrices can be daigonalized.
  \begin{proof}
    It is an easy check by hand to calculate $m_A(x) = (x-1)(x+1)^2 = x^3 + x^2 - x - 1$ and $m_B(x) = (x-1)(x+1) = x^2 - 1$.
    Hence the invariant factors for $A$ and $B$ are $x^3 + x^2 - x - 1$ and $x^2-1, x+1$, respectively.
    Thus the rational canonical forms for $A$ and $B$ are
    $$
    \begin{pmatrix}
      0 & 0  & 1\\
      1 & 0 & 1\\
      0 & 1 & -1
    \end{pmatrix}
    \quad \text{and} \quad
    \begin{pmatrix}
      -1 & 0 & 0\\
      0 & 0 & 1\\
      0 & 1 & 0
    \end{pmatrix},
    $$
    respectively.
    The elementary divisors for $A$ and $B$ are $x-1,(x+1)^2$ and $x-1, x+1, x+1$, respectively.
    Thus the Jordan canonical forms for $A$ and $B$ are 
    $$
    \begin{pmatrix}
      1 & 0 & 0\\
      0 & -1 & 1\\
      0 & 0 & -1
    \end{pmatrix}
    \quad \text{and} \quad
    \begin{pmatrix}
      1 & 0 & 0\\
      0 & -1 & 0\\
      0 & 0 & -1
    \end{pmatrix},
    $$
    respectively.
    Since the canonical forms are not the same, $A$ and $B$ are not similar.
    Moreover, $B$ is the only matrix of the two which can be diagonalized, since its Jordan canonical form is diagonal.
  \end{proof}
\end{thm}

\begin{thm}
  \label{Ex2}
  Determine the Jordan canonical form for the matrix
  $$
  \begin{pmatrix}
    1 & 2 & 0 & 0\\
    0 & 1 & 2 & 0\\
    0 & 0 & 1 & 2\\
    0 & 0 & 0 & 1
  \end{pmatrix}.
  $$
  \begin{proof}
    By simply subtracting twice the fourth fourth row from the third, then twice the third from the second and finally twice the second from the first we obtain the diagonal matrix
    $$
    \begin{pmatrix}
      1 & 0 & 0 & 0\\
      0 & 1 & 0 & 0\\
      0 & 0 & 1 & 0\\
      0 & 0 & 0 & 1
    \end{pmatrix}.
    $$
    Since this diagonal matrix is similar to the one we started with, it must be the Jordan canonical form.
  \end{proof}
\end{thm}

\begin{thm}
  \label{Ex3}
  Suppose $A$ is an $8 \times 8$ matrix with rational entries and 
  \begin{enumerate}
    \renewcommand{\theenumi}{\roman{enumi}.}
    \renewcommand{\labelenumi}{\theenumi}
  \item
    The row reduced form of $A + I$ has 5 non-zero rows,
  \item
    The row reduced form of $(A+I)^2$ has 2 non-zero rows,
  \item
    The row reduced form of $(A+I)^3$ has 1 non-zero row, and
  \item
    $(A+I)^4 = 0$.
    Find the Jordan canonical form for $A$.
  \end{enumerate}
  \begin{proof}
    If $(A+I)^4 = 0$, then $m_A(x) \mid (x+1)^4$.
    Since neither $A+I$, $(A+I)^2$, nor $(A+I)^3$ are zero we have that $m_A(x) = (x+1)^4$ and thus $c_A(x) = (x+1)^8$.
    Hence the Jordan canonical form has eight diagonal entries, all $-1$.
    Let $P$ be the matrix such that the Jordan canonical form of $A$ is given by $PAP^{-1}$.
    Using the fact that $A+I$ has five non-zero rows and so must $PAP^{-1} +I = P(A+I)P^{-1}$, it is clear that two of the elementary divisors must have degree 2.
    Hence the elementary divisors $(x+1)^4, (x+1)^2, (x+1)^2$ give rise to the Jordan canonical form
    $$
    \begin{pmatrix}
      -1 & 1 & 0 & 0 & 0 & 0 & 0 & 0\\
      0 & - 1 & 1 & 0 & 0 & 0 & 0 & 0\\
      0 & 0 & -1 & 1 & 0 & 0 & 0 & 0\\
      0 & 0 & 0 & -1 & 0 & 0 & 0 & 0\\
      0 & 0 & 0 & 0  & -1 & 1 & 0 & 0\\
      0 & 0 & 0 & 0  & 0 & -1 & 0 & 0\\
      0 & 0 & 0 & 0  & 0 & 0 & -1 & 1\\
      0 & 0 & 0 & 0  & 0 & 0 & 0 & -1\\
    \end{pmatrix}
    $$
  \end{proof}
\end{thm}

\begin{thm}
  \label{Ex4}
  Determine whether there are any $3\times3$ matrices $A$ with entries from $\Q$ of (multiplicative) order 8 (i.e., $A^8 = I$ but $A^4 \neq I$) and, if there are, determine how many such matrices there are up to similarity, with an example of each.
  \begin{proof}
    By the result of exercise 24, section 12.3, there are no $3\times3$ matrices $A$ over $\Q$ with $A^8 = I$ but $A^4 \neq I$.
  \end{proof}
\end{thm}

\begin{thm}
  \label{Ex5}
  Suppose $F$ is a field and $\alpha$ is an algebraic element over $F$ with $[F(\alpha):F]$ odd.
  Prove that $F(\alpha) = F(\alpha^2)$.
  
  \begin{proof}
    It is clear that $F(\alpha^2) \subseteq F(\alpha)$.  
    Hence it suffices to show the reverse inclusion.
    To that end, let $2k+1 = [F(\alpha):F]$, $m_{\alpha,F}(x) = x^{2k+1} + a_{2k}x^{2k} + \ldots + a_1x + a_0$ and consider the minimal polynomial over $F(\alpha^2)$. 
    Since the field now contains the element $\alpha^2$, we can rewrite the minimal polynomial as
    $$m_{\alpha,F(\alpha^2)}(x) = \alpha^{2k}x + a_{2k}\alpha^{2k} + \ldots + a_2\alpha^2 + a_1x + a_0.$$
    Collecting like terms, we see that $\alpha$ is a root of a degree 1 polynomial over $F(\alpha^2)$.
    Therefore we have the reverse containment $F(\alpha) \subseteq F(\alpha^2)$, completing the proof.
  \end{proof}
\end{thm}

\begin{thm}
  \label{Ex6}
  Suppose $f(x) \in \Q[x]$ is an irreducible polynomial of degree 5 and let $\alpha$ be a root of $f(x)$.
  \begin{alphaenum}
    \item
      Determine $[\Q(\alpha):\Q]$.
    \item
      Prove that $f(x)$ is still irreducible over $F = \Q\left(\sqrt{2},\sqrt{3}\right)$, i.e. is still an irreducible when viewed in $F[x]$.
  \end{alphaenum}
  
  \begin{proof}
    \begin{alphaenum}
    \item
      Since $\alpha$ is the root of an irreducible degree 5 polynomial, $[\Q(\alpha): \Q] = 5$.
    \item
      Let $K = \Q(\alpha)$.
      As we have proven, $[\Q\left(\sqrt{2},\sqrt{3}\right):\Q] = 4$.
      Consider the composite field $KF = F(\alpha) \cong F/(f(x))$.
      Since $(4,5) = 1$, it follows from Corollary 22 that $[KF:\Q] = 20$ and $[F(\alpha):F] = 5$.
      Therefore $f(x)$ is an irreducible polynomial of degree 5 over $F$.
    \end{alphaenum}
  \end{proof}
\end{thm}

\begin{thm}
  \label{Ex7}
  Show that $\Q\left(\sqrt{2 + \sqrt{2}}\right)$ is a cyclic quartic field, i.e., is a Galois extension of degree 4 with cyclic Galois group.
  \begin{proof}
    First, observe that 
    $$f(x) = x^4 - 4x^2 -2 = \left(x+\sqrt{2 + \sqrt{2}}\right)\left(x-\sqrt{2 + \sqrt{2}}\right)\left(x+\sqrt{2 - \sqrt{2}}\right)\left(x-\sqrt{2 - \sqrt{2}}\right)$$ 
    is irreducible over $\Q$ by Eisenstein's criterion using the prime 2.
    Hence the field $K = \Q\left(\sqrt{2 + \sqrt{2}}, \sqrt{2 - \sqrt{2}}\right)$ is a splitting field for $f$ and thus Galois.
    Let $D_1 =2 + \sqrt{2}, D_2 = 2 - \sqrt{2}$ and let $\alpha_1 = \sqrt{D_1}$ and $\alpha_2 = \sqrt{D_2}$.
    Observe that $\sqrt{2}$ is an element of both $K_1 = \Q(\alpha_1)$ and $K_2 = \Q(\alpha_2)$, and neither $D_1$ nor $D_2$ is a square in $L = Q\left(\sqrt{2}\right)$.
    So, we consider $K_1/L$ and $K_2/L$.
    Since $D_1D_2 = 2$ is a square in $L$, we have $[K:L] = 2$ and thus $K = L(\alpha_1) = K_1/L$ by the proof Exercise 8 of Section 13.2.
    This proves $K_1$ is Galois of degree 4.
    Consider an automorphism that sends $\sqrt{2}$ to $-\sqrt{2}$,
    $$
    \sigma: \begin{cases}
      \alpha_1 &\longmapsto -\alpha_2\\
      \alpha_2 &\longmapsto \alpha_1
    \end{cases}
    $$
    and note that $\sigma$ has order 4.
    Hence $\Gal(K_1/\Q)$ is not the Klein 4-group.
    Therefore $\Gal(K_1/\Q) \cong Z_4$
  \end{proof}
\end{thm}

I certify that I have neither given nor received any assistance on this exam.

\vspace{.75in}

Blake Farman
\end{document}
