\documentclass[10pt]{amsart}
\usepackage{amsmath,amsthm,amssymb,amsfonts,mymath}
\openup 5pt
\author{Blake Farman}
\title{Math-252:\\Homework 7}
\date{March 30, 2011}
\pdfpagewidth 8.5in
\pdfpageheight 11in
\usepackage[margin=1in]{geometry}
\begin{document}
\maketitle

\newtheorem{thm}{}

\begin{thm}
  \label{Ex1}
  Show that $p(x) = x^3 + 9x + 6$ is irreducible in $\Q[x]$.
  Let $\theta$ be a root of $p(x)$.
  Find the inverse of $1 + \theta$ in $\Q(\theta)$.
  \begin{proof}
    Since 3 divides both 9 and 6, but no higher power of 3 divides 6, $p$ is irreducible in $\Q[x]$ by Eisenstein's Criterion.
    With the extended Euclidean Algorithm and some routine algebra, it is easy to compute
    $$(1+x)(x^2 - x + 10) - (x^3 + 9x + 6)  = 4.$$
    Therefore the inverse of $1 + \theta$ is $(1/4)(\theta^2 - \theta + 10)$.
  \end{proof}
\end{thm}

\begin{thm}
  \label{Ex2}
  Let $F$ be a field of characteristic $\neq 2$.
  Let $D_1, D_2$ be elements of $F$, neither of which is a square in $F$.
  Prove that $F(\sqrt{D_1}, \sqrt{D_2})$ is of degree 4 over $F$ if $D_1D_2$ is not a square in $F$ and is of degree 2 over $F$ otherwise.
  When $F(\sqrt{D_1}, \sqrt{D_2})$ is of degree 4 over $F$ the field is called a biquadratic extension of $F$.

  \begin{proof}
    %Since both $D_1$ and $D_2$ were assumed not to be squares and we are working in a field it suffices to assume $D_1$ and $D_2$ are square-free.
    Seeing as $D_1$ and $D_2$ are both square-free up to a unit that will not affect the resulting field extensions, it suffices to assume they are both square-free. 
    %Let $D_1 = p_1\ldots p_r$ and $D_2 = q_1\ldots q_s$ be the prime factorizations of $D_1$ and $D_2$ in $F$.
    %Note that $p_i \neq p_j$ and $q_i \neq q_j$ holds for each $i \neq j$.
    Hence if $D_1D_2$ is a square, then it follows immediately that $D_1 = D_2$.
    Whence $[F(\sqrt{D_1},\sqrt{D_2}):F] = 2$.
    
    % Similarly, if $D_1D_2$ is not a square, then $p_i \neq q_j$ for at least one pair $i,j$.
    % Let $d = (D_1,D_1)$ and rearrange the factorizations so that $D_1 = dp_1p_2\ldots p_u$ and $D_2 = dq_1 q_2 \ldots q_v$, where $u < r$ and $v < s$.
    If $D_1D_2$ is not a square, then $D_1 \neq D_2$ and so it suffices to show that $\sqrt{D_2} \not \in F(\sqrt{D_1})$.
    Assume to the contrary that $\sqrt{D_2} \in F(\sqrt{D_1})$ and write $\sqrt{D_2} = a + b\sqrt{D_1}$ for some $a,b \in F$.
    Squaring both sides yields $D_2 = a^2 + b^2D_1 + 2ab\sqrt{D_2}$.
    Since $\{1, \sqrt{D_1}\}$ is a basis, we have the system $D_2 = a^2 + b^2D_1$ and $2ab\sqrt{D_2} = 0$.
    The characteristic of $F$ was assumed not to be 2, hence one of $a$ or $b$ must be zero.
    Noting that $D_2$ was assumed to be square-free allows us to reduce to $a = 0$ and $D_2 = b^2D_1$ .
    However, this is equally absurd and thus $\sqrt{D_2} \not \in F(\sqrt{D_1})$.
    Therefore, $F(\sqrt{D_1},\sqrt{D_2})$ has degree 4 over $F$.
    
  \end{proof}
\end{thm}

\begin{thm}
  \label{Ex3}
  Determine the degree of the extension $\Q(\sqrt{3 + 2\sqrt{2}})$ over $\Q$.
  
  \begin{proof}
    It is easy to compute $1 + \sqrt{2} = \sqrt{3 + 2\sqrt{2}}$, so it follows that $\Q(\sqrt{3 + 2\sqrt{2}}) = \Q(\sqrt{2})$.
    Therefore $[\Q(\sqrt{3 + 2\sqrt{2}}):\Q] = 2$.
  \end{proof}

  \begin{alphaenum}
    \item
      Let $\sqrt{3 + 4i}$ denote the square root of the complex number $3 + 4i$ that lies in the first quadrant and let $\sqrt{3 - 4i}$ denote the square root of $3 - 4i$ that lies in the fourth quadrant.
      Prove that $[\Q(\sqrt{3 + 4i} + \sqrt{3 - 4i}): \Q] = 1$.
    \item
      Determine the degree of the extension $\Q(\sqrt{1 + \sqrt{-3}} + \sqrt{1 - \sqrt{-3}})$ over $\Q$.
  \end{alphaenum}
  
  \begin{proof}
    \begin{alphaenum}
    \item
      Since $3 + 4i$ and $3 - 4i$ are complex conjugates, so too must their roots be.
      Hence $\sqrt{3 + 4i} + \sqrt{3 - 4i} = 2\operatorname{Re}{\sqrt{3+4i}} = 4$ has a degree 1 minimal polynomial.
      Therefore $[\Q(\sqrt{3 + 4i} + \sqrt{3 - 4i}): \Q] = 1$.
    \item
      Write $1 \pm \sqrt{-3} = 1 \pm i\sqrt{3}$ and it is then easy to compute $\sqrt{1 \pm \sqrt{-3}} = 2e^{\pm i\pi/6}$.
      It then follows that their sum is given by
      $$4\left(\frac{e^{i\pi/6} + e^{-i\pi/6}}{2}\right) = 4\cos(\pi/6) = 2\sqrt{3}.$$
      Therefore $[\Q(\sqrt{1 + \sqrt{-3}} + \sqrt{1 - \sqrt{-3}}):\Q] = 2$.
    \end{alphaenum}
  \end{proof}
\end{thm}

\begin{thm}
  \label{Ex4}
  Suppose $F = \Q(\alpha_1, \alpha_2, \ldots, \alpha_n)$ where $\alpha_i^2 \in \Q$ for $i = 1, 2, \ldots, n$.
  Prove that $\sqrt[3]{2} \not \in F$
  
  \begin{proof}
    Recursively define the extension fields $F_0 = \Q, F_{i+1} = F_i(\alpha_{i+1})$.
    Since  $\alpha_i$ satisfies the polynomial $x^2 - \alpha_i^2 \in \Q[x]$ for each $\alpha_i$, it follows that 
    $$[F:\Q] = \prod_{i=0}^{n-1} [F_{i+1} : F_i] \leq 2^n$$
    must be a power of 2.
    Since $x^3 - 2$ is the minimal polynomial of $\sqrt[3]{2}$, $[\Q(\sqrt[3]{2}):\Q] = 3$.
    Seeing as 2 and 3 are coprime, it follows from Theorem 14 that $\Q(\sqrt[3]{2}) \not \subset F$.
    Therefore, $\sqrt[3]{2} \not \in F.$
    
  \end{proof}
\end{thm}

\begin{thm}
  \label{Ex5}
  \newcommand{\BB}{\mathcal{B}}
  Let $f(x)$ be an irreducible polynomial of degree $n$ over a field $F$.
  Let $g(x)$ be any polynomial in $F[x]$.
  Prove that every irreducible factor of the composite polynomial $f \circ g$ has degree divisible by $n$.
  \begin{proof}
    Let $f \circ g = p_1p_2\ldots p_r$ be the prime factorization of the composition in $F[x]$.
    Consider a root, $\beta$, of an arbitrary $p_i$ and the extension $F(\beta)$ with degree equal to that of $p_i$.
    %, has a basis $\BB = \{1, \beta, \ldots, \beta^{m-1}\}$.
    Clearly $\beta$ is a root of $f \circ g$, so it follows that $\alpha = g(\beta)$ is a root of $f$.
    Moreover, $\alpha$, an $F$-linear combination of powers of $\beta$, is an element of $F(\beta)$.
    % . and so it which can be rewritten as an $F$-linear combination of elements of $\BB$.
    Hence $F(\alpha)$, a degree $n$ extension of $F$, is a subfield of $F(\beta)$.
    Therefore $n$ divides $\deg(p_i)$ holds for each $i$ by Theorem 14. 
    %Seeing as the choice of $p_i$ and $\beta$ were arbitrary, the result holds for each $i$.
  \end{proof}
\end{thm}

\begin{thm}
  \label{Ex6}
  Prove that it is impossible to construct the regular 9-gon
  \begin{proof}
    A regular 9-gon can be constructed using the $9^{th}$ roots of unity as the vertices.  
    Seeing as $\Q(\zeta_9)$ has degree $\varphi(9) = 6$ over $\Q$, the 9-gon is not constructible by Theorem 24.
  \end{proof}
  
  The construction of the regular 7-gon amounts to the constructability of $\cos(2\pi/7)$.  
  We shall see later that $\alpha = 2\cos(2\pi/7)$ satisfies the equation $x^3 + x^2 - 2x -1 = 0$.
  Use this to prove that the regular 7-gon is not constructible by straightedge and compass.
  
  \begin{proof}
    Observe that $x^3 + x^2 - 2x -1 \equiv x^2 + x^2 +1 \pmod{2}$ is irreducible over $\mathbb{F}_2$ and thus irreducible over $\Q$.
    Therefore $\Q(\alpha)$ has degree 3 over $\Q$ and by Theorem 24 the 7-gon is not constructible.
  \end{proof}

  Use the fact that $\alpha = 2 \cos(2\pi/5)$ satisfies the equation $x^2 + x - 1 = 0$ to conclude that the regular 5-gon is constructible by straightedge and compass.
  
  \begin{proof}
    Since $x^2 + x - 1$ is irreducible over $\mathbb{F}_2$, it is irreducible over $\Q$ and the extension $\Q(\alpha)$ has degree 2.
    Therefore the regular 5-gon is constructible by Theorem 24.
  \end{proof}
\end{thm}
\end{document}
