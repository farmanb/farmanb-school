\documentclass[10pt]{amsart}
\usepackage{amsmath,amsthm,amssymb,amsfonts}
\openup 5pt
\author{Blake Farman}
\title{Math-252:\\Homework 2}
\date{February 8, 2010}\pdfpagewidth 8.5in
\usepackage[margin=1in]{geometry}
\pdfpageheight 11in
\begin{document}
\maketitle

\newcommand{\Z}{\mathbb{Z}}
\newcommand{\R}{\mathbb{R}}
\newcommand{\Q}{\mathbb{Q}}

\renewcommand{\qedsymbol}{\(\blacksquare\)}
\newcommand{\znz}[1]{\Z / #1\Z}
\newcommand{\mznz}[1]{(\Z / #1\Z)^*}

\renewcommand{\phi}{\varphi}
\newenvironment{alphaenum}{
  \begin{enumerate}
    \renewcommand{\theenumi}{(\alph{enumi})}
    \renewcommand{\labelenumi}{\theenumi}
  }
  {\end{enumerate}}

\newcommand{\quadeq}[3]{\frac{-(#2) \pm \sqrt{(#2)^2 - 4(#1)(#3)}}{2(#3)}}
% \newcommand{quadeq}[3]{}

\newtheorem{thm}{}
% \newtheorem*{2}{2}
% \newtheorem*{3}{3}
% \newtheorem*{4}{4}
% \newtheorem*{5}{5}
% \newtheorem*{6}{6}
\begin{thm}
  \label{Ex1}
  Let $R$ be the quadratic integer ring \(\Z[\sqrt{-5}].\)
  Define the ideals \(I_2 = (2, 1 + \sqrt{-5}), I_3 = (3, 2 + \sqrt{-5})\) and \(I_3' = (3, 2 - \sqrt{-5})\).


  (a) Prove that \(I_2, I_3\) and \(I_3'\) are non-principal ideals in R.
  
  (b) Prove that the product of two non-principal ideals can be principal by showing that \(I_2^2\) is the principal ideal generated by 2, i.e. \(I_2^2 = (2)\).

  (c) Prove similarly that \(I_2I_3 = (1  - \sqrt{-5})\) and \(I_2I_3' = (1 + \sqrt{-5})\) are princpal.  
  Conclude that the principal ideal \((6)\) is the product of 4 ideals: \((6) = I_2^2I_3I_3'\).
  \begin{proof}
    (a) Suppose \(I_2 = (a + b \sqrt{-5})\) for some \(a,b \in \Z\).
    Then \(2 = \alpha(a + b\sqrt{-5})\) and \(1 + \sqrt{-5} = \beta(a + b\sqrt{-5})\) for some \(\alpha, \beta \in R\).
    Taking the norm of both sides of the former yields \[4 = N(a)(a^2 + 5b^2)\]
    so \(a^2 + 5b^2 = 1, 2 \text{ or } 4.\)  If \(a^2 + 5b^2 = 4\), then \(a + b\sqrt{-5} = \pm 2\), but \(1 + \sqrt{-5}\) is not divisible by 2 and thus this cannot be.
    Moreover \(a^2 + 5b^2 = 2\) does not have a solution in the integers, so the only possiblity is \(a^2 + 5b^2 = 1,\) which implies \((a + b\sqrt{-5}) = R\).
    If this is the case, then there must exist elements \(\gamma, \delta\) of \(R\) such that \[2\gamma + (1 + \sqrt{-5})\delta = 1.\]
    Multiplying both sides of the equation by $1 - \sqrt{-5}$ yields 
    \begin{equation}
      \label{1}
      2(\gamma(1-\sqrt{-5}) + 3\delta) = 1 - \sqrt{-5}.
    \end{equation}
    However, the right hand side of \eqref{1} is not divisible by \(2\).  
    Therefore \(I_2\) is not principal. 
    
    Suppose \(I_3' = (a + b\sqrt{-5})\) for some \(a,b \in \Z\) so that there exist elements \(\alpha, \beta\) of \(R\) such that \[3 = \alpha(a + b\sqrt{-5}) \quad \text{and} \quad 2 - \sqrt{-5} = \beta(a + b\sqrt{-5}).\]
    Taking the norm of both sides of the former yields \[9 = N(\alpha)(a^2 + 5b^2).\]
    The only possibilities are then \(a^2 + 5b^2 = 1, 3 \) or \(9\).
    As in the case of \(I_3\), \(a^2 + 5b^2\) cannot equal 3 or 9, so it must be 1.
    Hence there exist elements \(\gamma, \delta\) of \(R\) such that \[3\gamma + (2 - \sqrt{-5})\delta = 1.\]
    Multiplying both sides by \(2 + \sqrt{-5}\) it follows that
    \begin{equation}
      \label{2}
      3(\gamma(2+\sqrt{-5}) + 3\delta) = 2 + \sqrt{-5},
    \end{equation}
    however the right hand side of \eqref{2} is not divisible by 3.  Therefore \(I_3'\) is not principal.

    (b) By definition \(I_2^2 = \left\{ \sum_{k=1}^{n} a_kb_k\mid a_k,b_k \in I_2, n < \infty \right\}\), so it suffices to show that for any \(a,b \in I_2\) the product \(ab\) can always be written as \(ab = 2c\) for some \(c \in R\).
    If \(a,b \in I_2\), then there exist elements \(\alpha, \beta, \gamma\) and \(\delta\) of R such that \[a = 2\alpha + (1 + \sqrt{-5})\beta \quad \text{and} \quad b = 2\gamma + (1 + \sqrt{-5})\delta.\]
    Hence their product can be written as 
    \begin{align*}
      \begin{split}
        ab &= 4\alpha\gamma + 2(1 + \sqrt{-5})\alpha\delta + 2(1 + \sqrt{-5})\beta\delta + 6\beta\delta\\
        & = 2(2\alpha\gamma + (1 + \sqrt{-5})\alpha\delta + (1 + \sqrt{-5})\beta\delta + 3\beta\delta).
      \end{split}
    \end{align*}
    Therefore \(I_2^2 = (2)\), as desired.

    (c) As in the previous proof, it suffices to show that if \(a \in I_2\) and \(b \in I_3\), then \(ab = c(1 - \sqrt{-5})\) for some \(c \in R\).
    Write \(a = 2\alpha + \beta(1 + \sqrt{-5})\) and \(b = 3\gamma + \delta(2 + \sqrt{-5})\) for some \(\alpha, \beta, \gamma, \delta \in R\).
    Then with some simple algebra their product can be written as
    \begin{align*}
      \begin{split}
        ab &= 6\alpha\gamma + 2(2 + \sqrt{-5})\alpha\delta + 3(1 + \sqrt{-5})\beta\gamma - 3(1-\sqrt{-5})\beta\delta\\
        &= (1 - \sqrt{-5})[(1 + \sqrt{-5})\alpha\gamma - (1-\sqrt{-5})\alpha\delta - (2 - \sqrt{-5})\beta\gamma - 3\beta\delta].\\
      \end{split}
    \end{align*}
    Therefore \(I_2I_3 = (1 - \sqrt{-5})\).

    
    Now let \(a = 2\alpha + \beta(1 + \sqrt{-5}) \in I_2\) and \(b  = 3\gamma + \delta(2 - \sqrt{-5} \in I_3'\) be given.
    Then their product can be written as 
    \begin{align*}
      \begin{split}
        ab &= 6\alpha\gamma + 2(2 - \sqrt{-5})\alpha\delta + (1 = \sqrt{-5})[3\beta\gamma + (2 - \sqrt{-5})\beta\delta]\\
        &= (1 + \sqrt{-5})[(1 - \sqrt{-5})\alpha\gamma - (1+\sqrt{-5})\alpha\delta + 3\beta\gamma + (2 - \sqrt{-5})\beta\delta].
      \end{split}
    \end{align*}
    Therefore \(I_2I_3' = (1 + \sqrt{-5})\).

    Finally, for any \(a,b \in I_2, c \in I_3\) and \(d \in I_3'\) consider the product \(abcd\).
    Since the ring is commutative, we have \(abcd = acbd\) and by the previous results we can always write \[ac = \alpha(1 - \sqrt{-5}) \quad \text{and} \quad bd = \beta(1 + \sqrt{-5})\] for some \(\alpha, \beta \in R\).
    Hence \(abcd = 6\alpha\beta\) implies that it is always possible to factor a \(6\) from any element of \(I_2^2I_3I_3'\).
    Therefore \(I_2^2I_3I_3' = (6)\), as desired.
  \end{proof}
\end{thm}

\begin{thm}
  \label{Ex2}
  Let $G = Q^*$ be the multiplicative group of non-zero rational numbers.  
  If $\alpha = \frac{p}{q} \in G$, where $p$ and $q$ are relatively prime integers, let $\phi: G \longrightarrow G$ be the map which interchanges the primes $2$ and $3$ in the prime power factorization of $p$ and $q$.
  % \begin{enumerate}
  %   \renewcommand{\theenumi}{(\alph{enumi})}
  %   \renewcommand{\labelenumi}{\theenumi}
  \begin{alphaenum}
  \item Prove that $\phi$ is a group isomorphism.
  \item Prove that there are infinitely many isomorphisms of the group $G$ to itself.
  \item Prove that none of the isomorphisms above can be extended to an isomorphism of the ring $\Q$ to itself. In fact prove that the identity map is the only ring isomorphism of $\Q$.
  \end{alphaenum}
  \begin{proof}
    \begin{alphaenum}
    \item
      Let $g_1,g_2 \in G$ be given and suppose they have the prime factorizations $g_1 = 2^{\alpha_1}3^{\alpha_2}m$ and $g_2 = 2^{\beta_1}3^{\beta_2}n$, where $\alpha_i, \beta_j$ are (possibly negative) integers and $m, n$ are the rational parts of $g_1$ and $g_2$ with numerators and denominators relatively prime to 2 and 3, respectively.
      To see $\phi$ is an homomorphism apply $\phi$ to obtain
      \begin{align*}
        \begin{split}
          \phi(g_1g_2) &= \phi(2^{\alpha_1+\beta_1}3^{\alpha_2+\beta_2}mn)\\
          &= 2^{\alpha_2+\beta_2}3^{\alpha_1+\beta_1}mn\\
          & = 2^{\alpha_2}3^{\alpha_1}m2^{\beta_2}3^{\beta_1}n\\
          & = \phi(g_1)\phi(g_2).
        \end{split}
      \end{align*}
      Moreover $\phi$ is invertible.
      Indeed $\phi$ is its own inverse, which can be seen by applying the composition to $g_1$, $$(\phi \circ \phi)(g_1) = \phi(2^{\alpha_2}3^{\alpha_1}m) = 2^{\alpha_1}3^{\alpha_2}m = g_1.$$  
      Therefore $\phi$ is an automorphism.
    \item
      Let $\alpha$ be as above and fix an odd prime, $p$.  
      Define the map $\psi_p: G \longrightarrow G$ to be the map which interchanges the primes 2 and $p$ in the prime power factorization of $p$ and $q$.  
      By the same argument above, mutatis mutandis, $\psi_p$ is an automorphism of $G$.
      Since there are infinitely many primes, there are infinitely many automorphisms of $G$.
    \item
      First we show that none of the group automorphisms above can be extended to ring automorphisms of $\Q$.
      Fix any prime $p$ and apply it to sum of the elements $2^2$ and $p$ of $\Q$.
      Since neither $p$ nor $2$ divide their sum, we have that $\psi_p(2^2 + p) = 2^2 + p$.
      However, $\psi_p(2^2) + \psi_p(p) = p^2 + 2 \not = 2^2 + p$.
      Therefore $\psi_p$ is not a ring homomorphism.
      
      
      Assume $\phi: \Q \longrightarrow \Q$ is an automorphism.
      Let $\frac{r}{s} \in \Q$ be given and write $r = \sum_{k=1}^r \pm 1$.
      Since $\phi$ is assumed to be an automorphism, we know $\phi$ fixes $\pm1$ and thus 
      \begin{equation}
        \label{ints}
        \phi(r) = \sum_{k=1}^r \phi(\pm1) = r.
      \end{equation}
      Moreover, observe that by \eqref{ints} and the properties of automorphisms, 
      \begin{equation}
        \label{rats}
        1 = \phi(\frac{s}{s}) = s\phi(\frac{1}{s}).
      \end{equation}
      Dividing both sides by $s$ it is clear that $\phi(\frac{1}{s}) = \frac{1}{s}.$ 
      Hence by \eqref{ints} and \eqref{rats}, $$\phi(\frac{r}{s}) = \phi(r)\phi(\frac{1}{s}) = \frac{r}{s}.$$
      Therefore $\phi$ is the identity map, as desired.
    \end{alphaenum}
  \end{proof}
\end{thm}

\begin{thm}
  \begin{alphaenum}
  \item
    Determine all the representations of the integer $2130797 = 17^2 \cdot 73 \cdot 101$ as a sum of two squares.
  \item
    Prove that if an integer is the sum of two rational squares, then it is the sum of two integer squares.
  \end{alphaenum}
  \begin{proof}
    \begin{alphaenum}
    \item
      Since each of the primes in the factorization of $2130797$ are congruent to $1$ modulo $4$, each can be factored in the Gaussian Integers.
      Their factorizations are 
      \begin{align*}
        17 = (4+i)(4-i), \quad  73 = (8+3i)(8-3i) \quad \text{and} \quad 101 = (10+i)(10-i).\\
      \end{align*}
      So combining these gives the following factorization of $2130797$ (excluding their conjugates) in the Gaussian integers, up to units,
      \begin{align*}
        \begin{split}
          (4+i)^2(8+3i)(10+i) =  851 + i1186,& \quad (4+i)^2(8+3i)(10-i) = 1069 + i994,\\
          (4+i)^2(8-3i)(10+i) = 1421 + i334,& \quad (4+i)^2(8-3i)(10-i) = 1459 + i46,\\
          (4+i)(4-i)(8+3i)(10+i) = 1309 + i646,& \quad (4+i)(4-i)(8+3i)(10-i) = 1411 + i374.\\
        \end{split}
      \end{align*}
      Hence $2130797$ can be written as the following sums of squares
      \begin{align*}
        \begin{split}
          (\pm851)^2 + (\pm1186)^2,& \quad  (\pm1069)^2 + (\pm994)^2,\\
          (\pm1421)^2 + (\pm334)^2,& \quad  (\pm1459)^2 + (\pm46)^2,\\
          (\pm1309)^2 + (\pm646)^2,& \quad  (\pm1411)^2 + (\pm374)^2.\\
        \end{split}
      \end{align*}
    \item
      Let $n \in \Z$ be given.
      If $n$ can be written as the sum of two rational squares, $\frac{a}{b}$ and $\frac{c}{d}$, then $n(bd)^2 = (ad)^2 + (bc)^2$.
      Hence it follows from Corollary 19, that $n(bd)^2$ factors as $2^kp_1^{a_1} \ldots p_r^{a_r}q_1^{b_1} \ldots q_s^{b_s}$, where $p_i \equiv 1 \pmod{4}$,  $q_j \equiv 3 \pmod{4}$ and $b_k \equiv 0 \pmod{2}$ hold for each $i, j \text{ and } k$.
      If we write the prime factors of $bd$ and $n$ congruent to 3 modulo 4 as $q_1^{\alpha_1} \ldots q_s^{\alpha_s}$ and $q_1^{\beta_1} \ldots q_s^{\beta_s}$, respectively, then we can rewrite the factorization of $n(bd)^2$ as
      \begin{align*}
        2^kp_1^{a_1} \ldots p_r^{a_r}q_1^{b_1} \ldots q_s^{b_s} = 2^kp_1^{a_1} \ldots p_r^{a_r}q_1^{2\alpha_1 + \beta_1} \ldots q_s^{2\alpha_s + \beta_s}
      \end{align*}
      Since $b_i = 2\alpha_i + \beta_i$ must be even for each $i$, it follows that $\beta_i$ must also be even for each $i$.
      Therefore, by Corollary 19, $n$ can be written as the sum of two integer squares.
    \end{alphaenum}
  \end{proof}
\end{thm}


\begin{thm}
  \label{Ex4}
  Let $R = \Z[\sqrt{-n}]$ where $n$ is a square-free integer greater than 3.
  \begin{alphaenum}
  \item 
    Prove that $2, \sqrt{-n}$ and $1 + \sqrt{-n}$ are irreducibles in $R$.
  \item
    Prove that $R$ is not a U.F.D.  
    Conclude that the quadratic integer ring $\mathcal{O}$ is not a U.F.D. for $D \equiv 2,3 \pmod{4}, D < -3$.
  \item
    Give an explicit ideal in $R$ that is not principal.
  \end{alphaenum}
  \begin{proof}
    \begin{alphaenum}
    \item 
      First observe that if $a$ and $b \not = 0$ are integers, then $N(a + b\sqrt{-n}) = a^2 + b^2n \geq 5.$
      Suppose there are two elements $\alpha, \beta$ of $R$ such that $2 = \alpha\beta$.
      Taking norms of both sides we have $4 = N(\alpha)N(\beta)$ and by the observation above, $N(\alpha) = 1$ or $4$.  
      Hence one of $\alpha$ or $\beta$ must be a unit.
      Therefore 2 is irreducible.
      
      Since $N(\sqrt{-n}) = n$ and $n$ was assumed to be square free, $N(a + b\sqrt{-n}) = a^2 + b^2n = n$ if and only if $a=0, b=1$.
      Hence for any $\alpha,\beta \in R$ such that $n = \alpha\beta$, it must be the case that one of $\alpha$ or $\beta$ must be a unit.
      Therefore $\sqrt{-n}$ is irreducible.
      
      Let $\alpha = a + b\sqrt{-n}$ and $\beta = c + d\sqrt{-n}$ be elements of $R$ such that $1 + \sqrt{-n} = \alpha\beta$.
      Taking norms of both sides we have $n+1 = (ac)^2 + ((ad)^2 + (bc)^2)n + (bd)^2n^2$.
      Clearly one or both of $b,d$ must be zero.
      If both are zero, then $\alpha,\beta \in \Z$ and $ 1 + \sqrt{-n} = \alpha\beta = ac$.
      Clearly, this is not the case.
      Assume $b \not = 0$.  
      Then $1 + \sqrt{-n} = ac + bc\sqrt{n}$, but $a,b,c \in \Z$ implies $a = b = c= 1$.
      Hence $\beta$ is a unit.
      Therefore $1 + \sqrt{-n}$ is irreducible.
    \item
      Consider the integers $n = -(\sqrt{-n})^2$ and $n+1 = (1 + \sqrt{-n})(1 - \sqrt{-n})$.
      One of these must necessarily be even.
      Assume $n$ is even, so $n = 2k$ for some integer, $k$.  
      Since $2$ is irreducible, $\sqrt{-n}$ does not divide $2$, so assume it divides $k$.
      If $k = \frac{n}{2} = \alpha\sqrt{-n}$ for some $\alpha \in R$, then by taking the norm of both sides we obtain $$\frac{n}{4} = N(\alpha).$$
      However, this contradicts the assumption that $n$ is square-free.
      Hence $\sqrt{-n}$ does not divide either 2 or $k$.
      Similarly, if we assume that $n+1 = 2k$, for some integer, $k$, the same proof holds mutatis mutandis.
      This shows that one of $\sqrt{-n}$ or $1 + \sqrt{-n}$ is not prime.
      Therefore it follows from Proposition 12 that if $D < -3$ and $D \equiv 2,3 \pmod{4}$, then  $\mathcal{O} = R$ is not a U.F.D.
    \item
      If $n$ is odd, then the ideal $(2, 1 + \sqrt{-n})$ is a non-principal ideal.
      To see this is true, assume it were principal and generated by the element $a + b\sqrt{-n}$.
      Then there would exist elements $\alpha, \beta$ of $R$ such that $2 = \alpha(a + b\sqrt{-n})$ and $1 + \sqrt{-n} = \beta(a + b\sqrt{-n})$.
      Since $2$ and $1 + \sqrt{-n}$ are both irreducibles over $R$, it follows that $\alpha = 2, \beta = 1 + \sqrt{-n}$ and $a + b\sqrt{-n}=1$.
      If this is the case, then there must exist $\gamma,\delta \in R$ such that $$2\gamma + (1 + \sqrt{-n})\delta = 1.$$
      Now multiply both sides by $1 - \sqrt{-n}$ to obtain $$2(\gamma(1-\sqrt{-n} + 2k) = 1 - \sqrt{-n},\, k\in \Z.$$
      However, 2 does not divide $1 - \sqrt{-n}$.
      Therefore $(2, 1+\sqrt{-n})$ is a non-principal ideal.
      Similarly, if $n$ is even, then the ideal $(2, \sqrt{-n})$ would not be a principal ideal following the same proof, mutatis mutandis.
    \end{alphaenum}
  \end{proof}
\end{thm}

\begin{thm}
  \label{Ex5}
  Let $R$ be the quadratic integer ring $\Z[\sqrt{-5}]$ and define the ideals $I_2 = (2, 1 + \sqrt{-5}), I_3 = (3, 2 + \sqrt{-5}),$ and $I_3' = (3,2 - \sqrt{-5})$.
  \begin{alphaenum}
  \item
    Prove that $2,3, 1 + \sqrt{-5}$ and $1 - \sqrt{-5}$ are irreducibles in $R$, no two of which are associates in $R$, and that $6 = 2{\cdot}3 = (1+\sqrt{-5}) \cdot (1 - \sqrt{-5})$ are two distinct factorizations of 6 into irreducibles in $R$.
  \item
    Prove that $I_2, I_3,$ and $I_3'$ are prime ideals in $R$.
  \item
    Show that the factorizations in (a) imply the equality of ideals $(6) = (2)(3)$ and $(6) = (1+\sqrt{-5})(1-\sqrt{-5}).$  
    Show that these two ideal factorizations give the same factorization of the ideal (6) as the product of prime ideals.
  \end{alphaenum}
  \begin{proof}
    \begin{alphaenum}
    \item
      By \ref{Ex4}(a) we have $2, 1+\sqrt{-5}$ and $1 - \sqrt{-5}$ must all be irreducible.
      To see 3 is irreducible, assume $3 = \alpha\beta$ for some $\alpha,\beta\in R$.
      Take the norm of both sides to obtain $$9 = N(\alpha)N(\beta).$$
      There are then three options for $N(\alpha)$;  $N(\alpha)$ is either 1, 3 or 9.
      If $N(\alpha)$ is 1 or 9, then one of $\alpha$ or $\beta$ is a unit.
      So, the only case that remains is $N(\alpha) = 3$, but this is impossible.
      Let $\alpha = a + b\sqrt{-5}$ and observe that $a^2 + 5b^2 = 3$ does not have any integral solutions. 
      Therefore 3 is irreducible and the terms in the factorization of 6 above are all irreducible, as desired.
    \item
      First observe that by the Third Isomorphism Theorem $(R/(3))/(I_3/(3)) \cong R/I_3$.
      If $a + b\sqrt{-5} \in R$, then in $R/(3)$ there are three choices for $a$ and three choices for $b$.
      Hence there are a total of 9 elements in the quotient ring $R/(3)$.
      Let $\alpha 3 + \beta(2 + \sqrt{-5}) \in I_3$ be given, where $\alpha, \beta \in R$.
      Reducing these elements modulo 3, it is clear that the first term goes to zero.
      Hence the elements of $I_3/(3)$ are $\beta(2  +\sqrt{-5})$, reduced modulo 3.
      Let $\beta = c + d\sqrt{-5}$ so that 
      \begin{align*}
        \begin{split}
          \beta(2 + \sqrt{-5}) &\equiv 2c + d + (c + 2d)\sqrt{-5} \pmod{3}\\
          &\equiv -c - 2d + (c + 2d)\sqrt{-5} \pmod{3}\\
          &\equiv (c+2d)(2 + \sqrt{-5}) \pmod{3}.
        \end{split}
      \end{align*}
      Since $c+2d$ can only take on three values modulo 3, it follows that there are only three distinct elements in the quotient ring $I_3/(3)$.
      Hence the quotient ring $R/I_3$ has three elements and is isomorphic as a group under addition to $\Z/3\Z$, the only additive abelian group of order 3.
      Indeed, writing the elements of $R/(3)$ explicitly and reducing modulo the elements of $I_3/(3)$, it is easy to check by hand that the elements of $R/I_3$ are
      \begin{align*}
        \begin{split}
          \overline{0}&: 0, 2+\sqrt{-5}, 1 + 2\sqrt{-5}\\
          \overline{1}&: 1, \sqrt{-5}, 2 + 2\sqrt{-5}\\
          \overline{2}&: 2, 2\sqrt{-5}, 1 + \sqrt{-5}
        \end{split}
      \end{align*}
      and these behave like $\Z/3\Z$ under multiplication.
      Therefore since $R/I_3$ is a field, $I_3$ is a maximal and thus prime ideal.
      
      Using the same construction as above, the results are nearly identical with $I_3'$, up to a few minor sign changes.
      Likewise, constructing $R/I_2$ by using the isomorphic quotient $(R/(2))/(I_2/(2))$, we have $R/(2)$ with 4 elements and $I_2/(2)$ with 2 elements.
      The resulting quotient is then the field $\Z/2\Z$ and $I_2$ is therefore a prime ideal.
      \item
        From \ref{Ex1} we know $I_2^2I_3I_3' = (1+\sqrt{-5})(1-\sqrt{-5}) = (6)$ and $I_2^2 = (6)$, so it remains to show that $I_3I_3' = (3).$
        For simplicity of expression, take  $a = 3$ and $b = 2 + 2\sqrt{-5}$ and observe that $b\overline{b} = 9 = a^2$.
        As in \ref{Ex1}, we aim to show that every linear combination of elements in $I_3$ and $I_3'$ is a multiple of three by showing that for every $c = a\alpha + b\beta \in I_3$ and $d = a\gamma + \overline{b}\delta \in I_3'$, $cd$ is a multiple of 3.
        Now 
        \begin{align*}
          \begin{split}
            cd &= (a\alpha + b\beta)(a\gamma + \overline{b}\delta)\\
            &= a^2\alpha\gamma + a\overline{b}\alpha\delta + ab\beta\gamma + b\overline{b}\beta\delta\\
            &= 3(a(\alpha\gamma  + \beta\delta) + \overline{b}\alpha\delta + b\beta\delta).
          \end{split}
        \end{align*}
        Therefore $I_3I_3' = (3)$ and from the proof of \ref{Ex1} these factorizations of $(6)$ are identical.
    \end{alphaenum}
  \end{proof}
\end{thm}

\begin{thm}
  \label{Ex6}
  Let $F$ be a field. 
  Prove that the set $R$ of polynomials in $F[x]$ whose coefficient of $x$ is equal to 0 is a subring of $F[x]$ and that $R$ is not a U.F.D.
  \begin{proof}
    First we show that $R$ is closed under multiplication and addition.
    Let $p(x) = \sum_{k=0}^m a_kx^k, q(x) = \sum_{k=0}^n b_kx^k \in R$  be given.
    Clearly $p(x) + q(x) \in R$.
    From the definition of polynomial multiplication, the coefficient of $x$ in the product $p(x)q(x)$ is given by $a_0b_1 + a_1b_0 = 0$, since $a_1 = b_1 = 0$ by assumption.
    Since $R$ is closed, $R$ inherits associativity and distributivity from $F[x]$.
    Also, note that $R$ trivially contains the $0$ polynomial, which serves as the additive identity.
    It remains only to show that $R$ is closed under additive inverses.
    Consider $p(x)$ and note that its inverse in $F[x]$ is just $\sum_{k=0}^m -a_kx^k$, whose coefficient of $x$ is also equal to zero.
    Therefore $R$ is a subring of $F[x]$.
    
    Now to see that $R$ is not a U.F.D., observe that since $x \not \in R$, $x^2 = x*x$ and $x^3 = x^2*x = x*x*x$ are both irreducible.  
    Now consider the element $x^6 \in R$.
    We can write $x^6 = (x^2)^3 = (x^3)^2$, which are two factorizations of $x^6$ into irreducibles.
    Therefore, $R$ is not a U.F.D.
  \end{proof}
\end{thm}

\end{document}
