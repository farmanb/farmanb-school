\documentclass[10pt]{amsart}
\usepackage{amsmath,amsthm,amssymb,amsfonts,mymath}
\openup 5pt
\author{Blake Farman}
\title{Math-252:\\Homework 9}
\date{April 13, 2011}
\pdfpagewidth 8.5in
\pdfpageheight 11in
\usepackage[margin=1in]{geometry}
\begin{document}
\maketitle

\newtheorem{thm}{}

\begin{thm}
  \label{Ex1}
  \newcommand{\rt}{\sqrt[p]{2}}
  Let $p$ be a prime.
  Determine the elements of the Galois group of $x^p - 2$.
  
  \begin{proof}
    Let $\zeta$ be a primitive $p^{\text{th}}$ root of unity and let $\rt$ be real.
    The field $\Q(\rt,\zeta)$ is a splitting field for $x^p - 2$.
    We observe that $[\Q(\rt):\Q] = p$ and $[\Q(\zeta):\Q] = p-1$, hence $$|\Gal(\Q(\rt,\zeta)/\Q)| =[\Q(\rt,\zeta):\Q] = p(p-1).$$
    Therefore the automorphisms 
    $$
    \begin{cases}
      \zeta &\longmapsto \zeta^a, \quad a = 1, 2, \ldots, p-1,\\
      \rt &\longmapsto \zeta^b\rt, \quad b = 0,1,\ldots,p-1
    \end{cases}
    $$
    form the Galois group of $x^p - 2$.
  \end{proof}
  
  Prove that the Galois group of $x^p - 2$ is isomorphic to the group of matrices 
  $
  \begin{pmatrix}
    a & b\\
    0 & 1
  \end{pmatrix}
  $
  where $a,b \in \F_p, a \neq 0$. 

  \begin{proof}
    Let $G$ be the galois group of $x^p - 2$ and let $H \leq GL_2(\F_p)$ be the subgroup of matrices of the form 
    $
    \begin{pmatrix}
      a & b\\
      0 & 1
    \end{pmatrix}.
    $
    For $\sigma$ as above, define the map 
    \begin{align*}
      \phi : G &\longrightarrow H\\
      \sigma &\longmapsto \begin{pmatrix}
        a & b\\
        0 & 1
      \end{pmatrix}.
    \end{align*}
    Observe that by construction $\ker\phi = id_G$ and thus $\phi$ is bijective.
        Let 
    $$\sigma : \begin{cases}
      \zeta &\longmapsto \zeta^a,\\
      \rt &\longmapsto \zeta^b\rt
    \end{cases} 
    \quad \text{and} \quad 
    \tau : \begin{cases}
      \zeta &\longmapsto \zeta^c,\\
      \rt &\longmapsto \zeta^d\rt
    \end{cases}
    $$
    be two elements of $G$.
    The composition of these maps is given by
    $$\sigma\tau : \begin{cases}
      \zeta &\longmapsto \zeta^ac,\\
      \rt &\longmapsto \zeta^{ad+b}\rt
    \end{cases}.$$
    Then we have 
    $$     
      \phi(\sigma)\phi(\tau) = \begin{pmatrix}
        a & b\\
        0 & 1\\
      \end{pmatrix}
      \begin{pmatrix}
        c & d\\
        0 & 1\\
      \end{pmatrix}
      = \begin{pmatrix}
        ac & ad + b\\
        0 & 1\\
      \end{pmatrix}
      = \phi(\sigma\tau).
      $$
      Therefore $\phi$ is an isomorphism of $G$ and $H$.
  \end{proof}
\end{thm}

\begin{thm}
  \label{Ex2}
  Suppose $f(x) \in \Z[x]$ is an irreducible quartic polynomial whose splitting field has Galois group $S_4$  over $\Q$.
  Let $\theta$ be a root of $f(x)$ and set $K = \Q(\theta)$. 
  Prove that $K$ is an extension of $\Q$ of degree 4 which has no proper subfields.
  Are there any Galois extensions of $\Q$ of degree 4 with no proper subfields?

  \begin{proof}
    The elements of $S_4$ that fix any one of the roots in the splitting field form a subgroup isomorphic to $S_3$.
    Hence $[K:\Q] = [S_4:S_3] = 4$ follows from the Fundamental Theorem of Galois Theory.
        
    To see that $K$ does not have any proper subfields, it suffices to show that there is no subgroup $H$ of $S_4$ containing an isomorphic copy of $S_3$, or any group of order 6.
    To that end, suppose to the contrary that one exists and observe that $|H| = 12$ must hold.
    Moreover $A_4H \leq S_4$ and its order is given by $12^2/|A_4 \cap H| \leq 24$, from which it follows that $|A_4 \cap H|$ is either 6 or 12 by Lagrange's Theorem.
    However, as was proven in Math 251, $A_4$ does not have a subgroup of order 6.
    Therefore no such $H$ exists and $K$ does not have any proper subfields.

    There are 4 such isomorphic copies of $S_3$ contained within $S_4$, each obtained by fixing one of the roots of $f$.
    Since $S_3$ is not normal in $S_4$, none of these degree 4 extensions obtained by adjoining a root of $f$ to $\Q$ are Galois.
  \end{proof}
\end{thm}

\begin{thm}
  \label{Ex3}
  Prove that if the Galois group of the splitting field of a cubic over $\Q$ is the cyclic group of order 3, then all the roots of the cubic are real.
  
  \begin{proof}
    Suppose to the contrary that there were a complex root.
    Complex roots come in conjugate pairs, so complex conjugation restricted to the splitting field would be an element of the Galois group and it would necessarily fix the third real root, $\theta$.
    Hence $\Q(\theta)$ is a fixed field corresponding to a subgroup of the Galois group.
    However, $Z_3$ is simple.
    Therefore all the roots must be real.
  \end{proof}
\end{thm}

\begin{thm}
  \label{Ex4}
  Let $F$ be a field of characteristic $\neq 2$.
  \begin{alphaenum}
    \item
      If $K = F(\sqrt{D_1},\sqrt{D_2})$ where $D_1,D_2 \in F$ have the property that none of $D_1, D_2$ or $D_1D_2$ is a square in $F$, prove that $K/F$ is a Galois extension with $\Gal(K/F)$ isomorphic to the Klein 4-group.
    \item
      Conversely, suppose $K/F$ is a Galois extension with $\Gal(K/F)$ isomorphic to the Klein 4-group.
      Prove that $K = F(\sqrt{D_1},\sqrt{D_2})$ where $D_1,D_2 \in F$ have the property that none of $D_1,D_2$ or $D_1D_2$ is a square in $F$.
  \end{alphaenum}

  \begin{proof}
  \begin{alphaenum}
    \item
      By Exercise 8 of Section 13.2, $K$ is a biquadratic extension of degree 4.
      Moreover it is the splitting field of the separable polynomial $(x^2-D_1)(x^2-D_2)$ and hence $K/F$ is Galois by Theorem 13.
      The Galois group of $K$ is given by the four automorphisms
      $$
      \begin{cases}
        \sqrt{D_1} \longmapsto \pm\sqrt{D_1}\\
        \sqrt{D_2} \longmapsto \pm\sqrt{D_2}.
      \end{cases}
      $$
      Since each non-identity automorphism merely toggles the sign of a root, it is clear that each has order 2 and thus $\Gal(K/F) \cong V_4$.
    \item
      It follows from the Fundamental Theorem of Galois Theory that $K$ is a degree 4 extension of $F$ with three subfields, each quadratic Galois extensions of $F$.
      Any two of these subfields are isomorphic to $F(\sqrt{D_1})$ and $F(\sqrt{D_2})$ for some square-free elements $D_1,D_2$ of $F$.
      Since $K = F(\sqrt{D_1},\sqrt{D_2})$ has degree 4 it follows from Exercise 8 of Section 13.2 that $D_1D_2$ is not a square.
    \end{alphaenum}    
  \end{proof}
\end{thm}

\begin{thm}
  \label{Ex5}
  \begin{alphaenum}
    \item
      Prove that $x^4 - 2x^2 - 2$ is irreducible over $\Q$.
    \item
      Show the roots of this quartic are 
      \begin{align*}
        \begin{split}
          \alpha_1 = \sqrt{1 + \sqrt{3}} \qquad & \alpha_3 = -\sqrt{1 + \sqrt{3}}\\
          \alpha_2 = \sqrt{1 - \sqrt{3}} \qquad & \alpha_4 = -\sqrt{1 - \sqrt{3}}.
        \end{split}
      \end{align*}
    \item
      Let $K_1 = \Q(\alpha_1)$ and $K_2 = \Q(\alpha_2)$.  Show that $K_1 \neq K_2$ and $K_1 \cap K_2 = \Q(\sqrt{3}) = F$.
    \item
      Prove that $K_1$, $K_2$ and $K_1K_2$ are Galois over $F$ with $\Gal(K_1K_2/F)$ the Klein 4-group.
      Write out the elements of $\Gal(K_1K_2/F)$ explicitly.
      Determine all the subgroups of the Galois group and give their corresponding fixed subfields of $K_1K_2$ containing $F$.
    \item
      Prove that the splitting field of $x^4 - 2x^2 -2 $ over $\Q$ is of degree 8 with dihedral Galois group.
  \end{alphaenum}

  \begin{proof}
  \begin{alphaenum}
    \item
      The polynomial $x^4 - 2x^2 - 2$ is irreducible by an application of Eisenstein's Criterion using the prime 2.
    \item
      By completing the square, we get $$x^4 - 2x^2 - 2 = (x^2 - 1)^2 -3 = (x^2 - (1 + \sqrt{3}))(x^2 - (1 - \sqrt{3})).$$
      Therefore the roots are $\pm \sqrt{1 + \sqrt{3}}$ and $\pm \sqrt{1 - \sqrt{3}}$.
    \item
      Let $D_1 = 1 + \sqrt{3}$ and $D_2 = 1 - \sqrt{3}$ and observe that neither is a square in $F = \Q(\sqrt{3})$.
      Since both $K_1$ and $K_2$ contain $\sqrt{3}$ and thus contain $F$ as a subfield we can consider $K_1/\Q$ and $K_2/\Q$ as the extensions $K_1/F = F(\sqrt{D_1})$ and $K_2/F = F(\sqrt{D_2})$.
      Since $D_1D_2 = -2$ is not a square in $F$, the extension $K_1K_2/F$ is biquadratic by Exercise 8 of Section 13.2.
      Hence $K_1 \not = K_2$ and $K_1 \cap K_2 = F$.
    \item
      By Exercise \ref{Ex4} and the result of part (c), $\Gal(K_1K_2/F) \cong V_4$ and $K_1/F$, $K_2/F$ are quadratic Galois extensions.
      The elements of $\Gal(K_1K_2/F)$ are the identity automorphism along with
      \begin{align*}
        \begin{split}
          \sigma : \begin{cases}
            \alpha_1 & \longmapsto -\alpha_1\\
            \alpha_2 & \longmapsto \alpha_2
          \end{cases}
          \qquad
          \tau : \begin{cases}
            \alpha_1 & \longmapsto \alpha_1\\
            \alpha_2 & \longmapsto -\alpha_2
          \end{cases}
          \quad \text{and} \quad
          \sigma\tau : \begin{cases}
            \alpha_1 & \longmapsto -\alpha_1\\
            \alpha_2 & \longmapsto -\alpha_2.
          \end{cases}
        \end{split}
      \end{align*}
      The non-trivial subgroups are $<\sigma>$, which fixes $K_2$, $<\tau>$, which fixes $K_1$ and $<\sigma\tau>$, which fixes $F(\sqrt{D_1D_2}) = F(\sqrt{-2})$.
    \item
      Observe that $[K_1K_2:\Q] = [K_1K_2:F][F:\Q] = 8$, so it is clear that $|\Gal(K_1K_2/\Q)| = 8$.
      The remaining four automorphisms are obtained by mapping $\sqrt{3}$ to $-\sqrt{3}$.
      They are the map $\sigma$ as above along with the maps
      \begin{align*}
        \begin{split}
          \gamma : \begin{cases}
            \alpha_1 \longmapsto \alpha_2\\
            \alpha_2 \longmapsto -\alpha_1.
          \end{cases}
          \quad \gamma^2 = \sigma\tau: \begin{cases}
            \alpha_1 \longmapsto -\alpha_1\\
            \alpha_2 \longmapsto -\alpha_2
          \end{cases}      
          \quad \gamma^3 : \begin{cases}
            \alpha_1 \longmapsto -\alpha_2\\
            \alpha_2 \longmapsto \alpha_1
          \end{cases}\\
          \quad \sigma\gamma : \begin{cases}
            \alpha_1 \longmapsto \alpha_2\\
            \alpha_2 \longmapsto \alpha_1
          \end{cases}
          \quad \sigma\gamma^2 = \tau: \begin{cases}
            \alpha_1 \longmapsto \alpha_1\\
            \alpha_2 \longmapsto -\alpha_2
          \end{cases}
          \quad \sigma\gamma^3 : \begin{cases}
            \alpha_1 \longmapsto -\alpha_2\\
            \alpha_2 \longmapsto \alpha_1.
          \end{cases}
        \end{split}
      \end{align*}
      Here $\gamma$ and $\gamma^3$ have order 4 and all other maps have order 2.
      Moreover, $\gamma\sigma = \sigma\gamma^3$.
      Hence $\gamma$ behaves as $r$, $\sigma$ as $s$ and satisfy the relation $\gamma\sigma = \sigma\gamma^3$.
      Therefore $\Gal(K_1K_2/\Q) \cong D_8$.
  \end{alphaenum}
  \end{proof}
\end{thm}

\end{document}
