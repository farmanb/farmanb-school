\documentclass[10pt]{amsart}
\usepackage{amsmath,amsthm,amssymb,amsfonts,mymath}
\openup 5pt
\author{Blake Farman}
\title{Math-252:\\Homework 5}
\date{March 16, 2011}\pdfpagewidth 8.5in
\usepackage[margin=1in]{geometry}
\pdfpageheight 11in
\begin{document}
\maketitle

\newtheorem{thm}{}

\begin{thm}
  \label{Ex1}
  Prove that similar linear transformations of $V$ (or $n \times n$ matrices) have the same characteristic and the same minimal polynomial.
  \begin{proof}
    Let $S$ and $T$ be similar linear operators on $V$.  
    By Theorem 15, the $F[x]$-modules they generate are isomorphic to one another and thus they have the same invariant factors.
    Hence by Proposition 13, $m_S(x) = m_T(x)$.
    
    Let $U$ be the nonsingular operator on $V$ such that $S = UTU^{-1}$.
    The characteristic polynomial of $S$ is given by 
    \begin{align*}
      \begin{split}
        c_S(x) &= \det(xI - S)\\
        & = \det(xI - UTU^{-1})\\
        & = \det(U(xI - T)U^{-1})\\
        & = \det(U)c_T(x)\det(U^{-1})\\
        & = c_T(x).
        \end{split}
      \end{align*}
      Therefore the characteristic and minimal polynomials are the same.
  \end{proof}
\end{thm}

\begin{thm}
  \label{Ex2}
  \begin{alphaenum}
  \item
    Prove that two $2 \times 2$ matrices over $F$ which are not scalar matrices are similar if and only if they have the same characteristic polynomial.
  \item
    Prove that two $3 \times 3$ matrices over $F$ are similar if and only if they have the same characterstic and same minimal polynomials.
    Give an explicit counterexample to this assertion for $4 \times 4$ matrices.
  \end{alphaenum}
  \begin{proof}
    \begin{alphaenum}
    \item
      Let $A$ and $B$ be non-scalar $2 \times 2$ matrices.
      Assume $A$ and $B$ are similar.
      It follows from the previous exercise that the characteristic polynomials are the same.
      
      Conversely, suppose that $c_A(x) = c_B(x)$.
      Note that the degree of the characteristic polynomial is 2.
      Suppose that the degree of the minimal polynomial were degree 1.
      Then $m_A(x) = x - I\alpha$, for some $\alpha$ and $m_A(A) = A - I\alpha = 0$.
      This contradicts the assumption that $A$ is a non-scalar matrix.
      Hence the degree of the minimal polynomial must be 2 and thus equals the characteristic polynomial.
      A similar argument for $B$ shows that $m_A(x) = m_B(x) = c_A(x) = c_B(x)$.
      Therefore they have the same rational canonical form, proving their similarity.
    \item
      Let $A$ and $B$ be $3 \times 3$ matrices.
      Assume $A$ and $B$ are similar.
      By the result of the previous exercise, the minimal and characteristic polynomials are equal.
      
      Conversely, suppose that $c_A(x) = c_B(x)$ and $m_A(x) = m_B(x)$.
      There are three possibilities for the characteristic polynomial; either it is irreducible of degree 3, it factors into an irreducible quadratic and a linear term or it factors into three linear terms.
      Since the minimal and characterstic polynomials must share the same factors up to multiplicity, the only case of interest is the latter since $m_A(x) = m_B(x) = c_A(x) = c_B(x)$ in all other cases.
      In particular, we are concerned only with the case where at least two linear factors are the same and the minimal polynomial is of degree 2.
      We observe that because  $m_A(x) = m_B(x)$, there is no choice for the remaining invariant factor.
      Therefore $A$ and $B$ have the same rational canonical form, proving they are similar.
    \end{alphaenum}
  \end{proof}
\end{thm}

\begin{thm}
  \label{Ex3}
  Determine the eigenvalues of the matrix
  $$
  \begin{pmatrix}
    0 & 1 & 0 & 0\\
    0 & 0 & 1 & 0\\
    0 & 0 & 0 & 1\\
    1 & 0 & 0 & 0
  \end{pmatrix}.
  $$
  \begin{proof}
    Taking the determinant of the matrix
    $$
  \begin{pmatrix}
    \lambda & 1 & 0 & 0\\
    0 & \lambda & 1 & 0\\
    0 & 0 & \lambda & 1\\
    1 & 0 & 0 & \lambda
  \end{pmatrix}
  $$
  we obtain the polynomial  $\lambda^4 - 1 = (\lambda+i)(\lambda - i )(\lambda+1)(\lambda-1)$.
  Therefore the eigenvalues are $\pm i$ and $\pm 1$.
  \end{proof}
\end{thm}

\begin{thm}
  \label{Ex4}
  Find the rational canonical forms of
  $$
  \begin{pmatrix}
    0 & -1 & -1\\
    0 & 0 & 0 \\
    -1 & 0 & 0
  \end{pmatrix},
  \begin{pmatrix}
    c & 0 &-1\\
    0 & c & 1\\
    -1 & 1 & c
  \end{pmatrix}
  \quad
  \text{and}
  \quad
  \begin{pmatrix}
    422 & 465 & 15 & -30\\
    -420 & -463 & -15 & 30\\
    840 & 930 & 32 & -60\\
    -140 & -155 & -5 & 12
  \end{pmatrix}.
  $$
  \begin{proof}
    Let the matrices above be $A, B$ and $C$ respectively.
    The characteristic polynomial of $A$ is given by $c_A(x) = x(x+1)(x-1) = x^3 - x$.
    Since the factors are all different, the divisibility condition of the invariant factors allows us to conclude that $c_A(x) = m_A(x)$ thus the rational canonical form is 
    $$
    \begin{pmatrix}
      0 & 0 & 0\\
      1 & 0 & 1\\
      0 & 1 & 0
    \end{pmatrix}.
    $$
    
    The characteristic polynomial of $B$ can be calculated as $c_B(x) = (x-c)((x-c)^2 - 2) = x^3 - 3cx^2 + (3c^2-2)x + 2c - c^3$.
    Consider $(x-c)^2 - 2$; we aim to show this is irreducible.
    Suppose to the contrary that $(x+a)(x+b) = (x-c)^2 - 2$.
    Then $a+b = -2c$ and $ab = c^2 - 2$.
    Solving the first equation for $c$ yields $c = -(a+b)/2,$ from which it follows by some routine algebra that $4ab + 8 = (a+b)^2$.
    We observe that by subtracting $4ab$ from both sides we obtain $8 = (a-b)^2$.
    However, $8$ is not a square in $\Q$, a contradiction.
    Therefore we have $c_B(x) = m_B(x)$ and thus the rational canonical form is given by 
    $$
    \begin{pmatrix}
      0 & 0 & c^3 - 2c\\
      1 & 0 & 2 - 3c^2\\
      0 & 1 & 3c
    \end{pmatrix}.
    $$
    
    Finally, computing the characteristic polynomial of $C$ yields $c_C(x) = (x-2)^3(x+3)$.
    There are then three possibilities for the minimal polynomial: $$(x-2)(x+3), \quad (x-2)^2(x+3) \quad \text{or} \quad (x-2)^3(x+3).$$
    A quick check by hand shows that $(A-I2)(A+I3) = 0$ and so the invariant factors are $(x-2)(x+3),(x-2),(x-2).$
    Hence the rational canonical form is 
    $$
    \begin{pmatrix}
      2 & 0 & 0 & 0\\
      0 & 2 & 0 & 0\\
      0 & 0 & 0 & 6\\
      0 & 0 & 1 & -1
    \end{pmatrix}.
    $$
  \end{proof}
\end{thm}

\begin{thm}
  \label{Ex5}
  Find all similarity classes of $6 \times 6$ matrices over $\Q$ with minimal polynomial $(x+2)^2(x-1)$ (it suffices to give all lists of invariant factors and write out some of their corresponding matrices).
  \begin{proof}
    The list of invariant factors is as follows
    \begin{align*}
      (x+2)^2(x-1), (x-1), (x-1), (x-1),\\
      (x+2)^2(x-1), (x+2)(x-1), (x-1),\\
      (x+2)^2(x-1), (x+2)^2(x-1),\\
      (x+2)^2(x-1), (x+2)(x-1), (x+2),\\
      (x+2)^2(x-1), (x+2)^2, (x+2), \, \text{and}\\
      (x+2)^2(x-1), (x+2), (x+2), (x+2).
    \end{align*}
    The rational canonical form of the first is
    $$
    \begin{pmatrix}
      1 & 0 & 0 & 0 & 0 & 0\\
      0 & 1 & 0 & 0 & 0 & 0\\
      0 & 0 & 1 & 0 & 0 & 0\\
      0 & 0 & 0 & 0 & 0 & 4\\
      0 & 0 & 0 & 1 & 0 & 0\\
      0 & 0 & 0 & 0 & 1 & -3\\
    \end{pmatrix}.
    $$
  \end{proof}
\end{thm}

\begin{thm}
  \label{Ex6}
  Determine up to similarity all $2 \times 2$ rational matrices of precise order 4.  
  Do the same if the matrix has entries from $\C$.
  \begin{proof}
    The polynomial $x^4 - 1$ factors as $(x^2 + 1)(x+1)(x-1)$ over $\Q$.
    The possibilities for the minimal polynomial are $x+1, x-1, x^2+1, (x+1)(x-1)$.
    The lists of invariant factors are 
    \begin{align*}
      x+1, x+1,\\
      x-1, x-1,\\
      x^2+1,\\
      (x+1)(x-1)
    \end{align*}
    respectively.
    Computing the rational canonical forms, it is easy to check by hand that $x^2+1$ is the only element up to similarity of order 4, whose rational canonical form is 
    $$
    \begin{pmatrix}
      0 & -1\\
      1 & 0\\
    \end{pmatrix}.
    $$
    
    Over $\C$, we have the factorization $x^4 - 1 = (x+i)(x-i)(x+1)(x-1)$.
    This yields the new possibilities for the minimal polynomial: $x+i, x-i, (x+i)(x+1), (x+i)(x-1), (x-i)(x+1)$ and $(x-i)(x-1)$.
    The corresponding invariant factors are then 
    \begin{align*}
      x+i, x+i,\\ 
      x-i, x-i,\\ 
      (x+i)(x+1),\\ 
      (x+i)(x-1),\\ 
      (x-i)(x+1),\\
      (x-i)(x-1).
    \end{align*}
    The corresponding canonical rational forms are 
    $$
    iI, -iI, 
    \begin{pmatrix}
      0 & -i\\
      1 & -(1+i)
    \end{pmatrix},
    \begin{pmatrix}
      0 & i\\
      1 & 1-i
    \end{pmatrix},
    \begin{pmatrix}
      0 & i\\
      1 & i-1
    \end{pmatrix},
    \quad \text{and} \quad
    \begin{pmatrix}
      0 & -i\\
      1 & 1+i
    \end{pmatrix}.
    $$
    It is easily checked by hand that each of the newly introduced classes are of order 4.
  \end{proof}
\end{thm}

\begin{thm}
  \label{Ex7}
  Show that $x^5 - 1 = (x-1)(x^2 - 4x + 1)(x^2 + 5x + 1)$ in $\mathbb{F}_{19}[x]$.
  Use this to determine up to similarity all $2 \times 2$ matrices with entries from $\mathbb{F}_{19}$ of order 5.
  \begin{proof}
    Expanding the right hand side of the equation above we have $$(x-1)(x^2 - 4x + 1)(x^2 + 5x + 1) = x^5 -19x^3 + 19x^2 - 1 \equiv x^5 - 1 \pmod{19}.$$
    Then the possibilities for the minimum polynomial are $x-1, x^2 - 4x +1 $ and $x^2 + 5x + 1$.
    The corresponding invariant factors are 
    \begin{align*}
      x-1, x-1,\\
      x^2 - 4x + 1,\\
      x^2 + 5x + 1
    \end{align*}
    and their associated rational canonical forms are 
    \begin{align*}
      I, \quad
      \begin{pmatrix}
        0 & -1\\
        1 & 4
      \end{pmatrix},
      \quad \text{and} \quad
      \begin{pmatrix}
        0 & -1\\
        1 & -5
      \end{pmatrix}.
    \end{align*}
    Checking again by hand it is easy to see that the first is of order 1 and the other two are of order 5.
  \end{proof}
\end{thm}
\end{document}


%Let $W$ be a non-trivial submodule of $V$ and suppose $W$ is invariant under $T$.
%        If $i \not \in W$, then the imaginary axis cannot be contained in $W$ because for any $y \in \R$, there exists $1/y \in \R$ such that $1/y\cdot iy = i$.
%        Moreover, the real axis cannot be contained in $W$ since $T(1) = -i \not \in W$.
%        But this is absurd\\ consider $w + T(w)$ for any $w \in W$.
%        This sum lies either on the real or imaginary axis, both of which were supposed not to be contained by $W$.
%        Hence $i \in W$ and $T(i) = 1 \in W$.
%        Therefore $W = \C$ as desired.
