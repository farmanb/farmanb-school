\documentclass[10pt]{amsart}
\usepackage{amsmath,amsthm,amssymb,amsfonts}
\openup 5pt
\author{Blake Farman}
\title{Math-351:\\Homework 3}
\date{March 2, 2011}
\usepackage[margin=1in]{geometry}
\pdfpagewidth 8.5in
\pdfpageheight 11in
\begin{document}
\maketitle

\newcommand{\Z}{\mathbb{Z}}
\newcommand{\R}{\mathbb{R}}
\newcommand{\Q}{\mathbb{Q}}
\newcommand{\C}{\mathbb{C}}

\renewcommand{\qedsymbol}{\(\blacksquare\)}
\newcommand{\znz}[1]{\Z / #1\Z}
\newcommand{\mznz}[1]{(\Z / #1\Z)^*}

\renewcommand{\phi}{\varphi}
\newenvironment{alphaenum}{
  \begin{enumerate}
    \renewcommand{\theenumi}{(\alph{enumi})}
    \renewcommand{\labelenumi}{\theenumi}
  }
  {\end{enumerate}}

\newcommand{\quadeq}[3]{\frac{-(#2) \pm \sqrt{(#2)^2 - 4(#1)(#3)}}{2(#3)}}
\newcommand{\F}{\mathbb{F}}
\newcommand{\tor}[1]{\operatorname{Tor}(#1)}
\newcommand{\real}[1]{\operatorname{Re}(#1)}
\newcommand{\imag}[1]{\operatorname{Im}(#1)}

\newcommand{\A}{\mathfrak{a}}
\newcommand{\N}[1]{\mathit{N}(#1)}
\newcommand{\OO}{\mathcal{O}}

\newtheorem{thm}{}

\theoremstyle{definition}
\newtheorem{defn}{Definition}

\begin{defn}
  We define the {\it norm} of a nonzero ideal in $\OO_K$ to be its index in $\OO_K$.
  We will denote the norm of an ideal by $\N{\A}$.

\end{defn}

\begin{thm}
  \label{Ex1}
  Let $\mathfrak{a}$ be an integral ideal with basis $\alpha_1, \ldots, \alpha_n$.
  Show that $$[\det(\alpha_i^{(j)})]^2 = (\N\A)^2d_K.$$
  \begin{proof}
    Observe that by definition $[\det(\alpha_i^{(j)})]^2 = d_{K/\Q}(\alpha_1, \ldots, \alpha_n)$.
    Therefore it follows from Exercise 4.2.8 that $d_{K/\Q}(\alpha_1,\ldots,\alpha_n) = m^2d_K$, where $m = [\OO_K:\A] = \N{\A}$.
  \end{proof}
\end{thm}

\begin{thm}
  \newcommand{\AL}{{\alpha^{(i)}}}
  \label{Ex2}
  If the minimal polynomial of $\alpha$ is $f(x) = x^n + ax +b$, show that for $K = \Q(\alpha)$, 
  $$d_{K/\Q}(\alpha) = (-1)^{n \choose 2}(n^nb^{n-1} + a^n(1-n)^{n-1}).$$
  \begin{proof}
    We first observe that $f(\AL) = 0$ and thus we can easily obtain $\AL^{n} = -a\AL -b$.
    Now we take the first derivative of $f$ and make the substitution above.  
    After some routine algebra we obtain
    \begin{equation}
      \begin{split}
        \label{2.1}
        f'(\alpha^{(i)}) &= n\AL^{n-1} + a\\
        &= \frac{n\AL^{n} + a\AL}{\AL}\\
        &= \frac{a\AL(1-n) - bn}{\AL}\\
        &= \frac{-a(1-n)}{\AL}\left[\frac{bn}{a(1-n)} - \AL \right].
      \end{split}
    \end{equation}
    We now observe that, because the collection of $\AL$ are the roots of $f$, we can express $f$ as 
    \begin{equation}
      \label{2.2}
      f(x) = \Pi_{i=1}^n (x - \AL).
    \end{equation}
    Seeing as $\Pi_{i=1}^n -\AL$ accounts for the the constant term on the right-hand side, we conclude 
    \begin{equation}
      \label{2.3}
      b = (-1)^n\cdot \Pi_{i=1}^n \AL.
    \end{equation}
    Hence by \eqref{2.1}, \eqref{2.2} and \eqref{2.3} $$\Pi_{i=1}^{n}f^{\prime}(\alpha^{(i)}) = \frac{a^n(1-n)^n}{b}f\left(\frac{bn}{a(1-n)}\right)$$
    Therefore by Exercise 4.3.3 we have after expanding the right-hand side
    $$d_{K/\Q}(\alpha) = (-1)^{n \choose 2}\Pi_{i=1}^{n}f^{\prime}(\alpha^{(i)}) = (-1)^{n \choose 2}(n^nb^{n-1} + a^n(1-n)^{n-1}),$$
    as desired.
  \end{proof}
\end{thm}

\begin{thm}
  \label{Ex3}
  Determine an integral basis for $K = \Q(\theta)$ where $\theta^3 + 2\theta + 1 = 0$.
  \begin{proof}
    From the previous Exercise, we know $d_{K/\Q}(\theta) = d_{K/\Q}(1,\theta,\theta^2) = (-1)^{3}(27 + 8(2)^{2}) = - 59 = m^2d_K,$ where $m = [\OO_K : \Z[\theta]]$.
    Since 59 is a prime, we have $m = 1$ and thus $\OO_K = \Z[\theta]$.
    Therefore $1, \theta, \theta^2$ is an integral basis for $K$.
  \end{proof}
\end{thm}

%\begin{thm}
%  \label{Ex4}
%  Let $K$ and $L$ be algebraic number fields of degree $m$ and $n$, respectively, over $\Q$. 
%  Let $d = gcd(d_K, d_L)$.
%  Show that if $[KL:\Q] = mn$, then $\OO_{KL} \subseteq 1/d\OO_K\OO_L$.
%  \begin{proof}
%    
%  \end{proof}
%\end{thm}

\begin{thm}
  \label{Ex5}
  Let $p$ and $q$ be distrinct primes congruent to 1 modulo 4.
  Let $K = \Q(\sqrt{p})$, $L = \Q(\sqrt{q})$.
  Find a $\Z$-basis for $\Q(\sqrt{p},\sqrt{q})$.
  
  \begin{proof}
    By the result of Exercise 4.3.4, $\{1, (1 + \sqrt{p})/2\}$ and $\{1, (1 + \sqrt{q})/2\}$ are bases for $\OO_K$ and $\OO_L$ respectively.
    Observing that $(d_K,d_L) = (p,q) = 1$ we have, by Exercise 4.5.13, the basis $\{1, (1 + \sqrt{p})/2, (1 + \sqrt{q})/2, (1 + \sqrt{q})(1 + \sqrt{p})/2\}$.
  \end{proof}
\end{thm}
\end{document}
