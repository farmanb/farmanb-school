\documentclass[10pt]{amsart}
\usepackage{amsmath,amsthm,amssymb,amsfonts}
\openup 5pt
\author{Blake Farman}
\title{Math-351:\\Homework 1}
\date{January 31, 2010}\pdfpagewidth 8.5in
\pdfpageheight 11in
\begin{document}

\maketitle

\renewcommand{\qedsymbol}{\(\blacksquare\)}
\newcommand{\znz}[1]{\mathbb{Z} / #1\mathbb{Z}}
\newcommand{\mznz}[1]{(\mathbb{Z} / #1\mathbb{Z})^*}

\newtheorem*{1}{2.2.2}
\begin{1}
  Prove that if \(p\) is a positive prime, then there exists an element \(x \in  \znz{p}\) such that \(x^2 \equiv -1 \pmod{p}\) if and only if either \(p = 2\) or \(p \equiv 1 \pmod{4}\).
  \begin{proof}
    Both directions of the case where \(p = 2\) are immediate, so assume that \(p \equiv 1 \pmod{4}\).  
    Let \(x \in \znz{p}\) be such that \(x^2 \equiv -1 \pmod{4}\).
    The order of \(x\) in \(\mznz{p}\) is 4 and so it follows as a consequence of Lagrange's Theorem that \(4 | (p-1)\).
    Therefore \(p \equiv 1 \pmod{4}\).

    Conversely, let the prime \(p \equiv 1 \pmod{4}\) be given and consider the cyclic group \(G = \mznz{p}\).
    Let \(g\) generate \(G\) and write \(p = 4n + 1\) for some integer \(n\).
    Observe that \(G\) has elements of all orders diving \(4n\) and thus \(G\) has an element of order four, namely \(g^n\).
    From this observation it now follows directly that \[g^{2n} \equiv \pm 1 \pmod{p}.\]
    However, \(g\) was assumed to be a generator of \(G\), hence it must be the case that \[(g^n)^2 \equiv -1 \pmod{p}\] and the proof is finished.
  \end{proof}
\end{1}

\newtheorem*{2}{2.2.3}
\begin{2}
  Find all integer solutions to \(y^2 + 1 = x^3\) with \(x,y \not = 0\)
  \begin{proof}
    First observe that in the integers \(y^2 + 1 \geq 1\) and thus \(x \geq 1\) must hold.
    Hence, it suffices only to check solutions in the natural numbers.
    Now rearranging the equation yields \[x^3 - y^2 = 1.\]
    By Catalan's Conjecture, \(2^3\) and \(3^2\) are the only consecutive powers over the natural numbers.
    However, 9 is not a cube and 8 is not a square.
    Therefore the only solution is the trivial solution, \(x = 1\), \(y = 0\).
  \end{proof}
\end{2}

\newpage

\newtheorem*{3}{2.3.2}
\begin{3}
  (a) Show that \(\mathbb{Z}[\rho]\) is Euclidean.\\
  (b) Show that only units in \(\mathbb{Z}[\rho]\) are \(\pm 1, \pm \rho,\) and \(\pm \rho^2\).
  \begin{proof}
    (a) To see \(\mathbb{Z}[\rho]\) is Euclidean, it suffices to show that for some norm, \(N\), any two elements of \(\mathbb{Z}[\rho]\), \(\alpha\) and \(\beta\), can be written as \(\alpha = q\beta + \gamma\) for some \(q,\gamma \in \mathbb{Z}[\rho]\) with \(N(\gamma) < N(b)\).
  To that end, define the map
  \begin{align*}
      N: \mathbb{Z}[\rho] & \longrightarrow \mathbb{Z}\\ 
      a + \rho b & \longmapsto (a + \rho b)(a + \overline{\rho}b)
    \end{align*}
    and write the quotient of \(\alpha\) and \(\beta\) as
    \begin{align*}
      \frac{\alpha\overline{\beta}}{\beta\overline{\beta}} = r + {\rho}s,
    \end{align*}
    for some \(r,s \in \mathbb{Q}\).
    Take \(m,n \in \mathbb{Z}\) to be the integers closest to \(r\) and \(s\), respectively, so that
    \begin{align*}
      |r - m| \leq \frac{1}{2} \quad \text{and} \quad  |s - n| \leq \frac{1}{2}.
    \end{align*}
    Let \(q = m + {\rho}n\) and observe that
    \begin{equation}
      \label{norm}
      \begin{split}
        N(\frac{\alpha}{\beta} - q) &= (r - m)^2 - (r-m)(s-n) + (s-n)^2\\
        & \leq \frac{1}{4} + \frac{1}{4} + \frac{1}{4}\\
        & < 1.
      \end{split}
    \end{equation}
    Let \(\gamma = {\beta}(\frac{\alpha}{\beta} - q)\) and observe that by \eqref{norm}
    \begin{equation*}
      \begin{split}
        N(\gamma) &= N(\beta)N(\frac{\alpha}{\beta} - q)\\
        &< N(\beta).
      \end{split}
    \end{equation*}
    Hence \(\alpha = q\beta + \gamma\) with \(N(\gamma) < N(\beta)\), as desired.
    Therefore \(\mathbb{Z}[\rho]\) is Euclidean.

    (b) Assume \(u = a + {\rho}b\) is a unit and thus
    \begin{align*}
      N(u) - 1 = a^2 - ab + b^2 - 1 = 0.
    \end{align*}
    Taking the above to be a polynomial in \(b\) and applying the quadratic formula, it follows that
    \begin{align*}
      b = \frac{a \pm \sqrt{4 - 3a^2}}{2}.
    \end{align*}
    Since \(b\) was assumed to be an integer, it is clear that \(|a| \leq 1\).
    Fixing \(a = 0, \pm 1\) then determines the six values of \(b\) corresponding to \(\pm 1, \pm \rho,\) and \(\pm \overline{\rho}\) as desired.
  \end{proof}
\end{3}

\newpage

\newtheorem*{4}{2.3.3}
\begin{4}
  Let \(\lambda = 1 - \rho\).  
  Show that \(\lambda\) is irreducible, so we have a factorization of 3 (unique up to unit).
  \begin{proof}
    Using \(N\) as defined in the previous exercise, observe that 
    \begin{align*}
      N(\lambda) = (1 - \rho)(1 - \overline{\rho}) = 3.
    \end{align*}
    For any \(\alpha,\beta \in \mathbb{Z}[\rho]\), if \(\lambda = \alpha\beta\), then 
    \begin{align*}
      N(\lambda) = N(\alpha)N(\beta) = 3.
    \end{align*}
    Hence one of \(\alpha\) or \(\beta\) must be a unit.
    Therefore \(\lambda\) is irreducible.
  \end{proof}
\end{4}
\end{document}
