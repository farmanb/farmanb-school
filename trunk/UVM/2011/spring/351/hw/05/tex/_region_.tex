\message{ !name(hw05-351.tex)}
\message{ !name(hw05-351.tex) !offset(-2) }
\documentclass[10pt]{amsart}
\usepackage{amsmath,amsthm,amssymb,amsfonts,mymath}
\openup 5pt
\author{Blake Farman}
\title{Math-351:\\Homework 3}
\date{March 2, 2011}
\usepackage[margin=1in]{geometry}
\pdfpagewidth 8.5in
\pdfpageheight 11in
\begin{document}
\maketitle

\newtheorem{thm}{}

\newcommand{\OO}{\mathcal{O}}

\begin{thm}
  \label{Ex1}
  Show that the sum and product of two fractional ideals are again fractional ideals.
  \begin{proof}
    Let $A$ and $B$ be fractional ideals of $\OO_K$ for some algebraic number field $K$ and let $m,n \in \Z$ be such that $mA \subseteq \OO_K$ and $nB \subseteq \OO_K$.
    
    Consider first the product ideal $AB$.  
    Seeing as both are modules, so too is their product.
    Consider $nmAB$.
    Since $mA \subseteq \OO_K$ by assumption and $B$ is a module, $nmAB \subseteq nB \subseteq \OO_K$.
    Hence $AB$ is a fractional ideal.
    
    Now consider their sum, $A + B$.
    Again, $A + B$ is a module since both $A$ and $B$ are modules.
    Consider $mn(A + B) = n(mA) + m(nB)$.
    By assumption $mA \subseteq \OO_K$, so it follows that $nmA \subseteq O_K$.
    Similarly, $mnB \subset \OO_K$.
    Hence $mn(A+B) \subseteq \OO_K$.
    Therefore $A+B$ is a fractional ideal.
  \end{proof}
\end{thm}

\begin{thm}
  \label{Ex2}
  Show that, given any fractional ideal $A \not = 0$ in $K$, there exists a fractional ideal $A^{-1}$ such that $AA^{-1} = \OO_K$.
  \begin{proof}
    By the result of Exercise 5.3.7, we can write 
    $$A = \frac{p_1p_2\ldots p_r}{p_1^{\prime}p_2^{\prime}\ldots p_s^{\prime}} \quad \text{and} \quad A = \frac{p_1p_2\ldots p_r}{p_1^{\prime}p_2^{\prime}\ldots p_s^{\prime}}$$
  \end{proof}
\end{thm}

\begin{thm}
  \label{Ex3}
  \newcommand{\A}{\mathfrak{a}}
  \newcommand{\B}{\mathfrak{b}}
  \renewcommand{\C}{\mathfrak{c}}
  \newcommand{\D}{\mathfrak{d}}
  \newcommand{\E}{\mathfrak{e}}
  Suppose $\A,\B,\C$ are ideals of $\OO_K$.
  Show that if $\A\B = \C^g$ and $(\A,\B) = 1$, then $\A = \D^g$ and $\B = \E^g$ for some ideals $\D$ and $\E$ of $\OO_K$.
  \begin{proof}
    Factor the ideals $\A$ and $\B$ into primes as 
    $$\A = \prod_{i=1}^r p_i^{\alpha_i} \quad \text{and} \quad \B = \prod_{i=1}^s q_i^{\beta_i}.$$
    Seeing as we assumed $(\A,\B) = 1$, we know that $p_i \not = q_j$ for any $i,j$.
    If we write
    $$\C^g = p_1^{\alpha_1}\ldots p_r^{\alpha_r}q_1^{\beta_1}\ldots q_s^{\beta_s},$$
    then $\C$ must factor as 
    $$\C = p_1^{\gamma_1} \ldots p_r^{\gamma_r}q_1^{\delta_1}\ldots q_s^{\delta_s}.$$
    It then follows that $\alpha_i = g\gamma_i$ and $\beta_i = g\delta_i$.
    Hence if we take $\D = p_1^{\gamma_1}\ldots p_r^{\gamma_r}$ and $\E = q_1^{\delta_1}\ldots q_s^{\delta_s}$, we have $\A = \D^g$ and $\B = \E^g$, as desired.
  \end{proof}
\end{thm}

\end{document}

\message{ !name(hw05-351.tex) !offset(-74) }
