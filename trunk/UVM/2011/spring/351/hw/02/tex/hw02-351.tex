\documentclass[10pt]{amsart}
\usepackage{amsmath,amsthm,amssymb,amsfonts}
\openup 5pt
\author{Blake Farman}
\title{Math-351:\\Homework 2}
\date{February 7, 2011}\pdfpagewidth 8.5in
\pdfpageheight 11in
\begin{document}

\maketitle

\renewcommand{\qedsymbol}{\(\blacksquare\)}
\newcommand{\znz}[1]{\mathbb{Z} / #1\mathbb{Z}}
\newcommand{\mznz}[1]{(\mathbb{Z} / #1\mathbb{Z})^*}
\newcommand{\Z}{\mathbb{Z}}
\newcommand{\Q}{\mathbb{Q}}

\newtheorem*{1}{1}
\begin{1}
  Let $D$ be a square-free integer.  If $K = \Q(\sqrt{D})$, then the ring of integers of $K$, $\mathcal{O}_K$, is precisely $\Z[\omega],$ where 
  \begin{align*}
    \omega = \begin{cases}
      \sqrt{D} & \text{if $D \equiv 2,3 \pmod{4}$},\\
      \frac{1 + \sqrt{D}}{2}& \text{if $D \equiv 1 \pmod{4}$}.
      \end{cases}
  \end{align*}
  \begin{proof}
    Observe first that the monomials
    \begin{align*}
      x^2 - D \quad \text{and} \quad x^2 - x + \frac{1 - D}{4}
    \end{align*}
    have $\omega$ as a root when $D \equiv 2,3 \pmod{4}$ and $D \equiv 1 \pmod{4}$, respectively, and thus $\omega$ is an algebraic integer.
    By results from class we know that $\Z[\omega] \subseteq \mathcal{O}_K$ and hence it remains to show the reverse inclusion.
    
    Let $\alpha = a + b\sqrt{D} \in \Q$ be a given algebraic integer.  
    If $b = 0$, then $\alpha \in \Q$ and so $a \in \Z$.
    Assume that $b \not = 0$.
    Let $p(x) = x^2 + a_1x + a_0 = 0$ and solve $p(\alpha) = 0$ to obtain 
    \begin{align*}
      m_{\alpha}(x) = x^2 -2ax + (a^2 - b^2D).
    \end{align*}
    By results from class we know $m_{\alpha}$ has integer coefficients, so it follows that $2a, a^2 - b^2D \in \Z.$
    Moreover $4(a^2 - b^2D) = (2a)^2 - (2b^2)D \in \Z$ and thus $(2b)^2D$ must also be an integer.
    Indeed, since $D$ was assumed to be square-free, it must be the case that $2b$ is an integer.
    Now write $a = \frac{x}{2}$ and $b = \frac{y}{2}$ for some $x,y \in \Z$.
    Then since $a^2 - b^2D \in \Z$, it follows that $x^2 - y^2D \equiv 0 \pmod{4}.$ 
    Observe that $0$ and $1$ are the only squares modulo 4 and $D$ is not divisible by 4 because it was assumed to be square-free. 
    Hence either
    \begin{enumerate}
    \item
      $D \equiv 2,3 \pmod{4}$ and $x \equiv y \equiv 0\pmod{2}$, or
    \item
      $D \equiv 1 \pmod{4}$ and $x \equiv y \pmod{2}$.
    \end{enumerate}
    If (1), then $a,b \in \Z$ and thus $\alpha \in \Z[\omega]$.  
    If (2), then $a + b\sqrt{D} = \frac{x - y}{2} + y\omega$ with $\frac{x-y}{2}$ and $y$ both integers and thus $\alpha \in \Z[\omega]$.
    Therefore, $\Z[\alpha] = \mathcal{O}_K$, as desired.
  \end{proof}
\end{1}

\end{document}
