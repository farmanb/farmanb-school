\documentclass[10pt]{amsart}
\usepackage{amsmath,amsthm,amssymb,amsfonts,wasysym,mymath}
\openup 5pt
\author{Blake Farman}
\title{Math-331:\\Homework 5}
\date{March 11, 2011}\pdfpagewidth 8.5in
\usepackage[margin=1in]{geometry}
\pdfpageheight 11in
\begin{document}
\maketitle

\newtheorem{thm}{}

\begin{thm}
  \label{Ex1}
  Prove that if $f$ is meromorphic on an open set $U \subseteq \C$ with only a finite number of poles, then $f = g/h$ where $g$ and $h$ are analytic on $U$. 
\begin{proof}
  Let $S$ be the set of singularities of $f$.  
  Each $s \in S$ is a pole, hence we have by our Theorem from class that $f$ has a convergent Laurent series expansion for some deleted neighborhood of $s$,
  \begin{equation}
    \label{1.1}
    f(z) = \frac{c_{-n}}{(z-s)^n} + \frac{c_{-n+1}}{(z-s)^{n-1}} + \ldots + c_0 + c_1(z-s) + \ldots\,.
  \end{equation}
  Multiplying \eqref{1.1} by $(z-s)^n$, which is entire, we obtain
  $$(z-s)^nf(z) = c_{-n} + c_{-n+1}(z-s) + \ldots + c_0(z-s)^n + c_1(z-s)^{n+1} + \ldots\,.$$
  We observe that $(z-s)^nf(z)$ now converges not just on the deleted neighborhood, but at $s$ as well; indeed it takes on the value $c_{-n}$ at $z = s$.
  Hence $(z-s)^nf(z)$ is analytic at $s$.
  
  Let $s_1, s_2, \ldots, s_k$ be an enumeration of $S$ and define the function 
  \begin{equation}
    \label{1.2}
    g(z) = f(z)\cdot\prod_{i=1}^k(z-s_i)^{n_i}, \quad  n_i = -\ord_{s_i}{f}.
  \end{equation}
  If we rewrite \eqref{1.2} as $g(z) = (z-s_i)^{n_i}f(z){\cdot}\prod_{j=1,j \not = i}^k(z-s_j)^{n_j}$, then it is clear by the argument above that $g$ is analytic at each $s_i$.
  Since $f$ was assumed to be meromorphic on $U$, we have shown that $g$ is analytic on $U$.
  Moreover, if we take $h(z) = \prod_{i=1}^k(z-s_i)^{n_i}$, then $f = g/h$ is the ratio of functions analytic on $U$, as was to be shown.
\end{proof}
\end{thm}

\begin{thm}
  \label{Ex2}
  Let $$f(z) = \sum\limits_{n=1}^\infty \dfrac{1}{(z+n)^2}.$$
Prove that $f$ is meromorphic on $\C$ and determine its poles and their orders.
\begin{proof}
  Let $R>0$ be the given radius of a disc, $D$, centered at zero.
  Take $N > 2R$ and define the functions
  \begin{align*}
    g(z) = \sum_{n=1}^N \frac{1}{(z+n)^2} \quad \text{and} \quad h(z) = \sum_{n=N+1}^{\infty} \frac{1}{(z+n)^2}.
  \end{align*}
  Since $g$ is a finite sum it is apparent that it has poles of order 2 at each integer $-N \leq n < 0$.
  Hence $g$ is meromorphic.
  
  On $D$ we have that the terms of $h$ satisfy
  $$\frac{1}{(z+n)^2} \leq \frac{1}{R^2 + n^2} = \frac{1}{n^2\left(1 + \left(\frac{R}{n}\right)^2\right)}.$$
  From $n \geq 2R$ we obtain, with a little algebra, the inequality $$1 + \left(\frac{R}{n}\right)^2 \geq \frac{5}{4}.$$
  Whence the inequality 
  \begin{equation}
    \label{2.1}
    \frac{1}{(z+n)^2} \leq \frac{4}{5n^2}.
  \end{equation}
  The series comprised of the terms on the right hand side of \eqref{2.1} converge by results from Math 333 and thus $h$ converges uniformly by the Weierstrass M-test.
  Now let $\gamma$ be any closed rectangle contained in $D$.
  If $h_m(z) = \sum_{n=N+1}^m 1/(z+n)^2$ is the $m^{th}$ partial sum, then by a result from class we know that 
  $$\int_\gamma h(z)\,dz = \lim_{m\rightarrow\infty} \int_{\gamma} h_m(z)\,dz = 0.$$
  Therefore $h$ is holomorphic by Morera's Theorem on every such disc, $D$, and so $f = g + h$ is meromorphic on $\C$, as was to be shown.
\end{proof}
\end{thm}

\begin{thm}
  \label{Ex3}
  A set $S$ is called {\it star shaped} if there is a point $z_0 \in S$ such that
the line segment between $z_0$ and any point in $S$ is contained in $S$.
Prove that a star shaped set is simply connected.
\begin{proof}
  Let $S$ be a star shaped set and let $z_0$ be the point contained therein to which every other point in the set can be straight-line connected.
  Let $\gamma(t)$ be any closed curve contained in $S$ and reparameterize as necessary so that $\gamma$ is defined on the interval $[0,1]$.
  Consider the function $\Gamma:[0,1]\times[0,1] \longrightarrow S$ defined by $\Gamma(t,u) = u(z_0 - \gamma(t)) + \gamma(t)$.
  It is clear that $\Gamma$ is continuous and satisfies both 
  $$\Gamma(t,0) = \gamma(t) \quad \text{and} \quad \Gamma(t,1) = z_0.$$
  Moreover, for any fixed $t_0 \in [0,1]$ the curve $\Gamma(t_0,u) = u(z_0 - \gamma(t_0)) + \gamma(t_0)$ is the line from $z_0$ to $\gamma(t_0)$, which is contained in $S$ by assumption.
  Therefore $\Gamma$ is a homotopy and $S$ is simply connected, as desired.
\end{proof} 
\end{thm}

\begin{thm}
  \label{Ex4}
  Let $\gamma_0, \gamma_1$ and $\delta_0,\delta_1$ be four closed curves in $U$,
all of which have the same initial point $z_0$ (for some parameterizations).  
Assume that $\gamma_0$ and $\delta_0$ are
homotopic in $U$ with homotopy given by $\Gamma_0$;
and assume that $\gamma_1$ and $\delta_1$ are homotopic in $U$ with homotopy
given by $\Gamma_1$.
Prove that the product curves $\gamma_0 \gamma_1$ and $\delta_0 \delta_1$ are also homotopic
in $U$ by exhibiting an explicit homotopy in terms of $\Gamma_0$ and $\Gamma_1$
(and showing it is a homotopy).
\begin{proof}
  Seeing as we have proven in class that homotopy is independent of parameterization, we can reparameterize $\Gamma_0$ and $\Gamma_1$ as necessary so that they are both defined on the same rectangle, say $[a,b] \times [c,d]$.
  Now define the function $\Gamma: [a,b]\times[c,d] \longrightarrow U$ by $\Gamma(t,u) = \Gamma_0(t,u)\Gamma_1(t,u)$.
  Since $\Gamma_0$ and $\Gamma_1$ are homotopies, we know they are both continuous and they satisfy
  $$\Gamma(t,c) = \gamma_0\delta_0(t) \quad \text{and} \quad \Gamma(t,d) = \gamma_1\delta_1(t).$$
  Moreover, for any $u_0 \in [c,d]$, these homotopies must fix the endpoints and both $\Gamma_0(t,u_0)$ and $\Gamma_1(t,u_0)$ are closed curves, so we know that the product curve $\Gamma(t,u_0)$ is well defined.
  Therefore $\Gamma$ is a homotopy, as desired.
\end{proof}
\end{thm}

\begin{thm}
  \label{Ex5}
  Let $\gamma$ be a closed curve in $U$ with initial point $z_0$, and let
$\gamma^{-}$ denote its reverse curve.
Prove that $\gamma \gamma^{-}$ is null homotopic
in $U$ (exhibit and verify an explicit homotopy).
\begin{proof}
  Begin by reparameterizing as necessary so that $\gamma$ is defined on the interval $[0,1]$.  
  Consider the continuous function $\Gamma: [0,1]\times[0,1] \longrightarrow U$ defined by
  $$\Gamma(t,u) = \gamma((1-u)t)\gamma((1-t)(1-u)).$$
  Using the equality $\gamma^-(t) = \gamma(1-t)$, we have that $\Gamma(t,0) = \gamma\gamma^-(t)$ and that $\Gamma(t,1)$ is the just the point curve $z_0$.
    
  To see $\Gamma$ is a homotopy it remains to show that for any fixed ${u_0} \in [0,1]$ the function $\Gamma_{u_0}(t) = \Gamma(t,{u_0})$ is a closed curve.
  Observe that as $t$ varies over $[0,1]$, the function $\Gamma_{u_0}(t)$ first sweeps out the portion of the the original curve, $\gamma$, between  $z_0$ and $\gamma(1-u_0)$ and then sweeps out the same segment in the reverse direction.
  Since $\Gamma_{u_0}(0) = \Gamma_{u_0}(1)$, it follows from the definition that $\Gamma_{u_0}(t)$ is a closed curve.
  Therefore $\gamma\gamma^-$ is null homotopic.
\end{proof}
\end{thm}

\end{document}
