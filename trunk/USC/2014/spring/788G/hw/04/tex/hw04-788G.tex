\documentclass[10pt]{amsart}
\usepackage{amsmath,amsthm,amssymb,amsfonts,enumerate,mymath,mathtools,tikz-cd,mathrsfs,diagbox}
\openup 5pt
\author{Blake Farman\\University of South Carolina}
\title{Math 788G:\\Homework 04}
\date{May 1, 2014}
\pdfpagewidth 8.5in
\pdfpageheight 11in
\usepackage[margin=1in]{geometry}

\begin{document}
\maketitle

\providecommand{\p}{\mathfrak{p}}
\providecommand{\m}{\mathfrak{m}}
\providecommand{\Deck}[1]{\operatorname{Deck}\left(#1\right)}
%\newcommand{\Res}{\operatorname{Res}}
\newtheorem{thm}{}
\newtheorem{lem}{Lemma}
\newtheorem{prop}{Proposition}
\theoremstyle{definition}
\newtheorem{defn}{Definition}[thm]

\newcommand{\A}{\mathbb{A}}

\begin{thm}[5 Points]\label{Ex1}
  Verify directly, via brute force, that the discriminant of a binary cubic form is $\SL{2}{\Z}$-invariant.
  \begin{proof}
    Let $f(u,v) = au^3 + bu^2v + cuv^2 + dv^3$ be a binary cubic form and let $g = \left(\begin{array}{cc} \alpha & \beta \\ \gamma & \delta\end{array}\right) \in \SL{2}{\Z}$ be given.
      %      \begin{eqnarray*}
      %        f \cdot g(u,v) &=& \left(a\alpha^{3} 
      %        + b \alpha^{2} \gamma 
      %        + c \alpha  \gamma^{2} 
      %        + d \gamma^{3} \right)u^{3}\\
      %        &+& \left(3 a \alpha^{2} \beta 
      %        + b \alpha^{2}  \delta 
      %        + 2 b \alpha \beta \gamma 
      %        + 2 c \alpha \gamma \delta 
      %        + c \beta \gamma^{2} 
      %        + 3 d \gamma^{2} \delta \right)u^{2} v\\
      %        &+& \left(3 a \alpha\beta^{2} 
      %        + 2 b \alpha \beta \delta 
      %        + c \alpha \delta^{2} 
      %        + b \beta^{2} \gamma 
      %        + 2 c \beta \gamma \delta
      %        + 3 d \gamma \delta^{2} \right)u v^{2} \\
      %        &+& \left(a \beta^{3} 
      %        + b \beta^{2} \delta 
      %        + c \beta \delta^{2} 
      %        + d \delta^{3} \right)v^{3}
      %      \end{eqnarray*}
      Let
      \begin{eqnarray*}
        A &=& a\alpha^{3} 
        + b \alpha^{2} \gamma 
        + c \alpha  \gamma^{2} 
        + d \gamma^{3}\\ 
        B &=& 3 a \alpha^{2} \beta 
        + b \alpha^{2}  \delta 
        + 2 b \alpha \beta \gamma 
        + 2 c \alpha \gamma \delta 
        + c \beta \gamma^{2} 
        + 3 d \gamma^{2} \delta\\
        C &=& 3 a \alpha\beta^{2} 
        + 2 b \alpha \beta \delta 
        + c \alpha \delta^{2} 
        + b \beta^{2} \gamma 
        + 2 c \beta \gamma \delta
        + 3 d \gamma \delta^{2}\\
        D &=& a \beta^{3} 
        + b \beta^{2} \delta 
        + c \beta \delta^{2} 
        + d \delta^{3}\\
      \end{eqnarray*}
      so that 
      $$f \cdot g(u,v) = \frac{1}{\alpha\delta - \beta\gamma} f(\alpha u + \beta v, \gamma u + \delta v) = f(\alpha u + \beta v, \gamma u + \delta v) = Au^3 + Bu^2v + Cuv^2 + Du^3.$$
      The discriminant for $f \cdot g$ is
      $$B^2C^2 + 18ABCD - 4AC^3 - 4B^3D - 27A^2D^2,$$ 
      which expands to
      \begin{eqnarray*}
        &-& 27 a^{2} d^{2}\left(\alpha^{6} \delta^{6} 
        - 6 \alpha^{5}\delta^{5} \beta\gamma 
        + 15 \alpha^{4}\delta^{4} \beta^{2}\gamma^{2} 
        - 20 \alpha^{3}\delta^{3} \beta^{3}\gamma^{3} 
        + 15 \alpha^{2}\delta^{2} \beta^{4}\gamma^{4} 
        - 6 \alpha\delta \beta^{5}\gamma^{5} 
        + \beta^{6} \gamma^{6}\right)\\
        &+& 18a b c d \left( \alpha^{6} \delta^{6} 
        - 6 \alpha^{5}\delta^{5} \beta\gamma 
        + 15 \alpha^{4}\delta^{4} \beta^{2}\gamma^{2} 
        - 20 \alpha^{3}\delta^{3} \beta^{3}\gamma^{3} 
        + 15 \alpha^{2}\delta^{2} \beta^{4}\gamma^{4} 
        - 6 \alpha\delta \beta^{5}\gamma^{5} 
        + \beta^{6}\gamma^{6}\right)\\
        &-& 4 a c^{3} \left( \alpha^{6}\delta^{6} 
        - 6 \alpha^{5}\delta^{5} \beta\gamma 
        + 15 \alpha^{4}\delta^{4} \beta^{2}\gamma^{2} 
        - 20 \alpha^{3}\delta^{3} \beta^{3}\gamma^{3} 
        + 15 \alpha^{2}\delta^{2} \beta^{4}\gamma^{4} 
        - 6 \alpha\delta \beta^{5}\gamma^{5} 
        + \beta^{6}\gamma^{6}\right)\\
        &-& 4b^{3} d \left(\alpha^{6} \delta^{6} 
        - 6 \alpha^{5}\delta^{5} \beta\gamma 
        + 15 \alpha^{4}\delta^{4} \beta^{2}\gamma^{2} 
        - 20 \alpha^{3}\delta^{3} \beta^{3}\gamma^{3} 
        + 15 \alpha^{2}\delta^{2} \beta^{4}\gamma^{4} 
        - 6 \alpha\delta \beta^{5}\gamma^{5} 
        + 1 \beta^{6}\gamma^{6}\right)\\
        &+& b^{2} c^{2} \left(\alpha^{6}\delta^{6} 
        - 6 \alpha^{5}\delta^{5} \beta\gamma 
        + 15 \alpha^{4}\delta^{4} \beta^{2}\gamma^{2} 
        - 20 \alpha^{3}\delta^{3} \beta^{3}\gamma^{3} 
        + 15 \alpha^{2}\delta^{2} \beta^{4}\gamma^{4} 
        - 6 b^{2} c^{2}\alpha\delta \beta^{5}\gamma^{5} 
        + \beta^{6} \gamma^{6}\right).
      \end{eqnarray*}
      This then reduces to 
      \begin{equation}\label{1.1}
        (\alpha\delta - \beta\gamma)^6 \left(b^{2}c^{2} + 18abcd - 4 ac^{3} - 4b^{3}d - 27 a^2 d^2 \right) = \det(g)^6\disc{f} = \disc{f}.
      \end{equation}
  \end{proof}
\end{thm}

\begin{thm}[3 Points]
  Let $f(u,v) = u^3 + a_2u^2v + a_3uv^2 + v^3$ be a binary cubic form.
  Prove that its discriminant is equal to the polynomial discriminant obtained by setting either $u$ or $v$ equal to 1.

  \begin{proof}
    Let $g(u) = f(u,1) = u^3 + a_2u^2 + a_3u + 1$ and $h(v) = f(1,v) = v^3 + a_3v^2 + a_2v + 1$.
    By the formula for the discriminants we have
    \begin{eqnarray*}
      \disc{f} &=& a_2^{2}a_3^{2} + 18a_2a_3 - 4 a_3^{3} - 4a_2^{3} - 27,\\
      \disc{g} &=& a_2^{2}a_3^{2} + 18a_2a_3 - 4 a_3^{3} - 4a_2^{3} - 27,\, \text{and}\\
      \disc{h} &=& a_3^{2}a_2^{2} + 18a_3a_2 - 4 a_2^{3} - 4a_3^{3} - 27.
    \end{eqnarray*}
    Provided the ring over which $f$ takes its coefficients is commutative, $\disc{g} = \disc{h} = \disc{f}$.
  \end{proof}
\end{thm}

\begin{thm}[10 points]
  Let $f(u,v) = a_nu^n + a_{n-1}u^{n-1}v + \ldots + a_0v^n$.
  Prove
  \begin{enumerate}[(a)]
  \item
    If $a_n \neq 0$, then
    $$\disc{f} = a_n^{2n - 2}\disc{f(u,1)}$$
  \item
    If $a_n = 0$ and $a_{n-1} \neq 0$, then
    $$\disc{f} = a_{n-1}^2 \disc{a_{n-1}u^{n-1} + a_{n-2}u^{n-2}v + \ldots + a_0v^{n-1}}.$$
  \item
    If $a_n = 0$ and $a_{n-1} = 0$, then
    $$\disc{f} = 0.$$
  \end{enumerate}
  
  \begin{proof}
    \begin{enumerate}[(a)]
    \item\label{1.a}
      We observe that we may write 
      $$f(u,v) = \prod_{i = 1}^n (\alpha_iu - \beta_iv)$$
      and thus $a_n = \alpha_1\alpha_2 \cdots \alpha_n \neq 0$ implies $\alpha_i \neq 0$ holds for each $i$.
      The discriminant for $f$ is given by
      $$\disc{f} = \prod_{i < j} (\alpha_i\beta_j - \alpha_j\beta_i)^2 = \prod_{i < j} \alpha_i^2\alpha_j^2\left(\frac{\beta_j}{\alpha_j} - \frac{\beta_i}{\alpha_i}\right)^2.$$
      
      We first show that $\prod_{i < j} \alpha_i^2\alpha_j^2 = a_n^{2n - 2}$.
      We observe that the distinct elements in this product are the elements in the table below
      \begin{center}
        \begin{tabular}{c|cccccc}
          \diagbox{i}{j} & 2 & 3 & 4 & 5 & \ldots &n\\
          \hline
          1 &  $\alpha_1\alpha_2$ & $\alpha_1\alpha_3$ & $\alpha_1\alpha_4$ & $\alpha_1\alpha_5$ & $\ldots$ & $\alpha_1,\alpha_n$\\
          2 && $\alpha_2\alpha_3$ & $\alpha_2\alpha_4$ & $\alpha_2\alpha_5$ & $\ldots$ & $\alpha_2\alpha_n$\\
          3 &&       & $\alpha_3\alpha_4$ & $\alpha_3\alpha_5$ & $\ldots$ & $\alpha_3\alpha_n$\\
          4 &&       &       & $\alpha_4\alpha_5$ & $\ldots$ & $\alpha_4\alpha_n$\\
          $\vdots$ &&       &       &       & $\ddots$ & $\vdots$\\
          $n-1$&&&&&&$\alpha_{n-1}\alpha_n$
        \end{tabular}
      \end{center}
      We observe then that $\alpha_i$ appears $n - i$ times in the $i^\text{th}$ row, and once in each of the $i - 1$ rows before it, and so it follows that
      $$\prod_{i < j} \alpha_i^2\alpha_j^2 = \left[\prod_{i < j} \alpha_i\alpha_j\right]^2 = \left[\prod_{i < j} \alpha_i^{n-i + i + 1}\right]^2 = \prod_{i < j}(\alpha_i^{n-1})^2 = a_n^{2n-2}.$$
      Hence 
      $$\disc{f} = \prod_{i < j} (\alpha_i\beta_j - \alpha_j\beta_i)^2 = a_n^{2n-2}\prod_{i < j} \left(\frac{\beta_j}{\alpha_j} - \frac{\beta_i}{\alpha_i}\right)^2.$$
      
      %We now observe that if we assume $v \neq 0$, then by homogeneity
      %$$\frac{1}{v^n}f(u,v) = f\left(\frac{u}{v}, 1\right) = \prod_{i = 1}^n \left(\alpha_i \frac{u}{v} - \beta_i\right)$$
      We now obseve that when $v = 1$
      $$f(u,1) = \prod_{i = 1}^n \left(\alpha_iu - \beta_i\right)$$
      has as its roots the elements $\beta_i/\alpha_i$, for $i = 1, \ldots, n$.
      Therefore it follows from 
      $$\disc{f(u,1)} = \prod_{i < j} \left(\frac{\beta_j}{\alpha_j} - \frac{\beta_i}{\alpha_i}\right)^2$$
      that $\disc{f} = a_n^{2n - 2}\disc{f(u,1)}$.
    \item
      We observe first that $g(u,v) = a_{n-1}u^{n-1} + a_{n-2}u^{n-2}v + \ldots + a_0v^{n-1}$ is a binary $(n-1)$-ic form, so by Part~\ref{1.a}
      \begin{eqnarray*}
        a_{n-1}^2 \disc{g} &=& a_{n-1}^2 a_{n-1}^{2(n-1)-2} \disc{g(u,1)}\\
        &=& a_{n-1}^{2n-2} \disc{a_{n-1}u^{n-1} + a_{n-2}u^{n-2} + \ldots + a_0}
      \end{eqnarray*}
      If $a_n = 0$, then we may write
      \begin{eqnarray*}
        f(u,v) &=& v\left(a_{n-1}u^{n-1} + a_{n-2}u^{n-2}v + \ldots + a_0v^{n-1}\right)\\
        &=& (0u - v)\prod_{i = 1}^{n-1} (\alpha_i u - \beta_i v).
      \end{eqnarray*}
      Let $\alpha_n = 0$ and $\beta_n = 1$.
      %First we compute 
      %$$\prod_{i < j < n}(\alpha_i\beta_j - \alpha_j\beta_i)^2.$$
      Since $a_{n-1} \neq 0$ we may assume $\alpha_i \neq 0$ for $\alpha_i < n,$ and so by the same analysis, mutatis mutandis, as in Part~\ref{1.a}, we find that
      $$\prod_{i < j < n}(\alpha_i\beta_j - \alpha_j\beta_i)^2 = a_{n-1}^{2n - 4}\disc{g(u,1)}.$$
      To complete the computation, we observe that for $i < n$, we have
      $$\alpha_i\beta_n -\alpha_n\beta_i = \alpha_i.$$
      Therefore
      \begin{eqnarray*}
        \disc{f} &=& \prod_{i < j}(\alpha_i\beta_j - \alpha_j\beta_i)^2\\
        &=& \prod_{i < n} (\alpha_i\beta_n - \alpha_n\beta_i)^2\prod_{i < j < n}(\alpha_i\beta_j - \alpha_j\beta_i)^2\\
        &=& \prod_{i = 1}^{n-1} \alpha_i^2\prod_{i < j < n}(\alpha_i\beta_j - \alpha_j\beta_i)^2\\
        &=& a_{n-1}^2 a_{n-1}^{2n - 4}\disc{g(u,1)}\\
        &=& a_{n-1}^{2n - 2}\disc{g(u,1)}\\
        &=& a_{n-1}^2 \disc{a_{n-1}u^{n-1} + a_{n-2}u^{n-2}v + \ldots + a_0v^{n-1}}.
      \end{eqnarray*}
    \item
      If $a_n = a_{n-1} = 0$, then 
      $$f(u,v) = v^2\left(a_{n-2}u^{n-2} + a_{n-3}u^{n-3}v + \ldots + a_0v^{n-2}\right),$$
      has a repeated root.
      Therefore $\disc{f} = 0$.
      
    \end{enumerate}
  \end{proof}
\end{thm}

\begin{thm}[5 Points]
  Carry out the details of the computation given on p. 23.5 of the lecture notes.

  \begin{proof}
    Let $f(u,v) = u^3 + bu^2v + cuv^2 + dv^3$ and $g(u) = f(u,1) = (u - r)(u - s)(u - t)$.
    Let $v = u + b/3$ so that
    \begin{eqnarray*}
      g(u) &=& u^3 + bu^2 + 3\left(\frac{b}{3}\right)^2u + \left(\frac{b}{3}\right)^3 + \left(c - 3\left(\frac{b}{3}\right)^2\right)u  + \left(d - \left(\frac{b}{3}\right)^3\right)\\
      &=& \left(u + \frac{b}{3}\right)^3 + \left(c - 3\left(\frac{b}{3}\right)^2\right)u + \left(d - \left(\frac{b}{3}\right)^3\right)\\
      &=& v^3 + \left(c - 3\left(\frac{b}{3}\right)^2\right)\left(v - \frac{b}{3}\right) + \left(d - \left(\frac{b}{3}\right)^3\right)\\
      &=& v^3 + \left(c - \frac{b^2}{9}\right)v + \left(d - \frac{bc}{3} + 2\frac{b^3}{27}\right)\\
      &=& v^3 + \frac{1}{9}\left(9c - b^2\right) + \frac{1}{27}\left(2b^3 - 9bc + 27d\right)\\
      &=& v^3 + Cv + D\\
      &=& (v - R)(v - S)(v - T).
    \end{eqnarray*}
    Expanding and equating coefficients, we see that $\sigma_1 = R + S + T = 0$, $\sigma_2 = RS + RT + ST = C$, and $\sigma_3 = RST = - D$.
    We can compute the discriminant for $f$ using the relations on the elementary symmetric functions and the formula
    \begin{eqnarray*}
      -\disc{f} &=& g^\prime(R)g^\prime(S)g^\prime(T)\\
      &=& (3R^2 + C)(3S^2 + C)(3T^2 + C)\\
      &=& 27R^2S^2T^2 + 9C(R^2S^2 + R^2T^2 + S^2T^2) + 3C^2(R^2 + S^2 + T^2) + C^3\\
      &=& 27\sigma_3^2 + 9C(\sigma_2^2 - 2\sigma_1\sigma_3) + 3C^2(\sigma_1^2 - 2\sigma_2) + C^3\\
      &=& 27(-D)^2 + 9C(C^2) + 3C^2(-2C) + C^3\\
      &=& 27D^2 + 4C^3.
    \end{eqnarray*}
    Therefore $\disc{f} = -27D^2 - 4C^3 = -4 \, b^{3} d + b^{2} c^{2} + 18 \, b c d - 4 \, c^{3} - 27 \, d^{2}$.
  \end{proof}
\end{thm}

\setcounter{thm}{5}
\begin{thm}[5 Points]
  Prove that there are $\frac{1}{3}(p^2 - 1)(p^2 - p)$ irreducible binary cubic forms over $\F_p$.

  \begin{proof}
    Let $X$ be the set of irreducible binary cubic forms.
    By Delone-Faddeev, the set of irreducible binary cubic forms modulo the twisted $\GL{2}{\F_p}$ action is in bijection with the cubic extensions of $\F_p$.
    Since there is a unique cubic extension $\F_{p^3}/\F_p$, the action is necessarily transitive and so for any $f \in X$,
    $$\abs{X} = \left[\GL{2}{\F_p} \colon \GL{2}{F_p}_f \right] = \frac{\abs{\GL{2}{F_p}}}{\abs{\GL{2}{F_p}_f}}$$
    
    We observe that an element $A \in \Mat{2}{\F_p}$ is invertible if and only if the columns of the matrix form a basis for $\F_p^2$.
    There are $p^2 - 1$ non-zero choices for the first column, and excluding the columns not in the span of the first, there $p^2 - p$ choices for the second column.
    Hence $\abs{\GL{2}{\F_p}} = (p^2 - 1)(p^2 - p)$.
    It remains to show that the stabilizer of an element $f \in X$ has order three.
    
    Fix an element $f \in X$ and write $f = \prod_{i=1}^n(r_iu - s_iv)$.
    Let $\theta = s_i/r_i$ and observe that $\F_{p^3} \cong \F_p(\theta) \cong \F_p(\theta^p) \cong \F_p(\theta^{p^2})$ and $\Gal{\F_{p^3}/\F_p} \cong \Z/3\Z$ with generator
    \begin{align*}
      \sigma \colon \F_{p^3} & \rightarrow \F_{p^3}\\
      \alpha &\mapsto \alpha^p.
    \end{align*}
    By the proof of Delone-Faddeev, there exists an element $g = \left(\begin{array}{cc}\alpha & \beta\\ \gamma & \delta\end{array}\right)$ of $\GL{2}{\F_p}$ such that
      $$\sigma(\theta) = \theta^p = \theta \cdot g = \frac{\alpha \theta + \beta}{\gamma \theta + \delta}.$$
      Now we observe that 
      $$\theta^{p^2} = \sigma(\theta^p) = \sigma(\theta \cdot g) = \frac{\sigma\alpha \sigma\theta + \sigma\beta}{\sigma\gamma\sigma\theta + \sigma\delta} = \frac{\alpha\theta^p + \beta}{\gamma\theta^p + \delta} = \theta^p \cdot g$$
      and, similarly,
      $$\theta = \sigma(\theta^{p^2}) = \sigma(\theta^p \cdot g) = \frac{\sigma\alpha \sigma\theta^p + \sigma\beta}{\sigma\gamma\sigma\theta^p + \sigma\delta} = \frac{\alpha\theta^{p^2} + \beta}{\gamma\theta^{p^2} + \delta} = \theta^{p^2} \cdot g.$$
      Then we note that $g$ is not the identity matrix, giving a non-trivial automorphism of $f$ since it fixes the roots.
      The third automorphism of $f$ is the matrix that sends $\theta$ to $\theta^2$.
  \end{proof}
\end{thm}

\setcounter{thm}{8}
\begin{thm}[15 Points]
  Work out several explicit examples of the Delone-Faddeev correspondence over $\Z$.
  Your examples should include reducible and irreducible binary cubic forms, including a binary cubic form which factors as the product of a linear times a quadratic; integral domains, rings with zero divisors but no nilportents, and rings with nilpotents.
  Compute the relevant discriminants, and summarize your conclusions.
  
  \begin{proof}
    \begin{enumerate}[(a)]
    \item Irreducible binary cubic form (integral domain)
      
      Let $\alpha$ be such that $\alpha^3 - \alpha^2 + 1 = 0$.
      This is an irreducible, degree 3 polynomial over $\Z$ of discriminant $-23$.
      Take $\omega = \alpha - 1$ and $\theta = \alpha^2$ so that $\omega\theta = -1$
      Then 
      \begin{eqnarray*}
        \omega^2 &=& \alpha^2 - 2\alpha + 1 = -1 - 2\omega + \theta\\
        \theta^2 &=& \alpha^2 - \alpha - 1 = 2 - \omega + \theta
      \end{eqnarray*}
      giving the form
      $$u^3 + 2u^2v + uv^2 + v^3.$$
    \item Reducible binary cubic form
      \begin{enumerate}[(i)]
      \item
        Linear and quadratic
        
        Let $\rho \neq 1$ be such that $\rho^3 - 1 = 0$.
        We observe that $x^3 - 1 = (x - 1)(x^2 + x + 1)$ is the product of a linear term with an irreducible quadratic and the discriminant is -27.
        We have as a basis ${1, \rho, \rho^2}$ with $\rho\rho^2 = \rho^3 = 1$, so we take $\omega = \rho$ and $\theta = \rho^2$.
        Then
        \begin{eqnarray*}
          \omega^2 &=& \theta\\
          \theta^2 &=& -(-1)\omega\\
        \end{eqnarray*}
        and the corresponding form is
        $$u^3 - v^3 = (u - v)(u^2 + uv + v^2).$$
      \item Nilpotents
        
        Suppose $\alpha^3 - 3\alpha^2 + 3\alpha - 1 = 0$.
        We see that $(\alpha - 1)^3 = 0$, and the discriminant of this polynomial is necessarily 0.
        Let $\omega = \alpha - 3$, $\theta = \alpha^2 + 3$ so that $\omega\theta = -8$.
        Then 
        \begin{eqnarray*}
          \omega^2 &=& \alpha^2 - 6\alpha + 9  = -12 - 6\omega + \theta\\
          \theta^2 &=& 12\alpha^2 -8\alpha + 2 = -48 -8\omega + 12\theta,
        \end{eqnarray*}
        which gives the form
        $$u^3 - 6u^2v + 12uv^2 - 8v^3 = (u - 2v)^3,$$
        which also has discriminant 0.
      \item No nilpotents
        
        Suppose $\alpha^3 - 6\alpha^2 + 11\alpha - 6 = 0$.
        The discriminant of $x^3 - 6x^2 + 11x - 6$ is 4.
        %Factoring, we see $(\alpha - 1)(\alpha - 2)(\alpha - 3) = 0$, so in $\Z[\alpha]$, we have zero divisors, e.g. $(\alpha - 1)(\alpha^2 - 5\alpha + 6) = 0$.
        Since $f$ is not irreducible and $\Z[x]$ is a UFD, it follows that $(f)$ is not prime and so the ring $\Z[\alpha]$ has zero divisors; $\Z[\alpha]$ does not, however, have any nilpotents.
        %One can see this by first observing that $\Z[x]$ is a UFD, hence all the height one primes are principal, and so are the prime elements of $\Z[x]$; that is the primes of the form $(p) \in \operatorname{Spec}Z$ and $(g)$, where $g$ is a monic, irreducible polynomial.
        %The height 2 primes are the maximal primes ($\Z[x]$ has Krull dimension 2) of the form $(g, p)$ with $g$ irreducible over $\Z/p\Z$.
        %Since $f(x) = x^3 - 6x^2 + 11x - 6$ splits over $\F_p$ for every $p$ (i.e. $(x - 1), (x - 2), (x - 3) \in \F_p[x]$ for all $p$), we see that the only primes containing $f$ are the ideals generated by its linear factors.
        %Hence the prime ideals in $\Z[\alpha]$ are precisely $(\alpha - 1)$, $(\alpha - 2)$, and $(\alpha - 3)$.
        One can see this by first observing that the prime ideals $\mathfrak{p_i} = (x - i)$ for $i \in \{1, 2, 3\}$ satisfy $\mathfrak{p}_i + \mathfrak{p}_j = \Z[x]$ for $i \neq j$, so it follows that 
        $$\ker\left(\Z[x] \xrightarrow{\pi} \Z[\alpha]\right) = \bigcap_{i=1}^3 \mathfrak{p}_i = \prod_{i=1}^3 \mathfrak{p}_i.$$
        By the Lattice Isormorphism Theorem, the ideals $\overline{\mathfrak{p}}_i = \pi(\mathfrak{p}_i) = (\alpha - i)$ are prime.
        Therefore
        $$\sqrt{(0)} = \bigcap_{\mathfrak{p} \in \operatorname{Spec} \Z[\alpha]} \mathfrak{p} \subseteq \bigcap_{i=1}^3 \overline{\mathfrak{p}}_i = \prod_{i=1}^3 \overline{\mathfrak{p}}_i = (0),$$
        as desired.
        
        Take $\omega = \alpha - 6$ and $\theta = \alpha^2 + 11$ so that $\omega\theta = -60$.
        Then we have
        \begin{eqnarray*}
          \omega^2 &=& -47 - 12\omega + \theta\\
          \theta^2 &=& -720 -60\omega + 47\theta
        \end{eqnarray*}
        and the corresponding form is
        $$u^3 - 12u^2v + 47uv^2 - 60v^3 = (u - 5v)(u - 4v)(u - 3v),$$
        which also has discriminant 4.
      \end{enumerate}
    \end{enumerate}
  \end{proof}
\end{thm}
\begin{thm}[10 Points]
  Is the following true or false?
  
  Consider the cubic ring $\Z[\alpha]$, where $\alpha^3 + b\alpha^2 + c\alpha + d = 0$.
  Then, the corresponding cubic form is $u^3 + bu^2v + cuv^2 + dv^3$.
  
  \begin{proof}
    This statement is slightly ambiguous.
    Given $\alpha$ as above, the form obtained from Delone-Faddeev may not be $u^3 + bu^2 + cuv^2 + dv^3$ exactly, however it is $\GL{2}{\Z}$-equivalent.
    
    Let $\omega = \alpha + b$ and $\theta = \alpha^2 + c$.
    We show that the set $\{1, \omega, \theta\}$ is a basis for $\Z[\alpha]$.
    We first observe that $\{1, \alpha, \alpha^2\}$ is a basis for $\Z[\alpha]$, so for any $x \in \Z[\alpha]$, we see that
    $$x = \ell + m\alpha + n\alpha^2 = (\ell - mb - nc) + m\omega + n\theta$$
    and if
    $$0 = \ell + m\omega + n\theta = (\ell + mb + nc) + m\alpha + n\alpha^2,$$
    then it follows that $m = n = \ell = 0$.
    Moreover, from
    $$\omega\theta = \alpha^3 + b\alpha^2 + c\alpha + bc = -d + bc \in \Z$$
    we see that this is a normal basis.
    Then we can compute the corresponding form using Delone-Faddeev:
    $$\omega^2 = (-b^2 - c) + 2b(\alpha + b) + (\alpha^2 + c) = (-b^2 - c) - (-2b)\omega + \theta$$
    and
    \begin{eqnarray*}
      \theta^2 &=& \alpha^4 + 2c\alpha^2 + c^2\\
      %&=& \alpha(-b\alpha^2 - c\alpha - d) + 2c\alpha^2 + c^2\\
      %&=& -b\alpha^3 - c\alpha^2 - d\alpha + 2c\alpha^2 + c^2\\
      %&=& -b(-b\alpha^2 - c\alpha - d) + c\alpha^2 -d\alpha + c^2\\
      %&=& b^2\alpha^2 + bc\alpha + bd + c\alpha^2 - d\alpha + c^2\\
      &=& (bd + c^2) + (bc - d)\alpha + (b^2 + c)\alpha^2\\
      %&=& (b^2 + c)\alpha^2 + c(b^2 + c) - c(b^2 + c) + (bc - d)\alpha + b(bc - d) - b(bc - d) + (bd + c^2)\\
      &=& (bd + c^2 - c(b^2 + c) - b(bc - d)) + (bc - d)(\alpha + b) + (b^2 + c)(\alpha^2 + c)\\
      %&=& (b^2 + c)\theta + (bc - d)\omega + (bd + c^2 - c(b^2 + c) - b(bc - d)\\
      %&=& (bd + c^2 - c(b^2 + c) - b(bc - d)) - (d - bc)\omega + (b^2 + c)\theta\\
      &=& (bd + c^2 - c(b^2 + c) - b(bc - d)) - (d - bc)\omega + (b^2 + c)\theta
    \end{eqnarray*}
    give the form
    %$$f(u,v) = u^3 + (-2b)u^2v + (b^2 + 2c)uv^2 + (-bc)v^3$$
    $$f(u,v) = u^3 + (-2b)u^2v + (b^2 + c)uv^2 + (d - bc)v^3.$$
    which is not the desired form.
    However, we show that it is $\GL{2}{\Z}$-equivalent to the desired form.
    
    Let $h(u,v) = u^3 + bu^2v + cuv^2 + dv^3$ and suppose there is an element $g = \left(\begin{array}{cc} \alpha & \beta\\ \gamma & \delta\end{array}\right) \in \GL{2}{\Z}$ such that 
      \begin{eqnarray}\label{10.1}
        f(u,v) &=& h \cdot g(u,v)\\
        &=& \frac{1}{\alpha\delta - \beta\gamma}h(\alpha u + \beta v, \gamma u + \delta v)\\ 
        &=& \pm h(\alpha u + \beta v, \gamma u + \delta v).\label{10.1.3}
      \end{eqnarray}
      If the right-hand side of \eqref{10.1.3} is negative, then by homogeneity we note that
      $$- h(\alpha u + \beta v, \gamma u + \delta v) = (-1)^3 h(\alpha u + \beta v, \gamma u + \delta v) = h(-\alpha u - \beta v, -\gamma u - \delta v)$$
      and so by possibly replacing $g$ with $-g$ we may assume that $f(u,v) = h(\alpha u + \beta v, \gamma u + \delta v)$.
      %Then we see that by Problem~\ref{Ex1}
      %$$\disc{f} = \disc{h(\alpha u + \beta v, \gamma u + \delta v)} = (\alpha\delta - \beta\gamma)^6\disc{h} = (\pm 1)^6\disc{h} = \disc{h}.$$
      %Computing the discriminants, we see that they agree.
      We now note that necessarily $\alpha = 1$ and with this observation, let 
      \begin{eqnarray*}
        A &=& b\gamma 
        + c \gamma^{2} 
        + d \gamma^{3}\\ 
        B &=& 3 \beta 
        + b \delta 
        + 2 b \beta \gamma 
        + 2 c \gamma \delta 
        + c \beta \gamma^{2} 
        + 3 d \gamma^{2} \delta\\
        C &=& 3 \beta^{2} 
        + 2 b \beta \delta 
        + c \delta^{2} 
        + b \beta^{2} \gamma 
        + 2 c \beta \gamma \delta
        + 3 d \gamma \delta^{2}\\
        D &=& \beta^{3} 
        + b \beta^{2} \delta 
        + c \beta \delta^{2} 
        + d \delta^{3}
      \end{eqnarray*}
      so that, as in Exercise~\ref{Ex1}, $h(u + \beta v, \gamma u + \delta v) = Au^3 + Bu^2v + Cuv^2 + Dv^3$.
      With a little help from Sage, we solve the system $A = 1$, $B = -2b$, $C = b^2 + c$, $D = d - bc$ for $\beta$, $\gamma$, $\delta$ to obtain the matrix $g = \left(\begin{array}{cc} 1 & -b\\ 0 & 1\end{array}\right)$.
        It's easy to see that $h(u - bv, v) = f(u,v)$.
        %      \begin{eqnarray*}
        %        \disc{f} - \disc{h} &=& (-4b^3 d + b^2c^2 + 18bcd - 4c^3 - 27d^2)
        %        - (b^2 c^2 - 4b^3 - 4c^3 + 18bc - 27)\\
        %        &=& 4b^3(1 - d) + 18bc(d - 1) + 27(1 - d^2),
        %      \end{eqnarray*}
        %      which is not zero unless $d = 1$ or 
        %      $$d = -\frac{4}{27}b^{3} + \frac{2}{3}bc - 1 = \frac{1}{27}(-4b^3 + 18bc - 27).$$
  \end{proof}
\end{thm}
\end{document}
