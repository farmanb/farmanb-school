\documentclass[10pt]{amsart}
\usepackage{amsmath,amsthm,amssymb,amsfonts,enumerate,mymath,mathtools,tikz-cd,mathrsfs}
\openup 5pt
\author{Blake Farman\\University of South Carolina}
\title{Math 788G:\\Homework 02}
\date{April 21, 2014}
\pdfpagewidth 8.5in
\pdfpageheight 11in
\usepackage[margin=1in]{geometry}

\begin{document}
\maketitle

\providecommand{\Tr}[2]{\operatorname{Tr}_{#1}\left(#2\right)}
\providecommand{\p}{\mathfrak{p}}
\providecommand{\m}{\mathfrak{m}}
\providecommand{\Deck}[1]{\operatorname{Deck}\left(#1\right)}
%\newcommand{\Res}{\operatorname{Res}}
\newtheorem{thm}{}
\newtheorem{lem}{Lemma}
\newtheorem{prop}{Proposition}
\theoremstyle{definition}
\newtheorem{defn}{Definition}[thm]

\newcommand{\A}{\mathbb{A}}

\begin{thm}\label{Ex1}
	Let $R$ be the algebraic integers in an algebraic number field.
	Using Theorem 1 below, prove Theorem 2.
	
	\newtheorem{tthm}{Theorem}
	\begin{tthm}\label{T.1.1}
		For any ideal $B$ in $R$, there exists an ideal $C \neq (0)$ in $R$ such that $BC = (a)$ for some $a \in \Z$.
	\end{tthm}

	\begin{tthm}
		Let $B$, $C$, and $D$ be ideals in $R$ with $D \neq (0)$.
		If $BD = CD$, then $B = C$.
	\end{tthm}

	\begin{proof}
		By Theorem~\ref{T.1.1} there exists some ideal $E \neq (0)$ in $R$ and some integer $a$ such that $DE = (a)$.
		Since $D$ and $E$ are both not the zero ideal, there exist non-zero elements $e \in E$, $d \in D$ so that $0 \neq ed \in (a)$.
		That $a \neq 0$ then follows from the fact that $R$ is a domain.
		Now we have
		$$B(a) = BDE = CDE = C(a).$$
		Given $b \in B$, there exist elements $r_i \in R$ and $c_i \in C$ such that
		$$ba = \sum_{i=1}^n r_ic_ia_i = a\sum_{i=1}^n r_ic_i,$$
		Hence
		$$a(b - \sum_{i=1}^n r_ic_i) = 0.$$
		Since $R$ is a domain, $a \neq 0$ implies that $b = \sum_{i=1}^n r_ic_i$ and so $B \subseteq C$.
		The reverse inclusion follows from the same argument, mutatis mutandis.
	\end{proof}
\end{thm}

\begin{thm}\label{Ex2}
	Let $a$, $b$, and $d$ be integers (not necessarily positive) with $a^2 - b^2d$ divisible by 4.
	Explain why
	$$2013\left(\frac{a + b\sqrt{d}}{2}\right)^{2014} -2012\left(\frac{a + b\sqrt{d}}{2}\right)^{2013} + 2013\left(\frac{a - b\sqrt{d}}{2}\right)^{2014} - 2012\left(\frac{a - b\sqrt{d}}{2}\right)^{2013}$$
	is in $\Z$

	\begin{proof}
		Let $\alpha = (a + b\sqrt{d})/2$, $\overline{\alpha} = (a + b\sqrt{d})/2$, $\beta = 2013\alpha^{2014} - 2012\alpha^{2013} + 2013\overline{\alpha}^{2014} - 2012\overline{\alpha}^{2013}$, and let $K = \Q(\sqrt{d})$. 
		First we observe that 
		\begin{eqnarray*}
			(x - \alpha)(x - \overline{\alpha}) &=& x^2 - (\alpha + \overline{\alpha})x + \alpha\overline{\alpha}\\
			&=& x^2 - ax + a^2 - b^2d \in \Z[x]
		\end{eqnarray*}
		so that $\alpha \in \mathcal{O}_K$ and $[K : \Q] = 2$.
		If we let $\sigma$ be the generator of $\Gal{K/\Q}$, then we see that $\sigma\alpha = \overline{\alpha}$ so that
		$$\sigma\left(\beta\right) = 2013\overline{\alpha}^{2014} - 2012\overline{\alpha}^{2013} + 2013\alpha^{2014} - 2012\alpha^{2013} = \beta$$
		Therefore $\beta \in \Q \cap \mathcal{O}_K$ implies that $\beta \in \Z$.
	\end{proof}
\end{thm}

\begin{thm}\label{Ex3}
\end{thm}

\begin{thm}\label{Ex4}
\end{thm}

\end{document}
