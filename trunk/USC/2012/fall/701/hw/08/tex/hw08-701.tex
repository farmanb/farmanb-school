\documentclass[10pt]{amsart}
\usepackage{amsmath,amsthm,amssymb,amsfonts,enumerate,mymath}
\openup 5pt
\author{Blake Farman\\University of South Carolina}
\title{Math 701:\\Homework 08}
\date{November 9, 2012}
\pdfpagewidth 8.5in
\pdfpageheight 11in
\usepackage[margin=1in]{geometry}

\begin{document}
\maketitle

\newtheorem{thm}{}
\newtheorem{lem}{Lemma}

\begin{thm}
  Let $n \geq 5$.
  Show that $A_N = (S_n)^\prime = [S_n, S_n]$.
  \begin{proof}
    First consider the group homomorphism $\varepsilon \colon S_n \rightarrow \left\{-1,1\right\}$.
    Observe that the codomain is isomorphic to $\Z/2\Z$ and thus abelian.
    Then for any two elements $\sigma, \tau$ of $S_n$ we have 
    $$
    \varepsilon\left([\sigma,\tau]\right) = \varepsilon\left(\sigma\tau\sigma^{-1}\tau^{-1}\right) 
    = \varepsilon\left(\sigma\right)\varepsilon\left(\tau\right)\varepsilon\left(\sigma^{-1}\right)\varepsilon\left(\tau^{-1}\right) 
    = \varepsilon\left(\sigma\right)\varepsilon\left(\sigma\right)^{-1}\varepsilon\left(\tau\right)\varepsilon\left(\tau^{-1}\right) 
    = 1.
    $$
    Hence $(S_n)^\prime \leq A_n$.
    By Exercise 5 of Homework 6, $(S_n)^\prime$ contains a 3-cycle and hence is a non-trivial subgroup of $A_n$.
    Moreover, $(S_n)^\prime$ is normal in $S_n$ and thus normal in the simple group $A_n$.
    Therefore $A_n = (S_n)^\prime$.
  \end{proof}
\end{thm}

\begin{thm}
  \begin{enumerate}[(a)]
  \item
    Construct a composition series for $S_4$.
    Conclude that $S_4$ is solvable.
  \item
    Show that $S_n$ is not solvable for $n \geq 5$.
  \end{enumerate}
  \begin{proof}
    \begin{enumerate}[(a)]
    \item
      First consider the subgroup $\left<\left(1 \; 2\right)\left( 3 \; 4\right),  \left(1 \; 3\right)\left( 2 \; 4\right)\right>$ of $A_n$.
      Observe that this is an abelian group with four elements of order two (isomorphic to the Klein four-group), with fourth element $$\left(1 \; 2\right)\left( 3 \; 4\right)\left(1 \; 3\right)\left( 2 \; 4\right) = \left(1 \; 3\right)\left( 2 \; 4\right)\left(1 \; 2\right)\left( 3 \; 4\right) = \left(1 \; 4\right)\left( 2 \; 3\right).$$
      By Exercise 6 of Homework 6, this group is normal in $A_4$.
      Note also that the index of $\left<\left(1 \; 2\right)\left( 3 \; 4\right),  \left(1 \; 3\right)\left( 2 \; 4\right)\right>$ in $A_n$ is 3.
      
      Consider the composition series 
      $$\left<1\right> \unlhd \left<\left(1 \; 2\right)\left( 3 \; 4\right)\right> \unlhd \left<\left(1 \; 2\right)\left( 3 \; 4\right),  \left(1 \; 3\right)\left( 2 \; 4\right)\right> \unlhd A_4 \unlhd S_4.$$
      The factors are isomorphic to $\Z/2\Z$, $\Z/2\Z$, $\Z/3\Z$, and $\Z/2\Z$, respectively.
      Each of these are abelian, therefore $S_4$ is solvable.
    \item
      From the Corollary to the proof that $A_n$ is simple for $n \geq 5$, the only normal subgroups of $S_n$ are $\left< 1 \right>$, the simple group $A_n$, and $S_n$.
      Hence the only possible choices for series are $\left< 1 \right> \unlhd A_n \unlhd S_n$ and $\left< 1 \right> \unlhd S_n$.
      Since neither $A_n$ nor $S_n$ are abelian, $S_n$ is not solvable. 
    \end{enumerate}
  \end{proof}
\end{thm}

\begin{thm}
  Let $p$ be prime.
  Show that a finite $p$-group is solvable.
  \begin{proof}
  \end{proof}
\end{thm}

\begin{thm}
  A non-trivial abelian group $A$ is said to be divisible if and only if for each element $a \in A$ and for every $k \in \Z$, there exists $x \in A$ with $kx = a$.
  \begin{enumerate}[(a)]
  \item
    Prove that $\Q$, the additive group of rationals, is divisible.
  \item
    Prove that no finite abelian group is divisible.
  \item
    Prove that the quotient of a divisible group by a proper subgroup is divisible.
    Deduce that $\Q/\Z$ is divisible.
  \end{enumerate}
  \begin{proof}
    \begin{enumerate}[(a)]
    \item
      Let $a \in \Q$ and $k \in \Z$ be given.
      Since $a/k$ is an element of $\Q$, then $$k\left(\frac{a}{k}\right) = \underbrace{\frac{a}{k} + \frac{a}{k} + \ldots + \frac{a}{k}}_{k} = a$$
      shows that $\Q$ is divisible.
    \item
      Let $G$ be a finite abelian group of order $n > 1$.
      Let $0 \not = g \in G$ be given.
      Observe that for every element $g^\prime$ of $G$, $ng^\prime = 0 \not = g$.
      Hence there exists an element $g$ of $G$ and an integer $n$ such that for every element $g^\prime$ of $G$, $g \not = ng^\prime$.
      Therefore $G$ is not divisible.
    \item
      Let $G$ be a divisible group and let $H < G$ be given.
      Let $\pi \colon G \rightarrow G/H$ be the canonical projection homomorphism.
      Fix an element $g + H$ of $G/H$ and an integer $k$.
      Since $G$ is divisible, there exists an element $g^\prime$ of $G$ such that $g = kg^\prime$.
      Observe that $$\pi(kg^\prime) = \pi\left(\sum_{i=1}^k g^\prime\right) = \sum_{i=1}^k\pi\left(g^\prime\right) = k\pi\left(g^\prime\right) = k(g^\prime + H)$$ holds because $\pi$ is a homomorphism.
      Therefore
      $$k(g^\prime + H) = \pi(kg^\prime) =  \pi(g) = g + H$$
      shows $G/H$ is a divisible group.
      
      To see that $\Q/\Z$ is divisible, observe that $\Q$ is not cyclic by Exercise 5 of Homework 7.
      Therefore the containment $\left< 1 \right> = \Z \leq \Q$ must in fact be proper and the result follows directly from (a) and (c).
    \end{enumerate}
  \end{proof}
\end{thm}

\begin{lem}\label{normalizer}
  Let $G$ be a group and let $H,K \leq G$.
  If $H \leq N_G(K)$, then $HK$ is a subgroup.
  
  \begin{proof}
    Observe that by Exercise 4 of Homework 2 that it suffices to show $HK = KH$.
    Towards that end, let $h \in H$ and $k \in K$ be given.
    Consider the element $hk \in HK$.
    Since $H \leq N_G(K)$, we have $hkh^{-1} \in K$.
    Hence $$hk = hk(h^{-1}h) = (hkh^{-1})h \in KH$$ implies $HK \subseteq KH$.
    
    To see the reverse containment, consider the element $kh \in KH$.
    Since $H \leq N_G(K)$, it follows that $h^{-1}kh \in K$.
    Hence $$kh = (hh^{-1})kh = h(h^{-1}kh) \in HK$$ implies $KH \subseteq HK$.
    Therefore $KH = HK$, as desired.
   \end{proof}
 \end{lem}
 \begin{thm}
   Let $G = \SL{2}{\Z/3\Z}$.
   \begin{enumerate}[(a)]
   \item
     Find $\abs{G}$.
   \item
     Give all Sylow 3-subgroups of $G$.
   \item
     Prove that the subgroup of $G$ generated by 
     $\left(\begin{array}{cc}
       0 & -1\\
       1 & 0
     \end{array}\right)$ and 
     $\left(\begin{array}{cc}
       1 & 1\\
       -1 & -1
     \end{array}\right)$ 
     is the unique 2-Sylow subgroup of $G$.
   \item
     Show that $\cntr{G} = \left\{\pm \left(\begin{array}{cc}
       1 & 0\\
       0 & 1\\
     \end{array}\right)\right\}$.
     Conclude that $\PSL{2}{\Z/3\Z} = G/\cntr{G} \cong A_4$.
   \end{enumerate}
   \begin{proof}
     \begin{enumerate}[(a)]
     \item
       First observe that there are $3^4 = 81$ matrices in $\M{2}{\Z/3\Z}$.
       We first count the number of matrices of determinant zero, thereby determining the number of matrices in $\GL{2}{\Z/3\Z}$.  
       Then since $\Det \colon \GL{2}{\Z/3\Z} \rightarrow (\Z/3\Z)^\times \cong \Z/2\Z$ is a homomorphism with $\ker\Det = \SL{2}{\Z/3\Z}$, we will have $\abs{\SL{2}{\Z/3\Z}} = \abs{\GL{2}{\Z/3\Z}}/2$ by the First Isomorphism Theorem.

       Towards that end, observe that there are four possible choices of matrices with determinant zero: the zero matrix, matrices with three zeroes, matrices with two zeroes, and matrices with all non-zero entries.
       There are four possible arrangements of a matrix with three zeroes and two choices for the fourth entry.
       Hence there are eight such matrices.
       For matrices with two zeroes, there four possible arrangements of the zeroes and two choice for the remaining two entries.
       Hence there are 16 such matrices.
       Finally, for matrices with all non-zero entries, consider the equation $ad = bc$.
       There are two choices for $ad$--1 or 2--and two possible choices for the pair $(a,d)$: $(1,1)$, $(2,2)$, $(1,2)$, and $(2, 1)$.
       Similarly, for each pair $(a,d)$, there are two corresponding choices for the pair $(c,d)$.
       Hence there are eight such matrices.
       Therefore there are $16 + 8 + 8 + 1 = 33$ elements of $\M{2}{\Z/3\Z}$ with determinant zero, $81 - 33 = 48$ elements in $\GL{2}{\Z/3\Z}$, and 24 matrix elements in $\SL{2}{\Z/3\Z}$.
     \item
       By the Sylow Theorems, the number of Sylow 2-subgroups is either 1 or 3 and the number of 3-subgroups is either 1 or 4.
       With some computation, one finds the following four matrices of order 3:\\
       \begin{center}
         \begin{tabular}{c|c}
           $A$ & $A^2 = A^{-1}$\\
           \hline
           $\left(\begin{array}{cc}
             1 & 1\\
             0 & 1\\
           \end{array}\right)$& $\left(\begin{array}{cc}
             1 & 2\\
             0 & 1\\
           \end{array}\right)$\\
           \hline
           $\left(\begin{array}{cc}
             1 & 0\\
             2 & 1\\
           \end{array}\right)$ & $\left(\begin{array}{cc}
             1 & 0\\
             1 & 1\\
           \end{array}\right)$\\
           \hline
           $\left(\begin{array}{cc}
             2 & 1\\
             2 & 0\\
           \end{array}\right)$ & $\left(\begin{array}{cc}
             0 & 2\\
             1 & 2\\
           \end{array}\right)$\\
           \hline
           $\left(\begin{array}{cc}
             0 & 1\\
             2 & 2\\
           \end{array}\right)$ & $\left(\begin{array}{cc}
             2 & 2\\
             1 & 0\\
           \end{array}\right)$\\
         \end{tabular}
       \end{center}
       Since $n_3$ is either 1 or 4, these four must be the only Sylow 3-subgroups.
     \item
       Observe that the Sylow 2-subgroups each have 7 distinct non-identity elements.  
      Since there are four Sylow 3-subgroups, each with two distinct elements, if $n_2 = 3$, then there are at least $21 + 9 + 1 = 31$ elements in a group of order 24, a contradiction.
      Hence $n_2 = 1$ and so it suffices to show that the group generated by     
      $\left(\begin{array}{cc}
        0 & 2\\
        1 & 0
      \end{array}\right)$ and 
      $\left(\begin{array}{cc}
        1 & 1\\
        2 & 2
      \end{array}\right)$ is of order eight.
      
      
    \end{enumerate}
  \end{proof}
\end{thm}

\begin{thm}
  \begin{enumerate}[(a)]
  \item
    Let $p$ be prime and let 
    $$U_n(\Z/p\Z) = \left\{(x_{ij}) \in \GL{n}{\Z/p\Z} \;\middle\vert\; x_{ij} = 0 \text{ for all } i > j;\; x_{ii} = 1 \text{ for all } i \right\}.$$
    Show that $U_n(\Z/p\Z)$ is a Sylow $p$-subgroup of $\GL{N}{\Z/p\Z}$.
    It may be helpful to know that 
    $$\abs{\GL{N}{\Z/p\Z}} = \prod_{i=0}^{n-1}(p^n - p^i).$$
  \item
    Prove that the number of Sylow $p$-subgroups of $\GL{2}{\Z/p\Z}$ is $p + 1$.
  \end{enumerate}
  \begin{proof}
  \end{proof}
\end{thm}
\end{document}
