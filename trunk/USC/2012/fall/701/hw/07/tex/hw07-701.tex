\documentclass[10pt]{amsart}
\usepackage{amsmath,amsthm,amssymb,amsfonts,enumerate,mymath}
\openup 5pt
\author{Blake Farman\\University of South Carolina}
\title{Math 701:\\Homework 06}
\date{October 12, 2012}
\pdfpagewidth 8.5in
\pdfpageheight 11in
\usepackage[margin=1in]{geometry}

\begin{document}
\maketitle

\newtheorem{thm}{}
\newtheorem*{lem}{Lemma}

\begin{thm}
  Determine how many non-isomorphic abelian groups there are of order 360.
  \begin{enumerate}[(a)]
  \item
    Write out each in elementary divisor form.
  \item
    Write out each in invariant factor form.
  \end{enumerate}
  
  \begin{proof}
    \begin{enumerate}[(a)]
    \item
      The elementary divisors are given by the table below.\\
      \begin{center}
        \begin{tabular}{l| l | l}
          $p^\alpha$ & Partitions of $\alpha$ & Groups\\
          \hline
          $2^3$ & 3; 2, 1; 1, 1, 1 & $C_8$; $C_4 \times C_2$; $C_2 \times C_2 \times C_2$\\
          $3^2$ & 2; 1, 1 & $C_9$; $C_3 \times C_3$\\
          $5$ & 1 & $C_5$.
        \end{tabular}
      \end{center}
      The six abelian groups of order 360 are, up to isomorphism,\\
      \begin{center}
        \begin{tabular}{l l}
          $C_8 \times C_9 \times C_5$ & $C_8 \times C_3 \times C_3 \times C_5$\\
          $C_4 \times C_2 \times C_9 \times C_5$ & $C_4 \times C_2 \times C_3 \times C_3 \times C_5$\\
          $C_2 \times C_2 \times C_2 \times C_9 \times C_5$ & $C_2 \times C_2 \times C_2 \times C_3 \times C_3 \times C_5$.
        \end{tabular}
      \end{center}
    \item
      The invariant factors and the six isomorphism types are given by the table below.\\
      \begin{center}
        \begin{tabular}{l | l}
          Invariant Factors & Groups\\
          \hline
          $2^3 \cdot 3^2 \cdot 5$ & $C_{360}$\\
          $2^3 \cdot 3 \cdot 5$, 3 & $C_{120} \times C_3$\\
          $2^2 \cdot 3^2 \cdot 5$, 2 & $C_{180} \times C_2$\\
          $2^2 \cdot 3 \cdot 5$, $2 \cdot 3$ & $C_{60} \times C_6$\\
          $2 \cdot 3^2 \cdot 5$, 2, 2 & $C_{90} \times C_2 \times C_2$\\
          $2 \cdot 3 \cdot 5$, $2\cdot 3$, 2 & $C_{30} \times C_6 \times C_2$.\\
        \end{tabular}
      \end{center}
    \end{enumerate}
  \end{proof}
\end{thm}

\begin{thm}
  Let $G$ be a finite abelian group, let $p$ be a prime, and suppose that for all $g \in G$, we have $g^p = 1$.
  Show that there exists an integer $s \geq 0$ with $G$ isomorphic to the direct product of $s$ copies of $\Z/p\Z$.
  \begin{proof}
    Observe that for all $g \in G$, the order of $g$ divides $p$ and thus is either 1 or $p$.
    If the order of each element is 1, then $G$ is trivial and $G$ is isomorphic to zero copies of $\Z/p\Z$.
    Assume $G$ is non-trivial and then the order of each element of $G$ is $p$.
    Hence each non-identity element generates a subgroup of order $p$.
    
    Consider two non-identity elements $g,h \in G$.
    If $1 \not = \alpha \in \left<g\right> \cap \left<h\right>$, then it follows from closure under the group operation that $\left<\alpha\right> \leq \left<g\right> \cap \left<h\right>$ is a subgroup of order $p$ contained in the intersection of two groups of order $p$.
    Hence $\left<\alpha\right> = \left<g\right> = \left<h\right>$.
    
    Consider for some integer $s \geq 1$ the list of all elements of $G$, $g_1, g_2, \ldots, g_s$, that generate pairwise disjoint subgroups.
    Note that by the argument above, these elements necessarily generate all the elements of $G$.
    Since $G$ is abelian, $\left<g_i\right> \unlhd G$ holds for each $1 \leq i \leq s$, and thus $\left< g_1 \right>\left< g_2 \right> \cdots \left< g_s \right> \leq G$.
    Observe that for each $1 \leq i \leq s$ that $1 \in \left<g_i\right>$.
    Hence for each $1 \leq i \leq s$, we may write $g_i = 1 \cdot 1 \cdots g_i \cdots 1$, from which it follows that $g_i \in \left< g_1 \right> \cdots \left< g_s \right>$.
    Therefore $$G =  \left< g_1 \right> \cdots \left< g_s \right> \cong \underbrace{\Z/p\Z \times \ldots \times \Z/p\Z}_s.$$
  \end{proof}
\end{thm}

\begin{thm}
  Let $U$ and $V$ be non-abelian simple groups, and let $G = U \times V$.
  Show that $G$ has exactly 4 different normal subgroups.
  \begin{proof}
    Let $N \unlhd G = U \times V = \overline{U}\overline{V}$.
    Observe that $N \leq \overline{U}$ or $N \leq \overline{V}$ imply $N \unlhd \overline{U} \cong U$ and $N \unlhd \overline{V} \cong V$, respectively.
    Since $U$ and $V$ are both simple, both imply $N = \left< 1_G \right>$.

    Assume $N \not = \left< 1_G \right>$, and let $u \in U$ and $v \in V$ be such that $1_G \not = n = (u,v) \in N$.
    Observe that $\cntr{U} \unlhd U$ and $\cntr{V} \unlhd V$ imply, by the same argument above, that $\cntr{U} = \left< 1_U \right>$ and $\cntr{V} = \left< 1_V \right>$.
    If $u \not = 1_U$, then there exists some element $u^\prime \in U$ such that $[n,\overline{u}] \not = 1_G$, where $\overline{u} = (u^\prime, 1_V)$.
    Now it follows from the normality of $N$ and $\overline{U}$ that $$[n,\overline{u}] = (n\overline{u}n^{-1})\overline{u}^{-1} = n(\overline{u}n^{-1}\overline{u}^{-1}) \in \overline{U} \cap N.$$
    By the Diamond Isomorphism Theorem, we have that $\left<1_G\right> \not = N \cap \overline{U} \unlhd \overline{U}$, which implies, by the simplicity of $\overline{U}$, that $N \cap \overline{U} = \overline{U} \leq N$.
    Now by the Lattice Isomorphism Theorem, we have that $N/\overline{U} \unlhd G/\overline{U} \cong \overline{V}$.
    Since $\overline V$ is simple, it follows that either $N = G$ or $N = \overline{U}$.
    
    By the same argument, mutatis mutandis, we find that if $v \not = 1_V$, then $N = \overline{V}$ or $N = G$.
    Therefore the only four normal subgroups of $G$ are $\left<1_G\right>$, $\overline{U}$, $\overline{V}$, and $G$.
  \end{proof}
\end{thm}

\begin{thm}
  Let $n \geq 3$ be an integer.
  Show that 
  $$\cntr{D_{2n}} = \left\{\begin{array}{ll}
  \left<1\right> & n \text{ odd},\\
  \left\{1, r^k\right\} & n = 2k \text{ (even) },
  \end{array}\right.$$
  where $r$ has order $n$.
  \begin{proof}
    Since $D_{2n} = \left<r,s \;\middle\vert\; r^n = s^2 = 1, rs = sr^{-1} = sr^{n-1} \right>$, and since $n - 1 > 1$ by assumption, the commutation relation $rs = sr^{n-1}$ implies $s$ does not commute with $r$.
    Hence it suffices to show that some power of $r$ commutes with $s$.
    Observe that $r^\alpha s = s(r^\alpha)^{-1}$ if and only if $r^\alpha$ has order 2.
    By Homework 1, exercise 7, the order of $r^\alpha$ is $n/\gcd(\alpha,n)$, which is 2 if and only if $n = 2k$ and $\alpha = k$.
    Therefore   $$\cntr{D_{2n}} = \left\{\begin{array}{ll}
    \left<1\right> & n \text{ odd},\\
    \left\{1, r^k\right\} & n = 2k \text{ (even) }.
    \end{array}\right.$$
  \end{proof}
\end{thm}

\begin{thm}
  The group $G$ is finitely generated if and only if there is a finite subset $A \subseteq G$ with $G = \left< A \right>$.
  
  \begin{enumerate}[(a)]
  \item
    Prove that every finitely generated subgroup of $(\Q, +)$ is cyclic.
    Conclude that $\Q$ is not finitely generated.
  \item
    Use the first part of (a) to show that $\Q \not \cong \Q \times \Q$.
  \end{enumerate}
  \begin{proof}
    \begin{enumerate}[(a)]
    \item
      Let $A = \left\{\frac{a_1}{b_1}, \frac{a_2}{b_2}, \ldots, \frac{a_n}{b_n} \;\middle\vert\; a_i, b_i \in \mathbb{Z}, \gcd(a_i,b_i) = 1\right\} \subset \Q$.
      Consider the element $1/d \in \Q$, where $d = b_1 b_2 \ldots b_n$.
      Observe that for any $1 \leq i \leq n$, $\ell = a_i(d/b_i) \in \Z$ and thus the elment $1/d$ can be used to generate the elements of $A$ by adding $1/d$ to itself $\ell$ times.
      Namely,
      $$\sum_{k=1}^{\ell} \frac{1}{d} = \left(\frac{a_i d}{b_i}\right)\frac{1}{d} = \frac{a_i}{b_i}.$$
      
      To see that $\Q$ is not finitely generated, observe that since $\Q \leq \Q$ holds trivially.
      Hence if $\Q$ were finitely generated, then $\Q$ would be cyclic.
      Therefore It suffices, by the contrapositive, to show that $\Q$ is not cyclic.

      Towards that end, suppose to the contrary that $\Q$ were cyclic.
      Then there exist relatively prime integers $a$ and $b$ such that $\left<a/b\right> = \Q$.
      Let $q \in \Z$ be a prime such that $\gcd(q,b) = 1$.
      Since $\Q$ is generated by $a/b$, there exists some integer $c$ such that 
      $$c\left(\frac{a}{b}\right) = \underbrace{\frac{a}{b} + \frac{a}{b} + \ldots + \frac{a}{b}}_c = \frac{1}{q}.$$
      Then adding $c(a/b)$ to itself $b$ times, we we obtain 
      $$ca = \underbrace{c\left(\frac{a}{b}\right) +  c\left(\frac{a}{b}\right) + \ldots + c\left(\frac{a}{b}\right)}_b =  \frac{b}{q}.$$
      Since $ca \in \Z$, this implies $q$ divides $b$, contradicting the choice of $q$.
      Therefore $\Q$ is not cyclic.
    \item
      Observe that if $\Q$ were isomorphic to $\Q \times \Q$, then finitely generated subsets of $\Q \times \Q$ would be cyclic.
      Suppose this were the case and consider $A = \left\{(1/p, 0), (0, 1/q)\right\} \subseteq \Q \times \Q$, where $p, q \in \Z$ are prime.
      Suppose further for some $a,b,c,d \in \Z$ with $\gcd(a,b) = 1$ and $\gcd(c,d) = 1$ that $\left<(\frac{a}{b}, \frac{c}{d})\right> = \left<A\right>$.
      Then there exist non-zero integers $m$ and $n$ such that 
      $$m\left(\frac{a}{b}, \frac{c}{d}\right) = \left(\frac{ma}{b}, \frac{mc}{d}\right) = \left(\frac{1}{p},0\right) \;\text{ and }\; n\left(\frac{a}{b}, \frac{c}{d}\right) = \left(\frac{na}{b}, \frac{nc}{d}\right) = \left(0,\frac{1}{q}\right).$$
      However, this implies $mc = na = 0$, from which it follows that $a = c = 0$.
      But $(0,0)$ generates only the trivial group, a contradiction.
      Therefore $\Q \not \cong \Q \times \Q$.
    \end{enumerate}
  \end{proof}
\end{thm}

\begin{thm}
  {\bf Semi-direct products.}
  Let $H$ and $N$ be finite groups.
  Suppose that there exists a homomorphism $\phi \colon H \rightarrow \Aut{N}$ defined by $\phi \colon h \mapsto \phi_h$.
  Let $G = \left\{ (n,h) \;\middle\vert\; n \in N, h \in H \right\}$, and define an operation $\cdot$ on $G$ by $(n_1,h_1)\cdot(n_2,h_2) = (n\phi_{h1}(n_2), h_1h_2)$.
  \begin{enumerate}[(a)]
  \item
    Show that $G$ is a group of size $\abs{G} = \abs{N}\cdot\abs{N}$.
  \item
    Let $\overline{N} = \left\{(n, 1_H) \;\middle\vert\; n \in N \right\}$, and let $\overline{H} = \left\{(1_N, h) \;\middle\vert\; h \in H \right\}$.
    Show that $\overline{N} \cong N$ and $\overline{H} \cong H$.
  \item
    Show that $\overline{N} \unlhd G$.
    \newcounter{saveenum}
    \setcounter{saveenum}{\value{enumi}}
  \end{enumerate}
  The group $G$ is the semi-direct product of $N$ and $H$ with respect to $\phi$.
  We write $G = N \rtimes_\phi H$.
  \begin{enumerate}[(a)]
    \setcounter{enumi}{\value{saveenum}}
  \item
    Let $N \unlhd G$.
    Suppose that $H \subseteq G$ is a subgroup with $H \cap N = \left< 1 \right>$ and $NH = G$.
    Show that $G \cong N \rtimes_\phi H$ with $\phi \colon H \rightarrow \Aut{N}$ defined by $\phi \colon h \mapsto \phi_h$ such that $\phi_h(n) = hnh^{-1}$.
  \item
    Let $n \geq 3$ be an integer.
    Express the dihredral group, $D_{2n}$, as a semi-direct product.
  \item
    Let $p \not = q$ be primes
    Classify groups of order $pq$ up to isomorphism (we have done most of the work already; I am asking you to recognize the non-abelian isomorphism class as a semi-direct product).
  \end{enumerate}
  \begin{proof}
    \begin{enumerate}[(a)]
    \item
      Consider the element $(1_N, 1_H) \in G$.
      First observe that since $\phi$ is a homomorphism, for any $h \in H$ it follows from 
      $$\phi(1_Hh) = \phi(1_H)\phi(h) = \phi_{1_H}\phi_h = \phi_h = \phi(h)$$
      and
      $$\phi(h1_H) = \phi(h)\phi(1_H) = \phi_h\phi_{1_H} = \phi_h = \phi(h)$$
      that $\phi_{1_H} = 1_{\Aut{N}}$.
      Hence for any element $(n,h)$ of $G$
      $$(1_N,1_H) \cdot (n,h) = (1_N \phi_{1_H}(n), 1_Hh) = (\phi_{1_H}(n), h) = (n, h).$$
      Similarly, 
      $$(n,h) \cdot (1_N,1_H) = (n \phi_{h}(1_N), h1_H) = (n, h).$$
      Hence $1_G = (1_N, 1_H)$.
      
      To see that $G$ is associative, let $(n_1, h_1), (n_2,_h2), (n_3, h_3) \in G$ be given.
      Consider 
      $$\left((n_1,h_1) \cdot (n_2, h_2)\right) \cdot (n_3,h_3) = (n_1\phi_{h_1}(n_2), h_1h_2) \cdot (n_3,h_3) = (n_1\phi_{h_1}(n_2)\phi_{h_1h_2}(n_3), h_1h_2h_3)$$
      and
      $$(n_1,h_1) \cdot \left( (n_2, h_2) \cdot (n_3,h_3)\right) = (n_1,h_1) \cdot (n_2 \phi_{h_2}(n_3), h_2h_3) = (n_1 \phi_{h_1}(n_2\phi_{h_2}(n_3)), h_1h_2h_3).$$
      Now observe that $\phi_{h_1}\phi_{h_2} = \phi(h_1)\phi(h_2) = \phi(h_1h_2)$.
      Hence $$n_1 \phi_{h_1}(n_2\phi_{h_2}(n_3)) = n_1\phi_{h_1}(n_2)(\phi_{h_1}\phi_{h_2})(n_3) = n_1\phi_{h_1}(n_2)\phi_{h_1h_2}(n_3)$$
      and thus $\left((n_1,h_1) \cdot (n_2, h_2)\right) \cdot (n_3,h_3) = (n_1,h_1) \cdot \left( (n_2, h_2) \cdot (n_3,h_3)\right)$.
      
      Finally, consider for any $(n,h) \in G$, its product with the element $(1_N, h^{-1}) \cdot (n^{-1}, 1_H) = (1_N\phi_{h^{-1}}(n^{-1}), h^{-1}1_H) = (\phi_{h^{-1}}(n^{-1}), h^{-1})$.
      Then
      $$(n,h) \cdot (\phi_{h^{-1}}(n^{-1}), h^{-1}) = (n \phi_h\phi_{h^{-1}}(n^{-1}), hh^{-1}) = (nn^{-1}, 1_H) = (1_N, 1_H) = 1_G$$
      and
      $$(\phi_{h^{-1}}(n^{-1}), h^{-1}) \cdot (n,h) = (\phi_{h^{-1}}(n^{-1})\phi_{h^{-1}}(n), h^{-1}h) = (\phi_{h^{-1}}(nn^{-1}), 1_H) = (1_N, 1_H) = 1_G$$
      imply $(\phi_{h^{-1}}(n^{-1}), h^{-1}) = (n,h)^{-1}$.
      Therefore $G$ is a group.
    \item
      Define the surjective maps 
      \begin{align*}
        \varphi \colon \overline{N} & \rightarrow N && \text{and} & \psi \colon \overline{H} & \rightarrow H\\
        (n, 1_H) & \mapsto (n, 1_H) && & (1_N, h) & \mapsto h.
      \end{align*}
      Let $(n_1, 1_H),(n_2, 1_H) \in \overline{N}$ and $(1_N, h_1), (1_N, h_2) \in \overline{H}$ be given.
      To see that these maps are well-defined, assume $(n_1, 1_H) = (n_2, 1_H)$ and $(1_N, h_1) = (1_N, h_2)$.
      Then $\varphi(n_1,1_H) = n_1 = n_2 = \varphi(n_2,1_H)$ and $\psi(1_N, h_1) = h_1 = h_2 = \psi(1_N, h_2)$.
      
      To see that these are both homomorphisms, observe
      $$\varphi(n_1n_2, 1_H) = n_1n_2 = \varphi(n_1, 1_H)\varphi(n_2, 1_H)$$
      and
      $$\psi(1_N, h_1h_2) = h_1h_2 = \psi(1_N, h_1)\psi(1_N, h_2).$$
      Finally, observe that $\varphi(n,1_H) = 1_N$ and $\psi(1_N, h) = 1_H$ if and only if $n = 1_N$ and $h = 1_H$ imply $\ker\varphi = \ker\psi = 1_G$.
      Hence $\varphi$ and $\psi$ are both isomorphisms.
      Therefore $\overline{N} \cong N$ and $\overline{H} \cong H$.
    \item
      Let $(n_1, 1_H) \in \overline{N}$ be given and $(n_2, h) \in G$ be given.
      Conjugating by $(n_2, h)^{-1}$ we have
      $$(n_2, h)(n_1, 1_H)(\phi_{h^{-1}}(n_2^{-1}), h^{-1}) = (n_2, h)(n_1\phi_{h^{-1}}(n_2^{-1}), h^{-1}) = (n_2\phi_h(n_1\phi_{h^{-1}}(n_2^{-1})), 1_H) = (n_2\phi_h(n_1)n_2^{-1}, 1_H).$$
      Since $\phi_h(n_1) \in N$, it follows that $n_2\phi_h(n_1)n_2^{-1} \in N$, and thus $(n_2\phi_h(n_1)n_2^{-1}, 1_H) \in \overline{N}$ implies $\overline{N}$ is closed under conjugation from $G$.
      Therefore $\overline{N}$ is normal.
      
    \item
    \item
    \item
    \end{enumerate}
  \end{proof}
\end{thm}
\end{document}
