\documentclass[10pt]{amsart}
\usepackage{amsmath,amsthm,amssymb,amsfonts,enumerate,paralist}
\openup 5pt
\author{Blake Farman\\University of South Carolina}
\title{Math 703:\\Homework 02}
\date{September 20, 2012}
\pdfpagewidth 8.5in
\pdfpageheight 11in
\usepackage[margin=1in]{geometry}
\begin{document}
\maketitle

\providecommand{\norm}[1]{\lVert#1\rVert}
\renewcommand{\qedsymbol}{\ensuremath{\blacksquare}}
\newcommand{\abs}[1]{\left| #1 \right|}

\newtheorem*{ex3a}{p. 531, 3 (a)}
\newtheorem*{ex4}{p. 531, 4}
\newtheorem*{ex13}{p. 532, 13}
\newtheorem*{ex25}{p. 535, 25}

\begin{ex3a}
  Suppose that $(A,\rho)$ is a metric space, and define 
  $$\rho_1(u,v) = \frac{\rho(u,v)}{1 + \rho(u,v)}.$$
  Show that $(A, \rho_1)$ is a metric space.
  
  \begin{proof}
    Let $u,v \in A$ be given.
    Since $\rho$ is a metric, we have $\rho(u,v) \geq 0$ and $1 + \rho(u,v) \geq 1$.
    Hence $\rho_1(u,v) \geq 0 / 1 = 0$.
    Moreover, $\rho_1(u,v) = 0$ if and only if $\rho(u,v) = 0$.  
    Hence $\rho_1(u,v) = 0$ if and only if $u = v$.
    
    Since $\rho$ is a metric, we have
    $$\rho_1(u,v) = \frac{\rho(u,v)}{1 + \rho(u,v)} = \frac{\rho(v,u)}{1 + \rho(v,u)} = \rho_1(v,u).$$
    Finally, to see the triangle inequality, let $w \in A$ be given and define $A = \rho(u,w)$, $B = \rho(w,v)$, $C = \rho(u,v)$.
    Consider the sum
    $$ \frac{A}{1+A} + \frac{B}{1 + B} - \frac{C}{1+C} = \frac{A + B - C + 2AB + ABC}{(1+A)(1+B)(1+C)}.$$
    Since $A$, $B$, and $C$ are all positive, it is clear that the denominator of the right hand side is strictly positive, hence it suffices to show that the numerator is non-negative.
    Observe that $$A + B - C + 2AB + ABC \geq 2AB(1 + C)$$ holds by the triangle inequality for $\rho$ and that $2AB(1+C) \geq 0$ follows from $A,B,C$ all non-negative.
    Therefore the triangle inequality holds for $\rho_1$ and $(A,\rho_1)$ is a metric space.
  \end{proof}
\end{ex3a}

\begin{ex4}{}
  Let $(A,\rho)$ be a metric space, and let 
  $$\sigma(u,v) = \frac{\rho(u,v)}{1 + \rho(u,v)}.$$
  Show that a subset $A$ is open in $(A,\rho)$ if and only if it is open in $(A,\sigma)$.
  
  \begin{proof}
    Let $U \subseteq A$ be given and assume $U$ is open in $(A,\sigma)$.
    Let $u \in U$ be given.
    Since $U$ is open, there exists an $\varepsilon > 0$ such that $B_\sigma(u,\varepsilon) = \left\{a \in A \mid \sigma(a,u) < \varepsilon\right\} \subseteq U$.
    Consider the $\varepsilon$-neighborhood $B_\rho(u,\varepsilon) = \left\{ a \in A \mid \rho(a,u) < \varepsilon\right\}$.
    Let $a \in B_\rho(u,\varepsilon)$ be given and observe that $\sigma(a,u) = \rho(a,u)/(1 + \rho(a,u)) \leq \rho(a,u) < \varepsilon$.
    Hence $a \in B_\sigma(u,\varepsilon)$ implies $B_\rho(u,\varepsilon) \subseteq B_\sigma(u,\varepsilon) \subseteq U$.
    Therefore $U$ is open in $(A,\rho)$.
    
    Conversely, assume $U$ is open in $(A,\rho)$.
    Let $u \in U$ be given.
    Since $U$ is open, there exists an $\varepsilon > 0$ such that $B_\rho(u,\varepsilon) = \left\{a \in A \mid \rho(a,u) < \varepsilon\right\} \subseteq U$.
    Let $\epsilon^{\prime} = \varepsilon/(1 + \varepsilon)$ and consider the $\varepsilon^{\prime}$-neighborhood, $B_\sigma(u, \varepsilon^{\prime}) = \left\{a \in A \mid \sigma(a,u) < \varepsilon^{\prime} \right\}$.
    Let $a \in B_\sigma(a,\varepsilon^{\prime})$ be given and observe that $\sigma(a,u) = \rho(a,u)/(1 + \rho(a,u)) < \varepsilon^{\prime}$.
    Since $1 + \rho(a,u) \geq 1$, it follows that $\rho(a,u) < \varepsilon^{\prime}(1 + \rho(a,u)) < \varepsilon^{\prime}(1 + \varepsilon) = \varepsilon$.
    Hence $a \in B_\rho(u,\varepsilon)$ implies $B_\sigma(u,\varepsilon^{\prime}) \subseteq B_\rho(u,\varepsilon) \subseteq U$.
    Therefore $A$ is open in $(A,\sigma)$.
  \end{proof}
\end{ex4}{}

\begin{ex13}{}
  Prove\\
  \begin{inparaenum}[(a)]
  \item\label{13a}
    $(S_1 \cap S_2)^\circ = S_1^\circ \cap S_2^\circ$\hspace{10mm}
  \item\label{13b}
    $S_1^\circ \cup S_2^\circ \subseteq (S_1 \cup S_2)^\circ$
  \end{inparaenum}
  
  \begin{proof}
    \begin{enumerate}[(a)]
    \item
      Let $s \in (S_1 \cap S_2)^\circ$ be given.
      Then there exists an $\varepsilon > 0$ such that the $\varepsilon$-neighborhood about s, $B(s,\varepsilon)$, is contained in $S_1 \cap S_2$.
      From the definition of the intersection, we have $B(s,\varepsilon) \subseteq S_1$ and $B(s,\varepsilon) \subseteq S_2$.
      Hence, $s \in S_1^\circ \cap S_2^\circ$ implies $(S_1 \cap S_2)^\circ \subseteq S_1^\circ \cap S_2^\circ$
      
      To see the reverse containment, let $s \in S_1^\circ \cap S_2^\circ$ be given.
      There exist $\varepsilon_1, \varepsilon_2 > 0$ such that $B(s,\varepsilon_1) \subseteq S_1$ and $B(s,\varepsilon_2) \subseteq S_2$.
      Let $U = B(s,\varepsilon_1) \cap B(s,\varepsilon_2)$, then observe that $U$ is an open neighborhood of $s$ and $U \subseteq S_1 \cap S_2$.
      Therefore $s \in (S_1 \cap S_2)^\circ$ implies $S_1^\circ \cap S_2^\circ \subseteq (S_1 \cap S_2)^\circ$, as desired.
    \item
      Let $s \in S_1^\circ \cup S_2^\circ$ be given.
      Then there exists an $\varepsilon > 0$ for which either $B(s,\varepsilon) \subseteq S_1$ or $B(s,\varepsilon) \subseteq S_2$ holds.
      Hence $B(s,\varepsilon) \subseteq S_1 \cup S_2$ implies $s \in (S_1 \cup S_2)^\circ$, and thus $S_1^\circ \cup S_2^\circ \subseteq (S_1 \cup S_2)^\circ$.
      
      To see that the reverse containment does not hold in general, consider $\mathbb{R}$ with the usual metric and the sets $S_1 = [-1,0]$ and $S_2 = [0,1]$.  
      Then $0 \in (S_1 \cup S_2)^\circ = (-1,1)$, but $0 \not \in S_1^\circ \cup S_2^\circ = (-1,0) \cup (0,1)$.
    \end{enumerate}
  \end{proof}
\end{ex13}{}

\begin{ex25}{}
  \begin{enumerate}[(a)]
  \item\label{25a}
    Show that 
    $$\norm{f} = \int_a^b \abs{f(x)}dx$$
    is a norm on $C[a,b]$.
  \item\label{25b}
    Show that the sequence $\left\{f_n\right\}$ defined by 
    $$f_n(x) = \left(\frac{x-a}{b-a}\right)^n$$
    is a Cauchy sequence in $(C[a,b], \norm{\cdot})$.
  \item\label{25c}
    Show that $(C[a,b], \norm{\cdot})$ is not complete.
  \end{enumerate}

  \begin{proof}
    \begin{enumerate}[(a)]
    \item
      Let $f \in C[a,b]$ be given.
      We first show that $\norm{f} \geq 0$, with equality if and only if $f \equiv 0$.
      Note that since $f$ is continuous on $[a,b]$ and $\abs{\cdot}$ is continuous on $\mathbb{R}$, hence continuous on the (compact) image of $f$, $\abs{f}$ is continuous on $[a,b]$.
      Let $m = \min \left\{ \abs{f(x)} \mid x \in [a,b] \right\}$ and $M = \max \left\{ \abs{f(x)} \mid x \in [a,b] \right\}$.
      Observe that
      \begin{equation}\label{bound}
	0 \leq m(b-a) \leq \int_a^b \abs{f(x)}dx \leq M(b-a).
      \end{equation}
      From this inequality it follows that if $f \equiv 0$, then $\norm{f} = 0$ holds.
      Hence it suffices to show that if $f \not \equiv 0$, then the left-most inequality in \eqref{bound} is strict.

      To that end, assume the existence of some $c \in [a,b]$ such that $\abs{f(c)} > 0$ and let $\varepsilon = |f(c)|/2$.
      Since $\abs{f}$ is continuous, there exists some $\delta > 0$ such that $|f(B(c,\delta) \cap [a,b])| \subseteq B(|f(c)|,\varepsilon)$.
      Moreover, if $B(c,\delta) \cap [a,b]^c \not = \emptyset$, then we may choose a smaller $\delta$ such that $B(c,\delta) \subset [a,b]$.
      Hence it suffices to assume that $B(c,\delta) \subset [a,b]$.
      
      Now it follows from the left-most inequality in \eqref{bound} that
      $$\int_{c-\delta/2}^{c+\delta/2} \abs{f(x)}dx \leq \int_a^{c-\delta/2} \abs{f(x)}dx + \int_{c-\delta/2}^{c+\delta/2} \abs{f(x)}dx + \int_{c+\delta/2}^b \abs{f(x)}dx = \int_a^b \abs{f(x)}dx.$$
      By the construction above, we have $\abs{f(\overline{B}(c,\delta/2))} \subseteq B(\abs{f(c)},\varepsilon)$, from which it follows that $0 < \varepsilon \leq \max \left\{\abs{f(x)} | x \in \overline{B}(c,\delta/2) \right\}$.
      Therefore
      $$0 < \int_{c-\delta/2}^{c+\delta/2} \abs{f(x)}dx \leq \int_a^b \abs{f(x)}dx,$$
      holds by \eqref{bound}, as desired.
      
      The remaining axioms follow from the triangle inequality for $\abs{\cdot}$ and calculus.
      Namely, $\abs{(f + g)(x)} \leq \abs{f(x)} + \abs{g(x)}$ implies 
      $$\norm{f+g} = \int_a^b\abs{(f+g)(x)}dx \leq \int_a^b\abs{f(x)}dx + \int_a^b\abs{g(x)}dx = \norm{f} + \norm{g},$$
      and for any $a \in \mathbb{R}$, $$\norm{af} = \abs{a}\int_a^b\abs{f(x)}dx = \abs{a}\norm{f}.$$
    \item
      Let $\varepsilon > 0$ be given and take $N > 2(b-a)/\varepsilon$.
      Observe that $0 \leq f_n(x) \leq 1$, hence by the triangle inequality
      $$\norm{f_n - f_m} \leq \int_a^b \frac{(x-a)^n}{(b-a)^n}dx + \int_a^b\frac{(x-a)^m}{(b-a)^m}dx = \frac{b-a}{n+1} + \frac{b-a}{m+1} < \frac{\varepsilon}{2} + \frac{\varepsilon}{2}= \varepsilon$$
      holds whenever $m,n \geq N$.
      Therefore $f_n$ is Cauchy in $(C[a,b], \norm{\cdot})$.
    \item
      Let $\varepsilon > 0$ be given and take $N > 2(b-a)/\varepsilon$.
      Define the function
      $$g(x) = 
      \left\{
      \begin{array}{ll}
        1 & \text{if } x = b,\\
        0 & \text{if } a \leq x < b.
      \end{array}
      \right.$$

      Observe that by the argument in (\ref{25a}), we have
      \begin{equation}\label{intg} 
        \int_a^b g(x)dx\\ = \int_a^{b-\varepsilon/2} g(x)dx\\ + \int_{b-\varepsilon/2}^bg(x)dx = \int_{b-\varepsilon/2}^bg(x)dx \leq \int_{b-\varepsilon/2}^b dx = \frac{\varepsilon}{2}.
      \end{equation}
      By the triangle inequality and \eqref{intg}
      $$
      \norm{f_n - g} \leq \frac{b-a}{n+1} + \frac{\varepsilon}{2} < \frac{\varepsilon}{2} + \frac{\varepsilon}{2} = \varepsilon
      $$
      holds whenever $n \geq N$.
      Hence $f_n$ converges in $(C[a,b], \norm{\cdot})$ to $g$, which is discontinuous at $b$.
      Therefore $(C[a,b], \norm{\cdot})$ is not complete.
    \end{enumerate}
  \end{proof}
\end{ex25}{}

\end{document}
