\documentclass[12pt]{amsart}
\renewcommand{\baselinestretch}{1.5}
\usepackage{amsmath,amsthm,amssymb,amsfonts,enumerate,paralist,mymath}
\openup 5pt
\author{Blake Farman\\University of South Carolina}
\title{Math 703:\\Homework 06}
\date{November 20, 2012}
\pdfpagewidth 8.5in
\pdfpageheight 11in
\usepackage[margin=.75in]{geometry}
\begin{document}
\maketitle

\newtheorem{thm}{}
\setcounter{thm}{20}
\newtheorem{ex}{}[thm]
\newtheorem{lem}{Lemma}
\theoremstyle{definition}
\newtheorem{defn}{Definition}

\begin{thm}
  Express in the form $a + ib$.
  
  \begin{enumerate}[(a)]
  \item
    $(1 + i)^{20}$,
  \item
    $\frac{1 - 2i}{2 + i}$.
  \end{enumerate}
  
  \begin{proof}
    \begin{enumerate}[(a)]
      \item
        Write $1 + i$ as $\sqrt{2}e^{i\pi/4}$ and then 
        $$(1 + i)^{20} = (\sqrt{2}e^{i\pi/4})^{20} = 2^{10}e^{5\pi/4} = -2^{10}.$$
      \item
        Multiplying the numerator and the denominator by $2 - i$ we have
        $$\frac{1 - 2i}{2 + i} = \frac{(1 - 2i)(2 - i)}{(2 + i)(2 - i)} = \frac{2 - 2 - i(1 - 4)}{5} = \frac{-5i}{5} = -i.$$
    \end{enumerate}
  \end{proof}
\end{thm}

\begin{thm}
  Solve $z^2 - 4z + (4 + 2i) = 0$.
  \begin{proof}
    First write
    $$(4 + 2i) = 2(2 + i) = (1 + i)(1 - i)(2 + i) = (1 + i)(3 - i).$$
    Then it follows that 
    $$(z - (3 - i))(z - (i + 1)) = z^2 - z(1 + i + 3 - i) + (4 + 2i) = z^2 - 4z + (4 + 2i).$$
    Therefore the roots are $z = 3 - i$ and $z = 1 + i$.
  \end{proof}
\end{thm}

\begin{thm}
  Describe the sets whose points satisfy the following relation.
  Which of these sets are regions (i.e. open and connected sets)?
  \begin{enumerate}[(a)]
  \item
    $\abs{z + i} \leq 1$,
  \item
    $\abs{\frac{z - 1}{z + 1}} = 1$,
  \item
    $\abs{z - 3} > \abs{z - 2}$,
  \item
    $\frac{1}{z} = \overline{z}$.
  \end{enumerate}
  \begin{proof}
    \begin{enumerate}[(a)]
    \item
      The set $\abs{z + i} \leq 1$ is the closed disc of radius 1 centered at $-i$.
    \item
      First observe that since $\abs{\cdot}$ is a norm,  $\abs{\frac{z - 1}{z + 1}} = \abs{z - 1}\frac{1}{\abs{z + 1}} = 1$ implies $\abs{z - 1} = \abs{z + 1}$.
      Squaring both sides it follows that  $$(z - 1)(\overline{z} - 1) = (z + 1)(\overline{z} + 1).$$
      Expanding yields $$z\overline{z} - z - \overline{z} + 1 = z\overline{z} + z + \overline{z} + 1.$$
      Hence $2(z + \overline{z}) = 4\real{z} = 0$, from which it follows that $\real{z} = 0$.
      Therefore $\abs{\frac{z - 1}{z + 1}} = 1$ is the imaginary axis.
      Since for every point $iy \in \C$, $y \in \R$ and for every $\varepsilon > 0$ there exists some point in the open disc of radius $\varepsilon$ centered about $iy$ with non-zero real part, this set is not open. 
    \item
      It follows from $\abs{z - 3} > \abs{z - 2}$ that $$(z - 3)(\overline{z} - 3) > (z - 2)(\overline{z} - 2).$$
      Setting the two equal yields 
      $$z\overline{z} - 3z - 3\overline{z} + 9 = z\overline{z} - 2z - 2\overline{z} + 4.$$
      Rearranging one obtains $\real{z} = \frac{5}{2}$.
      Checking $z = 0$ it follows that $\abs{z - 3} > \abs{z - 2}$ is the region $\real{z} < \frac{5}{2}$.
    \item
      Multiplying both sides of $\frac{1}{z} = \overline{z}$ by $z$ we have 
      $$1 = z\overline{z} = \abs{z}^2.$$
      Taking the square root of both sides, it then follows from positivity of $\abs{\cdot}$ that this is the closed unit circle, $\abs{z} = 1$.
    \end{enumerate}
  \end{proof}
\end{thm}

\begin{thm}
  Find all solutions of the equation $\sin(z) = 3$.
  \begin{proof}
    First write $z = x + iy$ and then
    \begin{equation}\label{3.1}
      \sin(z) = \frac{e^{-y + ix} - e^{y - ix}}{2i} = \frac{e^{-y}(\cos(x) + i\sin(x)) - e^y(\cos(x) - i\sin(x))}{2i} = 3.
    \end{equation}
    Rearranging the right hand side of \eqref{3.1} yields
    $$\cos(x)(e^{-y} - e^{y}) + i\sin(x)(e^{-y} + e^y) = 6i.$$
    Equating real and imaginary parts we have $\cos(x)(e^{-y} - e^{y}) = 0$ and $\sin(x)(e^{-y} + e^y)$.
    The former has solutions $x = (2n + 1)\pi/2$, $n \in Z$, and $y = 0$.
    Observe that when $y = 0$, $$\sin(x)(e^{-y} + e^y) = 2\sin(x) = 6$$ has no solution in $x$.
    Similarly, when $x = (\pm 3n + 1)\pi/2$, $n \in \N$, $-(e^{-y} + e^y) = 6$ as no solution in $y$ since $e^{-y}$ and $e^y$ are both positive.
    Hence $x = (\pm 4n + 1)\pi/2$, $n \in \N$.
    Then letting $Y = e^y$ we have $6 = Y^{-1} + Y$, which rearranges to $Y^2 - 6Y + 1 = 0$.
    We then factor $1 = (3 + 2\sqrt{2})(3 - 2\sqrt{2})$ to obtain $$Y^2 - 6Y + 1 = (Y - (3 + 2\sqrt{2}))(Y - (3 - 2\sqrt{2})) = 0.$$
    %Solving for $Y$ we obtain $Y = 3 \pm 2\sqrt{2}$.
    Hence $y = \log(3 \pm 2 \sqrt{2})$.
    Therefore the solutions to $\sin(z) = 3$ are $z = (\pm 4n + 1)\pi/2 + i \log(3 \pm 2\sqrt{2})$, $n \in N$.
  \end{proof}
\end{thm}

\begin{thm}
  Find all solutions of 
  \begin{enumerate}[(a)]
  \item
    $e^z = -i$,
  \item
    $\sin(z) = 0$,
  \item
    $\Log{z} = 1 + i$.
  \end{enumerate}
  
  \begin{proof}
    \begin{enumerate}[(a)]
    \item
      Write $z = x + iy$ so that $$e^z = e^x(\cos(y) + i\sin(y)) = i.$$
      Then $e^x\cos(y) = 0$ implies $y = (2n+1)\pi/2$, $n \in \Z$.
      Hence $e^x\sin(y) = \pm e^x = 1$.
      Since $e^x$ is positive, it follows that $\sin(y) = 1$ must hold, hence $y = (\pm 4n + 1)\pi/2$, $n \in \N$.
      Then it follows that $e^x = 1$ and thus $x = 0$.
      Therefore the solutions to $e^z = -i$ are $z = i(\pm 4n + 1)\pi/2$, $n \in \N$.
    \item
      Writing $$\sin(z) = \frac{e^{iz} - e^{-iz}}{2i} = 0$$
      it is clear that $e^{iz} = e^{-iz}$.
      Multiplying both sides by $e^{iz}$ we have $(e^{iz})^2 = 1$ from which it follows that $e^{iz} = \pm 1$.
      Writing $z = x + iy$ we obtain $$e^{iz} = e^{-y + ix} = e^{-y}(\cos(x) + i\sin(x)) = \pm 1.$$
      Comparing coefficients, $\sin(x) = 0$ implies $x = n\pi, n \in \Z$.
      Hence $e^{-y}\cos(n\pi) = \pm e^{-y} = \pm {1}$ implies $y = 0$.
      Therefore the solutions to $\sin(z) = 0$ are $z = n\pi, n \in \Z$.
    \item
      Let $\theta = \Arg{z}$ and write $z = re^{i\theta}$.
      Then $\Log{z} = \log(r) + i\theta = 1 + i$ implies $r = e$ and $\theta = 1$.
      Hence $z = e^{1 + i}$.
      Therefore the solutions to $\Log{z} = 1 + i$ are the fibers of $e^{1 + i}$ under the $\operatorname{Arg}$ map, $z = e^{1 + i(2n\pi + 1)}, n \in \Z$
    \end{enumerate}
  \end{proof}
\end{thm}

\begin{thm}
  Prove that the following functions are nowhere differentiable.
  \begin{enumerate}[(a)]
    \item
      $f(z) = \real{Z}$,
    \item
      $f(z) = \abs{z}$.
  \end{enumerate}
  \begin{proof}
    \begin{enumerate}[(a)]
    \item
      Fix some point $z_0 \in \C$ and choose two sequences, $\left\{z_n = z_0 + \frac{1}{n}\right\}_{n=1}^\infty$ and $\left\{\omega_n = z_0 + i\frac{1}{n}\right\}_{n=1}^\infty$.
      Then we have for $z$ approaching $z_0$ by way of $\left\{z_n\right\}_{n=1}^\infty$ the limit
      $$\lim_{n \rightarrow \infty} \frac{\real{z_n} - \real{z_0})}{z_n - z_0} = \lim_{n\rightarrow \infty} = \frac{\frac{1}{n}}{\frac{1}{n}} = \lim_{n \rightarrow \infty} 1 = 1.$$
      However approaching by $\left\{\omega_n\right\}_{n=1}^\infty$ we have the limit
      $$\lim_{n \rightarrow \infty} \frac{\real{\omega_n} - \real{z_0}}{\omega_n - z_0} = \lim_{n\rightarrow \infty} = \frac{z_0 - z_0}{i\frac{1}{n}} = \lim_{n \rightarrow \infty} (-in) \cdot 0 = 0.$$
      Therefore the limit does not exist and, since $z_0$ was arbitrary, $f(z) = \real{z}$ is nowhere differentiable.
    \item
      First observe that by writing $z = x + iy$, we have $f(z) = \abs{z} = \sqrt{x^2 + y^2}$.
      Hence we may write $f(z) = u(x,y) + iv(x,y)$, where $u(x,y) = \sqrt{x^2 + y^2}$ and $v(x,y) = 0$.
      Fix some point $0 \neq z = x + iy \in \C$.
      Observe that since at least one of $x$ and $y$ is non-zero the Cauchy-Riemann equations,
      \begin{align*}
        \frac{\partial u}{\partial x} = \frac{x}{\sqrt{x^2 + y^2}} = 0& = \frac{\partial v}{\partial y} &&\text{and}\ & \frac{\partial u}{\partial y} = \frac{y}{\sqrt{x^2 + y^2}} = 0 &= -\frac{\partial v}{\partial x}.
      \end{align*}
      are not satisifed.
      Hence $f$ is not differentiable at $z \neq 0$ by the contrapositive of Theorem 3.3.
      
      Fix $z = 0$ and consider the two sequences $\left\{z_n = \frac{1}{n} \right\}_{n=1}^\infty$ and $\left\{ \omega_n = \frac{-1}{n} \right\}_{n=1}^\infty$.
      Approaching zero by way of $\left\{z_n\right\}_{n=1}^\infty$ we have the limit
      $$\lim_{n \rightarrow \infty} \frac{\abs{\frac{1}{n}} - 0}{\frac{1}{n} - 0} = \lim_{n \rightarrow \infty} \frac{\frac{1}{n}}{\frac{1}{n}} = \lim_{n \rightarrow \infty} 1 = 1.$$
      Approaching zero by way of $\left\{\omega_n\right\}_{n=1}^\infty$ we have the limit
      $$\lim_{n \rightarrow \infty} \frac{\abs{\frac{-1}{n}} - 0}{\frac{-1}{n} - 0} = \lim_{n \rightarrow \infty} \frac{\frac{1}{n}}{\frac{-1}{n}} = \lim_{n \rightarrow \infty} -1 = -1.$$
      Hence the limit does not exist and $f$ is not differentiable at $z = 0$.
      Therefore $f$ is nowhere differentiable.
    \end{enumerate}
  \end{proof}
\end{thm}
\end{document}
