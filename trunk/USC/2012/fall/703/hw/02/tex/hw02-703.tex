\documentclass[10pt]{amsart}
\usepackage{amsmath,amsthm,amssymb,amsfonts,enumerate,paralist}
\openup 5pt
\author{Blake Farman\\University of South Carolina}
\title{Math 703:\\Homework 01}
\date{August 30, 2012}
\pdfpagewidth 8.5in
\pdfpageheight 11in
\usepackage[margin=1in]{geometry}
\begin{document}
\maketitle

\providecommand{\norm}[1]{\lVert#1\rVert}
\renewcommand{\qedsymbol}{\ensuremath{\blacksquare}}
\newcommand{\abs}[1]{\left| #1 \right|}

\newtheorem*{ex3a}{p. 531, 3 (a)}
\newtheorem*{ex4}{p. 531, 4}
\newtheorem*{ex13}{p. 532, 13}
\newtheorem*{ex25}{p. 535, 25}

\begin{ex3a}
	Suppose that $(A,\rho)$ is a metric space, and define 
		$$\rho_1(u,v) = \frac{\rho(u,v)}{1 + \rho(u,v)}.$$
	Show that $(A, \rho_1)$ is a metric space.
	
	\begin{proof}
	Let $u,v \in A$ be given.
		Since $\rho$ is a metric, we have $\rho(u,v) \geq 0$ and $1 + \rho(u,v) \geq 1$.
		Hence $\rho_1(u,v) \geq 0 / 1 = 0$.
		Moreover, observe that since $1 + \rho(u,v) \geq 1$, $\rho_1(u,v) = 0$ if and only if $u = v$.
		
		Since $\rho$ is a metric, we have
			$$\rho_1(u,v) = \frac{\rho(u,v)}{1 + \rho(u,v)} = \frac{\rho(v,u)}{1 + \rho(v,u)} = \rho_1(v,u).$$
		Finally, to see the triangle inequality, let $w \in A$ be given and define $A = \rho(u,w)$, $B = \rho(w,v)$, $C = \rho(u,v)$.
		Consider the sum
			$$ \frac{A}{1+A} + \frac{B}{1 + B} - \frac{C}{1+C} = \frac{A + B - C + 2AB + ABC}{(1+A)(1+B)(1+C)}.$$
		Since $A$, $B$, and $C$ are all positive, it is clear that the denominator of the right hand side is strictly positive, hence it suffices to show that the numerator is non-negative.
		Observe that $$A + B - C + 2AB + ABC \geq 2AB(1 + C)$$ holds by the triangle inequality for $\rho$ and that $2AB(1+C) \geq 0$ follows from $A,B,C$ all non-negative.
		Therefore the triangle inequality holds for $\rho_1$ and $(A,\rho_1)$ is a metric space.
	\end{proof}
\end{ex3a}

\begin{ex4}{}
	Let $(A,\rho)$ be a metric space, and let 
		$$\sigma(u,v) = \frac{\rho(u,v)}{1 + \rho(u,v)}.$$
	Show that a subset $A$ is open in $(A,\rho)$ if and only if it is open in $(A,\sigma)$.
	
	\begin{proof}
		Let $U \subseteq A$ be given and assume $U$ is open in $(A,\sigma)$.
		Let $u \in U$ be given.
		Since $U$ is open, there exists an $\varepsilon > 0$ such that $B_\sigma(u,\varepsilon) = \left\{a \in A \mid \sigma(a,u) < \varepsilon\right\} \subseteq U$.
		Consider the $\varepsilon$-neighborhood $B_\rho(u,\varepsilon) = \left\{ a \in A \mid \rho(a,u) < \varepsilon\right\}$.
		Let $a \in B_\rho(u,\varepsilon)$ be given and observe that$\sigma(a,u) = \rho(a,u)/(1 + \rho(a,u)) \leq \rho(a,u) < \varepsilon$.
		Hence $a \in B_\sigma(u,\varepsilon)$ implies $B_\rho(u,\varepsilon) \subseteq B_\sigma(u,\varepsilon) \subseteq U$.
		Therefore $U$ is open in $(A,\rho)$.
		
		Conversely, assume $U$ is open in $(A,\rho)$.
		Let $u \in U$ be given.
		Since $U$ is open, there exists an $\varepsilon > 0$ such that $B_\rho(u,\varepsilon) = \left\{a \in A \mid \rho(a,u) < \varepsilon\right\} \subseteq U$.
		Let $\epsilon^{\prime} = \varepsilon/(1 + \varepsilon)$ and consider the $\varepsilon^{\prime}$-neighborhood, $B_\sigma(u, \varepsilon^{\prime}) = \left\{a \in A \mid \sigma(a,u) < \varepsilon^{\prime} \right\}$.
		Let $a \in B_\sigma(a,\varepsilon^{\prime})$ be given and observe that $\sigma(a,u) = \rho(a,u)/(1 + \rho(a,u)) < \varepsilon^{\prime}$.
		Since $1 + \rho(a,u) \geq 1$, it follows that $\rho(a,u) < \varepsilon^{\prime}(1 + \rho(a,u)) < \varepsilon^{\prime}(1 + \varepsilon) = \varepsilon$.
		Hence $a \in B_\rho(u,\varepsilon)$ implies $B_\sigma(u,\varepsilon^{\prime}) \subseteq B_\rho(u,\varepsilon) \subseteq U$.
		Therefore $A$ is open in $(A,\sigma)$.
	\end{proof}
\end{ex4}{}

\begin{ex13}{}
	Prove\\
	\begin{inparaenum}[(a)]
		\item\label{13a}
			$(S_1 \cap S_2)^0 = S_1^0 \cap S_2^0$\hspace{10mm}
		\item\label{13b}
			$S_1^0 \cup S_2^0 \subset (S_1 \cup S_2)^0$
	\end{inparaenum}
	
	\begin{proof}
	\end{proof}
\end{ex13}{}

\begin{ex25}{}
	\begin{enumerate}[(a)]
		\item\label{25a}
			Show that 
			$$\norm{f} = \int_a^b \abs{f(x)}dx$$
			is a norm on $C[a,b]$.
		\item\label{25b}
			Show that the sequence $\left\{f_n\right\}$ defined by 
				$$f_n(x) = \left(\frac{x-a}{b-a}\right)^n$$
			is a Cauchy sequence in $(C[a,b], \norm{\cdot})$.
		\item\label{25c}
			Show that $(C[a,b], \norm{\cdot})$ is not complete.
	\end{enumerate}
\end{ex25}{}

\end{document}
