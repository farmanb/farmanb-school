\documentclass[10pt]{amsart}
\usepackage{amsmath,amsthm,amssymb,amsfonts,enumerate}
\openup 5pt
\author{Blake Farman\\University of South Carolina}
\title{Math 784:\\Homework 01}
\date{August 30, 2012}
\pdfpagewidth 8.5in
\pdfpageheight 11in
\usepackage[margin=1in]{geometry}

\renewcommand{\qedsymbol}{\ensuremath{\blacksquare}}
\newcommand{\abs}[1]{\left| #1 \right|}

\begin{document}
\maketitle

\newtheorem{thm}{}

\begin{thm}
	
\end{thm}

\begin{thm}
Let $[x]$ denote the greatest integer $\leq x$. Determine whether the inequality
	\begin{equation}\label{ineq}
		\left[\frac{\pi}{2}b\right] < \frac{\pi}{2}b^2\sin\left(\frac{1}{b}\right) \leq \frac{\pi}{2}
	\end{equation}
holds for every positive integer b. If it does, supply a proof. If it doesn't, determine the
least six postive integers $b$ for which \eqref{ineq} does not hold.
	\begin{proof}
		Observe that for some convergents $a/b$ of $\pi/2$, with some rearrangement, we have
		$$\abs{\frac{\pi}{2}b - a} \leq \frac{1}{2b}.$$
		Using the denominators of the convergents, the first six for which \eqref{ineq} is not satisified are
		$340262731$, $1963319607$, $574364584667$, $47337186164411$, $136308121570117$, $158084910090576507$.
	\end{proof}
\end{thm}
\end{document}