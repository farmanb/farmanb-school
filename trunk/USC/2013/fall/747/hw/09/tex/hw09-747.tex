\documentclass[10pt]{amsart}
\usepackage{amsmath,amsthm,amssymb,amsfonts,enumerate,mymath,mathtools,tikz-cd}
\openup 5pt
\author{Blake Farman\\University of South Carolina}
\title{Math 747:\\Homework 08}
\date{September 25, 2013}
\pdfpagewidth 8.5in
\pdfpageheight 11in
\usepackage[margin=1in]{geometry}

\begin{document}
\maketitle

\providecommand{\p}{\mathfrak{p}}
\providecommand{\m}{\mathfrak{m}}

\newtheorem{thm}{}
\newtheorem{lem}{Lemma}
\newtheorem{prop}{Proposition}
\theoremstyle{definition}
\newtheorem{defn}{Definition}[thm]

\newcommand{\A}{\mathbb{A}}

\begin{thm}
  Let $X = \C \setminus \{\pm 1\}$ and $Y = \C \setminus \left\{\pi/2 + k\pi  \;\middle\vert\; k \in \Z \right\}$.
  Prove that $\Deck{Y \xrightarrow{\sin} X}$ consists of the following transformations
  \begin{enumerate}[(i)]
  \item
    $f_k(z) = z + 2k\pi, k \in \Z$,
  \item
    $g_k(z) = -z + (2k + 1)\pi, k \in \Z$.
  \end{enumerate}
  Calculate the products $f_k \circ f_\ell$, $f_k \circ g_\ell$, $g_l \circ f_k$, $g_k \circ g_\ell$.

  \begin{proof}
    By Exercise 4.1 it is clear that $\left\{f_k \;\middle\vert\; k \in \Z \right\} \cup \left\{g_k \;\middle\vert\; k \in \Z \right\} \subseteq \Deck{Y \xrightarrow{\sin} X}$.
    To see the reverse inclusion, let $f \in \Deck{Y \xrightarrow{\sin} X}$ be given.
    From the commutative diagram
    \begin{center}
      \begin{tikzcd}
        Y \arrow{r}{f} \arrow[swap]{rd}{\sin} & Y\arrow{d}{\sin}\\
        & X
      \end{tikzcd}
    \end{center}
    it follows that
    $$e^{if(z)} - e^{iz} = -\frac{e^{if(z)} - e^{iz}}{e^{i(f(z) + z)}},$$
    from which it follows that either $e^{i(f(z) + z)} = -1$ or $e^{if(z)} = e^{iz}$.
    In the first case it follows that
    $$f(z) = (2k + 1)\pi - z = g_k(z)$$
    for some $k \in \Z$.
    In the second case, we obtain 
    $$f(z) = z + 2k\pi = f_k(z)$$
    for some $k \in \Z$.

    For the products we have
    \begin{enumerate}[(i)]
    \item
      $f_k \circ f_\ell(z) = z + 2k\pi + 2\ell\pi = z + 2(k + \ell)\pi = f_{k + \ell}(z)$, 
    \item
      $f_k \circ g_\ell(z) =  (2\ell + 1)\pi - z + 2k\pi = (2(k + \ell) + 1)\pi - z = g_{\ell + k}(z)$,
    \item
      $g_l \circ f_k(z) = (2\ell + 1)\pi - z - 2k\pi = (2(\ell - k) + 1)\pi - z = g_{\ell - k}(z)$,
    \item
      $g_k \circ g_\ell(z) = (2k + 1)\pi - 2(\ell + 1)\pi + z = 2(k - \ell)\pi + z = f_{k - \ell}(z)$.
    \end{enumerate}
  \end{proof}
\end{thm}

\begin{thm}
  Determine the covering transformations of
  $$\tan \colon \C \rightarrow \mathbb{P}^1 \setminus \{\pm i\}.$$
  
  \begin{proof}
    From Exercise 4.4 it is clear that $\left\{ z \mapsto z + k\pi \;\middle\vert\; k \in \Z \right\} \subseteq \Deck{\C \xrightarrow{\tan} \mathbb{P}^1 \setminus \{\pm i\}}$.
    To see the reverse inclusion, let $f \in \Deck{\C \xrightarrow{\tan} \mathbb{P}^1 \setminus \{\pm i\}}$ be given.
    From the commutative diagram
    \begin{center}
      \begin{tikzcd}
        \C \arrow{r}{f} \arrow[swap]{rd}{\tan} & \C\arrow{d}{\tan}\\
        & \mathbb{P}^1 \setminus \{\pm i\}
      \end{tikzcd}
    \end{center}
    it follows that
    $$e^{i2f(z)} - e^{i2z} = -(e^{i2f(z)} - e^{i2z}),$$
    whence $e^{i2f(z)} = e^{i2z}$.
    Therefore $f(z) = z + \pi$.
  \end{proof}
\end{thm}

\begin{thm}
  Let $\Gamma$, $\Gamma^\prime \subset \C$ be lattices and let
  $$f \colon \C / \Gamma \rightarrow \C / \Gamma^\prime$$
  be a non-constant holomorphic map with $f(0) = 0$.
  Show that there exists a unique $\alpha \in \C^\times$ such that $\alpha\Gamma \subset \Gamma^\prime$ and the following diagram is commutative
  \begin{center}
    \begin{tikzcd}
      \C \arrow[swap]{d}{\pi} \arrow{r}{F} & \C \arrow{d}{\pi^\prime}\\
      \C/\Gamma \arrow{r}{f} & \C/\Gamma^\prime
    \end{tikzcd}
  \end{center}
  where $F(z) = \alpha z$, and $\pi$ and $\pi^\prime$ are the canonical projections.
  Prove that $f$ is an unbranched covering map and
  $$\Deck{\C/\Gamma \xrightarrow{f} \C/\Gamma^\prime} \cong \Gamma^\prime / \alpha\Gamma.$$

  \begin{proof}
    Regard $\C \xrightarrow{\pi} \C/\Gamma$ and $\C \xrightarrow{\pi^\prime} \C/\Gamma^\prime$ as covering maps.
    Since $\C$ is simply connected and $\pi^\prime$ is a covering map, we have by Theorem 4.17 the following lift
    \begin{center}
      \begin{tikzcd}
        \C \arrow[dotted]{r}{\exists ! F} \arrow[swap]{rd}{f \circ \pi} & \C \arrow{d}{\pi^\prime}\\
         & \C/\Gamma^\prime
      \end{tikzcd}
    \end{center}
    with $F(0) = 0$.
    This gives the commutative diagram
    \begin{center}
      \begin{tikzcd}
        \C \arrow[swap]{d}{\pi} \arrow{r}{F} & \C \arrow{d}{\pi^\prime}\\
        \C/\Gamma \arrow{r}{f} & \C/\Gamma^\prime.
      \end{tikzcd}
    \end{center}
    
    Fix a point $z \in \C$ and choose a neighbourhood $V$ of $z$ such that $\phi = \pi \mid_U \colon U \rightarrow V = \pi(U)$ is a homeomorphism.
    Choose neighbourhoods of $U$ of $F(z)$ and $V$ of $f \circ \pi(z)$ such that $\psi = \pi^\prime \mid_{U^\prime} \colon U^\prime \rightarrow V$ is a homeomorphism.
    Observing that these are both charts as in the construction from Forster, we have, after possibly shrinking $U$, by the commutativity of the diagram above that
    $$\psi \circ f \circ \phi= \psi^{-1} \circ \psi \circ F = F$$
    and, moreover, this gives $F$ holomorphic at $z$ since $f$ is holomorphic by assumption.
    Since $z$ was arbitrary we have that $F$ is holomorphic.
    
    Observe that if $z \in \C$ and $\gamma \in \Gamma$, it follows that
    $$\pi^\prime \circ F(z) = f \circ \pi(z) = f \circ \pi(z + \gamma) = \pi^\prime \circ F(z + \gamma),$$
    and so $F(z) - F(z + \gamma) \in \Gamma^\prime$.
    In particular, for fixed $\gamma \in \Gamma$, we have a map 
    \begin{align*}
      d \colon \C &\rightarrow \Gamma^\prime\\
      z &\mapsto F(z) - F(z + \gamma).
    \end{align*}
    Since $d$ is a holomorphic, hence continuous, map to a discrete set, we see that $d$ is constant and so $d^\prime \equiv 0$.
    Then it follows that $F^\prime(z) = F^\prime(z + \gamma)$; namely, $F^\prime$ is doubly periodic, hence constant.
    Therefore, since $F(0) = 0$, we obtain $F(z) = \alpha z$ for some $\alpha \in \C^\times$, as desired.
  \end{proof}
\end{thm}
\end{document}
