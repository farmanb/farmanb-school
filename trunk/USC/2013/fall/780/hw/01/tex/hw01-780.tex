\documentclass[10pt]{amsart}
\usepackage{amsmath,amsthm,amssymb,amsfonts,enumerate,mymath,mathtools}
\openup 5pt
\author{Blake Farman\\University of South Carolina}
\title{Math 780:\\Homework 01}
\date{September 4, 2013}
\pdfpagewidth 8.5in
\pdfpageheight 11in
\usepackage[margin=1in]{geometry}

\begin{document}
\maketitle

\providecommand{\p}{\mathfrak{p}}
\providecommand{\m}{\mathfrak{m}}

\newtheorem{thm}{}
\newtheorem{lem}{Lemma}

\begin{thm}\label{ex1}
  Let $I = \R \setminus \Q$ denote the set of irrational numbers.
  Determine whether each of the following is true or false.

  \begin{enumerate}[(a)]
  \item
    $\alpha \in I$ and $\beta \in I$ implies $\alpha + \beta \in I$.
  \item
    $\alpha \in I$ and $\beta \in I$ implies $\alpha\beta \in I$.
  \item
    $\alpha \in \Q \setminus \left\{ 0 \right\}$ and $\beta \in I$ implies $\alpha + \beta \in I$ and $\alpha\beta \in I$.
  \item
    $\alpha \in I$ and $\beta \in \Q \setminus \left\{0 \right\}$ implies $\alpha^\beta \in I$.
  \item
    $\alpha \in \Q \setminus \{1\}$ and $\beta \in I$ implies $\alpha^\beta \in I$.
  \end{enumerate}
  
  \begin{proof}
    \begin{enumerate}[(a)]
    \item
      This is false; $\sqrt{2} + (- \sqrt{2}) = 0 \in \Q$.
    \item
      This is false; $\sqrt{2}\sqrt{2} = 2 \in \Q$.
    \item
      This is true.
      Let $\alpha \in \Q \setminus \left\{ 0 \right\}$ and $\beta \in I$ be given.
      Suppose to the contrary that $\alpha + \beta \in \Q$ holds.
      Then
      $$(\alpha + \beta) - \alpha = \beta \in \Q,$$
      a contradiction.
      Similarly, if $\alpha\beta \in \Q$, then
      $$\alpha^{-1}(\alpha\beta) = (\alpha^{-1}\alpha)\beta = \beta \in \Q,$$ 
      again a contradiction.
    \item
      This is false; $\sqrt{2}^2 = 2 \in \Q$.
    \item
      This is false; for any positive $\beta \in I$, $0^\beta = 0 \in \Q$.
    \end{enumerate}
  \end{proof}
\end{thm}

\begin{thm}\label{ex2}
  Prove that $\sqrt{n}$ is irrational whenever $n$ is a positive integer which is not a square.

  \begin{proof}
    Let $n$ be a positive, square-free integer and suppose to the contrary that $\sqrt{n}$ were rational.
    There exist two coprime, positive integers $a$ and $b$ such that $\sqrt{n} = a/b$.
    Squaring both sides yields $n = a^2/b^2 \in \Z$.
    However, $a$ and $b$ were assumed to be coprime, whence $b = 1$ (the only positive unit in $\Z$) and $n = a^2$, a contradiction.
    Therefore $\sqrt{n}$ is irrational.
  \end{proof}
\end{thm}

\begin{thm}\label{ex3}
  Prove that $\sqrt{2} + \sqrt{3}$ is irrational.

  \begin{proof}
    Suppose to the contrary that there exist two positive, coprime integers $a$ and $b$ with $\sqrt{2} + \sqrt{3} = a/b$.
    After squaring both sides and performing some routine algebra, we have 
    $$\sqrt{6} = \frac{a^2 - 5b^2}{2b^2} \in \Q,$$
    contradicting Exercise~\ref{ex2}.
    Therefore $\sqrt{2} + \sqrt{3}$ is irrational.
  \end{proof}
\end{thm}

\begin{thm}\label{ex4}
  Prove that $\sqrt{2} + \sqrt{3} + \sqrt{5}$ is irrational.
  
  \begin{proof}
    Let $\alpha = \sqrt{2} + \sqrt{3} + \sqrt{5}$ and suppose to the contrary that $\alpha$ is rational.
    Observe that
    $$\alpha^2 = 2\sqrt{6} + 2\sqrt{5}(\sqrt{2} + \sqrt{3} + \sqrt{5}) = 2\sqrt{6} + 2\alpha\sqrt{5},$$
    whence $\alpha^2 - 2\alpha\sqrt{5} = 2\sqrt{6}$.
    Squaring both sides gives 
    $$\alpha^4 - 4\alpha^3\sqrt{5} + 20\alpha = 24.$$
    But then
    $$\sqrt{5} = \frac{\alpha^4 + 20\alpha - 24}{4\alpha^3} \in \Q,$$
    which contradicts Exercise~\ref{ex2}.
    Therefore $\alpha$ is irrational.
  \end{proof}
\end{thm}  

\begin{thm}\label{ex5}
  Prove that $\log_2{3}$ is irrational.
  
  \begin{proof}
    Suppose to the contrary that for two coprime integers $a$ and $b$, $\log_2{3} = a/b$.
    By the definition of the logarithm, we have $2^{a/b} = 3$, hence $2^a = 3^b$.
    Since $2$ and $3$ are both prime, the only such solution is $a = b = 0$, but this is absurd.
    Therefore $\log_2{3}$ is irrational.
  \end{proof}
\end{thm}
\end{document}
