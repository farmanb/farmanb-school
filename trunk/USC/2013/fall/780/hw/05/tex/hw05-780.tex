\documentclass[10pt]{amsart}
\usepackage{amsmath,amsthm,amssymb,amsfonts,enumerate,mymath,mathtools}
\openup 5pt
\author{Blake Farman\\University of South Carolina}
\title{Math 780:\\Homework 05}
\date{October 21, 2013}
\pdfpagewidth 8.5in
\pdfpageheight 11in
\usepackage[margin=1in]{geometry}

\begin{document}
\maketitle

\providecommand{\p}{\mathfrak{p}}
\providecommand{\m}{\mathfrak{m}}

\newtheorem{thm}{}
\newtheorem{lem}{Lemma}

\begin{lem}\label{lem1}
  Suppose $a$ has order $n$ modulo a fixed prime $p$.
  If $k$ is an integer, then the order of $a^k$ is 
  $$\abs{a^k} = \frac{n}{(n,k)}.$$
  
  \begin{proof}
    If $k = 0$, then we observe that $(n,k) = n$ and $\abs{a^k} = 1$, as desired.
    If $k < 0$, then we may replace $a$ by $a^{p-2}$ and take $k = -k$.
    Hence we may assume that $k > 0$.
    Let $d = (n,k)$ and observe that $(a^k)^{n/d} = (a^n)^\alpha = 1,$ for some integer $\alpha$.
    Consider any other integer $s$ such that $(a^k)^s = 1$.
    Since $n$ is the order of $a$, it follows that $sk$ must be a multiple of $n$.
    Minimality of $n/d$ then follows directly from the fact that $k(n/d) = \text{lcm}(n,k).$
  \end{proof}
\end{lem}

\begin{thm}\label{ex1}
  Let $p$ be a prime, let $g$ be a primitive root modulo $p$, and let $k$ be an integer.
  Prove that $g^k$ is a primitive root modulo $p$ if and only if $(k,p-1) = 1$.
  
  \begin{proof}
    Let $d = (k, p-1)$.
    By definition, $g^k$ is a primitive root if and only if the order of $g^k$ is $p-1$.
    By Lemma~\ref{lem1}, the order of $g^k$ is 
    $$\abs{g^k} = \frac{\abs{g}}{(k, \abs{g})} = \frac{p-1}{d}.$$
    Since $(p-1)/d = p-1$ if and only if $d = 1$, the result follows directly.
    %
    %Assume $g^k$ is a primitive root.
    %Assume that $d \neq 1$ and note that $(p-1)/d < p-1$.
    %Since $d \mid k$ it follows that
    %$$(g^k)^\frac{p-1}{d} = (g^{p-1})^\frac{k}{d} = 1^\frac{k}{d} = 1.$$
    %Hence the order of $g^k$ is strictly less than $p-1$ and so $g^k$ is not a primitive root.
    
    %Convsersely, assume $d = 1$.
    %It follows directly from the fact that the order of $g^k$ is 
    %$$\abs{g^k} = \frac{\abs{g}}{d} = \frac{p-1}{d} = p-1$$ 
    %that $g^k$ is a primitive root.
    %Therefore $g^k$ is a primitive root modulo $p$ if and only if $(k,p-1) = 1$.
  \end{proof}
\end{thm}

\begin{thm}\label{ex2}
  \begin{enumerate}[(a)]
  \item
    Prove that if $p$ is a prime congruent to $1 \pmod{3}$, then there are exactly $(p-1)/3$ non-zero incongruent cubes modulo $p$.
  \item
    Prove that if $p$ is a prime not congruent to $1 \pmod{3}$, then there are exactly $p-1$ non-zero incongruenct cubes modulo $p$.
  \item
    Generalize parts (a) and (b) to $k^\text{th}$ powers modulo a prime.
    In other words, find a precise description similar to the above for the number of $k^\text{th}$ powers modulo a prime
  \end{enumerate}
  
  \begin{proof}
    \begin{enumerate}[(a)]
    \item
      Let $g$ be a primitive root for $p$.
      Note that $3 \mid p-1$ since $p \equiv 1 \pmod{3}$.
      The non-zero cubes modulo $p$ are $(g^a)^3$ for $a \in \{1, 2, \ldots, p-1\}$.
      Since $(g^a)^3 = (g^3)^a$, the number of distinct cubes is then precisely the order of $g^3$, which is 
      $$\abs{g^3} = \frac{\abs{g}}{(3,\abs{g})} = \frac{p-1}{(3, p-1)} = \frac{p-1}{3}.$$
    \item
      Let $g$ be a primitive root for $p$.
      Note that $3 \nmid p-1$ since $p \not \equiv 1 \pmod{3}$, and so $(3, p-1) = 1$.
      The non-zero cubes modulo $p$ are $(g^a)^3$ for $a \in \{1, 2, \ldots, p-1\}$.
      Since $(g^a)^3 = (g^3)^a$, the number of distinct cubes is then precisely the order of $g^3$, which is 
      $$\abs{g^3} = \frac{\abs{g}}{(3,\abs{g})} = \frac{p-1}{(3,p-1)} = p-1.$$
    \item
      Fix an integer $k$ and let $g$ a primitive root modulo $p$.
      If $k = 0$, then there is precisely one such element, $1$.
      If $k < 0$, then we may replace $g$ by $g^{p-2}$, the multiplicative inverse of $g$, and take $k = -k$.
      Hence we may assume that $k > 0$.
      The non-zero $k^\text{th}$ powers modulo $p$ are $(g^a)^k$ for $a \in \{1, 2, \ldots, p-1\}$.
      Since $(g^a)^k = (g^k)^a$, the number of distinct $k^\text{th}$ powers is then precisely the order of $g^k$, which is 
      $$\abs{g^k} = \frac{\abs{g}}{(k,\abs{g})} = \frac{p-1}{(k,p-1)}.$$
    \end{enumerate}
  \end{proof}
\end{thm}

\begin{thm}\label{ex3}
  Calculate the Legendre symbols $\left(\frac{30}{71}\right)$ and $\left(\frac{-56}{103}\right)$.

  \begin{proof}
    Observe that $30 = 2\cdot 3 \cdot 5$  and $71$ is prime.
    Then 
    \begin{eqnarray*}
      \left(\frac{30}{71}\right) &=& \left(\frac{2}{71}\right)\left(\frac{3}{71}\right)\left(\frac{5}{71}\right)\\
      &=& -\left(\frac{2}{71}\right)\left(\frac{71}{3}\right)\left(\frac{71}{5}\right) \\
      &=& -\left(\frac{2}{71}\right)\left(\frac{2}{3}\right)\left(\frac{1}{5}\right)\\
      &=& -\left(\frac{2}{71}\right)(-1)(1)\\
      &=& 1
      \end{eqnarray*}
    since $12^2 = 144 = 71(2) + 2$.
    
    Observe that $-56 \equiv 47 \pmod{103}$ so we have
    \begin{eqnarray*}
      \left(\frac{-56}{103}\right) &=& \left(\frac{47}{103}\right)\\
      &=& -\left(\frac{103}{47}\right)\\
      &=& -\left(\frac{9}{47}\right)\\
      &=& -1
    \end{eqnarray*}
    since $9 = 3^2$.
  \end{proof}
\end{thm}
\begin{thm}\label{ex4}
  Let $p$ denote a prime.
  Prove that there is a solution to $x^2 - 3x + 3 \equiv 0 \pmod{p}$ if and only if $p = 3$ or $p \equiv 1 \pmod{3}$.

  \begin{proof}
    Observe that $x^2 - 3x + 3 \equiv x^2 + x + 1 \pmod{2}$ is irreducible, so we may assume $p$ is odd.
    Furthermore, note that $x \equiv 0 \pmod{p}$ is a root of $x^2 - 3x + 3$ modulo $p$ if and only if $p = 3$, so we may assume $p \not \equiv 0 \pmod{3}$.
    Since the roots of $x^2 - 3x + 3$ are 
    $$\frac{3 \pm \sqrt{-3}}{2},$$
    it is clear that $x^2 - 3x + 3$ has a root modulo $p$ if and only if $-3$ is quadratic residue modulo $p$.
    If $p \equiv 1 \pmod{4}$, then 
    $$\left(\frac{-3}{p}\right) = \left(\frac{-1}{p}\right)\left(\frac{3}{p}\right) = \left(\frac{3}{p}\right) = \left(\frac{p}{3}\right)$$
    and if $p \equiv 3 \pmod{4}$, then
    $$\left(\frac{-3}{p}\right) = \left(\frac{-1}{p}\right)\left(\frac{3}{p}\right) = -\left(\frac{-1}{p}\right)\left(\frac{p}{3}\right) = \left(\frac{p}{3}\right).$$
    Hence $x^2 - 3x + 3$ has a root modulo $p$ if and only if $p$ is a quadratic residue modulo $3$.
    Therefore, since $1$ is the only non-zero square modulo $3$, it follows directly that $x^2 - 3x + 3$ has a root modulo $p$ if and only if $p = 3$ or $p \equiv 1 \pmod{3}$.
  \end{proof}
\end{thm}

\begin{thm}
  Prove that for every prime $p$, there is an $a \in \{1,2,\ldots,9\}$ such that both $a$ and $a + 1$ are squares modulo $p$.

  \begin{proof}
    If $p = 2$, then $1 = 1^2$ and $1 + 1 = 2 \equiv 0 \equiv 0^2 \pmod{p}$.
    Assume $p$ is odd.
  \end{proof}
\end{thm}

\begin{thm}
  \begin{enumerate}[(a)]
  \item
    For every odd prime $p$, prove either $\left(\frac{-1}{p}\right) = 1$, $\left(\frac{2}{p}\right) = 1$, or $\left(\frac{-2}{p}\right) = 1$.
  \item
    Prove that $x^4 + 1$ is reducible modulo $p$ for every prime $p$.
  \end{enumerate}

  \begin{proof}
    \begin{enumerate}[(a)]
    \item
      If $p \equiv 1 \pmod{4}$, then $\left(\frac{-1}{p}\right) = 1$.
      Suppose $p \equiv 3 \pmod{4}$.
      Either $p \equiv \pm 1 \pmod{8}$ or $p \equiv \pm 3 \pmod{8}$.
      If the former holds, then
      $$\left(\frac{2}{p}\right) = 1.$$
      If the latter holds, then
      $$\left(\frac{-2}{p}\right) = \left(\frac{-1}{p}\right)\left(\frac{2}{p}\right) = (-1)(-1) = 1.$$
    \item
      If $p = 2$, then 
        $$x^4 + 1 = x^4 + 1^4 \equiv (x + 1)^4 \pmod{2}.$$
        If $p$ is odd, observe that $p^2 - 1$ is divisible by 8 since $p \equiv 1, 3, 5, 7 \pmod{8}$ and $1^2 \equiv 3^2 \equiv 5^2 \equiv 7^2 \equiv 1 \pmod{8}$.
        Let $p^2 - 1 = 8k$ for some integer $k$.
        Consider the polynomial $x^{p^2 - 1} - 1$.
        Writing $X = x^8$, it follows from $x^{p^2 - 1} - 1 = X^k - 1 = (X - 1)(X^{k - 1} + X^{k - 2} + \ldots + 1)$ that $x^8 - 1 = (x^4 + 1)(x^4 - 1)$ divides the polynomial $x^{p^2 - 1} - 1$, which in turn divides $x^{p^2} - x$.
        Regarding $\F_{p^2}$ as the splitting field of the latter polynomial, it follows that all the roots of $x^4 + 1$ lie in $\F_{p^2}$ and so have degree at most $2$ over $\F_p$.
        Therefore $x^4 + 1$ cannot be irreducible modulo $p$.
    \end{enumerate}
  \end{proof}
\end{thm}
\end{document}
