\documentclass[10pt]{amsart}
\usepackage{amsmath,amsthm,amssymb,amsfonts,enumerate,mymath,mathtools}
\openup 5pt
\author{Blake Farman\\University of South Carolina}
\title{Math 780:\\Homework 03}
\date{September 25, 2013}
\pdfpagewidth 8.5in
\pdfpageheight 11in
\usepackage[margin=1in]{geometry}

\begin{document}
\maketitle

\providecommand{\p}{\mathfrak{p}}
\providecommand{\m}{\mathfrak{m}}

\newtheorem{thm}{}
\newtheorem{lem}{Lemma}

\begin{thm}\label{ex1}
  Prove that 1105 and 1729 are absolute pseudo-primes.
  
  \begin{proof}
    Observe that $1105 = 5 \cdot 13 \cdot 17$, $1104 = 2^4 \cdot 3 \cdot 23$, $1729 = 7 \cdot 13 \cdot 19$, and $1728 = 3^3 \cdot 2^6$.
    Take $a \in \Z$ such that $(a, 1105) = 1$.
    Then we observe that $4$, $12$, and $16$ divide $1104$ and so by Fermat's Little Theorem
    \begin{eqnarray*}
      a^{1104} &\equiv 1 \pmod{5}\\
      a^{1104} &\equiv 1 \pmod{13}\\
      a^{1104} &\equiv 1 \pmod{17}.
    \end{eqnarray*}
    Therefore by the CRT, $a^{1104} \equiv 1 \pmod{1105}$, and $1105$ is an absolute pseudo-prime.
    
    Take $a \in \Z$ such that $(a, 1729) = 1$.
    Again we observe that $6$, $12$, and $18$ all divide $1728$ and so it follows from Fermat's Little Theorem that
    \begin{eqnarray*}
      a^{1104} &\equiv 1 \pmod{7}\\
      a^{1104} &\equiv 1 \pmod{13}\\
      a^{1104} &\equiv 1 \pmod{19}.
    \end{eqnarray*}
    Therefore by the CRT, $a^{1728} \equiv 1 \pmod{1729}$, and $1729$ is an absolute pseudo-prime.
  \end{proof}
\end{thm}

\begin{thm}\label{ex2}
  Prove that if $n$ is a pseudo-prime, then $2^n - 1$ is a pseudo-prime.

  \begin{proof}
    Let $N = 2^n - 1$ and observe that $N \equiv 1 \pmod{n}$, hence $N = kn + 1$ for some integer $k$.
    Then
    $$2^{N} - 1 = 2^{nk + 1} - 1 = 2(N+1)^k - 1 =\sum_{i=0}^{k-1} {k \choose i}N^{k-i} + 1 \equiv 1 \pmod{N}.$$
    Therefore $2^{N} \equiv 2 \pmod{N}$, as desired.
  \end{proof}
\end{thm}

\begin{thm}\label{ex1}
  Prove the converse of Wilson's Theorem.
  More specifically, prove that $n$ is an integer strictly larger than 1 for which $(n-1)! \equiv -1 \pmod{n}$, then $n$ is prime.
  
  \begin{proof}
    %Suppose to the contrary that $n$ is composite and $(n-1)! \equiv -1 \pmod{n}$.
    %When $n = 4$, we have $3! \equiv 2 \pmod{4}$.
    %Assume that $n > 4$.
    %Observe that for every proper divisor, $d$, of $n$, $a \mid (n-1)!$.
    %Hence 
    Suppose that we have a factorisation $n = kd$.
    Observe that $d \mid (n-1)!$, hence $(n-1)! = md$ for some integer $m$.
    By assumption, there exists an integer $\ell$ such that
    $$(n-1)! = \ell n - 1 = \ell(kd) - 1 = md,$$
    whence $1 = d(\ell k - m)$.
    Therefore $d$ is a unit and $n$ is irreducible, hence prime, as desired.
  \end{proof}
\end{thm}
\end{document}
