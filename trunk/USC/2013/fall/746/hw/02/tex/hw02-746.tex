\documentclass[10pt]{amsart}
\usepackage{amsmath,amsthm,amssymb,amsfonts,enumerate,mymath,mathtools}
\openup 5pt
\author{Blake Farman\\University of South Carolina}
\title{Math 746:\\Homework 02}
\date{October 10, 2013}
\pdfpagewidth 8.5in
\pdfpageheight 11in
\usepackage[margin=1in]{geometry}

\begin{document}
\maketitle

\providecommand{\p}{\mathfrak{p}}
\providecommand{\m}{\mathfrak{m}}
\providecommand{\Ann}[1]{\operatorname{Ann}\left(#1\right)}
\newtheorem{thm}{}
\newtheorem{lem}{Lemma}

\begin{thm}
	Let $A$ be a ring and $M$ an $A$-module.
	Let $I$ be an ideal of $A$ that is maximal amongst all annihilators of non-zero elements of $M$.
	Prove that $I$ is a prime ideal.

\begin{proof}
	There exists, by hypothesis, an element $m \in M$ such that $I = \Ann{m}$.
	Suppose that for some $a, b \in A$ that $ab \in I$.
	Assume that $b \not \in I$, so that $bm \neq 0$ and $a \in \Ann{bm} \subset A$.
	Observe that if $x \in I$, then
		$$x(bm) = (xb)m = (bx)m = b(xm) = 0$$
	implies that $I \subseteq \Ann{bm}$.
	Hence by maximality, $\Ann{bm} = I$.
	Therefore $a \in I$ implies that $I$ is prime, as desired.
\end{proof}
\end{thm}

\begin{thm}
  Let $A$ be a ring and $I$ an ideal that is maximal amongst all infinitely generated ideals in $A$.  Prove that $I$ is prime.
  
  \begin{proof}
    Suppose to the contrary that there exist elements $a$ and $b$ of $A$ with $a,b \not \in I$, but $ab \in I$.
    Let $\mathfrak{a} = (a) + I$ and observe that, since $I$ is maximal, this ideal must be finitely generated, say $\mathfrak{a} = (g_1, g_2, \ldots, g_n)$, where $g_i = a_ia + x_i$ with $a_i \in A$ and $x_i \in I$.
   % It is easy to see that this reduces to the ideal $\mathfrak{a} = (a, x_1, \ldots, x_n)$.
    
    Define the set $\mathfrak{b} = \left\{y \in A \;\middle\vert\; ay \in I \right\}$.
    Observe that for $y, y^\prime \in \mathfrak{b}$ we have 
    $$a(y - y^\prime) = ay - ay^\prime \in I$$ 
    so that $\mathfrak{b}$ is a subgroup of $A$, and for any $a^\prime \in A$, 
    $$a(a^\prime y) = a^\prime(ay) \in I,$$
    which establishes that $\mathfrak{b}$ is an ideal.
    It is clear that $\mathfrak{b}$ contains $I$, so we observe that $\mathfrak{b}$ does not contain a unit, for if $u \in A^\times \cap \mathfrak{b}$, then $ua \in I$ and so $a = u^{-1}(ua) \in I$, contrary to our assumption.
    Note that since $ab \in I$, it follows that $b \in \mathfrak{b}$, and hence the containment is necessarily strict.
    By maximality once more, we have that $\mathfrak{b}$ is finitely generated, say $\mathfrak{b} = (y_1, \ldots, y_m)$.
    
    Now consider the ideal $\mathfrak{c} = (ay_1, \ldots, ay_m, x_1, \ldots, x_n) \subseteq I$.
    Let $x \in I \subset \mathfrak{a}$ be given and write 
    $$x = c_1g_1 + \ldots + c_ng_n  = c_1(a_1a + x_1) + \ldots + c_n(a_na + x_n) = (c_1a_1 + \ldots c_na_n)a + c_1x_1 + \ldots c_nx_n$$% = a^\prime a + c_1x_1 + \ldots c_nx_n$$ 
    for some $c_1, \ldots, c_n \in A$.
   % It follows that $a^\prime a \in I$ and so $a^\prime \in \mathfrak{b}$.
    It follows that $(c_1a_1 + \ldots + c_na_n)a \in I$ and so $(c_1a_1 + \ldots + c_na_n) \in \mathfrak{b}$.
    Hence we may rewrite $x$ as 
    $$x = a(d_1y_1 + \ldots + d_my_m) + c_1x_1 + \ldots c_nx_n \in \mathfrak{c}$$
    for some $d_i \in A$.
    This establishes the reverse inclusion, $I \subseteq \mathfrak{c}$.
    However, this contradicts the assumption that $I$ is not finitely generated.
    Therefore one of $a \in I$ or $b \in I$ must hold, and thus $I$ is prime.
  \end{proof}
\end{thm}

\begin{thm}
	In a ring A, let I be an ideal which is maximal amongst non-principal ideals. 
	Prove that I is prime ideal.

	\begin{proof}
		Suppose to the contrary that $a, b \in A$ are such that $ab \in I$ but $a,b \not \in I$.
		Let $\mathfrak{a} = (a) + I$.
		Since $\mathfrak{a}$ properly contains $I$, it follows that there exists an $\alpha \in A$ with $\mathfrak{a} = (\alpha)$.
		Let $\mathfrak{b} = \left\{x \in A \;\middle\vert\; x\alpha \in I\right\}$.
		Observe that $\mathfrak{b}$ is a proper ideal of $A$ containing $b$ and $I$, hence $\mathfrak{b}$ properly contians $I$ and so there exists some $\beta \in A$ with $\mathfrak{b} = (\beta)$.

		Let $\mathfrak{c} = (\alpha \beta)$  and note that $\mathfrak{c} \subseteq I$.
		Let $x \in I$ be given and write $x = a^\prime \alpha$ for some $a^\prime \in A$.
		Then it follows that $a^\prime \in \mathfrak{b}$, so for some $c \in A$ we have 
		$$x = a^\prime \alpha = (c\beta)\alpha = c(\alpha \beta) \in \mathfrak{c}.$$
		Hence $I \subseteq \mathfrak{c}$, contrary to the assumption that $I$ is not principal.
		Therefore one of $a \in I$ or $b \in I$ must hold and $I$ is prime, as desired.
	\end{proof}
\end{thm}

\begin{thm}
	In a ring A, let I be an ideal which is maximal amongst ideals that are not countably generated. 
	Prove that I is prime ideal.

\begin{proof}
	Suppose to the contrary that $a, b \in A$ are such that $ab \in I$ but $a,b \not \in I$.
	Let $\mathfrak{a} = (a) + I$.
	Since $\mathfrak{a}$ properly contains $I$, it follows that there exists a countable set of generators $\{g_n\}_{n \in \N}$ with $g_n = a_n a + x_n$ with $a_n \in A$ and $x_n \in I$.
	Let $\mathfrak{b} = \left\{x \in A \;\middle\vert\; ax \in I\right\}$.
	Observe that $\mathfrak{b}$ is a proper ideal of $A$ containing $b$ and $I$, hence $\mathfrak{b}$ properly contains $I$ and so there exist countably many generators $\{y_n\}_{n \in \N}$

	Let $\mathfrak{c}$ be the ideal of $A$ generated by the set $\{ay_n\}_{n \in N} \cup \{x_n\}_{n \in \N}$ and note that $\mathfrak{c} \subseteq I$.
	Let $x \in I$ be given and write 
	$$x = \sum_{n \in \N} c_ng_n = a\sum_{n \in \N}a_nc_n + \sum_{n \in \N}c_n x_n$$
	where $c_n \in A$ and $c_n = 0$ for all but finitely many $n \in \N$.
	Then it follows that $\sum_{n \in \N} a_n c_n \in \mathfrak{b}$, so we have
	$$\sum_{n \in \N} a_n c_n = \sum_{n \in N} d_n y_n$$
	where $d_n \in A$ and $d_n = 0$ for all but finitely many $n \in N$.
	Hence
	$$x =  a\sum_{n \in \N}d_n y_n + \sum_{n \in \N}c_n x_n \in \mathfrak{c}$$
	implies that $I \subseteq \mathfrak{c}$, which contradicts the assumption that $I$ is not finitely generated.
	Therefore one of $a \in I$ or $b \in I$ hold and $I$ is prime, as desired.
\end{proof}
\end{thm}

\end{document}
