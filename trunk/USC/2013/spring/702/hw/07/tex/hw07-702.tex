\documentclass[10pt]{amsart}
\usepackage{amsmath,amsthm,amssymb,amsfonts,enumerate,mymath,mathtools,tikz}
\usetikzlibrary{shapes}

\openup 5pt
\author{Blake Farman\\University of South Carolina}
\title{Math 702:\\Homework 07}
\date{March 29, 2013}
\pdfpagewidth 8.5in
\pdfpageheight 11in
\usepackage[margin=1in]{geometry}

\begin{document}
\maketitle

\providecommand{\p}{\mathfrak{p}}
\providecommand{\m}{\mathfrak{m}}

\newtheorem{thm}{}
\newtheorem{lem}{Lemma}

\newcommand{\End}[2]{\operatorname{End}_{#1}\left(#2\right)}
\newcommand{\Hom}[2]{\operatorname{Hom}_{#1}\left(#2\right)}

\begin{thm}
  Let $F$ be a field of characteristic $p > 0$.
  Suppose that $f(x) = x^p - x - a \in F[x]$ has no root in $F$.
  Prove that $f$ has splitting field $K/F$, Galois with $\Gal{K/F} \cong \Z/p\Z$.

  \begin{proof}
    First observe that $D_x(f) = -1$ implies that $f$ is separable and thus has $p$ distinct roots.
    Let $\alpha$ be any root of $f$ in $K/F$.
    Since the prime subfield of $F$ is $\F_p$, it follows that for $1 \leq i \leq p-1$
    $$(\alpha + i)^p - (\alpha + i) - a = \alpha^p - i - (\alpha + i) - a = \alpha^p - \alpha - a = 0.$$
    Hence the $p$ roots of $f$ are $\alpha + i$ for $0 \leq i \leq p-1$.
    Then $F(\alpha)/F \cong F[x]/(f)$ is a degree $p$ extension of $F$ containing all the roots of $f$, and is the splitting field of $f$ over $F$.
    Hence $K/F = F(\alpha)/F$ is Galois of degree $p$.
    Therefore $\Gal{K/F}$ is cyclic of order $p$, as desired.
  \end{proof}
\end{thm}

\begin{thm}
  Let $K/F$ be a finite extension, and let $\alpha \in K$.
  Define the norm of $\alpha$ from $K$ to $F$ to be 
  $$\Norm{K/F}{\alpha} = \prod_{\sigma \in \Emb{K/F}} \sigma(\alpha),$$
  where $\Emb{K/F}$ denotes the embeddings of $K$ into an algebraic closure of $K$ containing $F$ which fix $F$.
  Note that if $K/F$ is Galois, then we have $\Emb{K/F} = \Gal{K/F}$.
  \begin{enumerate}[(a)]
  \item
    Prove that $\Norm{K/F}{\alpha} \in F$.
  \item
    Let $\alpha, \beta \in K$.
    Prove that $\Norm{K/F}{\alpha\beta} = \Norm{K/F}{\alpha}\Norm{K/F}{\beta}$.
  \item
    Suppose that $[K : F] = n$, and let $m_{\alpha, F}(x) = x^d + a_{d - 1}x^{d-1} + \ldots + a_1x + a_0 \in F[x]$ be the minimal polynomial of $\alpha$ over $F$.
    Prove that $d \mid n$, and that for fixed $\sigma \in \Emb{K/F}$, we have $\abs{\left\{\tau \in \Emb{K/F} \;\middle\vert\; \sigma(\alpha) = \tau(\alpha)\right\}} = n/d$.
    Conclude that $\Norm{K/F}{\alpha} = (-1)^n a_0^{n/d}$.
  \end{enumerate}

  \begin{proof}
    Let $L/F$ be the Galois closure of $K$.
    Let $H_1$ be the subgroup of $\Gal{L/F}$ that fixes $K$.
    \begin{enumerate}[(a)]
    \item
      Regard $\Emb{K/F}$ as the coset space $\Gal{L/F}/H_1$ and let $\sigma$ be an element of $\Gal{L/F}/H_1$.
      For any element $\tau$ of $\Gal{L/F}$, $\tau\sigma$ is again another set of coset representatives in $\Gal{L/F}/H_1$.
      Therefore it follows that
      $$\tau\left(\Norm{K/F}{\alpha}\right) = \prod_{\sigma \in \Emb{K/F}} \tau\sigma(\alpha) = \Norm{K/F}{\alpha}$$
      and, since $\tau$ was arbitrary, $\Norm{K/F}{\alpha} \in F$.
    \item
      Since each element $\sigma \in \Emb{K/F}$ is a homomorphism,
      \begin{eqnarray*}
        \Norm{K/F}{\alpha\beta} &=& \prod_{\sigma \in \Emb{K/F}} \sigma(\alpha\beta) = \prod_{\sigma \in \Emb{K/F}} \sigma(\alpha)\sigma(\beta)\\
        &=& \prod_{\sigma \in \Emb{K/F}}\sigma(\alpha)\prod_{\sigma \in \Emb{K/F}}\sigma(\beta)\\
        &=& \Norm{K/F}{\alpha}\Norm{K/F}{\beta}.
      \end{eqnarray*}
    \item
      First observe that because $\alpha \in K$, $F(\alpha)/F$ is a subfield of $K$ and by the multiplicity of degrees in towers, 
      $$[K : F(\alpha)] = \frac{[K : F]}{[F(\alpha) : F]} = n/d \in \Z.$$
      Let $H_2$ be the subgroup of $\Gal{L/F}$ that fixes $F(\alpha)$, and observe that $H_1 \subseteq H_2$, and so as coset spaces, we have $H_2/H_1 \subseteq \Gal{L/F} / H_1$.
      Note that if for fixed $\sigma \in \Emb{K/F}$, if $\sigma(\alpha) = \tau(\alpha)$, then 
      $$\tau^{-1}\sigma(\alpha) = \tau^{-1}\tau(\alpha) = \alpha$$
      and similarly $\sigma^{-1}\tau(\alpha) = \alpha$ imply that $\sigma, \tau$ both generate the same coset in $H_2/H_1$.
      Likewise, if they generate the same coset in $H_2/H_1$, then necessarily $\sigma(\alpha) = \tau(\alpha)$.
      Since $n = [\Gal{L/F} : H_1] = [\Gal{L/F} : H_2][H_2 : H_1] = d[H_2 : H_1]$, we have $[H_2 : H_1] = n/d$.
      Hence there are $n/d$ elements in the coset of $H_2/H_1$ represented by $\sigma$, and thus $\abs{\left\{\tau \in \Emb{K/F} \;\middle\vert\; \sigma(\alpha) = \tau(\alpha)\right\}} = n/d.$
      
      Now observe that 
      $$m_{\alpha, F}(x) = \prod_{\sigma \in \Emb{F(\alpha)/F}} (x - \sigma(\alpha))$$
      and so it follows from $\deg{m_{\alpha,F}} = d$ that
      $$\prod_{\sigma \in \Emb{F(\alpha)/F}} \sigma(\alpha) = (-1)^d a_0.$$
      Therefore, by partitioning the embeddings of $\Emb{K/F}$ as above, we have
      $$\prod_{\sigma \in \Emb{K/F}} \sigma(\alpha) = ((-1)^d a_0)^{n/d} = (-1)^n a_0^{n/d},$$
      as desired.
    \end{enumerate}
  \end{proof}
\end{thm}

\setcounter{thm}{3}
\begin{thm}
  Suppose that $p$ is prime and that $f(x) \in \Q[x]$ is irreducible of degree $p$ with precisely two non-real roots in $\C$.
  Prove that $f(x)$ has splitting field $K$ with $\Gal{K/F} \cong S_p$.
  Use this result to determine the Galois group of the splitting field of $f(x) = x^5 - 2x^3 - 8x + 2$.
  
  \begin{proof}
    Identify $\Gal{K/F}$ with a subgroup of $S_p$.
    Observe that because $\Q[x]/(f)$ is a subfield of $K$ of degree $p$, it follows from Cauchy's Theorem that $\Gal{K/F}$ contains an element of order $p$, which can be identified with some $p$-cycle.
    Then since there are two complex roots, complex conjugation is an involution that may be identified with a transposition.
    Therefore by Exercise 2 of Homework 6 from last semester, it follows that these two elements generate $S_p$ and $\Gal{K/F} \cong S_p$, as desired.
  \end{proof}
\end{thm}
\begin{thm}
  Let $\zeta \in \mu_3 \subseteq \C$ be primitive $(\neq 1)$, and let $\sigma, \tau \in \Aut{\C(t)/\C}$ with
  $$\sigma \colon t \mapsto \zeta t,\ \ \tau \colon t \mapsto \frac{1}{t}.$$
  \begin{enumerate}[(a)]
  \item
    Define $H = \left< \sigma, \tau \right>$.
    Prove that $\sigma^3 = \tau^2 = 1$, $\tau\sigma = \sigma^{-1}\tau$, and $\abs{H} = 6$.
    Identify $H$ as a familiar group.
  \item
    Let $u = t^3 + 1/t^3$.
    Prove that $\C(t)^H = \C(u)$.
    Conclude that $\C(t)/C(u)$ is Galois with $\Gal{\C(t)/\C(u)} \cong H$.
  \item
    Find a cubic $f(x) \in \C(u)[x]$ with splitting field $\C(t)$.
  \end{enumerate}
  
  \begin{proof}
    \begin{enumerate}[(a)]
    \item
      To see that $\sigma^3 = \tau^2 = 1$, observe that 
      $$t \xmapsto{\sigma} \zeta t \xmapsto{\sigma} \zeta^2t \xmapsto{\sigma} \zeta^3 t = t\ \text{and}\ t \xmapsto{\tau} \frac{1}{t} \xmapsto{\tau} \frac{1}{\frac{1}{t}} = t.$$
      To see that $\tau\sigma = \sigma^{-1}\tau = \sigma^2\tau$, observe that
      $$t \xmapsto{\tau} \frac{1}{t} \xmapsto{\sigma} \frac{1}{\zeta t} \xmapsto{\sigma} \frac{1}{\zeta^2t}\ \text{and}\ t \xmapsto{\sigma} \zeta t \xmapsto{\sigma} \zeta^2t \xmapsto{\tau} \frac{1}{\zeta^2t}.$$
      Then it is clear that $H = \left< \sigma, \tau \;\middle\vert\; \sigma^3 = \tau^2 = 1,\, \sigma \tau = \sigma^{-1}\tau \right> \cong S_3$ (or $D_6$), the unique non-abelian group of order 6 up to isomorphism.
    \item
      To see that $H$ fixes $\C(u)$, compute
      $$\sigma(u) = \sigma(t)^3 + \frac{1}{\sigma(t)^3} = (\zeta t)^3 + \frac{1}{(\zeta t)^3} = t^3 + \frac{1}{t^3} = u$$
      and
      $$\tau(u) = \tau(t)^3 + \frac{1}{\tau(t)^3} = \left(\frac{1}{t}\right)^3 + \frac{1}{\left(\frac{1}{\frac{1}{t}}\right)^3} = \frac{1}{t^3} + {t^3} = u.$$
      Then since $u = (t^6 + 1)/t^3 \in \C(t)$, it follows from $[\C(t) : \C(u)] = \max\left\{\deg{t^6 + 1}, \deg{t^3}\right\} = 6$ that $\C(u) = \C(t)^H$.
      Therefore $\C(t)/\C(u)$ is Galois with $\Gal{\C(t)/\C(u)} \cong H$ by the Fundamental Theorem.
    \item
      Using the symmetrizing functions $1 + \tau$, $1 + \tau\sigma$, and $1 + \tau\sigma^2$ and the same argument above, mutatis mutandis, we have that the degree three subfields of $\C(t)$ are $\C(t + 1/t)$, $\C(t + \zeta/t)$, and $\C(t + \zeta^2/t)$.
      Now by computing 
      $$(t + 1/t)^3 = t^3 + 3t + 3/t + 1/t^3 = u + 3(t + 1/t)$$
      it follows that $t + 1/t$ is a root of $x^3 - 3x - u$.
      Similarly, 
      $$(\zeta(t + \zeta/t))^3 = t^3 + \zeta t + \zeta^2/t + 1/t^3 = u + 3(\zeta(t + \zeta/t))$$
      and
      $$(\zeta^2(t + \zeta^2/t))^3 = t^3 + \zeta^2 t + \zeta/t + 1/t^3 = u + 3(\zeta^2(t + \zeta^2/t))$$
      show that these are the other two roots.
      Observe that $\zeta(\zeta^2(t + \zeta^2/t)) = t + \zeta^2/t$ and $\zeta^2(\zeta(t + \zeta/t)) = t + \zeta/t$.
      Hence any field containing all the roots of $x^3 - 3x - u$ contains all the subfields of degree 3, and the smallest such field is $\C(t)$.
      Therefore $f(x) = x^3 - 3x - u$ is a degree three polynomial with coefficients in $\C(u)$ with splitting field $\C(t)$, as desired.
    \end{enumerate}
  \end{proof}
\end{thm}
\end{document}
