\documentclass[10pt]{amsart}
\usepackage{amsmath,amsthm,amssymb,amsfonts,enumerate,mymath,mathtools}
\openup 5pt
\author{Blake Farman\\University of South Carolina}
\title{Math 702:\\Homework 10}
\date{May 8, 2013}
\pdfpagewidth 8.5in
\pdfpageheight 11in
\usepackage[margin=1in]{geometry}

\begin{document}
\maketitle

\providecommand{\p}{\mathfrak{p}}
\providecommand{\m}{\mathfrak{m}}

\newtheorem{thm}{}
\newtheorem{lem}{Lemma}

\begin{thm}
  Suppose that $G$ is an abelian group with generators $\{a_1, a_2, a_3\}$ and relations
  $$2a_1 - a_2 = 0;\ a_1 - 3a_2 = 0;\ a_1 + a_2 + a_3 = 0.$$
  Find the rank of $G$ and give its torsion subgroup in invariant factor form.
  \begin{proof}
    The relations matrix is 
    $$\left(
    \begin{array}{ccc}
      2 & 1 & 1\\
      -1 & -3 & 1\\
      0 & 0 & 1
    \end{array}\right)
    $$
    which reduces with elementary row and column operations to 
    $$\left(
    \begin{array}{ccc}
      1 & 0 & 0\\
      0 & 1 & 0\\
      0 & 0 & 5
    \end{array}\right).
    $$
    Hence the group is isomorphic to $\Z^{\oplus3}/\Z \oplus \Z \oplus \Z/(5)\cong \Z/(5)$.
  \end{proof}
\end{thm}  

\begin{thm}
  Suppose that $R = (r_{ji}) \in \GL{n}{\Z}$.
  What can you say about a finitely generated abelian group $G$ with presentation
  $$\left< a_1, \ldots, a_n \;\middle\vert\; \sum_{j = 1}^n r_{ji}a_j = 0\ \text{for}\ 1 \leq i \leq n \right>?$$
  
  \begin{proof}
    Since the relations matrix is invertible, it can be reduced with elementary row and column operations to $I_n$.
    Hence the group is trivial.
  \end{proof}
\end{thm}  

\begin{thm}
  Suppose that $F$ is a field, that $V$ is an $F$-vector space, that $T \in \Hom{F}{V, v}$ is a linear transformation, that $v \in V$, and that $W = F[x]v \subseteq V_T$ is a cyclic submodule with $W \cong F[x] / (f(x))$, where $f(x) \in F[x]$ has degree $k \geq 1$.
  \begin{enumerate}[(a)]
  \item
    Prove that $B = \{v, T(v), \ldots, T^{k-1}(v)\}$ is a basis for $W$ as an $F$-vector space.
  \item
    Prove that $(T)_B = C(f)$.
  \end{enumerate}
  \begin{proof}
    \begin{enumerate}[(a)]
    \item
      Since there is an isomorphism of $W$ with $F[x]/(f)$, it suffices to show that the image of $\{1, x, \ldots, x^{k-1}\}$ is a basis for $F[x]/(f)$.
      Since $F$ is a field, we may assume without loss of generality that each element $g \in F[x]$ is monic and so may be written uniquely as $g = fq + r$ for some $q,r \in F[x]$ and $\deg{r} < k$ or $r = 0$.
      Hence we may identify the elements of $F[x]/(f)$ with polynomials of degree at most $k-1$, and so $\{1, x, \ldots, x^{k-1}\}$ certainly span the space and are clearly linearly independent.
      The result then follows from the action of $x$ on $W = F[x]v$.
    \item
      Let $f(x) = x^k + a_{k-1}x^{k-1} + \ldots + a_1x + a_0$.
      Since $T$ cycles the basis elements and $T(T^{k-1}) = T^k = - a_{k-1}T^{k-1} - \ldots - a_1T - a_0$, the matrix for $T$ with respect to $B$ is
      $$(T)_B = \left(\begin{array}{cccc}
        0 & 0 & \ldots &-a_0\\
        1 & 0 & \ldots &-a_1\\
        0 & 1 & \ldots &-a_2\\
        \vdots& & \ddots &\vdots\\
        0 & 0 & \ldots & -a_{k-1}\\
      \end{array}\right) = C(f).$$
  \end{enumerate}
  \end{proof}
\end{thm}  

\begin{thm}
  Let $F$ be a field, let $B$ be the standard basis for $F^n$, and let $f(x) \in F[x]$.
  Suppose that $T \in \Hom{F}{V,V}$ has $(T)_B = C(f)$.
  Prove that $f(x) = m_T$, the minimal polynomial for $T$.
  
  \begin{proof}
    First observe that $T$ simply cycles the basis elements as above; if $n = \deg{f}$, for each $1 \leq i \leq n-1$, $T(e_i) = e_{i+1}$ and so we may rewrite $\{e_1, \ldots e_n\} = \{e_1, T(e_1), \ldots, T^{n-1}(e_1)\}$.
    Therefore the vector space is cyclic and isomorphic to $F[x]/(f)$, and so $f = m_T$, the generator for the annihilator.
  \end{proof}
\end{thm}  

\begin{thm}
  Suppose that $T \in \Hom{\R}{\R^3, \R^3}$ has matrix representation in the standard ordered basis $B$ given by 
  $$(T)_B = A = \left(
  \begin{array}{ccc}
    1 & 3 & 3\\
    3 & 1 & 3\\
    -3 & -3 & -5
  \end{array}
  \right) \in \Mat{3 \times 3}{\R}$$
  \begin{enumerate}[(a)]
  \item
    Compute the invariant factors for the $\R[x]$-module $(\R^3)_T$.
  \item
    Compute the rational canonical form for $A$.
  \item
    Compute a basis for $\R^3$ with respect to which the matrix $T$ is in rational canonical form.
  \item
    Give a matrix $P \in \GL{3}{\R}$ with $P^{-1}AP$ in rational canonical form.
  \end{enumerate}
  
  \begin{proof}
    \begin{enumerate}[(a)]
    \item
      Using the elementary row and column operations\\
      \begin{tabular}{l}
        $R_1 + R_3 \mapsto R_1$\\
        $C_3 - C_1 \mapsto C_3$\\
        $C_3 \leftrightarrow C_1$\\
        $R_3 \leftrightarrow R_2$\\
        $R_1 + R_2 \mapsto R_1$\\
        $C_2 + \frac{x - 1}{3}C_3 \mapsto C_2$\\
        $R_3 + \frac{x + 2}{3}R_2 \mapsto R_3$\\
        $C_2 - C_1 \mapsto C_2$\\
        $3R_3 \mapsto R_3$\\
        $\frac{-1}{3}R_2 \mapsto R_3$\\
        $R_1 \leftrightarrow R_2$\\
        $C_2 \leftrightarrow C_3$\\
        $C_1 \leftrightarrow C_2$
      \end{tabular}\\
      the matrix $xI - A$ diagonalizes to 
      $$\left(
      \begin{array}{ccc}
        1 & 0 & 0\\
        0 & x + 2 & 0\\
        0 & 0 & (x+2)(x-1)
      \end{array}
      \right).$$
      Therefore the invariant factors are $x+2$ and $(x+2)(x-1)$.
    \item
      Using the companion matrices for the invariant factors above, the rational canonical form is 
      $$\left(
      \begin{array}{ccc}
        -2 & 0 & 0\\
        0 & 0 & 2\\
        0 & 1 & -1
      \end{array}
      \right).$$
    \item
      Using the row operations above, the basis is given by 
      $$\{-3e_2 + 3e_3 - 3e_1 + (x + 2)e_1, e_3 - e_1, \frac{1}{3}e_1\}$$
      Using the relation $T(e_1) = e_1  + 3e_2 - 3e_3$, this reduces to 
      $$\{0, e_3 - e_1, \frac{1}{3}e_1\},$$
      corresponding to $1, x + 2, (x + 2)(x - 1)$.
    \item
      Using the non-zero elements above and $\frac{1}{3}T(e_1) = \frac{1}{3}e_1 + e_2 - e_3$ we have
      $$P^{-1}AP = \left(
      \begin{array}{ccc}
        0 & 1 & 1\\
        3 & 2 & 3\\
        0 & 1 & 0
      \end{array}
      \right)
      \left(
      \begin{array}{ccc}
        1 & 3 & 3\\
        3 & 1 & 3\\
        -3 & -3 & -5
      \end{array}
      \right)
      \left(
      \begin{array}{ccc}
        -1 & \frac{1}{3} & \frac{1}{3}\\
        0 &  0 & 1\\
        1 & 0 & -1
      \end{array}
      \right) = 
      \left(
      \begin{array}{ccc}
        -2 & 0 & 0\\
        0 & 0 & 2\\
        0 & 1 & -1
      \end{array}
      \right).$$
    \end{enumerate}
  \end{proof}
\end{thm}  
\end{document}
