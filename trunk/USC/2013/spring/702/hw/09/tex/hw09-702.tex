\documentclass[10pt]{amsart}
\usepackage{amsmath,amsthm,amssymb,amsfonts,enumerate,mymath,mathtools}
\openup 5pt
\author{Blake Farman\\University of South Carolina}
\title{Math 702:\\Homework 09}
\date{April 17, 2013}
\pdfpagewidth 8.5in
\pdfpageheight 11in
\usepackage[margin=1in]{geometry}

\begin{document}
\maketitle

\providecommand{\p}{\mathfrak{p}}
\providecommand{\m}{\mathfrak{m}}

\newtheorem{thm}{}
\newtheorem{lem}{Lemma}

\begin{thm}
  Let $R$ be a ring, let $M$ be a (left) $R$-module, and let $\Tor{R}{M}$ denote the set of torsion elements of $M$.
  \begin{enumerate}[(a)]
  \item
    Suppose that $R$ is an integral domain.
    Prove that $\Tor{R}{M}$ is a submodule of $M$, the torsion submodule of $M$.
  \item
    Suppose that $R$ is commutative.
    Give an example of an $R$-module $M$ in which $\Tor{R}{M}$ is not a submodule of $M$.
  \end{enumerate}
  
  \begin{proof}
    \begin{enumerate}[(a)]
    \item
      Since $0 \in \Tor{R}{M}$, it is non-empty.
      Let $x,y \in \Tor{R}{M}$ and $0 \neq r \in R$ be given.
      There exist elements $r_x, r_y$ of $R$ such that $r_x x = r_y y = 0$.
      Hence, since $R$ is an integral domain,
      $$r_x r_y(x + ry) = r_y(r_x x) + r_x r(r_y y) = 0$$
      with $r_x r_y \neq 0$.
      Therefore $x + ry \in \Tor{R}{M}$ and $\Tor{R}{M}$ is a submodule, as desired.
    \item
      Take the $\Z$-module $\Z/6\Z$.
      The elements 2 and 3 are both torsion elements, $3 \cdot (2) = 0$ and $2 \cdot (3) = 0$.
      However, $5 = 2 + 3 \not \in \Tor{R}{M}$.
    \end{enumerate}
  \end{proof}
\end{thm}

\begin{thm}
  Let $R$ be a ring, let $M$ be a (left) $R$-module, and let $\Ann{R}{N}$ denote the annihilator of $N$ in $R$.
  Prove that $\Ann{R}{N}$ is an ideal (two-sided).
  
  \begin{proof}
    First observe that $0 \in \Ann{R}{N}$, and so it is not empty.
    Let $a_1, a_2 \in \Ann{R}{N}$ be given.
    For any $n \in N$, it follows from
    $$(a_1 + a_2)n = a_1 n + a_2 n = 0$$
    that $a_1 + a_2 \in \Ann{R}{N}$.
    Then for any $r \in R$, by the definition of the $R$-action on $N$, 
    $$(a_1r)n = a_1(rn) = 0\ \text{and}\ (ra_1)n = r(a_1n) = 0.$$
    Therefore $\Ann{R}{N}$ is a two-sided ideal.
  \end{proof}
\end{thm}

\begin{thm}
  Let $R$ be a commutative ring, let $M$, $A$, $B$ be $R$-modules, and let $F$ be a free $R$-module.
  \begin{enumerate}[(a)]
  \item
    Prove the following.
    \begin{enumerate}[(i)]
    \item
      $\Hom{R}{A,B}$ is an $R$-module.
    \item
      $\Hom{R}{R,M} \cong M$ as $R$-modules.
    \item
      Prove that $\End{R}{M} = \Hom{R}{M,M}$ is a ring, the endomorphism ring of $M$.
  \end{enumerate}
\item
  Prove one of the following $R$-module isomorphisms.
  \begin{enumerate}[(i)]
  \item
    $\Hom{R}{A \times B, M} \cong \Hom{R}{A,M} \times \Hom{R}{B,M}$.
  \item
    $\Hom{R}{M, A \times B} \cong \Hom{R}{M,A} \times \Hom{R}{M,B}$.
  \end{enumerate}
\item
  Suppose that $\operatorname{rank}_R(F) = n$ is finite.
  \begin{enumerate}[(i)]
  \item
    Prove that $\Hom{R}{F,R} \cong F$ as $R$-modules.
  \item
    Prove that $\Hom{R}{F,M} \cong M \times \ldots M$ ($n$ times).
  \end{enumerate}
\end{enumerate}

\begin{proof}
  \begin{enumerate}[(a)]
  \item
    \begin{enumerate}[(i)]
    \item
      Let $\varphi, \psi \in \Hom{R}{A,B}$ be given.
      Under pointwise addition, $\Hom{R}{A,B}$ is an abelian group:
      \begin{enumerate}
      \item
	The unique map $0 \colon A \rightarrow 0 \hookrightarrow B$ is the additive identity,
      \item
	$(\varphi + \psi)(a) = \varphi(a) + \psi(a) = \psi(a) + \varphi(a) = (\psi + \varphi)(a)$, for each $a \in A$
      \item
	$(\varphi - \varphi)(a) = \varphi(a) - \varphi(a) = 0$, for each $a \in A$.
      \end{enumerate}
      Define an $R$-action
      \begin{align*}
	R \times \Hom{R}{A,B} &\rightarrow \Hom{R}{A,B}\\
	(r, \varphi) &\mapsto r\varphi.
      \end{align*}
      Since $R$ is a commutative ring, this action is well-defined.  Namely $r\varphi$ is a homomorphism,
      $$r\varphi(a + sb) = r(\varphi(a) + s\varphi(b)) = r\varphi(a) + rs\varphi(b) = (r\varphi)(a) + s(r\varphi)(b).$$
      For any $a \in A$ and $r, s \in R$, using the $R$-action on $B$, this action satisfies the module axioms
      \begin{enumerate}
      \item
	$((r + s)\varphi)(a) = (r + s)\varphi(a) = r\varphi(a) + s\varphi(a) = (r\varphi)(a) + (s\varphi)(a)$,
      \item
	$((rs)\varphi)(a) = (rs)\varphi(a) = r(s\varphi(a)) = r(s\varphi)(a)$,
      \item
	$r(\varphi + \psi)(a) = r(\varphi(a) + \psi(a)) = r\varphi(a) + r\psi(a) = (r\varphi)(a) + (r\psi)(a)$, and
      \item
	$(1\varphi)(a) = 1 \varphi(a) = \varphi(a)$.
      \end{enumerate}
      Therefore $\Hom{R}{A,B}$ is an $R$-module.
    \item
      %Let $\varphi \in \Hom{R}{M}$ be given.
      %Since $R$ is cyclically generated by $1$ as an $R$-module, for $r \in R$, $\varphi(r) = r\varphi(1)$.
      %Hence $\varphi$ is determined by the image of 1.
      For each element $m \in M$, define a map 
      \begin{align*}
        \varphi_m \colon R &\rightarrow M\\
        r &\mapsto rm.
      \end{align*}
      For any $r, s \in R$, we have by the definition of the action of $R$ on $M$ that $\varphi(r) = r\varphi(1) = rm$ and
      $$\varphi_m(r + s) = (r + s)\varphi_m(1) = (r + s)m = rm + sm = \varphi_m(r) + \varphi_m(s).$$
      Hence $\varphi_m$ is a homomorphism and so we may define the surjective map 
      \begin{align*}
        \psi \colon \Hom{R}{R,M} &\twoheadrightarrow M\\
        \varphi &\mapsto \varphi(1).
      \end{align*}
      To see that $\psi$ is a homomorphism, observe that 
      $$\psi(\varphi_1 + r\varphi_2) = (\varphi_1 + r\varphi_2)(1) = \varphi_1(1) + (r\varphi_2)(1) = \varphi_1(1) + r\varphi_2(1) = \psi(\varphi_1) + r\psi(\varphi_2).$$
      Finally, note that if $\psi(\varphi) = 0$, then $\varphi(1) = 0$ implies $\varphi = 0$.
      Therefore $\ker\psi = 0$ and $\Hom{R}{R,M} \cong M$ by the First Isomorphism Theorem.
    \item
      Let $\varphi, \psi, \sigma \in \End{R}{M}$, $m, n \in M$, and $r \in R$ be given.
    Associativity and commutativity of pointwise addition are inherited from $M$.
    Since $\varphi, \psi$ are homomorphisms, closure under pointwise addition follows from 
    $$(\varphi + \psi)(m + n) = \varphi(m + n) + \psi(m + n) = \varphi(m) + \varphi(n) + \psi(m) + \psi(n) = (\varphi + \psi)(m) + (\varphi + \psi)(n)$$
    and 
    $$(\varphi + \psi)(rm) = \varphi(rm) + \psi(rm) = r\varphi(m) + r\psi(m) = r(\varphi(m) + \psi(m))= r(\varphi + \psi)(m).$$
    The trivial homomorphism,
    \begin{align*}
      0 \colon M & \rightarrow M\\
      m &\mapsto 0,
    \end{align*}
    provides an additive identity and the additive inverse of $\varphi$ is given by $-\varphi$; namely for all $m \in M$
    $$(\varphi + 0)(m) = \varphi(m) + 0(m) = \varphi(m)\ \text{and}\ (\varphi + -\varphi)(m) = \varphi(m) - \varphi(m) = 0.$$

    Since composition is associative in general, it remains to show closure under $\circ$ and distribution.
    Closure under $\circ$ follows from
    $$(\varphi \circ \psi)(m + n) = \varphi(\psi(m) + \psi(n)) = (\varphi\circ\psi)(m) + (\varphi\circ\psi)(n)$$
    and
    $$(\varphi \circ \psi)(rm) = \varphi(r\psi(m)) = r\varphi(\psi(m)) = r(\varphi\circ\psi)(m).$$
    Finally, distribution follows from 
    $$(\sigma \circ (\varphi + \psi))(m) = \sigma((\varphi+ \psi)(m)) = \sigma(\varphi(m) + \psi(m)) = \sigma(\varphi(m)) + \sigma(\psi(m)) = (\sigma\circ)\varphi(m) + (\sigma\circ\psi)(m).$$
    By defining $1_{\End{R}{M}} = \operatorname{id}_M$, it is clear that this is a unital ring.
\end{enumerate}
\item
\begin{enumerate}[(ii)]
\item
  Observe that $A \times B$ comes equipped with projections 
  $$\pi_A \colon A \times B \twoheadrightarrow A\ \text{and}\ \pi_B \colon A \times B \twoheadrightarrow B.$$
  Define the map
  \begin{align*}
    \psi_1 \colon \Hom{R}{M, A \times B} &\rightarrow \Hom{R}{M, A} \times \Hom{R}{M,B}\\
    \varphi &\mapsto (\pi_A \circ \varphi, \pi_B \circ \varphi).
  \end{align*}
  Let $\phi_1, \phi_2 \in \Hom{R}{M, A \times B}$, $r \in R$.
  Then 
  \begin{eqnarray*}
    \psi_1(\phi_1 + r\phi_2) &=& (\pi_A \circ (\phi_1 + r\phi_2), \pi_B \circ (\phi_1 + r\phi_2))\\
    &=& (\pi_A \circ \phi_1, \pi_B \circ \phi_1) + (r \pi_A \circ \phi_2, r \pi_B \circ \phi_2))
  \end{eqnarray*}
  follows from 
  \begin{eqnarray*}
    \pi_A \circ (\phi_1 + r\phi_2)(m) &=& \pi_A(\phi_1(m) + r\phi_2(m)) = \pi_A(\phi_1(m)) + r\pi_A(\phi_2(m))\\
    &=& (\pi_A\circ \phi_1 + r\pi_A \circ \phi_2)(m)
  \end{eqnarray*}and the same computation, mutatis mutandis, for $\pi_B$.
  Similarly, define the homomorphism
  \begin{align*}
    \psi_2 \colon \Hom{R}{M, A} \times \Hom{R}{M,B} &\rightarrow \Hom{R}{M, A \times B}\\
    (\varphi_1, \varphi_2) &\mapsto \{\varphi \colon m \mapsto (\varphi_1(m), \varphi_2(m))\}.
  \end{align*}
  Then from
  $$\varphi \xmapsto{\psi_1} (\pi_A \circ \varphi, \pi_B \circ \varphi) \xmapsto{\psi_2} \{m \mapsto (\pi_A \circ \varphi(m), \pi_B \circ \varphi(m)\} = \varphi$$
  and
  $$(\varphi_1, \varphi_2) \xmapsto{\psi_2} \{\varphi \colon m \mapsto (\varphi_1(m), \varphi_2(m))\} \xmapsto{\psi_1} (\pi_A \circ \varphi, \pi_B \circ \varphi) = (\varphi_1, \varphi_2)$$
  it is clear that $\psi_1 \circ \psi_2 = 1$ and $\psi_2 \circ \psi_1 = 1$.
  Therefore $\Hom{R}{M, A \times B} \cong \Hom{R}{M, A} \times \Hom{R}{M,B}$.
\end{enumerate}
\item
  \begin{enumerate}[(i)]
  \item
    Since $F \cong R^n$, it suffices to show that $\Hom{R}{R^n, R} \cong R^n$.
    Inducting on $n$, observe that when $n = 1$, it follows from part (ii) of 3 (a) that $\Hom{R}{R, R} \cong R$.
    Assume the hypothesis holds up to $n$.
    Then by part (i) of 3 (b) and the induction hypothesis
    $$\Hom{R}{R^n, R} = \Hom{R}{R^{n-1} \times R, R} \cong \Hom{R}{R^{n-1}, R} \times \Hom{R}{R, R} \cong R^{n-1} \times R = R^n.$$
  \item
    By the same induction argument, $\Hom{R}{R,M} \cong M$ and so 
    $$\Hom{R}{R^n,M} \cong \Hom{R}{R^{n-1}, M} \times \Hom{R}{R,M} = \underbrace{(M \times \ldots M)}_{n-1} \times M = \underbrace{M \times \ldots \times M}_n.$$
  \end{enumerate}
\end{enumerate}
\end{proof}

\end{thm}

\begin{thm}[Schur's Lemma]
  let $R$ be a ring.
  An $R$-module $M$ is {\it simple} if and only if it is nonzero and its only submodules are $\left\{0\right\}$ and $M$.
  Suppose that $M$ is a simple $R$-module.
  Prove that $\End{R}{M}$ is a division ring.

\begin{proof}
  Let $\varphi \in \End{R}{M}$ be given.
Observe that $\varphi(0) = \varphi(0 + 0) = \varphi(0) + \varphi(0)$ implies by cancellation that $\varphi(0) = 0 \in \varphi(M)$.
For any two elements $a, b \in M$ and any ring element $r$,
$$\varphi(a) + r\varphi(b) = \varphi(a) + \varphi(rb) = \varphi(a + rb) \in \varphi(M),$$
and so $\varphi(M)$ is a submodule of $M$.
Hence $\varphi$ is either the zero morphism, or an automorphism.
Since by part iii of 3 (a) $\End{R}{M}$ is a ring, it follows that $\End{R}{M}^\times = \operatorname{Aut}_R(M)$.
		Therefore $\End{R}{M}$ is a (not necessarily commutative) division ring.
	\end{proof}
\end{thm}  
\end{document}
