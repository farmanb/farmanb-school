\documentclass[10pt]{amsart}
\usepackage{amsmath,amsthm,amssymb,amsfonts,enumerate,mymath,tikz-cd,pbox,mathtools}
\openup 5pt
\author{Blake Farman\\University of South Carolina}
\title{Math 788p:\\Homework 05}
\date{March 22, 2013}
\pdfpagewidth 8.5in
\pdfpageheight 11in
\usepackage[margin=1in]{geometry}

\begin{document}
\maketitle

\providecommand{\p}{\mathfrak{p}}
\providecommand{\m}{\mathfrak{m}}
\newcommand{\legendre}[2]{\left(\frac{#1}{#2}\right)}
\theoremstyle{plain}
\newtheorem{thm}{}
\newtheorem{lem}{Lemma}
\theoremstyle{definition}
\newtheorem{defn}{Definition}


\setcounter{thm}{2}
\begin{thm}
  Compute the class group of $K = \Q(\sqrt{-33})$.
  
  \begin{proof}
    Let $\alpha = i \sqrt{33}$.
    Since $\mathcal{O}_K = \Z[\alpha]$, we have
    $$\disc{K} = \operatorname{det}\left(\begin{array}{cc}
      \Tr{K/\Q}{1} & \Tr{K/\Q}{\alpha}\\
      \Tr{K/\Q}{\alpha} & \Tr{K/\Q}{\alpha^2}
    \end{array}
    \right)
    = 
    \left(\begin{array}{cc}
      2 & 0\\
      0 & -66
      \end{array}
    \right) = -132.$$
    Then $B_K = \frac{2}{\pi} \sqrt{132} \in (7, 8)$.
    Hence it suffices to check ideals of norm at most 7.
    Factoring $x^2 + 33$ modulo the primes 2, 3, 5, and 7 we have $(2) = (2, \alpha + 1)^2 = \mathfrak{p}_2^2$, $(3) = (3, \alpha)^2 = \mathfrak{p}_3^2$, $(5)$ is inert, and $(7) = (7, \alpha + 3)(7, \alpha -3) = \mathfrak{p}_7\mathfrak{p}_7^\prime$.
    Then $\p_2 \not \sim \p_3$, for otherwise $\p_2\p_3^{-1} = \p_2\p_3$ has norm 6, and thus cannot be principle.
    Similarly, $\p_2\p_3 \not \sim \p_2$ since $\p_3\p_2\p_2^{-1} = \p_3$ and $\p_2\p_3 \not \sim \p_3$.
    Hence $\{1, \p_2, \p_3, \p_2\p_3\} \leq \operatorname{Cl}(K)$ is a subgroup isomorphic to the Klein 4-group.
    Since $2 \mid \abs{\operatorname{Cl}(K)}$, the only candidates for the class group are $\Z/(6)$, $\Z/(4)$.
    Since neither contain subgroups isomorphic to $V_4$, $\operatorname{Cl}(K) \cong V_4$.
  \end{proof}
\end{thm}

\begin{thm}
  Compute the class group of $K = \Q(\sqrt{-163})$.
  
  \begin{proof}
    Let $\alpha = i\sqrt{163}$ and let $\omega = \frac{1 + \alpha}{2}$.
    Since $\mathcal{O}_K = \Z[\omega]$ and 
    $$\omega^2 = \frac{\alpha^2 + 2\alpha + 1}{4} = \frac{2\alpha - 163}{4} = \frac{\alpha - 81}{2} = \frac{\alpha+ 1 - 82}{2} = \omega - 41,$$
    it follows that
    $$\disc{K} = \operatorname{det}\left(\begin{array}{cc}
      \Tr{K/\Q}{1} & \Tr{K/\Q}{\omega}\\
      \Tr{K/\Q}{\omega} & \Tr{K/\Q}{\omega^2}
    \end{array}
    \right) =
    \left(\begin{array}{cc}
      2 & 1\\
      1 & -81
      \end{array}
    \right) = -163$$
    Then $B_K = \frac{2}{\pi}\sqrt{163} \in (8,9)$.
    Hence it suffices to check ideals of norm at most 8.
    Since $x^2 - x + 41$ is irreducible modulo 2, 3, 5, and 7, each prime remains inert and so there are no ideals (other than $(1)$) of norm at most 8.
    Therefore the class group is trivial and $\mathcal{O}_K$ is a P.I.D.
  \end{proof}
\end{thm}

\begin{thm}
  Compute the class group of $K = \Q(\sqrt{-14})$.
  
  \begin{proof}
    Let $\alpha = i\sqrt{14}$.
    Since $\mathcal{O_K} = \Z[\alpha]$ we have
    $$
    \disc{K} = \operatorname{det}\left(\begin{array}{cc}
      \Tr{K/\Q}{1} & \Tr{K/\Q}{\alpha}\\
      \Tr{K/\Q}{\alpha} & \Tr{K/\Q}{\alpha^2}
    \end{array}
    \right) = 
    \left(\begin{array}{cc}
      2 & 0\\
      0 & -28
      \end{array}
    \right) =
    -56.
    $$
    Then $B_K = \frac{2}{\pi}\sqrt{56} \in (4,5)$.
    Hence it suffices to check ideals of norm at most 4.
    Factoring $x^2 + 14$ modulo 2 and 3 we have $(2) = (2, \alpha)^2 = \p_2^2$ and $(3) = (3, \alpha + 1)(3, \alpha - 1) = \p_3\p_3^\prime$.

    Consider the ideal $(2 + \alpha)$ of norm $18 = 2\cdot 3^2$.
    Observe that $3 \nmid 2 + \alpha$ and $2 \nmid 2 + \alpha$, but clearly $(2 + \alpha) \subseteq \p_2$.
    Moreover, $2 + \alpha = 3 + (\alpha - 1) \in \p_3^\prime$, and so it follows that $(2 + \alpha) = \p_2(\p_3^\prime)^2$, which in turn implies $\p_2 \sim \p_3^2$.
    Then $\p_3^3 \sim \p_3^\prime$ since 
    $$p_3^3(\p_3^\prime)^{-1} = \p_3^4 \sim (\p_2)^2 \sim 1.$$
    Therefore $\operatorname{Cl}_K$ is generated by $\p_3$, which has order four, and is thus isomorphic to $\Z/(4)$.
  \end{proof}
\end{thm}
\end{document}
