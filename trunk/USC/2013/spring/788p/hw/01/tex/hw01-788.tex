\documentclass[10pt]{amsart}
\usepackage{amsmath,amsthm,amssymb,amsfonts,enumerate}
\openup 5pt
\author{Blake Farman\\University of South Carolina}
\title{Math 788p:\\Homework 01}
\date{January 25, 2013}
\pdfpagewidth 8.5in
\pdfpageheight 11in
\usepackage[margin=1in]{geometry}

%Field names
\newcommand{\Z}{\mathbb{Z}}
\newcommand{\R}{\mathbb{R}}
\newcommand{\Q}{\mathbb{Q}}
\newcommand{\C}{\mathbb{C}}
\newcommand{\F}{\mathbb{F}}
\newcommand{\N}{\mathbb{N}}
\newcommand{\uhp}{\mathfrak{h}}

%Operator names
\newcommand{\ord}{\operatorname{ord}}
\newcommand{\Det}{\operatorname{Det}}
\newcommand{\Gal}{\operatorname{Gal}}
\newcommand{\Inn}[1]{\operatorname{Inn}\left(#1\right)}
\newcommand{\Aut}[1]{\operatorname{Aut}\left(#1\right)}
\newcommand{\real}[1]{\operatorname{\mathfrak{Re}}\left(#1\right)}
\newcommand{\imag}[1]{\operatorname{\mathfrak{Im}}\left(#1\right)}
\newcommand{\Syl}[2]{\operatorname{Syl}_{#1}\left(#2\right)}
\newcommand{\SL}[2]{\operatorname{SL}_#1\left(#2\right)}
\newcommand{\GL}[2]{\operatorname{GL}_#1\left(#2\right)}
\newcommand{\M}[2]{\operatorname{M}_#1\left(#2\right)}
\newcommand{\PSL}[2]{\operatorname{PSL}_#1\left(#2\right)}
\newcommand{\Mat}[2]{\operatorname{Mat}_{#1}\left(#2\right)}
\providecommand{\norm}[1]{\lVert#1\rVert}
\newcommand{\dist}[2]{\operatorname{dist}\left(#1,#2\right)}
\newcommand{\cntr}[1]{\mathbf{Z}\left(#1\right)}
\newcommand{\abs}[1]{\left| #1 \right|}
\newcommand{\orbit}[1]{\mathcal{O}_{#1}}
\newcommand{\card}[1]{\operatorname{card}#1}
\newcommand{\Log}[1]{\operatorname{Log}\left(#1\right)}
\newcommand{\Arg}[1]{\operatorname{Arg}\left(#1\right)}

\renewcommand{\qedsymbol}{\(\blacksquare\)}
\renewcommand{\epsilon}{\varepsilon}

\begin{document}
\maketitle

\providecommand{\p}{\mathfrak{p}}
\providecommand{\m}{\mathfrak{m}}

\newtheorem{thm}{}
\newtheorem{lem}{Lemma}

\begin{thm}
  Prove that for any prime $p$ other than 2, and any $x \in \Z/p\Z$, the equation $x = a^2 + b^2$ can be solved in $\Z/p\Z$.
  \begin{proof}
    Let $p$ be an odd prime.
    Since the multiplicative group of units, $\left(\Z/p\Z\right)^\times = \Z/p\Z \setminus \left\{0\right\}$, is cyclic of order $p-1$ there exists some element $g$ such that $\left(\Z/p\Z\right)^\times = \left<g\right>.$
    Define a subset of distinct squares $$S = \left\{(g^\alpha)^2 \;\middle\vert\; 1 \leq \alpha \leq (p-1)/2 \right\} \cup \left\{0\right\} \subseteq \Z/p\Z.$$
    Note that by construction this set has $(p-1)/2 + 1 = (p + 1)/2$ elements.
    %Consider $G = \Z/p\Z$ as an additive group and 
    %Define the set of additive inverses of $S$, 
    %$$S^{-1} = \left\{-g^{2\alpha} \;\middle\vert\; 1 \leq \alpha \leq (p-1)/2\right\} \cup \left\{0\right\}.$$
    
    Let $x \in \Z/p\Z$ be given.
    It suffices to show that there exist elements $s$ and $s^\prime$ of $S$ such that $x = s + s^\prime$.
    Consider the set
    $$x - S = \left\{x - s \;\middle\vert\; s \in S\right\} \subseteq \Z/p\Z.$$
    For any two elements $s, s^\prime \in S$, by left (additive) cancellation and the uniqueness of additive inverses, $x - s = x - s^\prime$ if and only if $s = s^\prime$.
    Hence the set map $S \rightarrow x - S$ given by $s \mapsto x - s$ is a bijection.
    %Thus $\abs{S} = \abs{h - S} = (p + 1) / 2$.
    It now follows that these two sets must have a non-empty intersection; for otherwise we would have $(p + 1) / 2$ distinct elements of $\Z/p\Z$ in each set, and hence $p + 1$ distinct elements in a field of order $p$.
    Now observe that for any element $h$ of $S \cap x - S$ we may write $$h = s = x - s^\prime,$$ for some $s,s^\prime \in S$. 
    Therefore $x = s + s^\prime$, as desired.
  \end{proof}
\end{thm}

\begin{lem}\footnote{This might all be overkill, but I wasn't quite sure what was quotable.}\label{wilson}
For any rational prime, $p$, $(p-1)! \equiv -1 \pmod{p}$.
\begin{proof}
Consider the difference of the elements $g(x) = \prod_{a \in \F_p}(x - a)$ and $h(x) = x^{p-1} - 1$ of $A = \F_p[x]$.
Since $g$ and $h$ are both monic of degree $p-1$, we have that $\deg{g - h} = p - 2$.
It now follows from Lagrange's Theorem that $h(a) \equiv 0 \pmod{p}$ for each $a \in \F_p$.
Thus $(g-h)(a) \equiv 0 \pmod{p}$ for each $a \in \F_p$.
Since $g-h$ is a degree $p-2$ polynomial with $p-1$ roots, it follows that $g-h = 0_A$.
Now observe that, by construction, the constant coefficient of $g$ is $(p-1)!$ and thus it follows from $g - h = 0_A$ that $(p-1)! + 1 = 0 \pmod{p}$.
Therefore $(p-1)! \equiv -1 \pmod{p}$.
\end{proof}
\end{lem}

\begin{lem}\label{square}
If $p$ is a rational prime congruent to $1$ modulo $4$, then there exists some integer $x$ such that $x^2 = -1 \pmod{p}$.
\end{lem}
\begin{proof}
By Lemma~\ref{wilson} we have $(p-1)! \equiv -1 \pmod{p}$.
Write $p = 4n + 1$ for some $n \in \Z$ and then we have 
$$(p-1)! = 1 \cdot 2 \cdot \ldots \cdot (2n) \cdot (p - (2n)) \cdot \ldots \cdot (p - 1) = (2n)!(-1)^{\frac{p-1}{2}}(2n!) \equiv -1 \pmod{p}.$$
Since $(p-1)/2 = 2n$, it follows that $(2n!)^2 \equiv -1 \pmod{p}$, as desired.
\end{proof}

\begin{lem}
A rational prime, $p$, is a sum of two squares if and only if $p \equiv 1 \pmod{4}$.
\begin{proof}
Assume $p$ is the sum of two squares.
Then since the squares are 0 and 1 modulo 4, it follows directly that $p \not \equiv 3 \pmod{4}$.
Hence $p \equiv 1 \pmod{4}$.

Conversely, assume $p \equiv 1 \pmod{4}$.
Consider the ring $A = \F_p[x]/(x^2 + 1)$.
By Lemma~\ref{square}, there  exists some element of $\F_p$ such that $x^2 + 1 = (x + a)(x - a)$.
Since $\F_p[x]$ is a principal ideal domain, it follows that $(x^2 + 1)$ is not prime and thus $A$ is not an integral domain.
It now follows from the isomorphism $\Z[i]/(p) \cong \F_p[x]/(x^2 + 1)$ that $p$ is not prime in $\Z[i]$.
Hence there exists a non-trivial factorization $p = \alpha\beta$ for some $\alpha, \beta \in \Z[i]$.
Taking norms of both sides, it follows that $p = N(\alpha) = a^2 + b^2$ for some $a,b \in \Z$.
\end{proof}
\end{lem}
\end{document}
