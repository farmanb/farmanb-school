\documentclass[10pt]{amsart}
\usepackage{amsmath,amsthm,amssymb,amsfonts,enumerate,mymath,tikz-cd,pbox,mathtools}
\openup 5pt
\author{Blake Farman\\University of South Carolina}
\title{Math 788p:\\Homework 07}
\date{May 8, 2013}
\pdfpagewidth 8.5in
\pdfpageheight 11in
\usepackage[margin=1in]{geometry}

\begin{document}
\maketitle

\providecommand{\p}{\mathfrak{p}}
\providecommand{\m}{\mathfrak{m}}
\newcommand{\legendre}[2]{\left(\frac{#1}{#2}\right)}
\theoremstyle{plain}
\newtheorem{thm}{}
\newtheorem{lem}{Lemma}
\theoremstyle{definition}
\newtheorem{defn}{Definition}
\newtheorem{prop}{Proposition}
\newtheorem{cor}{Corollary}

\setcounter{thm}{6}

\begin{thm}
  Describe, in around two to three pages, something interesting you have learned this semester, about number theory or a related topic, outside of this course and your other coursework.
\end{thm}

The bulk of my extra-curricular learning this semester has been devoted in large part to developing the basic machinery for algebraic number theory, particularly category theory.  
One of the interesting results is Yoneda's Lemma, which is a roughly similar result to the embedding of a module in its ring of endomorphisms.  
Namely, a category, $\mathcal{C}$, can be identified as a subcategory of the category of presheaves (i.e. the category of covariant functors $\mathcal{C}^\text{op} \rightarrow {\bf Set}$), $[\mathcal{C}^\text{op}, {\bf Set}]$.

The proof of Yoneda's Lemma follows quickly from a proposition, which requires the following
\begin{defn}{Univseral Arrow}
  If $S: \mathcal{D} \rightarrow \mathcal{C}$ is a functor and $c$ an object of $\mathcal{C}$, a universal arrow from $c$ to $S$ is a pair $\left<r, u\right>$ consisting of an object $r$ of $\mathcal{D}$ and an arrow $u \colon c \rightarrow Sr$ of $C$, such that every pair $\left<d,f\right>$ with $d$ an object of $\mathcal{D}$ and $f \colon c \rightarrow Sd$, there is a unique arrow $f^\prime \colon r \rightarrow d$ of $\mathcal{D}$ such that the diagram
  \begin{center}
    \begin{tikzcd}
      c \arrow[equals]{d}\arrow{r}{u} & Sr \arrow[dashed]{d}{Sf^\prime} & r \arrow[dashed]{d}{\exists! f^\prime}\\
      c \arrow{r} & Sd & d
    \end{tikzcd}
  \end{center}
  commutes.
\end{defn}

\begin{prop}
  For a functor $S \colon \mathcal{D} \rightarrow \mathcal{C}$ a pair $\left<r, u \colon c \rightarrow Sr\right>$ is a universal from $c$ to $S$ if and only if the functor sending each $f^\prime \colon r \rightarrow d$ into $Sf^\prime \circ u \colon c \rightarrow Sd$ is a bijection of hom-sets
  \begin{equation}\label{eq1}
    \mathcal{D}(r,d) \cong \mathcal{C}(S,d),
  \end{equation}
  where $D$ and $C$ are the usual covariant hom-functors.
  This bijection is natural in $d$.
  Conversely, any natural isomorphism \eqref{eq1} is determined in this way by a unique arrow $u \colon c \rightarrow Sr$ such that $\left<r,u\right>$ is universal from $c$ to $S$.
  
  \begin{proof}
    The bijection follows directly from the definition of the universal.
    Namely, for every element $f \colon c \rightarrow Sd$, there is a unique arrow $f^\prime r \rightarrow d$ with $f = Sf^\prime \circ u$.
    Naturality is just the statement $S(g^\prime \circ f^\prime) \circ u = Sg^\prime \circ Sf^\prime \circ u$, which follows directly from functoriality of $S$ and associativity of composition.

    For the converse, let $\varphi \colon \mathcal{D}(r,\_) \rightarrow \mathcal{C}(c, {S\_})$ be a natural transformation and let $u \colon c \rightarrow Sr$ to be the image of the identity on $r$ under the component $\varphi_r$.
    For every morphism $f^\prime \colon r \rightarrow d$, we have a morphism
    \begin{align*}
      \mathcal{D}(r,r) &\rightarrow \mathcal{D}(r,d)\\
      g^\prime &\mapsto f^\prime \circ g^\prime.
    \end{align*}
    In particular, when $g^\prime = 1_r$, we have by naturality $Sf^\prime \circ \varphi_r(1_r) = Sf^\prime \circ u = \varphi_d(f^\prime \circ 1_r) = \varphi_d(f^\prime)$, which is precisely the statement that $\left<r,u\right>$ is universal.
  \end{proof}
\end{prop}
  This gives a natural isomorphism between the functors $\mathcal{D}(r,\_) \cong \mathcal{C}(c, S\_)$ and gives rise to the 
  \begin{defn}
    Let $\mathcal{D}$ have small hom-sets.
    A representation of a functor $K \colon \mathcal{D} \rightarrow {\bf Set}$ is a pair $\left<r, \psi\right>$, $r$ an object of $\mathcal{D}$, and
    $$\psi \colon \mathcal{D}(r, \_) \cong K$$
    a natural isomorphism.
    The object $r$ is called the representing object.
    The functor $K$ is said to be representable when such a representation exists.
  \end{defn}
  With this in hand, the Yoneda Lemma is then
  \begin{lem}
    If $K \colon \mathcal{D} \rightarrow {\bf Set}$ and $r$ is an object of $\mathcal{D}$ having small hom-sets, there is a bijection
    $$y \colon \operatorname{Nat}(D(r, \_), K) \cong Kr,$$
    which sends each natural transformation $\alpha \colon D(r, \_) \rightarrow K$ to $\alpha_r(1_r)$, the image of the identity $1_r$ on $r$, to an element of $Kr$.
    The map $y$ is natural in $K$ and $r$.
    \begin{proof}
      The proof follows from Proposition 1 and the diagram
      \begin{center}
        \begin{tikzcd}
          \mathcal{D}(r, r) \arrow{d}[left]{f \circ \_}\arrow{r}{\alpha_r} & Kr \arrow{d}{Kf} & r \arrow{d}{f}\\
          \mathcal{D}(r, d) \arrow{r}{\alpha_d} & Kd & d.
        \end{tikzcd}
      \end{center}
    \end{proof}
  \end{lem}
  
  Using Yoneda and Proposition 1, we have the 
  \begin{cor}
    For objects $r, s \in \mathcal{D}$, each natural transformation $\mathcal{D}(r, \_) \rightarrow \mathcal{D}(S, _)$ has the form $\mathcal{D}(h, \_)$ for a unique arrow $h \colon s \rightarrow r$. 
  \end{cor}
  With this corollary, taking as object function $r \mapsto \mathcal{D}(r, \_)$ and arrow function $(f \colon s \rightarrow r) \mapsto  
\end{document}
