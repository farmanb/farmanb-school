\documentclass[12pt]{amsart}
\usepackage{amsmath,amsthm,amssymb,amsfonts,enumerate,mymath}
\openup 5pt
\author{Blake Farman\\University of South Carolina}
\title{Math 704:\\Exam 01}
\date{March 21, 2013}
\pdfpagewidth 8.5in
\pdfpageheight 11in
\usepackage[margin=1in]{geometry}

\begin{document}
\maketitle

\providecommand{\p}{\mathfrak{p}}
\providecommand{\m}{\mathfrak{m}}

\newtheorem{thm}{}
\newtheorem{lem}{Lemma}

\setcounter{thm}{5}

\begin{thm}
  Prove that if $E$ is a measurable subset of $[0,1]$ of Lebesgue measure zero, then the complement of $E$ is dense in $[0,1]$.

  \begin{proof}
    Suppose to the contrary that the complement of $E$ is not dense in $[0,1]$.
    That is, there exists some non-empty open set $U \subseteq [0,1]$ such that $U \cap E^\text{c} = U \setminus E = \emptyset$.
    Then $U \subseteq E$ and so $\mu(U) \leq \mu(E) = 0$ implies $\mu(U) = 0$.
    Since $U$ was supposed to be an non-trivial open subset of $[0,1]$, it may be written as the union of disjoint intervals, at least one of which is non-empty.
    Hence $\mu(U) > 0$, a contradiction.
    Therefore $E^\text{c}$ is dense in $[0,1]$.
  \end{proof}
\end{thm}
\newpage

\setcounter{thm}{7}

\begin{thm}
  If $f$ is a measurable function, prove that $f^{-1}(E)$ is measurable whenever $E$ is a Borel subset of the real line.
  
  \begin{proof}
    Since $f$ is measurable, the pre-image of bounded open intervals are measurable sets by Proposition 5.6.
    Then since the measurable sets form a $\sigma$-algebra, it follows that the countable unions and complements of measurable sets are again measurable.
    Observe that the Borel sets are generated by countable unions and complements of these bounded open intervals.
    Hence the result follows from the following properties of functions: given a collection $\mathcal{S}$
    \begin{enumerate}[(i)]
      \item
        $f^{-1}\left(\bigcup_{S \in \mathcal{S}} S\right) = \bigcup_{S \in \mathcal{S}} f^{-1}(S),$
      \item
        $f^{-1}\left(\bigcap_{S \in \mathcal{S}} S\right) = \bigcap_{S \in \mathcal{S}} f^{-1}(S),$  
      \item
        $f^{-1}(S^\text{c}) = (f^{-1}(S))^\text{c}.$
    \end{enumerate}
  \end{proof}
\end{thm}
\newpage

\setcounter{thm}{9}

\begin{thm}
  Suppose on a finite measure space that $\left\{f_n\right\}_{n = 1}^\infty$ is a sequence of real-valued, integrable functions tending to $f$ uniformly.
  Prove that 
  $$\lim_{n \rightarrow \infty} \int_X f_n d\mu = \int_X f d\mu.$$

  \begin{proof}
    Let $\varepsilon > 0$ be given.
    Choose $N \in \N$ such that 
    $$\abs{f_n(x) - f(x)} < \varepsilon/\mu(X)$$
    holds for all $x \in X$ whenever $n \geq N$.
    Then by linearity of the integral and Corollary 5.21 we have
    \begin{eqnarray*}
      \abs{\int_X (f_n - f) d\mu} & = & \abs{\int_X f_n - \int_X f d\mu}\\
      & \leq & \int_X \abs{f_n - f} d\mu\\
      & < & \frac{\varepsilon}{\mu(X)} \int_X \chi_X d\mu\\
      & = & \frac{\varepsilon}{\mu(X)} \mu(X) = \varepsilon
    \end{eqnarray*}
    whenever $n \geq N$, as desired.
  \end{proof}
\end{thm}
\newpage

\setcounter{thm}{16}

\begin{thm}
  Let $(X, \mathcal{A}, \mu)$ be a finite measure space.
  Suppose that $f_n$ and $f$ are measurable functions with values in $R$ such that $\lim_n f_n = f$ pointwise.
  Consider the sets
  $$E_{MN} = \left\{x \in X \;\middle\vert\; \abs{f_n(x) - f(x)} < \frac{1}{M}\ \text{for}\ n \geq N\right\}$$
  for $M$ fixed and $N$ varying, prove that if $\varepsilon > 0$ is given, then there exists a measurable subset $E$ of $X$ with $\mu(E) < \varepsilon$ such that $\lim_n f_n = f$ uniformly on $E^c$.
  \begin{proof}
    Fix $M \in \N$ and observe that for each $N \in \N$.
    Let $$U = \bigcup_{x \in \R} (f(x) - \frac{1}{M}, f(x) + \frac{1}{M})$$
    and observe that as the union of open sets is itself open.
    Hence for each $n \in \N$, $f_n^{-1}(U)$ is measurable.
    Writing $E_{MN}$ as the countable intersection of measurable sets,
    $$E_{MN} = \bigcap_{n \geq N} f_n^{-1}(U),$$
    it is clear that each $E_{MN}$ is measurable.
    For each $N \in \N$ it is clear from the definition of pointwise convergence that if $x \in E_{MN}$, then $x \in E_{M(N+1)}$ and so $E_{MN} \subseteq E_{M(N+1)}$.
    Moreover, since $f_n \rightarrow f$ pointwise everywhere, it follows that $E_{MN} \uparrow_N X$ and so $\mu(E_{MN}) \uparrow_N \mu(X)$, which is finite by assumption.
    
    Let $\varepsilon > 0$ be given.
    Since $\mu(E_{MN}) \uparrow_N \mu(X) < \infty$, it follows that for each $M$ there exists an $N_M \in \N$ such that 
    $$\mu(E_{MN_M}^c) = \mu(X \setminus E_{MN_M}) = \mu(X) - \mu(E_{MN_M}) < 2^{-M}\varepsilon.$$
    Let 
    $$E = \bigcup_{M \in \N} E_{MN_M}^c = \bigcup_{M \in \N} \left\{x \in X \;\middle\vert\; \abs{f_n(x) - f(x)} \geq \frac{1}{M}\ \text{for some}\ n \geq N_M \right\}$$
    and then note that
    $$\mu(E) \leq \sum_{M \in \N} \mu(E_{MN_M}^c) < \sum_{M \in \N} 2^{-M}\varepsilon = \varepsilon.$$
    %Since $\varepsilon$ was arbitrary, we have $\mu(E) = 0$ and so i
    It remains only to show that $f_n \rightarrow f$ uniformly on $E^c$.
    Observe that if $x \in E^c$, then $x \in E_{MN_M}$ for each $M \in \N$ by DeMorgan's Laws.
    Fix an $M > 1/\varepsilon$.
    If $n \geq N_M$, then
    $$\abs{f_n(x) - f(x)} < 1/M < \varepsilon$$
    holds for all $x \in E^c$.
    Therefore $f_n \rightarrow f$ uniformly on $E^\text{c}$, as desired.
  \end{proof}
\end{thm}
\newpage
  
\end{document}
