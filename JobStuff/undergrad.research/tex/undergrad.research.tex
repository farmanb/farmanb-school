\documentclass[12pt]{amsart}
%\renewcommand{\familydefault}{\rmdefault}
\openup 5pt
\usepackage[margin=1in]{geometry}

\usepackage{tcolorbox}

\newtheorem{theorem}{Theorem}[section]
\newtheorem{objective}{Objective}
\newtheorem*{theorem*}{Theorem}
\theoremstyle{definition}
\newtheorem*{question*}{Question}
\title{Undergraduate Research Plans}
\author{Blake A. Farman}
\begin{document}
\maketitle

Given the breadth of pre-requisite knowledge for algebraic geometry, it is likely unreasonable to expect undergraduate students to participate fruitfully in work on the author's research program.
However, the author is eager to conduct related research that is accessible to undergraduate students.
While the author would be more than willing to work with students in any area of algebra in identifying and carrying out research, of particular interest to the author is the marriage of his background in computer science to the study of geometry, algebra, and number theory.

Depending upon student interest, the author would enjoy offering a course or independent study following an undergraduate text such as \cite{algos}.
While such a course would be interesting in its own right as a concrete introduction to the study of algebraic and arithmetic geometry through computation, it would mainly serve as preparation for further study of the subject through student research.
Students would gain valuable programming skills through problem solving with computer algebra systems such as SageMath or Mathematica, and a number of suggested projects from such a text could serve as the basis for a thesis, while those whose interests range beyond solving problems would also be equipped with the tools to undertake a computational research project within algebra, geometry, or number theory.
The author would be delighted to have the opportunity to supervise either type of project.

\bibliographystyle{amsalpha}
\bibliography{biblio}
\end{document}
