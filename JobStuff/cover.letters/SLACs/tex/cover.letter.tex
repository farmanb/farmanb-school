\documentclass[10pt]{letter}

\usepackage[margin=1in]{geometry}
\usepackage{amsmath}

\signature{Blake Farman}
\address{
  306 Cattell St\\
  Floor 2\\
  Easton, PA 18042
}
\date{\today}

%\def\MSPRF{As requested in the posting, I have applied for an NSF Mathematical Sciences Postdoctoral Fellowship.
%If awarded, I plan to use it to carry out the proposed research laid out in my research statement at the University of Glasgow under the direction of Michael Wemyss.}
\def\position{Assistant Professor in Algebra position}
\def\posloc{MathJobs.org}
\def\materials{the standard AMS cover sheet,
  a curriculum vitae,
  a research statement,
  and a teaching statement}
%a publication list,
%and a teaching portfolio}
%and an unofficial transcript}
\def\contacts{Tom Cassidy}

\def\numresrefs{three}
\def\numteachrefs{two}
\def\refs{Matthew Ballard,
  David Favero,
  and Colin Ingalls}
\def\teachingrefs{Matthew Boylan and Cliff Reiter}

\begin{document}
\begin{letter}{
    Search Committee\\
    Colgate University Department of Mathematics\\
    13 Oak Drive\\
    Hamilton NY 13346-1398
  }
  \opening{Dear Members of the Search Committee,}


    My name is Blake Farman, and I am currently a Visiting Assistant Professor at Lafayette College.
  I received my Ph.D. from the University of South Carolina on May 12, 2018 under the direction of Matthew Ballard in the area of Noncommutative Projective Algebraic Geometry.
  I am writing to apply for the \position\ advertised on \posloc.

  My research lies at the intersection of Kontsevich and Artin-Zhang style noncommutative geometry, where I study the derived category of quasi-coherent sheaves on the noncommutative projective schemes of Artin-Zhang via the framework of differential graded categories.
  Fundamentally, the goal of this research is concerned with constructing tools that allow one to turn the lens of modern geometric intuition towards understanding the structure of noncommutative rings.
  This work draws from a wide range of subjects that fall broadly within algebra, geometry, and category theory, and provides a broad perspective conducive to working with undergraduate students across a variety of interests.

  In particular, the perspective granted by my research preparation positions me to provide a meaningful dictionary between abstract algebraic concepts and tangible geometric examples.
  This skill is clearly effective for communicating concepts effectively to mathematical audiences, predominantly in the form of guest lectures aimed at advanced undergraduate/early graduate audiences, seminars, and colloquium talks.
  However, it has proven itself especially invaluable in the courses for which I have served as instructor of record.
  In courses ranging from college algebra through the calculus sequence, where I encounter many students whose interests lie outside of mathematics, I find .
  

  %As an instructor, I am always excited to share this and any other insight I can with students.
  %Indeed, I take great joy especially in seeing the proverbial 'light bulb' turn on in a student's mind.
  
    %As an instructor, I have a deep appreciation for the beauty and power of mathematics that I am always excited to share with students.
  %Additionally, I believe I am not only excited, but well situated, to supervise research projects in algebra or geometry broadly.

%  My research lies at the intersection of Kontsevich and Artin-Zhang style noncommutative geometry, where I study the derived category of quasi-coherent sheaves on the noncommutative projective schemes of Artin-Zhang via the framework of differential graded categories.
  %My dissertation provides a noncommutative definition of integral transform and proves, under appropriate cohomological assumptions, that to any equivalence between derived categories of quasi-coherent sheaves one may associate an integral transform.
  %In the commutative situation, the analogous result is due to Orlov (and To\"{e}n and Lunts in more generality), and forms a corner stone for our understanding of derived categories in algebraic geometry.
  
  %In all the courses that I teach, I feel that it is imperative to foster an inclusive environment.
  %It is my firmly held belief that everyone should have equal access to education and, along this line, I believe that traditionally underrepresented groups should be embraced and encouraged.
  %Moreover, the prominent role of colleges and universities in preparing students for an increasingly technical labor market makes this especially important within mathematics and other technical disciplines.
  %As a graduate student, I have had some experience promoting diversity through the University's high school math contest, and I look forward to broadening my service as my career progresses.
  
  With my application, I include \materials.
  I have also arranged for \numresrefs\  letters of recommendation regarding my research from \refs, and \numteachrefs\ letters of recommendation regarding my teaching from \teachingrefs.
  %I also include \numresrefs\ letters of recommendation regarding my research from \refs.

  %\MSPRF
  
  Please let me know if there is anything else I can provide.
  Thank you in advance for your time and consideration.
  \closing{Sincerely,}
  \encl{
    Curriculum Vitae\\
    Research Statement\\
    Teaching Statement\\
    %Publication List\\
    %Unofficial Transcript
    %Teaching Portfolio
  }
\end{letter}
\end{document}
