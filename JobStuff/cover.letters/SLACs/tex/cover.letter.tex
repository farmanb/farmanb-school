\documentclass[12pt]{letter}

\usepackage[margin=1in]{geometry}
\openup 5pt
\pdfpagewidth 8.5in
\pdfpageheight 11in
\usepackage{amsmath}

\signature{Blake Farman}
\address{
  306 Cattell St\\
  Floor 2\\
  Easton, PA 18042
}
\date{\today}

\def\school{Central Connecticut State University}
\def\courses{linear and abstract algebra, and topology}
\def\position{Assistant Professor position}
\def\posloc{MathJobs.org}

\def\diversityblurb{In all the courses that I teach, I feel that it is imperative to foster an inclusive environment.
  It is my firmly held belief that everyone should have equal access to education and, along this line, I believe that traditionally underrepresented groups should be embraced and encouraged, a sentiment shared by the community at Lafayette College.
  During my time here, I have had the opportunity to review literature regarding inclusivity in STEM, and to discuss these ideas and their implementation with fellow faculty members.
  I have also taken the opportunity to participate in biweekly teaching lunches within the mathematics department, where colleagues share their philosophies and methods for fostering an inclusive and effective teaching environment.
  As a direct result, I have incorporated examples from various disciplines which are relevant to students' lives, and implemented active learning techniques to reach a wide variety of learning styles.
  I look forward to the opportunity to continue this work as my career progresses.}

\def\materials{the standard AMS cover sheet,
  a curriculum vitae,
  a teaching statement
  a research statement,
  and transcripts}
\def\contacts{Tom Cassidy}

\def\numresrefs{three}
\def\numteachrefs{two}
\def\refs{Matthew Ballard,
  David Favero,
  and Colin Ingalls}
\def\teachingrefs{Matthew Boylan and Cliff Reiter}

\begin{document}
\begin{letter}{
    Central Connecticut State University\\
    Department of Mathematical Sciences\\
    1615 Stanley Street\\
    New Britain, CT 06053
  }
  \opening{Dear Members of the Search Committee,}
  
    My name is Blake Farman, and I am currently a Visiting Assistant Professor at Lafayette College.
  I received my Ph.D. from the University of South Carolina on May 12, 2018 under the direction of Matthew Ballard in the area of Noncommutative Projective Algebraic Geometry.
  I am writing to apply for the \position\ advertised on \posloc.

  My research lies at the intersection of Kontsevich and Artin-Zhang style noncommutative geometry, where I study the derived category of quasi-coherent sheaves on noncommutative projective schemes via the framework of differential graded categories.
  Fundamentally, the goal of this research is concerned with constructing tools that allow one to turn the lens of modern geometric intuition towards understanding the structure of noncommutative rings.
  This work draws from a wide range of subjects that fall largely within algebra, geometry, and category theory, and provides a broad perspective conducive to working with students across a variety of interests.
  

  In particular, the perspective granted by my research preparation allows me to provide a meaningful dictionary between abstract algebraic concepts and tangible geometric examples, a skill that is clearly effective for communicating concepts to general mathematical audiences.
  I have made extensive use of this skill while teaching at the advanced undergraduate and graduate levels, where I have delivered guest lectures, seminars, and colloquia.
  However, it has proven itself especially invaluable in the introductory courses in which I have served as instructor of record, where I encounter many students whose interests lie outside of mathematics and for whom such imagery is most helpful in achieving a deeper conceptual understanding of the material.
  % beyond rote manipulation.

  Teaching at each of these levels presents its own set of challenges, which I have found rewarding, both as a graduate student and, especially, in my current position at Lafayette College.
  I am eager to have the opportunity to teach advanced courses such as \courses, in addition to conducting research, both along the program outlined in my research statement, as well as on related topics accessible to students.
  I believe that \school\ will provide the ideal environment to achieve this goal, and, moreover, I believe that my experience at Lafayette has prepared me to excel in a diverse, student-centered setting.
  %in a small liberal arts setting.
  %student-centered liberal arts setting.
  
  \diversityblurb
  
With my application, I include \materials.
I have also arranged for \numresrefs\  letters of recommendation regarding my research from \refs, and \numteachrefs\ letters of recommendation regarding my teaching from \teachingrefs.
%I have also provided contact information for \numresrefs\ letters of recommendation regarding my research from \refs, and \numteachrefs\ letters of recommendation regarding my teaching from \teachingrefs.



  Please let me know if there is anything else I can provide.
  Thank you in advance for your time and consideration.
  \closing{Sincerely,}
  \encl{
    Curriculum Vitae\\
    Teaching Statement\\
    Research Statement\\
    Transcripts
  }
\end{letter}
\end{document}
