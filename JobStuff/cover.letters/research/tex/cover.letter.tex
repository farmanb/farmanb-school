\documentclass[10pt]{letter}

\usepackage[margin=1in]{geometry}

\signature{Blake Farman}
\address{
  2807 Holt Dr\\
  Columbia, SC 29205
}
\date{\today}

%\def\MSPRF{As requested in the posting, I have applied for an NSF Mathematical Sciences Postdoctoral Fellowship.
%If awarded, I plan to use it to carry out the proposed research laid out in my research statement at the University of Glasgow under the direction of Michael Wemyss.}
\def\position{Visiting Assistant Professor position}
\def\materials{the standard AMS cover sheet,
  a curriculum vitae,
  a research statement,
  a teaching statement,
  a publication list,
  and a teaching portfolio}
  %and an unofficial transcript}
\def\contacts{Manuel Reyes}
\def\numresrefs{three}
\def\numteachrefs{one}
\def\refs{Matthew R. Ballard, David Favero, and Colin Ingalls}
\def\teachingrefs{Matthew Boylan}

\begin{document}
\begin{letter}{
    Mathematics\\
    Bowdoin College\\
    8600 College Station\\
    Brunswick, ME 04011-8486
}
  \opening{To the Hiring Committee,}
  
  My name is Blake Farman.
  I am a graduate student at the University of South Carolina, working in the area of Derived Categories in Noncommutative Projective Algebraic Geometry.
  I expect to receive a Ph.D. in May of 2018, under the direction of Matthew R. Ballard.
  I am writing to apply for the \position.
  %I am particularly interested in working with \contacts.

  %For the former, one studies a not necessarily commutative graded ring via an analog of Serre's description for sheaves on a projective variety.
  %For the latter, one focuses on the dg-category of quasi-coherent sheaves on a variety.
  %So for a regular old projective variety, this passage is going from the abelian category of coherent or quasi-coherent sheaves to its derived category.
  %This is what we need to do follow the edicts of physics, to understand deep but subtle connections in moduli theory, and to gain new insights into birational geometry.
%  Consequently, for varieties, an impressive machine has been constructed by Mukai, Bondal, Orlov, Bridgeland, and others to understand the homological algebra in terms of the geometry.
%  In particular, thanks to Orlov (and Toen and Lunts in more generality) we know that any equivalence between D(X) and D(Y) is an integral transform associated to a kernel on the product X x Y.
%  If we think noncommutatively, this is a derived Morita statement and we can ask if it holds for Artin-Zhang noncommutative projective schemes.
%  Blake's result is that (given some standard homological assumptions) it does.
%  This is a consequence of a stronger result on dg-functors between D(X) and D(Y).
%  For more details, please see the enclosed research statement. 

  As a researcher, my interests lie in noncommutative algebraic geometry, which align particularly well with the interests of Professor \contacts, and I believe that collaboration could prove fruitful. 
  As a teacher, I have a deep appreciation for the beauty and power of mathematics that I am always excited to share with students.
  While I am particularly interested in teaching courses related to my research area, generally I am delighted to teach mathematics at large.

  In all the courses that I teach, I feel that it is imperative to foster an inclusive environment.
  It is my firmly held belief that everyone should have equal access to education and, along this line, I believe that traditionally underrepresented groups should be embraced and encouraged.
  Moreover, the prominent role of colleges and universities in preparing students for an increasingly technical labor market makes this especially important within mathematics and other technical disciplines.
  As a graduate student, I have had some experience promoting diversity through the University's high school math contest, and I look forward to broadening my service as my career progresses.
  
  %My research lies at the intersection of Artin-Zhang style noncommutative algebraic geometry and Kontsevich style noncommutative geometry, which I believe aligns well with the interests of Professor \contacts.
  %My research focuses on methods of derived categories in noncommutative projective algebraic geometry.
  %In recent joint work with my advisor, we have established an analogue of Fourier-Mukai kernels for Artin and Zhang's noncommutative projective schemes.
  %My research is aimed towards importing methods of projective algebraic geometry into the study of noncommutative graded algebras.
  %Please see the enclosed research statement for a more detailed treatment of my proposed research program.

  %By the end of my tenure as a graduate student, I will have served as instructor of record for 20 undergraduate level courses ranging from college algebra to calculus II;
  %the specific courses can be found in my curriculum vitae.
  %In addition, I have also spent time tutoring students one-on-one in department sponsored tutoring labs and covered lectures at the graduate level.

  With my application, I include \materials.
  I have also arranged for \numresrefs\  letters of recommendation regarding my research from \refs, and \numteachrefs\ letter of recommendation regarding my teaching from \teachingrefs.
  %I also include \numresrefs\ letters of recommendation regarding my research from \refs.

  %\MSPRF
  
  Please let me know if there is anything else I can provide.
  Thank you in advance for your consideration.
  \closing{Sincerely,}
  \encl{
    Curriculum Vitae\\
    Research Statement\\
    Teaching Statement\\
    Publication List\\
    %Unofficial Transcript
    Teaching Portfolio
  }
\end{letter}
\end{document}
