\documentclass[11pt]{amsart}
\usepackage[margin=1in]{geometry}
\usepackage{style}
\title{Thesis Abstract}
\author{Blake A. Farman}
\date{}
\begin{document}
\maketitle
Noncommutative Projective Schemes were introduced by Michael Artin and J.J. Zhang in their 1994 paper of the same name as a generalization of projective schemes to the setting of not necessarily commutative algebras over a commutative ring.
In this work, we study the derived category of quasi-coherent sheaves associated to a noncommutative projective scheme with a primary emphasis on the triangulated equivalences between two such categories.

We adapt Artin and Zhang's noncommutative projective schemes for the language of differential graded categories and work in \(\Ho{\dgcat{k}}\), the homotopy category of differential graded categories, making extensive use of Bertrand To\"en's Derived Morita Theory.
For two noncommutative projective schemes, \(X\) and \(Y\), we associate differential graded enhancements, \(\D(X)\) and \(\D(Y)\), of the respective derived categories of quasi-coherent sheaves.
Under appropriate cohomological conditions, we provide a noncommutative geometric description of the subcategory, \(\mathbf{R}\underline{\op{Hom}}_c(\D(X), \D(Y))\), of the internal Hom category in \(\Ho{\dgcat{k}}\).
As an immediate application, we show that, under these conditions, any triangulated equivalence between the derived categories induces an equivalence of Fourier-Mukai type, with kernel an object of the derived category of quasi-coherent sheaves on the appropriate analogue of the product.
\end{document}
