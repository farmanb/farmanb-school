\documentclass[teaching.portfolio.tex]{subfiles}
\begin{document}
\subsection{University of Vermont}
Below is a list of courses taught while working towards a Master of Science degree at the University of Vermont.
Included in each section is the course description provided by the university, the semesters taught, and the role.
\subsubsection{Math 017: Applications of Finite Math}\hfill\\

\begin{tcolorbox}
  \begin{desc}
    Introduction to mathematics of finite systems with applications, such as probability, statistics, graph theory, fair division and apportionment problems, voting systems.
  \end{desc}
\end{tcolorbox}
\noindent
\textbf{Fall 2009}: Served as instructor of record for one section.\\
\textbf{Spring 2009}: Served as instructor of record for one section.\\
\textbf{Summer 2009}: Served as instructor of record for one section.
\subsubsection{Math 019: Fundamentals of Calculus I}\hfill\\
\begin{tcolorbox}
  \begin{desc}
    Introduction to limits and differential calculus with a wide variety of applications.
  \end{desc}
\end{tcolorbox}

\noindent
\textbf{Spring 2010}: Served as instructor of record for two sections.
\subsection{University of South Carolina}
Below is a list of courses taught while working towards a Doctorate of Philosophy at the University of South Carolina.
Included in each section is the course description provided by the university, the semesters taught, and the role.
\subsubsection{Math 111: Basic College Mathematics}\hfill\\
\begin{tcolorbox}
  \begin{desc}
    Basic college algebra; linear and quadratic equations, inequalities, functions and graphs of functions, exponential and logarithm functions, systems of equations.
  \end{desc}
\end{tcolorbox}

\noindent
\textbf{Summer 2013}: Served as instructor of record for one section.\\
\textbf{Spring 2015}: Served as instructor of record for one section.\\
\textbf{Fall 2017}: Served as instructor of record for one section.

\subsubsection{Math 115: Precalculus Mathematics}\hfill\\
\begin{tcolorbox}
  \begin{desc}
    Topics in algebra and trigonometry specifically needed for MATH 141, 142, 241. Subsets of the real line, absolute value; polynomial, rational, inverse, logarithmic, exponential functions; circular functions; analytic trigonometry.
  \end{desc}
\end{tcolorbox}

\noindent
\textbf{Fall 2014}: Served as instructor of record for one section.

\subsubsection{Math 116: Brief Precalculus Mathematics}\hfill\\
\begin{tcolorbox}
  \begin{desc}
    Essential algebra and trigonometry topics for Calculus, including working with equations that involve polynomials, rational functions, exponential and logarithmic functions, and trigonometric and inverse trigonometric functions. Intended for students with prior experience in Precalculus, but not ready for MATH 141.
  \end{desc}
\end{tcolorbox}

\begin{rmk}
  This is a half semester course, starting in October, intended for students in Calculus I who are struggling with algebra and trigonometry.
Generally, students drop into this course from Calculus I after the withdraw period has passed.
This course covers the same material as Math 115 (above) in this shortened period.\\
\end{rmk}

\noindent
\textbf{Fall 2013}: Served as instructor of record for two sections.\\
\textbf{Fall 2015}: Served as instructor of record for two sections.\\
\textbf{Fall 2016}: Served as instructor of record for two sections.

\subsubsection{Math 122: Calculus for Business Administration and Social Sciences}\hfill\\
\begin{tcolorbox}
  \begin{desc}
    Derivatives and integrals of elementary algebraic, exponential, and logarithmic functions. Maxima, minima, rate of change, motion, work, area under a curve, and volume.
  \end{desc}
\end{tcolorbox}

\begin{rmk}
  This class only introduces the formal calculus of the functions mentioned above.
  Concepts such as limits, continuity, and trigonometric functions are explicitly \textit{not} to be covered in the course.
\end{rmk}
\noindent
\textbf{Spring 2017}: Served as instructor of record for one section.

\subsubsection{Math 141: Calculus I}\hfill\\
\begin{tcolorbox}
  \begin{desc}
    Functions, limits, derivatives, introduction to integrals, the Fundamental Theorem of Calculus, applications of derivatives and integrals.
  \end{desc}
\end{tcolorbox}

\noindent
\textbf{Fall 2013}: Served as teaching assistant for two sections of Calculus I, responsible for running one 50 minute recitation and one 50 minute Maple lab each week.

\subsubsection{Math 142: Calculus II}\hfill\\
\begin{tcolorbox}
  \begin{desc}
    Methods of integration, sequences and series, approximations.
  \end{desc}
  \end{tcolorbox}

\noindent
\textbf{Fall 2012}: Served as teaching assistant for two sections of Calculus II, responsible for running one 50 minute recitation and one 50 minute Maple lab each week.\\
\textbf{Summer 2014}: Served as instructor of record for one section.\\
\textbf{Spring 2018}: Will serve as instructor of record for one section.

\subsubsection{Math 170: Finite Mathematics}\hfill\\
\begin{tcolorbox}
  \begin{desc}
    Elementary matrix theory; systems of linear equations; permutations and combinations; probability and Markov chains; linear programming and game theory.
  \end{desc}
\end{tcolorbox}

\noindent
\textbf{Spring 2014}: Served as instructor of record for one section.\\
\textbf{Spring 2016}: Served as instructor of record for one section.

\end{document}
