\documentclass[11pt]{amsart}
\usepackage[margin=1in]{geometry}

\title{Teaching Philosophy}
\author{Blake A. Farman}
\date{}
\begin{document}
\maketitle

To my mind, the enterprise of mathematics is, in some sense, largely a linguistic endeavour, and it is this philosophy that shapes my outlook on teaching.
Of particular note are two professors, David S. Dummit and Richard M. Foote, to whom I owe a great intellectual debt for this invaluable perspective.
Both demanded of those of us who passed through their classes to not only master the mundane technical details of the subject, but to achieve fluency in the expression of the underlying mathematical ideas.
Both professors pushed us to converse, literally, in class without the aid of writing, an exercise that forced us to internalize the material by attaching meaning to the objects of study and thereby to develop a mathematical vocabulary beyond the rudimentary grammar of formal manipulation.
This is something that I strive for, to varying degrees, in all of the courses I teach.

Since mathematics is the natural choice of language to express a wide array of real world phenomena, it is unsurprising that many students with no interest in pursuing mathematics in its own right find themselves in mathematics courses in order to develop the skills necessary to express ideas within their chosen field of study.
At every level, it is extremely important to me that I push my students to take away from any course I teach a solid conceptual understanding of the material.
That is to say, I want my students to leave my courses with an understanding not just of the formal manipulations, but a genuine understanding of what the objects involved are and \textit{why} they are manipulated in the ways they are.

As an instructor, I encourage my students to be active participants in a (somewhat one-sided) semester long conversation complemented by independent work on exercises intended to facilitate mastery of the mechanics.
The latter requires a great deal of practice, often times involving tedious repetition, and I find that it is unenlightening for the students to watch another perform these tasks.
I curate the exercises to strike to the heart of the material and encourage the students to ask questions--either in class or during office hours--whenever they may get stuck.
In my view, the importance of my role as instructor lies in expediting the journey to a deeper understanding of the material beyond the technical details.
Indeed, were the technical details sufficient, one could simply offer the students a well written set of lecture notes and be done with the whole affair.

While I do not think it is reasonable to expect that one could somehow transmit their own understanding to another, I can at least attempt to ascribe meaning to the material by presenting the fruits of my own personal contemplations.
While I may do the majority of the talking during the lectures, the lion's share of the burden, in some sense, lies on the students: it is imperative that they grapple with and probe the material by asking and answering questions in the classroom, which I view as their end of the conversation.
As such, I work to foster an environment where my students feel comfortable asking and answering questions during class.
The questions I pose are generally intended to push the students to reflect on whether they understand the material.
I encourage the students to always ask for clarification whenever the answer to that reflection is negative.

By encouraging my students to actively engage with the material and modeling my classroom on the instructors who have had the most profound effect on my mathematical development, my students leave my classes with a solid conceptual understanding of the material.

\end{document}
