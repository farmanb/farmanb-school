\documentclass[reqno, 12pt]{amsart}
\usepackage{style}

\title[Noncommutative kernels]{Kernels for noncommutative projective schemes}
\author{Matthew Ballard and Blake Farman}

\email{ballard@math.sc.edu \\ farmanb@math.sc.edu}
\address{University of South Carolina, Columbia, SC, USA}

\keywords{moduli spaces, derived categories, variation of stability}

\thanks{Both authors were partially supported by a NSF Standard Grant DMS-1501813.}

\begin{document}
\begin{abstract}
  In their 1994 paper, Noncommutative Projective Schemes, Michael Artin and J.J. Zhang introduce a generalization of usual projective schemes to the setting of not necessarily commutative algebras over a commutative ring. Gon\c{ç}alo Tabuada in \cite{Tab05} endows the category of differential graded categories with the structure of a model category and, in \cite{Toen}, Toën shows that its homotopy category is symmetric monoidal closed. In this talk, we’ll give a brief overview of these results, adapting Artin and Zhang’s noncommutative projective schemes for the language of DG categories, and discuss a “geometric” description of this internal Hom for two noncommutative projective schemes in the style of To\"en's Morita theory for DG categories.
  As an immediate application, we give a noncommutative projective derived Morita statement along the lines of Rickard and Orlov.
\end{abstract}

\maketitle

\section{Morita Theory  for Rings} \label{section: Morita for Rings}
The classical example of Morita theory is for rings:

\begin{theorem}\label{thm: Morita}
  If $A$ and $B$ are unital rings, then the categories $\Mod{A}$ and $\Mod{B}$ of left modules are equivalent if and only if there exists a $B$-$A$-bimodule, $P$ and an $A$-$B$-bimodule $Q$ such that
  $$Q \otimes_B P \cong B\ \text{and}\ P \otimes_A Q \cong A.$$
\end{theorem}

\begin{lemma}\label{lem: endomorphisms of identity functor are the center}
  Let $A$ be a unital ring.
  Denoting by $\Abcat$ the 2-category of \newline$\Ab$-enriched categories, define 
  $$\End_{\Mod{A}} = \Abcat(\Mod{A},\Mod{A}).$$
There is an isomorphism
$$Z(A) \cong \End_{\Mod{A}}(\id_{\Mod{A}}, \id_{\Mod{A}}).$$
\end{lemma}

\begin{remark}
  We note that when $A$ is not necessarily commutative, it is imperative that $z \in Z(A)$, for otherwise the associated morphism $\eta_A$ does not commute with left multiplication by some element of $a \in A$, since
  $$\varphi_a \circ \eta_A(a^\prime) = a(za^\prime) \neq z(aa^\prime) = \eta_A \circ \varphi_a(a^\prime),$$
  which breaks the naturality of $\eta$.
\end{remark}

\begin{corollary}\label{cor: morita implies isomorphic centers}
  If $A$ and $B$ are Morita equivalent, then $Z(A) \cong Z(B)$.
\end{corollary}

We now obtain the obvious corollary
\begin{corollary}
  If $A$ and $B$ are commutative, then $\Mod{A}$ and $\Mod{B}$ are equivalent if and only if $A$ and $B$ are isomorphic.
\end{corollary}

\begin{corollary}\label{cor: morita example}
  If $A$ is a unital ring, then $A$ is Morita equivalent to the matrix ring $M_n(A)$.
\end{corollary}

\section{Morita for $D(\Mod{A})$}

One attempts to slacken the relationship by extending Morita to the setting of derived categories, as was done by Rickard in \cite{JR89}.
\begin{theorem}
  Let $A$ and $B$ be rings.  If
  $$F \colon D(\Mod{A}) \to D(\Mod{B})$$
  is an equivalence of derived categories of modules, then there exists a complex of $A$-$B$-bimodules which is perfect as a complex of $A$-modules and whose derived endomorphisms as $B$-modules are just $B$ in degree 0.
  Consequently, $P \otimes^\mathbf{L}_A -$ is an equivalence.
\end{theorem}

However, even here two commutative rings are derived Morita equivalent if and only if they are isomorphic.

\section{Global Morita}
To obtain more interesting relationships between commutative rings, one should globalize the notion of a commutative ring by passing to schemes.
Orlov gives a specific version for smooth projective schemes in \cite{Orlov}, and later the results from \cite{Toen} and \cite{Lunts Orlov} give the more general result.

\begin{theorem}[Lunts-Orlov \cite{Lunts Orlov}, To\"en \cite{Toen}]
  Let $X$ and $Y$ be quasi-compact and separated schemes over a field $k$.
  If
  $$F \colon D(\Qcoh{X}) \to D(\Qcoh{Y})$$
  is an equivalence, then there exists an object $E \in D(\Qcoh{X \otimes_k Y})$ such that
  $$\Phi_E \colon D(\Qcoh{X}) \overset{\sim}\to D(\Qcoh{Y})$$
  is an equivalence.
\end{theorem}

One can describe $\Phi_E$ as follows.
The object $X \times_k Y$ is equipped with the two projections,
$$\begin{tikzcd}
  X  & X \times_k Y \arrow[swap]{l}{\pi_X}\arrow{r}{\pi_Y}& Y
\end{tikzcd}$$
which induce morphisms
$$\pi_X^\ast \colon \Qcoh{X} \to \Qcoh{X \times_k Y}$$
and
$${\pi_{Y}}_\ast \colon \Qcoh{X \times_k Y} \to \Qcoh{Y}.$$
An object, $M$, of $D(\Qcoh{X})$ is sent by $\Phi_E$ to the object
$$\mathbf{R}{\pi_Y}_\ast \left(E \otimes^{\mathbf{L}} \mathbf{L}\pi_X^\ast M\right).$$

\begin{remark}
  The category $\Qcoh{X}$ is a Grothendieck category and thus we may equip $\Ch{\Qcoh{X}}$ with the cofibrantly generated injective model structure.
  For any morphism $f \colon X \to Y$ of schemes we have an adjunction
  $$\Qcoh{X}(f^\ast G, F) \cong \Qcoh{Y}(G,f_\ast F)$$
  This adjunction is Quillen if and only if $f$ is flat and, in this case, we can define
  $$\mathbf{R}f_\ast(F) = f_\ast(RF)\ \text{and}\ \mathbf{L}f^\ast(G) = f^\ast(QG) = f^\ast(G),$$
  where $RF$ is a fibrant replacement of $F$ and $QG = G$ is a cofibrant replacement (since every object is cofibrant in the injective model structure).
  In the case where $X$ and $Y$ are flat over $k$, then both projections are flat, so both $\mathbf{L}\pi_X^\ast$ and $\mathbf{R}{\pi_Y}_\ast$ are well-defined.

  Note that when $F$ is just a quasi-coherent sheaf, $RF$ is an injective resolution, and the cohomology groups of the total right derived functor correspond to the classical right derived functors:
  $$H^i\mathbf{R}f^\ast(F) \cong \mathbf{R}^i f^\ast (F).$$
  For more information, one should consult \cite{Hov01}.
  
\end{remark}
\section{Noncommutative Global Morita}

\subsection{Noncommutative Geometry}
Let $k$ be a commutative ring, and let $A$ be a $\Z$-graded $k$ algebra.
Define $A_{\geq n} = \oplus_{n \leq d} A_d$.
We denote by $\Gr{A}$ the category of $\Z$-graded $A$-modules and by $\Tors{A}$ the Serre subcategory of graded $A$-modules, where $M$ is torsion if for every element $m \in M$ there exists some $n$, possibly depending on $m$, such that $A_{\geq n}m = 0$.

\begin{definition}
  Following \cite{AZ}, we denote the quotient category of graded $A$-modules of torsion by
  $$\QGr{A} = \Gr{A}/\Tors{A}.$$
\end{definition}
By definition, the objects of this category are the objects of $\Gr{A}$ and the morphisms are given by the colimit over the directed system of pairs of subobjects $(M^\prime, N^\prime)$ such that $M/M^\prime$ and $N^\prime$ are objects of $\Tors{A}$
$$\QGr{A}(\pi M, \pi N) = \colim_{(M^\prime, N^\prime)} \Gr{A}(M^\prime, N/N^\prime)$$
where $\pi: \Gr{A} \to \QGr{A}$ is the canonical projection functor.
It is well known that $\pi$ is exact and admits a fully faithful right adjoint,
$$\omega \colon \QGr{A} \to \Gr{A}$$
and this allows us to view $\QGr{A}$ as the full subcategory of torsion-free graded $A$-modules, $M \cong QM$, that are injective with respect to the class $J$ of monomorphisms $M \to N$ with torsion cokernel.
In the language of \cite{DCA}, the objects in the image of $\QGr{A}$ under $\omega$ are precisely the $\Tors{A}$-closed objects.
In particular, defining $Q = \omega\pi$ allows one to view $\QGr{A}$ as $Q(\Gr{A})$.

The category $\QGr{A}$ is defined to be the quasi-coherent sheaves on the noncommutative projective scheme $X$ associated to $A$.
In general, $X$ is not a space in the traditional sense (see \cite{AZ}), however, when $A$ is commutative and $X = \operatorname{Proj}(A)$, then this definition is justified by the following famous result of Serre.
\begin{theorem}[Serre]
  If $A$ is generated over $k$ by elements of degree 1, then the functor
  $$\begin{tikzcd}[row sep=tiny]
    \Gamma_\ast \colon \Qcoh{X} \arrow{r}& \QGr{A}\\
    M \arrow[mapsto]{r} & \bigoplus_{d \in \Z} H^0(X, M \otimes \mathcal{O}_X(d))
  \end{tikzcd}$$
  is an equivalence.
\end{theorem}

\subsection{What is the analogue of $X \times Y$?}
Following the line of reasoning in the globalized Morita statement for commutative rings, one is naturally led to think about a globalized version for noncommutative rings.
The framework of \cite{AZ} allows one to pose such a question.
In analogy with the way that one associates the scheme $X \times_k Y$ to the tensor product of commutative rings $A \otimes_k B$, one can associate a noncommutative (bi)projective scheme $X^\opp \times Y$ to the bigraded $k$-algebra $A^\opp \otimes_k B$.

\subsection{Cohomological Conditions}
In general, good behavior of $\QGr{A}$ occurs with some cohomological assumptions on the ring $A$.
We recall here some common assumptions.
\begin{definition}
  For a connected graded $k$-algebra, $A$, we say
  \begin{enumerate}[(i)]
  \item
    (\cite{VdB}) $A$ is \textbf{Ext-finite} if the ungraded Ext-groups, $\Ext^n_A(k,k)$, are finite dimensional $k$-modules for $n \geq 0$, and
  \item
    (\cite{AZ}) $A$ satisfies $\chi^\circ(M)$ if $\EXT_A^n(k,M)$ has \textbf{right limited grading} for each $0 \leq n$; that is, there exists some $d_0$ such that $\EXT_A^n(k,M)_d = 0$ for all $d_0 < d$.
  \end{enumerate}
\end{definition}

We recall some basic facts about Ext-finite from \cite{VdB}.
\begin{lemma}[{\cite[Lemma 4.2]{VdB}}]
  If $A$ is Ext-finite, then so is $A^\opp$.
  Moreover, for each $n$, $A_n$ is a finite-dimensional $k$-vector space.

  If in addition, $B$ is Ext-finite, then so is $A \otimes_k B$.
\end{lemma}

We can give a more geometric interpretation of $\chi^\circ$ through noncommutative sheaf cohomology.
As in the commutative case, one views the image of $A$ in $\QGr{A}$ as the structure sheaf, $\mathcal{O}_X$, of the noncommutative projective scheme $X$ associated to $A$.
From standard torsion theory, $\Tors{A}$ and the full subcategory of torsion-free objects gives a torsion pairing on $\Gr{A}$, which implies that $\Tors{A}$ is a coreflective subcategory of $\Gr{A}$.
The right adjoint to the inclusion is the functor $\tau$, which takes $M$ to its maximal torsion submodule, $\tau(M)$.
There is an exact sequence
$$0 \to \tau M \to M \to Q M \to \mathbf{R}^1\tau M \to 0$$
and isomorphisms $\mathbf{R}^iQ M \cong \mathbf{R}^{i+1}\tau M$ for $1 \leq i$.
Of particular interest here is the fact that both $\tau$ and $Q$ preserve injective objects (see, e.g., \cite{AZ}), hence for an injective object, $E$, of $\Gr{A}$ we have a split exact sequence of injectives
$$0 \to \tau E \to E \to Q E \to 0.$$
\begin{definition}
  One defines sheaf cohomology of a quasi-coherent sheaf $\mathcal{M} := \pi(M)$ to be
  $$\underline{H}^i(\mathcal{M}) := \EXT^i_{\QGr{A}}(\mathcal{O}_X, \mathcal{M}) \cong \mathbf{R}^iQ (M)$$
  and the un-graded sheaf cohomology by
  $$H^i(\mathcal{M}) := \underline{H}^i(\mathcal{M})_0 \cong (\mathbf{R}^iQ(M))_0.$$
\end{definition}

With this in hand, we can understand $\chi^\circ$ geometrically through the following proposition.
\begin{proposition}
  Let $\mathcal{M} = \pi(M)$.
  Then $\chi_i^\circ(M)$ holds if and only if
  $$\colim_n \EXT^j_A(A/A_{\geq n}, M) \cong \mathbf{R}^j\tau_A(M) \cong \mathbf{R}^{j-1}Q_A (M) = \underline{H}^{j-1}(\mathcal{M})$$
  is right limited for $2 \leq j \leq i$.
\end{proposition}
\begin{proof}
  The first isomorphism is \cite[Lem. 4.1.2]{VdB} and the rest of the statement is \cite[Cor. 3.6 (3)]{AZ}.
\end{proof}
To say that $A$ satisfies $\chi^\circ(A)$ is to say that $\chi_i^\circ(A)$ holds for all $i$, which is to say that for all $1 \leq j$ the cohomology of the twisting sheaves $\mathcal{O}_X(n)$ vanish for sufficiently large $n$.
One can think of a ring $A$ which is Ext-finite, and satisfies $\chi^\circ(A)$ and $\chi^\circ(A^\opp)$ as requiring Serre vanishing for the twisting sheaves (on $X$ and $X^\opp$) plus some local finite-dimensionality over $k$.

%One can relate $\chi$ to $\chi^\circ$ through the following proposition.
We make the following definition
\begin{definition}
  Let $k$ be a commutative ring and let $A$ and $B$ be connected graded $k$-algebras.
  If $A$ is Ext-finite, left and right Noetherian, and satisfies $\chi^\circ(A)$ and $\chi^\circ(A^\opp)$, then we say that $A$ is \textbf{truly tasty}.
  If $A$ and $B$ are both truly tasty, then we say that $A$ and $B$ form a \textbf{tasty pair}.
\end{definition}

\subsection{Some examples of algebras that satisfy $\chi$}
The following are some examples of algebras that actually satisfy the stronger condition, $\chi$, from \cite{AZ}:
$A$ satisfies $\chi(M)$ if for each $n$ and $x$ there is a $y_0$ such that $\EXT^n_A(A/A_{\geq y},M)_{\geq x}$ is a finite $A$-module for each $n$, $x$, and $y_0 \leq y$.
We say that $A$ satisfies $\chi$ if for all finite graded $A$-modules, $M$, $A$ satisfies $\chi(M)$.
In particular, these algebras satisfy $\chi(A)$.

The connection between $\chi$ and $\chi^\circ$ is the following.

\begin{proposition}
  Let $A$ be left Noetherian and let $M$ be a graded left $A$-module. Then $\chi(M)$ holds if and only if $\chi^\circ(M)$ holds and for each $n,x$ the graded module $\underline{\op{Ext}}_A^n(A/A_{\geq y},M)_{\geq x}$ is a finitely generated $k$-module for $y \gg 0$.
\end{proposition}
\begin{proof}
  This is \cite[Prop 3.8 (1)]{AZ}.
\end{proof}

\subsubsection{\cite{AS} Regular}
We say a graded algebra, $A$, over a field $k$ is \textbf{regular} if it is connected graded and if it has the following properties
\begin{enumerate}
\item
  A has finite global dimension, $d$,
\item
  $A$ has finite GK-dimension (Gelfand-Kirillov),
\item
  $A$ is Gorenstein, meaning that $\EXT^i_A(k,A) = 0$ if $i \neq d$ and for some $l$, $\EXT^d_A(k,A) \cong k[l]$.
\end{enumerate}

\begin{remark}
  If $A$ is a noetherian connected graded algebra having global dimension 1, then $A$ is isomorphic to $k[x]$, where $\op{deg}(x) = n$ for some $0 < n$.
\end{remark}

\subsubsection{PI-Algebras (Polynomial Identity)}
If $A$ is such that for some non-zero element $P$ of the free algebra $k<X_1, X_2, \ldots, X_N>$ and for all $a \in A^N$
$$P(a) = 0$$
then $A$ satisfies $\chi$.

\subsubsection{Veronese Subrings}
Let $B$ be a left Noetherian graded algebra.
If the Veronese subring, $A = B^{(r)}$, for some $2 \leq r$, is Noetherian, then $B$ satisfies $\chi$ if and only if $A$ satisfies $\chi$.
\section{The Main Result}
As an application of our main result, we have a Morita-style statement for the noncommutative projective schemes associated to a tasty pair
\begin{theorem}[Ballard, F.]
  Let $k$ be a field, let $A$ and $B$ be a tasty pair, and let $X$ and $Y$ be noncommutative projective schemes associated to $A$ and $B$, respectively.
  If there exists an equivalence of categories
  $$F \colon D(\Qcoh{X}) \to D(\Qcoh{Y})$$
  then there exists an object $P \in D(\Qcoh{X \times Y})$ such that
  $$\begin{tikzcd}[row sep=tiny]
    \Phi_P \colon D(\Qcoh{X}) \arrow{r}& D(\Qcoh{Y})\\
    M \arrow[mapsto]{r} & \mathbf{R}{\pi_Y}_\ast(E \otimes^\mathbf{L} \pi_X^\ast M)
  \end{tikzcd}$$
  is an equivalence.
\end{theorem}

\section{Internals}

Our result is a corollary of a more general result characterizing internal Homs in the homotopy category of the category of small differential graded categories,
$\Ho{\dgcat{k}}$.
We briefly recall the basic definitions, and state the relevant theorems of To\"en.

A dg-category over a commutative ring $k$ is simply a category enriched over the symmetric monoidal closed category of chain complexes of $k$-modules.
The category of small differential graded categories, $\dgcat{k}$, has all small dg-categories as objects, and $\Ch{k}$-enriched functors as morphisms.

\begin{example}
  The canonical example of a dg-category is the category $\CH{A}$ of chain complexes of modules over a commutative $k$-algebra, $A$, equipped with the Hom total complex.
\end{example}

One of the major benefits of working with dg-categories is as suitable replacements of triangulated categories, which are well known to have somewhat unsatisfying properties.
In particular, one has pretriangulated dg-categories, for which the category $H^0(\A)$, which has the same objects as $\A$ and morphisms
$$H^0(\A)(X,Y) = H^0(\A(X,Y)),$$
are triangulated.
\begin{definition}
  One defines an \textbf{enhancement} of a triangulated category, $\D$, as a pretriangulated dg-category $\A$ equipped with an equivalence
  $$H^0(\A) \to \D$$
  of triangulated categories.
\end{definition}
In \cite{Lunts Orlov} it is shown that in sufficiently nice situations, these enhancements are unique up to quasi-equivalence.
In particular, under mild hypotheses, derived categories of Grothendieck categories have unique enhancements; this makes dg-categories a natural choice for investigating derived categories of noncommutative projective schemes.

Since we are working with derived categories of Grothendieck categories, we're interested in coproduct preserving triangulated functors between them.
We define $\Ho{\dgcat{k}}$ and motivate its usage.
One says that a dg-functor, $F \colon \A \to \B$, is
\begin{enumerate}
\item
  \textbf{quasi-fully faithful} if for any two objects $A_1$, $A_2$ of $\A$ the morphism
  $$F(A_1, A_2) \colon \A(A_1, \A_2) \to \B(FA_1, FA_2)$$
  is a quasi-isomorphism of chain complexes,
\item
  \textbf{quasi-essentially surjective} if the induced functor $H^0(F) \colon H^0(\A) \to H^0(\B)$ is essentially surjective,
\item
  a \textbf{quasi-equivalence} if $F$ is quasi-fully faithful and quasi-essentially surjective,
\end{enumerate}
Quasi-equivalences could reasonably be thought of as analogous to quasi-isomorphisms in the setting of homological algebra.
By analogy with the derived category of modules, one obtains the category $\Ho{\dgcat{k}}$ by formally inverting the quasi-equivalences in $\dgcat{k}$.
The morphisms of $\Ho{\dgcat{k}}$ between $\A$ and $\B$ are denoted $[\A,\B]$.

Before \cite{Toen}, it was well known that $\dgcat{k}$ is symmetric monoidal closed, that $\Ho{\dgcat{k}}$ is symmetric monoidal, and it was suspected that $\Ho{\dgcat{k}}$ was also closed, although the most obvious techniques of model categories fail to yield the result.
To\"en constructs for two small dg-categories, $\A$ and $\B$, a dg-category $\RHom{\A,\B}$ and proves the following theorem.
\begin{theorem}[{\cite[Theorem 1.1]{Toen}}] \label{theorem: Toen}
  Let $\A$, $\B$, and $\C$ be objects of $\dgcat{k}$.
  There is a natural bijection
  $$[\A \otimes \B, \C] \overset{1:1}\longleftrightarrow [\A, \RHom{\B,\C}]$$
  proving that the symmetric monoidal category $\Ho{\dgcat{k}}$ is closed.
\end{theorem}

Given a dg-category $\A$, we denote by $\D(\A)$ the enhancement of the triangulated category $D(\A)$.

\begin{example}
  As a particularly nice example, one should keep in mind the dg-category $\CH{A}$ for $A$ a commutative $k$-algebra, where $k$ is a commutative ring; in this case, $H^0(\CH{A})$ is equivalent to the usual homotopy category, $K(A)$, $D(\CH{A})$ is obtained by inverting quasi-isomorphisms and is equivalent to the usual derived category $D(\Mod{A})$, which has an enhancement $\D(\CH{A})$ by \cite{Lunts Orlov}.
\end{example}
Of particular interest here is the subcategory $\RHomc{\D(A),\D(B)}$, which if we view the objects of $\RHom{\D(\A),\D(\B)}$ as dg-functors $F \colon \D(\A) \to \D(\B)$, is the subcategory of functors that induce coproduct preserving triangulated functors
$$H^0(F) \colon D(\A) \to D(\B).$$
As a corollary to the main theorem, To\"en proves the following, which goes by the name of Derived Morita Theory.
\begin{corollary}[{\cite[7.6]{Toen}}] \label{corollary: Toen}
  Given two dg categories $\A$ and $\B$, there is an isomorphism in $\Ho{\dgcat{k}}$
  $$\RHomc{\D(\A), \D(\B)} \cong \D(\A^\opp \otimes \B).$$
\end{corollary}

%Should say something about sorting out the homological conditions.
\section{Further Questions}
The statement of our theorem specifies that given two noncommutative projective schemes, $X$ and $Y$, and an equivalence of their derived categories, $F \colon D(\Qcoh{X}) \to D(\Qcoh{Y})$, there exists a quasi-equivalence $G \colon \D(\Qcoh{X}) \to \D(\Qcoh{Y})$, which in turn corresponds to a bimodule $P \in \D(\A^\opp \otimes \B)$ such that $H^0(G) = \Phi_P$.

The first obvious question is
\begin{question}
  What is a tasty pair, $A$ and $B$, such that $X$ and $Y$ are derived equivalent?
\end{question}
It is known that noncommutative projective curves are actually commutative (see \cite{AS}), so standard techniques from the commutative situation can be used to produce equivalences.
Similarly, for $A$ and $B$ commutative, one can produce derived equivalences by identifying the noncommutative projective schemes with the commutative projective schemes, and our result coincides with To\"en's.
However, we don't yet know any other examples of derived equivalent noncommutative projective schemes.

The next natural question is also obvious:
\begin{question}
  Given a bimodule $P$, when does $P$ induce an equivalence of the associated derived categories?
\end{question}
Further still is the less precise range of questions, generally expressed as
\begin{question}
  What properties/invariants are preserved by such a Morita equivalence?
\end{question}
Obvious answers, as is the case for commutative rings, should include purely categorical properties/invariants.
%TODO: Maybe add something more precise here?

In a different direction, one could also investigate various invariants such as Hochschild (co)homology.
Work has been done in this direction for general dg-categories (see, e.g., \cite{Kel06}).
\section{Appendix: Proofs}
\begin{proof}[Proof of Theorem~\ref{thm: Morita}]
  If $L \colon \Mod{A} \to \Mod{B}$ is an equivalence, then $L$ is fully faithful with fully faithful right adjoint, $R \colon \Mod{B} \to \Mod{A}$.
  We have an isomorphism $\varphi \colon A \to \Mod{A}(A,A)$ given by
  $$\varphi_a(a^\prime) = aa^\prime,$$
  hence taking $P = L(A)$ it's clear that the isomorphism of abelian groups
  $$A \cong \Mod{A}(A,A) \cong \Mod{B}(LA, LA) = \Mod{B}(P,P)$$
  that the action
  $$p \cdot a = L(\varphi_a)(p)$$
  endows $P$ with the structure of a right $A$-module which is compatible with the left $B$-module structure in the sense that
  $$(b\cdot p)\cdot a = L(\varphi_a)(b \cdot p) = b \cdot L(\varphi_a)(p) = b \cdot (p \cdot a).$$
  Similarly, if we define $Q = R(B)$, the isomorphism of abelian groups
  $$B \cong \Mod{B}(B,B) \cong \Mod{A}(RB,RB) = \Mod{A}(Q,Q)$$
  endows $Q$ with the structure of an $A$-$B$-bimodule.

  The functor
  $$P \otimes_A - \colon \Mod{A} \to \Mod{B}$$
  has right adjoint
  $$\Mod{B}(P, -) \colon \Mod{B} \to \Mod{A}$$
  and hence we obtain the following natural isomorphisms
  \begin{eqnarray*}
    \Mod{A}(M, \Mod{B}(P,N)) &\cong& \Mod{A}(M, \Mod{A}(A,RN))\\
    &\cong& \Mod{A}(M, RN)
  \end{eqnarray*}
  and
  \begin{eqnarray*}
    \Mod{B}(P \otimes_A M, N) &\cong& \Mod{A}(M, \Mod{B}(P,N))\\
    &\cong& \Mod{A}(M, \Mod{A}(A, RN))\\
    &\cong& \Mod{A}(M,RN)\\
    &\cong& \Mod{B}(LA, N),
  \end{eqnarray*}
  giving  $L \cong P \otimes_A -$ and $R \cong \Mod{B}(P, -)$ by Yoneda's Lemma.
  It's now immediate that
  $$P \otimes_A Q \cong LQ = LRB \cong B$$
  and
  \begin{eqnarray*}
    Q \otimes_B P &\cong& RL(Q \otimes_BP)\\
    &\cong& R(P \otimes_A (Q \otimes_B P))\\
    &\cong& R((P \otimes_A Q) \otimes_B P)\\
    &\cong& R(B \otimes_B P)\\
    &\cong& R(P)\\
    &=& RLA \cong A.
  \end{eqnarray*}

  Conversely, assume that $P$ and $Q$ are as in the statement.
  If we define $$L = P \otimes_A -\ \text{and}\ R = Q \otimes_B -,$$ then we obtain the unit and counit of adjunction from
  $$RL(M) = Q \otimes_B (P \otimes_A M) \cong (Q \otimes_B P) \otimes_A M \cong A \otimes_A M \cong M$$
  and
  $$LR(N) = P \otimes_A (Q \otimes_B N) \cong (P \otimes_A Q) \otimes_B N \cong B \otimes_B N \cong N.$$
  This establishes that $R,L$ give the desired equivalence.
\end{proof}

\begin{proof}[Proof of Lemma~\ref{lem: endomorphisms of identity functor are the center}]
  The data of a natural transformation $\eta \in \End_{\Mod{A}}(\id_{\Mod{A}},\id_{\Mod{A}})$ is just the data of a collection of morphisms $\eta_M$ for each object $M$ of $\Mod{A}$ such that all diagrams
$$\begin{tikzcd}
  M \arrow{r}{\eta_M}\arrow{d}{f} & M\arrow{d}{f}\\
  M^\prime \arrow{r}{\eta_{M^\prime}} & M^\prime
\end{tikzcd}$$
commute.
We note that by choosing a morphism $f \colon A \to M$ such that $f(1_A) = m$ we have
\begin{equation}\label{eqn: eta determined by 1_A}
  \eta_M(m) = \eta_M \circ f(1_A) = f \circ \eta_A(1_A) = \eta_A(1_A) f(1_A) = \eta_A(1_A) m
\end{equation}
by the $A$-linearity of $f$.
This implies that $\eta_M$ is determined by $\eta_A(1_A)$.

Taking $M = A$ we consider two interesting types of morphisms in $\Mod{A}(A,A)$:
For any $a \in A$ there is the left multiplication morphism, $\varphi_a(a^\prime) = aa^\prime$, and also the right multiplication morphism
$$\psi_a(a^\prime) = a^\prime a = (a \cdot \varphi_{1_A})(a^\prime)$$
coming from the left $A$-module structure on $\Mod{A}(A,A)$
$$(a \cdot f)(a^\prime) := f(a^\prime a).$$
Thus for any $a \in A$ we have $\varphi_a(1_A) = a = \psi_a(1_A)$
and so it follows from (\ref{eqn: eta determined by 1_A}) that
$$a \eta_A(1_A) = \varphi_a(\eta_A(1_A)) = \eta_A(a) = \psi_a(\eta_A(1_A))= \eta_A(1_A)a.$$
The mapping $\eta \mapsto \eta_A(1_A)$ is by definition $\Z$-linear and respects the ring structures since
\begin{eqnarray*}
  \nu_A(1_A) \eta_A(1_A) &=& \eta_A(1_A) \nu_A(1_A)\\
  &=& \varphi_{\eta_A(1_A)} \circ \nu_A(1_A)\\
  &=& \nu_A \circ \varphi_{\eta_A(1_A)}(1_A)\\
  &=& \nu_A \circ \eta_A(1_A)
\end{eqnarray*}
holds for any other endomorphism, $\nu$.

For the reverse mapping, we note that it's clear that an element $z \in Z(A)$ determines for each module $M$ a morphism $\eta_M(m) = zm$ and this is natural in $M$ since for any $f \in \Mod{A}(M,M^\prime)$ we have
$$\eta_M \circ f(m) = zf(m) = f(zm) = f \circ \eta_M(m).$$
This mapping gives the inverse to the morphism of rings constructed above.
\end{proof}

\begin{proof}[Proof of Corollary~\ref{cor: morita implies isomorphic centers}]
  Denote by $\varepsilon \colon \id_{\Mod{A}} \to RL$ and $\eta \colon LR \to \id_{\Mod{B}}$ the unit and counit of the equivalence.
  Since the equivalence is necessarily $\Ab$-enriched, we have $\Ab$-enriched functors
  $$\begin{tikzcd}[row sep=tiny]
    \Abcat(\Mod{A},\Mod{A}) \arrow{r} & \Abcat(\Mod{B},\Mod{B})\\
    F \arrow[mapsto]{r} & LFR\\
    \varphi \colon F_1 \to F_2 \arrow[mapsto]{r} & L(\varphi_{R(-)}) \colon LF_1R \to LF_2R
  \end{tikzcd}$$
  and
  $$\begin{tikzcd}[row sep=tiny]
    \Abcat(\Mod{B},\Mod{B}) \arrow{r} & \Abcat(\Mod{A},\Mod{A})\\
    G \arrow[mapsto]{r} & RGL\\
    \psi \colon G_1 \to G_2 \arrow[mapsto]{r} & R(\varphi_{L(-)}) \colon RG_1L \to RG_2L
  \end{tikzcd}$$
  which we show give an equivalence of categories.
  We handle one composition, and note that the other composition is the same argument, swapping $\eta$ for $\varepsilon$.
  Given an endomorphism $F$ of $\Mod{A}$, for any object $M$ of $\Mod{A}$, it's clear that we have the isomorphism
  $$FM \overset{\varepsilon_{FM}}\longrightarrow RLFM \overset{RLF(\varepsilon_M)}\longrightarrow RLFRLM$$
  and for any morphism $\varphi \colon F_1 \to F_2$ the diagram
  $$\begin{tikzcd}[column sep=huge]
    F_1M \arrow{d}{\varepsilon_{F_1M}}\arrow{r}{\varphi_M} & F_2M\arrow{d}{\varepsilon_{F_2M}}\\
    RLF_1M \arrow{d}{RLF_1(\varepsilon_{M})}\arrow{r}{RL(\varphi_M)} & RLF_2M\arrow{d}{RLF_2(\varepsilon_M)}\\
    RLF_1RLM \arrow{r}{RL(\varphi_{RLM})} & RLF_2RLM
  \end{tikzcd}$$
  commutes since the top square commutes by naturality of $\varepsilon$ and the bottom square is just $RL$ applied to the top square.
  Taking $F_1 = \id_{\Mod{A}} = F_2$, the equivalence gives the desired isomorphism
  $$Z(A) \cong \End_{\Mod{A}}(\id_{\Mod{A}}, \id_{\Mod{A}}) \cong \End_{\Mod{B}}(\id_{\Mod{B}},\id_{\Mod{B}}) \cong Z(B).$$
\end{proof}

\begin{proof}[Proof of Corollary~\ref{cor: morita example}]
  View $P = A^n$ as column vectors equipped with the natural $M_n(A)$-$A$-bimodule structure by matrix multiplication on the left, and scalar multiplication on the right, and $Q = A^n$ as row vectors equipped with the natural $A$-$M_n(A)$-bimodule structure by scalar multiplication on the left and matrix multiplication on the right.
  The two morphisms
  $$\begin{tikzcd}[row sep=tiny]
    P \otimes_A Q \arrow{r} & A\\
    p \otimes q \arrow[mapsto]{r} & \sum p_i q_i
  \end{tikzcd}$$
  and
  $$\begin{tikzcd}[row sep=tiny]
    Q \otimes_{M_n(A)} P \arrow{r} & M_n(A)\\
    q \otimes p \arrow[mapsto]{r} & (q_ip_j)
  \end{tikzcd}$$
  give the morphita equivalence.
\end{proof}

%%%%%%%%%%%%%%%%%%%%%%%%%%%%%%%%%%%%%%%%%%%%%%%%%%
%                                                %
%                 Bibliography                   %
%                                                %
%%%%%%%%%%%%%%%%%%%%%%%%%%%%%%%%%%%%%%%%%%%%%%%%%%

\begin{thebibliography}{99}
\bibitem[AS]{AS} M. Artin and W. Schelter, Graded algebras of dimension 3, Adv. Math. \textbf{66} (1987), 172-216.
  
\bibitem[AZ94]{AZ}
  M. Artin, J. J. Zhang, \emph{Noncommutative projective schemes}, Adv. Math. 109 (1994), no. 2, pp. 228--287.
\bibitem[AS95]{AS}
  Artin, M., and Stafford, J.T.. \emph{Noncommutative graded domains with quadratic..}, Inventiones mathematicae 122.2 (1995): 231-276
\bibitem[BvdB03]{BV}
  A. Bondal, M. van den Bergh, \emph{Generators and representability of functors in commutative and noncommutative geometry}, Mosc. Math. J., 3 (2003), no. 1, pp. 1--36, 258.
  
\bibitem[CS15]{CS}
  A. Canonaco, P. Stellari, \emph{Internal Homs via extensions of dg functors}, Adv. Math. 277 (2015), 100-123.

\bibitem[Dyc11]{Dyckerhoff}
  Dyckerhoff, Tobias. Compact generators in categories of matrix factorizations. Duke Math. J. 159 (2011), no. 2, 223--274.

\bibitem[Ga73]{DCA}
  Gabriel, P.. "Des catégories abéliennes." Bulletin de la Société Mathématique de France 90 (1962): 323-448.
\bibitem[Hov01]{Hov01}{}
  M. Hovey, \emph{Model category structures on chain complexes of sheaves}, Trans. Amer. Math. Soc. 353 (2001), no. 6, 2441--2457.

\bibitem[Kel95]{Kel95}
  B. Keller, \emph{Deriving DG categories}, Ann. Sci. \'{E}cole Norm. Sup. \textbf{27} (1995), 63-102.

\bibitem[Kel06]{Kel06}
  B. Keller, \emph{On differential graded categories}, International Congress of Mathematicians, Vol. II, 151–190, Eur. Math. Soc., Zu ̈rich, 2006.
  
\bibitem[LO10]{Lunts Orlov}
  Lunts, Valery A.; Orlov, Dmitri O. {\em Uniqueness of enhancement for triangulated categories}. J. Amer. Math. Soc. 23 (2010), no. 3, 853--908.

\bibitem[Orl97]{Orlov}
  Orlov, D. O. {\em Equivalences of derived categories and K3 surfaces}. Algebraic geometry, 7. J.
  Math. Sci. (New York) 84 (1997), no. 5, 1361–1381.
\bibitem[Ric89]{JR89}
  Rickard, J. {\em Morita Theory for Derived Categories}. Journal of the London Mathematical Society \textbf{2} (1989), no. 3, 436.

\bibitem[Spa88]{Spaltenstein}
  Spaltenstein, N. {\em Resolutions of unbounded complexes}. Compositio Math. 65 (1988), no. 2, 121--154.

\bibitem[Tab05]{Tab05}
  G. Tabuada, \emph{Une structure de cat\'{e}gorie de mod\`{e}les de Quillen sur la cat\'{e}gorie des dg-cat\`{e}gories}, Comptes Rendus de l'Academie de Sciences de Paris \textbf{340} (2005), 15--19.
  
\bibitem[To\"e07]{Toen}
  B. To\"en. \emph{The homotopy theory of dg-categories and derived Morita theory}. Invent. Math. 167 (2007), no. 3, 615--667. 
  
\bibitem[VdB97]{VdB}
  M. van den Bergh, \emph{Existence theorems for dualizing complexes over non-commutative graded and filtered rings}, J. Algebra 195 (1997), no. 2, pp. 662--679.

\end{thebibliography}
\end{document}
