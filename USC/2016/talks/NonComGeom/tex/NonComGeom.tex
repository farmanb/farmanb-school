\documentclass{beamer}

\mode<presentation> {
	\usetheme{PaloAlto}
}

%%
\makeatletter
\setbeamertemplate{subsubsection in sidebar}
  {\vspace*{-\baselineskip}}
\setbeamertemplate{subsubsection in sidebar shaded}
  {\vspace*{-\baselineskip}}
\makeatother
%%

%%
\setbeamertemplate{theorems}[numbered]
%%

\definecolor{Garnet}{RGB}{130,0,20}
\usecolortheme[named=Garnet]{structure}

\logo{\includegraphics[width=1.5cm]{USClogo.png}}

%\setbeamercolor{title}{fg=red!60!black,bg=white!50!black}
%\usecolortheme{beaver}
%\usecolortheme{crane}
\usefonttheme{structuresmallcapsserif}
\usefonttheme[onlysmall]{structurebold}



\usepackage{graphicx}
\usepackage{mathtools}
\usepackage{latexsym}
\usepackage{amsfonts}
\usepackage[only,ninrm,elvrm,twlrm,sixrm,egtrm,tenrm]{rawfonts}
\usepackage{indentfirst}
\usepackage[noend]{algorithmic}
\usepackage{algorithm}
\usepackage{enumerate}
\usepackage{graphicx,psfrag}
\usepackage{epsfig}
%\usepackage[pdflatex]{graphicx}
%\usepackage{epstopdf}
\usepackage{ulem}
\usepackage{animate} %need the animate.sty file
\usepackage{tikz}
\usepackage{amsmath,amsthm,amssymb,amsfonts,enumerate,mymath,mathtools,tikz-cd,mathrsfs}

\newtheorem{thm}{Theorem}
\newtheorem{lem}{Lemma}
\newtheorem{prop}{Proposition}
\theoremstyle{definition}
\newtheorem{defn}{Definition}
\newtheorem{rmk}{Remark}

\newcommand{\A}{\mathscr{A}}
\renewcommand{\C}{\mathscr{C}}

\newcommand*{\defeq}{\mathrel{\vcenter{\baselineskip0.5ex \lineskiplimit0pt
                     \hbox{\scriptsize.}\hbox{\scriptsize.}}}%
                     =}
\DeclarePairedDelimiter\ceil{\lceil}{\rceil}
\DeclarePairedDelimiter\floor{\lfloor}{\rfloor}

\input epsf



\usepackage[english]{babel}
% or whatever

\usepackage[latin1]{inputenc}
% or whatever

\usepackage{times}
\usepackage[T1]{fontenc}
% Or whatever. Note that the encoding and the font should match. If T1
% does not look nice, try deleting the line with the fontenc.

\title % (optional, use only with long paper titles)
{Noncommutative Projective Schemes}


\author[Farman]
{Blake Farman~\inst{1}}

\institute[USC]{
\inst{1}
University of South Carolina, Columbia, SC USA}
%\inst{2}
%East Carolina University, Greenville, NC USA\\
%\inst{3}
%University of Johannesburg, Auckland Park, South Africa}

\date[May 28, 2015]
{Temple Graduate Student Conference in Algebra, Geometry, and Topology}

%\subject{Irredundant and Mixed Ramsey Numbers}
\setbeamercolor{alerted text}{fg=red!60!black}
\setbeamercolor{block title}{bg=white!50!black,fg=red!60!black}

\begin{document}

\begin{frame}
  \titlepage
\end{frame}

\begin{frame}
  \frametitle{Outline}
  \tableofcontents[pausesections]
\end{frame}




\section{Serre Finiteness}

\begin{frame}{Commutative Case}
  \begin{thm}[Serre]
    Let $X$ be a projective scheme over a noetherian ring $A$, and let $\mathcal{O}_X(1)$ be a very ample invertible sheaf on $X$ over $\operatorname{Spec} A$.
    Let $\mathscr{F}$ be a coherent sheaf on $X$. Then:
  \end{thm}
\end{frame}

\begin{frame}{Commutative Case}
  \setcounter{thm}{0}
  \begin{thm}[Serre]
    Let $X$ be a projective scheme over a noetherian ring $A$, and let $\mathcal{O}_X(1)$ be a very ample invertible sheaf on $X$ over $\operatorname{Spec} A$.
    Let $\mathscr{F}$ be a coherent sheaf on $X$. Then:
    \begin{enumerate}[(i)]
    \item
      for each $0 \leq i$, $H^i(X,\mathscr{F})$ is a finitely generated $A$-module;
    \end{enumerate}
  \end{thm}
\end{frame}

\begin{frame}{Commutative Case}
  \setcounter{thm}{0}
  \begin{thm}[Serre]
    Let $X$ be a projective scheme over a noetherian ring $A$, and let $\mathcal{O}_X(1)$ be a very ample invertible sheaf on $X$ over $\operatorname{Spec} A$.
    Let $\mathscr{F}$ be a coherent sheaf on $X$. Then:
    \begin{enumerate}[(i)]
    \item
      for each $0 \leq i$, $H^i(X,\mathscr{F})$ is a finitely generated $A$-module;
    \item
      there is an integer $n_0$, depending on $\mathscr{F}$ such that for each $0 < i$ and each $n_0 \leq n$, $H^i(X, \mathscr{F}(n)) = 0$.
    \end{enumerate}
  \end{thm}
\end{frame}

\begin{frame}{Noncommutative Case}
  \setcounter{thm}{1}
  \begin{thm}[Artin-Zhang]
    Let $A$ be a right noetherian $\Z_{\geq 0}$-graded algebra over a commutative noetherian ring $k$ satisfying $\chi$ and let $\pi(M)$ be an object of $\qgr{A}$.
  \end{thm}
\end{frame}

\begin{frame}{Noncommutative Case}
  \setcounter{thm}{1}
  \begin{thm}[Artin-Zhang]
          Let $A$ be a right noetherian $\Z_{\geq 0}$-graded algebra over a commutative noetherian ring $k$ satisfying $\chi$ and let $\pi(M)$ be an object of $\qgr{A}$.
      Then
    \begin{enumerate}[(i)]
    \item
      \begin{center}
        \begin{flushleft}
          (H4) for every $0 \leq j$, $H^j(\pi(M))$ is a finite right $A_0$-module, and\\
        \end{flushleft}
      \end{center}
    \end{enumerate}
  \end{thm}
\end{frame}

\begin{frame}{Noncommutative Case}
  \setcounter{thm}{1}
  \begin{thm}[Artin-Zhang]
          Let $A$ be a right noetherian $\Z_{\geq 0}$-graded algebra over a commutative noetherian ring $k$ satisfying $\chi$ and let $\pi(M)$ be an object of $\qgr{A}$.
      Then
    \begin{enumerate}[(i)]
    \item
      \begin{center}
        \begin{flushleft}
          (H4) for every $0 \leq j$, $H^j(\pi(M))$ is a finite right $A_0$-module, and\\
          (H5) for every $1 \leq j$, $\underline{H}^j(\pi(M))$ is right bounded; i.e., there is an integer $d_0$ such that for all $d_0 \leq d$, $H^{j}(\pi(M)[d]) = 0$.
        \end{flushleft}
      \end{center}
    \end{enumerate}
  \end{thm}
\end{frame}

\begin{frame}{Noncommutative Case}
  \setcounter{thm}{1}
  \begin{thm}[Artin-Zhang]
    Let $A$ be a right noetherian $\Z_{\geq 0}$-graded algebra over a commutative noetherian ring $k$ satisfying $\chi$ and let $\pi(M)$ be an object of $\qgr{A}$.
    Then
    \begin{enumerate}[(i)]
    \item
      \begin{center}
        \begin{flushleft}
          (H4) for every $0 \leq j$, $H^j(\pi(M))$ is a finite right $A_0$-module, and\\
          (H5) for every $1 \leq j$, $\underline{H}^j(\pi(M))$ is right bounded; i.e., there is an integer $d_0$ such that for all $d_0 \leq d$, $H^{j}(\pi(M)[d]) = 0$.
        \end{flushleft}
      \end{center}
    \item
      Conversely, if $A$ satisfies $\chi_1$ and if (H4) and (H5) hold for every $\pi(M) \in \qgr{A}$, then $A$ satisfies $\chi$.
    \end{enumerate}
  \end{thm}
\end{frame}

\section{Quotient Categories}

\begin{frame}{Preliminaries}
  Let $k$ be a Noetherian commutative ring, $A$ a $\Z_{\geq 0}$-graded right Noetherian algebra over $k$.
  Denote by $\Gr{A}$ (resp. $\gr{A}$) the category of graded right $A$-modules (resp. finite) with morphisms
  $$\Hom{\Gr{A}}{M,N} = \left\{f \in \Hom{A}{M,N} \;\middle\vert\; f(M_d) \subseteq N_d\right\}.$$
\end{frame}

\begin{frame}[fragile]{Preliminaries}
  $\Gr{A}$ is a Grothendieck category with injective envelopes.
\end{frame}

\begin{frame}[fragile]{Preliminaries}
  $\Gr{A}$ is a Grothendieck category with injective envelopes.
  That is,
  \begin{itemize}
  \item
    $\Gr{A}$ is abelian (zero object, finite biproducts, all kernels and cokernels, monics and epics are normal--every monic is a kernel and every epic is a cokernel),
  \end{itemize}
\end{frame}

\begin{frame}[fragile]{Preliminaries}
  $\Gr{A}$ is a Grothendieck category with injective envelopes.
  That is,
  \begin{itemize}
  \item
    $\Gr{A}$ is abelian (zero object, finite biproducts, all kernels and cokernels, monics and epics are normal--every monic is a kernel and every epic is a cokernel),
  \item
    every family of objects has a coproduct,
  \end{itemize}
\end{frame}

\begin{frame}[fragile]{Preliminaries}
  $\Gr{A}$ is a Grothendieck category with injective envelopes.
  That is,
  \begin{itemize}
  \item
    $\Gr{A}$ is abelian (zero object, finite biproducts, all kernels and cokernels, monics and epics are normal--every monic is a kernel and every epic is a cokernel),
  \item
    every family of objects has a coproduct,
  \item
    filtered colimits are exact,
  \end{itemize}
\end{frame}

\begin{frame}[fragile]{Preliminaries}
  $\Gr{A}$ is a Grothendieck category with injective envelopes.
  That is,
  \begin{itemize}
  \item
    $\Gr{A}$ is abelian (zero object, finite biproducts, all kernels and cokernels, monics and epics are normal--every monic is a kernel and every epic is a cokernel),
  \item
    every family of objects has a coproduct,
  \item
    filtered colimits are exact,
  \item
    $\Gr{A}$ has a generator: the functor $h^A \colon \Gr{A} \rightarrow \Sets$ is faithful; for any morphism $M \rightarrow N$ the morphism
    $$\begin{tikzcd}
      \Hom{\Gr{A}}{M, N} \arrow{r}& \Hom{\Sets}{h^A(M), h^A(N)}
    \end{tikzcd}$$
    is a monomorphism of sets, and
  \end{itemize}
\end{frame}

\begin{frame}[fragile]{Preliminaries}
  $\Gr{A}$ is a Grothendieck category with injective envelopes.
  That is,
  \begin{itemize}
  \item
    $\Gr{A}$ is abelian (zero object, finite biproducts, all kernels and cokernels, monics and epics are normal--every monic is a kernel and every epic is a cokernel),
  \item
    every family of objects has a coproduct,
  \item
    filtered colimits are exact,
  \item
    $\Gr{A}$ has a generator: the functor $h^A \colon \Gr{A} \rightarrow \Sets$ is faithful; for any morphism $M \rightarrow N$ the morphism
    $$\begin{tikzcd}
      \Hom{\Gr{A}}{M, N} \arrow{r}& \Hom{\Sets}{h^A(M), h^A(N)}
    \end{tikzcd}$$
    is a monomorphism of sets, and
  \item
    every object has an injective enevelope.
  \end{itemize}
\end{frame}

\begin{frame}[fragile]{Preliminaries}
  \begin{defn}
    A full subcategory, $\A$, of an abelian category $\C$ is called a Serre (or \'{e}paisse/thick/dense) subcategory if for any short exact sequence
    $$\begin{tikzcd}
      0 \arrow{r} & X^\prime \arrow{r} & X \arrow{r} & X^{\prime\prime} \arrow{r} & 0
    \end{tikzcd}$$
    of $\C$, $X$ is an object of $\A$ if and only if both $X^\prime$ and $X^{\prime\prime}$ are.
  \end{defn}
\end{frame}

\begin{frame}[fragile]{Preliminaries}
  \setcounter{defn}{0}
  \begin{defn}
    A full subcategory, $\A$, of an abelian category $\C$ is called a Serre (or \'{e}paisse/thick/dense) subcategory if for any short exact sequence
    $$\begin{tikzcd}
      0 \arrow{r} & X^\prime \arrow{r} & X \arrow{r} & X^{\prime\prime} \arrow{r} & 0
    \end{tikzcd}$$
    of $\C$, $X$ is an object of $\A$ if and only if both $X^\prime$ and $X^{\prime\prime}$ are.
  \end{defn}
  The full subcategory $\Tors$ (resp. $\tors$) of $\Gr{A}$ (resp. $\gr{A}$) with objects $M$ of $\Gr{A}$ (resp. $\gr{A}$) satisfying
  $$\tau(M) = \left\{m \in M \;\middle\vert\; mA_{\geq s} = 0\ \text{for some}\ s\right\} = M$$
  is a Serre subcategory.
\end{frame}

\begin{frame}[fragile]{Preliminaries}
  Let $M$ and $N$ be objects of $\Gr{A}$.
  Define the category $\mathscr{I}$ with
\end{frame}

\begin{frame}[fragile]{Preliminaries}
  Let $M$ and $N$ be objects of $\Gr{A}$.
  Define the category $\mathscr{I}$ with
  \begin{itemize}
    \item
      objects pairs of subobjects $(M^\prime, N^\prime)$ such that $M/M^\prime$ and $N^\prime$ are torsion and 
  \end{itemize}
\end{frame}

\begin{frame}[fragile]{Preliminaries}
  Let $M$ and $N$ be objects of $\Gr{A}$.
  Define the category $\mathscr{I}$ with
  \begin{itemize}
    \item
      objects pairs of subobjects $(M^\prime, N^\prime)$ such that $M/M^\prime$ and $N^\prime$ are torsion and 
    \item
      a unique morphism $$(M^\prime, N^\prime) \rightarrow (M^{\prime\prime}, N^{\prime\prime})$$ if and only if $M^{\prime\prime} \subseteq M^\prime$ and $N^\prime \subseteq N^{\prime\prime}$.
  \end{itemize}
\end{frame}

\begin{frame}[fragile]{Preliminaries}
  Let $M$ and $N$ be objects of $\Gr{A}$.
  Define the category $\mathscr{I}$ with
  \begin{itemize}
    \item
      objects pairs of subobjects $(M^\prime, N^\prime)$ such that $M/M^\prime$ and $N^\prime$ are torsion and 
    \item
      a unique morphism $$(M^\prime, N^\prime) \rightarrow (M^{\prime\prime}, N^{\prime\prime})$$ if and only if $M^{\prime\prime} \subseteq M^\prime$ and $N^\prime \subseteq N^{\prime\prime}$.
  \end{itemize}
  The category $\mathscr{I}$ is filtered.
\end{frame}

\begin{frame}[fragile]{Definition}
  \begin{defn}
    Define the quotient category, $\QGr{A} = \Gr{A}/\Tors$, to be the category with
  \end{defn}
\end{frame}

\begin{frame}[fragile]{Definition}
  \setcounter{defn}{1}
  \begin{defn}
    Define the quotient category, $\QGr{A} = \Gr{A}/\Tors$, to be the category with
    \begin{itemize}
    \item
      objects the objects of $\Gr{A}$, and
    \end{itemize}
  \end{defn}
\end{frame}

\begin{frame}[fragile]{Definition}
  \setcounter{defn}{1}
  \begin{defn}
    Define the quotient category, $\QGr{A} = \Gr{A}/\Tors$, to be the category with
    \begin{itemize}
    \item
      objects the objects of $\Gr{A}$, and
    \item
      morphisms defined by the filtered colimit
      $$\Hom{\QGr{A}}{M,N} = \operatorname{colim}_{\mathscr{I}} \Hom{\Gr{A}}{M^\prime, N/N^\prime}.$$
    \end{itemize}
  \end{defn}
\end{frame}

\begin{frame}[fragile]{Definition}
  \setcounter{defn}{1}
  \begin{defn}
    Define the quotient category, $\QGr{A} = \Gr{A}/\Tors$, to be the category with
    \begin{itemize}
    \item
      objects the objects of $\Gr{A}$, and
    \item
      morphisms defined by the filtered colimit
      $$\Hom{\QGr{A}}{M,N} = \operatorname{colim}_{\mathscr{I}} \Hom{\Gr{A}}{M^\prime, N/N^\prime}.$$
    \end{itemize}
  \end{defn}
  \begin{itemize}
  \item
    $\QGr{A}$ is abelian.
  \end{itemize}
\end{frame}

\begin{frame}[fragile]{Definition}
  \setcounter{defn}{1}
  \begin{defn}
    Define the quotient category, $\QGr{A} = \Gr{A}/\Tors$, to be the category with
    \begin{itemize}
    \item
      objects the objects of $\Gr{A}$, and
    \item
      morphisms defined by the filtered colimit
      $$\Hom{\QGr{A}}{M,N} = \operatorname{colim}_{\mathscr{I}} \Hom{\Gr{A}}{M^\prime, N/N^\prime}.$$
    \end{itemize}
  \end{defn}
  \begin{itemize}
  \item
    $\QGr{A}$ is abelian.
  \item
    There is a functor $\pi : \Gr{A} \rightarrow \QGr{A}$ that is the identity on objects and sends a morphism $f \in \Hom{\Gr{A}}{M,N}$ to its image, $\pi(f)$, in the colimit.
  \end{itemize}
\end{frame}

\begin{frame}[fragile]{Definition}
  \setcounter{defn}{1}
  \begin{defn}
    Define the quotient category, $\QGr{A} = \Gr{A}/\Tors$, to be the category with
    \begin{itemize}
    \item
      objects the objects of $\Gr{A}$, and
    \item
      morphisms defined by the filtered colimit
      $$\Hom{\QGr{A}}{M,N} = \operatorname{colim}_{\mathscr{I}} \Hom{\Gr{A}}{M^\prime, N/N^\prime}.$$
    \end{itemize}
  \end{defn}
  \begin{itemize}
  \item
    $\QGr{A}$ is abelian.
  \item
    There is a functor $\pi : \Gr{A} \rightarrow \QGr{A}$ that is the identity on objects and sends a morphism $f \in \Hom{\Gr{A}}{M,N}$ to its image, $\pi(f)$, in the colimit.
  \item
    $\qgr{A}$ is defined analogously.
  \end{itemize}
\end{frame}

\begin{frame}{Morphisms}
  In $\Gr{A}$, we have a somewhat more explicit formulation of the $\operatorname{Hom}$-sets:
\end{frame}

\begin{frame}{Morphisms}
  In $\Gr{A}$, we have a somewhat more explicit formulation of the $\operatorname{Hom}$-sets:
  \begin{itemize}
    \item
      Given two objects $M$, $N$ of $\Gr{A}$, 
      $$\Hom{\QGr{A}}{\pi(M), \pi(N)} = \operatorname{colim}_{M^\prime}\Hom{\Gr{A}}{M^\prime, N/\tau(N)}.$$
  \end{itemize}
\end{frame}

\begin{frame}{Morphisms}
  In $\Gr{A}$, we have a somewhat more explicit formulation of the $\operatorname{Hom}$-sets:
  \begin{itemize}
    \item
      Given two objects $M$, $N$ of $\Gr{A}$, 
      $$\Hom{\QGr{A}}{\pi(M), \pi(N)} = \operatorname{colim}_{M^\prime}\Hom{\Gr{A}}{M^\prime, N/\tau(N)}.$$
    \item
      If in addition $M$ is an object of $\gr{A}$, then 
      $$\Hom{\QGr{A}}{\pi(M), \pi(N)} = \lim_{n \rightarrow \infty} \Hom{\Gr{A}}{M_{\geq n}, N}$$
      where 
      $$M_{\geq n} = \bigoplus_{d \geq n} M_d.$$
  \end{itemize}
\end{frame}

\begin{frame}[fragile]{Properties of $\pi$}
  \begin{itemize}
  \item
    given an exact sequence
    $$\begin{tikzcd}
      0 \arrow{r} & K \arrow{r}{\ker{f}} & M \arrow{r}{f} & N \arrow{r}{\coker{f}} & C \arrow{r} & 0.
    \end{tikzcd}$$
  \end{itemize}
\end{frame}

\begin{frame}[fragile]{Properties of $\pi$}
  \begin{itemize}
  \item
    given an exact sequence
    $$\begin{tikzcd}
      0 \arrow{r} & K \arrow{r}{\ker{f}} & M \arrow{r}{f} & N \arrow{r}{\coker{f}} & C \arrow{r} & 0.
    \end{tikzcd}$$
    \begin{enumerate}[(i)]
    \item
      $\pi(f) = 0$ if and only if $f(M) \cong M/K$ is torsion,
    \end{enumerate}
  \end{itemize}
\end{frame}

\begin{frame}[fragile]{Properties of $\pi$}
  \begin{itemize}
  \item
    given an exact sequence
    $$\begin{tikzcd}
      0 \arrow{r} & K \arrow{r}{\ker{f}} & M \arrow{r}{f} & N \arrow{r}{\coker{f}} & C \arrow{r} & 0.
    \end{tikzcd}$$
    \begin{enumerate}[(i)]
    \item
      $\pi(f) = 0$ if and only if $f(M) \cong M/K$ is torsion,
    \item
      $\pi(f)$ is a monomorphism if and only if $K$ is torsion, 
    \end{enumerate}
  \end{itemize}
\end{frame}

\begin{frame}[fragile]{Properties of $\pi$}
  \begin{itemize}
  \item
    given an exact sequence
    $$\begin{tikzcd}
      0 \arrow{r} & K \arrow{r}{\ker{f}} & M \arrow{r}{f} & N \arrow{r}{\coker{f}} & C \arrow{r} & 0.
    \end{tikzcd}$$
    \begin{enumerate}[(i)]
    \item
      $\pi(f) = 0$ if and only if $f(M) \cong M/K$ is torsion,
    \item
      $\pi(f)$ is a monomorphism if and only if $K$ is torsion, 
    \item
      $\pi(f)$ is an epimorphism if and only if $C$ is torsion,
    \end{enumerate}
  \end{itemize}
\end{frame}

\begin{frame}[fragile]{Properties of $\pi$}
  \begin{itemize}
  \item
    given an exact sequence
    $$\begin{tikzcd}
      0 \arrow{r} & K \arrow{r}{\ker{f}} & M \arrow{r}{f} & N \arrow{r}{\coker{f}} & C \arrow{r} & 0.
    \end{tikzcd}$$
    \begin{enumerate}[(i)]
    \item
      $\pi(f) = 0$ if and only if $f(M) \cong M/K$ is torsion,
    \item
      $\pi(f)$ is a monomorphism if and only if $K$ is torsion, 
    \item
      $\pi(f)$ is an epimorphism if and only if $C$ is torsion,
    \end{enumerate}
  \item
    $\pi$ is exact and admits a fully faithful adjoint, $\omega : \QGr{A} \rightarrow \Gr{A}$,
  \end{itemize}
\end{frame}

\begin{frame}[fragile]{Properties of $\pi$}
  \begin{itemize}
  \item
    given an exact sequence
    $$\begin{tikzcd}
      0 \arrow{r} & K \arrow{r}{\ker{f}} & M \arrow{r}{f} & N \arrow{r}{\coker{f}} & C \arrow{r} & 0.
    \end{tikzcd}$$
    \begin{enumerate}[(i)]
    \item
      $\pi(f) = 0$ if and only if $f(M) \cong M/K$ is torsion,
    \item
      $\pi(f)$ is a monomorphism if and only if $K$ is torsion, 
    \item
      $\pi(f)$ is an epimorphism if and only if $C$ is torsion,
    \end{enumerate}
  \item
    $\pi$ is exact and admits a fully faithful adjoint, $\omega : \QGr{A} \rightarrow \Gr{A}$,
  \item
    $\pi$ preserves injectives.
  \end{itemize}
\end{frame}

\section{Cohomology}

\begin{frame}[fragile]{Closed Objects}
  \begin{defn}
    We say an object $M$ of $\Gr{A}$ is $\Tors$-closed if $M$ is torsion-free and any short exact sequence 
      $$\begin{tikzcd}
          0 \arrow{r} & M \arrow{r}{f} & X \arrow{r}{\coker{f}} & X/M \arrow{r} & 0
        \end{tikzcd}$$
      with $X/M$ torsion splits.
  \end{defn}
\end{frame}

\begin{frame}[fragile]{Closed Objects}
  \begin{defn}
    We say an object $M$ of $\Gr{A}$ is $\Tors$-closed if $M$ is torsion-free and any short exact sequence 
      $$\begin{tikzcd}
          0 \arrow{r} & M \arrow{r}{f} & X \arrow{r}{\coker{f}} & X/M \arrow{r} & 0
        \end{tikzcd}$$
      with $X/M$ torsion splits.
  \end{defn}
    \begin{rmk}
    It's immediate that every torsion-free injective is $\Tors$-closed.
  \end{rmk}
\end{frame}

\begin{frame}[fragile]{Closed Objects}
  \begin{prop}[Gabriel]
    For $M$ an object of $\Gr{A}$, the following are equivalent:
    \begin{enumerate}
    \item
      Any exact sequence
      $$\begin{tikzcd}
          0 \arrow{r} & K \arrow{r}{\ker{f}} & X \arrow{r}{f} & Y \arrow{r}{\coker{f}} & C \arrow{r} & 0
        \end{tikzcd}$$
      with $K$ and $C$ torsion implies $h_M(f) \colon h_M(Y) \cong h_M(X)$,
    \end{enumerate}
  \end{prop}
\end{frame}

\begin{frame}[fragile]{Closed Objects}
  \setcounter{prop}{0}
  \begin{prop}[Gabriel]
    For $M$ an object of $\Gr{A}$, the following are equivalent:
    \begin{enumerate}
    \item
      Any exact sequence
      $$\begin{tikzcd}
          0 \arrow{r} & K \arrow{r}{\ker{f}} & X \arrow{r}{f} & Y \arrow{r}{\coker{f}} & C \arrow{r} & 0
        \end{tikzcd}$$
      with $K$ and $C$ torsion implies $h_M(f) \colon h_M(Y) \cong h_M(X)$,
    \item
      $M$ is $\Tors$-closed,
    \end{enumerate}
  \end{prop}
\end{frame}

\begin{frame}[fragile]{Closed Objects}
  \setcounter{prop}{0}
  \begin{prop}[Gabriel]
    For $M$ an object of $\Gr{A}$, the following are equivalent:
    \begin{enumerate}
    \item
      Any exact sequence
      $$\begin{tikzcd}
          0 \arrow{r} & K \arrow{r}{\ker{f}} & X \arrow{r}{f} & Y \arrow{r}{\coker{f}} & C \arrow{r} & 0
        \end{tikzcd}$$
      with $K$ and $C$ torsion implies $h_M(f) \colon h_M(Y) \cong h_M(X)$,
    \item
      $M$ is $\Tors$-closed,
    \item
      For any object $N$ of $\Gr{A}$
      $$\pi \colon \Hom{\Gr{A}}{N, M} \cong \Hom{\QGr{A}}{\pi(N), \pi(M)}.$$ 
    \end{enumerate}
  \end{prop}
\end{frame}

\begin{frame}{Injective Objects}
  \begin{rmk}
    \begin{enumerate}
    \item
      If $M$ is torsion-free and $i \colon M \rightarrow E(M)$ is an injective envelope, then $E(M)$ is torsion free, hence $\Tors$-closed.
    \end{enumerate}
  \end{rmk}
\end{frame}

\begin{frame}{Injective Objects}
  \setcounter{rmk}{1}
  \begin{rmk}
    \begin{enumerate}
    \item
      If $M$ is torsion-free and $i \colon M \rightarrow E(M)$ is an injective envelope, then $E(M)$ is torsion free, hence $\Tors$-closed.
      In such a case, it can be shown that $\pi(i)$ is an injective envelope.
    \end{enumerate}
  \end{rmk}
\end{frame}

\begin{frame}{Injective Objects}
  \setcounter{rmk}{1}
  \begin{rmk}
    \begin{enumerate}
    \item
      If $M$ is torsion-free and $i \colon M \rightarrow E(M)$ is an injective envelope, then $E(M)$ is torsion free, hence $\Tors$-closed.
      In such a case, it can be shown that $\pi(i)$ is an injective envelope.
      Since $\pi(M) \cong \pi(M/\tau(M))$, it follows that $\QGr{A}$ has injective envelopes.
    \end{enumerate}
  \end{rmk}
\end{frame}

\begin{frame}{Injective Objects}
  \setcounter{rmk}{1}
  \begin{rmk}
    \begin{enumerate}
    \item
      If $M$ is torsion-free and $i \colon M \rightarrow E(M)$ is an injective envelope, then $E(M)$ is torsion free, hence $\Tors$-closed.
      In such a case, it can be shown that $\pi(i)$ is an injective envelope.
      Since $\pi(M) \cong \pi(M/\tau(M))$, it follows that $\QGr{A}$ has injective envelopes.
    \item
      It can be shown (see Artin-Zhang, Prop 2.2) that if $M$ is torsion, then so is $E(M)$.
    \end{enumerate}
  \end{rmk}
\end{frame}

\begin{frame}{Injective Objects}
  \setcounter{rmk}{1}
  \begin{rmk}
    \begin{enumerate}
    \item
      If $M$ is torsion-free and $i \colon M \rightarrow E(M)$ is an injective envelope, then $E(M)$ is torsion free, hence $\Tors$-closed.
      In such a case, it can be shown that $\pi(i)$ is an injective envelope.
      Since $\pi(M) \cong \pi(M/\tau(M))$, it follows that $\QGr{A}$ has injective envelopes.
    \item
      It can be shown (see Artin-Zhang, Prop 2.2) that if $M$ is torsion, then so is $E(M)$.
      In the case that we have an injective object, $Q$, $\tau(Q)$ is injective and gives the decomposition $Q \cong \tau(Q) \oplus Q/\tau(Q) \cong \tau(Q) \oplus \omega\pi(Q)$.
      In fact, it follows that $Q/\tau(Q) \cong \omega\pi(Q)$ is injective.
    \end{enumerate}
  \end{rmk}
\end{frame}

\begin{frame}{Injective Objects}
  \setcounter{rmk}{1}
  \begin{rmk}
    \begin{enumerate}
    \item
      If $M$ is torsion-free and $i \colon M \rightarrow E(M)$ is an injective envelope, then $E(M)$ is torsion free, hence $\Tors$-closed.
      In such a case, it can be shown that $\pi(i)$ is an injective envelope.
      Since $\pi(M) \cong \pi(M/\tau(M))$, it follows that $\QGr{A}$ has injective envelopes.
    \item
      It can be shown (see Artin-Zhang, Prop 2.2) that if $M$ is torsion, then so is $E(M)$.
      In the case that we have an injective object, $Q$, $\tau(Q)$ is injective and gives the decomposition $Q \cong \tau(Q) \oplus Q/\tau(Q) \cong \tau(Q) \oplus \omega\pi(Q)$.
      In fact, it follows that $Q/\tau(Q) \cong \omega\pi(Q)$ is injective.
    \item
      Every injective object of $\QGr{A}$ is isomorphic to $\pi(Q/\tau(Q))$ for some injective object $Q$ of $\Gr{A}$.
    \end{enumerate}
  \end{rmk}
\end{frame}

\begin{frame}{Computing $\operatorname{Ext}$}
  Since $\QGr{A}$ has enough injectives, we can define $\operatorname{Ext}$ for $\QGr{A}$.
  Let's compute $\Ext{i}{\QGr{A}}{\pi(M), \pi(N)}$:
\end{frame}

\begin{frame}[fragile]{Computing $\operatorname{Ext}$}
  Since $\QGr{A}$ has enough injectives, we can define $\operatorname{Ext}$ for $\QGr{A}$.
  Let's compute $\Ext{i}{\QGr{A}}{\pi(M), \pi(N)}$:
  \begin{enumerate}
  \item
    Take an injective resolution $\begin{tikzcd}Q^\cdot : 0 \arrow{r} & N \arrow{r} & Q^0 \arrow{r} & Q^1 \arrow{r} & \cdots \end{tikzcd}$.
  \end{enumerate}
\end{frame}

\begin{frame}[fragile]{Computing $\operatorname{Ext}$}
  Since $\QGr{A}$ has enough injectives, we can define $\operatorname{Ext}$ for $\QGr{A}$.
  Let's compute $\Ext{i}{\QGr{A}}{\pi(M), \pi(N)}$:
  \begin{enumerate}
  \item
    Take an injective resolution $\begin{tikzcd}Q^\cdot : 0 \arrow{r} & N \arrow{r} & Q^0 \arrow{r} & Q^1 \arrow{r} & \cdots \end{tikzcd}$.
  \item
    $\pi(Q^\cdot)$ is an injective resolution of $\pi(N)$ by the comments above, so 
    $$h^i(\Hom{\QGr{A}}{\pi(M), \pi(Q^\cdot)}) \cong \Ext{i}{\QGr{A}}{\pi(M), \pi(N)}$$
  \end{enumerate}
\end{frame}

\begin{frame}[fragile]{Computing $\operatorname{Ext}$}
  Since $\QGr{A}$ has enough injectives, we can define $\operatorname{Ext}$ for $\QGr{A}$.
  Let's compute $\Ext{i}{\QGr{A}}{\pi(M), \pi(N)}$:
  \begin{enumerate}
  \item
    Take an injective resolution $\begin{tikzcd}Q^\cdot : 0 \arrow{r} & N \arrow{r} & Q^0 \arrow{r} & Q^1 \arrow{r} & \cdots \end{tikzcd}$.
  \item
    $\pi(Q^\cdot)$ is an injective resolution of $\pi(N)$ by the comments above, so $$h^i(\Hom{\QGr{A}}{\pi(M), \pi(Q^\cdot)}) \cong \Ext{i}{\QGr{A}}{\pi(M), \pi(N)}$$
  \item
    From the adjunction we get an isomorphism of complexes
    $$\Hom{\QGr{A}}{\pi(M), \pi(Q^\cdot)}) \cong \Hom{\Gr{A}}{M, \omega\pi(Q^\cdot)}$$
    and we see that 
    $$\Ext{i}{\QGr{A}}{\pi(M), \pi(N)} \cong R^i\Hom{\Gr{A}}{M, \omega\pi(N)}$$
  \end{enumerate}
\end{frame}

\begin{frame}{Graded $\operatorname{Hom}$}
  \begin{defn}
    Define the graded modules
  $$\HOM{\Gr{A}}{M,N} = \bigoplus_{d \in \Z} \Hom{\Gr{A}}{M, N[d]}$$ 
  and 
  $$\HOM{\QGr{A}}{\pi(M), \pi(N)} = \bigoplus_{d \in \Z} \Hom{\QGr{A}}{\pi(M), \pi(N)[d]}.$$
  \end{defn}
\end{frame}

\begin{frame}{Graded $\operatorname{Ext}$}
  The right derived functors are
  $$\EXT{i}{\Gr{A}}{M,N} = \bigoplus_{d \in \Z}\Ext{i}{\Gr{A}}{M,N[d]}$$
  and
  $$\EXT{i}{\QGr{A}}{\pi(M),\pi(N)} = \bigoplus_{d \in \Z}\Ext{i}{\QGr{A}}{\pi(M),\pi(N)[d]}.$$
\end{frame}

\begin{frame}{Graded $\operatorname{Ext}$ (cont'd)}
  For $Q^\cdot$ an injective resolution of $N$, 
  \begin{eqnarray*}
    \EXT{i}{\QGr{A}}{\pi(M), \pi(N)} 
    &\cong& h^i\left(\HOM{\Gr{A}}{M, \omega\pi(Q^\cdot)}\right)\\
    &\cong& R^i\HOM{\Gr{A}}{M, \omega\pi(N)}.
  \end{eqnarray*}
\end{frame}

\begin{frame}{Cohomology}
  Define the cohomology functors
  $$H^i(\pi(M)) = \Ext{i}{\QGr{A}}{\pi(A), \pi(M)} \cong h^i(\omega\pi(Q^\cdot))_0$$
  and
  $$\underline{H}^i(\pi(M)) = \bigoplus_{d \in \Z} H^i(\pi(M)[d]) \cong h^i(\omega\pi(Q^\cdot)).$$
\end{frame}



\section{The $\chi$ Condition}

\begin{frame}{Bounded Modules}
  \begin{defn}
    Let $M$ be an object of $\Gr{A}$.
    \begin{enumerate}[(i)]
    \item
      We say $M$ is left bounded if there exists some $\ell$ such that $M_d = 0$ for all $d \leq \ell$.
    \end{enumerate}
  \end{defn}
\end{frame}

\begin{frame}{Bounded Modules}
  \setcounter{defn}{5}
  \begin{defn}
    Let $M$ be an object of $\Gr{A}$.
    \begin{enumerate}[(i)]
    \item
      We say $M$ is left bounded if there exists some $\ell$ such that $M_d = 0$ for all $d \leq \ell$.
    \item
      We say $M$ is right bounded if there exists some $r$ such that $M_d = 0$ for all $r \leq d$.
    \end{enumerate}
  \end{defn}
\end{frame}

\begin{frame}{Bounded Modules}
  \setcounter{defn}{5}
  \begin{defn}
    Let $M$ be an object of $\Gr{A}$.
    \begin{enumerate}[(i)]
    \item
      We say $M$ is left bounded if there exists some $\ell$ such that $M_d = 0$ for all $d \leq \ell$.
      \item
        We say $M$ is right bounded if there exists some $r$ such that $M_d = 0$ for all $r \leq d$.
      \item
        We say $M$ is bounded if it is left and right bounded.
    \end{enumerate}
  \end{defn}
\end{frame}

\begin{frame}{The $\chi$ Condition}
  \begin{defn}
    \begin{enumerate}
    \item
      We say $\chi_i^0(M)$ holds if $\EXT{j}{\Gr{A}}{A_0, M}$ is bounded for all $j \leq i$.
    \end{enumerate}
  \end{defn}
\end{frame}

\begin{frame}{The $\chi$ Condition}
  \setcounter{defn}{6}
  \begin{defn}
    \begin{enumerate}
    \item
      We say $\chi_i^0(M)$ holds if $\EXT{j}{\Gr{A}}{A_0, M}$ is bounded for all $j \leq i$.
    \item
      If $\chi^0_i(M)$ holds for every object $M$ of $\gr{A}$, then we say that $\chi^0_i$ holds for $A$.
    \end{enumerate}
  \end{defn}
\end{frame}

\begin{frame}{The $\chi$ Condition}
  \setcounter{defn}{6}
  \begin{defn}
    \begin{enumerate}
    \item
      We say $\chi_i^0(M)$ holds if $\EXT{j}{\Gr{A}}{A_0, M}$ is bounded for all $j \leq i$.
    \item
      If $\chi^0_i(M)$ holds for every object $M$ of $\gr{A}$, then we say that $\chi^0_i$ holds for $A$.
    \item
      If $\chi^0_i(M)$ holds for $A$ for every $i$, then we say that $\chi^0$ holds for $A$.
    \end{enumerate}
  \end{defn}
\end{frame}

\begin{frame}{The $\chi$ Condition}
  \setcounter{defn}{6}
  \begin{defn}
    \begin{enumerate}
    \item
      We say $\chi_i^0(M)$ holds if $\EXT{j}{\Gr{A}}{A_0, M}$ is bounded for all $j \leq i$.
    \item
      If $\chi^0_i(M)$ holds for every object $M$ of $\gr{A}$, then we say that $\chi^0_i$ holds for $A$.
    \item
      If $\chi^0_i(M)$ holds for $A$ for every $i$, then we say that $\chi^0$ holds for $A$.
    \item
      We say that $\chi_i(M)$ holds for an object of $\Gr{A}$ if for all $d$ and all $j \leq i$, there is an integer $n_0$ such that $\EXT{j}{\Gr{A}}{A/A_{\geq n}, M}_{\geq d}$ is an object of $\gr{A}$ when $n_0 \leq n$.
    \end{enumerate}
  \end{defn}
\end{frame}

\begin{frame}{The $\chi$ Condition}
  \setcounter{defn}{6}
  \begin{defn}
    \begin{enumerate}
    \item
      We say $\chi_i^0(M)$ holds if $\EXT{j}{\Gr{A}}{A_0, M}$ is bounded for all $j \leq i$.
    \item
      If $\chi^0_i(M)$ holds for every object $M$ of $\gr{A}$, then we say that $\chi^0_i$ holds for $A$.
    \item
      If $\chi^0_i(M)$ holds for $A$ for every $i$, then we say that $\chi^0$ holds for $A$.
    \item
      We say that $\chi_i(M)$ holds for an object of $\Gr{A}$ if for all $d$ and all $j \leq i$, there is an integer $n_0$ such that $\EXT{j}{\Gr{A}}{A/A_{\geq n}, M}_{\geq d}$ is an object of $\gr{A}$ when $n_0 \leq n$.
    \item
      If $\chi_i$ holds for every object of $\gr{A}$, then we say that $\chi_i$ holds for $A$.
    \end{enumerate}
  \end{defn}
\end{frame}

\begin{frame}{The $\chi$ Condition}
  \setcounter{defn}{6}
  \begin{defn}
    \begin{enumerate}
    \item
      We say $\chi_i^0(M)$ holds if $\EXT{j}{\Gr{A}}{A_0, M}$ is bounded for all $j \leq i$.
    \item
      If $\chi^0_i(M)$ holds for every object $M$ of $\gr{A}$, then we say that $\chi^0_i$ holds for $A$.
    \item
      If $\chi^0_i(M)$ holds for $A$ for every $i$, then we say that $\chi^0$ holds for $A$.
    \item
      We say that $\chi_i(M)$ holds for an object of $\Gr{A}$ if for all $d$ and all $j \leq i$, there is an integer $n_0$ such that $\EXT{j}{\Gr{A}}{A/A_{\geq n}, M}_{\geq d}$ is an object of $\gr{A}$ when $n_0 \leq n$.
    \item
      If $\chi_i$ holds for every object of $\gr{A}$, then we say that $\chi_i$ holds for $A$.
    \item
      If $\chi_i$ holds for every $i$, then we say that $\chi$ holds for $A$.
    \end{enumerate}
  \end{defn}
\end{frame}

\begin{frame}
\begin{center}
Thank you!
\end{center}
\end{frame}






\end{document}
