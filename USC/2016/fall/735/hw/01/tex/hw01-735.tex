\documentclass[10pt]{amsart}
\usepackage{amsmath,amsthm,amssymb,amsfonts,enumerate,mymath,mathtools,tikz-cd,mathrsfs}
\openup 5pt
\author{Blake Farman\\University of South Carolina}
\title{Math 735:\\Homework 01}
\date{September 14, 2016}
\pdfpagewidth 8.5in
\pdfpageheight 11in
\usepackage[margin=1in]{geometry}

\begin{document}
\maketitle

\providecommand{\p}{\mathfrak{p}}
\providecommand{\m}{\mathfrak{m}}
\providecommand{\Deck}[1]{\operatorname{Deck}\left(#1\right)}
%\newcommand{\Res}{\operatorname{Res}}
\newtheorem{thm}{Theorem}
\newtheorem{ex}{}
\newtheorem{lem}{Lemma}
\newtheorem{cor}{Corollary}
\newtheorem{prop}{Proposition}
\theoremstyle{definition}
\newtheorem{defn}{Definition}

\newcommand{\A}{\mathbb{A}}

\begin{lem}\label{lem1}
  If $G$ is a Lie group and $g \in G$, the maps
  $$\begin{tikzcd}
    G \arrow{r}{\varphi_g} & G & \text{and} & G \arrow{r}{_g\varphi} & G\\
    x \arrow[mapsto]{r} & gx & & x \arrow[mapsto]{r} & xg
  \end{tikzcd}$$
  are homeomorphisms.
  \begin{proof}
    Since $G$ is a lie group, the multiplication map 
    $\begin{tikzcd}
      G \times G \arrow{r}{\mu} & G 
    \end{tikzcd}$
      is continuous.
      We have an obvious morphism of topological spaces
      $$\begin{tikzcd}
        G \arrow{r}{g} & G\\
        x \arrow[mapsto]{r} & g
      \end{tikzcd}$$
      which gives rise by the universal property for the product of topological spaces to morphisms
      $$\begin{tikzcd}
        & G \arrow[swap]{ld}{\operatorname{id}_G} \arrow{rd}{g} \arrow[dashed]{d}{\exists !\imath_g} & &\text{and} & & G\arrow[swap]{ld}{g} \arrow{rd}{\operatorname{id}_G}\arrow[dashed]{d}{\exists ! _g\imath}\\
        G & \arrow{l}{\pi_1} G \times G \arrow[swap]{r}{\pi_2} & G & & G & \arrow{l}{\pi_1} G \times G \arrow[swap]{r}{\pi_2} & G
      \end{tikzcd}$$
      Therefore $\varphi_g = \mu \circ \imath_g$ and $_g\varphi = \mu \circ\,_g\imath$ are both continuous with continuous inverses $\varphi_{g^{-1}}$ and $_{g^{-1}}\varphi$.
  \end{proof}
\end{lem}

\begin{cor}\label{cor1}
  Let $G$ be a Lie group and let $g \in G$ be given.
  The map
  $$\begin{tikzcd}
    G \arrow{r}{\Phi_g} &G\\
    x \arrow[mapsto]{r}& xgx^{-1}
  \end{tikzcd}$$
  is continuous.

  \begin{proof}
    Since $G$ is a lie group, the inversion map
    $$\begin{tikzcd}
      G \arrow{r}{\_^{-1}} &G\\
      x \arrow[mapsto]{r} & x^{-1} 
    \end{tikzcd}$$
    is continuous.
    Combining this with Lemma~\ref{lem1}, we obtain morphisms from the universal property for products in $\mathfrak{Top}$,
    $$\begin{tikzcd}
      & G\arrow[swap]{ld}{\operatorname{id}_G}\arrow{rd}{\operatorname{id}_G}\arrow[dashed]{d}{\exists !\Delta}\\
      G\arrow[swap]{d}{_g\varphi} & \arrow{l}{\pi_1} G \times G \arrow[dashed]{d}{\exists !_g\varphi \times \_^{-1}} \arrow[swap]{r}{\pi_2} & G\arrow{d}{\_^{-1}}\\
      G & \arrow{l}{\pi_1} G \times G \arrow[swap]{r}{\pi_2} & G\\
    \end{tikzcd}$$
    and observe that 
    $$\mu \circ \left(_g\varphi \times \_^{-1}\right) \circ \Delta(x)
    = \mu \circ \left(_g\varphi \times \_^{-1}\right)(x,x)
    = \mu (_g\varphi(x),x^{-1})
    = xgx^{-1}
    = \Phi_g(x)$$
    is continuous, as desired.
  \end{proof}
\end{cor}

\begin{ex}\label{thm1}
  \begin{enumerate}[(a)]
  \item
    Let $A_i$ be a sequence of matrices in $\operatorname{U}(n) = \U{n}{\C}$ converging to a matrix $A \in \GL{n}{\C}$.
    Prove that $A \in \U{n}{\C}$ (and thereby establish that $\U{n}{\C}$ is indeed a matrix Lie group).
  \item
    Prove the same for $\SU{n}{\C}$.
  \end{enumerate}
  \begin{proof}
    \begin{enumerate}[(a)]
    \item
      Write $A_i = \left(a_{j,k}^{(i)}\right)$ and $A = \left(a_{j,k}\right)$ so that, by assumption, $a_{j,k}^{(i)} \rightarrow a_{j,k}$ and thus
      \begin{eqnarray*}
        (AA^*)_{j,k} &=& \sum_{\ell = 1}^n a_{j,\ell}\overline{a_{k,\ell}}\\
        &=& \sum_{\ell = 1}^n \left(\lim_{i \rightarrow \infty}a_{j,\ell}^{(i)}\right)\overline{\left(\lim_{i \rightarrow \infty}a_{k,\ell}^{(i)}\right)}\\
        &=& \lim_{i \rightarrow \infty}\sum_{\ell = 1}^n a_{j,\ell}^{(i)}\overline{a_{k,\ell}^{(i)}}\\
        &=& \lim_{i \rightarrow \infty} (A_iA_i^*)_{j,k}\\
        &=& \delta_{j,k}
      \end{eqnarray*}
      and hence $AA^* = I$.
      Therefore $A^{-1} = A^*$ and hence $A \in \U{n}{\C}$.
    \item
      By definition, $\SU{n}{\C} = \U{n}{\C} \cap \SL{n}{\C}$.
      By part (a), $\U{n}{\C}$ is closed, $\SL{n}{\C}$ is also closed, hence $\SU{n}{\C}$ is also closed.
      Therefore every convergent sequence of $\SL{n}{\C}$ converges to a matrix in $\SL{n}{\C}$.
    \end{enumerate}
  \end{proof}
\end{ex}

\begin{ex}
  Read the definition of the Euclidean group $\operatorname{E}(n) = \left\{\{x,R\} \;\middle\vert\; x \in \R^n, R \in \operatorname{O}(n)\right\}$ in Section 1.2.5 of the book, and verify that $\operatorname{E}(n)$ is a matrix Lie group by checking (1.11).
  \begin{proof}
    First observe that if $\{x,R\} = \{y, S\}$, then by the definition of the action of $\R^n$,
    $$x = \{x,R\}0 = \{y,S\}0 = y$$
    and hence for any basis element $e_i$ of $\R^n$,
    $$Re_i + x = \{x,R\}e_i = \{y,S\}e_i = Se_i + y = Se_i + x$$
    implies $Re_i = Se_i$ and thus $R = S$.
    Hence the map
    $$\begin{tikzcd}
      \operatorname{E}(n) \arrow{r}{\varphi} & \GL{n+1}{\R}\\
      \{x,R\} \arrow[mapsto]{r} & \left(
      \begin{matrix}
        &   & x_1\\
        & R & \vdots\\
        &   & x_n\\
        0 & \ldots 0 & 1
      \end{matrix}
      \right)
    \end{tikzcd}$$
    is a well defined bijection onto its image.
    We see that this is a morphism of groups for if $\{x,R\}$, $\{y,S\} \in \operatorname{E}(n)$, then the product of their images
    $$
      \left(
      \begin{matrix}
        r_{1,1} & r_{1,2} & \ldots & r_{1,n} & x_1\\
        r_{2,1} & r_{2,2} & \ldots & r_{2,n} & x_2\\
        \vdots & \vdots & \ddots & \vdots & \vdots\\
        r_{n,1} & r_{n,2} & \ldots & r_{n,n} & x_n\\
        0 & 0 & \ldots & 0 & 1
      \end{matrix}
      \right)
      \left(
      \begin{matrix}
        s_{1,1} & s_{1,2} & \ldots & s_{1,n} & y_1\\
        s_{2,1} & s_{2,2} & \ldots & s_{2,n} & y_2\\
        \vdots & \vdots & \ddots & \vdots & \vdots\\
        s_{n,1} & s_{n,2} & \ldots & s_{n,n} & y_n\\
        0 & 0 & \ldots & 0 & 1
      \end{matrix}
      \right)\\
      $$
      is given by the matrix
      $$\left(
      \begin{matrix}
        \sum_{i = 1}^n r_{1,i}s_{i,1} & \ldots & \sum_{i = 1}^n r_{1,i}s_{i,n} & \sum_{i = 1}^n r_{1,i}y_i + x_1\\
        \sum_{i = 1}^n r_{2,i}s_{i,1} & \ldots & \sum_{i = 1}^n r_{2,i}s_{i,n} \ & \sum_{i = 1}^n r_{2,i}y_i + x_2\\
        \vdots & \ddots & \vdots & \vdots\\
        \sum_{i = 1}^n r_{n,i}s_{i,1} & \ldots & \sum_{i = 1}^n r_{n,i}s_{i,n} \ & \sum_{i = 1}^n r_{n,i}y_i + x_n\\
        0 & \ldots & 0 & 1
      \end{matrix}
      \right) = 
      \left(
      \begin{matrix}
        (RS)_{1,1} & \ldots & (RS)_{1,n} & (Ry)_1 + x_1\\
        (RS)_{2,1} & \ldots & (RS)_{2,n} & (Ry)_2 + x_2\\
        \vdots & \ddots & \vdots & \vdots\\
        (RS)_{n,1} & \ldots & (RS)_{n,n} & (Ry)_n + x_n\\
        0 & \ldots & 0 & 1
      \end{matrix}
      \right)$$
      which is visibly the image of 
      $$\{Ry + x, RS\} = \{x, R\}\{y,S\}.$$
      Therefore $E(n)$ is isomorphic to a subgroup of $\GL{n+1}{\R}$, as desired.
  \end{proof}
\end{ex}

\begin{ex}
  Characterize the image of the 'restriction of scalars' map $\M{n}{\mathbb{H}} \rightarrow \M{2n}{\C}$ in as nice of a way as you can.
  \begin{proof}
    First observe that we can view an element $a + ib + jc + kd$ of $\mathbb{H}$ as an element of $\M{2n}{\C}$
    $$M = \left(\begin{matrix}
      \alpha & -\overline{\beta}\\
      \beta & \overline{\alpha}
    \end{matrix}\right)$$
    where $\alpha = a + id$ and $\beta = b - ic$.
    If we denote 
    $$\Omega = \left(\begin{matrix}
      0 & 1\\
      -1 & 0
    \end{matrix}\right)$$
    then we observe that
    $$-\Omega M \Omega = 
    \left(\begin{matrix}
      \overline{\alpha} & -\beta\\
      \overline{\beta} & \alpha
    \end{matrix}\right)
    = \overline{M}.$$
    Indeed, this characterizes the image of $\mathbb{H}$ in $\M{2n}{\C}$, for if
    $$\left(\begin{matrix}
    \overline{z_{1,1}} & \overline{z_{1,2}}\\
    \overline{z_{2,1}} & \overline{z_{2,2}}
    \end{matrix}\right) 
    = 
    -\Omega \left(\begin{matrix}
      z_{1,1} & z_{1,2}\\
      z_{2,1} & z_{2,2}
    \end{matrix}\right)
    \Omega
    = 
    \left(\begin{matrix}
      z_{2,2} & -z_{2,1}\\
      -z_{1,2} & z_{1,1}
    \end{matrix}\right)$$
    then we see that $z_{2,2} = \overline{z_{1,1}}$ and $z_{1,2} = -\overline{z_{2,1}}$.
    
    Now, if we consider an element $M = \left(x_{i,j}\right)$ of $\M{n}{\mathbb{H}}$, then identifying $x_{i,j}$ with the matrix
    $$M_{i,j} = \left(\begin{matrix}
      \alpha_{i,j} & -\overline{\beta_{i,j}}\\
      \beta_{i,j} & \overline{\alpha_{i,j}}
    \end{matrix}\right)$$
    the image of $x_{i,j}$ under $\operatorname{Res}_{\mathbb{H}/\C}$ is a $2n \times 2n$ block matrix
    $$M = \left(\begin{matrix}
      M_{1,1} & M_{1,2} & \ldots & M_{1,n}\\
      M_{2,1} & M_{2,2} & \ldots & M_{2,n}\\
      \vdots & \vdots & \ddots & \vdots\\
      M_{n,1} & M_{n,2} & \ldots & M_{n,n}
    \end{matrix}\right).$$
    If we denote by $0_2$ the $2 \times 2$ matrix of zeroes, we see that
    \begin{eqnarray*}
      \left(\begin{matrix}
      -\Omega &  0_2 & \ldots & 0_2\\
      0_2 & -\Omega & \ldots & 0_2\\
      \vdots & \vdots & \ddots & \vdots\\
      0_2 & 0_2 & \ldots & -\Omega
    \end{matrix}\right)
    M
    \left(\begin{matrix}
      \Omega &  0_2 & \ldots & 0_2\\
      0_2 & \Omega & \ldots & 0_2\\
      \vdots & \vdots & \ddots & \vdots\\
      0_2 & 0_2 & \ldots & \Omega
    \end{matrix}\right)
    &=&
    \left(\begin{matrix}
        -\Omega M_{1,1}\Omega & \ldots & -\Omega M_{1,n}\Omega\\
        \vdots & \ddots & \vdots\\
        -\Omega M_{n,1} \Omega & \ldots & -\Omega M_{n,n} \Omega
      \end{matrix}\right)\\
    &=& 
    \left(\begin{matrix}
      \overline{M_{1,1}} & \ldots & \overline{M_{1,n}}\\
        \vdots & \ddots & \vdots\\
        \overline{M_{n,1}} & \ldots & \overline{M_{n,n}}
      \end{matrix}\right)\\
    &=& \overline{M}.
    \end{eqnarray*}
    Using the characterization from the embedding of $\mathbb{H}$ into $\M{2}{\C}$, we see that the image of $\operatorname{Res}_{\mathbb{H}/\C}$ is the set of matrices satisfying the conjugation relation above.
  \end{proof}
\end{ex}

\begin{ex}
  Let $\omega$ be the skew-symmetric bilinear form on $\R^{2n}$ given by 
  $$\omega(x,y) = \sum_{j = 1}^n (x_jy_{n + j} - x_{n+j}y_j).$$
  Let $\Omega$ be the $2n \times 2n$ matrix
  $$\Omega = \left(\begin{matrix}
    0 & I\\
    -I & 0
    \end{matrix}\right).$$
  Show that for all $x,y \in \R^{2n}$, we have
  $$\omega(x,y) = \left<x, \Omega y\right>.$$
  Show that a $2n \times 2n$ matrix, $A$, belongs to $\operatorname{Sp}_n(\R)$ if and only if $-\Omega A^\text{tr} \Omega = A^{-1}$.\\
  {\it Note}: A similar analysis applies to $\operatorname{Sp}_n(\C)$.
  \begin{proof}
    First we observe that 
    $$\left(\Omega y\right)_{i} = \sum_{j = 1}^{2n} \Omega_{i,j}y_i 
    = \left\{
    \begin{array}{cc}
      y_{n + i} & i \leq n\\
      -y_{n - i} & \text{else}.
    \end{array}
    \right.$$
    Hence we have
    \begin{eqnarray*}
      \langle x, \Omega y \rangle &=& \sum_{i=1}^{2n} x_i(\Omega y)_i \\
      &=& \sum_{i=1}^{n} x_i(\Omega y)_i + \sum_{i=n+1}^{2n} x_i(\Omega y)_i \\
      &=& \sum_{i=1}^{n} x_iy_{n+i} + \sum_{i=n+1}^{2n} x_i(-y_{n -i})\\
      &=& \sum_{i=1}^{n} x_iy_{n+i} - \sum_{i=1}^{n} x_{n + i}y_{i}\\
      &=& \sum_{i=1}^{2n} \left(x_iy_{n+i} - x_{n + i} y_{i}\right)\\
      &=& \omega(x,y).
    \end{eqnarray*}

    By definition, $A$ belongs to $\operatorname{Sp}_n(\R)$ if and only if 
    $$\langle Ax, \Omega Ay\rangle = \omega(Ax,Ay) = \omega(x,y) = \langle x, \Omega y\rangle$$ for all $x,y \in \R^{2n}$.
    This is equivalent to 
    \begin{eqnarray*}
      0 = \langle Ax, \Omega Ay\rangle - \langle x, \Omega y\rangle
      &\iff& 0 = \langle x, A^\text{tr}\Omega Ay\rangle - \langle x, \Omega y\rangle\\
      &\iff& 0 = \langle x, A^\text{tr}\Omega Ay - \Omega y\rangle\\
      &\iff& A^\text{tr}\Omega Ay - \Omega y = 0\\
      &\iff& A^\text{tr}\Omega Ay = \Omega y.
    \end{eqnarray*}
    It is straightforward to check that $-\Omega = \Omega^{-1}$, and hence it follows from
    $$y = -\Omega(\Omega y) = 
    -\Omega(A^\text{tr}\Omega A y)
    = (-\Omega A^\text{tr}\Omega)(Ay)
    = ((-\Omega A^\text{tr}\Omega)A)y$$
    that $A^{-1} = -\Omega A^\text{tr} \Omega$.
    Therefore $A$ belongs to $\operatorname{Sp}_n(\R)$ if and only if $A^{-1} = -\Omega A^\text{tr} \Omega$.
  \end{proof}
\end{ex}

\begin{ex}
  Show that the symplectic group $\operatorname{Sp}_1(\R) \subset \GL{2}{\R}$ is equal to $\SL{2}{\R}$.
  Show that $\operatorname{Sp}_1(\C) = \SL{2}{\C}$ and that $\operatorname{Sp}(1) = \operatorname{SU}(2)$.
  \begin{proof}
    Given a $2 \times 2$ matrix with real entries
    $$A = \left(\begin{matrix}
      a & b\\
      c & d
    \end{matrix}\right)$$
    we observe that
    $$-\Omega A^\text{tr} \Omega = \left(\begin{matrix}
      0 & -1\\
      1 & 0
    \end{matrix}\right)
    \left(\begin{matrix}
      a & c\\
      b & d
    \end{matrix}\right)
    \left(\begin{matrix}
      0 & 1\\
      -1 & 0
    \end{matrix}\right)
    =
    \left(\begin{matrix}
      d & -b\\
      -c & a
    \end{matrix}\right)$$
    and hence 
    $$(-\Omega A^\text{tr} \Omega) A = \left(\begin{matrix}
      ad - bc & 0\\
      0 & ad - bc
    \end{matrix}\right) = \operatorname{id}_{\Mat{2}{\R}}$$
    if and only if $\det{A} = 1$.
    Therefore by the previous exercise, $\operatorname{Sp}_2(\R) = \SL{2}{\R}$.
    Changing the entries from $\R$ to $\C$, the same proof shows $\operatorname{Sp}_1(\C) = \SL{2}{\C}$ by Hall's remark at the end of the statement of the last exercise.
    The fact that $\operatorname{Sp}(1) = \operatorname{SU}(2)$ then follows from the definition 
    $$\operatorname{SU}(2) = \SL{2}{\C} \cap \operatorname{U}_2(\C) = \operatorname{Sp}_1(\C) \cap \operatorname{U}_2(\C) = \operatorname{Sp}(1).$$
  \end{proof}
\end{ex}

\begin{ex}
  Show that if $\alpha$ and $\beta$ are arbitrary complex numbers satisfying $|\alpha|^2 + |\beta|^2 = 1$, then the matrix
  $$A = \left(\begin{matrix}
    \alpha & -\overline{\beta}\\
    \beta & \overline{\alpha}
  \end{matrix}\right)$$
  is in $\operatorname{SU}(2)$.
  Show that every $A \in \operatorname{SU}(2)$ can be expressed in this form for a unique pair $(\alpha, \beta)$ satisfying $|\alpha|^2 + |\beta|^2 = 1$.
  \begin{proof}
    By the previous exercises, we note that
    \begin{eqnarray*}
    \left(\begin{matrix}
      \alpha & -\overline{\beta}\\
      \beta & \overline{\alpha}
    \end{matrix}\right)
    \left[
    \left(\begin{matrix}
      0 & -1\\
      1 & 0
    \end{matrix}\right)
    \left(\begin{matrix}
      \alpha & \beta\\
      -\overline{\beta} & \overline{\alpha}
    \end{matrix}\right)
    \left(\begin{matrix}
      0 & 1\\
      -1 & 0
    \end{matrix}\right)\right]
    &=&
    \left(\begin{matrix}
      \alpha & -\overline{\beta}\\
      \beta & \overline{\alpha}
    \end{matrix}\right)
    \left(\begin{matrix}
      \overline{\alpha} & \overline{\beta}\\
      -\beta & \alpha
    \end{matrix}\right)\\
    &=&
    AA^*\\
    &=&
    \left(\begin{matrix}
      |\alpha|^2 + |\beta|^2 & 0\\
      0 & |\alpha|^2 + |\beta|^2
    \end{matrix}\right)\\
    &=& \operatorname{id}_{\M{2}{\C}}
    \end{eqnarray*}
    implies that $A \in \operatorname{Sp}_1(\C) \cap \operatorname{U}_2(\C) = \operatorname{SU}(2)$.
    This gives a monomorphism of sets
    $$\left\{\left(\begin{matrix}
    \alpha & -\overline{\beta}\\
    \beta & \overline{\alpha}
  \end{matrix}\right) \;\middle\vert\; |\alpha|^2 + |\beta|^2 = 1\right\} \rightarrow \operatorname{SU}(2)$$
    For bijectivity, let
    $$A = \left(\begin{matrix}
      a & b\\
      c & d
    \end{matrix}\right) \in \operatorname{SU}(2)$$
    be given and observe that, by assumption,
    $$\left(\begin{matrix}
      d & -b\\
      -c & a
      \end{matrix}\right) = 
    A^{-1} = 
    A^* =
    \left(\begin{matrix}
      \overline{a} & \overline{c}\\
      \overline{b} & \overline{d}
      \end{matrix}\right)$$
    and hence $d = \overline{a}$ and $b = -\overline{c}$.
    Therefore 
    $$A = \left(\begin{matrix}
      a & -\overline{c}\\
      c & \overline{a}
    \end{matrix}\right)$$
    with $|a|^2 + |b|^2 = 1$, as desired.
  \end{proof}
\end{ex}

\begin{ex}
  Determine the center $\cntr{H}$ of the Heisenberg group,
  $$H = \left\{\left(\begin{matrix}1 & a & b\\0 & 1 & c\\0 & 0 & 1\end{matrix}\right) \in \M{3}{\R} \;\middle\vert\; a,b,c \in \R\right\}.$$
    Show that the quotient group $H/\cntr{H}$ is commutative.
    \begin{proof}
      Let 
      $$Z = \left(\begin{matrix}
      1 & z_1 & z_2\\
      0 & 1 & z_3\\
      0 & 0 & 1
      \end{matrix}\right) \in \cntr{H}$$
      be given.
      For 
      $$A = \left(\begin{matrix}1 & a & b\\0 & 1 & c\\0 & 0 & 1\end{matrix}\right) \in H$$
      we have  
      $$A = Z^{-1}AZ = \left(\begin{array}{rrr}
        1 & a & b + az_3 - cz_1\\
        0 & 1 & c \\
        0 & 0 & 1
      \end{array}\right)$$
      implies $az_3 = cz_1$ for all $a,c \in \R$ and hence $z_1 = z_3 = 0$.
      Therefore
      $$\cntr{H} = \left\{\left(\begin{matrix}1 & 0 & b\\0 & 1 & 0\\0 & 0 & 1\end{matrix}\right) \in \M{3}{\R} \;\middle\vert\; b \in \R\right\}.$$
        
        To see that $H/\cntr{H}$ is abelian, let
        %$$A = \left(\begin{matrix}1 & a_1 & a_2\\0 & 1 & a_3\\0 & 0 & 1\end{matrix}\right)$$
        %and
        $$B = \left(\begin{matrix}1 & e & f\\0 & 1 & g\\0 & 0 & 1\end{matrix}\right) \in H$$
          be given and observe that
        $$A^{-1}B^{-1}AB = \left(\begin{array}{rrr}
          1 & 0 & ag - ce\\
          0 & 1 & 0 \\
          0 & 0 & 1
        \end{array}\right) \in \cntr{H}.$$
    \end{proof}
\end{ex}

\begin{ex}
  A subset $E$ of a matrix Lie group, $G$, is called {\bf discrete} if for each $A$ in $E$, there is a neighbourhood $U$ of $A$ in $G$ such that $U$ contains no point in $E$ except for $A$.
  Suppose that $G$ is a connected matrix Lie group and $N$ is a discrete normal subgroup of $G$.
  Show that $N$ is contained in the center of $G$.
  \begin{proof}
    First, observe that $N$, endowed with the subspace topology inherited from $G$, is discrete; indeed, every point $n \in N$ is open since, by assumption, we can always find a neighbourhood $U \subseteq G$ of $n$ with $U \cap N = \{n\}$.
    %Since $G$ is a Lie group and $N$ is normal, the map
    
    Since $N$ is normal, the morphism of groups
    \begin{align*}
      \Phi_n \colon G &\rightarrow N\\
      g & \mapsto gng^{-1}
    \end{align*}  
    for fixed $n \in N$ is well-defined and continuous by Corollary~\ref{cor1}.
    Since $G$ is connected, the image of $\Phi_n$ is also connected, and hence must be a singleton.
    Therefore as $\Phi_n(1) = n$, the image is $\{n\}$ and $N$ is central, as desired.
  \end{proof}
\end{ex}
\end{document}
