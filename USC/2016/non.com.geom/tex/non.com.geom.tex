\documentclass[10pt]{amsart}
\usepackage{amsmath,amsthm,amssymb,amsfonts,enumerate,mymath,mathtools,tikz-cd,mathrsfs}
\openup 5pt
\author{Blake Farman\\University of South Carolina}
\title{Math 748:\\Homework 01}
\date{April 24, 2016}
\pdfpagewidth 8.5in
\pdfpageheight 11in
\usepackage[margin=1in]{geometry}

\begin{document}
%\maketitle

\providecommand{\p}{\mathfrak{p}}
\providecommand{\m}{\mathfrak{m}}
\providecommand{\Deck}[1]{\operatorname{Deck}\left(#1\right)}
%\newcommand{\Res}{\operatorname{Res}}
\newtheorem{thm}{Theorem}
\newtheorem{ex}{}
\newtheorem{lem}{Lemma}
\newtheorem{cor}{Corollary}
\newtheorem{prop}{Proposition}
\theoremstyle{definition}
\newtheorem{defn}{Definition}
\newtheorem{rmk}{Remark}

\newcommand{\A}{\mathscr{A}}
\renewcommand{\C}{\mathscr{C}}

\section{Commutative Case}
\begin{defn}
  Say $X$ is projective over $\operatorname{Spec} A$ if there is a closed immersion $X \rightarrow \mathbb{P}^r_A$ for some $r$.
\end{defn}

\begin{rmk}
  $X$ is projective over $\operatorname{Spec} A$ if and only if it is isomorphic to $\operatorname{Proj} S$ for some connected graded ring $S$ (i.e. $S_0 = A$) and $S$ is finitely generated by $S_1$ as an $A = S_0$-algebra.
\end{rmk}

\begin{defn}
  Say an invertible sheaf sheaf $\mathscr{L}$ on a scheme $X$ is very ample relative to $Y$ if there is an immersion $i : X \rightarrow \mathbb{P}^r_A$ for some $r$ such that $i^*(\mathcal{O}(1)) \cong \mathscr{L}$.
\end{defn}
\begin{thm}
  Let $X$ be a projective scheme over a noetherian ring $A$, and let $\mathcal{O}_X(1)$ be a very ample invertible sheaf on $X$ over $\operatorname{Spec} A$.
  Let $\mathscr{F}$ be a coherent sheaf on $X$. Then:
  \begin{enumerate}[(a)]
  \item
    for each $0 \leq i$, $H^i(X,\mathscr{F})$ is a finitely generated $A$-module;
  \item
    there is an integer $n_0$, depending on $\mathscr{F}$ such that for each $0 < i$ and each $n_0 \leq n$, $H^i(X, \mathscr{F}(n)) = 0$.
  \end{enumerate}
\end{thm}

\begin{rmk}
  To say that $\mathcal{O}_X(1)$ is very ample is to say there is a closed immersion $i \colon X \rightarrow \mathbb{P}^r_A$ for some $r$ and $\mathcal{O}_X(1) \cong i^*(\mathcal{O}_{\mathbb{P}^r_A})$.
\end{rmk}
\section{Preliminaries}
Let $k$ be a Noetherian commutative ring, $A$ a $\Z_{\geq 0}$-graded right Noetherian ring.
Denote by $\Gr{A}$ (resp. $\gr{A}$) the category of graded right $A$-modules (resp. finite) with morphisms
$$\Hom{\Gr{A}}{M,N} = \left\{f \in \Hom{A}{M,N} \;\middle\vert\; f(M_d) \subseteq N_d\right\}.$$
This is a Grothendieck category with injective envelopes.
That is,
\begin{itemize}
\item
  $\Gr{A}$ is abelian (zero object, finite biproducts, all kernels and cokernels, monics and epics are normal--every monic is a kernel and every epic is a cokernel),
\item
  every family of objects has a direct limit/filtered colimit,
\item
  the presheaf $h^A \colon \Gr{A} \rightarrow \Sets$ is faithful; for any morphism $M \rightarrow N$ the morphism
  $$\begin{tikzcd}
    \Hom{\Gr{A}}{M, N} \arrow{r}& \Hom{\Sets}{h^A(M), h^A(N)}\\
    f \arrow[mapsto]{r} & h^A(f)
  \end{tikzcd}$$
  %    $$\begin{tikzcd}
  %      \Hom{\Gr{A}}{A, M} \arrow{r} & \Hom{\Gr{A}}{A, N}\\
  %      A \rightarrow M \arrow[mapsto]{r} & A \rightarrow M \rightarrow N
  %    \end{tikzcd}$$
  is a monomorphism of sets.
\end{itemize}

\begin{defn}
  A full subcategory, $\A$, of $\C$ is called a Serre (or \'{e}paisse/thick/dense) subcategory if for any short exact sequence
  $$\begin{tikzcd}
    0 \arrow{r} & X^\prime \arrow{r} & X \arrow{r} & X^{\prime\prime} \arrow{r} & 0
  \end{tikzcd}$$
  of $\C$, $X$ is an object of $\A$ if and only if both $X^\prime$ and $X^{\prime\prime}$ are.
\end{defn}

Denote by $\Tors$ (resp. $\tors$) the Serre subcategory of torsion modules (resp. finite), where a module $M$ is called torsion if
$$\tau(M) = \left\{m \in M \;\middle\vert\; xA_{\geq s} = 0\ \text{for some}\ s\right\} = M$$

\section{Quotient Categories}
Throughout, let $\C$ be an abelian category.

\begin{defn}
  Let $X$ be an object of $\C$.
  For two subobjects $i_1 \colon X_1 \rightarrow X$ and $i_2 \colon X_2 \rightarrow X$ denote by $X_1 \cap X_2$ the fibered product
  $$\begin{tikzcd}
    X_1 \cap X_2 \arrow{d}{i_2^\prime}\arrow{r}{i_1^\prime} & X_1\arrow{d}{i_1}\\
    X_2 \arrow{r}{i_2} & X
  \end{tikzcd}$$
  and denote by $X_1 + X_2$ the fibered coproduct 
  $$\begin{tikzcd}
    X_1 \cap X_2 \arrow{r}{i_1^\prime} \arrow{d}{i_2^\prime} & X_1\arrow{d}{u_1}\\
    X_2 \arrow{r}{u_2} & X_1 + X_2.
  \end{tikzcd}$$
  These are both subobjects of $X$ and endow the subobjects of $X$ with lattice structure under the relation
  $$X_1 \leq X_2$$
  if there exists a monomorphism making the diagram
  $$\begin{tikzcd}
    X_1\arrow{rd}{i_1}\arrow[dashed]{rr}{\exists !} & & X_2 \arrow{ld}{i_2}\\
    & X &
  \end{tikzcd}$$
  commute.
\end{defn}

\begin{rmk}
  Alternatively, one can construct $X_1 + X_2 \rightarrow X$ as the image of the morphism $s$ below
  $$\begin{tikzcd}
    X_1\arrow[bend right]{rdd}{i_1}\arrow{rd} & & X_2 \arrow{ld} \arrow[bend left]{ldd}{i_2}\\
    & X_1 \coprod X_2\arrow[dashed]{d}{\exists !s} &\\
    & X. &
  \end{tikzcd}$$
\end{rmk}

\begin{defn}
  Given an object $X$ of $\C$, an essential extension is a monomorphism $i \colon X \rightarrow E$ such that for any non-zero subobject $E^\prime \rightarrow E$, $E \cap X$ is non-zero.
  
  If $E$ is an injective object, then we say that $i$ is an injective envelope/hull.
\end{defn}

\begin{prop}
  Let $f : X \rightarrow Y$ be a morphism of $\C$.
  The following are equivalent
  \begin{enumerate}
  \item
    For any subobject $Y^\prime \rightarrow Y$, in the pullback diagram
    $$\begin{tikzcd}
      f^{-1}(Y^\prime) \arrow{r}\arrow{d} & X\arrow{d}{f}\\
      Y^\prime \arrow{r} & Y
    \end{tikzcd}$$
    $f^{-1}(Y^\prime) = 0$ implies $Y^\prime = 0$, and
  \item
    if $\zeta : Z \rightarrow X$ is a morphism such that $\ker{(f \circ \zeta)} = \ker{f}$, then $f$ is a monomorphism.
  \end{enumerate}
  In particular, $f$ is an essential extension if and only if whenever $f \circ\zeta$ is a monomorphism, $\zeta$ is a morphism.
\end{prop}

\begin{prop}
  Let $Q$ be an object of $\C$.
  The following are equivalent.
  \begin{enumerate}[(a)]
  \item
    $Q$ is injective,
  \item
    every morphism $X \rightarrow Q$ lifts over monics,
    $$\begin{tikzcd}
      0 \arrow{r} & X\arrow{d} \arrow{r} & Y \arrow[dashed]{ld}{\exists}\\
      & Q
    \end{tikzcd}$$
  \item
    the presheaf of abelain groups $h_Q$ is exact, and
  \item
    every short exact sequence
    $$\begin{tikzcd}
      0 \arrow{r} & Q \arrow{r} & X \arrow{r} & X/Q \arrow{r} & 0
    \end{tikzcd}$$
    splits.
  \end{enumerate}
\end{prop}

\begin{defn}
  We say that a subobject, $X^\prime$, of an object, $X$, is an $\A$-subobject of $X$ if $X^\prime$ is an object of $\A$.
  We say that an $\A$-subobject, $X^\prime$, is maximal if for every $\A$-subobject $X^{\prime\prime}$ we have a commutative diagram
  \begin{center}
    \begin{tikzcd}
      X^{\prime\prime}\arrow{rd}\arrow[dashed]{rr}{\exists !h} & & X^\prime\arrow{ld}\\
      & X &
    \end{tikzcd}
  \end{center}
  
  If $X$ has no non-zero $\A$ subobjects, then we say that $X$ is $\A$-torsionfree.
\end{defn}

\begin{prop}
  Let $X$ and $Y$ be objects of $\C$.
  The collection of pairs of subobjects $(X^\prime, Y^\prime)$ such that $X/X^\prime$ and $Y^\prime$ are objects of $\A$ is directed by the relation
  $$(X^\prime, Y^\prime) \leq (X^{\prime\prime}, Y^{\prime\prime})$$
  if $X^{\prime\prime} \leq X^\prime$ and $Y^\prime \leq Y^{\prime\prime}$.
  
  Moreover, the system of Abelian groups
  $$\Hom{\C}{X^\prime, Y/Y^\prime}$$
  induced by pairs $(X^\prime, Y^\prime)$ above is a directed system with morphisms
  $$\begin{tikzcd}
    \Hom{\C}{X^\prime, Y/Y^\prime} \arrow{r} &\Hom{\C}{X^{\prime\prime}, Y^{\prime\prime}}\\
    (X^\prime \rightarrow Y/Y^\prime) \arrow[mapsto]{r} & (X^{\prime\prime} \rightarrow X^\prime \rightarrow Y/Y^\prime \rightarrow Y/Y^{\prime\prime})
  \end{tikzcd}$$
  whenever $(X^\prime, Y^\prime) \leq (X^{\prime\prime}, Y^{\prime\prime})$.
\end{prop}

\begin{defn}
  Define the quotient category, $\C/\A$, to be the category with objects the objects of $\C$ and morphisms
  $$\Hom{\C/\A}{X, Y} = \operatorname{colim}_{(X^\prime, Y^\prime)} \Hom{\C}{X^\prime, Y/Y^\prime}.$$
  Let $\pi \colon \C \rightarrow \C/\A$ be the canonical projection functor, defined by $\pi(X) = X$ and sending a morphism $f \colon X \rightarrow Y$ to its image, $\pi(f)$, in the colimit.
\end{defn}

\begin{lem}
  The quotient category, $\C/\A$, is an additive category and $\pi$ is an additive functor.
\end{lem}

\begin{lem}
  Let $f \colon X \rightarrow Y$ be a morphism of $\C$.  
  We have a factorization of $f$
  $$\begin{tikzcd}
    X \arrow{rd}{\coim{f}}\arrow{rr}{f} & & Y\\
    & f(X)\arrow{ur}{\im{f}} & 
  \end{tikzcd}$$
  and an exact sequence 
  $$\begin{tikzcd}
    0 \arrow{r} & K \arrow{r}{\ker{f}} & X \arrow{r}{f} & Y \arrow{r}{\coker{f}} & C \arrow{r} & 0.
  \end{tikzcd}$$
  Then
  \begin{enumerate}[(i)]
  \item
    $\pi(f) = 0$ if and only if $f(X)$ is an object of $\A$,
  \item
    $\pi(f)$ is a monomorphism if and only if $K$ is an object of $\A$, and
  \item
    $\pi(f)$ is an epimorphism if and only if $C$ is an object of $\A$.
  \end{enumerate}
\end{lem}

\begin{lem}
  For any morphism $f \colon X \rightarrow Y$ of $\C$, we have an exact sequence 
  $$\begin{tikzcd}
    0 \arrow{r} & K \arrow{r}{\ker{f}} & X \arrow{r}{f} & Y \arrow{r}{\coker{f}} & C \arrow{r} & 0.
  \end{tikzcd}$$ 
  The morphism $\pi(f)$ has a kernel and a cokernel,
  $$\begin{tikzcd}
    0 \arrow{r} & \mathcal{K} \arrow{r}{\ker{\pi(f)}} & \pi(X) \arrow{r}{\pi(f)} & \pi(Y) \arrow{r}{\coker{\pi(f)}} & \mathcal{C} \arrow{r} & 0.
  \end{tikzcd}$$
  Moreover, $\pi(\ker{f})$ induces an isomorphism $\pi(K) \cong \mathcal{K}$ and $\pi(\coker{f})$ induces an isomorphism $\pi(C) \cong \mathcal{C}$.
\end{lem}

\begin{lem}
  Given an exact sequence 
  $$\begin{tikzcd}
    0 \arrow{r} & K \arrow{r}{\ker{f}} & X \arrow{r}{f} & Y \arrow{r}{\coker{f}} & C \arrow{r} & 0.
  \end{tikzcd}$$ 
  of $\C$, $f$ is an isomorphism if and only if $K$ and $C$ are both objects of $\A$.
\end{lem}

\begin{prop}
  The quotient category $\C/\A$ is an abelian category and $\pi$ is an exact functor.
\end{prop}

\section{Proj}
Denote by $\QGr{A}$ (resp. $\qgr{A}$) the quotient category $\Gr{A}/\Tors$ (resp. $\gr{A}/\tors$).
It can be shown that $\QGr{A}$ is an $\operatorname{Ab} 5$ category; see III.4 of Des Cat\`{e}gories Ab\'{e}lienne.
We view $\QGr{A}$ as the analogue of quasi-coherent sheaves and $\qgr{A}$ as the analogue of coherent sheaves.

\begin{defn}
  \begin{enumerate}[(i)]
  \item
    Let $\C$, and $\C^\prime$ be $k$-linear abelian categories; that is categories enriched over $\Mod{k}$.
    For $X$ and $X^\prime$ objects of $\C$ and $\C^\prime$, a morphism of pairs
    $$(\C, X) \rightarrow (\C^\prime, X^\prime)$$
    is a pair $(f, \theta)$ consisting of an isomorphism $\theta \colon f(X) \rightarrow X^\prime$ and a $k$-linear functor $f \colon \C \rightarrow \C^\prime$; that is, the canonical morphism
    $$\begin{tikzcd}
      \Hom{\C}{A,B}\arrow{r}& \Hom{\C^\prime}{f(A), f(B)}.
    \end{tikzcd}$$
    is $k$-linear.
    \begin{enumerate}[(a)]
    \item
      A morphism of pairs is said to be an isomorphism if $f$ is an equivalence of categories.
    \item
      A morphism of pairs is said to be right exact if $f$ preserves direct limits.
    \item
      Two morphisms of pairs $(f,\theta)$ and $(f^\prime, \theta^\prime)$ are said to be equivalent if there is a natural isomorphism $\eta \colon f \rightarrow f^\prime$ compatible with $\theta$ and $\theta^\prime$.
    \end{enumerate}
  \item
    Given two $k$-linear abelian categories $\C$ and $\C^\prime$ equipped with autoequivalences $s : \C \rightarrow \C$ and $s^\prime : \C^\prime \rightarrow \C^\prime$, and objects $X$ of $\C$ and $X^\prime$ of $\C$, a morphism of triples
    $$(\C, X, s) \rightarrow (\C^\prime, X^\prime, s^\prime)$$
    is a triple $(f, \theta, \mu)$ with $f : \C \rightarrow \C^\prime$ a $k$-linear functor, $\theta : f(X) \rightarrow X^\prime$ an isomorphism, and $\mu : f \circ s \rightarrow s^\prime \circ f$ a natural isomorphism.
    \begin{enumerate}[(a)]
    \item
      A morphism of triples is said to be right exact if $f$ preserves direct limits.
    \item
      Two morphisms of triples $(f_1, \theta_1, \mu_1)$ and $(f_2, \theta_2, \mu_2)$ are said to be equivalent if there exists a natural isomorphism $\eta : f_1 \rightarrow f_2$ such that
      $$\theta_1 = \theta_2 \circ \eta(A)$$
      and for all objects $A$ of $\C$
      $$(s^\prime \circ \eta(A)) \circ \mu_1 = \mu_2 \circ \eta(s(A)).$$
    \item
      A morphism of triples is said to be an isomorphism if $f$ is an equivalence of categories.
    \end{enumerate}
  \item
    Let $s$ be the twist functor, $s(M) = M[1]$, $s^d(M) = M[d]$, which is an automorphism of $\Gr{A}$.
    Since $\QGr{A}$ is a quotient category, it inherits this autoequivalence in an obvious way.
    The general (resp. Noetherian) projective scheme of $A$, $\Proj{A}$ (resp $\proj{A}$), is the pair $(\QGr{A}, \pi(A))$ (resp. $(\qgr{A}, \pi(A))$).
  \item
    A morphism $F :\Proj{B} \rightarrow \Proj{A}$ is an equivalence class of right exact morphisms of pairs $(\QGr{A}, \pi(A)) \rightarrow (\QGr{B}, \pi(B))$.
  \item
    A morphism of general schemes is an equivalence class of morphisms of triples $(\QGr{A}, \pi(A), s_A) \rightarrow (\QGr{B}, \pi(B), s_B)$.
  \end{enumerate}
  Analogous definitions are made for $\proj{A}$ by substituting $\QGr{A}$ with $\qgr{A}$ as necessary.
\end{defn}

The next two propositions describe the morphisms of $\QGr{A}$ explicitly.

\begin{prop}
  Given two objects $M$, $N$ of $\Gr{A}$, 
  $$\Hom{\QGr{A}}{\pi(M), \pi(N)} = \operatorname{colim}_{M^\prime}\Hom{\Gr{A}}{M^\prime, N/\tau(N)}.$$
  
  \begin{proof}
    Consider the indexing category $\mathscr{I}$ with objects pairs of subobjects $(M^\prime, N^\prime)$ such that $M/M^\prime$ and $N^\prime$ are objects of $\Tors$ and morphisms induced by the relation $\leq$ defined above.
    We note that because $N^\prime$ is a subobject of $\tau(N)$ for all $N^\prime$, the full subcategory, $\mathscr{J}$, with objects $(M^\prime, \tau(N))$ is cofinal.
    Therefore
    $$\Hom{\QGr{A}}{\pi(M), \pi(N)} = \operatorname{colim}_{\mathscr{I}}\Hom{\Gr{A}}{M^\prime, N/N^\prime} = \operatorname{colim}_{\mathscr{J}}\Hom{\Gr{A}}{M^\prime, N/\tau(N)}.$$
  \end{proof}
\end{prop}

\begin{prop}
  If $M$ is an object of $\gr{A}$, then 
  $$\Hom{\QGr{A}}{\pi(M), \pi(N)} = \lim_{n \rightarrow \infty} \Hom{\Gr{A}}{M_{\geq n}, N}$$
  where 
  $$M_{\geq n} = \bigoplus_{d \geq n} M_d.$$
  \begin{proof}
    Let $M^\prime \rightarrow M$ be a subobject with torsion quotient.
    By definition, for each $m \in M$ there is an $n_m$ such that $mA_{\geq n_m} \subseteq M^\prime$.
    Let $m_1, \ldots, m_s$ be a set of generators for $M$ and let $n = \max\left\{\deg(m_i) + n_{m_i} \;\middle\vert\; i = 1, \ldots , s\right\}$.
    The subobject $M_{\geq n}$ has torsion quotient, $M/M_{\geq n}$, and we get the kernel diagram
    $$\begin{tikzcd}
      M_{\geq n} \arrow[bend left]{rr}{0} \arrow[dashed]{rd}{\exists !}\arrow{r} & M \arrow{r} & M/M^\prime\\
      & M^\prime \arrow{u}
    \end{tikzcd}$$
    because for each $m \in M_{d}$ with $n \leq d$ we can write
    $$m = a_1m_1 + \ldots a_sm_s$$
    and by construction we have
    $$n_{m_i} = (\deg(m_i) + n_{m_i}) - deg(m_i) \leq n - \deg(m_1) \leq d - \deg(m_1) = \deg(a_i).$$
    With $\mathscr{I}$ as above, we see that for any object $(M^\prime, N^\prime)$ of $\mathscr{I}$ there exists some $n$ such that
    $$(M^\prime, N^\prime) \leq (M_{\geq n}, N)$$
    and hence the full subcategory $\mathscr{J}$ with objects $(M_{\geq n}, N)$ is cofinal.
    Therefore
    \begin{eqnarray*}
      \Hom{\QGr{A}}{\pi(M), \pi(N)} &=& \operatorname{colim}_{\mathscr{I}}\Hom{\Gr{A}}{M^\prime, N^\prime}\\
      &=& \operatorname{colim}_{\mathscr{J}}\Hom{\Gr{A}}{M_{\geq n}, N}\\
      &=& \lim_{n \rightarrow \infty} \Hom{\Gr{A}}{M_{\geq n}, N}.
    \end{eqnarray*}
  \end{proof}
\end{prop}

\section{The Section Functor}

\begin{lem}\label{lem1}
  If $X$ is an object of $\C$, then the following are equivalent.
  \begin{enumerate}
  \item\label{lem1.1}
    Given a short exact sequence 
    \begin{center}
      \begin{tikzcd}
        0 \arrow{r} & K \arrow{r}{\ker{f}} & Z \arrow{r}{f} & Y \arrow{r}{\coker{f}} & C \arrow{r} & 0
      \end{tikzcd}
    \end{center}
    with $K$ and $C$ objects of $\A$, then the canonical morphism
    $$h_X(f) \colon h_X(Y) \rightarrow h_X(Z)$$
    is an isomorphism,
  \item\label{lem1.2}
    $X$ is $\A$-torsionfree and 
    %Every monic $X^\prime \rightarrow X$ with $X^\prime$ an object of $\A$ is zero and 
    any short exact sequence 
    \begin{center}
      \begin{tikzcd}
        0 \arrow{r} & X \arrow{r}{f} & Y \arrow{r}{\coker{f}} &C \arrow{r} & 0
      \end{tikzcd}
    \end{center}
    with $C$ an object of $\A$ splits, and
  \item\label{lem1.3}
    For any object $Y$ of $\C$, $\pi \colon \C \rightarrow \C/\A$ induces an isomorphism
    $$\Hom{\C}{Y, X} \cong \Hom{\C/\A}{\pi(Y), \pi(X)}.$$
  \end{enumerate}
  
  \begin{proof}
    (\ref{lem1.1}) $\implies$ (\ref{lem1.2}).  Given an $\A$-subobject $i \colon X^\prime \rightarrow X$, then we have the short exact sequence
    \begin{center}
      \begin{tikzcd}
        0 \arrow{r} & X^\prime \arrow{r}{i} & X \arrow{r}{\coker{i}} & X/X^\prime \arrow{r} & 0
      \end{tikzcd}
    \end{center}
    both $X^\prime$ and $0$ are objects of $\A$, hence an isomorphism
    $$h_X(\coker{i}) \colon \Hom{\C}{X/X^\prime, X} \rightarrow \Hom{\C}{X,X}$$
    which implies that $\coker{i}$ is monic.
    Therefore $\coker{i} \circ i = 0$ implies $i = 0$.

    Now, if we let 
    \begin{center}
      \begin{tikzcd}
        0 \arrow{r} & X \arrow{r}{f} & Y \arrow{r}{p} & C \arrow{r} & 0
      \end{tikzcd}
    \end{center}
    be a short exact sequence with $C$ an object of $\A$, then the isomorphism
    $$h_X(f) \colon \Hom{\C}{Y,X} \rightarrow \Hom{\C}{X,X}$$
    yields a section $s \colon Y \rightarrow X$ of $f$, so the sequence splits.

    (\ref{lem1.2}) $\implies$ (\ref{lem1.3}).  Let $Y$ be an object of $\C$.
    Given a morphism $f : \pi(Y) \rightarrow \pi(X)$, we lift to a morphism $f^\prime \colon Y^\prime \rightarrow X/X^\prime$ with $Y/Y^\prime$ and $X^\prime$ objects of $\A$.
    Since we have assumed that $X$ has no non-trivial $\A$-subobjects, it follows that $X/X^\prime = X$.
    By dualizing the relevant theorems on fiber products, this gives the commutative diagram with exact rows
    \begin{center}
      \begin{tikzcd}
        0 \arrow{r} & Y^\prime \arrow{r}{j} \arrow{d}{f^\prime} & Y \arrow{d}{f^{\prime\prime}} \arrow{r}{\coker{j}} & Y/Y^\prime \arrow{r}\arrow[dashed]{d}{\exists ! h} & 0\\
        0 \arrow{r} & X \arrow{r}{i} & Y \coprod_{Y^\prime} X \arrow{r}{\coker{i}} & \left(Y \coprod_{Y^\prime} X\right)/X \arrow{r} & 0
      \end{tikzcd}
    \end{center}
    and with $h$ an isomorphism.
    Since $Y/Y^\prime$ was assumed to be an object of $\A$, so too is $\left(Y \coprod_{Y^\prime} X\right)/X$ and thus there exists a section $s : Y \coprod_{Y^\prime} X \rightarrow X$ of $i$ so that
    $$f^\prime = id_X \circ f^\prime = s\circ i \circ f^\prime = s \circ f^{\prime\prime} \circ j.$$
    By commutativity of the diagram
    \begin{center}
      \begin{tikzcd}
        \Hom{\C}{Y,X} \arrow{rr}{\_\, \circ j} \arrow{rd} & & \Hom{\C}{Y^\prime,X} \arrow{ld}\\
        & \Hom{\C/\A}{\pi(Y), \pi(X)} &
      \end{tikzcd}
    \end{center}
    we see that $\pi(s \circ f^{\prime\prime}) = f$ and thus
    $$\Hom{\C}{Y,X} \rightarrow \Hom{\C/\A}{\pi(Y), \pi(X)}$$
    is surjective.
    For injectivity, suppose that $f \colon Y \rightarrow X$ satisfies $\pi(f) = 0$.
    Then $f(Y)$ is an object of $\A$ and from the short exact sequence
    \begin{center}
      \begin{tikzcd}
        0 \arrow{r} & f(Y) \arrow{r}{\im{f}} & X \arrow{r}{\coker{f}} & C \arrow{r} & 0
      \end{tikzcd}
    \end{center}
    we see that $\im{f} = 0$.
    Therefore $f = \im{f} \circ \coim{f} = 0$, as desired.
    
    (\ref{lem1.3}) $\implies$ (\ref{lem1.1}).  Let
    \begin{center}
      \begin{tikzcd}
        0 \arrow{r} & K \arrow{r}{i} & Z \arrow{r}{f} & Y \arrow{r}{p} & C \arrow{r} & 0
      \end{tikzcd}
    \end{center}
    be an exact sequence with $K$ and $C$ objects of $\A$.
    We have the commutative diagram
    \begin{center}
      \begin{tikzcd}
        Z\arrow{d}{f} & & \Hom{\C}{Y,X} \arrow{d}{h_X(f)}\arrow{r}& \Hom{\C/\A}{\pi(Y), \pi(X)} \arrow{d}{h_{\pi(X)}(\pi(f))}\\
        Y & & \Hom{\C}{Z,X} \arrow{r}{\sim}& \Hom{\C/\A}{\pi(Z), \pi(X)}
      \end{tikzcd}
    \end{center}
    with $h_{\pi(X)}(\pi(f))$ an isomorphism because $\pi(f)$ is.
    Therefore $h_X(f)$ is an isomorphism, as desired.
  \end{proof}
\end{lem}

\begin{defn}
  \begin{enumerate}[(i)]
  \item
    If $X$ is an object of $\C$ satisfying any of the conditions in Lemma~\ref{lem1}, then we say that $X$ is $\A$-closed.
  \item
    A morphism $X \rightarrow Y$ is an $\A$-envelope if in the exact sequence
    \begin{center}
      \begin{tikzcd}
        0 \arrow{r} & K \arrow{r} & X \arrow{r} & Y \arrow{r} & C \arrow{r} & 0
      \end{tikzcd}
    \end{center}
    $Y$ is $\A$-closed, and both $K$ and $C$ are objects of $\A$.
  \end{enumerate}
\end{defn}

\begin{lem}\label{lem2}
  If $X$ has a maximal $\A$-subobject, $X_\A$, then $X/X_\A$ is $\A$-torsionfree.
  \begin{proof}
    Let $j \colon Y \rightarrow X/X_\A$ be a monic with $Y$ an object of $\A$.
    We have the commutative diagram
    \begin{center}
      \begin{tikzcd}
        0 \arrow{r} & K \arrow[dashed]{d}{\exists ! h}\arrow{r}{\ker p^\prime} & X \times_{X/X_\A} Y \arrow{d}{i^\prime}\arrow{r}{p^\prime} & Y \arrow{d}{j}\arrow{r}{\coker{p^\prime}} & C \arrow[dashed]{d}{\exists ! h^\prime}\\
        0 \arrow{r} & X_\A \arrow{r}{i} & X \arrow{r}{p} & X/X_\A \arrow{r}{\coker p} & 0
      \end{tikzcd}
    \end{center}
    with $h$ an isomorphism, and $h^\prime$ monic, hence an isomorphism.
    The top row gives us the short exact sequence
    \begin{center}
      \begin{tikzcd}
        0 \arrow{r} & K \arrow{r}{\ker p^\prime} & X \times_{X/X_\A} Y \arrow{r}{p^\prime} & Y \arrow{r}{\coker{p^\prime}} & 0
      \end{tikzcd}
    \end{center}
    with $K$ and $Y$ objects of $\A$, hence $X \times_{X/X_\A} Y$ is also an object of $\A$.
    By maximality of $X_\A$,  $i^\prime$ factors through $i$ uniquely,
    \begin{center}
      \begin{tikzcd}
        X \times_{X/X_\A} Y \arrow{rd}{i^\prime}\arrow[dashed]{rr}{i^{\prime\prime}}& & X_\A \arrow{ld}{i}\\
        & X &
      \end{tikzcd}
    \end{center}
    and so we see
    $$j \circ p^\prime = p \circ i = p \circ (i \circ i^{\prime\prime}) = (p \circ i) \circ i^{\prime\prime} = 0$$
    implies, because $p^\prime$ is epic, that $j = 0$.
    Therefore $X/X_\A$ is $\A$-torsionfree, as desired.
  \end{proof}
\end{lem}

\begin{lem}\label{lem3}
  If $\C$ is such that every object of $\C$ has a maximal $\A$-subobject and every $\A$-torsionfree object has a monomorphism to an $\A$-closed object, then every object of $\C$ has an $\A$-envelope.
  %In particular, if $X$ is an object of $\C$ and $X^\prime$ its maximal $\A$-subobject, then the canonical epimorphism $X \rightarrow X/X^\prime$ is an $\A$-envelope. 
  \begin{proof}
    Let $X$ be an object of $\C$ and let $X_\A$ be its maximal $\A$-subobject, so we have the short exact sequence
    $$\begin{tikzcd}
      0 \arrow{r}& X_\A \arrow{r}{i} & X \arrow{r}{p} & X/X_\A \arrow{r} & 0.
    \end{tikzcd}$$
    By assumption, there exists an $\A$-closed object $Y$ and a short exact sequence
    $$\begin{tikzcd}
      0 \arrow{r} & X/X_\A \arrow{r}{j} & Y \arrow{r}{q} & C \arrow{r} & 0
    \end{tikzcd}$$
    from which we construct the pullback
    $$\begin{tikzcd}
      0 \arrow{r} & K \arrow[dashed]{d}{\exists !h}\arrow{r}{\ker{q^\prime}} & q^{-1}(C_\A) \arrow{r}{q^\prime} \arrow{d}{k^\prime} & C_\A\arrow{d}{k} \arrow{r}&0\\
      0 \arrow{r} & X/X_\A \arrow{r}{j} & Y \arrow{r}{q} & C \arrow{r} & 0,
    \end{tikzcd}$$
    with $h$ an isomorphism.
    Then from the short exact sequence 
    $$\begin{tikzcd}
      0 \arrow{r} & X/X_\A \cong K \arrow{r}{\ker{q^\prime}} & q^{-1}(C_\A) \arrow{r}{q^\prime} & C_\A\arrow{r} & 0
    \end{tikzcd}$$
    it suffices to show that $q^{-1}(C_\A)$ is $\A$-closed.
    
    It's clear that $q^{-1}(C_\A)$ is $\A$-torsionfree because it is a subobject of the $\A$-closed object $Y$.
    If we have any short exact sequence
    $$\begin{tikzcd}
      0 \arrow{r} & q^{-1}(C_\A) \arrow{r}{s} & A \arrow{r}{\coker{s}} & B \arrow{r} & 0
    \end{tikzcd}$$
    with $B$ an object of $\A$, then by Lemma~\ref{lem1.1} there is a unique morphism $\varphi : A \rightarrow Y$ such that 
    $$k^\prime = \varphi \circ s = h_Y(s)(\varphi).$$
    Now we have the commutative diagram
    $$\begin{tikzcd}
      A \arrow[dashed]{rd}{\exists ! r} \arrow[bend right]{ddr}{\varphi} \arrow[bend left]{rrd}{0}\\
      & q^{-1}(C_\A) \arrow{r}{q^\prime} \arrow{d}{k^\prime} & C_\A \arrow{d}{k}\\
      & Y \arrow{r}{q} & C
    \end{tikzcd}$$
    from which we see that 
    $$k^\prime \circ id_{q^{-1}(C_\A)} = k^\prime = \varphi \circ s = (k^\prime \circ r) \circ s = k^\prime (\circ r \circ s)$$
    and thus $r \circ s = id_{q^{-1}(C_\A)}.$
    Therefore $q^{-1}(C_\A)$ is $\A$-closed by Lemma~\ref{lem1}.\ref{lem1.2}, as desired.
  \end{proof}
\end{lem}

\begin{lem}\label{lem4}
  If $\pi \colon \C \rightarrow \C/\A$ has a right adjoint, $\omega \colon \C/\A \rightarrow \C$, then
  \begin{enumerate}
  \item\label{lem4.1}
    for each object $Y$ of $\C$, $\omega\pi(Y)$ is $\A$-closed,
  \item\label{lem4.2}
    for $Y$ an object of $\C$, the morphism $\eta_{\pi(Y)} : \pi\omega\pi(Y) \rightarrow \pi(Y)$ is an isomorphism, and
  \item\label{lem4.3}
    $\omega$ is fully faithful.
  \end{enumerate}
  
  \begin{proof}
    \begin{enumerate}
    \item
      Given an exact sequence 
      \begin{center}
        \begin{tikzcd}
          0 \arrow{r} & K \arrow{r} & Z \arrow{r}{f} & Y \arrow{r} & C \arrow{r} & 0
        \end{tikzcd}
      \end{center}
      with $K$ and $C$ objects of $\A$, we have that $\pi(f)$ is an isomorphism and hence $h_{\pi(Y)}(\pi(f))$ is also an isomorphism.
      From the adjunction we get the commutative diagram
      \begin{center}
        \begin{tikzcd}
          \Hom{\C}{X, \omega\pi(Y)} \arrow{r}{\sim} \arrow{d}{h_{\omega\pi(Y)}(f)} & \Hom{\C/\A}{\pi(X), \pi(Y)} \arrow{d}{h_{\pi(Y)}(\pi(f))}\\
          \Hom{\C}{Z,\omega\pi(Y)} \arrow{r}{\sim} & \Hom{\C/\A}{\pi(Z), \pi(Y)}
        \end{tikzcd}
      \end{center}
      which shows that $h_{\pi(Y)}{\pi(F)}$ is an isomorphism.
      Therefore $\omega\pi(Y)$ is $\A$-closed by part \ref{lem1.1} of Lemma~\ref{lem1}.
    \item
      We have the commutative diagram
      \begin{center}
        \begin{tikzcd}
          %\Hom{\C}{\omega\pi(Y), \omega\pi(Y)} \arrow{r}{\sim} \arrow{d}{h_{\omega\pi(Y)}(\varepsilon_{Y})} & \Hom{\C/\A}{\pi(Y), \pi(Y)} \arrow{d}{h_{\pi(Y)}(\pi(\varepsilon_Y))}\\
          %\Hom{\C}{Y,\omega\pi(Y)} \arrow{r}{\sim} & \Hom{\C/\A}{\pi(Y), \pi(Y)}\\
          \Hom{\C}{\omega\pi(Y),\omega\pi(Y)}\arrow{rr}{\sim}\arrow{rd}{\sim} & & \Hom{\C/\A}{\pi\omega\pi(Y), \pi\omega\pi(Y)}\arrow{ld}{h_{\pi\omega\pi(Y)}(\eta_{\pi(Y)})}\\
          & \Hom{\C/\A}{\pi\omega\pi(Y), \pi(Y)}&  
        \end{tikzcd}
      \end{center}
      since for any morphism $f \colon \omega\pi(Y) \rightarrow \omega\pi(Y)$, the image under the adjunction isomorphism is just $\eta_{\pi(Y)} \circ \pi(f)$.
      This immediately implies that $h_{\pi\omega\pi(Y)}(\eta_{\pi(Y)})$ is an isomorphism, and hence so is $\eta_{\pi(Y)}$.
    \item
      Since $\omega$ being fully faithful is equivalent to $\eta$ being a natural isomorpism, this is a consequence of the definition of $\C/\A$.  
      Indeed, every object of $\C/\A$ is $\pi(X)$ for some object $X$ of $\C$, and the result follows.
    \end{enumerate}
  \end{proof}
\end{lem}

%\begin{cor}\label{cor1}
%  If $\pi \colon \C \rightarrow \C/\A$ has a right adjoint, $\omega \colon \C/\A \rightarrow \C$, then $\omega$ is fully faithful.
%\end{cor}

\begin{thm}\label{thm1}
  The following are equivalent.
  \begin{enumerate}
  \item
    $\pi \colon \C \rightarrow \C/\A$ has a right adjoint, and
  \item
    Every object of $\A$ has a maximal $\A$-subobject and every $\A$-torsionfree object has a monomorphism into an $\A$-closed object.
  \end{enumerate}

  \begin{proof}
    First assume that $\pi$ has a right adjoint, $\omega \colon \C/\A \rightarrow \C$, and let $Y$ be an object of $\C$.
    There are then two natural transformations of adjunction, $\varepsilon : \id_\C \rightarrow \omega\pi$ (unit) and $\eta : \pi\omega \rightarrow \id_{\C/\A}$ (counit), the latter being an isomorphism by Lemma~\ref{lem4}.
    It follows from the commutative diagram
    \begin{center}
      \begin{tikzcd}
        \pi(Y) \arrow[swap]{rd}{\id_{\pi(Y)}} \arrow{r}{\pi(\varepsilon_Y)} & \pi\omega\pi(Y) \arrow{d}{\eta_{\pi(Y)}}\\
        & \pi(Y)
      \end{tikzcd}
    \end{center}
    that $\pi(\varepsilon_Y) = \eta_{\pi(Y)}^{-1}$ is an isomorphism, whence in the short exact sequence    
    \begin{center}
      \begin{tikzcd}
        0 \arrow{r} & K \arrow{r} & Y \arrow{r}{\varepsilon_Y} & \omega\pi(Y) \arrow{r} & C \arrow{r} & 0
      \end{tikzcd}
    \end{center}
    both $K$ and $C$ are objects of $\A$.
    We show that $K$ is the desired subobject.
    Indeed, let $j \colon Y^\prime \rightarrow Y$ be a subobject of $Y$ with $Y^\prime$ and object of $\A$.
    We have the commutative diagram
    \begin{center}
      \begin{tikzcd}
        Y^\prime \arrow{rr}{\varepsilon_Y \circ j} \arrow[swap]{rd}{\coim{(\varepsilon_Y \circ j)}} & & \omega\pi(Y)\\
        & \varepsilon_Y(Y^\prime)\arrow[swap]{ur}{\im{(\varepsilon_Y \circ j})} &
      \end{tikzcd}
    \end{center}
    and we note that because $\omega\pi(Y)$ is $\A$-closed and $\varepsilon_Y(Y^\prime) \cong Y^\prime/\left(Y^\prime \cap K\right)$ is an object of $\A$, the monic $\im{(\varepsilon_Y \circ j)}$ is zero.
    Therefore by the kernel diagram
    \begin{center}
      \begin{tikzcd}
        Y^\prime \arrow[swap,dashed]{rd}{\exists ! j^\prime}\arrow{r}{j}\arrow[bend left]{rr}{0} & Y\arrow{r}{\varepsilon_Y} & \omega\pi(Y)\\
        & K\arrow{u} &
      \end{tikzcd}
    \end{center}
    we see that $j^\prime$ is monic, and $K$ is maximal, as desired. 
    
    Conversely, assume that every object of $\C$ has a maximal $\A$-subobject and every $\A$-torsionfree object has a monomorphism into an $\A$-closed object.
    Let $Y$ be an object of $\C$.
    By Lemma~\ref{lem3}, $Y$ has an $\A$-envelope $Y \rightarrow E$.
    Hence $\pi(Y) \cong \pi(E)$ and by the natural isomorphisms
    $$\Hom{\C}{\,\_\, , E} \cong \Hom{\C/\A}{\pi(\,\_\,), \pi(E)} \cong \Hom{\C/\A}{\pi(\,\_\,), \pi(Y)},$$
    the presheaf $\Hom{\C/\A}{\pi(\_), \pi(Y)}$ on $\C$ is representable.
    Therefore $\pi$ admits a right adjoint.
  \end{proof}
\end{thm}

\begin{defn}
  If $\pi$ has a right adjoint, then we say that $\A$ is a localizing subcategory.
\end{defn}

\begin{cor}\label{cor1}
  Assume that $\pi$ has a right adjoint, $\omega \colon \C/\A \rightarrow \C$.
  Then 
  \begin{enumerate}
  \item
    $\A$-envelopes are unique up to unique isomorphism,
  \item
    for every object $X$ of $\C$, $\omega\pi(X) \cong E$, where $X \rightarrow E$ is an $\A$-envelope of $X$,
  \end{enumerate}
  \begin{proof}
    \begin{enumerate}
    \item
      By the proof of Theorem~\ref{thm1}, an $\A$-envelope of an object $Y$ of $\C$ represents the presheaf $\Hom{\C/\A}{\pi(\,\_\,), Y}$ and thus is unique up to unique isomorphism.
    \item
      This is immediate from Yoneda's Lemma.
    \end{enumerate}
  \end{proof}
\end{cor}

\begin{lem}\label{lem5}
  Assume that $\A$ is a localizing subcategory, $X$, $Y$, objects of $\C$, $X_\A$, $Y_\A$, their maximal $\A$-subobjects.
  A morphism $f \colon X \rightarrow Y$ induces a morphism
  $$\begin{tikzcd}
    X_\A \arrow{r}{i} & X \arrow{r}{f}\arrow{d}{p} & Y\arrow{d}{q}\\
    & X/X_\A \arrow[dashed]{r}{\exists !h} & Y/Y_\A
  \end{tikzcd}$$
  and the morphism $\pi(f)$ is an essential extension if and only if $h$ is.

  \begin{proof}
    We first note that $\pi(p)$ and $\pi(h)$ are isomorphisms, hence essential extensions, so 
    $$\pi(f) = \pi(q)^{-1} \circ \pi(h) \circ \pi(p)$$
    is an essential extension if and only if $\pi(h)$ is.
    Hence it suffices to assume that $X_\A = Y_\A = 0$ and $h = f$.
    
    Assume first that $\pi(f)$ is an essential extension.
    Given a subobject $k : Y^\prime \rightarrow Y$ we get the pullback 
    $$\begin{tikzcd}
      \pi(Y^\prime \times_Y X) \arrow{r}{\pi(k^\prime)}\arrow{d}{\pi(f^\prime)} & \pi(X) \arrow{d}{\pi(f)}\\
      \pi(Y^\prime) \arrow{r}{\pi(k)} &\pi(Y)
    \end{tikzcd}$$
    because $\pi$ is exact.
    We note that so long as $Y^\prime$ is not an object of $\A$, $\pi(Y^\prime)$ is not zero.
    Since $Y$ was assumed to be $\A$-torsionfree, this is equivalent to $Y^\prime$ being non-zero.
    Therefore $\pi(Y^\prime \times_Y X)$ is non-zero whenever $Y^\prime$ is non-zero because $\pi(f)$ is essential and hence so is $Y^\prime \times_Y X$.
    
    Conversely, assume that $f$ is an essential extension.
    Given $i : \pi(Z) \rightarrow \pi(Y)$ a non-zero subobject, we may lift to a morphism 
    $$\begin{tikzcd}
      0 \arrow{r} & K \arrow{r}{\ker{j}} & Z^\prime \arrow{r}{j} & Y
    \end{tikzcd}$$
    with $Z/Z^\prime$ and $K$ objects of $\A$ since $k$ is monic and $Y$ has no non-zero $\A$-subobjects.
    Since $f$ is an essential extension we have the non-zero pullback
    $$\begin{tikzcd}
      Z^\prime / K \times_Y X \arrow{r}{k}\arrow{d}{f^\prime} & X\arrow{d}{f} \\
      Z/K \arrow{r}{\im{j}}& Y
    \end{tikzcd}$$
    As $K$ is an object of $\A$, the short exact sequence
    $$\begin{tikzcd}
      0 \arrow{r} & K \arrow{r}{\ker{j}} & Z^\prime \arrow{r}{\coim{j}} & Z^\prime/K \arrow{r} & 0
    \end{tikzcd}$$
    gives the isomorphism
    $$\pi(Z^\prime/K) \cong \pi(Z^\prime) \cong \pi(Z).$$
    Therefore
    $$\pi(Z^\prime/K \times_Y X) \cong \pi(Z^\prime/K) \times_{\pi(Y)} \pi(X) \cong \pi(Z) \times_{\pi(Y)} \pi(X)$$
    is non-zero, as desired.
  \end{proof}
\end{lem}

\begin{lem}\label{lem6}
  If $Q$ is an $\A$-closed injective, then $\pi(Q)$ is injective.

  \begin{proof}
    Given a short exact sequence
    $$\begin{tikzcd}
      0 \arrow{r} & \pi(Q) \arrow{r}{s} & \pi(X) \arrow{r}{\coker{s}} & \pi(X/Q) \arrow{r} & 0
    \end{tikzcd}$$
    it is enough to show that $s$ is a section; that is, there exists a morphism $r \colon \pi(X) \rightarrow \pi(Q)$ such that $r \circ s = \id_{\pi(Q)}$.
    We can lift $s$ to a morphism 
    $$\begin{tikzcd}
      0 \arrow{r} & K \arrow{r}{\ker{t}} & Q^\prime \arrow{r}{t} & X/X^\prime
    \end{tikzcd}$$
    with $K$, $Q/Q^\prime$, and $X^\prime$ objects of $\A$.
    Since we have assumed that $Q$ is $\A$-closed, the diagram
    $$\begin{tikzcd}
      K \arrow{r}{\ker t}\arrow[bend left]{rr}{0} & Q^\prime \arrow{r}{i} & Q
    \end{tikzcd}$$
    commutes and thus we see that $\ker{t} = 0$ because $i$ is a monomorphism.
    Since $Q$ was assumed to be injective, we have the lift
    $$\begin{tikzcd}
      0 \arrow{r} & Q^\prime \arrow{r}{t} \arrow{d}{i} & X/X^\prime\arrow[dashed]{ld}{\exists r}\\
      & Q.
    \end{tikzcd}$$
    If we let $q \colon X \rightarrow X/X^\prime$ be the canonical projection, then we have the diagram
    $$\begin{tikzcd}
      \pi(Q^\prime)\arrow[swap]{d}{\pi(i)}\arrow{r}{\pi(t)} & \pi(X/X^\prime)\arrow[swap]{ld}{\pi(r)}\\
      \pi(Q)\arrow{r}{s} & \pi(X)\arrow[swap]{u}{\pi(q)}
    \end{tikzcd}$$
    with $\pi(i)$ and $\pi(q)$ isormoprhisms, the top left triangle commutative, and the outer square commutative.
    Therefore
    $$\id_{\pi(Q)} \circ \pi(i) = \pi(i) = \pi(r) \circ \pi(t) = \pi(r) \circ \pi(q) \circ s \circ \pi(i)$$
    implies, because $\pi(i)$ is an isomorphism, that
    $$\left(\pi(r) \circ \pi(q)\right) \circ s = \id_{\pi(Q)},$$
    as desired.
  \end{proof}
\end{lem}

\begin{lem}\label{lem7}
  If $i \colon X \rightarrow E$ is an injective envelope and $X$ is $\A$-torsionfree, then $E$ is $\A$-closed and the morphism $\pi(i) \colon \pi(X) \rightarrow \pi(E)$ is an injective envelope.

  \begin{proof}
    Since $E$ is injective, every short exact sequence
    $$\begin{tikzcd}
      0 \arrow{r} & E \arrow{r} & A \arrow{r} & B \arrow{r} & 0
    \end{tikzcd}$$
    splits.  
    To see that $E$ is $\A$-closed, it then suffices by Lemma~\ref{lem1}.\ref{lem1.2} to show that $E$ is $\A$-torsionfree.
    Given an $\A$-subobject $j : E^\prime \rightarrow E$, we have the pullback 
    $$\begin{tikzcd}
      E^\prime \times_E X \arrow{r}{j^\prime}\arrow{d}{i^\prime} & X\arrow{d}{i}\\
      E^\prime \arrow{r}{j} & E
    \end{tikzcd}$$
    and the morphism $i^\prime$ gives $E^\prime \times_E X$ $E$-subobject structure, hence is an object of $\A$.
    Since $X$ is $\A$-torsionfree by assumption, $E^\prime \times_E X = 0$ and thus $E^\prime$ is also zero because $i$ is essential.
    
    By Lemma~\ref{lem6} we see that $\pi(E)$ is injective, so it remains to show that $\pi(i)$ is essential.
    To see this, we note that the assumption $\A$ is a localizing subcategory in Lemma~\ref{lem5} was only used to produce maximal $\A$-subobjects, and hence the same argument shows that $\pi(i)$ is essential.
    Therefore $\pi(i)$ is an injective envelope.
  \end{proof}
\end{lem}

\begin{prop}\label{prop1}
  Assume that $\A$ is a localizing subcategory.
  If $\C$ has injective envelopes, then
  \begin{enumerate}[(i)]
  \item
    $\C/\A$ has injective envelopes,
  \item
    Every injective object of $\C/\A$ is isomorphic to $\pi(Q)$ for some $\A$-closed injective, $Q$, and
  \item
    Every injective object $Q$ of $\C$ is isomorphic to $E \oplus \omega(Q_2)$, where $Q_\A \rightarrow E$ is an injective envelope of the maximal $\A$-subobject of $Q$ and $Q_2$ is an injective object of $\C/\A$.
  \end{enumerate}
  
  \begin{proof}
    \begin{enumerate}[(i)]
    \item
      Given an object $\pi(X)$ of $\C/\A$, let $X_\A$ be the maximal $\A$-subobject of $X$.
      Since $\C$ has injective envelopes and $X/X_\A$ is $\A$-torsionfree, an injective envelope $X/X_\A \rightarrow E$ gives the injective envelope
      $$\pi(X) \cong \pi(X/X_\A) \rightarrow \pi(E)$$
      by Lemma~\ref{lem7}.
    \item
      Given an injective object $\pi(Q)$ of $\C/\A$, $\omega\pi(Q)$ is $\A$-closed by Lemma~\ref{lem4}.\ref{lem4.1} and is injective because $\pi$ is exact.
      Therefore by Lemma~\ref{lem4}.\ref{lem4.3}, $\pi(Q) \cong \pi(\omega\pi(Q))$.
    \item
      Let $Q$ be an injective object of $\C$, let $i \colon Q_\A \rightarrow Q$ be its maximal $\A$-subobject, and let $j \colon Q_\A \rightarrow E$ be an injective envelope.
      Since $Q$ is injective we have the lift
      $$\begin{tikzcd}
        0 \arrow{r} & Q_\A \arrow{r}{j} \arrow{d}{i}& E \arrow[dashed]{ld}{\exists k}\\
        & Q
      \end{tikzcd}$$
      with $k$ a monomorphism because $j$ is essential and $\ker{(k \circ j)} = \ker{i} = 0$.
      Because $E$ is injective we get the split exact sequence
      $$\begin{tikzcd}
        0 \arrow{r} & E \arrow[shift left=0.5]{r}{k} & Q \arrow[shift left=0.5]{r}{p}\arrow[shift left=0.5ex]{l}{r} & Q/E \arrow[shift left=0.5ex]{l}{s} \arrow{r} & 0
      \end{tikzcd}$$
      so we need only show that $Q/E$ is an $\A$-closed injective, for then $\pi(Q/E)$ is injective by Lemma~\ref{lem6}, and $Q/E \cong \omega\pi(Q/E)$. 
      
      The fact that $Q/E$ is injective follows from the fact that both $Q$ and $E$ are injective.
      Thus every monomorphism out of $Q/E$ splits, so by Lemma~\ref{lem1}.\ref{lem1.2} it is enough to show that $Q/E$ is $\A$-torsionfree.
      Given an $\A$-subobject $\varphi \colon X \rightarrow E$, the fact that $Q/Q_\A$ is $\A$-torsionfree gives the kernel diagram
      $$\begin{tikzcd}
        X \arrow[swap,dashed]{rrd}{\exists !h}\arrow[bend left]{rrr}{0} \arrow{r}{\varphi} & Q/E \arrow{r}{s} & Q \arrow{r} & Q/Q_\A\\
        & & Q_\A\arrow{u}{i}
      \end{tikzcd}$$
      Therefore
      $$\varphi = \id_{Q/E} \circ \varphi = p \circ s \circ \varphi = p \circ i \circ h = p \circ k \circ j \circ h = 0,$$
      as desired.
    \end{enumerate}
  \end{proof}
\end{prop}

\begin{cor}\label{cor2}
  Assume that $\A$ is a localizing subcategory and that $\C$ has injective envelopes.
  If the injective envelope of an object of $\A$ is a morphism of $\A$, then
  \begin{enumerate}[(i)]
  \item\label{cor2.1}
    the maximal subobject of an injective is injective and thus the $\A$-envelope of an injective, $Q$, is $Q \rightarrow Q/Q_\A$, where $Q_\A$ is the maximal $\A$-subobject, 
  \item\label{cor2.2}
    $\pi$ preserves injectives, and
    %      \item
    %        $\pi$ preserves injective envelopes.
  \end{enumerate}
  \begin{proof}
    \begin{enumerate}[(i)]
    \item
      Let $Q$ be an injective object of $\C$ and $i \colon Q_\A \rightarrow Q$ its maximal $\A$-subobject.
      Given an injective envelope $j \colon Q_\A \rightarrow E$, we have the lift
      $$\begin{tikzcd}
        0 \arrow{r} & Q_\A \arrow{d}{i} \arrow{r}{j} & E \arrow[dashed]{ld}{\exists k}\\
        & Q
      \end{tikzcd}$$
      with $k$ a monomorphism because $j$ is essential.
      Since $E$ is assumed to be an object of $\A$, $k$ factors through $i$ uniquely,
      $$\begin{tikzcd}
        E \arrow{rd}{k}\arrow[dashed]{r}{\exists !\varphi} & Q_\A\arrow{d}{i}\\
        & Q.
      \end{tikzcd}$$
      So we see that
      $$k \circ j \circ \varphi = i \circ \varphi = k = k \circ id_E$$
      implies that  $j \circ \varphi = id_E$
      and
      $$i \circ \varphi \circ j = k \circ j = i = i \circ id_{Q_\A}$$
      implies $\varphi \circ j = id_{Q_\A}$.
      Hence $\varphi$ is an isomorphism.
      Therefore by Proposition~\ref{prop1} we have the short exact sequence
      $$\begin{tikzcd}
        0 \arrow{r} & Q_\A \arrow{r} & Q_\A \oplus Q/Q_\A \arrow{r} & Q/Q_\A \arrow{r} & 0
      \end{tikzcd}$$
      and $Q/Q_\A$ is $\A$-closed, as desired.
    \item
      If $Q$ is an injective object of $\C$, then by the above $Q/Q_\A$ is $\A$-closed and hence $\pi(Q) \cong \pi(Q/Q_\A)$ is injective by Lemma~\ref{lem6}.
      %      \item
      %        Given an object $X$, let $i \colon X_\A \rightarrow X$ be the maximal $\A$-subobject of $X$, let $k \colon X \rightarrow E$ be an injective envelope, and let $j \colon E_\A \rightarrow E$ be the maximal $\A$-subobject of $E$.
      %        We have already shown that $\pi(Q)$ is injective, so we need only show that $\pi(i)$ is an essential extension.
      
      %        It's clear that $k^{-1}(E_\A) \cong X_\A$, so we have an induced monomorphism 
      %        $$\begin{tikzcd}
      %          0 \arrow{r} & X_\A \arrow{r}{i}\arrow{d}{k^\prime} & X \arrow{r}{p}\arrow{d}{k} & X/X_\A \arrow{r}\arrow[dashed]{d}{\exists !h} & 0\\
      %          0 \arrow{r} & E_\A \arrow{r}{j} & E \arrow{r}{q} & E/E_\A \arrow{r} & 0
      %        \end{tikzcd}$$
      %Since the left-most square is Cartesian if and only if the sequence
      %$$\begin{tikzcd}
      %  0 \arrow{r} & X_\A \arrow{r}{i} & X \arrow{r}{q \circ k} & E
      %\end{tikzcd}$$
      %is exact, we note that $i = \ker{(q \circ k)}$.
      
      %        Suppose that we are given a morphism $f \colon E/E_\A \rightarrow Y$ such that $f \circ h$ is monic and consider $\ker{f} \colon K \rightarrow E/E_\A$.
      %        We have the pullback
      %        $$\begin{tikzcd}
      %          E_\A \cap q^{-1}(K) \arrow{d}\arrow{r}{\ker{(q \circ \ker{(f \circ q)})}} & q^{-1}(K) \arrow{d}{\ker{(f \circ q)}} \arrow{r}{q^\prime} & K\arrow{d}{\ker{f}}\\
      %          E_\A \arrow{r}{j} & E \arrow{r}{q} & E/E_\A \arrow{r}{f} & Y
      %        \end{tikzcd}$$
      %        Then we see that $h^{-1}(K) \cong K \cap X/X_\A = 0$, so it follows that 
      %        $$(q \circ k)^{-1}(K) = (h \circ p)^{-1}(K) \cong p^{-1}(h^{-1}(K)) = p^{-1}(0) \cong X_\A.$$
      %        Consider
      %        $$\begin{tikzcd}
      %          X_\A \arrow{r}\arrow{d}{i} & q^{-1}(K) \arrow{r}\arrow{d} & K \arrow{d}{\ker{f}}\\
      %          X \arrow[bend right]{rr}{h \circ p}\arrow{r}{k} & E \arrow{r}{q} & E/E_\A \arrow{r}{f} & Y
      %        \end{tikzcd}$$
      %Given a non-zero subobject \begin{tikzcd} 0 \arrow{r} & Z \arrow{r}{\varphi} & E/E_\A\end{tikzcd}, we have
      %  $$p^{-1}(h^{-1}(Z)) \cong (h \circ p)^{-1}(Z) =  (q \circ k)^{-1}(Z) \cong k^{-1}(q^{-1}(Z))$$
      %  and the pullback diagram
      %  $$\begin{tikzcd}
      %    0 \arrow{r} & X_\A \cap k^{-1}(q^{-1}(Z)) \arrow{d}{\varphi^{\prime\prime\prime}}\arrow{r}{i^\prime} & k^{-1}(q^{-1}(Z)) \arrow{r}{k^\prime}\arrow{d}{\varphi^{\prime\prime}} & q^{-1}(Z) \arrow{r}{q^\prime}\arrow{d}{\varphi^\prime} & Z \arrow{d}{\varphi}\\
      %    0 \arrow{r} & X_\A \arrow{r}{i} & X \arrow{r}{k} & E \arrow{r}{q} & E/E_\A
      %  \end{tikzcd}$$
      %  with all squares cartesian.
      %  Since $\varphi$ is not zero, clearly $Z$ is not an object of $\A$ and hence, because $q^\prime$ is epic, it follows that $q^{-1}(Z)$ is not an object of $\A$.
      %  As $k$ is an essential extension, $\varphi^{\prime\prime} : k^{-1}(q^{-1}(Z)) \rightarrow X$ is a non-zero subobject of $X$.
      
      %  that is not $X_\A$ and hence it must be that $h^{-1}(Z)$ is also a non-zero subobject of $X/X_\A$.
      %  This implies that $h$ is an essential extension.
      %  Therefore $\pi(i)$ is an essential extension by Lemma~\ref{lem5}, as desired.
      %  $$\begin{tikzcd}
      %    p^{-1}(h^{-1}(Z)) \arrow{r}{p^\prime}\arrow{d}{\psi^{\prime}} & h^{-1}(Z) \arrow{r}{h^\prime}\arrow{d}{\psi} & Z \arrow{d}{\varphi}\\
      %    X \arrow{r}{p} & X/X_\A \arrow{r}{h} & E/E_\A
      %  \end{tikzcd}$$
    \end{enumerate}
  \end{proof}
\end{cor}

\begin{prop}
  Let $A$ be a right noetherian $\Z_{\geq 0}$ graded algebra over a commutative noetherian ring, $k$.
  If $i \in \Hom{\Gr{A}}{M, N}$ is an essential extension, then
  \begin{enumerate}[(a)]
  \item
    the right bounds of $N$ and $M$ are equal; that is, if there exists some $0 \ll n$ such that $M_d = 0$ for all $n \leq d$, then $N_d = 0$ for all $n \leq d$, and
  \item
    if $M$ is torsion, then so is $N$.
  \end{enumerate}
  \begin{proof}
    See Proposition 2.2 on page 234 of {\it Noncommutative Projective Schemes}.
  \end{proof}
\end{prop}

\begin{rmk}
  In this case, every object $M$ of $\Gr{A}$ has a maximal $\Tors$-subobject, $\tau(M)$, and hence $\Tors$ is a localizing subcategory.
  In particular, every injective object, $Q$, of $\Gr{A}$ decomposes as $\tau(Q) \oplus Q/\tau(Q)$ with $Q/\tau(Q)$ a $\Tors$-closed object, and $\omega\pi(Q) \cong Q/\tau(Q)$ and
  the injective objects of $\QGr{A}$ are precisely $\pi(Q/Q(\tau))$ for $Q$ an injective object of $\Gr{A}$.
\end{rmk}

\section{Cohomology}

\begin{cor}\label{cor3}
  Assume that $\A$ is a localizing subcategory and that $\C$ has injective envelopes.
  If the injective envelope of an object of $\A$ is a morphism of $\A$, then for objects $X$ and $Y$ of $\C$
  $$\Ext{i}{\C/\A}{\pi(X), \pi(Y)} \cong R^i \Hom{\C}{X, \omega\pi(Y)}.$$
  
  \begin{proof}
    Take an injective resolution
    $$\begin{tikzcd}
      Q^\cdot \colon 0 \arrow{r} & Y = Q^0 \arrow{r}{d^0} & Q^1 \arrow{r}{d^1} \arrow{r} & \cdots 
    \end{tikzcd}$$
    of $Y$.
    By Corollary~\ref{cor2}.\ref{cor2.2}, $\pi(Q^\cdot)$ is an injective resolution of $\pi(Y)$.
    Using the natural transformation $\varepsilon \colon id_{\C} \rightarrow \omega\pi$ we have for each $n$ an isomorphism of adjunction
    $$\begin{tikzcd}
      \Phi^n \colon \Hom{\C/\A}{\pi(X), \pi(Q^n)} \arrow{r} & \Hom{\C}{X, \omega\pi(Q^n)}\\
      \varphi \arrow[mapsto]{r} & \omega(\varphi) \circ \varepsilon_X
    \end{tikzcd}$$
    and so we get an isomorphism of chain complexes
    $$\begin{tikzcd}[column sep=large]
      \Hom{\C/\A}{\pi(X), \pi(Q^\cdot)} \colon 0 \arrow{r} & \Hom{\C/\A}{\pi(X),\pi(Y)} \arrow{d}{\Phi^0}\arrow{r}{h^{\pi(X)}(\pi(d^0))} & \Hom{\C/\A}{\pi(X), \pi(Q^1)}\arrow{d}{\Phi^1} \arrow{r}{h^{\pi(X)}(\pi(d^1))} & \cdots\\
      \Hom{\C}{X, \omega\pi(Q^\cdot)} \colon 0 \arrow{r} & \Hom{\C}{X,\omega\pi(Y)} \arrow{r}{h^X(\omega\pi(d^0))} & \Hom{\C}{X, \omega\pi(Q^1)} \arrow{r}{h^X(\omega\pi(d^1))} & \cdots
    \end{tikzcd}$$
    since for $\varphi \in \Hom{\C/\A}{\pi(X), Q^n}$ we have
    $$h^X(\omega\pi(d^n)) \circ \Phi^n(\varphi) = \omega\pi(d^n) \circ \omega(\varphi) \circ \varepsilon_X = \omega(\pi(d^n) \circ \varphi) \circ \varepsilon_X = \Phi^{n+1} \circ h^X(\pi(d^n))(\varphi).$$
    Therefore 
    $$\Ext{i}{\C/\A}{\pi(X), \pi(Y)} \cong h^i\left(\Hom{\C/\A}{\pi(X), \pi(Q^\cdot)}\right) \cong h^i\left(\Hom{\C}{X, \omega\pi(Q^\cdot)}\right) \cong R^i \Hom{\C}{X, \omega\pi(Y)}$$
  \end{proof}
\end{cor}

From here on, let $A$ be a right Noetherian $\Z_{\geq 0}$-graded algebra over a commutative Noetherian ring $k$, $\C = \Gr{A}$, $\A = \Tors$, $\C/\A = \QGr{A}$.

\begin{defn}
  Define the graded modules
  $$\HOM{\Gr{A}}{M,N} = \bigoplus_{d \in \Z} \Hom{\Gr{A}}{M, N[d]}$$ 
  and 
  $$\HOM{\QGr{A}}{\pi(M), \pi(N)} = \bigoplus_{d \in \Z} \Hom{\Gr{A}/A}{\pi(M), \pi(N)[d]}.$$
\end{defn}


\begin{prop}\label{prop2}
  The right derived functors of $\HOM{\Gr{A}}{M,N}$ and $\HOM{\QGr{A}}{\pi(M), \pi(N)}$ are
  $$\EXT{i}{\Gr{A}}{M,N} = \bigoplus_{d \in \Z}\Ext{i}{\Gr{A}}{M,N[d]}$$
  and
  $$\EXT{i}{\QGr{A}}{\pi(M),\pi(N)} = \bigoplus_{d \in \Z}\Ext{i}{\QGr{A}}{\pi(M),\pi(N)[d]}.$$
  Moreover, for $Q^\cdot$ an injective resolution of $N$, 
  $$\EXT{i}{\QGr{A}}{\pi(M), \pi(N)} \cong h^i\left(\HOM{\Gr{A}}{M, \omega\pi(Q^\cdot)}\right) \cong R^i\HOM{\Gr{A}}{M, \omega\pi(N)}.$$
  
  \begin{proof}
    The first is a consequence of the shift operator, $s$, being an automorphism of $\Gr{A}$ and cohomology being an additive functor.
    In particular, if $Q^\cdot$ is an injective resolution of $N$, then we have
    \begin{eqnarray*}
      h^i(\HOM{\Gr{A}}{M, Q^\cdot} &=& h^i\left(\bigoplus_{d \in \Z}\Hom{\Gr{A}}{M, (Q^\cdot)[d]}\right)\\
      &\cong& \bigoplus_{d \in \Z} h^i\left(\Hom{\Gr{A}}{M, (Q^\cdot)[d]}\right)\\
      &=& \bigoplus_{d \in \Z} \Ext{i}{\Gr{A}}{M, N[d]}.
    \end{eqnarray*}

    Similarly, for $\QGr{A}$ we have
    \begin{eqnarray*}
      h^i(\HOM{\QGr{A}}{M, (Q^\cdot)[d]} &=& h^i\left(\bigoplus_{d \in \Z}\Hom{\QGr{A}}{M,(Q^\cdot[d])}\right)\\ 
      &\cong& \bigoplus_{d \in \Z} h^i\left(\Hom{\QGr{A}}{M, (Q^\cdot[d])}\right)\\
      &=& \bigoplus_{d \in \Z} \Ext{i}{\QGr{A}}{M, N[d]}.\\
      &\cong& \bigoplus_{d \in \Z} R^i\Hom{\Gr{A}}{M, \omega\pi(N)[d]}.\\
      &\cong& \bigoplus_{d \in \Z} h^i(\Hom{\Gr{A}}{M, \omega\pi(Q^\cdot)[d]}\\
      &\cong& h^i\left(\bigoplus_{d \in \Z}\Hom{\Gr{A}}{M, \omega\pi(Q^\cdot)[d]}\right)\\
      &=& h^i\left(\HOM{\Gr{A}}{M, \omega\pi(Q^\cdot)}\right)
    \end{eqnarray*}
  \end{proof}
\end{prop}

\begin{prop}
  For any object $M$ of $\Gr{A}$, the canonical morphism
  $$\begin{tikzcd}
    \Hom{\Gr{A}}{A, M} \arrow{r} & M_0\\
    \varphi \arrow[mapsto]{r} & \varphi(1)
  \end{tikzcd}$$
  is an isomorhism.
  In particular, taking degree zero is an exact because $A$ is a projective object in $\Gr{A}$ and, moreover, $\Hom{\Gr{A}}{A, M[d]} \cong M_d$.
\end{prop}

\begin{defn}
  Let $M$ be an object of $\Gr{A}$ and $Q^\cdot$ an injective resolution.
  Define the cohomology functors
  $$H^i(\pi(M)) = \Ext{i}{\QGr{A}}{\pi(A), \pi(M)}
  \cong h^i(\Hom{\Gr{A}}{A, \omega\pi(Q^\cdot)})
  \cong h^i(\omega\pi(Q^\cdot)_0)
  \cong h^i(\omega\pi(Q^\cdot))_0,$$
  and note that the last isomorphism follows from the fact that taking degree 0 is an exact functor.
  Define the graded cohomology functors
  $$\underline{H}^i(\pi(M)) = \bigoplus_{d \in \Z} H^i(\pi(M)[d])
  \cong \bigoplus_{d \in \Z} h^i(\omega\pi(Q^\cdot))_d
  \cong h^i(\omega\pi(Q^\cdot)).$$
\end{defn}

\begin{rmk}
  Note that for $M$ an object of $\Gr{A}$ we have
  $$H^0(\pi(M)) = \Ext{0}{\QGr{A}}{\pi(A), \pi(M)} \cong R^0\Hom{\Gr{A}}{A, \omega\pi(M)} = \Hom{\Gr{A}}{A, \omega\pi(M)} \cong \omega\pi(M)_0$$
  and so it follows that $\underline{H}^0(\pi(M)) \cong \omega\pi(M)$.
\end{rmk}

\begin{prop}
  Let $M$ be an object of $\Gr{A}$ and let $N$ be an object of $\gr{A}$.
  Then
  \begin{enumerate}
  \item
    For $i \geq 0$
    $$\EXT{i}{\QGr{A}}{\pi(N), \pi(M)} \cong \lim_{n \rightarrow \infty} \EXT{i}{\Gr{A}}{N_{\geq n}, M}$$
    and
    $$\underline{H}^i(\pi(M)) \cong \lim_{n \rightarrow \infty} \EXT{i}{\Gr{A}}{A_{\geq n}, M}.$$
  \item
    There is an exact sequence
    $$\begin{tikzcd}
      0 \arrow{r} & \tau(M) \arrow{r} & M \arrow{r} & \underline{H}^0(\pi(M)) \arrow{r} & \lim_{n \rightarrow \infty}\EXT{1}{\Gr{A}}{A/A_{\geq n}, M} \arrow{r} & 0
    \end{tikzcd}$$
    and for $i \geq 1$, 
    $$\underline{H}^i(\pi(M)) \cong \lim_{n \rightarrow \infty}\EXT{i+1}{\Gr{A}}{A/A_{\geq n}, M} \cong h^{i + 1}(\tau(Q^\cdot))$$
    for $Q$ an injective resolution of $M$.
  \item
    $\underline{H}^i(\pi(M))$ is an object of $\Tors$ if $i \geq 1$.
  \item
    $\EXT{i}{\QGr{A}}{\pi(N), \pi(M)}$ and $\underline{H}^i(\pi(M))$ are compatible with direct limits of objects $\pi(M)$.
  \end{enumerate}

  \begin{proof}
    \begin{enumerate}
    \item
      Let $Q^\cdot$ be an injective resolution of $M$.
      For each $0 \leq i$ and for each $d$ we have 
      \begin{eqnarray*}
        \lim_{n \rightarrow \infty}\Hom{\Gr{A}}{N_{\geq n}, Q^i[d]} &\cong& 
        %\lim_{n \rightarrow \infty}\Hom{\Gr{A}}{N_{\geq n}, \tau(Q^i)[d] \oplus Q^i/\tau(Q^i)[d]}\\
        %&\cong& \lim_{n \rightarrow \infty}\Hom{\Gr{A}}{N_{\geq n}, \tau(Q^i)[d]} \oplus \lim_{n \rightarrow \infty} \Hom{\Gr{A}}{N_{\geq n},Q^i/\tau(Q^i)[d]}\\
        %&\cong& \Hom{\QGr{A}}{\pi(N), \pi(\tau(Q^i))[d]} \oplus \Hom{\QGr{A}}{\pi(N), \pi(Q^i/\tau(Q^i)[d])}\\
        %&\cong& 0 \oplus \Hom{\QGr{A}}{\pi(N), \pi(Q^i)[d]}\\
        \Hom{\QGr{A}}{\pi(N), \pi(Q^i)[d]}.
      \end{eqnarray*}
      Since $\Gr{A}$ is an $\operatorname{Ab} 5$ category we have
      \begin{eqnarray*}
        \lim_{n \rightarrow \infty}\HOM{\Gr{A}}{N_{\geq n}, Q^i} &=&
        \lim_{n \rightarrow \infty}\bigoplus_{d \in \Z} \Hom{\Gr{A}}{N_{\geq n}, Q^i[d]}\\
        &\cong& \bigoplus_{d \in \Z} \lim_{n \rightarrow \infty}\Hom{\Gr{A}}{N_{\geq n}, Q^i[d]}\\
        &\cong& \bigoplus_{d \in \Z} \Hom{\QGr{A}}{\pi(N), \pi(Q^i)[d]}\\
        &=&\HOM{\QGr{A}}{\pi(N), \pi(Q^i)}.
      \end{eqnarray*}
      Next we note that
      $$\left(\lim_{n \rightarrow \infty} \HOM{\Gr{A}}{N_{\geq n}, Q^\cdot}\right)_i \cong \lim_{n \rightarrow \infty}\HOM{\Gr{A}}{N_{\geq n}, Q^i} \cong \HOM{\QGr{A}}{\pi(N), \pi(Q^i)}$$ and hence
      \begin{eqnarray*}
        \lim_{n \rightarrow \infty} \EXT{i}{\Gr{A}}{N_{\geq n}, M} &=& 
        \lim_{n \rightarrow \infty} h^i(\HOM{\Gr{A}}{N_{\geq n}, Q^\cdot})\\
        &\cong& h^i\left(\lim_{n \rightarrow \infty} \HOM{\Gr{A}}{N_{\geq n}, Q^\cdot}\right)\\
        &\cong& h^i\left(\HOM{\QGr{A}}{\pi(N), \pi(Q^\cdot)}\right)\\
        &\cong& \EXT{i}{\QGr{A}}{\pi(N), \pi(M)}
      \end{eqnarray*}
      That $\underline{H}^i(\pi(M)) \cong \lim_{n \rightarrow \infty}\EXT{i}{\Gr{A}}{A_{\geq n}, M}$ follows by taking $N = A$.
    \item
      The short exact sequence
      $$\begin{tikzcd}
        0 \arrow{r} & A_{\geq n} \arrow{r} & A \arrow{r} & A/A_{\geq n} \arrow{r} & 0
      \end{tikzcd}$$
      gives rise to a long exact sequence
      $$\begin{tikzcd}
        0 \arrow{r} & \HOM{\Gr{A}}{A/A_{\geq n}, M} \arrow{r} & M \arrow{r} & \HOM{\Gr{A}}{A_{\geq n}, M}\arrow[rounded corners,out = -5, in=170]{lld}\\
        & \EXT{1}{\Gr{A}}{A/A_{\geq n}, M} \arrow{r} & \EXT{1}{\Gr{A}}{A, M} = 0 \arrow{r} & \ldots
      \end{tikzcd}$$
      %We see that 
      %\begin{eqnarray*}
      %  \lim_{n \rightarrow \infty}\HOM{\Gr{A}}{A_{\geq n}, M} &\cong& 
      %  \HOM{\QGr{A}}{\pi(A), \pi(M)}\\
      %  &\cong& \HOM{\Gr{A}}{A, \omega\pi(M)}\\
      %  &\cong& \omega\pi(M)\\
      %  &\cong& \underline{H}^0(\pi(M))
      %\end{eqnarray*}
      and by taking limits, we get from part (a) the exact sequence
      $$\begin{tikzcd}
        0 \arrow{r} & \lim_{n \rightarrow \infty}\HOM{\Gr{A}}{A/A_{\geq n}, M} \arrow{r}{\ker{\varphi}} & M \arrow{r}{\varphi} & \underline{H}^0(\pi(M))\arrow{r}{\coker{\varphi}} & \lim_{n \rightarrow \infty}\EXT{1}{\Gr{A}}{A/A_{\geq n}, M}  \arrow{r} & 0
      \end{tikzcd}$$
      It remains only to show that $\lim_{n \rightarrow \infty}\HOM{\Gr{A}}{A/A_{\geq n}, M} \cong \tau(M)$.
      
      Let $p : A \rightarrow A/A_{\geq n}$ be the canonical epimorphism.
      For each $n$, given $f \in \HOM{\Gr{A}}{A/A_{\geq n}, M}$, $\varphi(f)(1_A) = f \circ p(1_A) \in M$ is torsion; 
      for if $a \in A_{\geq n}$, then 
      $$f \circ p(1_A) \cdot a = f \circ p(a) = f \circ p(0) = 0$$
      and hence the kernel diagram
      $$\begin{tikzcd}
        \HOM{\Gr{A}}{A/A_{\geq n}, M} \arrow[swap,dashed]{rd}{\exists !h_n}\arrow{r}{\ker{\varphi}} & M \arrow{r}{q} & M/\tau(M) \\
        & \tau(M)\arrow{u}{\ker q}
      \end{tikzcd}$$
      with $h_n$ a monomorphism.
      It's clear from the definition that for $m \leq n$ the diagram
      $$\begin{tikzcd}
        \HOM{\Gr{A}}{A/A_{\geq n}, M} \arrow{rd}{h_n} \arrow{rr} & & \HOM{\Gr{A}}{A/A_{\geq m}, M}\arrow{ld}{h_m}\\
        & \tau(M)
        ;    \end{tikzcd}$$
      commutes and hence we have a commutative diagram for each $m \leq n$
      $$\begin{tikzcd}
        \HOM{\Gr{A}}{A/A_{\geq n}, M}\arrow{rd}{i_n}\arrow[bend right]{rdd}{h_n}\arrow[bend right]{rddd} & & \HOM{\Gr{A}}{A/A_{\geq m}, M}\arrow{ld}{i_n}\arrow[bend left]{ldd}{h_n}\arrow[bend left]{lddd}\\
        & \lim_{n \rightarrow \infty}\HOM{\Gr{A}}{A/A_{\geq n}, M}\arrow[dashed]{d}{\exists ! h}\arrow[bend right, dashed, swap,yshift=2ex]{dd}{\exists !h^\prime} &\\
        & \tau(M)\arrow{d}{\ker q} &\\
        & M &
      \end{tikzcd}$$
      We note that by unicity, $h^\prime = \ker{\varphi}$.
      However, we note that because $\omega\pi(M)$ is $\Tors$-closed, we also have a kernel diagram
      $$\begin{tikzcd}
        \tau(M) \arrow[bend left]{rr}{0}\arrow[dashed,swap]{rd}{\exists !g}\arrow{r}{\ker q} & M \arrow{r}{\varphi} & \omega\pi(M)\\
        & \lim_{n \rightarrow \infty} \HOM{\Gr{A}}{A/A_{\geq n}, M} \arrow{u}{\ker{\varphi}}
      \end{tikzcd}$$
      with $g$ a monomorphism.
      Therefore
      $$\ker{\varphi} \circ g \circ h  = \ker{q} \circ h = \ker{\varphi}$$
      and
      $$\ker{q} \circ h \circ g = \ker{\varphi} \circ g = \ker{q}$$
      imply that $h$ is an isomorphism with inverse $g$.

      When $1 \leq i$, the isomorphism 
      $$h^i(\omega\pi(Q^\cdot)) \cong \underline{H}^i(\pi(M)) \cong \lim_{n \rightarrow \infty}\EXT{i+1}{\Gr{A}}{A/A_{\geq n}, M}$$ 
      results from $A$ being projective, part (a), and the limit of the induced long exact sequence of $\operatorname{\underline{Ext}}$.
      From the short exact sequence of complexes
      $$\begin{tikzcd}
        0 \arrow{r} & \tau(Q^\cdot) \arrow{r} & Q^\cdot \arrow{r} & \omega\pi(Q^\cdot) \arrow{r} & 0
      \end{tikzcd}$$
      we get the long exact sequence of cohomology
      $$\begin{tikzcd}
        \cdots \arrow{r} & h^i(\tau(Q^\cdot)) \arrow{r} & h^i(Q^\cdot) \arrow{r} & h^i(\omega\pi(Q^\cdot)) \arrow{r} & h^{i+1}(\tau(Q^\cdot) \arrow{r} & h^{i+1}(Q^\cdot) \arrow{r} & \cdots
      \end{tikzcd}$$
      which gives $h^{i+1}(\tau(Q^\cdot)) \cong h^i(\omega\pi(Q))$ because $Q^\cdot$ is an injective resolution of $M$, hence exact.
    \item
      By the last isomorphism we have the exact sequence
      $$\begin{tikzcd}
        0 \arrow{r} & B^{i+1}(\tau(Q^\cdot)) \arrow{r} & Z^{i + 1} (\tau(Q^\cdot)) \arrow{r} & h^{i+1}(\tau(Q^\cdot)) \cong \underline{H}^i(\pi(M)) \arrow{r} & 0.
      \end{tikzcd}$$
      Since $Z^{i+1}(\tau(Q^\cdot))$ is a subobject of $\tau(Q)$ it is torsion and therefore so too is $\underline{H}^i(\pi(M))$ because $\Tors$ is a Serre subcategory.
    \item
      Note that because filtered colimits commute, by part 1 it suffices to show that for a directed system of objects $\{M_\alpha\}$
      $$\operatorname{colim}_{\alpha}\EXT{i}{\Gr{A}}{N,M_\alpha} \cong \EXT{i}{\Gr{A}}{N,\operatorname{colim}_{\alpha} M_\alpha}.$$
      Let 
      $$\begin{tikzcd}
        P_\cdot : \cdots \arrow{r} & A^{r_2} \arrow{r} & A^{r_1} \arrow{r} & N \arrow{r} & 0
      \end{tikzcd}$$
      be a projective resolution of $N$.
      For each $\alpha$ and for each $d$ we have the complex
      $$\begin{tikzcd}
        \Hom{\Gr{A}}{P_\cdot, M_\alpha[d]} \colon \cdots \arrow{r} & (M_\alpha)_d^{r_2} \arrow{r} & (M_\alpha)_d^{r_1} \arrow{r} & \Hom{\Gr{A}}{N, M_\alpha[d]} \arrow{r} & 0
      \end{tikzcd}$$
      and $\Ext{i}{\Gr{A}}{N, M_\alpha[d]} = h^i\Hom{\Gr{A}}{P_\cdot, M_\alpha[d]}$.
      Since $\Gr{A}$ is Ab 5 the colimit is exact and hence
      \begin{eqnarray*}
        \operatorname{colim}_\alpha \Ext{i}{\Gr{A}}{N, M_\alpha[d]}
        &\cong& \operatorname{colim}_\alpha h^i\Hom{\Gr{A}}{P_\cdot, M_\alpha[d]}\\
        &\cong& h^i\left(\operatorname{colim}_\alpha \Hom{\Gr{A}}{P_\cdot, M_\alpha[d]} \right)\\
        &\cong& h^i\left(\Hom{\Gr{A}}{P_\cdot, \operatorname{colim}_\alpha M_\alpha[d]}\right)\\
        &\cong& \Ext{i}{\Gr{A}}{N, \operatorname{colim}_\alpha M_\alpha[d]}.
      \end{eqnarray*}
      Therefore 
      \begin{eqnarray*}
        \operatorname{colim}_\alpha \EXT{i}{\Gr{A}}{N,M} 
        &\cong& \bigoplus_{d \in \Z}\operatorname{colim}_\alpha \Ext{i}{\Gr{A}}{N, M_\alpha[d]}\\
        &\cong& \bigoplus_{d \in \Z}\Ext{i}{\Gr{A}}{N, \operatorname{colim}_\alpha M_\alpha[d]}\\
        &\cong& \EXT{i}{\Gr{A}}{N, \operatorname{colim}_\alpha M_\alpha}.
      \end{eqnarray*}
      The second assertion follows by taking $A = N$.
    \end{enumerate}
  \end{proof}

  \begin{defn}
    Let $M$ be an object of $\Gr{A}$.
    \begin{enumerate}[(i)]
    \item
      We say $M$ is left bounded if there exists some $d_0$ such that $M_d = 0$ for all $d \leq d_0$.
      We say $M$ is right bounded if there exists some $d_0$ such that $M_d = 0$ for all $d_0 \leq d$.
      We say $M$ is bounded if it is left and right bounded.
    \item
      We say $\chi_i^0(M)$ holds if $\EXT{j}{\Gr{A}}{A_0, M}$ is bounded for all $j \leq i$.
    \item
      If $\chi^0_i(M)$ holds for every object $M$ of $\gr{A}$, then we say that $\chi^0_i$ holds for $A$.
    \item
      If $\chi^0_i(M)$ holds for $A$ for every $i$, then we say that $\chi^0$ holds for $A$.
    \item
      We say that $\chi_i(M)$ holds for an object of $\Gr{A}$ if for all $d$ and all $j \leq i$, there is an integer $n_0$ such that $\EXT{j}{\Gr{A}}{A/A_{\geq n}, M}_{\geq d}$ is an object of $\gr{A}$ when $n_0 \leq n$.
    \item
      If $\chi_i$ holds for every object of $\gr{A}$, then we say that $\chi_i$ holds for $A$.
    \item
      If $\chi_i$ holds for every $i$, then we say that $\chi$ holds for $A$.
    \end{enumerate}
  \end{defn}

  \begin{prop}
    \begin{enumerate}[(i)]
    \item
      Let
      $$\begin{tikzcd}
        0 \arrow{r} & M^\prime \arrow{r} & M \arrow{r} & M^{\prime\prime} \arrow{r} & 0
      \end{tikzcd}$$
      be a short exact sequence of $\Gr{A}$.
      If $\chi^0_i(M^\prime)$ and $\chi^0_i(M^{\prime\prime})$ hold, then $\chi^0_i(M)$ holds.
    \item
      $\chi^0_i$ holds for $A$ if and only if every non-zero object $M$ of $\gr{A}$ has a non-zero subobject $M^\prime \rightarrow M$ such that $\chi^0_i(M^\prime)$ holds.
    \end{enumerate}
    \begin{proof}
      \begin{enumerate}[(i)]
      \item
        We note that from the long exact sequence of $\operatorname{\underline{Ext}}$ it suffices to show that whenever
        $$\begin{tikzcd}
          N^\prime \arrow{r}{g} & N \arrow{r}{f} & N^{\prime\prime}
        \end{tikzcd}$$
        is an exact sequence of $\Gr{A}$, then $N^\prime$ and $N^{\prime\prime}$ bounded imply that $N$ is bounded.
        Let $\ell^\prime, \ell^{\prime\prime}$ and $r^\prime, r^{\prime\prime}$ be the left and right bounds of $N^\prime$ and $N^{\prime\prime}$, and let $\ell = \min\{\ell^\prime, \ell^{\prime\prime}\}$, $r = \max\{r^\prime, r^{\prime\prime}\}$.
        Given elements $n_d \in N_d$ and $n_e \in N_e$ with $d \leq \ell$, $r \leq e$, we see that because $f$ preserves degrees $f(n_d) = 0 = f(n_e)$.
        Hence there exists elements $n^\prime_d \in N^\prime_d$ and $n^\prime_e \in N^\prime_e$ such that $n_d = g(n^\prime_d)$ and $n_e = g(n^\prime_e)$.
        However, both of these elements must be zero, and hence $N_d = N_e = 0$.
        Therefore $N$ is bounded.
      \item
        First note that if $\chi^0_i$ holds for $A$, then the implication is trivial.
        Assume that every finite non-zero object $M$ of $\gr{A}$ has a non-zero subobject $i \colon M^\prime \rightarrow M$ for which $\chi^0_i(M^\prime)$ holds.
        If $j \colon M^{\prime\prime} \rightarrow M$ is another subobject such that $i \leq j$ and  $\chi^0_i(M^{\prime\prime})$ holds, then we could replace $M^\prime$ by $M^{\prime\prime}$.
        Hence we may assume that for any subobject $j \colon M^{\prime\prime} \rightarrow M$ for which $i < j$, $\chi^0_i(M^{\prime\prime})$ does not hold.
        
        By assumption, $M/M^\prime$ has a non-zero subobject $j \colon M^{\prime\prime} \rightarrow M/M^\prime$ for which $\chi^0_i(M^{\prime\prime})$ holds.
        If we let $p \colon M \rightarrow M/M^\prime$ be the canonical epimorphism, then we get the pullback diagram
        $$\begin{tikzcd}
          0 \arrow{r} & K \arrow{r}{\ker{p^\prime}}\arrow[dashed]{d}{\exists !j^{\prime\prime}} & p^{-1}(M^{\prime\prime}) \arrow{d}{j^\prime}\arrow{r}{p^\prime} & M^{\prime\prime}\arrow{d}{j} \arrow{r} & 0\\
          0 \arrow{r} & M^\prime \arrow{r}{i} & M \arrow{r}{p} & M/M^\prime \arrow{r} & 0
        \end{tikzcd}$$
        with $j^{\prime\prime}$ an isomorphism.
        By the first part, we see that $\chi^0_i(p^{-1}(M^{\prime\prime}))$ holds.
        Moreover, $M^\prime$ is a subobject of $p^{-1}(M^{\prime\prime})$ and hence by our assumptions on $M^\prime$, we see that $p^{-1}(M^{\prime\prime}) \cong M^\prime$.
        It follows that $M^{\prime\prime} = 0$ and hence it must be the case that $M/M^\prime = 0$.
        Therefore $M \cong M^\prime$ and $\chi^0_i(M)$ holds, as desired.
      \end{enumerate}
    \end{proof}
  \end{prop}
  
  \begin{thm}[Serre's Finiteness Theorem]
    \begin{enumerate}
    \item
      Let $A$ be a right noetherian $\Z_{\geq 0}$-graded algebra over a commutative noetherian ring $k$ satisfying $\chi$ and let $\pi(M)$ be an object of $\qgr{A}$.
      Then
      \begin{center}
        \begin{flushleft}
          (H4) for every $0 \leq j$, $H^j(\pi(M))$ is a finite right $A_0$-module, and\\
          (H5) for every $1 \leq j$, $\underline{H}^j(\pi(M))$ is right bounded; i.e., there is an integer $d_0$ such that for all $d_0 \leq d$, $H^{j}(\pi(M)[d]) = 0$.
        \end{flushleft}
      \end{center}
    \item
      Conversely, if $A$ satisfies $\chi_1$ and if (H4) and (H5) hold for every $\pi(M) \in \qgr{A}$, then $A$ satisfies $\chi$.
    \end{enumerate}
  \end{thm}
\end{prop}

\end{document}
