\documentclass[10pt]{amsart}
\usepackage{amsmath,amsthm,amssymb,amsfonts,enumerate,mymath,tikz-cd,pbox,mathtools,cite,mathrsfs}
\openup 5pt
\author{Blake Farman\\University of South Carolina}
\title{Student Algebraic Geometry, Commutative Algebra, and Number Theory Seminar}
\date{July 10, 2015}
\pdfpagewidth 8.5in
\pdfpageheight 11in
\usepackage[margin=1in]{geometry}

\begin{document}
\maketitle

\providecommand{\p}{\mathfrak{p}}
\providecommand{\m}{\mathfrak{m}}
\newcommand{\legendre}[2]{\left(\frac{#1}{#2}\right)}
\newcommand{\CC}{\mathscr{C}}
\theoremstyle{plain}
\newtheorem{thm}{Theorem}
\newtheorem{lem}{Lemma}
\theoremstyle{definition}
\newtheorem{defn}{Definition}
\newtheorem{prop}{Proposition}
\newtheorem{cor}{Corollary}
\newtheorem{rmk}{Remark}



\section{The Yoneda Embedding}
Throughout, we assume that $\mathscr{C}$ and $\mathscr{D}$ are locally small categories; that is, $\Hom{\mathscr{C}}{X,Y}$ is a small set, rather than a proper class, for any two objects $X$ and $Y$ of $\mathscr{C}$.
If $X$ is any object of $\mathscr{C}$, define the presheaf of sets (i.e. a contravariant functor from $\mathscr{C}$ to the category of sets) $h_X \colon \mathscr{C} \rightarrow \mathfrak{Set}$ with object map $X \mapsto \Hom{\mathscr{C}}{\line(1,0){10}, X}$ and arrow map sending $f : Z \rightarrow Y$ to the arrow
\begin{align*}
  f^*_X \colon \Hom{\mathscr{C}}{Y,X} &\rightarrow \Hom{\mathscr{C}}{Z,X}\\
  g &\mapsto g \circ f.
\end{align*}
Similarly, define a covariant functor $h^X : \mathscr{C} \rightarrow \mathfrak{Set}$ which takes objects to $\Hom{\mathscr{C}}{X,\, \line(1,0){10}}$ and morphisms to
\begin{align*}
  (f_*)_X \colon \Hom{\mathscr{C}}{X,Z} &\rightarrow \Hom{\mathscr{C}}{X,Y}\\
  g &\mapsto f \circ g.
\end{align*}

\begin{prop}\label{p.1}
  For any two objects $X$ and $Y$ of $\mathscr{C}$, a morphism $f : Y \rightarrow X$ of $\mathscr{C}$ determines a natural transformation $f_* : h_Y \rightarrow h_X$.
  
  \begin{proof}
    %For this, we suppose that we are given a morphism $f : Y \rightarrow X$ of $\mathscr{C}$.
    For any object $Z$ of $\mathscr{C}$, we see $h_Y(Z) = h^Z(Y)$ and $h_X(Z) = h^Z(X)$, so by the above, we have a morphism
    \begin{align*}
      (f_*)_Z \colon h_Y(Z) &\rightarrow h_X(Z)\\
      g &\mapsto f \circ g.
    \end{align*}
    Now we can see that this morphism makes the diagram
    $$\begin{tikzcd}
      Z^\prime\arrow{d}{g} & h_Y(Z)\arrow{d}{g^*_Y}\arrow{r}{(f_*)_Z} & h_X(Z)\arrow{d}{g^*_{X}}\\ 
      Z & h_Y(Z^\prime) \arrow{r}{(f_*)_{Z^\prime}} &  h_X(Z^\prime)
    \end{tikzcd}$$
    commute by chasing an element $\varphi : Z \rightarrow Y$ through the diagram.
    Through the top right, we get
    $$\varphi \mapsto f \circ \varphi \mapsto (f \circ \varphi) \circ g $$
    and through the bottom left, we get
    $$\varphi \mapsto \varphi \circ g \mapsto f \circ (\varphi \circ g) $$
    which are the same since function composition is associative.
  \end{proof}
\end{prop}

\begin{rmk}
  By almost exactly the same proof, a morphism $f : Y \rightarrow X$ of $\mathscr{C}$ determines a natural transformation $f^* : h^X \rightarrow h^Y$.
\end{rmk}


%\begin{prop}
%  For any two objects $X$ and $Y$ of $\mathscr{C}$, a morphism $f : Y \rightarr%ow X$ of $\mathscr{C}$ determines a natural transformation $f^* : h^X \rightarr%ow h^Y$.
%  
%  \begin{proof}
%    For any object $Z$ of $\mathscr{C}$, we see that $h^Y(Z) = h_Z(Y)$ and $h^X%(Z) = h_Z(X)$, so by the above, we have a morphism
%    \begin{align*}
%      f^*_Z : h^Y(Z) &\rightarrow h^X(Z)\\
%      g &\mapsto g \circ f
%    \end{align*}
%    Now we can see that this morphism makes the diagram
%    $$\begin{tikzcd}
%      Z\arrow{d}{g} & h^X(Z)\arrow{d}{(g_*)_X} \arrow{r}{f^*_Z} & h^Y(Z)\arrow{%d}{(g_*)_Y}\\
%      Z^\prime & h^X(Z^\prime) \arrow{r}{f^*_{Z^\prime}} & h^Y(Z^\prime)
%    \end{tikzcd}$$
%    commute by chasing an element $\varphi : X \rightarrow Z$ through the diagr%am.
%    Through the top right, we get
%    $$\varphi \mapsto \varphi \circ f \mapsto g \circ (\varphi \circ f)$$
%    and through the bottom left, we get
%    $$\varphi \mapsto g \circ \varphi \mapsto (g \circ \varphi) \circ f$$
%    which are teh same since function composition is associative.
%  \end{proof}
%\end{prop}

\begin{defn}
  A presheaf $F \colon \mathscr{C} \rightarrow \mathfrak{Set}$ is called {\em representable} if there exists an object $X$ of $\mathscr{C}$ and a natural isomorphism $h_X \cong F$.
  We call $X$ the {\em representing object}, and $h_X$ is called a {\em representation} of $F$.
\end{defn}

\begin{defn}
  Given a functor $F : \mathscr{C} \rightarrow \mathscr{D}$, there is an induced map for each object $X$ and $Y$ of $\mathscr{C}$
  \begin{align*}
    \Hom{\mathscr{C}}{X,Y} &\rightarrow \Hom{\mathscr{D}}{F(X), F(Y)}\\
    f &\mapsto F(f).
  \end{align*}
  We say $F$ is {\em faithful} if this map is injective, {\em full} if this map is surjective, and {\em fully faithful} if this map is a bijection.
\end{defn}

\begin{prop}\label{p.2}
  Let $F \colon \mathscr{C} \rightarrow \mathscr{D}$ be a fully faithful functor.
  A morphism $f \colon X \rightarrow Y$ of $\mathscr{C}$ is an isomorphism if and only if $F(f) \colon F(X) \rightarrow F(Y)$ is an isomorphism of $\mathscr{D}$.

  We say that $F$ {\em reflects isomorphisms}.

  \begin{proof}
    If $f$ is an isomorphism, then $F(f)$ is an isomorphism almost by the definition of a functor: 
    $$1_{F(X)} = F(1_X) = F(f \circ f^{-1}) = F(f) \circ F(f^{-1}) \ \text{and}\ 1_{F(Y)} = F(1_Y) = F(f^{-1} \circ f) = F(f^{-1}) \circ F(f).$$
    Conversely, assume that $F(f)$ is an isomorphism with inverse $g : F(Y) \rightarrow F(X)$.
    Since $F$ is full, there exists some $h : Y \rightarrow X$ such that $g = F(h)$.
    Hence
    $$F(1_X) = 1_{F(X)} = F(f) \circ F(h) = F(f \circ h)$$
    and
    $$F(1_Y) = 1_{F(Y)} = F(h) \circ F(f)= F(h \circ f).$$
    Therefore by faithfulness, $h$ is the inverse of $f$, as desired.
  \end{proof}
\end{prop}
\begin{rmk}
  Note that by possibly passing to the opposite category, every functor is covariant.
\end{rmk}

\begin{lem}[Yoneda]
  If $F \colon \mathscr{C} \rightarrow \mathfrak{Sets}$ is any presheaf of sets, then for any object $X$ of $\mathscr{C}$ of $\mathscr{C}$, there is a natural isomorphism of sets
  $$\operatorname{Nat}(h_X, F) \cong F(X).$$

  \begin{proof}
    First we define the morphisms of sets
    \begin{align*}
      \varphi \colon \Nat{h_x, F} &\rightarrow F(X)\\
      \alpha &\mapsto \alpha_X(1_X).
    \end{align*}
    Next we define a morphism of sets $\psi \colon F(X) \rightarrow \Nat{h_x, F}$ as follows:
    given a point $p \in F(X)$, define for each object $Y$ a morphism
    \begin{align*}
      \beta_Y \colon h_X(Y) &\rightarrow F(Y)\\
      f &\mapsto F(f)(p).
    \end{align*}
    Note that $F$ is a presheaf, so $F(f)$ is a morphism from $F(X)$ to $F(Y)$!
    It's easy to see that this definition makes the naturality square
    $$\begin{tikzcd}
      Z\arrow{d}{g} & h_X(Y)\arrow{d}{g^*}\arrow{r}{\beta_Y} & F(Y)\arrow{d}{F(g)}\\
      Y & h_X(Z)\arrow{r}{\beta_Z} & F(Z)
    \end{tikzcd}$$
    since for any $f \in h_X(Y)$
    $$F(g) \circ \beta_X(f) = F(g) \circ F(f) (p) = F(g \circ f)(p) = F(g^*(f))(p) = \beta_Z \circ g^* (f)$$
    and so we let $\beta : h_X \rightarrow F$ be the image of $p$ under $\psi$.
    We show that these these give the desired isomorphism.

    Given a natural transformation $\alpha \colon h_x \rightarrow F$, the element $\alpha_X(1_X)$ of $F(X)$ determines the natural transformation $\beta : h_x \rightarrow F$ with component morphisms
    \begin{align*}
      \beta_Y \colon h_X(Y) &\rightarrow F(Y)\\
      f &\mapsto F(f)(\alpha_X(1_X)).
    \end{align*}
    Chasing the element $1_X$ through the naturality square
    $$\begin{tikzcd}
      Y\arrow{d}{f} & h_X(X)\arrow{d}{f^*}\arrow{r}{\alpha_X} & F(X)\arrow{d}{F(f)}\\
      X & h_X(Y)\arrow{r}{\alpha_Y} & F(Y)
    \end{tikzcd}$$
    we see that for any object $Y$ of $\mathscr{C}$
    $$\beta_Y(f) = F(f) \circ \alpha_X (1_X) = \alpha_y \circ f^*(1_X) = \alpha_Y(1_x \circ f) = \alpha_Y(f)$$
    implies $\psi \circ \varphi(\alpha) = \alpha$.
    Now if $p \in F(X)$, then the natural transformation $\beta : h_X \rightarrow F$ determined by $p$ satisfies
    $$\varphi(\beta) = \beta_X(1_X)(p) = F(1_X)(p) = 1_{F(x)}(p) = p$$
    so that $\varphi \circ \psi (p) = p$, as desired.
    Therefore $\varphi$ is an isomorphism.

    Next we show that $\varphi$ is natural in $X$.
    For this, we suppose that we are given a morphism $f : Y \rightarrow X$ of $\mathscr{C}$.
    By Proposition~\ref{p.1}, this induces a natural transformation $f_* : h_Y \rightarrow h_X$ and so we obtain by composition a morphism, which we call $f^*$ by abuse of notation,
    \begin{align*}
      f^* \colon \Nat{h_X, F} &\rightarrow \Nat{h_Y,F}\\
      \alpha & \mapsto \alpha \circ f_*.
    \end{align*}
    We check that the naturality square
    $$\begin{tikzcd}
      Y\arrow{d}{f} & \Nat{h_X, F}\arrow{r}{\varphi}\arrow{d}{f^*} & F(X)\arrow{d}{F(f)}\\
      X & \Nat{h_Y, F}\arrow{r}{\varphi} & F(Y)
    \end{tikzcd}$$
    commutes by chasing a natural transformation $\alpha : h_X \rightarrow F$ through.
    Through the top right, we have
    $$\alpha \mapsto \alpha_X(1_X) \mapsto F(f)(\alpha_X(1_X))$$
    and through the bottom left, we have
    $$\alpha \mapsto \alpha \circ f_* \mapsto \alpha_Y \circ (f_*)_Y (1_Y) = \alpha_Y(f).$$
    Now by chasing $1_X$ through the naturality square
    $$\begin{tikzcd}
      Y\arrow{d}{f} & h_X(X)\arrow{r}{\alpha_X}\arrow{d}{f^*} & F(X)\arrow{d}{F(f)}\\
      X & h_X(Y)\arrow{r}{\alpha_Y} & F(Y)
    \end{tikzcd}$$
    we see that
    $$F(f)(\alpha_X(1_X)) = \alpha_Y \circ f^* (1_X) = \alpha_Y(f),$$
    as desired.

    Now, for naturality in $F$ we assume that we are given a natural transformation $\beta : F \rightarrow G$.
    Chasing the natural transformation $\alpha : h_x \rightarrow F$ through the naturality square
    $$\begin{tikzcd}
      F\arrow{d}{\beta} & \Nat{h_X, F} \arrow{r}{\varphi}\arrow{d}{\beta_*} & F(X)\arrow{d}{\beta_X} \\
      G & \Nat{h_X, G} \arrow{r}{\varphi} & G(X)
    \end{tikzcd}$$
    we see that
    $$\beta_X \circ \varphi(\alpha) = \beta_X \circ \alpha_X(1_X) = \varphi( \beta \circ \alpha) = \varphi \circ \beta_* ( \alpha).$$
    Therefore $\varphi$ is an isomorphism of sets, natural in both $X$ and $F$.
  \end{proof}
\end{lem}

This lemma allows us to define the Yoneda functor
$$Y : \mathscr{C} \rightarrow \hat{\mathscr{C}} = \operatorname{Fun}(\CC^\text{op}, \mathfrak{Set})$$
where $\operatorname{Fun}(\CC^\text{op}, \mathfrak{Set})$ is the category with objects presheaves of sets on $\mathscr{C}$ and morphisms natural transformations.
This functor takes an object $X$ of $\mathscr{C}$ to the presheaf $h_X$ and takes a morphism $f : Y \rightarrow X$ to the natural transformation $f_* : h_Y \rightarrow h_X$ of Proposition~\ref{p.1}.

\begin{cor}\label{c.1}
  The Yoneda functor is fully faithful.

  \begin{proof}
    We must show that for any two objects $X$ and $X^\prime$ of $\mathscr{C}$, the map of sets
    $$\Hom{\mathscr{C}}{X,X^\prime} = h_{X^\prime} \rightarrow \Nat{h_X,h_{X^\prime}}$$
    is a bijection.
    Taking $F = h_{X^\prime}$ and applying Yoneda's Lemma gives the desired bijection.
  \end{proof}
\end{cor}

\begin{cor}
  Representing objects are unique up to isomorphism.

  \begin{proof}
    Suppose $F$ is a presheaf of sets on $\mathscr{C}$ and $X, X^\prime$ are objects of $\mathscr{C}$ that represent $F$.
    By definition, we have natural isomorphisms $h_X \cong F$ and $h_{X^\prime} \cong F$, so by composing we obtain a natural isomorphism
    $$Y(X) = h_X \cong F \cong h_{X^\prime} \cong Y(X^\prime).$$
    By Corollary~\ref{c.1} and Proposition~\ref{p.2}, $Y$ reflects isomorphisms and thus $X \cong X^\prime$, as desired.
  \end{proof}
\end{cor}

\section{Fiber Products in $\hat{\mathscr{C}}$}

\begin{defn}[Fiber Products]
  For any category $\mathscr{C}$, given objects $A$, $B$, $C$ of $\mathscr{C}$, and morphisms
  $$\begin{tikzcd}
    A \arrow{r}{f} & C & \arrow[swap]{l}{g} B
  \end{tikzcd}$$
  we define the {\em fiber product} $A \times_C B$ to be an object of $C$ equipped with morphisms called {\em projections} making the diagram
  $$\begin{tikzcd}
    A \times_C B \arrow{d}{p} \arrow{r}{q} & B\arrow{d}{g}\\
    A\arrow{r}{f} & C
  \end{tikzcd}$$
  commute and final amongst all such objects.
  That is to say, for any object $Z$ equipped with projects to $A$ and $B$, the following diagram commutes
  $$\begin{tikzcd}
    Z\arrow[bend right]{ddr}\arrow[bend left]{rrd} \arrow[dotted]{rd}{\exists ! h}& &\\
    & A \times_C B \arrow{r}{q}\arrow{d}{p} & B\arrow{d}{g}\\
    & A\arrow{r}{f} & C
  \end{tikzcd}$$
  Note that by definition $A \times_C B$ is universal and hence unique up to unique isomorphism.
\end{defn}

The category $\mathfrak{Set}$ has all fiber products.  Given set maps $f: A \rightarrow C$ and $g : B \rightarrow C$, it's straightforward to check that
$$A \times_C B = \left\{(a,b) \in A \times B \;\middle\vert\; f(a) = g(b) \in C\right\}$$
satisfies the universal property.

\begin{prop}
  For $\mathscr{C}$ locally small, $\hat{\mathscr{C}}$ has all fiber products.
  
  \begin{proof}
    Given $A$, $B$, and $C$ presheaves on $\mathscr{C}$ and natural transformations
    $$\begin{tikzcd}A \arrow{r}{f} & C & B \arrow[swap]{l}{g}\end{tikzcd}$$
    define the presheaf $A \times_C B$ with object map that takes an object $X$ of $\mathscr{C}$ to $A(X) \times_{C(X)} B(X)$ and arrow map that takes a morphism $\varphi : Y \rightarrow X$ of $\mathscr{C}$ to the unique map of sets
    $$\begin{tikzcd}
      A \times_{C} B(X) \arrow{rr}{q_X}\arrow{dd}{p_X} \arrow[dotted]{rd}{\exists !A \times_C B (\varphi)}& & B(X)\arrow[dashed]{dd}[pos=0.6]{g_X}\arrow{rd}{B(\varphi)}&\\
      & A \times_{C} B(Y)\arrow{dd}[description, pos=0.39]{p_Y} \arrow{rr}[pos=0.4,description]{q_Y} & & B(Y)\arrow{dd}{g_Y}\\
      A(X)\arrow{rd}{A(\varphi)}\arrow[dashed]{rr}[pos=0.4]{f_X} & & C(X)\arrow[dashed]{rd}{C(\varphi)} & \\
      & A(Y)\arrow{rr}{f_Y} & & C(Y)
    \end{tikzcd}$$
    commute.
    We can write this morphism of sets explicitly as
    \begin{align*}
      A \times_C B(\varphi) \colon A \times_C B(X) &\rightarrow A \times_C B (Y)\\
      (a,b) &\mapsto (A(\varphi)(a), B(\varphi)(b))
    \end{align*}
    since the left face of the cube commutes by
    $$p_Y \circ A \times_C B(a,b)  = p_Y(A(\varphi)(a), B(\varphi(b))) = A(\varphi)(a) = A(\varphi) \circ p_X(a,b),$$
    the top face of the cube commutes by
    $$q_Y \circ A \times_C B(a,b)  = q_Y(A(\varphi)(b), B(\varphi(b))) = A(\varphi)(b) = A(\varphi) \circ q_X(a,b),$$
    the back and front faces of the cube commutes by assumption, and the other two faces of the cube commute by naturality of $f$ and $g$.
    Moreover, this shows that $A \times_C B$ is equipped with two natural projections,
    $$\begin{tikzcd}
      A & A \times_C B \arrow{r}{q} \arrow[swap]{l}{p} & B
    \end{tikzcd}$$
    and so it remains only to show that this object is indeed final amongst all such objects of $\hat{\mathscr{C}}$.

    Suppose that we have another presheaf $Z$ equipped with projections
    $$\begin{tikzcd}
      A & Z \arrow{r}{\beta} \arrow[swap]{l}{\alpha} & B
    \end{tikzcd}$$
    Given any object $X$ of $\mathscr{C}$, the morphisms $\alpha_X$ and $\beta_X$ induce by the universal property the morphism of sets
    \begin{align*}
      (\alpha \times \beta)_X \colon Z(X) &\rightarrow A \times_C B(X)\\
      z &\mapsto (\alpha(z), \beta(z)),
    \end{align*}
    making the diagram
    $$\begin{tikzcd}
      Z(X)\arrow[bend left]{rrd}{\beta_X}\arrow[bend right]{rdd}{\alpha_X} \arrow[dotted]{rd}{\exists !(\alpha \times \beta)_X} \\
      & A \times_C B(X) \arrow{r}{q_X}\arrow{d}{p_X} & B(X)\arrow{d}{g_X}\\
      & A(X) \arrow{r}{f_X} & C(X).
    \end{tikzcd}$$
    For any morphism $\varphi \colon Y \rightarrow X$ of $\mathscr{C}$, naturality of $\alpha$ and $\beta$ ensure that the diagram
    $$\begin{tikzcd}
      A(X)\arrow{d}{A(\varphi)} & Z(X)\arrow[swap]{l}{\alpha_X}\arrow{r}{\beta_X}\arrow{d}{Z(\varphi)} & B(X)\arrow{d}{B(\varphi)}\\
      A(Y) & Z(Y)\arrow[swap]{l}{\alpha_Y}\arrow{r}{\beta_Y} & B(Y)
    \end{tikzcd}$$
    commutes.
    This is enough to ensure that the naturality square
    $$\begin{tikzcd}
      Z(X) \arrow[swap]{d}{(\alpha \times \beta)_X}\arrow{r}{Z(\varphi)}& Z(Y)\arrow{d}{(\alpha \times \beta)_Y}\\
      A \times_C B(X) \arrow{r}{A \times_C B(\varphi)} & A \times_C B(Y)
    \end{tikzcd}$$
    commutes because
    \begin{eqnarray*}
      A \times_C B (\varphi) \circ (\alpha \times \beta)_X(z) &=& A \times_C B (\varphi)(\alpha_X(z), \beta_X(z))\\
      &=& (A(\varphi) \circ \alpha_X(z), B(\varphi) \circ \beta_X(z))\\
      &=& (\alpha_Y \circ Z(\varphi)(z), \beta_Y \circ Z(\varphi)(z))\\
      &=& (\alpha \times \beta)_Y \circ Z(\varphi)(z)
    \end{eqnarray*}
    and hence $\alpha \times \beta \colon Z \rightarrow A \times_C B$ is a natural transformation.
    The uniquness follows from the fact that each of the components were defined from the universal property in $\mathfrak{Set}$.
  \end{proof}
\end{prop}

\begin{prop}\label{p.4}
  Let $\mathscr{C}$ be a locally small category.
  Given natural transformations $f \colon h_X \rightarrow h_S$ and \\$g \colon h_Y \rightarrow h_S$, the fiber product $h_X \times_{h_S} h_Y$ in $\hat{\mathscr{C}}$ is representable if and only if the fiber $X \times_S Y$ exists in $\mathscr{C}$ and, in this case, is represented by $X \times_S Y$.
  
  %    While the notation $X \times_S Y$ is potentially ambiguous (the fiber prod%uct explicitly depends on the choice of maps $X \rightarrow S$ and $Y \rightar%row S$), it will become clear in the course of the proof that there can be no %confusion.
  
  \begin{proof}
    We first note that by Yoneda's Lemma, we have two isomorphisms $\Nat{h_X,h_S} \cong h_S(X)$ and $\Nat{h_Y, h_S} \cong h_S(Y)$, so we let $\varphi: X \rightarrow S$ be the image of $f$ and $\psi: Y \rightarrow S$ be the image of $g$.
    Assume that the fibered product
    $$\begin{tikzcd}
      X \times_S Y \arrow{r}{q}\arrow{d}{p} & Y\arrow{d}{\psi}\\
      X \arrow{r}{\varphi} & S
    \end{tikzcd}$$
    exists in $\mathscr{C}$.
    Using the naturality of the Yoneda embedding in the second slot, the natural transformations $f$ and $g$ induce commutative diagrams
    \begin{center}
      \begin{tikzcd}
        h_X(X \times_S Y) \arrow{d}[rotate=90,xshift=-1ex,yshift=1ex]{\sim}\arrow{r}{\varphi_*} & h_S(X \times_S Y)\arrow{d}[rotate=90,xshift=-1ex,yshift=1ex]{\sim}\\
        \Nat{h_{X \times_S Y}, h_X} \arrow{r}{f_*} & \Nat{h_{X \times_S Y}, h_S}
      \end{tikzcd}
      and
      \begin{tikzcd}
        h_Y(X \times_S Y) \arrow{d}[rotate=90,xshift=-1ex,yshift=1ex]{\sim}\arrow{r}{\psi_*} & h_S(X \times_S Y)\arrow{d}[rotate=90,xshift=-1ex,yshift=1ex]{\sim}\\
        \Nat{h_{X \times_S Y}, h_Y} \arrow{r}{g_*} & \Nat{h_{X \times_S Y}, h_S}
      \end{tikzcd}
    \end{center}
    which gives us a commutative diagram
    $$\begin{tikzcd}
      h_{X \times_S Y}\arrow[bend right]{ddr}\arrow[bend left]{rrd}\arrow[dotted]{rd}{\exists !h} \\
      &h_X \times_{h_S} h_Y \arrow{d}\arrow{r}& h_Y\arrow{d}{g}\\
      &h_X\arrow{r}{f} & h_Y
    \end{tikzcd}$$
    where the projections from $h_{X \times_S Y}$ are the images of $p$ and $q$ in $\Nat{h_{X \times_S Y}, h_X}$ and $\Nat{h_{X \times_S Y}, h_Y}$, respectively.
    It is enough to show that each morphism $\gamma : Z \rightarrow X \times_S Y$ comes from an element of $h_X \times_{h_S} h_Y(Z)$, but this follows from the universal property:  the element $h_Z(\gamma)$ is a pair of morphisms $\alpha : Z \rightarrow X$, $\beta : Z \rightarrow Y$ such that the diagram
    $$\begin{tikzcd}
      Z \arrow{r}{\beta}\arrow{d}{\alpha} & Y\arrow{d}{\psi}\\
      X \arrow{r}{\varphi} & S
    \end{tikzcd}$$
    commutes, which of course induces a unique morphism of $h_{X \times_S Y}(Z)$, and this must be $\gamma$.
    Therefore $h_Z$ is an isomorphism of sets, and so $h$ is a natural isomorphism.

    Conversely, suppose that $h_X \times_{h_S} h_Y$ is represented by an object $Z$ of $\mathscr{C}$.
    The projections $h_Z \rightarrow h_X$ and $h_Z \rightarrow h_Y$ induce projections $p : Z \rightarrow X$ and $q : Z \rightarrow Y$ making the diagram
    $$\begin{tikzcd}
      Z \arrow{r}{q}\arrow{d}{p} & Y\arrow{d}{\psi}\\
      X\arrow{r}{\varphi} & S
    \end{tikzcd}$$
    commute.
    Given any other object $W$ equipped projections, $W \rightarrow X$ and $W \rightarrow Y$, we obtain morphisms $\alpha : h_W \rightarrow h_X$ and $\beta : h_W \rightarrow h_Y$ making the diagram
    $$\begin{tikzcd}
      h_W\arrow[dotted]{rd}{\exists !}\arrow[bend left]{rrd}{\beta}\arrow[bend right]{ddr}{\alpha}\\
      & h_Z \arrow{r}\arrow{d}& h_Y\arrow{d}{g}\\
      & h_X \arrow{r}{f}& h_S
    \end{tikzcd}$$
    commute.
    This of course induces a unique morphism $W \rightarrow Z$ by the natural isomorphism
    $$\Nat{h_W, h_Z} \cong h_W(Z).$$
    Therefore, as $Z$ satisfies the universal property, we have $Z \cong X \times_S Y$.
  \end{proof}
\end{prop}

Some nice properties derived from fiber products.

\begin{prop}\label{p.5}
  A morphism $f : A \rightarrow B$ of $\mathscr{C}$ is a monomorphism if and only if the fiber product $A \times_B A$ exists and the diagram
  $$\begin{tikzcd}
    A \arrow{d}{\operatorname{id}_A}\arrow{r}{\operatorname{id}_A} & A\arrow{d}{f}\\
    A\arrow{r}{f} & B
  \end{tikzcd}$$
  is a fiber product diagram.

  \begin{proof}
    First assume that the fiber product exists and the above diagram is a fiber product diagram.
    Given a commutative diagram
    $$\begin{tikzcd}
      Z \arrow[yshift=0.5ex]{r}{g} \arrow[yshift=-0.5ex,swap]{r}{h}& A \arrow{r}{f}& B
    \end{tikzcd}$$
    we have an induced diagram
    $$\begin{tikzcd}
      Z \arrow[bend right]{ddr}{g} \arrow[bend left]{rrd}{h} \arrow[dotted]{rd}{\exists !h^\prime}\\
      & A\arrow{d}{\operatorname{id}_A}\arrow{r}{\operatorname{id}_A} & A\arrow{d}{f}\\
      & A\arrow{r}{f} & B.
    \end{tikzcd}$$
    By assumption, taking $h^\prime = h$ or $h^\prime = g$ makes the diagram commute, so by uniqueness of $h^\prime$ we have $h = g$.
    Therefore $f$ is monic.

    Conversely, assume that $f$ is monic.
    It suffices to show that $A$ equipped with the two identity morphisms as projections satisfies the universal property for the fiber product.
    Given any object $Z$ equipped with projections $g : Z \rightarrow A$ and $h : Z \rightarrow A$ making the diagram
    $$\begin{tikzcd}
      Z \arrow{d}{g} \arrow{r}{h} & A\arrow{d}{f}\\
      A\arrow{r}{f} & B
    \end{tikzcd}$$
    commute, we see that because $f$ is monic,
    $f \circ g = f \circ h$
    and thus $g = h$.
    Therefore we have a unique morphism $g = h$ making the diagram
    $$\begin{tikzcd}
      Z \arrow[bend right]{ddr}{g} \arrow[bend left]{rrd}{h} \arrow[dotted]{rd}{\exists !g = h}\\
      & A\arrow{d}{\operatorname{id}_A}\arrow{r}{\operatorname{id}_A} & A\arrow{d}{f}\\
      & A\arrow{r}{f} & B.
    \end{tikzcd}$$
    commute, as desired.
  \end{proof}
\end{prop}

\begin{cor}
  The Yoneda functor preserves monomorphisms.

  \begin{proof}
    Given a monomorphism $f \colon A \rightarrow B$ of $\mathscr{C}$, we must show that the induced morphism $\hat{f} \colon h_A \rightarrow h_B$ is a monomorphism.
    By Proposition~\ref{p.5} and the fact that $\hat{\operatorname{id}_A} = \operatorname{id}_{h_A}$, this is equivalent to showing that the diagram
    $$\begin{tikzcd}
      h_A \arrow{d}{\operatorname{id}_{h_A}}\arrow{r}{\operatorname{id}_{h_A}} & h_A\arrow{d}{\hat{f}}\\
    h_A \arrow{r}{\hat{f}} & h_B
  \end{tikzcd}$$
    is a fiber product diagram.
    But this follows from Proposition~\ref{p.4}: $h_A \times_{h_B} h_A$ exists and is represented by $A \times_B A \cong A$ if and only if $A \times_B A$ exists.
    Therefore $h_A \times_{h_B} h_A \cong h_{A \times_B A} \cong h_A$, and so it follows that $\hat{f}$ is a monomorphism.
  \end{proof}
\end{cor}

\section{Representable Morphisms}
\begin{defn}
  Let $S$ be a scheme.
  We say $X$ is a {\em scheme over $S$} if $X$ is equipped with a morphism $X \rightarrow S$.
  We denote by $\mathfrak{Sch}/S$ the slice category of $\mathfrak{Sch}$ with objects schemes over $S$ and morphisms $X \rightarrow Y$ making the diagram
  $$\begin{tikzcd}
    X\arrow{r}\arrow{rd} & Y\arrow{d}\\
    & S
  \end{tikzcd}$$
  commute.
  When $S = \Spec{A}$, we will denote this category by $\Sch/A$.
\end{defn}

\begin{defn}
  Let $F \colon \mathscr{C} \rightarrow \mathscr{D}$ be a functor.
  A subfunctor of $F^\prime$ of $F$ is a functor $F \colon \mathscr{C} \rightarrow \mathscr{D}$ equipped with a natural transformation $i : F^\prime \rightarrow F$ such that for each object $X$ of $\mathscr{C}$, the morphism $i_X : F^\prime(X) \rightarrow F(X)$ of $\mathscr{D}$ is a monomorphism; that is, for for any two morphisms $f,g \colon Z \rightarrow F^\prime(X)$ of $\mathscr{D}$ such that the diagram
  $$\begin{tikzcd}
    Z \arrow[yshift=-.5ex,swap]{r}{g}\arrow[yshift=.5ex]{r}{f} & F^\prime(X) \arrow{r}{i_X} & F(X)
  \end{tikzcd}$$
  commutes, $f = g$.
\end{defn}

Let $F$ and $G$ be presheaves of sets on $\mathfrak{Sch}$.
Let $X$ be a scheme and let $g \colon h_X \rightarrow G$ be a morphism in $\hat{\mathfrak{Sch}}$.
Let $f \colon F \rightarrow G$ be a morphism of $\hat{\mathfrak{Sch}}$ and consider the fiber product
$$\begin{tikzcd}
  F \times_G h_X \arrow{r}{p}\arrow{d}{q} & F\arrow{d}{f}\\
  h_X \arrow{r}{g} & G.
\end{tikzcd}$$

\begin{defn}
  A morphism $f \colon F \rightarrow G$ is called representable if for all schemes $X$ and all morphisms $g : h_X \rightarrow G$, the presheaf $F \times_G h_X$ is representable.
\end{defn}

Let $F$ be a presheaf of sets on $\mathfrak{Sch}$.
If $i : U \rightarrow X$ is an open immersion of $S$-schemes (e.g. $U$ is an open subset of $X$ and $i$ is the induced immersion), the morphism of sets
$$F(i) : F(X) \rightarrow F(U)$$
is called the {\em restriction morphism}.
We think of elements $\varphi \in F(X)$ as functions on $X$ and as such we denote $F(i)(\varphi)$ by  $\varphi\mid_U$ for simplicity of notation, which we regard as a function on $U$.

\begin{rmk}
  This mode of thought is justified in the case of representable functors, which will be of the most interest to us.  When $F$ is representable, it is isomorphic to $h_Y$ for some scheme $Y$, and so an element
  $$\varphi \in F(X) \cong h_Y(X) = \Hom{\mathfrak{Sch}/S}{X,Y}$$
  is, legitimately, a morphism of $S$-schemes, $\varphi : X \rightarrow Y$, and the element $F(i)(\varphi)$ is the just restriction,
  $$\begin{tikzcd}
    U \arrow{r}{i} & X\arrow{r}{\varphi} & Y.
  \end{tikzcd}$$
\end{rmk}

\begin{defn}
  We say that a presheaf $F$ on $\mathfrak{Sch}$ is a {\em Zariski sheaf} (or a {\em sheaf for the Zariski topology}) if for every $S$-scheme $X$ and for every open cover $\{U_i\}_{i \in I}$, the following axioms are satisfied:
  \begin{enumerate}[(i)]
  \item
    If $\xi_1, \xi_2 \in F(X)$ satisfy $\xi_1\mid_{U_i} = \xi_2\mid_{U_i}$ for each $i$, then $\xi_1 = \xi_2$, and
  \item
    If for each $i$ we have $\xi_i \in F(U_i)$ satisfying $\xi_i\mid_{U_i \cap U_j} = \xi_j\mid_{U_i \cap U_j}$, then there exists some $\xi \in F(X)$ such that $\xi\mid_{U_i} = \xi_i$ for each $i$.
  \end{enumerate}
\end{defn}

In light of this definition we have the following corollary, which is immediate for those familiar with, say, section II.3 of Hartshorne:

\begin{cor}
  Every representable presheaf on $\mathfrak{Sch}/S$ is a Zariski sheaf.
  \begin{proof}
    It suffices to assume the functor in question is $h_Y$ for some $S$-scheme $Y$.
    For each $S$-scheme $X$, this is exactly the statement that for an open cover $\{U_i\}_{i \in I}$ of $X$, a collection of morphisms $f_i \colon U_i \rightarrow Y$ glue to a unique $S$-morphism $f : X \rightarrow Y$ such that $f\mid_{U_i} = f_i$.
  \end{proof}
\end{cor}

\begin{defn}
  Let $F$ and $G$ be presheaves on $\mathfrak{Sch}/S$ and let $f \colon F \rightarrow G$ be a representable morphism.
  For every $S$-scheme $X$ and every natural transformation $h_X \rightarrow G$, there exists and $S$-scheme $Z$, depending on $X$, such that $F \times_G h_X \cong h_Z$.
  We say that $f$ is an {\em open immersion} if every projection $h_Z \cong F \times_G h_X \rightarrow h_X$ induces an open immersion of $S$-schemes, $Z \rightarrow X$.
  
  If $F$ is also a subfunctor of $G$, then we say that $F$ is an {\em open subfunctor} of $G$.
\end{defn}

\begin{defn}
  A family $\{f_i : F_i \rightarrow F\}_{i \in I}$ of open subfunctors is called a {\em Zariski open covering} of $F$ if for every $S$-scheme $X$ and every morphism $g \colon h_X \rightarrow F$, the images of the morphisms of $S$-schemes induced by the projections
  $$F_i \times_F h_X \rightarrow h_X$$
  cover $X$.
\end{defn}

\begin{thm}
  Let $F$ be a Zariski sheaf of sets on $\mathfrak{Sch}/S$.
  If $F$ admits a Zariski open covering
  $$\{f_i : F_i \rightarrow F\}_{i \in I}$$
  by representable functors $F_i$, then $F$ is a representable functor.

  \begin{proof}
    For each $i$, let $X_i$ be the scheme representing $F_i$.
    We will show that we may glue these schemes to a scheme $X$ and that this scheme represents $F$.

    In order to glue the $X_i$ together, we must produce for each $i$ and $j$ open subsets $U_{i,j} \subseteq X_i$ and isomorphisms of schemes $\varphi_{i,j} : U_{i,j} \rightarrow U_{j,i}$ satisfying
    \begin{enumerate}[(i)]
    \item
       $\varphi_{i,j}^{-1} = \varphi_{j,i}$,
    \item
      $\varphi_{i,j}(U_{i,j} \cap U_{i,k}) = U_{j,i} \cap U_{j,k}$ for all $i$, $j$, and $k$, and
    \item
      the diagram
      $$\begin{tikzcd}
        U_{i,j} \cap U_{i,k} \arrow{rd}{\varphi_{i,k}}\arrow{r}{\varphi_{i,j}} & U_{j,i} \cap U_{j,k}\arrow{d}{\varphi_{j,k}}\\
        & U_{k,j} \cap U_{k,i}
      \end{tikzcd}$$
      commutes for all $i$, $j$, and $k$.
    \end{enumerate}

    For each $i$ and $j$, let $F_{i,j}$ denote the fiber product $F_i \times_F F_j$.
    Each $F_{i,j}$ is, by assumption, a representable Zariski sheaf.
    If we let $X_{i,j}$ be the scheme representing $F_{i,j}$, then the induced morphism of schemes,\\ $X_{i,j} \rightarrow X_i$ is an open immersion, and hence a monomorphism of $\mathfrak{Sch}/S$.
    Since the Yoneda functor preserves monomorphisms, it follows that the projections $F_{i,j} \rightarrow F_i$ and $F_{i,j} \rightarrow F_j$ are also monomorphisms; that is, we may identify $F_{i,j}$ as a subfunctor of $F$.
    In particular, for each $S$-scheme $T$, we may identify each $F_i(T)$ as a subset of $F(T)$, and thus identify $F_{i,j}(T)$ with $F_i(T) \cap F_j(T) \subseteq F(T)$.
    Moreover, we see from the diagram
    $$\begin{tikzcd}
      F_{i,j}\arrow[dotted]{rd}{\exists !}\arrow[bend right]{ddr} \arrow[bend left]{rrd}\\
      & F_{j,i}\arrow{r}\arrow{d} & F_i\arrow{d}{f_i}\\
      & F_{j}\arrow{r}{f_j} & F
    \end{tikzcd}$$
    that under these identifications, the $T$ component of the induced morphism is simply the identity map
    $$F_i(T) \cap F_j(T) \rightarrow F_j(T) \cap F_i(T),$$
    giving the identification $F_{i,j} = F_{j,i}$.
    In particular, note that this immediately implies that $F_{i,i} = F_i$.
    Similarly, we may identify the fiber products $F_{i,j,k} = F_i \times_F F_j \times_F F_k$ for each permutation of the ordered set ${i,j,k}$ by making the identification $F_{i,j,k}(T) = F_i(T) \cap F_j(T) \cap F_k(T)$.

    Now, the open immersions $X_{i,j} \rightarrow X_i$ induce an isomorphism of schemes
    $$\psi_{i,j} : X_{i,j} \rightarrow U_{i,j}$$
    with $U_{i,j}$ an open subset of $X_i$ and we define isomorphisms
    $$\begin{tikzcd}
      U_{i,j} \arrow[swap]{rd}{\varphi_{i,j}}\arrow{r}{\psi_{i,j}^{-1}} & X_{i,j} \cong X_{j,i}\arrow{d}{\psi_{j,i}}\\
      & U_{j,i}
    \end{tikzcd}$$
    which clearly satisfy $\varphi_{i,j}^{-1} = \varphi_{j,i}$.
    Identifying the schematic intersections $U_{i,j} \cap U_{i,k}$ with the fiber product $X_{i,j} \times_{X_i} X_{i,k}$ yields an isomorphism $h_{U_{i,j} \cap U_{i,k}} \cong F_{i,j} \times_{F_{i}} F_{i,k}$.
    The diagram
    $$\begin{tikzcd}
      F_{i,j} \times_{F_i} F_{i,k}\arrow[bend right]{rdd}\arrow[bend left]{rrd}\arrow[dotted]{rd}{\exists !}\\
      & F_{i,j,k}\arrow{r}\arrow{d} & F_{i,j}\arrow{d}\\
      & F_{i,k}\arrow{r} & F
    \end{tikzcd}$$
    implies that for any $S$-scheme $T$, the $T$ component of the induced morphism identifies the intersections
    $$(F_i(T) \cap F_j(T)) \cap (F_i(T) \cap F_k(T)) \rightarrow F_i(T) \cap F_j(T) \cap F_k(T),$$
    where on the left-hand side the parenthetical intersections are taking inside $F(T)$, and then considered as the intersection of subsets of $F_i(T)$, and agree with the three intersections taken within $F(T)$ and considered as a subset of $F_i(T)$.
    This allows us to identify $h_{U_{i,j} \cap U_{i,k}}$ with $F_{i,j,k}$.
    This induces a commutative diagram
    $$\begin{tikzcd}
      h_{U_{i,j} \cap U_{i,k}}\arrow{d}[rotate=90,yshift=1ex,xshift=-1ex]{\sim}\arrow{r} & h_{U_{j,i} \cap U_{j,k}}\arrow{d}[rotate=90,yshift=1ex,xshift=-1ex]{\sim}\\
      F_{i,j,k} \arrow{d}\arrow{r}{\sim} & F_{j, i, k}\arrow{d}\\
      F_{i,j} \arrow{r}{\sim} & F_{j,i}
    \end{tikzcd}$$
    which gives $\varphi_{i,j}(U_{i,j} \cap U_{i,k}) = U_{j,i} \cap U_{j,k}$ since $\varphi_{i,j}$ was induced by the bottom isomorphism.
    The composition condition is verified by the observation that
    $$\varphi_{j,k} \circ \varphi_{i,j} (U_{i,j} \cap U_{i,k}) = \varphi_{j,k}(U_{j,i} \cap U_{j,k}) = U_{k,i} \cap U_{k,j} = U_{k,j} \cap U_{k,i} = \varphi_{i,k}(U_{i,j} \cap U_{i,k}).$$
    Thus we may glue the schemes $X_i$ to a scheme $X$.% that is equipped with morphisms $\xi_i: X_i \rightarrow X$ for each $i$ such that
    %\begin{enumerate}[(i)]
    %\item
    %  $\xi_i$ is an open immersion onto an open subset $U_i$ of $X$, 
    %\item
    %  the $U_i = \xi_i(X_i)$ cover $X$,
    %\item
    %  $\xi_i(U_{i,j}) = \xi_i(X_i) \cap \xi_j(X_j) = U_i \cap U_j$, and
    %\item
    %  $\xi_i = \xi_j \circ \varphi_{i,j}$ on $U_{i,j}$.
    %\end{enumerate}

    Now, we note that by identifying the $X_i$ with open subschemes of $X$ by way of the induced open immersion $\imath_i : X_i \rightarrow X$, the open immersions $\imath_{i,j} : X_i \cap X_j \rightarrow X_i$ correspond to the morphisms $F_{i,j} \rightarrow F_i$.
    Now from the naturality square
    $$\begin{tikzcd}
      F_i(X_i) \arrow{r}\arrow{d}{f_{i,X_i}}& F_i(X_i \cap X_j)\arrow{d}{f_{i,X_i \cap X_j}}\\
      F(X_i) \arrow{r} & F(X_i \cap X_j)
    \end{tikzcd}$$
    and the fiber product diagram
    $$\begin{tikzcd}
      F_{i,j} \arrow{r}\arrow{d}& F_i\arrow{d}\\
      F_j \arrow{r} & F
    \end{tikzcd}$$
    we have
    $$f_{i,X_i}(1_{X_i})\mid_{X_i \cap X_j} = f_{i,X_i \cap X_j}(\imath_{i,j}) = f_{j,X_i \cap X_j}(\imath_{i,j}) = f_{j, X_j}(1_{X_j}).$$
    Since $F$ is a Zariski sheaf, the elements $f_{i,X_i}(1_{X_i})$ glue to give an element $f \in F(X)$.
    Let $\eta : h_X \rightarrow F$ be the image of $f$ under the isomorphism
    $$F(X) \cong \Nat{h_X, F}.$$
    We note that this gives the factorization $f_i = \eta \circ \hat{\imath_i}$, where $\hat{\imath_i} : F_i \rightarrow h_X$ is the morphism induced by $\imath_i$ and has components
    \begin{align*}
      \hat{\imath_i}_Z \colon F_i(Z) &\rightarrow h_X(Z)\\
      \varphi &\mapsto \imath_i \circ \varphi.
    \end{align*}
    The equality is verified by checking that for any $S$-scheme, $Z$, and any $S$-morphism $\varphi : X_i \rightarrow Z$, we have
    \begin{eqnarray*}
      \eta_Z \circ \hat{\imath_i}_Z(\varphi) &=& \eta_Z(\imath_i \circ \varphi)\\
      &=& F(\imath_i \circ \varphi)(f)\\
      &=& F(\varphi) \circ F(\imath_i)(f)\\
      &=& F(\varphi)(f\mid_{X_i})\\
      &=& F(\varphi) \circ f_{i,X_i}(1_{X_i})\\
      &=& f_{i,Z} \circ F_i(\varphi)(1_{X_i})\\
      &=& f_{i,Z}(\varphi).
    \end{eqnarray*}
    

    To see that $\eta$ is an isomorphism, let $Z$ be a scheme over $S$.
    Given $\zeta \in F(Z)$, we have the natural transformation $\nu \in \Nat{h_Z, F}$ with components
    \begin{align*}
      \nu_T \colon h_Z(T) &\rightarrow F(T)\\
      \varphi &\mapsto F(\varphi)(\zeta).
    \end{align*}
    Since we have assumed that the $f_i$ are a Zariski open cover of $F$, there is an induced open cover $\{V_i\}_I$ of $Z$ such that for each $i$, the diagram
    $$\begin{tikzcd}
      h_{V_i} \arrow{r}{p_i}\arrow{d}{q_i} & F_i\arrow{d}{f_i}\\
      h_Z \arrow{r}{\nu} & F
    \end{tikzcd}$$
    is a fiber product diagram; that is, $h_{V_i} \cong F_i \times_F h_Z$.
    Let $j_i = q_{i,V_i} : V_i \rightarrow Z$ be the open immersion corresponding to the inclusion $V_i \subseteq Z$.
    %If we let $\hat{\imath_i} : F_i \rightarrow h_X$ be the natural transformation induced by the inclusion $\imath_i : X_i \rightarrow X$, we have the commutative diagram
    $$\begin{tikzcd}
      h_{V_i}(V_i) \arrow{r}{p_{i,V_i}} \arrow{d}{q_{i,V_i}} & F_i(V_i)\arrow{d}{f_{i,V_i}} \arrow{r}{\hat{\imath_i}} & h_X(V_i)\arrow{ld}{\eta_{V_i}}\\
      h_Z(V_i)\arrow{r}{\nu_{V_i}} & F(V_i)
    \end{tikzcd}$$
    %since the diagram
    %$$\begin{tikzcd}
    %  F_i(X_i) \arrow{r}{f_{i,X_i}} \arrow[swap]{d}{F_i(p_i(1_{V_i}))} & F(X_i)\arrow{d}{F(p_i(1_{V_i}))}\\
    %  F_i(V_i) \arrow{r}{f_{i,V_i}} & F(V_i)
    %\end{tikzcd}$$
    commutes, which gives
    \begin{eqnarray*}
      \eta_{V_i}\left(\imath_i \circ p_{i,V_i}(1_{V_i})\right) &=& F(\imath_i \circ p_{i,V_i}(1_{V_i}))(f)\\
      &=& F(p_{i,V_i}(1_{V_i})) \circ F(\imath_i)(f)\\
      &=& F(p_{i,V_i}(1_{V_i}))(f\mid_{X_i})\\
      &=& F(p_{i,V_i}(1_{V_i})) \circ f_{i,X_i}(1_{X_i})\\
      &=& f_{i,V_i} \circ F_i(p_{i,V_i}(1_{V_i})(1_{X_i})\\
      &=& f_{i,V_i} \circ p_{i,V_i}(1_{V_i})\\
      &=& \nu_{V_i} \circ q_{i,V_i}(1_{V_i})\\
      &=& \nu_{V_i}(j_i).
    \end{eqnarray*}
    We wish to rephrase this in terms of restriction maps.
    We observe that by definition,
    $$\nu_{V_i}(j_i) = F(j_i)(\zeta) = \zeta\mid_{V_i}$$
    Now, using the restriction maps
    $$\begin{tikzcd}
      F(X) \arrow{r}{F(\imath_i)} \arrow[swap]{rd}{F(\imath_i \circ p_{i,V_i}(1_{V_i}))} & F(X_i)\arrow{d}{F(p_{i,V_i}(1_{V_i}))}\\
      & F(V_i)
    \end{tikzcd}$$
    we obtain the more intuitive equality
    $$\zeta\mid_{V_i} = \eta_{V_i}\left(\imath_i \circ p_{i,V_i}(1_{V_i})\right) = F(p_{i,V_i}(1_{V_i})) \circ F(\imath_i)(f) = F(p_{i,V_i}(1_{V_i}))(f\mid_{X_i}) = \left(f\mid_{X_i}\right)\mid_{V_i} = f\mid_{V_i}.$$
    Since $f_{i,V_i} : F_i(V_i) \rightarrow F(V_i)$ is an injection, we can identify $p_{i,V_i}(1_{V_i})$ with $\zeta\mid_{V_i}$, and thus identify $\zeta\mid_{V_i}$ as an element of $h_X(V_i)$ under the inclusion
    $$\hat{\imath_i}_{V_i} \colon F_i(V_i) \rightarrow h_X(V_i).$$
    By assumption
    $$(\zeta\mid_{V_i})\mid_{V_i \cap V_k} = \zeta\mid_{V_i \cap V_k} = (\zeta\mid_{V_k})\mid_{V_i \cap V_k},$$
    so it follows that these $\zeta\mid_{V_i}$ glue to a unique element of $\xi \in h_X(Z)$ such that $\xi\mid_{V_i} = \imath_i \circ p_{i,V_i}(1_{V_i})$.
    By commutativity of the naturality square
    $$\begin{tikzcd}
      h_X(Z) \arrow{r}{\eta_Z}\arrow{d} & F(Z)\arrow{d}\\
      h_X(V_i) \arrow{r}{\eta_{V_i}} & F(V_i)
    \end{tikzcd}$$
    it follows that
    $$\eta_Z(\xi)\mid_{V_i} = \eta_{V_i}(\xi\mid_{V_i}) = \eta_{V_i} \circ \imath_i \circ p_{i,V_i}(1_{V_i}) = f_{i,V_i} \circ p_{i,V_i}(1_{V_i}) = \zeta\mid_{V_k}$$
    and so $\eta_Z(\xi) = \zeta$ by the Zariski sheaf condition.
    Hence every element $\zeta \in F(Z)$ is the image of a unique $\xi \in h_X(Z)$ and so $\eta_Z$ is a bijection (i.e. an isomorphism of sets).
    Therefore $\eta$ is a natural isomorphism, as desired.

    %By the isomorphisms $F_i \cong h_{X_i}$ and  $\Nat{h_{X_i}, F} \cong F(X_i)$, we may consider each of the $f_i : F_i \rightarrow F$ as an element of $F(X_i)$, $x_i = f_{i,X_i}(\operatorname{id}_{X_i})$.
    %We consider the image of $x_i$ under the morphism
    %$$\Nat{F_i, F} \cong F(X_i) \rightarrow F(U_{i,j}) \cong \Nat{F_{i,j}, F}$$
    %which is the morphism induced by the projection $F_{i,j} \rightarrow F_{i}$. 
    %Since the projection $F_{i,j} \rightarrow F_i$ is the same as the projection $F_{j,i} \rightarrow F_i$, we see that the restrictions $f_i \mid_{U_i \cap U_j}$ and $f_j \mid_{U_i \cap U_j}$ must agree and so we may glue the $f_i$ to a morphism $h_X \rightarrow F$.  It's routine to check that for each $S$-scheme $T$, this is an isomorphism.
    %Therefore we have $h_X \cong F$ is representable, as desired.
  \end{proof}
\end{thm}

\section{Affine Restriction}
Consider a scheme $X$ over an affine scheme $\Spec{A}$.
We can consider the coslice category $A-\mathfrak{alg}$, with objects ring morphisms $A \rightarrow B$ and morphisms $B \rightarrow C$ such that the diagram
$$\begin{tikzcd}
  A\arrow{r}\arrow{rd} & B\arrow{d}\\
  & C
\end{tikzcd}$$
commute and we define the restriction of the presheaf $h_X$ to the category $A-\mathfrak{alg}$ by
$$h_X^*(B) =  h_X(\Spec{B}) = \Hom{\mathfrak{Sch}/A}{\Spec{B}, X}.$$

\begin{prop}
  The functor
  \begin{align*}
    \Sch/A &\rightarrow \operatorname{Fun}({A-\mathfrak{alg}}, \Sets)\\
    X &\mapsto h_X^*
  \end{align*}
  is fully faithful.
  \begin{proof}
    First we describe the map
    $$\Hom{\Sch/A}{X,Y} \rightarrow \Nat{h_X^*, h_Y^*}.$$
    Let $f \in \Hom{\Sch/A}{X,Y}$ be given.
    Just as with the Yoneda embedding, $f$ determines for each $A$-algebra, $B$, a map
    \begin{align*}
      \Hom{\Sch/A}{\Spec{B},X} &\rightarrow \Hom{\Sch/A}{\Spec{B},Y}\\
      g &\mapsto f \circ g,
    \end{align*}
    natural in $B$.
    For injectivity, let $f,g : X \rightarrow Y$ be given.
    Let $\eta^*$, $\nu^* : h_X \rightarrow h_Y$ be the natural transformations associated to $f$, $g$, respectively, and assume that $\eta^* = \nu^*$.
    If $i : U \rightarrow X$ is the inclusion of an open affine subset, we have
    $$f\mid_U = f \circ i = \eta^*_U(i) = \nu^*_U(i) = g \circ i = g\mid_U$$
    so that $f = g$.
    
    Now assume that we are given a natural transformation $\eta^* \colon h_X^* \rightarrow h_Y^*$.
    We will show that we may extend $\eta^*$ to a natural transformation $\eta : h_X \rightarrow h_Y$, so that we have a surjection
    $$\Hom{\Sch/A}{X,Y} \cong \Nat{h_X, h_Y} \twoheadrightarrow \Nat{h_X^*, h_Y^*}.$$

    Given an $A$-morphism $f : Z \rightarrow X$, we define for each affine open $U \subseteq Z$ the restriction $f_U = f\mid_U$
    $$\begin{tikzcd}
      U \arrow{r}{i_U} \arrow[swap]{rd}{f_U} & Z\arrow{d}{f}\\
      & X.
    \end{tikzcd}$$
    Let $g_U = \eta^*_U(f_U)$.
    We will show that the $g_U : U \rightarrow Y$ glue to a morphism $g : Z \rightarrow Y$.
    If $V$ is any other affine open, we note $U \cap V$ is not necessarily affine.
    Since $\eta^*$ is only defined on affine schemes, in order to check gluability, we must pass to affine subsets of $W \subseteq U \cap V$.
    %, for which we use Nike's Trick to produce an open $W \subseteq U \cap V$ that is a basic affine open in $U$ and $V$, and these $W$ cover $U \cap V$.
    
    It is clear from the definition that $f_U\mid_W  = f_V\mid_W$, so by commutativity of the naturality square
    $$\begin{tikzcd}
      h_X(U)\arrow{d}{\eta^*_U}\arrow{r} & h_X(W)\arrow{d}{\eta_W^*}\\
      h_Y(U)\arrow{r} & h_Y(W)
    \end{tikzcd}$$
    we have
    $$g_U\mid_W = \eta^*_U(f_U)\mid_W = \eta^*_W(f_U\mid_W) = \eta^*_W(f_V\mid_W) = \eta^*_V(f_V)\mid_W = g_V\mid_W.$$
    Since we may $U \cap V$ by affines, this is enough to guarantee that $g_U\mid_W$ and $g_V\mid_W$ glue to the same morphism, $g_U\mid_{U \cap V} = g_V\mid_{U \cap V}$.
    Therefore the $g_U$ glue to a morphism, $g : Z \rightarrow Y$, as desired.
    We define $\eta(f) = g$, and we note that $\eta(f)\mid_U = g_u = \eta^*(f\mid_U)$.

    Suppose that $g : Z^\prime \rightarrow Z$ is any morphism of $A$-schemes.
    We must check that the diagram
    $$\begin{tikzcd}
      h_X(Z) \arrow{r}{\eta_Z}\arrow{d} & h_Y(Z)\arrow{d}\\
      h_X(Z^\prime)\arrow{r}{\eta_{Z^\prime}} & h_Y(Z^\prime)
    \end{tikzcd}$$
    commutes.
    For $f : Z \rightarrow X$, this amounts to checking that
    $$\eta_{Z^\prime}(f \circ h) = \eta_Z(f) \circ h.$$
    Since we may cover $Z$ by affines, the preimages of the affines cover $Z^\prime$, and each can be covered in turn by open affines of $Z^\prime$, it suffices to show
    $$\eta_{Z^\prime}(f \circ h)\mid_V = \left(\eta_Z(f) \circ h\right) \mid_V,$$
    for $U \subseteq Z$ open affine and $V \subseteq h^{-1}(U)$ open affine.
    But, this follows directly from the commutative diagram
    $$\begin{tikzcd}
      V\arrow{d}{h\mid_V} & h_X^*(U)\arrow{r}{\eta_U^*}\arrow{d} & h_Y^*(U)\arrow{d}\\
      U & h_X^*(V)\arrow{r}{\eta_V^*} & h_Y^*(V)
    \end{tikzcd}$$
    since
    $$\eta_{Z^\prime}(f \circ h)\mid_V = \eta_V^*\left(\left(f \circ h\right) \mid_V\right) = \eta_V^*(f_U \circ h\mid_V) = \eta_U^*(f_U) \circ h\mid_V = \eta_Z(f)\mid_U \circ h\mid_V = \left(\eta_Z(f) \circ h\right) \mid_V.$$
    This establishes that $\eta$ is a natural transformation, and by construction $\eta_B = \eta^*_B$ for any $A$-algebra, $B$.
    Therefore
    $$\Hom{\Sch/A}{X,Y} \rightarrow \Nat{h_X^*, h_Y^*}$$
    is a bijection, as desired.
  \end{proof}
\end{prop}

\begin{defn}
  A morphism $\eta : F \rightarrow G$ of $Fun(A-\mathfrak{Alg}, \mathfrak{Set})$ is called representable if for each morphism $\nu : h_{\Spec{B}} \rightarrow G$, the fibered product
  $$\begin{tikzcd}
    F \times_G h_{\Spec{B}}\arrow{d}\arrow{r} & F\arrow{d}{\nu}\\
    h_{\Spec{B}}\arrow{r}{\nu} & G
  \end{tikzcd}$$
  is representable, $F \times_G h_{\Spec{B}} \cong h_{\Spec{C}}$ for some $A$-algebra, $C$.
  \begin{defn}
    A representable morphism $\nu : F \rightarrow G$ is called an open subfunctor if for each ring $B$, the induced morphism of rings $\Spec{C} \rightarrow \Spec{B}$ induced by the morphism $h_{\Spec{C}} \rightarrow h_{\Spec{B}}$ is an open immersion.
    That is to say, the morphism $B \rightarrow C$ of $A$-algebras identifies $B$ with some localization, $C_f$, of $C$.
  \end{defn}
\end{defn}

\end{document}

