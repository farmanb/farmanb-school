\documentclass[10pt]{amsart}
\usepackage{amsmath,amsthm,amssymb,amsfonts,enumerate,mymath,mathtools,tikz-cd,mathrsfs,enumitem,calligra}
\openup 5pt
\author{Blake Farman\\University of South Carolina}
\title{Dissertation?}
\date{October 3, 2016}
\pdfpagewidth 8.5in
\pdfpageheight 11in
\usepackage[margin=1in]{geometry}

\begin{document}
%\maketitle

\providecommand{\F}{\mathscr{F}}
\providecommand{\G}{\mathscr{G}}
\providecommand{\p}{\mathfrak{p}}
\providecommand{\m}{\mathfrak{m}}
\providecommand{\Deck}[1]{\operatorname{Deck}\left(#1\right)}
%\newcommand{\Res}{\operatorname{Res}}
\newtheorem{thm}{Theorem}
\newtheorem{ex}{}
\newtheorem{lem}{Lemma}
\newtheorem{cor}{Corollary}
\newtheorem{prop}{Proposition}
\theoremstyle{definition}
\newtheorem{defn}{Definition}
\newtheorem{rmk}{Remark}
\newtheorem{eg}{Example}

\newcommand{\A}{\mathbb{A}}
\newcommand{\T}{\mathscr{T}}
\newcommand{\CC}{\mathscr{C}}
\newcommand{\D}{\mathscr{D}}

\section{Preliminaries}
\begin{defn}
  Let $\mathscr{C}$ be a category.
  \begin{itemize}
  \item
    We say a subcategory $\mathscr{C^\prime}$ of $\mathscr{C}$ is {\it full} if 
    $\Hom{\mathscr{C^\prime}}{X,Y} = \Hom{\mathscr{C}}{X,Y}$
    for all objects $X,Y$ of $\mathscr{C^\prime}$.
  \item
    We say a full subcategory $\mathscr{C^\prime}$ of $\mathscr{C}$ is {\it strictly full} if whenever $X \rightarrow Y$ is an isomorphism with $X$ an object of $\mathscr{C^\prime}$, then $Y$ is an object of $\mathscr{C^\prime}$.
  \end{itemize}
\end{defn}

\begin{defn}
  Let $\T$ be an additive category equipped with autoequivalences of categories, $[n] : \mathscr{C} \rightarrow \mathscr{C}$ for $n \in \Z$, and a class of distinguished triangles.
  We say $\T$ is {\it triangulated} if the distinguished triangles satisfy the following axioms:
  \begin{description}[style=nextline]
    \item[TR1]\label{TR1}
      \begin{itemize}
      \item
        Every morphism $X \rightarrow Y$ can be embedded in a distinguished triangle $X \rightarrow Y \rightarrow Z \rightarrow X[1]$.
      \item
        The triangle \begin{tikzcd} X \arrow{r}{\id_X} & X \arrow{r} & 0 \arrow{r} & X[1]\end{tikzcd} is distinguished.
      \item
        Any triangle isomorphic to a distinguished triangle is also distinguished.
      \end{itemize}
    \item[TR2]\label{TR2}
      If 
      \begin{tikzcd}
        X \arrow{r}{f} & Y \arrow{r}{g} & Z \arrow{r}{h} & X[1]
      \end{tikzcd} 
      is a distinguished triangle, then the two rotated triangles
      $$\begin{tikzcd}
        Y \arrow{r}{g} & Z \arrow{r}{h} & X[1] \arrow{r}{-f[1]} & Y[1]
      \end{tikzcd}$$
      and
      $$\begin{tikzcd}
        Z[-1] \arrow{r}{-h[1]} & X \arrow{r}{f} & Y \arrow{r}{g} & Z
      \end{tikzcd}$$
      are also distinguished.
    \item[TR3]\label{TR3}
      Given distinguished triangles
      $$\begin{tikzcd}
        X_i \arrow{r}{f_i} & Y_i\arrow{r}{g_i} & Z_i \arrow{r}{h_i} & Z_i
      \end{tikzcd}$$
      for $i = 1, 2$ and a diagram
      $$\begin{tikzcd}
        X_1 \arrow{d}{\alpha}\arrow{r}{f_1} & Y_1 \arrow{d}{\beta}\arrow{r}{g_1} & Z_1 \arrow[dashed]{d}{\exists \gamma}\arrow{r}{h_1} & X_1[1]\arrow{d}{\alpha[1]}\\
        X_2 {\alpha}\arrow{r}{f_2} & Y_2 \arrow{r}{g_2} & Z_2 \arrow{r}{h_2} & X_2[1]
      \end{tikzcd}$$
      such that the left-hand square commutes, the dashed arrow can be filled in with a morphism $\gamma$, which is not necessarily unique.
      
    \item[TR4]\label{TR4}
      Given three distinguished triangles
      $$\begin{tikzcd}
        X \arrow{r}{f} & Y \arrow{r}{i} & X^\prime \arrow{r}{\ell} & X[1]\\
        X \arrow{r}{g \circ f} & Z \arrow{r}{j} & Y^\prime \arrow{r}{m} & X[1]\\
        Y \arrow{r}{g} & Z \arrow{r}{k} & Z^\prime \arrow{r}{n} & Y[1]
      \end{tikzcd}$$
      then there exists a distinguished triangle
      $$\begin{tikzcd}
        X^\prime \arrow{r}{u} & Y^\prime \arrow{r}{v} & Z^\prime \arrow{r}{i[1]\circ n} & X^\prime[1]
      \end{tikzcd}$$
      making the diagram
      $$\begin{tikzcd}
        X\arrow{d}{\id_X}\arrow{r}{f} & Y\arrow{r}{i}\arrow{d}{g} & X^\prime\arrow{r}{\ell}\arrow{d}{u} & X[1]\arrow{d}{id_{X[1]}}\\
        X \arrow{d}{f}\arrow{r}{g \circ f} & Z\arrow{r}{j}\arrow{d}{\id_Z} & Y^\prime\arrow{r}{m}\arrow{d}{v} & X[1]\arrow{d}{f[1]}\\
        Y\arrow{r}{g}\arrow{d}{i} & Z \arrow{r}{k}\arrow{d}{j} & Z^\prime \arrow{r}{n}\arrow{d}{\id_{Z^\prime}} & Y[1]\arrow{d}{i[1]}\\
        X^\prime \arrow{r}{u} & Y^\prime \arrow{r}{v} & Z^\prime \arrow{r}{i[1] \circ n} & X^\prime[1]
      \end{tikzcd}$$
  \end{description}
\end{defn}
commute.

\begin{defn}
  Let $\T$ be a triangulated category.
  \begin{itemize}
  \item
    We say a subcategory $\T^\prime$ of $\T$ is a triangulated subcategory if $\T^\prime$ is triangulated and the inclusion functor preserves distinguished triangles.
  \item
    We say a full triangulated subcategory $\T^\prime$ of $\T$ is {\it saturated} if whenever $X \oplus Y$ is isomorphic to an object of $T^\prime$, both $X$ and $Y$ are isomorphic to objects of $T^\prime$.
  \end{itemize}
\end{defn}

\begin{prop}\label{fulltriangles}
  Let $\T$ be a triangulated category, let $\T^\prime$ be a full triangulated subcategory, and let 
  $$D\colon 
  \begin{tikzcd}
    X \arrow{r}{f} & Y \arrow{r}{g} & Z \arrow{r}{h} & X[1] 
  \end{tikzcd}$$
  be a distinguished triangle of $\T$.
  \begin{enumerate}
  \item
    If $D$ is distinguished in $\T$ and two out of three of $X,Y,Z$ are objects of $\T^\prime$, then the third is isomorphic to an object of $T^\prime$.
    Moreover, if $T^\prime$ is strictly full, then all three are objects of $T^\prime$.
  \item
    If $D$ is a triangle in $T^\prime$, then it is distinguished in $\T^\prime$.
  \end{enumerate}

  \begin{proof}
    For the first assertian, note that because we may rotate, it suffices to assume that $X$ and $Y$ are objects of $T^\prime$.
    We observe that by (TR1) we obtain a distinguished triangle of $T^\prime$ 
    $$D^\prime : 
    \begin{tikzcd}
      X \arrow{r}{f} & Y\arrow{r}{u} & Z^\prime \arrow{r}{v} & X[1]
    \end{tikzcd},$$
    which is also distinguished in $\T$.
    By (TR3) we have a morphism of distinguished triangles of $\T$ to $D$
    $$\begin{tikzcd}
      D^\prime: & X \arrow{d}{\id_X}\arrow{r}{f} & Y \arrow{d}{\id_Y}\arrow{r}{u} & Z^\prime \arrow[dashed]{d}{\exists \gamma}\arrow{r}{v} & X[1]\arrow{d}{\id_{X[1]}}\\
      D: & X \arrow{r}{f} & Y \arrow{r}{g} & Z \arrow{r}{h} & X[1]
    \end{tikzcd}$$
    which is an isomorphism by the 5 Lemma.
    This proves the first assertian.
    
    For the second, we note that when $Z$ is an object of $\T^\prime$, we have $\Hom{\T^\prime}{Z,Z^\prime} = \Hom{\T}{Z,Z^\prime}$, so that the isomorphism above is an isomorphism of triangles in $T^\prime$.
  \end{proof}
\end{prop}

\begin{prop}
  Let $\T$ be a triangulated category.
  If 
  \begin{tikzcd} X \arrow{r}{f} & Y \arrow{r}{g} & Z \arrow{r}{h} & X[1]\end{tikzcd} 
  is a distinguished triangle, then so is
  $$\begin{tikzcd}X[n] \arrow{r}{f[n]} & Y[n] \arrow{r}{g[n]} & Z[n] \arrow{r}{h[n]} & X[n+1]\end{tikzcd}.$$

  \begin{proof}
    We note that it suffices to show that the result holds for $n = 1$ since $[n] = [n - 1] \circ [1]$.
    We have by (TR2) the successive rotations
    $$\begin{tikzcd}
      Y \arrow{r}{g} & Z\arrow{r}{h} & X[1]\arrow{r}{-f[1]} & Y[1]\\
      Z \arrow{r}{h}& X[1]\arrow{r}{-f[1]} & Y[1]\arrow{r}{-g[1]} & Z[1]\\
      X[1] \arrow{r}{-f[1]} & Y[1] \arrow{r}{-g[1]} & Z[1]\arrow{r}{-h[1]} & X[2],
    \end{tikzcd}$$
    all of which are distinguished triangles.
    We then have an isomorphism of triangles
    $$\begin{tikzcd}
      X[1] \arrow{d}{\id_{X[1]}}\arrow{r}{f[1]} & Y[1] \arrow{d}{-\id_{Y[1]}}\arrow{r}{g[1]} & Z[1]\arrow{d}{\id_{Z[1]}}\arrow{r}{h[1]} & X[2]\arrow{d}{-\id_{X[2]}}\\
      X[1] \arrow{r}{-f[1]} & Y[1] \arrow{r}{-g[1]} & Z[1]\arrow{r}{-h[1]} & X[2].
    \end{tikzcd}$$
    Therefore 
    $$\begin{tikzcd}
      X[1] \arrow{r}{f[1]} & Y[1] \arrow{r}{g[1]} & Z[1]\arrow{r}{h[1]} & X[2]
    \end{tikzcd}$$
    is distinguished, as desired.
  \end{proof}
\end{prop}

\begin{defn}
  Let $\T$ be a triangulated category and let $\mathscr{A}$ be an abelian category.
  An additive functor $H : \T \rightarrow \mathscr{A}$ (resp. $H : \T^\text{op}\rightarrow A$) is called {\it homological} (resp. {\it cohomological}) if for every distinguished triangle 
  \begin{tikzcd}
    X \arrow{r}{f} & Y \arrow{r}{g} & Z \arrow{r}{h} & X[1]
  \end{tikzcd}
  the sequence
  $$\begin{tikzcd}
    HX \arrow{r}{H(f)} & HY \arrow{r}{H(g)} & HZ
  \end{tikzcd}$$
  (resp. $\begin{tikzcd}
    HZ \arrow{r}{H(g)} & HY \arrow{r}{H(f)} & HX
  \end{tikzcd}$)
  is exact in $\mathscr{A}$ 
\end{defn}

\begin{prop}
  Let $\T$ be a triangulated category.
  If 
  \begin{tikzcd}
    X \arrow{r}{f} & Y \arrow{r}{g} & Z \arrow{r}{h} & X[1]
  \end{tikzcd} 
  is a distinguished triangle, then $g \circ f = 0$ and $h \circ g = 0$.

  \begin{proof}
    By (TR3) we obtain a morphism of distinguished triangles
    $$\begin{tikzcd}
      X \arrow{d}{\id_X}\arrow{r}{\id_X} & X \arrow{d}{f}\arrow{r} & 0\arrow[dashed]{d}{\exists !0} \arrow{r} & X[1]\arrow{d}{\id_{X[1]}}\\
      X \arrow{r}{f} & Y \arrow{r}{g} & Z \arrow{r}{h} & X[1]
    \end{tikzcd}$$
    and thus $g \circ f = 0$.

    That $h \circ g = 0$ follows from the same argument applied to the distinguished triangle
    $$\begin{tikzcd}
      Y \arrow{r}{g} & Z \arrow{r}{h} & X[1] \arrow{r}{-g[1]} & Y[1].
    \end{tikzcd}$$
  \end{proof}
\end{prop}

\begin{prop}
  Let $\T$ be a triangulated category.
  For any object $X$ of $\T$, the functor $h^X = \Hom{\T}{X,\_}$ (resp. $h_X = \Hom{\T}{\_,X}$) is homological (resp. cohomological).
  
  \begin{proof}
    Given a distinguished triangle \begin{tikzcd}A \arrow{r}{f} & B \arrow{r}{g} & C \arrow{r}{h} & A[1]\end{tikzcd}, we obtain a complex
    $$\begin{tikzcd}
      \Hom{\T}{X,A} \arrow{r}{f_*} &\Hom{\T}{X,B} \arrow{r}{g_*} & \Hom{\T}{X,C}.
    \end{tikzcd}$$
    since for any morphism $\varphi : X \rightarrow A$ 
    $$g_* \circ f_*(\varphi) = g \circ f \circ \varphi = 0.$$
    We need only show that $f_* = \ker g_*$.
    
    Given a morphism $\psi : X \rightarrow B$ such that $g_*(\psi) = \psi \circ g= 0$, we have a morphism of distinguished triangles by (TR1)
    $$\begin{tikzcd}
      X \arrow{r}\arrow{d}{\psi} & 0 \arrow{r}\arrow{d} & X[1] \arrow[dashed]{d}{\exists \gamma}\arrow{r}{\id_{X[1]}} & X[1]\arrow{d}{\psi[1]}\\
      B \arrow{r}{g} & C \arrow{r}{h} & A[1] \arrow{r}{-f[1]} & B[1]
    \end{tikzcd},$$
    which yields $f_*(-\gamma[-1]) = -f \circ \gamma[-1] = \psi$.
    Therefore $f_* = \ker g_*$, as desired.
    
    By replacing $\T$ by $\T^\text{op}$, we see that $h_X$ is cohomological.
  \end{proof}
\end{prop}


\begin{prop}\label{isotriangle}
  Let $\T$ be a triangulated category.
  If 
  \begin{tikzcd}
    X \arrow{r}{f} & Y \arrow{r} & Z \arrow{r} & X[1]
  \end{tikzcd} is a distinguished triangle, then $f$ is an isomorphism if and only if $Z \cong 0$.
  
  \begin{proof}
    Assume that $f$ is an isomorphism.
    We have by (TR3) a morphism of triangles
    $$\begin{tikzcd}
      X \arrow{d}{\id_X}\arrow{r}{f} & Y \arrow{d}{f^{-1}}\arrow{r} & Z \arrow[dashed]{d}{\exists}\arrow{r} & X[1]\arrow{d}{\id_{X[1]}}\\
      X \arrow{r}{\id_X} & X \arrow{r} & 0 \arrow{r} & X[1]
    \end{tikzcd}$$
    which is an isomorphism of triangles by the 5 Lemma.
    Hence $Z \cong 0$.
    
    Conversely, if $Z \cong 0$, then we have morphisms of distinguished triangles by (TR3)
    $$\begin{tikzcd}
      Z \arrow{d}\arrow{r} & X[1] \arrow{d}{\id_{X[1]}}\arrow{r}{f[1]} & Y[1]\arrow[dashed]{d}{\exists g} \arrow{r} & Z[1]\arrow{d}\\
      0 \arrow{r} & X[1] \arrow{r}{\id_{X[1]}} & X[1] \arrow{r} & 0
    \end{tikzcd}$$
    and the induced morphism is an isomorphism by the 5 Lemma.
    We also have a morphism of distinguished triangles by (TR3)
    $$\begin{tikzcd}
      Z \arrow{d}\arrow{r} & X[1] \arrow{d}{g^{-1}}\arrow{r}{f[1]} & Y[1]\arrow[dashed]{d}{\exists h} \arrow{r} & Z[1]\arrow{d}\\
      0 \arrow{r} & Y[1] \arrow{r}{\id_{Y[1]}} & Y[1] \arrow{r} & 0
    \end{tikzcd}$$
    We see that $g^{-1} = h \circ f[1]$ and thus
    $$\id_{Y[1]} = g^{-1} \circ g = h \circ (f[1] \circ g) = h$$
    yields $f[1]$ an isomorphism with inverse $g$.
    Moreover $f$ is also an isomorphism because $[1]$ is an equivalence of categories and hence reflects isomorphisms.
    Therefore $f$ is an isomorphism if and only if $Z \cong 0$.
  \end{proof}
\end{prop}

\begin{prop}\label{sumtriangles}
  Let $\T$ be a triangulated category admitting coproducts indexed by a set $I$.
  If 
  $$\left\{D_i : \begin{tikzcd}X_1 \arrow{r}{f_i} & Y_i \arrow{r}{g_i} & Z_i\arrow{r}{h_i} & X_i[1]\end{tikzcd}\right\}_{i\in I}$$ 
  is a collection of distinguished triangles, then the triangle 
  $$\bigoplus_{i \in I} D_i : \begin{tikzcd} \bigoplus_{i \in I} X_i \arrow{r}{\oplus f_i} & \bigoplus_{i \in I}Y_i \arrow{r}{\oplus g_i} & \bigoplus_{i \in I} Z_i \arrow{r}{\oplus h_i} & \bigoplus_{i \in I} X_i[1]\end{tikzcd}$$
  is distinguished.

  \begin{proof}
    We first note that because $[1]$ is an autoequivalence it is left adjoint to $[-1]$ and hence necessarily commutes with coproducts, so that there is a unique isomorphism $\left(\bigoplus_{i \in I}X_i\right)[1] \cong \bigoplus_{i \in I}X_i[1]$.
    By (TR1) and (TR3) we have for each $i$ a morphism of distinguished triangles
    $$\begin{tikzcd} 
      X_i \arrow{r}{f_i}\arrow{d} & Y_i\arrow{d}\arrow{r}{g_i} & Z_i \arrow[dashed]{d}{\exists \gamma_i}\arrow{r}{h_i} & X_i[1] \arrow{d}\\
      \bigoplus_{i \in I} X_i \arrow{r}{\oplus f_i} & \bigoplus_{i \in I}Y_i \arrow{r}{g} & Z \arrow{r}{h} & \bigoplus_{i \in I} X_i[1]
    \end{tikzcd}.$$
    For any object $Z^\prime$ of $\T$, applying the cohomological functor $h_{Z^\prime}$ to the triangle $D_i$ gives a long exact sequence
    $$h_{Z^\prime}(D_i) : 
    \begin{tikzcd} 
      \cdots \arrow{r} & h_{Z^\prime}(Y_i[1]) \arrow{r} & h_{Z^\prime}(X_i[1]) \arrow{r} & h_{Z^\prime}(Z_i) \arrow{r} & h_{Z^\prime}(Y_i) \arrow{r} & h_{Z^\prime}(X_i) \arrow{r} & \cdots
    \end{tikzcd},$$
    and similarly gives a long exact sequence
    $$\begin{tikzcd}
      \cdots \arrow{r} & h_{Z^\prime}\left(\bigoplus_{i \in I}Y_i[1]\right) \arrow{r} & h_{Z^\prime}\left(\bigoplus_{i \in I}X_i[1]\right) \arrow{r} & h_{Z^\prime}(Z) \arrow{r} & h_{Z^\prime}\left(\bigoplus_{i \in I}Y_i\right) \arrow{r} & h_{Z^\prime}\left(\bigoplus_{i \in I} X_i\right) \arrow{r} & \cdots
    \end{tikzcd}$$
    We note that cohomology commutes with direct products, so we obtain the commutative diagram
    $$\begin{tikzcd} 
      \prod_{i \in I}h_{Z^\prime}(Y_i[1]) \arrow{r}\arrow{d}{\alpha_1} & \prod_{i \in I}h_{Z^\prime}(X_i[1]) \arrow{r}\arrow{d}{\alpha_2} & \prod_{i \in I}h_{Z^\prime}(Z_i) \arrow{r}\arrow{d}{\alpha_3} & \prod_{i \in I}h_{Z^\prime}(Y_i) \arrow{r}\arrow{d}{\alpha_4} & \prod_{i \in I}h_{Z^\prime}(X_i)\arrow{d}{\alpha_5}\\
      h_{Z^\prime}\left(\bigoplus_{i \in I}Y_i[1]\right) \arrow{r} & h_{Z^\prime}\left(\bigoplus_{i \in I}X_i[1]\right) \arrow{r} & h_{Z^\prime}(Z) \arrow{r} & h_{Z^\prime}\left(\bigoplus_{i \in I}Y_i\right) \arrow{r} & h_{Z^\prime}\left(\bigoplus_{i \in I} X_i\right)
    \end{tikzcd}$$
    with exact rows and $\alpha_1, \alpha_2, \alpha_4, \alpha_5$ all isomorphisms.
    By the 5 Lemma we see that $\alpha_3$ is an isomorphism, and thus by the Yoneda Lemma
    $$h^Z(Z^\prime) = h_{Z^\prime}(Z) \cong \prod_{i \in I}h_{Z^\prime}(Z_i) \cong h_{Z^\prime}\left(\bigoplus_{i \in I}Z_i\right) = h^{\bigoplus_{i \in I} Z_i}(Z^\prime).$$
    implies there is an isomorphism $\gamma : \bigoplus_{i \in I} Z_i \rightarrow Z$ which yields an isomorphism of triangles
    $$\begin{tikzcd}
      \bigoplus_{i \in I} X_i \arrow{d}{\id}\arrow{r}{\oplus f_i} & \bigoplus_{i \in I}Y_i \arrow{d}{\id}\arrow{r}{\oplus g_i} & \bigoplus_{i \in I} Z_i \arrow{d}{\gamma}\arrow{r}{\oplus h_i} & \bigoplus_{i \in I} X_i[1]\arrow{d}{\id}\\
      \bigoplus_{i \in I} X_i \arrow{r}{\oplus f_i} & \bigoplus_{i \in I}Y_i \arrow{r}{g} & Z \arrow{r}{h} & \bigoplus_{i \in I} X_i[1]
    \end{tikzcd}.$$
    Therefore $\bigoplus_{i \in I}D_i$ is distinguished, as desired.
  \end{proof}
\end{prop}

\begin{cor}\label{corsumtriangles}
  Let $\T$ be a triangulated category and let $X,Y$ be objects of $\T$.
  The triangle
  $$\begin{tikzcd}
    X \arrow{r} & X \oplus Y \arrow{r} & Y \arrow{r}{0} & X[1]
  \end{tikzcd}$$
  is distinguished.
  
  \begin{proof}
    Realize 
    $$\begin{tikzcd}
      X \arrow{r} & X \oplus Y \arrow{r} & Y \arrow{r}{0} & X[1]
    \end{tikzcd}$$
    as the coproduct of the distinguished triangles
    $$\begin{tikzcd}
      X \arrow{r} & X \arrow{r} & 0\arrow{r} & X[1]
    \end{tikzcd}
    \text{and}
    \begin{tikzcd}
      0 \arrow{r} & Y \arrow{r} & Y \arrow{r} & 0.
    \end{tikzcd}$$
  \end{proof}
\end{cor}

\begin{cor}\label{splittriangle}
  Let $\T$ be a triangulated category.
  If 
  \begin{tikzcd}
    X \arrow{r} & Y \arrow{r} & Z \arrow{r}{0} & X[1]
  \end{tikzcd}
  is a distinguished triangle, then $Y \cong X \oplus Z$.
  
  \begin{proof}
    By taking rotates, we obtain a morphism of distinguished triangles
    $$\begin{tikzcd}
      Z[-1] \arrow{d}{\id_{Z[-1]}}\arrow{r}{0} & X \arrow{d}{\id_X}\arrow{r} & Y \arrow{r}\arrow[dashed]{d}{\exists} & Z\arrow{d}{\id_{Z}}\\
      Z[-1] \arrow{r}{0} & X \arrow{r} & X \oplus Z \arrow{r} & X[1]
    \end{tikzcd}$$
    and the induced morphism is an isomorphism by the 5 Lemma.
  \end{proof}
\end{cor}
\section{Generators of Triangulated Categories}

\begin{defn}
  Let $\T$ be a triangulated category and let $E$ be an object of $\T$.
  \begin{itemize}
    \item
      Denote by $\langle E \rangle_1$ the strictly full subcategory of $\T$ with objects isomorphic to direct summands of finite coproducts
      $$\bigoplus_{i = 1}^r E[d_i].$$
    \item
      For $1 < n$, denote by $\langle E \rangle_n$ the full subcategory of $\T$ with objects isomorphic to direct summands of objects $X$ which fit into a distinguished triangle
      $$A \rightarrow X \rightarrow B \rightarrow A[1]$$
      with $A$ an object of $\langle E \rangle_1$ and $B$ an object of $\langle E \rangle_{n-1}$.
    \item
      Denote by $\langle E \rangle$ the subcategory of $\T$ with objects
      $$\bigcup_{n>0} \langle E \rangle_n.$$
  \end{itemize}
\end{defn}

\begin{lem}
  Let $\T$ be a triangulated category and let $E$ be an object of $\T$.
  The category $\langle E \rangle_n$ is a strictly full, additive subcategory of $\T$ preserved under shifts and under taking summands.
  
  \begin{proof}
    We proceed by induction on $n$.
    We first note that it's clear from the definition that $\langle E \rangle_n$ is strictly full, closed under taking summands, and preadditive since $\Hom{\langle E \rangle_n}{X,Y} = \Hom{\T}{X,Y}$.
    Thus it remains only to show that $\langle E \rangle_n$ is[ closed under sums, and closed under shifts.
    
    The case $n = 1$ is clear from the definition.
    Assume the result holds up to some $1 < n$, and let $E^\prime$ be an object of $\langle E \rangle_n$.
    Choose an object $X$ fitting in a distinguished triangle
    $$\begin{tikzcd}
      A \arrow{r}{f} & X \arrow{r}{g} & B \arrow{r}{h} & A[1]
    \end{tikzcd}$$
    such that $X \cong E \oplus E^{\prime\prime}$, $A$ is an object of $\langle E \rangle_1$, and $B$ is an object of $\langle E \rangle_{n-1}$.
    We note that the triangle
    $$\begin{tikzcd}
      A[d] \arrow{r}{f[d]} & X[d] \arrow{r}{g[d]} & Y[d] \arrow{r}{h[d]} & A[d+1]
    \end{tikzcd}$$
    is distinguished, and so by induction $A[d]$ is an object of $\langle E \rangle_1$ and $B[d]$ is an object of $\langle E \rangle_{n-1}$.
    Since $X[d] \cong E^\prime[d] \oplus E^{\prime\prime}[d]$, it follows that $E^\prime[d]$ is an object of $\langle E \rangle_n$.
    
    Similarly, if $E^\prime, E^{\prime\prime}$ are objects of $\langle E \rangle_n$, then we have objects $X, X^\prime$ fitting into distinguished triangles
    $$\begin{tikzcd}
      A \arrow{r}{f} & X \arrow{r}{g} & B \arrow{r}{h} & A[1]
    \end{tikzcd}$$
    and
    $$\begin{tikzcd}
      A^\prime \arrow{r}{f^\prime} & X^\prime \arrow{r}{g^\prime} & B^\prime \arrow{r}{h} & A^\prime[1]
    \end{tikzcd}$$
    By the induction hypothesis and Proposition~\ref{sumtriangles} the triangle
    $$\begin{tikzcd}
      A \oplus A^\prime \arrow{r}{f \oplus f^\prime} & X \oplus X^\prime \arrow{r}{g \oplus g^\prime} & B \oplus B^\prime \arrow{r}{h \oplus h^\prime} & A[1] \oplus A^\prime[1]
    \end{tikzcd}$$
    is distinguished with $A \oplus A^\prime$ an object of $\langle E \rangle_1$ and $B \oplus B^\prime$ an object of $\langle E \rangle_{n-1}$.
    Since $E^\prime \oplus E^{\prime\prime}$ is a summand of $X \oplus X^\prime$, it follows that $E^\prime \oplus E^{\prime\prime}$ is an object of $\langle E \rangle_n$.
  \end{proof}
\end{lem}

\begin{lem}
  Let $\T$ be a triangulated category and let $E$ be an object of $\T$.
  If $A$ is an object of $\langle E \rangle_a$, $B$ is an object of $\langle E \rangle_b$, and 
  \begin{tikzcd} A \arrow{r}{f} & X \arrow{r}{g} & B \arrow{r}{h} & A[1]\end{tikzcd}
  is a distinguished triangle, then $X$ is an object of $\langle E \rangle_{a + b}$.

  \begin{proof}
    Given an object $E^\prime$ of $\langle E \rangle_n$, we have by Corollary~\ref{corsumtriangles} the distinguished triangle
    $$\begin{tikzcd}
      E \arrow{r} & E^\prime \oplus E \arrow{r} & E^\prime \arrow{r} & E[1]
    \end{tikzcd}$$
    which shows $E^\prime \oplus E$ is an object of $\langle E \rangle_{n+1}$.
    Since $\langle E \rangle_{n+1}$ is closed under taking summands, it follows that $E^\prime$ is an object of $\langle E \rangle_{n+1}$.
    Hence $\langle E \rangle_n \subseteq \langle E \rangle_{n+1}$ and so $A$, $B$ are both objects of $\langle E \rangle_{a + b}$.
    Therefore $X$ is an object of $\langle E \rangle_{a + b}$ by Proposition~\ref{fulltriangles}.
  \end{proof}
\end{lem}

\begin{prop}
  Let $\T$ be a triangulated category and let $E$ be an object of $\T$.
  The subcategory
  $$\langle E \rangle = \bigcup_{n > 0} \langle E \rangle_n$$
  is a strictly full, saturated, triangulated subcategory of $\T$ and it is the smallest such subcategory of $\T$ containing $E$.

  \begin{proof}
    By construction, $\langle E \rangle$ is a full subcategory.
    To see that it is strictly full, observe that given an object $E^\prime$ of $\langle E \rangle$, choose an $n$ such that $E^\prime$ is an object of $\langle E \rangle_n$.
    Given an isomorphism $E^\prime \rightarrow Y$ we have a distinguished triangle by Proposition~\ref{isotriangle}
    $$\begin{tikzcd}
      E^\prime \arrow{r} & Y \arrow{r} & 0 \arrow{r} & E^\prime[1]
    \end{tikzcd}$$
    which implies $Y$ is an object of $\langle E \rangle_n \subseteq \langle E \rangle$ by Proposition~\ref{fulltriangles}.
    
    To see that $\langle E \rangle$ is saturated, it suffices to observe that $\langle E \rangle_n$ is strictly full and closed under taking summands.
    Indeed, if $X \oplus Y$ is isomorphic to an object $E^\prime$ of $\langle E \rangle$, then $E^\prime$ lies in some $\langle E \rangle_n$.
    Hence $X$ and $Y$ are both objects of $\langle E \rangle_n \subseteq \langle E \rangle$.
    
    For the triangulated stucture, take the distinguished triangles of $\langle E \rangle$ to be the distinguished triangles 
    $$\begin{tikzcd}
      X \arrow{r} & Y \arrow{r} & Z \arrow{r} & X[1]
    \end{tikzcd}$$
    of $\T$ with $X$, $Y$, and $Z$ objects of $\langle E \rangle$.
    We observe that (TR1) and (TR4) are satisfied by virtue of $\langle E \rangle$ being strictly full and Proposition~\ref{fulltriangles}.
    The axiom (TR2) is inherited from $\T$ and (TR3) is satisfied because $\langle E \rangle$ is full.
    The inclusion functor preserves the distinguished triangles by definition.

    Suppose that $T^\prime$ is any other strictly full, saturated, triangulated subcategory of $\T$ containing $E$.
    We show by induction that $\langle E \rangle_n$ is a subcategory of $\T^\prime$.
    First note that by definition, $\langle E \rangle_1$ is necessarily a subcategory of $\T^\prime$.
    Suppose that $\langle E \rangle_i$ is a subcategory of $\T^\prime$ for all $i < n$.
    Given an object $E^\prime$ of $\langle E \rangle_n$, there exist objects $A$ of $\langle E \rangle_1$, $B$ of $\langle E \rangle_{n-1}$, and $X \cong E^\prime \oplus E^{\prime\prime}$ and a distinguished triangle
    $$\begin{tikzcd}
      A \arrow{r} & X \arrow{r} & B \arrow{r} & A[1].
    \end{tikzcd}$$
    Since $\T^\prime$ is strictly full and saturated, it follows by Proposition~\ref{fulltriangles} that $X$ is an object of $\T^\prime$ and thus $E^\prime$ is also an object of $\T^\prime$.
    Therefore $\langle E \rangle$ is a subcategory of $\T^\prime$.
  \end{proof}
\end{prop}

\begin{defn}
  Let $\T$ be a triangulated category and let $E$ be an object of $\T$.
  \begin{enumerate}
  \item
    We say $E$ is a {\it classical generator} of $\T$ if $\T = \langle E \rangle$.
  \item
    We say $E$ is a {\it strong generator} of $\T$ if $\T = \langle E \rangle_n$ for some $1 \leq n$.
  \item
    We say $E$ is a {\it (weak) generator} of $\T$ if for every non-zero object $X$ of $\T$, there eixsts an $n$ such that $\Hom{\T}{E,X[n]} \neq 0$.
  \end{enumerate}
\end{defn}

\begin{lem}\label{classicalgenhom}
  Let $\T$ be a triangulated category.
  If $E$ and $X$ are objects of $\T$, then $\Hom{\T}{E,X[n]} = 0$ for all $n$ if and only if $\Hom{\T}{E^\prime, X} = 0$ for all objects $E^\prime$ of $\langle E \rangle$.
  
  \begin{proof}
    The converse is immediate.
    Assume that $\Hom{\T}{E,X[n]} = 0$ for all $n$.
    First we observe that because $[n]$ is an autoequivalence, we have a natural isomorphism
    $$0 = \Hom{\T}{E[n], X} \cong \Hom{\T}{E, X[-n]}.$$
    
    We proceed by induction.
    Given an object $E^\prime$ of $\langle E \rangle_1$, we may choose by the definition of $\langle E \rangle_1$ an object  $E^{\prime\prime}$ such that $E^\prime \oplus E^{\prime\prime} \cong \bigoplus_{i = 1}^r E[d_i]$.
    Hence
    $$\Hom{\T}{E^\prime,X} \oplus \Hom{\T}{E^{\prime\prime}, X} \cong \Hom{\T}{\bigoplus_{i=1}^rE[d_i], X} = 0$$
    implies that $\Hom{\T}{E^\prime, X} = 0$.
    
    Assume the result holds up to some $1 < n$.
    Given an object $E^\prime$ of $\langle E \rangle_n$, choose an object $E^{\prime\prime}$ and a distinguished triangle
    $$\begin{tikzcd}
      A \arrow{r} & E^\prime \oplus E^{\prime\prime} \arrow{r} & B \arrow{r} & A[1]
    \end{tikzcd}$$
    with $A$ an object of $\langle E \rangle_1$ and $B$ an object of $\langle E \rangle_{n-1}$.
    By induction, $\Hom{\T}{A,X} = 0 = \Hom{\T}{B,X}$, and because $h_X$ is a cohomological functor we get an exact sequence
    $$\begin{tikzcd}
      0 \arrow{r} & \Hom{\T}{E^\prime,X} \oplus \Hom{\T}{E^{\prime\prime}, X} \arrow{r} & 0
    \end{tikzcd}$$
    which implies $\Hom{\T}{E^\prime,X} = 0$.
  \end{proof}
\end{lem}

\begin{lem}\label{classicalgenisgen}
  Let $\T$ be a triangulated category and let $E$ be an object of $\T$.
  If $E$ is a classical generator, then $E$ is a generator.
  
  \begin{proof}
    Given an object $X$ of $\T$ such that $\Hom{\T}{E,X[n]} = 0$ for all $n$, we observe that because $\T = \langle E \rangle$, Lemma~\ref{classicalgenhom} implies $\Hom{\T}{X,X} = 0$.
    Therefore $\id_{X} = 0$ implies $X = 0$ and $E$ is a generator.
  \end{proof}
\end{lem}

\begin{defn}
  Let $\mathscr{C}$ be an additive category with arbitrary coproducts.
  A {\it compact object} of $\mathscr{C}$ is an object $K$ such that the canonical morphism
  $$\bigoplus_{i \in I} \Hom{\mathscr{C}}{K, E_i} \rightarrow \Hom{\mathscr{C}}{K, \bigoplus_{i \in I}E_i}$$
  is an isomorphism for any indexing set $I$ and any collection $\{E_i\}_I$ of objects of $\mathscr{C}$.
\end{defn}

\begin{lem}
  Let $\T$ be a triangulated category with coproducts.
  The compact objects of $\T$ form a saturated, strictly full, triangulated subcategory, $\T_c$, of $\T$.
\end{lem}

%May need some hypothesis on $\T$ being Ab3-enriched?
\begin{lem}\label{homhocolimiscolim}
  Let $\T$ be a triangulated category with coproducts.
  Let $K$ be an object of $\T_c$.
  Given a directed system of $\T$
  $$\begin{tikzcd}
    \cdots \arrow{r}{f_{n+1}} & X_n \arrow{r}{f_n} & X_{n-1} \arrow{r}{f_{n-1}} & \cdots
  \end{tikzcd}$$
  for which $X = \operatorname{hocolim}X_n$ exists, the canonical morphism
  $$\begin{tikzcd}
    \operatorname{colim} \Hom{\T}{K,X_n} \arrow{r} & \Hom{\T}{K,X}
  \end{tikzcd}$$
  is an isomorphism.
\end{lem}

\begin{lem}\label{hocolim}
  Let $\T$ be a triangulated category with coproducts.
  If $\left\{E_i\right\}_{i \in I}$ is a family of compact objects of $\T$ such that $\bigoplus_{i \in I}{E_i}$ generates $\T$, then every object $X$ of $\T$ can be written as 
  $$X = \operatorname{hocolim}X_n,$$
  where $X_1$ is a direct sum of shifts of the $E_i$ and each transition morphism fits into a distinguished triangle $Y_n \rightarrow X_n \rightarrow X_{n+1} \rightarrow Y_n[1]$, where $Y_n$ is a direct sum of shifts of the $E_i$.
\end{lem}

\begin{lem}\label{hocolimfactorization}
  With the same assumptions and notation as in the Lemma above, if $K$ is a compact object and $K \rightarrow X_n$ is a morphism, then there is a factorization 
  $$K \rightarrow E \rightarrow X_n,$$
  where $E$ is an object of $\langle \bigoplus_{J}E_j \rangle$ for some finite $J \subseteq I$.
  \begin{proof}
    We proceed by induction.
    When $n = 1$, $X_1 = \bigoplus_S E_i[m]$ over the triples
    $$S = \left\{(i,m,\varphi) \;\middle\vert\; i \in I, m \in \Z, \varphi : E_i[m] \rightarrow X\right\}.$$
    Because $K$ is compact, we have
    $$\Hom{\T}{K,X_1} \cong \bigoplus_S\Hom{\T}{K, E_i[m]}$$
    and thus a morphism $K \rightarrow X_1$ maps under this ismorphism to a finite collection of morphism, $$\left\{K \rightarrow E_i[m]\right\}_{(i,m,\varphi) \in S^\prime}.$$
    Hence $K \rightarrow X_1$ factors as
    $$K \rightarrow \bigoplus_{(i,m,\varphi) \in S^\prime}E_i[m] \rightarrow X_1.$$

    Assume the result holds up to some $1 < n$.
    By assumption, we have a distinguished triangle
    $$Y_{n-1} \rightarrow X_{n-1} \rightarrow X_{n} \rightarrow Y_{n-1}[1]$$
    and, since $Y_{n-1}$ is of the same form as $X_1$, we get a factorization
    $$\begin{tikzcd}
      K \arrow{rd}\arrow{r} & X_n \arrow{r} & Y_{n-1}[1]\\
      & E^\prime[1]\arrow{ur}
    \end{tikzcd}$$
    for some $E^\prime \in \langle \bigoplus_{i \in I^\prime}E_i \rangle$ with $I^\prime \subset I$ finite.
    By the axioms for a triangulated category, we have a morphism of distinguished triangles
    $$\begin{tikzcd}
      K[-1] \arrow{d}\arrow{r} & E^\prime \arrow{r}\arrow{d} & K^\prime \arrow{r}\arrow[dashed]{d}{\exists} & K\arrow{d}\\
      X_n[-1] \arrow{r} & Y_{n-1} \arrow{r} & X_{n-1} \arrow{r} & X_n
    \end{tikzcd}$$
    and we note that $K^\prime$ is compact because $K$ and $E^\prime$ are, and the triangulated subcategory of compact objects is strictly full.
    By induction, the induced morphism factors as
    $$\begin{tikzcd}
      K^\prime \arrow{rr}\arrow{rd} & & X_{n-1}\\
      & E^{\prime\prime}\arrow{ur}
    \end{tikzcd}$$
    for some object $E^{\prime\prime} \in \langle \bigoplus_{i \in I^{\prime\prime}} E_i \rangle$ with finite $I^{\prime\prime} \subseteq I$.
    Taking the composition $E^\prime \rightarrow K^\prime \rightarrow E^{\prime\prime}$ we get a distinguished triangle $E^\prime \rightarrow E^{\prime\prime} \rightarrow E \rightarrow E^\prime[1]$ for some object $E \in \langle \bigoplus_{i \in I^\prime \cup I^{\prime\prime}}E_i\rangle$ and morphisms of distinguished triangles
    $$\begin{tikzcd}
      E^\prime \arrow{r}\arrow{d} & K^\prime \arrow{r}\arrow{d} & K \arrow{r}\arrow[dashed]{d}{\exists} & E^\prime[1]\arrow{d}\\
      E^\prime \arrow{r}\arrow{d} & E^{\prime\prime} \arrow{r}\arrow{d} & E \arrow{r}\arrow[dashed]{d}{\exists} & E^\prime[1]\arrow{d}\\
      Y_{n-1} \arrow{r} & X_{n-1} \arrow{r} & X_n \arrow{r} & Y_{n-1}[1]
    \end{tikzcd}$$
    
    The composition of the induced morphisms gives a morphism $K \rightarrow X_n$, which need not be the original morphism, however the compositions with $X_n \rightarrow Y_{n-1}[1]$ agree by construction.
    We observe that because $h^{K}$ is a homological functor, the distinguished triangle $Y_{n-1} \rightarrow X_{n-1} \rightarrow X_{n} \rightarrow Y_{n-1}[1]$ gives rise to an exact sequence
    $$\begin{tikzcd}
      \Hom{\T}{K,X_{n-1}} \arrow{r} & \Hom{\T}{K,X_n} \arrow{r} & \Hom{\T}{K,Y_{n-1}[1]}
    \end{tikzcd}$$
    and, by the observation above, the difference of these two morphisms can be expressed as a morphism $K \rightarrow X_{n-1} \rightarrow X_n$.
    By induction, we have a factorization 
    $$\begin{tikzcd}
      K \arrow{rd}\arrow{rr} && X_{n-1}\\
      & E^{\prime\prime\prime}\arrow{ur}
    \end{tikzcd}$$
    through an object of $\langle \bigoplus_{i \in I^{\prime\prime\prime}} E_i\rangle$ with $I^{\prime\prime\prime} \subseteq I$ finite.
    Then the composition of the induced morphisms
    $$\begin{tikzcd}
      & K\arrow[dashed]{d}{\exists !}\arrow{ld}\arrow{rd}\\
      E\arrow{rd} & E \oplus E^{\prime\prime\prime} \arrow[dashed]{d}{\exists!}\arrow{r}\arrow{l} & E^{\prime\prime\prime}\arrow{d}\arrow{ld}\\
      & X_n & \arrow{l}X_{n-1}
    \end{tikzcd}$$
    gives the desired morphism with $E \oplus E^{\prime\prime\prime} \in \langle \bigoplus_{i \in I^\prime \cup I^{\prime\prime} \cup I^{\prime\prime\prime}} E_i\rangle.$
  \end{proof}
\end{lem}

\begin{defn}
  Let $\T$ be a triangulated category with coproducts.
  We say $\T$ is compactly generated if there exists a family $\{E_i\}_{i \in I}$ of compact objects such that $\bigoplus_{i \in I} E_i$ generates $\T$.
\end{defn}

\begin{prop}
  Let $\T$ be a triangulated category with coproducts.
  If $E$ is a compact object of $\T$, then
  $E$ is a classical generator for $\T_c$ and $\T$ is compactly generated 
  if and only if
  $E$ is a generator for $\T$.
  
  \begin{proof}
    First assume that $E$ is a classical generator for $\T_c$, and $\{E_i\}_{i \in I} \subseteq \T_c$ generates $\T$.
    By Lemma~\ref{classicalgenisgen}, $E$ is a generator for $\T_c$, and $\bigoplus_{i \in I}E_i \in \langle E \rangle = \T_c$.
    Hence, given an object $X$ of $\T$ such that $\Hom{\T}{E,X[n]} = 0$ for all $n$, it follows from Lemma~\ref{classicalgenhom} that
    $$\Hom{\T}{\bigoplus_{i \in I}E_i, X} = 0$$
    and thus $X = 0$.
    Therefore $E$ is a generator for $\T$.
    
    Conversely, assume that $E$ is a generator for $\T$.
    By definition, $\T$ is compactly generated, so it remains to show that $\T_c = \langle E \rangle$.
    Towards that end, let $X$ be an object of $\T_c$.
    By applying Lemma~\ref{hocolim} with $I = \{1\}$ and $E_1 = E$, using the same notation, we can write $X = \operatorname{hocolim}X_n$.
    By Lemma~\ref{homhocolimiscolim}, there exists a factorization
    $$\begin{tikzcd}
      X \arrow{rd}\arrow{rr}{\sim} & & \operatorname{hocolim} X_n\\
      & X_n\arrow{ur}
    \end{tikzcd}$$
    for some $n$.
    Using Proposition~\ref{isotriangle}, (TR1), and (TR4), we obtain a commutative diagram 
    $$\begin{tikzcd}
      X \arrow{r}\arrow{d} & X_n \arrow{r}\arrow{d} & A \arrow{d}\arrow{r} & X[1]\arrow{d}\\
      X \arrow{d}\arrow{r} & \operatorname{hocolim}X_n \arrow{d}\arrow{r} & 0 \arrow{d}\arrow{r} & X[1]\arrow{d}\\
      X_n \arrow{d}\arrow{r} & \operatorname{hocolim}X_n \arrow{d}\arrow{r} & B \arrow{r}\arrow{d} & X_n[1]\arrow{d}\\
      A \arrow{r} & 0 \arrow{r} & B \arrow{r} & A[1]
    \end{tikzcd}$$
    with rows distinguished triangles, showing that $A \rightarrow X[1]$ is the zero morphism.
    
    Now, by Lemma~\ref{hocolimfactorization}, we obtain a factorization
    $$\begin{tikzcd}
      X \arrow{rr}\arrow{rd} & & X_n\\
      & E^\prime \arrow{ur}
    \end{tikzcd}$$
    with $E^\prime$ an object of $\langle E \rangle$.
    Applying (TR1) and (TR4) yields a commutative diagram
    $$\begin{tikzcd}
      X \arrow{d}\arrow{r} & E^\prime \arrow{r}\arrow{d} & A^\prime \arrow{r}\arrow{d} & X[1]\arrow{d}\\
      X \arrow{r}\arrow{d} & X_n \arrow{r}\arrow{d} & A \arrow{r}{0}\arrow{d} & X[1]\arrow{d}\\
      E^\prime \arrow{r}\arrow{d} & X_n \arrow{r}\arrow{d} & B^\prime \arrow{r}\arrow{d} & E^\prime[1]\arrow{d}\\
      A^\prime \arrow{r} & A \arrow{r} & B^\prime \arrow{r} & A^\prime[1]
    \end{tikzcd}$$
    with rows distinguished triangles, which implies the morphism $A^\prime \rightarrow X[1]$ is the zero morphism.
    Therefore $E^\prime \cong X \oplus A^\prime$ by Corollary~\ref{splittriangle}, and $X$ is an object of $\langle E \rangle$, as desired.
  \end{proof}
\end{prop}

\section{dg categories}
{\noindent Throughout, let $k$ be a commutative ring.}

\begin{defn}
  A {\it differential graded $k$-module}, $M$, is a complex of $k$-modules
  $$M \colon \begin{tikzcd}
    \cdots \arrow{r}{d_M^{n-2}} & M^{n-1} \arrow{r}{d_M^{n-1}} & M^n \arrow{r}{d_M^n} & M^{n+1} \arrow{r}{d_M^{n+1}} & \cdots
  \end{tikzcd},$$
  or, equivalently, a $\Z$-graded module over the graded ring $k$, concentrated in degree 0, equipped with a degree one morphism $d_M : M \rightarrow M[1]$ such that $d_M^2 = 0$.
  A {\it morphism of differential graded $k$-modules} is a morphism of chain complexes.
  
  The {\it shift} of a dg $k$-module is the shifted complex
  $$M[1] \colon \begin{tikzcd}
    \cdots \arrow{r}{-d_M^{n-1}} & M^{n} \arrow{r}{-d_M^{n}} & M^{n+1} \arrow{r}{-d_M^{n+1}} & M^{n+2} \arrow{r}{-d_M^{n+2}} & \cdots
  \end{tikzcd}$$
  
  The tensor product of two dg $k$-modules is the usual tensor product in $\mathcal{C}(k)$, the category of chain complexes of $k$-modules.
\end{defn}

\begin{defn}
  We say a category, $\mathscr{C}$, is a {\it $k$-category} if
  \begin{enumerate}
    \item
      for each pair of objects $X$ and $Y$ of $\mathscr{C}$, $\Hom{\mathscr{C}}{X,Y}$ is a $k$-module, and
      \item
        composition 
        $$\Hom{\mathscr{C}}{X,Y} \otimes_k \Hom{\mathscr{C}}{Y,Z} \rightarrow \Hom{\mathscr{C}}{X,Z}$$
        in $\mathscr{C}$ is $k$-linear, associative, admitting units $\id_X \in \Hom{\mathscr{C}}{X,X}$.
  \end{enumerate}
\end{defn}

\begin{defn}
  A {\it differential graded} or {\it dg category} is a $k$-category, $\mathscr{C}$, satisfying
  \begin{enumerate}
  \item
    $\Hom{\mathscr{C}}{X,Y}$ is a dg $k$-module for all objects $X,Y$ of $\mathscr{C}$, and
  \item
    composition 
    $$\begin{tikzcd}
      \Hom{\CC}{X,Y} \otimes_k \Hom{\CC}{Y,Z} \arrow{r} & \Hom{\CC}{X,Z}
    \end{tikzcd}$$
    in $\mathscr{C}$ is a morphism of dg $k$-modules;
    that is a morphism of $\mathcal{C}(k)$.
  \end{enumerate}
  
  A {\it morphism of dg categories}, $\mathscr{F} \colon \CC \rightarrow \D$, is a functor such that
  $$F(X,Y) \colon 
  \begin{tikzcd}
    \Hom{\CC}{X,Y} \arrow{r} & \Hom{\D}{FX,FY}
  \end{tikzcd}$$
  is a morphism of dg $k$-modules.
  Denote by $\operatorname{dgcat_k}$ the category with objects small dg categories and morphisms dg functors.
\end{defn}

\begin{eg}
  \begin{description}[style=nextline]
  \item[dg category with one object]
    Let $A$ be a dg $k$-algebra; that is, a graded $k$-algebra equipped with a differential
    $$d(fg) = d(f)g + (-1)^nfd(g),\ f \in A^n, g \in A.$$
    Define $\mathscr{A}$ to be the category with one object, $\ast$, and morphisms $\mathscr{A}(\ast, \ast) = A$, with composition defined by the multiplication in $A$.
  \item[dg $k$-modules]
    Define the category $\mathcal{C}_\text{dg}(k)$ to have objects chain complexes of $k$-modules, and morphisms the graded module of graded morphisms; that is, a morphism $f \in \mathcal{C}_\text{dg}(k)(C^\bullet, D^\bullet)^m$ is a collection of morphisms $f^n \colon C^n \rightarrow D^{n+m}$.
    Equip $\mathcal{C}_\text{dg}(k)(C^\bullet, D^\bullet)$ with the differential
    $$\begin{tikzcd}
      \mathcal{C}_\text{dg}(k)(C^\bullet, D^\bullet)^n \arrow{r} & \mathcal{C}_\text{dg}(k)(C^\bullet, D^\bullet)^{n+1}\\
      f \arrow[mapsto]{r} & d_D \circ f + (-1)^{n+1}f \circ d_C
    \end{tikzcd}$$
    and define composition of morphisms by the tensor product in $\mathcal{C}(k)$,
    $$\begin{tikzcd}
      \mathcal{C}_\text{dg}(k)(D^\bullet, E^\bullet) \otimes_k \mathcal{C}_\text{dg}(k)(C^\bullet, D^\bullet) \arrow{r} & \mathcal{C}_\text{dg}(k)(C^\bullet, E^\bullet).
    \end{tikzcd}$$
  \end{description}
\end{eg}

\begin{defn}
  Let $\CC$ and $\D$ be objects of $\operatorname{dgcat}_k$.
  The {\it tensor product}, $\CC \otimes \D$, is the dg category with objects $\operatorname{ob}\CC \times \operatorname{ob}\D$ and
  morphisms
  $$\begin{tikzcd}
    \Hom{\CC \otimes \D}{(X, Y), (X^\prime, Y^\prime)} = \Hom{\CC}{X,X^\prime} \otimes_k \Hom{\D}{Y,Y^\prime}.
  \end{tikzcd}$$
  For ease of notation, we will denote by $X \otimes Y$ the object $(X,Y)$ of $\CC \otimes \D$.
\end{defn}

\begin{defn}
  Given two dg functors $\mathscr{F},\mathscr{G} \colon \CC \rightarrow \D$, define $\SHom{\mathscr{F},\mathscr{G}}^n$ to be the $k$-module of degree $n$ natural transformations.
  That is, morphisms of functors, $\eta \colon \mathscr{F} \rightarrow \mathscr{G}$ such that for each object $X$ of $\CC$, $\eta(X) \in \Hom{\D}{\mathscr{F}(X),\mathscr{G}(X)}^n$.
\end{defn}

\begin{prop}
  Given a degree $n$ natural transformation $\eta \colon \mathscr{F} \rightarrow \mathscr{G}$, the collection of morphisms
  $$d^n_{\Hom{\D}{\mathscr{F}X, \mathscr{G}X}}\left( \eta(X) \right) \in \Hom{\D}{\mathscr{F}X, \mathscr{G}X}^{n+1}$$
  define a natural transformation and hence endow $\SHom{\mathscr{F},\mathscr{G}}$ with the structure of a dg $k$-module, where the differential, $d_{\SHom{\mathscr{F},\mathscr{G}}}^n$ sends $\eta$ to this natural transformation.
  
  \begin{proof}
    It's clear that so long as the collection of $d^n_{\D(\F X, \G X)}(\eta(X))$ defines a natural transformation, the resulting sequence
    $$\begin{tikzcd}
      \cdots \arrow{r} & \SHom{\F, \G}^n \arrow{r} & \SHom{\F,\G}^{n+1} \arrow{r} & \cdots
    \end{tikzcd}$$
    will be a complex.
    
    First we note that, by definition, any dg functor necessarily commutes with the differentials:
    $$d_{\D(\mathscr{F}X, \mathscr{F}X^\prime)}\left(\mathscr{F}(f) \right) = \mathscr{F}\left(d_{\CC(X,X^\prime)}(f)\right)$$
    and composition is a morphism of dg $k$-modules, so we have the commutative diagrams
    $$\begin{tikzcd}
      \D(\F X^\prime, \G X^\prime) \otimes_k \D(\F X, \F X^\prime) \arrow{r}\arrow[swap]{d}{d \otimes 1 + 1 \otimes d}& \D(\F X, \G X^\prime)\arrow{d}{d}\\
      \D(\F X^\prime, \G X^\prime) \otimes_k \D(\F X, \F X^\prime)[1] \arrow{r}& \D(\F X, \G X^\prime)[1]
    \end{tikzcd}$$
    and
    $$\begin{tikzcd}
      \D(\G X, \G X^\prime) \otimes_k \D(\F X, \G X) \arrow{r}\arrow[swap]{d}{d \otimes 1 + 1 \otimes d}& \D(\F X, \G X^\prime)\arrow{d}{d}\\
      \D(\G X, \G X^\prime) \otimes_k \D(\F X, \G X)[1] \arrow{r}& \D(\F X, \G X^\prime)[1]
    \end{tikzcd}$$
    For a morphism $f \in \CC(X,X^\prime)$, chasing $\eta(X^\prime) \otimes \F(f)$ and $\G(f) \otimes \eta(X)$ through the diagram gives
    \begin{eqnarray*}
      d(\eta(X^\prime)) \circ \F(f) + \eta(X^\prime) \circ d(\F(f)) 
      &=& d(\eta(X^\prime) \circ \F(f))\\
      &=& d(\G(f) \circ \eta(X))\\
      &=& d(\G(f)) \circ \eta(X) + \G(f) \circ d(\eta(X)).
    \end{eqnarray*}
    By the fact that $\F$,$\G$ commute with differentials and $\eta$ is a natural tranformation, we see
    $$\eta(X^\prime) \circ d(\F(f)) = \eta(X^\prime) \circ \F(d(f)) = \G(d(f)) \circ \eta(X) = d(\G(f)) \circ \eta(X).$$
    Therefore
    $$d(\eta(X^\prime)) \circ \F(f) = \G(f) \circ d(\eta(X)),$$
    as desired.
  \end{proof}
\end{prop}

\begin{defn}
  For two objects $\CC$ and $\D$ of $\operatorname{dgcat}_k$, define the object $\SHom{\CC,\D}$ of $\operatorname{dgcat}_k$ to be the category with objects dg functors $\mathscr{F} \colon \CC \rightarrow \D$ and morphisms $\SHom{\mathscr{F},\mathscr{G}}$.
  
  Given two dg functors $\F,\G \colon \CC \rightarrow \D$, define a {\it morphism of dg functors}, $\eta \colon \F \rightarrow G$ to be a closed, degree zero natural transformation.
  That is, $\eta \in \SHom{\F, \G}^0$ and its image in $\SHom{\F,\G}^1$ under the differential is zero.
\end{defn}

\begin{rmk}
  There is a natural isomorphism of bifunctors
  $$\Hom{\operatorname{dgcat}_k}{\CC \otimes \D, \mathscr{E}} \cong \Hom{\operatorname{dgcat}_k}{\CC, \SHom{\D,\mathscr{E}}},$$
  which endows $\operatorname{dgcat}_k$ with a symmetric closed monoidal structure.
\end{rmk}

\begin{defn}
  Let $\CC$ be a dg category.
  Define
  \begin{enumerate}
  \item
    the category $Z^0(\CC)$ to be the category with objects those of $\CC$ and morphisms
    $$\Hom{Z^0(\CC)}{X,Y} = Z^0\left(\Hom{\CC}{X,Y}\right),$$
  \item
    the category $H^0(\CC)$ to be the category with objects those of $\CC$ and morphisms
    $$\Hom{H^0(\CC)}{X,Y} = H^0\left(\Hom{\CC}{X,Y}\right),$$
  \item
    the {\it homology category}, $H^\ast(\CC)$, to be the category with objects those of $\CC$ and morphisms 
    $$H^\ast\CC(X,Y) = \bigoplus H^n\CC(X,Y).$$
  \end{enumerate}
\end{defn}

\begin{rmk}
    Note that given a dg functor, $\mathscr{F} \colon \CC \rightarrow \D$, for $X$ and $Y$ objects of $\CC$, 
    $$\mathscr{F}(X,Y) \colon \CC(X,Y) \rightarrow \D(\mathscr{F}X, \mathscr{F}Y)$$
    is a morphism of $\mathcal{C}(k)$.
    Hence $H^0$ induces a functor
    $H^0(\mathscr{F}) \colon H^0(\CC) \rightarrow H^0(\D)$
    with $H^0(\mathscr{F})(X) = \mathscr{F}(X)$ and $H^0(\mathscr{F})(X,Y) = H^0(\mathscr{F}(X,Y))$.
\end{rmk}

\begin{defn}
  Let $\CC$ be a small dg category and let $\mathscr{M} : \CC \rightarrow \mathcal{C}_\text{dg}(k)$ be a dg functor.
  \begin{enumerate}
  \item
    We say that $\mathscr{M}$ is a right (resp. left) dg $\CC$-module if $\mathscr{M}$ is contravariant (resp. covariant).
  \item
    The {\it homology of a dg $\CC$-module}, $\mathscr{M}$, is the induced functor
    $$\begin{tikzcd}
      H^\ast(\CC) \arrow{r}{H^\ast(\mathscr{M})}& \Gr{k}\\
      X \arrow[mapsto]{r} & H^\ast(\mathscr{M}(X)),
    \end{tikzcd}$$
    where $\Gr{k}$ denotes the category of graded modules.
  \item
    Define the category $\mathcal{C}_\text{dg}(\CC) = \SHom{\CC^\text{op}, \mathcal{C}_\text{dg}(k)}$, and the category of right dg $\CC$-modules by
    $$\mathcal{C}(\CC) = Z^0 \mathcal{C}_\text{dg}(\CC).$$
  \item
    The {\it category up to homotopy of dg $\CC$-modules} is $\mathcal{H}(\CC) = H^0(\mathcal{C}_\text{dg}(\CC))$.
  \item
    A morphism $\eta \colon \mathscr{L} \rightarrow \mathscr{M}$ of dg $\CC$-modules is called a {\it quasi-isomorphism} if the induced morphism $H^\ast(\eta) \colon H^\ast(\mathscr{L}) \rightarrow H^\ast(\mathscr{M})$ is an isomorphism.
  \end{enumerate}
\end{defn}

\begin{defn}
  Let $\CC$ be a small dg category.
  The derived category $\mathcal{D}(\CC)$ is the localization of $\mathcal{C}(\CC)$ at the class of quasi-isomorphisms.

  TODO: This needs some justification.
\end{defn}

\begin{defn}
  Let $\mathscr{C}$ be a small dg category.
  We have the Yoneda embedding 
  $$\begin{tikzcd}
    Z^0(\mathscr{C}) \arrow{r}{h_{\_}} & \mathcal{C}(\mathscr{C})\\
    X \arrow[mapsto]{r} & h_X.
  \end{tikzcd}$$
  We say that $\mathscr{C}$ is {\it pretriangulated} if
  \begin{enumerate}
  \item
    for each object $X$ of $Z^0(\mathscr{C})$, $h_X[n] \cong h_{X[n]}$, and
  \item
    for any morphism $f \in Z^0\Hom{\mathscr{C}}{A,B}$, $\operatorname{cone}(h_X(f)) = h_X(\operatorname{cone}(f))$.
  \end{enumerate}
\end{defn}

\begin{prop}
  If $\mathscr{C}$ is a pretriangulated dg category, then $H^0(\mathscr{C})$ is triangulated.
  
  \begin{proof}
    ?
  \end{proof}
\end{prop}

\begin{defn}
  Let $\mathscr{C}$ be a pretriangulated dg category admitting coproducts, let $E$ be an object of $\mathscr{C}$, and let $\mathscr{T} = H^0(\mathscr{C})$.
  By abuse of notation, 
  \begin{enumerate}
  \item
    We say $E$ is a {\it classical generator} of $\mathscr{C}$ if $E$ is a classical generator of $\mathscr{T}$,
  \item
    We say $E$ is a {\it strong generator} of $\mathscr{C}$ if $E$ is a strong generator of $\mathscr{T}$,
  \item
    We say $E$ is a {\it (weak) generator} of $\mathscr{C}$ if $E$ is a (weak) generator of $\mathscr{T}$, and
  \item
    If $\mathscr{C}$ has coproducts, we say $\mathscr{C}$ is compactly generated if $\T$ is.
  \end{enumerate}
\end{defn}

%\begin{lem}[Brown representability]
%  Let $\T$ be a triangulated category with direct sums which is compactly generated.
%  Let $H$ be a cohomological functor on $\T$ which transforms direct sums into direct products.
%  Then $H$ is representable.
%\end{lem}

\begin{defn}
  A dg functor $\mathscr{F} \colon \CC \rightarrow \D$ is called a {\it quasi-equivalence} if 
  \begin{enumerate}
  \item
    for all objects $X$ and $Y$ of $\CC$, the morphism
    $$\mathscr{F}(X,Y) \colon
    \begin{tikzcd}
       \CC(X,Y) \arrow{r} & \D(\mathscr{F}X,\mathscr{F}Y)
    \end{tikzcd}$$
    of $\mathcal{C}(k)$ is a quasi-isomorphism, and
  \item
    the induced functor $H^0(\mathscr{F}) \colon H^0(\CC) \rightarrow H^0(\D)$ is an equivalence.
  \end{enumerate}
\end{defn}

%\begin{lem}
%  Let $\CC$ be a pretriangulated dg category and let $\T = H^0(\CC)$.
%  If $E_i$ is a family of generators for $\mathscr{C}$, then the compact objects of $\T$ are the representables.

%  \begin{proof}
    
%  \end{proof}
%\end{lem}

\begin{thm}
  Let $\mathscr{C}$ and $\mathscr{D}$ be pretriangulated dg-categories.
  If $E_i$ is a family of generators for $\mathscr{C}$ and $F_i$ is a family of generators for $\mathscr{D}$, then the $E_i \otimes F_j$ are a family of generators for $\mathcal{D}\left(\mathscr{C} \otimes \mathscr{D}\right)$.
  
  \begin{proof}
    
  \end{proof}
\end{thm}

\end{document}

