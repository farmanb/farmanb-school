\documentclass[10pt]{amsart}
\usepackage{amsmath,amsthm,amssymb,amsfonts,enumerate,mymath,mathtools,tikz-cd,mathrsfs,enumitem}
\openup 5pt
\author{Blake Farman\\University of South Carolina}
\title{Math 735:\\Homework 01}
\date{September 14, 2016}
\pdfpagewidth 8.5in
\pdfpageheight 11in
\usepackage[margin=1in]{geometry}

\begin{document}
%\maketitle

\providecommand{\p}{\mathfrak{p}}
\providecommand{\m}{\mathfrak{m}}
\providecommand{\Deck}[1]{\operatorname{Deck}\left(#1\right)}
%\newcommand{\Res}{\operatorname{Res}}
\newtheorem{thm}{Theorem}
\newtheorem{ex}{}
\newtheorem{lem}{Lemma}
\newtheorem{cor}{Corollary}
\newtheorem{prop}{Proposition}
\theoremstyle{definition}
\newtheorem{defn}{Definition}

\newcommand{\A}{\mathbb{A}}
\newcommand{\T}{\mathcal{T}}
\section{Preliminaries}
\begin{defn}
  Let $\mathcal{C}$ be a category.
  \begin{itemize}
  \item
    We say a subcategory $\mathcal{C^\prime}$ of $\mathcal{C}$ is {\it full} if 
    $\Hom{\mathcal{C^\prime}}{X,Y} = \Hom{\mathcal{C}}{X,Y}$
    for all objects $X,Y$ of $\mathcal{C^\prime}$.
  \item
    We say a full subcategory $\mathcal{C^\prime}$ of $\mathcal{C}$ is {\it strictly full} if whenever $X \rightarrow Y$ is an isomorphism with $X$ an object of $\mathcal{C^\prime}$, then $Y$ is an object of $\mathcal{C^\prime}$.
  \end{itemize}
\end{defn}

\begin{defn}
  Let $\T$ be an additive category equipped with autoequivalences of categories, $[n] : \mathcal{C} \rightarrow \mathcal{C}$ for $n \in \Z$, and a class of distinguished triangles.
  We say $\T$ is {\it triangulated} if the distinguished triangles satisfy the following axioms:
  \begin{description}[style=nextline]
    \item[TR1]\label{TR1}
      \begin{itemize}
      \item
        Every morphism $X \rightarrow Y$ can be embedded in a distinguished triangle $X \rightarrow Y \rightarrow Z \rightarrow X[1]$.
      \item
        The triangle \begin{tikzcd} X \arrow{r}{\id_X} & X \arrow{r} & 0 \arrow{r} & X[1]\end{tikzcd} is distinguished.
      \item
        Any triangle isomorphic to a distinguished triangle is also distinguished.
      \end{itemize}
    \item[TR2]\label{TR2}
      If 
      \begin{tikzcd}
        X \arrow{r}{f} & Y \arrow{r}{g} & Z \arrow{r}{h} & X[1]
      \end{tikzcd} 
      is a distinguished triangle, then the two rotated triangles
      $$\begin{tikzcd}
        Y \arrow{r}{g} & Z \arrow{r}{h} & X[1] \arrow{r}{-f[1]} & Y[1]
      \end{tikzcd}$$
      and
      $$\begin{tikzcd}
        Z[-1] \arrow{r}{-h[1]} & X \arrow{r}{f} & Y \arrow{r}{g} & Z
      \end{tikzcd}$$
      are also distinguished.
    \item[TR3]\label{TR3}
      Given distinguished triangles
      $$\begin{tikzcd}
        X_i \arrow{r}{f_i} & Y_i\arrow{r}{g_i} & Z_i \arrow{r}{h_i} & Z_i
      \end{tikzcd}$$
      for $i = 1, 2$ and a diagram
      $$\begin{tikzcd}
        X_1 \arrow{d}{\alpha}\arrow{r}{f_1} & Y_1 \arrow{d}{\beta}\arrow{r}{g_1} & Z_1 \arrow[dashed]{d}{\exists \gamma}\arrow{r}{h_1} & X_1[1]\arrow{d}{\alpha[1]}\\
        X_2 {\alpha}\arrow{r}{f_2} & Y_2 \arrow{r}{g_2} & Z_2 \arrow{r}{h_2} & X_2[1]
      \end{tikzcd}$$
      such that the left-hand square commutes, the dashed arrow can be filled in with a morphism $\gamma$, which is not necessarily unique.
      
    \item[TR4]\label{TR4}
      Given three distinguished triangles
      $$\begin{tikzcd}
        X \arrow{r}{f} & Y \arrow{r}{i} & X^\prime \arrow{r}{\ell} & X[1]\\
        X \arrow{r}{g \circ f} & Z \arrow{r}{j} & Y^\prime \arrow{r}{m} & X[1]\\
        Y \arrow{r}{g} & Z \arrow{r}{k} & Z^\prime \arrow{r}{n} & Y[1]
      \end{tikzcd}$$
      then there exists a distinguished triangle
      $$\begin{tikzcd}
        X^\prime \arrow{r}{u} & Y^\prime \arrow{r}{v} & Z^\prime \arrow{r}{i[1]\circ n} & X^\prime[1]
      \end{tikzcd}$$
      making the diagram
      $$\begin{tikzcd}
        X\arrow{d}{\id_X}\arrow{r}{f} & Y\arrow{r}{i}\arrow{d}{g} & X^\prime\arrow{r}{\ell}\arrow{d}{u} & X[1]\arrow{d}{id_{X[1]}}\\
        X \arrow{d}{f}\arrow{r}{g \circ f} & Z\arrow{r}{j}\arrow{d}{\id_Z} & Y^\prime\arrow{r}{m}\arrow{d}{v} & X[1]\arrow{d}{f[1]}\\
        Y\arrow{r}{g}\arrow{d}{i} & Z \arrow{r}{k}\arrow{d}{j} & Z^\prime \arrow{r}{n}\arrow{d}{\id_{Z^\prime}} & Y[1]\arrow{d}{i[1]}\\
        X^\prime \arrow{r}{u} & Y^\prime \arrow{r}{v} & Z^\prime \arrow{r}{i[1] \circ n} & X^\prime[1]
      \end{tikzcd}$$
  \end{description}
\end{defn}
commute.

\begin{prop}\label{fulltriangles}
  Let $\T$ be a triangulated category, let $\T^\prime$ be a full triangulated subcategory, and let 
  $$D\colon 
  \begin{tikzcd}
    X \arrow{r}{f} & Y \arrow{r}{g} & Z \arrow{r}{h} & X[1] 
  \end{tikzcd}$$
  be a distinguished triangle of $\T$.
  \begin{enumerate}
  \item
    If $D$ is distinguished in $\T$ and two out of three of $X,Y,Z$ are objects of $\T^\prime$, then the third is isomorphic to an object of $T^\prime$.
    Moreover, if $T^\prime$ is strictly full, then all three are objects of $T^\prime$.
  \item
    If $D$ is a triangle in $T^\prime$, then it is distinguished in $\T^\prime$.
  \end{enumerate}

  \begin{proof}
    For the first assertian, note that because we may rotate, it suffices to assume that $X$ and $Y$ are objects of $T^\prime$.
    We observe that by (TR1) we obtain a distinguished triangle of $T^\prime$ 
    $$D^\prime : 
    \begin{tikzcd}
      X \arrow{r}{f} & Y\arrow{r}{u} & Z^\prime \arrow{r}{v} & X[1]
    \end{tikzcd},$$
    which is also distinguished in $\T$.
    By (TR3) we have a morphism of distinguished triangles of $\T$ to $D$
    $$\begin{tikzcd}
      D^\prime: & X \arrow{d}{\id_X}\arrow{r}{f} & Y \arrow{d}{\id_Y}\arrow{r}{u} & Z^\prime \arrow[dashed]{d}{\exists \gamma}\arrow{r}{v} & X[1]\arrow{d}{\id_{X[1]}}\\
      D: & X \arrow{r}{f} & Y \arrow{r}{g} & Z \arrow{r}{h} & X[1]
    \end{tikzcd}$$
    which is an isomorphism by the 5 Lemma.
    This proves the first assertian.
    
    For the second, we note that when $Z$ is an object of $\T^\prime$, we have $\Hom{\T^\prime}{Z,Z^\prime} = \Hom{T}{Z,Z^\prime}$, so that the isomorphism above is an isomorphism of triangles in $T^\prime$.
  \end{proof}
\end{prop}

\begin{prop}
  Let $\T$ be a triangulated category.
  If 
  \begin{tikzcd} X \arrow{r}{f} & Y \arrow{r}{g} & Z \arrow{r}{h} & X[1]\end{tikzcd} 
  is a distinguished triangle, then so is
  $$\begin{tikzcd}X[n] \arrow{r}{f[n]} & Y[n] \arrow{r}{g[n]} & Z[n] \arrow{r}{h[n]} & X[n+1]\end{tikzcd}.$$

  \begin{proof}
    We note that it suffices to show that the result holds for $n = 1$ since $[n] = [n - 1] \circ [1]$.
    We have by (TR2) the successive rotations
    $$\begin{tikzcd}
      Y \arrow{r}{g} & Z\arrow{r}{h} & X[1]\arrow{r}{-f[1]} & Y[1]\\
      Z \arrow{r}{h}& X[1]\arrow{r}{-f[1]} & Y[1]\arrow{r}{-g[1]} & Z[1]\\
      X[1] \arrow{r}{-f[1]} & Y[1] \arrow{r}{-g[1]} & Z[1]\arrow{r}{-h[1]} & X[2],
    \end{tikzcd}$$
    all of which are distinguished triangles.
    We then have an isomorphism of triangles
    $$\begin{tikzcd}
      X[1] \arrow{d}{\id_{X[1]}}\arrow{r}{f[1]} & Y[1] \arrow{d}{-\id_{Y[1]}}\arrow{r}{g[1]} & Z[1]\arrow{d}{\id_{Z[1]}}\arrow{r}{h[1]} & X[2]\arrow{d}{-\id_{X[2]}}\\
      X[1] \arrow{r}{-f[1]} & Y[1] \arrow{r}{-g[1]} & Z[1]\arrow{r}{-h[1]} & X[2].
    \end{tikzcd}$$
    Therefore 
    $$\begin{tikzcd}
      X[1] \arrow{r}{f[1]} & Y[1] \arrow{r}{g[1]} & Z[1]\arrow{r}{h[1]} & X[2]
    \end{tikzcd}$$
    is distinguished, as desired.
  \end{proof}
\end{prop}

\begin{defn}
  Let $\T$ be a triangulated category and let $\mathcal{A}$ be an abelian category.
  An additive functor $H : \T \rightarrow \mathcal{A}$ (resp. $H : \T^\text{op}\rightarrow A$) is called {\it homological} (resp. {\it cohomological}) if for every distinguished triangle 
  \begin{tikzcd}
    X \arrow{r}{f} & Y \arrow{r}{g} & Z \arrow{r}{h} & X[1]
  \end{tikzcd}
  the sequence
  $$\begin{tikzcd}
    HX \arrow{r}{H(f)} & HY \arrow{r}{H(g)} & HZ
  \end{tikzcd}$$
  (resp. $\begin{tikzcd}
    HZ \arrow{r}{H(g)} & HY \arrow{r}{H(f)} & HX
  \end{tikzcd}$)
  is exact in $\mathcal{A}$ 
\end{defn}

\begin{prop}\label{sumtriangles}
  Let $\T$ be a triangulated category admitting direct sums indexed by a set $I$.
  If 
  $$\left\{D_i : \begin{tikzcd}X_1 \arrow{r}{f_i} & Y_i \arrow{r}{g_i} & Z_i\arrow{r}{h_i} & X_i[1]\end{tikzcd}\right\}_{i\in I}$$ 
  is a collection of distinguished triangles, then the triangle 
  $$\bigoplus_{i \in I} D_i : \begin{tikzcd} \bigoplus_{i \in I} X_i \arrow{r}{\oplus f_i} & \bigoplus_{i \in I}Y_i \arrow{r}{\oplus g_i} & \bigoplus_{i \in I} Z_i \arrow{r}{\oplus h_i} & \bigoplus_{i \in I} X_i[1]\end{tikzcd}$$
  is distinguished.

  \begin{proof}
    We first note that because $[1]$ is an autoequivalence it is left adjoint to $[-1]$ and hence necessarily commutes with coproducts, so that there is a unique isomorphism $\left(\oplus_{i \in I}X_i\right)[1] \cong \oplus_{i \in I}X_i[1]$.
    By (TR1) and (TR3) we have for each $i$ a morphism of distinguished triangles
    $$\begin{tikzcd} 
      X_i \arrow{r}{f_i}\arrow{d} & Y_i\arrow{d}\arrow{r}{g_i} & Z_i \arrow[dashed]{d}{\exists \gamma_i}\arrow{r}{h_i} & X_i[1] \arrow{d}\\
      \bigoplus_{i \in I} X_i \arrow{r}{\oplus f_i} & \bigoplus_{i \in I}Y_i \arrow{r}{g} & Z \arrow{r}{h} & \bigoplus_{i \in I} X_i[1]
    \end{tikzcd}.$$
    For any object $Z^\prime$ of $\T$, applying the cohomological functor $h_{Z^\prime}$ to the triangle $D_i$ gives a long exact sequence
    $$h_{Z^\prime}(D_i) : 
    \begin{tikzcd} 
      \cdots \arrow{r} & h_{Z^\prime}(Y_i[1]) \arrow{r} & h_{Z^\prime}(X_i[1]) \arrow{r} & h_{Z^\prime}(Z_i) \arrow{r} & h_{Z^\prime}(Y_i) \arrow{r} & h_{Z^\prime}(X_i) \arrow{r} & \cdots
    \end{tikzcd},$$
    and similarly gives a long exact sequence
    $$\begin{tikzcd}
      \cdots \arrow{r} & h_{Z^\prime}\left(\bigoplus_{i \in I}Y_i[1]\right) \arrow{r} & h_{Z^\prime}\left(\bigoplus_{i \in I}X_i[1]\right) \arrow{r} & h_{Z^\prime}(Z) \arrow{r} & h_{Z^\prime}\left(\bigoplus_{i \in I}Y_i\right) \arrow{r} & h_{Z^\prime}\left(\bigoplus_{i \in I} X_i\right) \arrow{r} & \cdots
    \end{tikzcd}$$
    We note that cohomology commutes with direct products, so we obtain the commutative diagram
    $$\begin{tikzcd} 
      \prod_{i \in I}h_{Z^\prime}(Y_i[1]) \arrow{r}\arrow{d}{\alpha_1} & \prod_{i \in I}h_{Z^\prime}(X_i[1]) \arrow{r}\arrow{d}{\alpha_2} & \prod_{i \in I}h_{Z^\prime}(Z_i) \arrow{r}\arrow{d}{\alpha_3} & \prod_{i \in I}h_{Z^\prime}(Y_i) \arrow{r}\arrow{d}{\alpha_4} & \prod_{i \in I}h_{Z^\prime}(X_i)\arrow{d}{\alpha_5}\\
      h_{Z^\prime}\left(\bigoplus_{i \in I}Y_i[1]\right) \arrow{r} & h_{Z^\prime}\left(\bigoplus_{i \in I}X_i[1]\right) \arrow{r} & h_{Z^\prime}(Z) \arrow{r} & h_{Z^\prime}\left(\bigoplus_{i \in I}Y_i\right) \arrow{r} & h_{Z^\prime}\left(\bigoplus_{i \in I} X_i\right)
    \end{tikzcd}$$
    with exact rows and $\alpha_1, \alpha_2, \alpha_4, \alpha_5$ all isomorphisms.
    By the 5 Lemma we see that $\alpha_3$ is an isomorphism, and thus by the Yoneda Lemma
    $$h^Z(Z^\prime) = h_{Z^\prime}(Z) \cong \prod_{i \in I}h_{Z^\prime}(Z_i) \cong h_{Z^\prime}\left(\bigoplus_{i \in I}Z_i\right) = h^{\bigoplus_{i \in I} Z_i}(Z^\prime).$$
    implies there is an isomorphism $\gamma : \bigoplus_{i \in I} Z_i \rightarrow Z$ which yields an isomorphism of triangles
    $$\begin{tikzcd}
      \bigoplus_{i \in I} X_i \arrow{d}{\id}\arrow{r}{\oplus f_i} & \bigoplus_{i \in I}Y_i \arrow{d}{\id}\arrow{r}{\oplus g_i} & \bigoplus_{i \in I} Z_i \arrow{d}{\gamma}\arrow{r}{\oplus h_i} & \bigoplus_{i \in I} X_i[1]\arrow{d}{\id}\\
      \bigoplus_{i \in I} X_i \arrow{r}{\oplus f_i} & \bigoplus_{i \in I}Y_i \arrow{r}{g} & Z \arrow{r}{h} & \bigoplus_{i \in I} X_i[1]
    \end{tikzcd}.$$
    Therefore $\bigoplus_{i \in I}D_i$ is distinguished, as desired.
  \end{proof}
\end{prop}

\begin{cor}\label{corsumtriangles}
  Let $\T$ be a triangulated category and let $X,Y$ be objects of $\T$.
  The triangle
  $$\begin{tikzcd}
    X \arrow{r} & X \oplus Y \arrow{r} & Y \arrow{r}{0} & X[1]
  \end{tikzcd}$$
  is distinguished.
  
  \begin{proof}
    Realize 
    $$\begin{tikzcd}
      X \arrow{r} & X \oplus Y \arrow{r} & Y \arrow{r}{0} & X[1]
    \end{tikzcd}$$
    as the sum of the distinguished triangles
    $$\begin{tikzcd}
      X \arrow{r} & X \arrow{r} & 0\arrow{r} & X[1]
    \end{tikzcd}
    \text{and}
    \begin{tikzcd}
      0 \arrow{r} & Y \arrow{r} & Y \arrow{r} & 0.
    \end{tikzcd}$$
  \end{proof}
\end{cor}

\section{Generators of Triangulated Categories}

\begin{defn}
  Let $\T$ be a triangulated category and let $E$ be an object of $\T$.
  \begin{itemize}
    \item
      Denote by $\langle E \rangle_1$ the strictly full subcategory of $\T$ with objects isomorphic to direct summands of finite direct sums
      $$\bigoplus_{i = 1}^r E[d_i].$$
    \item
      For $1 < n$, denote by $\langle E \rangle_n$ the full subcategory of $\T$ with objects isomorphic to direct summands of objects $X$ which fit into a distinguished triangle
      $$A \rightarrow X \rightarrow B \rightarrow A[1]$$
      with $A$ an object of $\langle E \rangle_1$ and $B$ an object of $\langle E \rangle_{n-1}$.
    \item
      Denote by $\langle E \rangle$ the subcategory of $\T$ with objects
      $$\bigcup_{n>0} \langle E \rangle_n.$$
  \end{itemize}
\end{defn}

\begin{lem}
  Let $\T$ be a triangulated category and let $E$ be an object of $\T$.
  The category $\langle E \rangle_n$ is a strictly full, additive subcategory of $\T$ preserved under shifts and under taking summands.
  
  \begin{proof}
    We proceed by induction on $n$.
    We first note that it's clear from the definition that $\langle E \rangle_n$ is strictly full, closed under taking summands, and preadditive since $\Hom{\langle E \rangle_n}{X,Y} = \Hom{\T}{X,Y}$.
    Thus it remains only to show that $\langle E \rangle_n$ is[ closed under sums, and closed under shifts.
    
    The case $n = 1$ is clear from the definition.
    Assume the result holds up to some $1 < n$, and let $E^\prime$ be an object of $\langle E \rangle_n$.
    Choose an object $X$ fitting in a distinguished triangle
    $$\begin{tikzcd}
      A \arrow{r}{f} & X \arrow{r}{g} & B \arrow{r}{h} & A[1]
    \end{tikzcd}$$
    such that $X \cong E \oplus E^{\prime\prime}$, $A$ is an object of $\langle E \rangle_1$, and $B$ is an object of $\langle E \rangle_{n-1}$.
    We note that the triangle
    $$\begin{tikzcd}
      A[d] \arrow{r}{f[d]} & X[d] \arrow{r}{g[d]} & Y[d] \arrow{r}{h[d]} & A[d+1]
    \end{tikzcd}$$
    is distinguished, and so by induction $A[d]$ is an object of $\langle E \rangle_1$ and $B[d]$ is an object of $\langle E \rangle_{n-1}$.
    Since $X[d] \cong E^\prime[d] \oplus E^{\prime\prime}[d]$, it follows that $E^\prime[d]$ is an object of $\langle E \rangle_n$.
    
    Similarly, if $E^\prime, E^{\prime\prime}$ are objects of $\langle E \rangle_n$, then we have objects $X, X^\prime$ fitting into distinguished triangles
    $$\begin{tikzcd}
      A \arrow{r}{f} & X \arrow{r}{g} & B \arrow{r}{h} & A[1]
    \end{tikzcd}$$
    and
    $$\begin{tikzcd}
      A^\prime \arrow{r}{f^\prime} & X^\prime \arrow{r}{g^\prime} & B^\prime \arrow{r}{h} & A^\prime[1]
    \end{tikzcd}$$
    By the induction hypothesis and Proposition~\ref{sumtriangles} the triangle
    $$\begin{tikzcd}
      A \oplus A^\prime \arrow{r}{f \oplus f^\prime} & X \oplus X^\prime \arrow{r}{g \oplus g^\prime} & B \oplus B^\prime \arrow{r}{h \oplus h^\prime} & A[1] \oplus A^\prime[1]
    \end{tikzcd}$$
    is distinguished with $A \oplus A^\prime$ an object of $\langle E \rangle_1$ and $B \oplus B^\prime$ an object of $\langle E \rangle_{n-1}$.
    Since $E^\prime \oplus E^{\prime\prime}$ is a summand of $X \oplus X^\prime$, it follows that $E^\prime \oplus E^{\prime\prime}$ is an object of $\langle E \rangle_n$.
  \end{proof}
\end{lem}

\begin{lem}
  Let $\T$ be a triangulated category and let $E$ be an object of $\T$.
  If $A$ is an object of $\langle E \rangle_a$, $B$ is an object of $\langle E \rangle_b$, and 
  \begin{tikzcd} A \arrow{r}{f} & X \arrow{r}{g} & B \arrow{r}{h} & A[1]\end{tikzcd}
  is a distinguished triangle, then $X$ is an object of $\langle E \rangle_{a + b}$.

  \begin{proof}
    Given an object $E^\prime$ of $\langle E \rangle_n$, we have by Corollary~\ref{corsumtriangles} the distinguished triangle
    $$\begin{tikzcd}
      E \arrow{r} & E^\prime \oplus E \arrow{r} & E^\prime \arrow{r} & E[1]
    \end{tikzcd}$$
    which shows $E^\prime \oplus E$ is an object of $\langle E \rangle_{n+1}$.
    Since $\langle E \rangle_{n+1}$ is closed under taking summands, it follows that $E^\prime$ is an object of $\langle E \rangle_{n+1}$.
    Hence $\langle E \rangle_n \subseteq \langle E \rangle_{n+1}$ and so $A$, $B$ are both objects of $\langle E \rangle_{a + b}$.
    Therefore $X$ is an object of $\langle E \rangle_{a + b}$ by Proposition~\ref{fulltriangles}.
  \end{proof}
\end{lem}

\begin{prop}
  Let $\T$ be a triangulated category and let $E$ be an object of $\T$.
  The subcategory
  $$\langle E \rangle = \bigcup_{n > 0} \langle E \rangle_n$$
  is a strictly full, saturated, triangulated subcategory of $\T$ and it is the smallest such subcategory of $\T$ containing $E$.

  \begin{proof}
    By construction, $\langle E \rangle$ is a full subcategory.
    To see that it is strictly full, observe that given an object $E^\prime$ of $\langle E \rangle$ and an isomorphism $E^\prime \rightarrow Y$ we have a distinguished triangle
    $$\begin{tikzcd}
      E^\prime \arrow{r} & Y \arrow{r} & 0 \arrow{r} & E^\prime[1]
    \end{tikzcd}$$
    which implies $Y$ is an object of $\langle E \rangle$ by Proposition~\ref{fulltriangles}.
  \end{proof}
\end{prop}

\begin{lem}
  Let $\T$ be a triangulated category with direct sums.
  If $E_i$ for $i \in I$ is a family of compact objects of $\T$ such that $\oplus{E_i}$ generates $\T$, then every object $X$ of $\T$ can be written as 
  $$X = \operatorname{hocolim}X_n,$$
  where $X_1$ is a direct sum of shifts of the $E_i$ and each transition morphism fits into a distinguished triangle $Y_n \rightarrow X_n \rightarrow X_{n+1} \rightarrow Y_n[1]$, where $Y_n$ is a direct sum of shifts of the $E_i$.
\end{lem}

\begin{lem}
  With the same assumptions and notation as in the Lemma above, if $K$ is a compact object and $K \rightarrow X_n$ is a morphism, then there is a factorization 
  $$K \rightarrow E \rightarrow X_n,$$
  where $E$ is an object of $\langle \oplus_{J}E_j \rangle$ for some finite $J \subseteq I$.
  \begin{proof}
    We proceed by induction.
    When $n = 1$, $X_1 = \oplus_S E_i[m]$ over the triples
    $$S = \left\{(i,m,\varphi) \;\middle\vert\; i \in I, m \in \Z, \varphi : E_i[m] \rightarrow X\right\}.$$
    Because $K$ is compact, we have
    $$\Hom{\T}{K,X_1} \cong \bigoplus_S\Hom{T}{K, E_i[m]}$$
    and thus a morphism $K \rightarrow X_1$ maps under this ismorphism to a finite collection of morphism, $$\left\{K \rightarrow E_i[m]\right\}_{(i,m,\varphi) \in S^\prime}.$$
    Hence $K \rightarrow X_1$ factors as
    $$K \rightarrow \bigoplus_{(i,m,\varphi) \in S^\prime}E_i[m] \rightarrow X_1.$$

    Assume the result holds up to some $1 < n$.
    By assumption, we have a distinguished triangle
    $$Y_{n-1} \rightarrow X_{n-1} \rightarrow X_{n} \rightarrow Y_{n-1}[1]$$
    and, since $Y_{n-1}$ is of the same form as $X_1$, we get a factorization
    $$\begin{tikzcd}
      K \arrow{rd}\arrow{r} & X_n \arrow{r} & Y_{n-1}[1]\\
      & E^\prime[1]\arrow{ur}
    \end{tikzcd}$$
    for some $E^\prime \in \langle \oplus_{i \in I^\prime}E_i \rangle$ with $I^\prime \subset I$ finite.
    By the axioms for a triangulated category, we have a morphism of distinguished triangles
    $$\begin{tikzcd}
      K[-1] \arrow{d}\arrow{r} & E^\prime \arrow{r}\arrow{d} & K^\prime \arrow{r}\arrow[dashed]{d}{\exists} & K\arrow{d}\\
      X_n[-1] \arrow{r} & Y_{n-1} \arrow{r} & X_{n-1} \arrow{r} & X_n
    \end{tikzcd}$$
    and we note that $K^\prime$ is compact because $K$ and $E^\prime$ are, and the triangulated subcategory of compact objects is strictly full.
    By induction, the induced morphism factors as
    $$\begin{tikzcd}
      K^\prime \arrow{rr}\arrow{rd} & & X_{n-1}\\
      & E^{\prime\prime}\arrow{ur}
    \end{tikzcd}$$
    for some object $E^{\prime\prime} \in \langle \bigoplus_{i \in I^{\prime\prime}} E_i \rangle$ with finite $I^{\prime\prime} \subseteq I$.
    Taking the composition $E^\prime \rightarrow K^\prime \rightarrow E^{\prime\prime}$ we get a distinguished triangle $E^\prime \rightarrow E^{\prime\prime} \rightarrow E \rightarrow E^\prime[1]$ for some object $E \in \langle \oplus_{i \in I^\prime \cup I^{\prime\prime}}E_i\rangle$ and morphisms of distinguished triangles
    $$\begin{tikzcd}
      E^\prime \arrow{r}\arrow{d} & K^\prime \arrow{r}\arrow{d} & K \arrow{r}\arrow[dashed]{d}{\exists} & E^\prime[1]\arrow{d}\\
      E^\prime \arrow{r}\arrow{d} & E^{\prime\prime} \arrow{r}\arrow{d} & E \arrow{r}\arrow[dashed]{d}{\exists} & E^\prime[1]\arrow{d}\\
      Y_{n-1} \arrow{r} & X_{n-1} \arrow{r} & X_n \arrow{r} & Y_{n-1}[1]
    \end{tikzcd}$$
    
    The composition of the induced morphisms gives a morphism $K \rightarrow X_n$, which need not be the original morphism, however the compositions with $X_n \rightarrow Y_{n-1}[1]$ agree by construction.
    We observe that because $h^{K}$ is a homological functor, the distinguished triangle $Y_{n-1} \rightarrow X_{n-1} \rightarrow X_{n} \rightarrow Y_{n-1}[1]$ gives rise to an exact sequence
    $$\begin{tikzcd}
      \Hom{\T}{K,X_{n-1}} \arrow{r} & \Hom{\T}{K,X_n} \arrow{r} & \Hom{\T}{K,Y_{n-1}[1]}
    \end{tikzcd}$$
    and, by the observation above, the difference of these two morphisms can be expressed as a morphism $K \rightarrow X_{n-1} \rightarrow X_n$.
    By induction, we have a factorization 
    $$\begin{tikzcd}
      K \arrow{rd}\arrow{rr} && X_{n-1}\\
      & E^{\prime\prime\prime}\arrow{ur}
    \end{tikzcd}$$
    through an object of $\langle \oplus_{i \in I^{\prime\prime\prime}} E_i\rangle$ with $I^{\prime\prime\prime} \subseteq I$ finite.
    Then the composition of the induced morphisms
    $$\begin{tikzcd}
      & K\arrow[dashed]{d}{\exists !}\arrow{ld}\arrow{rd}\\
      E\arrow{rd} & E \oplus E^{\prime\prime\prime} \arrow[dashed]{d}{\exists!}\arrow{r}\arrow{l} & E^{\prime\prime\prime}\arrow{d}\arrow{ld}\\
      & X_n & \arrow{l}X_{n-1}
    \end{tikzcd}$$
    gives the desired morphism with $E \oplus E^{\prime\prime\prime} \in \langle \oplus_{i \in I^\prime \cup I^{\prime\prime} \cup I^{\prime\prime\prime}} E_i\rangle.$
  \end{proof}
\end{lem}

\begin{defn}
  Let $\T$ be a triangulated category with direct sums.
  We say $\T$ is compactly generated if there exists a family $E_i$, $i \in I$, of compact objects such that $\oplus_{i \in I} E_i$ generates $\T$.
\end{defn}

\begin{prop}
  Let $\T$ be a triangulated category with direct sums.
  If $E$ is a compact object of $\T$, then
  $E$ is a classical generator for $\T_c$ and $\T$ is compactly generated 
  if and only if
  $E$ is a generator for $\T$.
  
  \begin{proof}
    First assume that $E$ is a classical generator for $\T_c$.
    Because $\T_c$ is a strictly full subcategory and $\T = \langle E \rangle$, it follows that $\T_c = \langle E \rangle = \T$.
    
  \end{proof}
\end{prop}

\begin{thm}
  Let $\mathcal{C}$ and $\mathcal{D}$ be pretriangulated dg-categories.
  If $E_i$ is a family of generators for $\mathcal{C}$ and $F_i$ is a family of generators for $\mathcal{D}$, then the $E_i \otimes F_j$ are a family of generators for $D\left(\mathcal{C} \otimes \mathcal{D}\right)$.
\end{thm}

\end{document}
p
