\documentclass[dissertation.tex]{subfiles}
\begin{document}
\subsection{Enriched Categories}
\begin{defn}[\cite{Borceux2}]
  Let $\mathscr{V}$ be a monoidal category.
  A $\mathscr{V}$-category, $\CC$, is a category such that $\CC(X,Y)$ is an object of $\mathscr{V}$ for all objects $X$, $Y$ of $\CC$ with an associative composition 
  $$c_{XYZ} \colon \CC(X,Y) \otimes \CC(Y,Z) \to \CC(X,Z)$$
  making the diagram
  $$\begin{tikzcd}
    \left(\CC(X,Y) \otimes \CC(Y,Z)\right) \otimes \CC(Z,W)\arrow{rr}{c_{XYZ} \otimes 1}\arrow{dd}{a_{\CC(X,Y),\CC(Y,Z),\CC(Z,W)}} && \CC(X,Z) \otimes \CC(Z,W)\arrow{dddd}{c_{XZW}}\\
    \\
    \CC(X,Y) \otimes \left(\CC(Y,Z) \otimes \CC(Z,W)\right)\arrow{dd}{1 \otimes c_{YZW}}\\
    \\
    \CC(X,Y) \otimes \CC(Y,W)\arrow{rr}{c_{XYW}} && \CC(X,W)
  \end{tikzcd}$$
  commutes for all quadruples $X,Y,Z,W$ of $\CC$
  and a unit morphism
  $$u_X \colon I \to \CC(X,X)$$
  for each object $X$ of $\CC$ making the diagram 
  $$\begin{tikzcd}
    I \otimes \CC(X,Y) \arrow{rr}{l_\CC(X,Y)}\arrow{dd}{u_X \otimes 1} && \CC(X,Y) \arrow{dd}{1}&& \CC(X,Y) \otimes I \arrow[swap]{ll}{r_{\CC(X,Y)}}\arrow{dd}{1 \otimes u_Y}\\
    \\
    \CC(X,X) \otimes \CC(X,Y) \arrow{rr}{c_{XXY}} && \CC(X,Y) && \CC(X,Y) \otimes \CC(Y,Y)\arrow[swap]{ll}{c_{XYY}}
  \end{tikzcd}$$
  commute.
\end{defn}

\begin{defn}[\cite{Borceux2}]
  Let $\mathscr{V}$ be a monoidal category.
  Given $\mathscr{V}$-categories $\CC$, $\D$, a $\mathscr{V}$-functor $\F : \CC \to \D$ is a functor such that for each pair of objects $X,X^\prime$ of $\CC$
  $$\F_{X,X^\prime} \colon \CC(X,X^\prime) \to \D(\F(X),\F(X^\prime))$$
  is a morphism of $\mathscr{V}$
  such that for all triples $X,X^\prime, X^{\prime\prime}$ of $\CC$ the diagram
  $$\begin{tikzcd}
    \CC(X,X^\prime) \otimes \CC(X^\prime, X^{\prime\prime}) \arrow{rr}{C_{XX{^\prime}X^{\prime\prime}}}\arrow{dd}{\F_{XX^\prime} \otimes \F_{X^\prime X^{\prime\prime}}} && \CC(X,X^{\prime\prime})\arrow{dd}{\F_{XX^{\prime\prime}}}\\
    \\
    \D(\F X, \F X^\prime) \otimes \D(\F X^\prime, \F X^{\prime\prime}) \arrow{rr}{c_{\F X,\F X^\prime,\F X^{\prime\prime}}}&& \D(\F X, \F X^{\prime\prime})
  \end{tikzcd}$$
  commutes (composition)
  and for every object $X$ of $\CC$ the diagram
  $$\begin{tikzcd}
    I \arrow{rr}{u_X}\arrow{rrdd}{u_{\F X}} && \CC(X,X)\arrow{dd}{\F_{XX}}\\
    \\
    &&\D(\F X, \F X)
  \end{tikzcd}$$
  commutes (units).
\end{defn}

\begin{defn}[\cite{Borceux2}]
  Let $\mathscr{V}$ be a monoidal category.
  Let $\CC$,$\D$, be two $\mathscr{V}$-categories and let $\F,\G : \CC \to \D$ be two $\mathscr{V}$-functors.
  A $\mathscr{V}$-natural transformation $\eta \colon \F \to \G$ is a collection of morphisms
  $$\eta_X \colon I \to \D(\F(A), \G(A))$$
  of $\mathscr{V}$ making the diagram
  $$\begin{tikzcd}
    && \CC(X,X^\prime) \arrow[swap]{lldd}{l^{-1}_{\CC(X,X^\prime)}} \arrow{rrdd}{r^{-1}_{\CC(X,X^\prime)}}\\
    \\
    I \otimes \CC(X,X^\prime)\arrow{dd}{\eta_X \otimes \G_{XX^\prime}} &&&& \CC(X,X^\prime) \otimes I\arrow{dd}{\F_{X,X^\prime} \otimes \eta_{X^\prime}}\\
    \\
    \D(\F X, \G X) \otimes \D(\G X, \G X^\prime) \arrow{rrdd}{c_{\F X, \G X, \G X^\prime}} &&&& \D(\F X, \F X^\prime) \otimes \D(\F X^\prime, \G X^\prime) \arrow[swap]{lldd}{c_{\F X, \F X^\prime, \G X^\prime}}\\
    \\
    && \D(\F X, \G X^\prime)
  \end{tikzcd}$$
  commute (naturality).
\end{defn}

\begin{prop}[\cite{Borceux2}]
  %TODO: Should probably define 2-categories somewhere...
  Let $\V$ be a monoidal category.
  The small $\V$ categories together with $\V$-functors and $\V$-natural transformations form a 2-categoriy, $\V$-cat.
\end{prop}

\begin{prop}[\cite{Borceux2}]
  Let $\V$ be a symmetric monoidal category.
  The category of small $\V$-categories and $\V$-functors is itself provided with the structure of a symmetric monoidal category.
\end{prop}

%TODO: This might require some of the nonsense about $\V$-distributors..?
\begin{prop}[\cite{Borceux2}]
  Let $\V$ be a complete symmetric monoidal closed category.
  Given two $\V$-categories, $\CC$, $\D$, with $\CC$ small, the category of $\V$-functors $\CC \to \D$ and $\V$-natural transformations can be provided with the structure of a $\V$-category, written $\V[\CC,\D]$.
\end{prop}

\begin{cor}[\cite{Borceux2}]
  Let $\V$ be a complete symmetric monoidal closed category.
  The category of small $\V$-categories and $\V$-fucntors is itself provided with the structure of a symmetric monoidal closed category.
\end{cor}

\begin{thm}[Enriched Yoneda Lemma, \cite{Borceux2}]
  Let $\V$ be a symmetric monoidal closed category and $\CC$ a small $\V$-category.
  For every object $X$ of $\CC$ and every $\V$-functor $\F \colon \CC \to \V$, the object of $\V$-natural transformations from $\CC(X,-)$ to $\F$ exists and there is an isomorphism in $\V$
  $$\operatorname{\V-Nat}(\CC(X,-),\F) \cong \F(X),$$
  natural in both $\F$ and $X$.
\end{thm}

\begin{cor}[Enriched Yoneda Embedding, \cite{Borceux2}]
  Let $\V$ be a complete symmetric monoidal closed category.
  For every small $\V$-category, $\CC$, the mapping
  $$Y \colon \CC \to \operatorname{\V-Nat},\, X \mapsto \CC(X,-)$$
  can be extended to a $\V$-functor, called the $\V$-Yoneda Embedding.
\end{cor}
\end{document}
