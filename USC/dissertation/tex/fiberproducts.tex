\documentclass[dissertation.tex]{subfiles}
\begin{document}

\begin{defn}[Fiber Products]
  For any category $\mathscr{C}$, given objects $A$, $B$, $C$ of $\mathscr{C}$, and morphisms
  $$\begin{tikzcd}
    A \arrow{r}{f} & C & \arrow[swap]{l}{g} B
  \end{tikzcd}$$
  we define the {\em fiber product} $A \times_C B$ to be an object of $C$ equipped with morphisms called {\em projections} making the diagram
  $$\begin{tikzcd}
    A \times_C B \arrow{d}{p} \arrow{r}{q} & B\arrow{d}{g}\\
    A\arrow{r}{f} & C
  \end{tikzcd}$$
  commute and final amongst all such objects.
  That is to say, for any object $Z$ equipped with projects to $A$ and $B$, the following diagram commutes
  $$\begin{tikzcd}
    Z\arrow[bend right]{ddr}\arrow[bend left]{rrd} \arrow[dotted]{rd}{\exists ! h}& &\\
    & A \times_C B \arrow{r}{q}\arrow{d}{p} & B\arrow{d}{g}\\
    & A\arrow{r}{f} & C
  \end{tikzcd}$$
  Note that by definition $A \times_C B$ is universal and hence unique up to unique isomorphism.
\end{defn}

The category $\mathfrak{Set}$ has all fiber products.  Given set maps $f: A \rightarrow C$ and $g : B \rightarrow C$, it's straightforward to check that
$$A \times_C B = \left\{(a,b) \in A \times B \;\middle\vert\; f(a) = g(b) \in C\right\}$$
satisfies the universal property.

\begin{prop}
  For $\mathscr{C}$ locally small, $\hat{\mathscr{C}}$ has all fiber products.
  
  \begin{proof}
    Given $A$, $B$, and $C$ presheaves on $\mathscr{C}$ and natural transformations
    $$\begin{tikzcd}A \arrow{r}{f} & C & B \arrow[swap]{l}{g}\end{tikzcd}$$
    define the presheaf $A \times_C B$ with object map that takes an object $X$ of $\mathscr{C}$ to $A(X) \times_{C(X)} B(X)$ and arrow map that takes a morphism $\varphi : Y \rightarrow X$ of $\mathscr{C}$ to the unique map of sets
    $$\begin{tikzcd}
      A \times_{C} B(X) \arrow{rr}{q_X}\arrow{dd}{p_X} \arrow[dotted]{rd}{\exists !A \times_C B (\varphi)}& & B(X)\arrow[dashed]{dd}[pos=0.6]{g_X}\arrow{rd}{B(\varphi)}&\\
      & A \times_{C} B(Y)\arrow{dd}[description, pos=0.39]{p_Y} \arrow{rr}[pos=0.4,description]{q_Y} & & B(Y)\arrow{dd}{g_Y}\\
      A(X)\arrow{rd}{A(\varphi)}\arrow[dashed]{rr}[pos=0.4]{f_X} & & C(X)\arrow[dashed]{rd}{C(\varphi)} & \\
      & A(Y)\arrow{rr}{f_Y} & & C(Y)
    \end{tikzcd}$$
    commute.
    We can write this morphism of sets explicitly as
    \begin{align*}
      A \times_C B(\varphi) \colon A \times_C B(X) &\rightarrow A \times_C B (Y)\\
      (a,b) &\mapsto (A(\varphi)(a), B(\varphi)(b))
    \end{align*}
    since the left face of the cube commutes by
    $$p_Y \circ A \times_C B(a,b)  = p_Y(A(\varphi)(a), B(\varphi(b))) = A(\varphi)(a) = A(\varphi) \circ p_X(a,b),$$
    the top face of the cube commutes by
    $$q_Y \circ A \times_C B(a,b)  = q_Y(A(\varphi)(b), B(\varphi(b))) = A(\varphi)(b) = A(\varphi) \circ q_X(a,b),$$
    the back and front faces of the cube commutes by assumption, and the other two faces of the cube commute by naturality of $f$ and $g$.
    Moreover, this shows that $A \times_C B$ is equipped with two natural projections,
    $$\begin{tikzcd}
      A & A \times_C B \arrow{r}{q} \arrow[swap]{l}{p} & B
    \end{tikzcd}$$
    and so it remains only to show that this object is indeed final amongst all such objects of $\hat{\mathscr{C}}$.

    Suppose that we have another presheaf $Z$ equipped with projections
    $$\begin{tikzcd}
      A & Z \arrow{r}{\beta} \arrow[swap]{l}{\alpha} & B
    \end{tikzcd}$$
    Given any object $X$ of $\mathscr{C}$, the morphisms $\alpha_X$ and $\beta_X$ induce by the universal property the morphism of sets
    \begin{align*}
      (\alpha \times \beta)_X \colon Z(X) &\rightarrow A \times_C B(X)\\
      z &\mapsto (\alpha(z), \beta(z)),
    \end{align*}
    making the diagram
    $$\begin{tikzcd}
      Z(X)\arrow[bend left]{rrd}{\beta_X}\arrow[bend right]{rdd}{\alpha_X} \arrow[dotted]{rd}{\exists !(\alpha \times \beta)_X} \\
      & A \times_C B(X) \arrow{r}{q_X}\arrow{d}{p_X} & B(X)\arrow{d}{g_X}\\
      & A(X) \arrow{r}{f_X} & C(X).
    \end{tikzcd}$$
    For any morphism $\varphi \colon Y \rightarrow X$ of $\mathscr{C}$, naturality of $\alpha$ and $\beta$ ensure that the diagram
    $$\begin{tikzcd}
      A(X)\arrow{d}{A(\varphi)} & Z(X)\arrow[swap]{l}{\alpha_X}\arrow{r}{\beta_X}\arrow{d}{Z(\varphi)} & B(X)\arrow{d}{B(\varphi)}\\
      A(Y) & Z(Y)\arrow[swap]{l}{\alpha_Y}\arrow{r}{\beta_Y} & B(Y)
    \end{tikzcd}$$
    commutes.
    This is enough to ensure that the naturality square
    $$\begin{tikzcd}
      Z(X) \arrow[swap]{d}{(\alpha \times \beta)_X}\arrow{r}{Z(\varphi)}& Z(Y)\arrow{d}{(\alpha \times \beta)_Y}\\
      A \times_C B(X) \arrow{r}{A \times_C B(\varphi)} & A \times_C B(Y)
    \end{tikzcd}$$
    commutes because
    \begin{eqnarray*}
      A \times_C B (\varphi) \circ (\alpha \times \beta)_X(z) &=& A \times_C B (\varphi)(\alpha_X(z), \beta_X(z))\\
      &=& (A(\varphi) \circ \alpha_X(z), B(\varphi) \circ \beta_X(z))\\
      &=& (\alpha_Y \circ Z(\varphi)(z), \beta_Y \circ Z(\varphi)(z))\\
      &=& (\alpha \times \beta)_Y \circ Z(\varphi)(z)
    \end{eqnarray*}
    and hence $\alpha \times \beta \colon Z \rightarrow A \times_C B$ is a natural transformation.
    The uniquness follows from the fact that each of the components were defined from the universal property in $\mathfrak{Set}$.
  \end{proof}
\end{prop}

\begin{prop}\label{p.4}
  Let $\mathscr{C}$ be a locally small category.
  Given natural transformations $f \colon h_X \rightarrow h_S$ and \\$g \colon h_Y \rightarrow h_S$, the fiber product $h_X \times_{h_S} h_Y$ in $\hat{\mathscr{C}}$ is representable if and only if the fiber $X \times_S Y$ exists in $\mathscr{C}$ and, in this case, is represented by $X \times_S Y$.
  
  %    While the notation $X \times_S Y$ is potentially ambiguous (the fiber prod%uct explicitly depends on the choice of maps $X \rightarrow S$ and $Y \rightar%row S$), it will become clear in the course of the proof that there can be no %confusion.
  
  \begin{proof}
    We first note that by Yoneda's Lemma, we have two isomorphisms $\Nat{h_X,h_S} \cong h_S(X)$ and $\Nat{h_Y, h_S} \cong h_S(Y)$, so we let $\varphi: X \rightarrow S$ be the image of $f$ and $\psi: Y \rightarrow S$ be the image of $g$.
    Assume that the fibered product
    $$\begin{tikzcd}
      X \times_S Y \arrow{r}{q}\arrow{d}{p} & Y\arrow{d}{\psi}\\
      X \arrow{r}{\varphi} & S
    \end{tikzcd}$$
    exists in $\mathscr{C}$.
    Using the naturality of the Yoneda embedding in the second slot, the natural transformations $f$ and $g$ induce commutative diagrams
    \begin{center}
      \begin{tikzcd}
        h_X(X \times_S Y) \arrow{d}[rotate=90,xshift=-1ex,yshift=1ex]{\sim}\arrow{r}{\varphi_*} & h_S(X \times_S Y)\arrow{d}[rotate=90,xshift=-1ex,yshift=1ex]{\sim}\\
        \Nat{h_{X \times_S Y}, h_X} \arrow{r}{f_*} & \Nat{h_{X \times_S Y}, h_S}
      \end{tikzcd}
      and
      \begin{tikzcd}
        h_Y(X \times_S Y) \arrow{d}[rotate=90,xshift=-1ex,yshift=1ex]{\sim}\arrow{r}{\psi_*} & h_S(X \times_S Y)\arrow{d}[rotate=90,xshift=-1ex,yshift=1ex]{\sim}\\
        \Nat{h_{X \times_S Y}, h_Y} \arrow{r}{g_*} & \Nat{h_{X \times_S Y}, h_S}
      \end{tikzcd}
    \end{center}
    which gives us a commutative diagram
    $$\begin{tikzcd}
      h_{X \times_S Y}\arrow[bend right]{ddr}\arrow[bend left]{rrd}\arrow[dotted]{rd}{\exists !h} \\
      &h_X \times_{h_S} h_Y \arrow{d}\arrow{r}& h_Y\arrow{d}{g}\\
      &h_X\arrow{r}{f} & h_Y
    \end{tikzcd}$$
    where the projections from $h_{X \times_S Y}$ are the images of $p$ and $q$ in $\Nat{h_{X \times_S Y}, h_X}$ and $\Nat{h_{X \times_S Y}, h_Y}$, respectively.
    It is enough to show that each morphism $\gamma : Z \rightarrow X \times_S Y$ comes from an element of $h_X \times_{h_S} h_Y(Z)$, but this follows from the universal property:  the element $h_Z(\gamma)$ is a pair of morphisms $\alpha : Z \rightarrow X$, $\beta : Z \rightarrow Y$ such that the diagram
    $$\begin{tikzcd}
      Z \arrow{r}{\beta}\arrow{d}{\alpha} & Y\arrow{d}{\psi}\\
      X \arrow{r}{\varphi} & S
    \end{tikzcd}$$
    commutes, which of course induces a unique morphism of $h_{X \times_S Y}(Z)$, and this must be $\gamma$.
    Therefore $h_Z$ is an isomorphism of sets, and so $h$ is a natural isomorphism.

    Conversely, suppose that $h_X \times_{h_S} h_Y$ is represented by an object $Z$ of $\mathscr{C}$.
    The projections $h_Z \rightarrow h_X$ and $h_Z \rightarrow h_Y$ induce projections $p : Z \rightarrow X$ and $q : Z \rightarrow Y$ making the diagram
    $$\begin{tikzcd}
      Z \arrow{r}{q}\arrow{d}{p} & Y\arrow{d}{\psi}\\
      X\arrow{r}{\varphi} & S
    \end{tikzcd}$$
    commute.
    Given any other object $W$ equipped projections, $W \rightarrow X$ and $W \rightarrow Y$, we obtain morphisms $\alpha : h_W \rightarrow h_X$ and $\beta : h_W \rightarrow h_Y$ making the diagram
    $$\begin{tikzcd}
      h_W\arrow[dotted]{rd}{\exists !}\arrow[bend left]{rrd}{\beta}\arrow[bend right]{ddr}{\alpha}\\
      & h_Z \arrow{r}\arrow{d}& h_Y\arrow{d}{g}\\
      & h_X \arrow{r}{f}& h_S
    \end{tikzcd}$$
    commute.
    This of course induces a unique morphism $W \rightarrow Z$ by the natural isomorphism
    $$\Nat{h_W, h_Z} \cong h_W(Z).$$
    Therefore, as $Z$ satisfies the universal property, we have $Z \cong X \times_S Y$.
  \end{proof}
\end{prop}

Some nice properties derived from fiber products.

\begin{prop}\label{p.5}
  A morphism $f : A \rightarrow B$ of $\mathscr{C}$ is a monomorphism if and only if the fiber product $A \times_B A$ exists and the diagram
  $$\begin{tikzcd}
    A \arrow{d}{\op{id}_A}\arrow{r}{\op{id}_A} & A\arrow{d}{f}\\
    A\arrow{r}{f} & B
  \end{tikzcd}$$
  is a fiber product diagram.

  \begin{proof}
    First assume that the fiber product exists and the above diagram is a fiber product diagram.
    Given a commutative diagram
    $$\begin{tikzcd}
      Z \arrow[yshift=0.5ex]{r}{g} \arrow[yshift=-0.5ex,swap]{r}{h}& A \arrow{r}{f}& B
    \end{tikzcd}$$
    we have an induced diagram
    $$\begin{tikzcd}
      Z \arrow[bend right]{ddr}{g} \arrow[bend left]{rrd}{h} \arrow[dotted]{rd}{\exists !h^\prime}\\
      & A\arrow{d}{\op{id}_A}\arrow{r}{\op{id}_A} & A\arrow{d}{f}\\
      & A\arrow{r}{f} & B.
    \end{tikzcd}$$
    By assumption, taking $h^\prime = h$ or $h^\prime = g$ makes the diagram commute, so by uniqueness of $h^\prime$ we have $h = g$.
    Therefore $f$ is monic.

    Conversely, assume that $f$ is monic.
    It suffices to show that $A$ equipped with the two identity morphisms as projections satisfies the universal property for the fiber product.
    Given any object $Z$ equipped with projections $g : Z \rightarrow A$ and $h : Z \rightarrow A$ making the diagram
    $$\begin{tikzcd}
      Z \arrow{d}{g} \arrow{r}{h} & A\arrow{d}{f}\\
      A\arrow{r}{f} & B
    \end{tikzcd}$$
    commute, we see that because $f$ is monic,
    $f \circ g = f \circ h$
    and thus $g = h$.
    Therefore we have a unique morphism $g = h$ making the diagram
    $$\begin{tikzcd}
      Z \arrow[bend right]{ddr}{g} \arrow[bend left]{rrd}{h} \arrow[dotted]{rd}{\exists !g = h}\\
      & A\arrow{d}{\op{id}_A}\arrow{r}{\op{id}_A} & A\arrow{d}{f}\\
      & A\arrow{r}{f} & B.
    \end{tikzcd}$$
    commute, as desired.
  \end{proof}
\end{prop}

\begin{cor}
  The Yoneda functor preserves monomorphisms.

  \begin{proof}
    Given a monomorphism $f \colon A \rightarrow B$ of $\mathscr{C}$, we must show that the induced morphism $\hat{f} \colon h_A \rightarrow h_B$ is a monomorphism.
    By Proposition~\ref{p.5} and the fact that $\hat{\op{id}_A} = \op{id}_{h_A}$, this is equivalent to showing that the diagram
    $$\begin{tikzcd}
      h_A \arrow{d}{\op{id}_{h_A}}\arrow{r}{\op{id}_{h_A}} & h_A\arrow{d}{\hat{f}}\\
    h_A \arrow{r}{\hat{f}} & h_B
  \end{tikzcd}$$
    is a fiber product diagram.
    But this follows from Proposition~\ref{p.4}: $h_A \times_{h_B} h_A$ exists and is represented by $A \times_B A \cong A$ if and only if $A \times_B A$ exists.
    Therefore $h_A \times_{h_B} h_A \cong h_{A \times_B A} \cong h_A$, and so it follows that $\hat{f}$ is a monomorphism.
  \end{proof}
\end{cor}
\end{document}
