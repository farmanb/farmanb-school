\documentclass[dissertation.tex]{subfiles}
\begin{document}
\subsection{Triangulated Categories}

\begin{defn}
  Let $\CC$ be a category.
  \begin{itemize}
  \item
    We say a subcategory $\CC^\prime$ of $\CC$ is {\it full} if 
    $\Hom{\CC^\prime}{X,Y} = \Hom{\CC}{X,Y}$
    for all objects $X,Y$ of $\CC^\prime$.
  \item
    We say a full subcategory $\CC^\prime$ of $\CC$ is {\it strictly full} if whenever $X \rightarrow Y$ is an isomorphism with $X$ an object of $\CC^\prime$, then $Y$ is an object of $\CC^\prime$.
  \end{itemize}
\end{defn}

\begin{defn}
  Let $\T$ be an additive category equipped with autoequivalences of categories, $[n] : \CC \rightarrow \CC$ for $n \in \Z$, and a class of distinguished triangles.
  We say $\T$ is {\it triangulated} if the distinguished triangles satisfy the following axioms:
  \begin{description}[style=nextline]
    \item[TR1]\label{TR1}
      \begin{itemize}
      \item
        Every morphism $X \rightarrow Y$ can be embedded in a distinguished triangle $X \rightarrow Y \rightarrow Z \rightarrow X[1]$.
      \item
        The triangle \begin{tikzcd} X \arrow{r}{\id_X} & X \arrow{r} & 0 \arrow{r} & X[1]\end{tikzcd} is distinguished.
      \item
        Any triangle isomorphic to a distinguished triangle is also distinguished.
      \end{itemize}
    \item[TR2]\label{TR2}
      If 
      \begin{tikzcd}
        X \arrow{r}{f} & Y \arrow{r}{g} & Z \arrow{r}{h} & X[1]
      \end{tikzcd} 
      is a distinguished triangle, then the two rotated triangles
      $$\begin{tikzcd}
        Y \arrow{r}{g} & Z \arrow{r}{h} & X[1] \arrow{r}{-f[1]} & Y[1]
      \end{tikzcd}$$
      and
      $$\begin{tikzcd}
        Z[-1] \arrow{r}{-h[1]} & X \arrow{r}{f} & Y \arrow{r}{g} & Z
      \end{tikzcd}$$
      are also distinguished.
    \item[TR3]\label{TR3}
      Given distinguished triangles
      $$\begin{tikzcd}
        X_i \arrow{r}{f_i} & Y_i\arrow{r}{g_i} & Z_i \arrow{r}{h_i} & Z_i
      \end{tikzcd}$$
      for $i = 1, 2$ and a diagram
      $$\begin{tikzcd}
        X_1 \arrow{d}{\alpha}\arrow{r}{f_1} & Y_1 \arrow{d}{\beta}\arrow{r}{g_1} & Z_1 \arrow[dashed]{d}{\exists \gamma}\arrow{r}{h_1} & X_1[1]\arrow{d}{\alpha[1]}\\
        X_2 {\alpha}\arrow{r}{f_2} & Y_2 \arrow{r}{g_2} & Z_2 \arrow{r}{h_2} & X_2[1]
      \end{tikzcd}$$
      such that the left-hand square commutes, the dashed arrow can be filled in with a morphism $\gamma$, which is not necessarily unique.
      
    \item[TR4]\label{TR4}
      Given three distinguished triangles
      $$\begin{tikzcd}
        X \arrow{r}{f} & Y \arrow{r}{i} & X^\prime \arrow{r}{\ell} & X[1]\\
        X \arrow{r}{g \circ f} & Z \arrow{r}{j} & Y^\prime \arrow{r}{m} & X[1]\\
        Y \arrow{r}{g} & Z \arrow{r}{k} & Z^\prime \arrow{r}{n} & Y[1]
      \end{tikzcd}$$
      then there exists a distinguished triangle
      $$\begin{tikzcd}
        X^\prime \arrow{r}{u} & Y^\prime \arrow{r}{v} & Z^\prime \arrow{r}{i[1]\circ n} & X^\prime[1]
      \end{tikzcd}$$
      making the diagram
      $$\begin{tikzcd}
        X\arrow{d}{\id_X}\arrow{r}{f} & Y\arrow{r}{i}\arrow{d}{g} & X^\prime\arrow{r}{\ell}\arrow{d}{u} & X[1]\arrow{d}{id_{X[1]}}\\
        X \arrow{d}{f}\arrow{r}{g \circ f} & Z\arrow{r}{j}\arrow{d}{\id_Z} & Y^\prime\arrow{r}{m}\arrow{d}{v} & X[1]\arrow{d}{f[1]}\\
        Y\arrow{r}{g}\arrow{d}{i} & Z \arrow{r}{k}\arrow{d}{j} & Z^\prime \arrow{r}{n}\arrow{d}{\id_{Z^\prime}} & Y[1]\arrow{d}{i[1]}\\
        X^\prime \arrow{r}{u} & Y^\prime \arrow{r}{v} & Z^\prime \arrow{r}{i[1] \circ n} & X^\prime[1]
      \end{tikzcd}$$
  \end{description}
\end{defn}
commute.

\begin{defn}
  Let $\T$ be a triangulated category.
  \begin{itemize}
  \item
    We say a subcategory $\T^\prime$ of $\T$ is a triangulated subcategory if $\T^\prime$ is triangulated and the inclusion functor preserves distinguished triangles.
  \item
    We say a full triangulated subcategory $\T^\prime$ of $\T$ is {\it saturated} if whenever $X \oplus Y$ is isomorphic to an object of $T^\prime$, both $X$ and $Y$ are isomorphic to objects of $T^\prime$.
  \end{itemize}
\end{defn}

\begin{prop}\label{fulltriangles}
  Let $\T$ be a triangulated category, let $\T^\prime$ be a full triangulated subcategory, and let 
  $$D\colon 
  \begin{tikzcd}
    X \arrow{r}{f} & Y \arrow{r}{g} & Z \arrow{r}{h} & X[1] 
  \end{tikzcd}$$
  be a distinguished triangle of $\T$.
  \begin{enumerate}
  \item
    If $D$ is distinguished in $\T$ and two out of three of $X,Y,Z$ are objects of $\T^\prime$, then the third is isomorphic to an object of $T^\prime$.
    Moreover, if $T^\prime$ is strictly full, then all three are objects of $T^\prime$.
  \item
    If $D$ is a triangle in $T^\prime$, then it is distinguished in $\T^\prime$.
  \end{enumerate}

  \begin{proof}
    For the first assertian, note that because we may rotate, it suffices to assume that $X$ and $Y$ are objects of $T^\prime$.
    We observe that by (TR1) we obtain a distinguished triangle of $T^\prime$ 
    $$D^\prime : 
    \begin{tikzcd}
      X \arrow{r}{f} & Y\arrow{r}{u} & Z^\prime \arrow{r}{v} & X[1]
    \end{tikzcd},$$
    which is also distinguished in $\T$.
    By (TR3) we have a morphism of distinguished triangles of $\T$ to $D$
    $$\begin{tikzcd}
      D^\prime: & X \arrow{d}{\id_X}\arrow{r}{f} & Y \arrow{d}{\id_Y}\arrow{r}{u} & Z^\prime \arrow[dashed]{d}{\exists \gamma}\arrow{r}{v} & X[1]\arrow{d}{\id_{X[1]}}\\
      D: & X \arrow{r}{f} & Y \arrow{r}{g} & Z \arrow{r}{h} & X[1]
    \end{tikzcd}$$
    which is an isomorphism by the 5 Lemma.
    This proves the first assertian.
    
    For the second, we note that when $Z$ is an object of $\T^\prime$, we have $\Hom{\T^\prime}{Z,Z^\prime} = \Hom{\T}{Z,Z^\prime}$, so that the isomorphism above is an isomorphism of triangles in $T^\prime$.
  \end{proof}
\end{prop}

\begin{prop}
  Let $\T$ be a triangulated category.
  If 
  \begin{tikzcd} X \arrow{r}{f} & Y \arrow{r}{g} & Z \arrow{r}{h} & X[1]\end{tikzcd} 
  is a distinguished triangle, then so is
  $$\begin{tikzcd}X[n] \arrow{r}{f[n]} & Y[n] \arrow{r}{g[n]} & Z[n] \arrow{r}{h[n]} & X[n+1]\end{tikzcd}.$$

  \begin{proof}
    We note that it suffices to show that the result holds for $n = 1$ since $[n] = [n - 1] \circ [1]$.
    We have by (TR2) the successive rotations
    $$\begin{tikzcd}
      Y \arrow{r}{g} & Z\arrow{r}{h} & X[1]\arrow{r}{-f[1]} & Y[1]\\
      Z \arrow{r}{h}& X[1]\arrow{r}{-f[1]} & Y[1]\arrow{r}{-g[1]} & Z[1]\\
      X[1] \arrow{r}{-f[1]} & Y[1] \arrow{r}{-g[1]} & Z[1]\arrow{r}{-h[1]} & X[2],
    \end{tikzcd}$$
    all of which are distinguished triangles.
    We then have an isomorphism of triangles
    $$\begin{tikzcd}
      X[1] \arrow{d}{\id_{X[1]}}\arrow{r}{f[1]} & Y[1] \arrow{d}{-\id_{Y[1]}}\arrow{r}{g[1]} & Z[1]\arrow{d}{\id_{Z[1]}}\arrow{r}{h[1]} & X[2]\arrow{d}{-\id_{X[2]}}\\
      X[1] \arrow{r}{-f[1]} & Y[1] \arrow{r}{-g[1]} & Z[1]\arrow{r}{-h[1]} & X[2].
    \end{tikzcd}$$
    Therefore 
    $$\begin{tikzcd}
      X[1] \arrow{r}{f[1]} & Y[1] \arrow{r}{g[1]} & Z[1]\arrow{r}{h[1]} & X[2]
    \end{tikzcd}$$
    is distinguished, as desired.
  \end{proof}
\end{prop}

\begin{defn}
  Let $\T$ be a triangulated category and let $\A$ be an abelian category.
  An additive functor $H : \T \rightarrow \A$ (resp. $H : \T^\text{op}\rightarrow A$) is called {\it homological} (resp. {\it cohomological}) if for every distinguished triangle 
  \begin{tikzcd}
    X \arrow{r}{f} & Y \arrow{r}{g} & Z \arrow{r}{h} & X[1]
  \end{tikzcd}
  the sequence
  $$\begin{tikzcd}
    HX \arrow{r}{H(f)} & HY \arrow{r}{H(g)} & HZ
  \end{tikzcd}$$
  (resp. $\begin{tikzcd}
    HZ \arrow{r}{H(g)} & HY \arrow{r}{H(f)} & HX
  \end{tikzcd}$)
  is exact in $\A$ 
\end{defn}

\begin{prop}
  Let $\T$ be a triangulated category.
  If 
  \begin{tikzcd}
    X \arrow{r}{f} & Y \arrow{r}{g} & Z \arrow{r}{h} & X[1]
  \end{tikzcd} 
  is a distinguished triangle, then $g \circ f = 0$ and $h \circ g = 0$.

  \begin{proof}
    By (TR3) we obtain a morphism of distinguished triangles
    $$\begin{tikzcd}
      X \arrow{d}{\id_X}\arrow{r}{\id_X} & X \arrow{d}{f}\arrow{r} & 0\arrow[dashed]{d}{\exists !0} \arrow{r} & X[1]\arrow{d}{\id_{X[1]}}\\
      X \arrow{r}{f} & Y \arrow{r}{g} & Z \arrow{r}{h} & X[1]
    \end{tikzcd}$$
    and thus $g \circ f = 0$.

    That $h \circ g = 0$ follows from the same argument applied to the distinguished triangle
    $$\begin{tikzcd}
      Y \arrow{r}{g} & Z \arrow{r}{h} & X[1] \arrow{r}{-g[1]} & Y[1].
    \end{tikzcd}$$
  \end{proof}
\end{prop}

\begin{prop}
  Let $\T$ be a triangulated category.
  For any object $X$ of $\T$, the functor $h^X = \Hom{\T}{X,\_}$ (resp. $h_X = \Hom{\T}{\_,X}$) is homological (resp. cohomological).
  
  \begin{proof}
    Given a distinguished triangle \begin{tikzcd}A \arrow{r}{f} & B \arrow{r}{g} & C \arrow{r}{h} & A[1]\end{tikzcd}, we obtain a complex
    $$\begin{tikzcd}
      \Hom{\T}{X,A} \arrow{r}{f_*} &\Hom{\T}{X,B} \arrow{r}{g_*} & \Hom{\T}{X,C}.
    \end{tikzcd}$$
    since for any morphism $\varphi : X \rightarrow A$ 
    $$g_* \circ f_*(\varphi) = g \circ f \circ \varphi = 0.$$
    We need only show that $f_* = \ker g_*$.
    
    Given a morphism $\psi : X \rightarrow B$ such that $g_*(\psi) = \psi \circ g= 0$, we have a morphism of distinguished triangles by (TR1)
    $$\begin{tikzcd}
      X \arrow{r}\arrow{d}{\psi} & 0 \arrow{r}\arrow{d} & X[1] \arrow[dashed]{d}{\exists \gamma}\arrow{r}{\id_{X[1]}} & X[1]\arrow{d}{\psi[1]}\\
      B \arrow{r}{g} & C \arrow{r}{h} & A[1] \arrow{r}{-f[1]} & B[1]
    \end{tikzcd},$$
    which yields $f_*(-\gamma[-1]) = -f \circ \gamma[-1] = \psi$.
    Therefore $f_* = \ker g_*$, as desired.
    
    By replacing $\T$ by $\T^\text{op}$, we see that $h_X$ is cohomological.
  \end{proof}
\end{prop}


\begin{prop}\label{isotriangle}
  Let $\T$ be a triangulated category.
  If 
  \begin{tikzcd}
    X \arrow{r}{f} & Y \arrow{r} & Z \arrow{r} & X[1]
  \end{tikzcd} is a distinguished triangle, then $f$ is an isomorphism if and only if $Z \cong 0$.
  
  \begin{proof}
    Assume that $f$ is an isomorphism.
    We have by (TR3) a morphism of triangles
    $$\begin{tikzcd}
      X \arrow{d}{\id_X}\arrow{r}{f} & Y \arrow{d}{f^{-1}}\arrow{r} & Z \arrow[dashed]{d}{\exists}\arrow{r} & X[1]\arrow{d}{\id_{X[1]}}\\
      X \arrow{r}{\id_X} & X \arrow{r} & 0 \arrow{r} & X[1]
    \end{tikzcd}$$
    which is an isomorphism of triangles by the 5 Lemma.
    Hence $Z \cong 0$.
    
    Conversely, if $Z \cong 0$, then we have morphisms of distinguished triangles by (TR3)
    $$\begin{tikzcd}
      Z \arrow{d}\arrow{r} & X[1] \arrow{d}{\id_{X[1]}}\arrow{r}{f[1]} & Y[1]\arrow[dashed]{d}{\exists g} \arrow{r} & Z[1]\arrow{d}\\
      0 \arrow{r} & X[1] \arrow{r}{\id_{X[1]}} & X[1] \arrow{r} & 0
    \end{tikzcd}$$
    and the induced morphism is an isomorphism by the 5 Lemma.
    We also have a morphism of distinguished triangles by (TR3)
    $$\begin{tikzcd}
      Z \arrow{d}\arrow{r} & X[1] \arrow{d}{g^{-1}}\arrow{r}{f[1]} & Y[1]\arrow[dashed]{d}{\exists h} \arrow{r} & Z[1]\arrow{d}\\
      0 \arrow{r} & Y[1] \arrow{r}{\id_{Y[1]}} & Y[1] \arrow{r} & 0
    \end{tikzcd}$$
    We see that $g^{-1} = h \circ f[1]$ and thus
    $$\id_{Y[1]} = g^{-1} \circ g = h \circ (f[1] \circ g) = h$$
    yields $f[1]$ an isomorphism with inverse $g$.
    Moreover $f$ is also an isomorphism because $[1]$ is an equivalence of categories and hence reflects isomorphisms.
    Therefore $f$ is an isomorphism if and only if $Z \cong 0$.
  \end{proof}
\end{prop}

\begin{prop}\label{sumtriangles}
  Let $\T$ be a triangulated category admitting coproducts indexed by a set $I$.
  If 
  $$\left\{D_i : \begin{tikzcd}X_1 \arrow{r}{f_i} & Y_i \arrow{r}{g_i} & Z_i\arrow{r}{h_i} & X_i[1]\end{tikzcd}\right\}_{i\in I}$$ 
  is a collection of distinguished triangles, then the triangle 
  $$\bigoplus_{i \in I} D_i : \begin{tikzcd} \bigoplus_{i \in I} X_i \arrow{r}{\oplus f_i} & \bigoplus_{i \in I}Y_i \arrow{r}{\oplus g_i} & \bigoplus_{i \in I} Z_i \arrow{r}{\oplus h_i} & \bigoplus_{i \in I} X_i[1]\end{tikzcd}$$
  is distinguished.

  \begin{proof}
    We first note that because $[1]$ is an autoequivalence it is left adjoint to $[-1]$ and hence necessarily commutes with coproducts, so that there is a unique isomorphism $\left(\bigoplus_{i \in I}X_i\right)[1] \cong \bigoplus_{i \in I}X_i[1]$.
    By (TR1) and (TR3) we have for each $i$ a morphism of distinguished triangles
    $$\begin{tikzcd} 
      X_i \arrow{r}{f_i}\arrow{d} & Y_i\arrow{d}\arrow{r}{g_i} & Z_i \arrow[dashed]{d}{\exists \gamma_i}\arrow{r}{h_i} & X_i[1] \arrow{d}\\
      \bigoplus_{i \in I} X_i \arrow{r}{\oplus f_i} & \bigoplus_{i \in I}Y_i \arrow{r}{g} & Z \arrow{r}{h} & \bigoplus_{i \in I} X_i[1]
    \end{tikzcd}.$$
    For any object $Z^\prime$ of $\T$, applying the cohomological functor $h_{Z^\prime}$ to the triangle $D_i$ gives a long exact sequence
    $$h_{Z^\prime}(D_i) : 
    \begin{tikzcd} 
      \cdots \arrow{r} & h_{Z^\prime}(Y_i[1]) \arrow{r} & h_{Z^\prime}(X_i[1]) \arrow{r} & h_{Z^\prime}(Z_i) \arrow{r} & h_{Z^\prime}(Y_i) \arrow{r} & h_{Z^\prime}(X_i) \arrow{r} & \cdots
    \end{tikzcd},$$
    and similarly gives a long exact sequence
    $$\begin{tikzcd}
      \cdots \arrow{r} & h_{Z^\prime}\left(\bigoplus_{i \in I}Y_i[1]\right) \arrow{r} & h_{Z^\prime}\left(\bigoplus_{i \in I}X_i[1]\right) \arrow{r} & h_{Z^\prime}(Z) \arrow{r} & h_{Z^\prime}\left(\bigoplus_{i \in I}Y_i\right) \arrow{r} & h_{Z^\prime}\left(\bigoplus_{i \in I} X_i\right) \arrow{r} & \cdots
    \end{tikzcd}$$
    We note that cohomology commutes with direct products, so we obtain the commutative diagram
    $$\begin{tikzcd} 
      \prod_{i \in I}h_{Z^\prime}(Y_i[1]) \arrow{r}\arrow{d}{\alpha_1} & \prod_{i \in I}h_{Z^\prime}(X_i[1]) \arrow{r}\arrow{d}{\alpha_2} & \prod_{i \in I}h_{Z^\prime}(Z_i) \arrow{r}\arrow{d}{\alpha_3} & \prod_{i \in I}h_{Z^\prime}(Y_i) \arrow{r}\arrow{d}{\alpha_4} & \prod_{i \in I}h_{Z^\prime}(X_i)\arrow{d}{\alpha_5}\\
      h_{Z^\prime}\left(\bigoplus_{i \in I}Y_i[1]\right) \arrow{r} & h_{Z^\prime}\left(\bigoplus_{i \in I}X_i[1]\right) \arrow{r} & h_{Z^\prime}(Z) \arrow{r} & h_{Z^\prime}\left(\bigoplus_{i \in I}Y_i\right) \arrow{r} & h_{Z^\prime}\left(\bigoplus_{i \in I} X_i\right)
    \end{tikzcd}$$
    with exact rows and $\alpha_1, \alpha_2, \alpha_4, \alpha_5$ all isomorphisms.
    By the 5 Lemma we see that $\alpha_3$ is an isomorphism, and thus by the Yoneda Lemma
    $$h^Z(Z^\prime) = h_{Z^\prime}(Z) \cong \prod_{i \in I}h_{Z^\prime}(Z_i) \cong h_{Z^\prime}\left(\bigoplus_{i \in I}Z_i\right) = h^{\bigoplus_{i \in I} Z_i}(Z^\prime).$$
    implies there is an isomorphism $\gamma : \bigoplus_{i \in I} Z_i \rightarrow Z$ which yields an isomorphism of triangles
    $$\begin{tikzcd}
      \bigoplus_{i \in I} X_i \arrow{d}{\id}\arrow{r}{\oplus f_i} & \bigoplus_{i \in I}Y_i \arrow{d}{\id}\arrow{r}{\oplus g_i} & \bigoplus_{i \in I} Z_i \arrow{d}{\gamma}\arrow{r}{\oplus h_i} & \bigoplus_{i \in I} X_i[1]\arrow{d}{\id}\\
      \bigoplus_{i \in I} X_i \arrow{r}{\oplus f_i} & \bigoplus_{i \in I}Y_i \arrow{r}{g} & Z \arrow{r}{h} & \bigoplus_{i \in I} X_i[1]
    \end{tikzcd}.$$
    Therefore $\bigoplus_{i \in I}D_i$ is distinguished, as desired.
  \end{proof}
\end{prop}

\begin{cor}\label{corsumtriangles}
  Let $\T$ be a triangulated category and let $X,Y$ be objects of $\T$.
  The triangle
  $$\begin{tikzcd}
    X \arrow{r} & X \oplus Y \arrow{r} & Y \arrow{r}{0} & X[1]
  \end{tikzcd}$$
  is distinguished.
  
  \begin{proof}
    Realize 
    $$\begin{tikzcd}
      X \arrow{r} & X \oplus Y \arrow{r} & Y \arrow{r}{0} & X[1]
    \end{tikzcd}$$
    as the coproduct of the distinguished triangles
    $$\begin{tikzcd}
      X \arrow{r} & X \arrow{r} & 0\arrow{r} & X[1]
    \end{tikzcd}
    \text{and}
    \begin{tikzcd}
      0 \arrow{r} & Y \arrow{r} & Y \arrow{r} & 0.
    \end{tikzcd}$$
  \end{proof}
\end{cor}

\begin{cor}\label{splittriangle}
  Let $\T$ be a triangulated category.
  If 
  \begin{tikzcd}
    X \arrow{r} & Y \arrow{r} & Z \arrow{r}{0} & X[1]
  \end{tikzcd}
  is a distinguished triangle, then $Y \cong X \oplus Z$.
  
  \begin{proof}
    By taking rotates, we obtain a morphism of distinguished triangles
    $$\begin{tikzcd}
      Z[-1] \arrow{d}{\id_{Z[-1]}}\arrow{r}{0} & X \arrow{d}{\id_X}\arrow{r} & Y \arrow{r}\arrow[dashed]{d}{\exists} & Z\arrow{d}{\id_{Z}}\\
      Z[-1] \arrow{r}{0} & X \arrow{r} & X \oplus Z \arrow{r} & X[1]
    \end{tikzcd}$$
    and the induced morphism is an isomorphism by the 5 Lemma.
  \end{proof}
\end{cor}
\end{document}
