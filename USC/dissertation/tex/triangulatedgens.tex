\documentclass[dissertation.tex]{subfiles}
\begin{document}

\begin{defn}
  Let $\T$ be a triangulated category and let $E$ be an object of $\T$.
  \begin{itemize}
    \item
      Denote by $\langle E \rangle_1$ the strictly full subcategory of $\T$ with objects isomorphic to direct summands of finite coproducts
      $$\bigoplus_{i = 1}^r E[d_i].$$
    \item
      For $1 < n$, denote by $\langle E \rangle_n$ the full subcategory of $\T$ with objects isomorphic to direct summands of objects $X$ which fit into a distinguished triangle
      $$A \rightarrow X \rightarrow B \rightarrow A[1]$$
      with $A$ an object of $\langle E \rangle_1$ and $B$ an object of $\langle E \rangle_{n-1}$.
    \item
      Denote by $\langle E \rangle$ the subcategory of $\T$ with objects
      $$\bigcup_{n>0} \langle E \rangle_n.$$
  \end{itemize}
\end{defn}

\begin{lem}
  Let $\T$ be a triangulated category and let $E$ be an object of $\T$.
  The category $\langle E \rangle_n$ is a strictly full, additive subcategory of $\T$ preserved under shifts and under taking summands.
  
  \begin{proof}
    We proceed by induction on $n$.
    We first note that it's clear from the definition that $\langle E \rangle_n$ is strictly full, closed under taking summands, and preadditive since $\Hom{\langle E \rangle_n}{X,Y} = \Hom{\T}{X,Y}$.
    Thus it remains only to show that $\langle E \rangle_n$ is closed under sums, and closed under shifts.
    
    The case $n = 1$ is clear from the definition.
    Assume the result holds up to some $1 < n$, and let $E^\prime$ be an object of $\langle E \rangle_n$.
    Choose an object $X$ fitting in a distinguished triangle
    $$\begin{tikzcd}
      A \arrow{r}{f} & X \arrow{r}{g} & B \arrow{r}{h} & A[1]
    \end{tikzcd}$$
    such that $X \cong E \oplus E^{\prime\prime}$, $A$ is an object of $\langle E \rangle_1$, and $B$ is an object of $\langle E \rangle_{n-1}$.
    We note that the triangle
    $$\begin{tikzcd}
      A[d] \arrow{r}{f[d]} & X[d] \arrow{r}{g[d]} & Y[d] \arrow{rr}{(-1)^{d}h[d]} && A[d+1]
    \end{tikzcd}$$
    is distinguished, and so by induction $A[d]$ is an object of $\langle E \rangle_1$ and $B[d]$ is an object of $\langle E \rangle_{n-1}$.
    Since $X[d] \cong E^\prime[d] \oplus E^{\prime\prime}[d]$, it follows that $E^\prime[d]$ is an object of $\langle E \rangle_n$.
    
    Similarly, if $E^\prime, E^{\prime\prime}$ are objects of $\langle E \rangle_n$, then we have objects $X, X^\prime$ fitting into distinguished triangles
    $$\begin{tikzcd}
      A \arrow{r}{f} & X \arrow{r}{g} & B \arrow{r}{h} & A[1]
    \end{tikzcd}$$
    and
    $$\begin{tikzcd}
      A^\prime \arrow{r}{f^\prime} & X^\prime \arrow{r}{g^\prime} & B^\prime \arrow{r}{h} & A^\prime[1]
    \end{tikzcd}$$
    By the induction hypothesis and Proposition~\ref{sumtriangles} the triangle
    $$\begin{tikzcd}
      A \oplus A^\prime \arrow{r}{f \oplus f^\prime} & X \oplus X^\prime \arrow{r}{g \oplus g^\prime} & B \oplus B^\prime \arrow{r}{h \oplus h^\prime} & A[1] \oplus A^\prime[1]
    \end{tikzcd}$$
    is distinguished with $A \oplus A^\prime$ an object of $\langle E \rangle_1$ and $B \oplus B^\prime$ an object of $\langle E \rangle_{n-1}$.
    Since $E^\prime \oplus E^{\prime\prime}$ is a summand of $X \oplus X^\prime$, it follows that $E^\prime \oplus E^{\prime\prime}$ is an object of $\langle E \rangle_n$.
  \end{proof}
\end{lem}

\begin{lem}
  Let $\T$ be a triangulated category and let $E$ be an object of $\T$.
  If $A$ is an object of $\langle E \rangle_a$, $B$ is an object of $\langle E \rangle_b$, and 
  \begin{tikzcd} A \arrow{r}{f} & X \arrow{r}{g} & B \arrow{r}{h} & A[1]\end{tikzcd}
  is a distinguished triangle, then $X$ is an object of $\langle E \rangle_{a + b}$.

  \begin{proof}
    Given an object $E^\prime$ of $\langle E \rangle_n$, we have by Corollary~\ref{corsumtriangles} the distinguished triangle
    $$\begin{tikzcd}
      E \arrow{r} & E^\prime \oplus E \arrow{r} & E^\prime \arrow{r} & E[1]
    \end{tikzcd}$$
    which shows $E^\prime \oplus E$ is an object of $\langle E \rangle_{n+1}$.
    Since $\langle E \rangle_{n+1}$ is closed under taking summands, it follows that $E^\prime$ is an object of $\langle E \rangle_{n+1}$.
    Hence $\langle E \rangle_n \subseteq \langle E \rangle_{n+1}$ and so $A$, $B$ are both objects of $\langle E \rangle_{a + b}$.
    Therefore $X$ is an object of $\langle E \rangle_{a + b}$ by Proposition~\ref{fulltriangles}.
  \end{proof}
\end{lem}

\begin{prop}
  Let $\T$ be a triangulated category and let $E$ be an object of $\T$.
  The subcategory
  $$\langle E \rangle = \bigcup_{n > 0} \langle E \rangle_n$$
  is a strictly full, saturated, triangulated subcategory of $\T$ and it is the smallest such subcategory of $\T$ containing $E$.

  \begin{proof}
    By construction, $\langle E \rangle$ is a full subcategory.
    To see that it is strictly full, observe that given an object $E^\prime$ of $\langle E \rangle$, choose an $n$ such that $E^\prime$ is an object of $\langle E \rangle_n$.
    Given an isomorphism $E^\prime \rightarrow Y$ we have a distinguished triangle by Proposition~\ref{isotriangle}
    $$\begin{tikzcd}
      E^\prime \arrow{r} & Y \arrow{r} & 0 \arrow{r} & E^\prime[1]
    \end{tikzcd}$$
    which implies $Y$ is an object of $\langle E \rangle_n \subseteq \langle E \rangle$ by Proposition~\ref{fulltriangles}.
    
    To see that $\langle E \rangle$ is saturated, it suffices to observe that $\langle E \rangle_n$ is strictly full and closed under taking summands.
    Indeed, if $X \oplus Y$ is isomorphic to an object $E^\prime$ of $\langle E \rangle$, then $E^\prime$ lies in some $\langle E \rangle_n$.
    Hence $X$ and $Y$ are both objects of $\langle E \rangle_n \subseteq \langle E \rangle$.
    
    For the triangulated stucture, take the distinguished triangles of $\langle E \rangle$ to be the distinguished triangles 
    $$\begin{tikzcd}
      X \arrow{r} & Y \arrow{r} & Z \arrow{r} & X[1]
    \end{tikzcd}$$
    of $\T$ with $X$, $Y$, and $Z$ objects of $\langle E \rangle$.
    We observe that (TR1) and (TR4) are satisfied by virtue of $\langle E \rangle$ being strictly full and Proposition~\ref{fulltriangles}.
    The axiom (TR2) is inherited from $\T$ and (TR3) is satisfied because $\langle E \rangle$ is full.
    The inclusion functor preserves the distinguished triangles by definition.

    Suppose that $T^\prime$ is any other strictly full, saturated, triangulated subcategory of $\T$ containing $E$.
    We show by induction that $\langle E \rangle_n$ is a subcategory of $\T^\prime$.
    First note that by definition, $\langle E \rangle_1$ is necessarily a subcategory of $\T^\prime$.
    Suppose that $\langle E \rangle_i$ is a subcategory of $\T^\prime$ for all $i < n$.
    Given an object $E^\prime$ of $\langle E \rangle_n$, there exist objects $A$ of $\langle E \rangle_1$, $B$ of $\langle E \rangle_{n-1}$, and $X \cong E^\prime \oplus E^{\prime\prime}$ and a distinguished triangle
    $$\begin{tikzcd}
      A \arrow{r} & X \arrow{r} & B \arrow{r} & A[1].
    \end{tikzcd}$$
    Since $\T^\prime$ is strictly full and saturated, it follows by Proposition~\ref{fulltriangles} that $X$ is an object of $\T^\prime$ and thus $E^\prime$ is also an object of $\T^\prime$.
    Therefore $\langle E \rangle$ is a subcategory of $\T^\prime$.
  \end{proof}
\end{prop}

\begin{defn}
  Let $\T$ be a triangulated category and let $E$ be an object of $\T$.
  \begin{enumerate}
  \item
    We say $E$ is a {\it classical generator} of $\T$ if $\T = \langle E \rangle$.
  \item
    We say $E$ is a {\it strong generator} of $\T$ if $\T = \langle E \rangle_n$ for some $1 \leq n$.
  \item
    We say $E$ is a {\it (weak) generator} of $\T$ if for every non-zero object $X$ of $\T$, there eixsts an $n$ such that $\Hom{\T}{E,X[n]} \neq 0$.
  \end{enumerate}
\end{defn}

\begin{lem}\label{classicalgenhom}
  Let $\T$ be a triangulated category.
  If $E$ and $X$ are objects of $\T$, then $\Hom{\T}{E,X[n]} = 0$ for all $n$ if and only if $\Hom{\T}{E^\prime, X} = 0$ for all objects $E^\prime$ of $\langle E \rangle$.
  
  \begin{proof}
    The converse is immediate.
    Assume that $\Hom{\T}{E,X[n]} = 0$ for all $n$.
    First we observe that because $[n]$ is an autoequivalence, we have a natural isomorphism
    $$0 = \Hom{\T}{E[n], X} \cong \Hom{\T}{E, X[-n]}.$$
    
    We proceed by induction.
    Given an object $E^\prime$ of $\langle E \rangle_1$, we may choose by the definition of $\langle E \rangle_1$ an object  $E^{\prime\prime}$ such that $E^\prime \oplus E^{\prime\prime} \cong \bigoplus_{i = 1}^r E[d_i]$.
    Hence
    $$\Hom{\T}{E^\prime,X} \oplus \Hom{\T}{E^{\prime\prime}, X} \cong \Hom{\T}{\bigoplus_{i=1}^rE[d_i], X} = 0$$
    implies that $\Hom{\T}{E^\prime, X} = 0$.
    
    Assume the result holds up to some $1 < n$.
    Given an object $E^\prime$ of $\langle E \rangle_n$, choose an object $E^{\prime\prime}$ and a distinguished triangle
    $$\begin{tikzcd}
      A \arrow{r} & E^\prime \oplus E^{\prime\prime} \arrow{r} & B \arrow{r} & A[1]
    \end{tikzcd}$$
    with $A$ an object of $\langle E \rangle_1$ and $B$ an object of $\langle E \rangle_{n-1}$.
    By induction, $\Hom{\T}{A,X} = 0 = \Hom{\T}{B,X}$, and because $h_X$ is a cohomological functor we get an exact sequence
    $$\begin{tikzcd}
      0 \arrow{r} & \Hom{\T}{E^\prime,X} \oplus \Hom{\T}{E^{\prime\prime}, X} \arrow{r} & 0
    \end{tikzcd}$$
    which implies $\Hom{\T}{E^\prime,X} = 0$.
  \end{proof}
\end{lem}

\begin{lem}\label{classicalgenisgen}
  Let $\T$ be a triangulated category and let $E$ be an object of $\T$.
  If $E$ is a classical generator, then $E$ is a generator.
  
  \begin{proof}
    Given an object $X$ of $\T$ such that $\Hom{\T}{E,X[n]} = 0$ for all $n$, we observe that because $\T = \langle E \rangle$, Lemma~\ref{classicalgenhom} implies $\Hom{\T}{X,X} = 0$.
    Therefore $\id_{X} = 0$ implies $X = 0$ and $E$ is a generator.
  \end{proof}
\end{lem}

\subsection{Compactly Generated Triangulated Categories}
\begin{defn}
  Let $\mathscr{C}$ be an additive category with arbitrary coproducts.
  A {\it compact object} of $\mathscr{C}$ is an object $K$ such that the canonical morphism
  $$\bigoplus_{i \in I} \Hom{\mathscr{C}}{K, E_i} \rightarrow \Hom{\mathscr{C}}{K, \bigoplus_{i \in I}E_i}$$
  is an isomorphism for any indexing set $I$ and any collection $\{E_i\}_I$ of objects of $\mathscr{C}$.
\end{defn}

\begin{lem}\label{compactobjstriangulated}
  Let $\T$ be a triangulated category with coproducts.
  The compact objects of $\T$ form a saturated, strictly full, triangulated subcategory, $\T_c$, of $\T$.
\end{lem}

%May need some hypothesis on $\T$ being Ab3-enriched?
\begin{lem}\label{homhocolimiscolim}
  Let $\T$ be a triangulated category with coproducts.
  Let $K$ be an object of $\T_c$.
  Given a directed system of $\T$
  $$\begin{tikzcd}
    \cdots \arrow{r}{f_{n+1}} & X_n \arrow{r}{f_n} & X_{n-1} \arrow{r}{f_{n-1}} & \cdots
  \end{tikzcd}$$
  for which $X = \operatorname{hocolim}X_n$ exists, the canonical morphism
  $$\begin{tikzcd}
    \operatorname{colim} \Hom{\T}{K,X_n} \arrow{r} & \Hom{\T}{K,X}
  \end{tikzcd}$$
  is an isomorphism.
\end{lem}

\begin{lem}\label{hocolim}
  Let $\T$ be a triangulated category with coproducts.
  If $\left\{E_i\right\}_{i \in I}$ is a family of compact objects of $\T$ such that $\bigoplus_{i \in I}{E_i}$ generates $\T$, then every object $X$ of $\T$ can be written as 
  $$X = \operatorname{hocolim}X_n,$$
  where $X_1$ is a direct sum of shifts of the $E_i$ and each transition morphism fits into a distinguished triangle $Y_n \rightarrow X_n \rightarrow X_{n+1} \rightarrow Y_n[1]$, where $Y_n$ is a direct sum of shifts of the $E_i$.
\end{lem}

\begin{lem}\label{hocolimfactorization}
  With the same assumptions and notation as in the Lemma above, if $K$ is a compact object and $K \rightarrow X_n$ is a morphism, then there is a factorization 
  $$K \rightarrow E \rightarrow X_n,$$
  where $E$ is an object of $\langle \bigoplus_{J}E_j \rangle$ for some finite $J \subseteq I$.
  \begin{proof}
    We proceed by induction.
    When $n = 1$, $X_1 = \bigoplus_S E_i[m]$ over the triples
    $$S = \left\{(i,m,\varphi) \;\middle\vert\; i \in I, m \in \Z, \varphi : E_i[m] \rightarrow X\right\}.$$
    Because $K$ is compact, we have
    $$\Hom{\T}{K,X_1} \cong \bigoplus_S\Hom{\T}{K, E_i[m]}$$
    and thus a morphism $K \rightarrow X_1$ maps under this ismorphism to a finite collection of morphism, $$\left\{K \rightarrow E_i[m]\right\}_{(i,m,\varphi) \in S^\prime}.$$
    Hence $K \rightarrow X_1$ factors as
    $$K \rightarrow \bigoplus_{(i,m,\varphi) \in S^\prime}E_i[m] \rightarrow X_1.$$

    Assume the result holds up to some $1 < n$.
    By assumption, we have a distinguished triangle
    $$Y_{n-1} \rightarrow X_{n-1} \rightarrow X_{n} \rightarrow Y_{n-1}[1]$$
    and, since $Y_{n-1}$ is of the same form as $X_1$, we get a factorization
    $$\begin{tikzcd}
      K \arrow{rd}\arrow{r} & X_n \arrow{r} & Y_{n-1}[1]\\
      & E^\prime[1]\arrow{ur}
    \end{tikzcd}$$
    for some $E^\prime \in \langle \bigoplus_{i \in I^\prime}E_i \rangle$ with $I^\prime \subset I$ finite.
    By the axioms for a triangulated category, we have a morphism of distinguished triangles
    $$\begin{tikzcd}
      K[-1] \arrow{d}\arrow{r} & E^\prime \arrow{r}\arrow{d} & K^\prime \arrow{r}\arrow[dashed]{d}{\exists} & K\arrow{d}\\
      X_n[-1] \arrow{r} & Y_{n-1} \arrow{r} & X_{n-1} \arrow{r} & X_n
    \end{tikzcd}$$
    and we note that $K^\prime$ is compact because $K$ and $E^\prime$ are, and the triangulated subcategory of compact objects is strictly full.
    By induction, the induced morphism factors as
    $$\begin{tikzcd}
      K^\prime \arrow{rr}\arrow{rd} & & X_{n-1}\\
      & E^{\prime\prime}\arrow{ur}
    \end{tikzcd}$$
    for some object $E^{\prime\prime} \in \langle \bigoplus_{i \in I^{\prime\prime}} E_i \rangle$ with finite $I^{\prime\prime} \subseteq I$.
    Taking the composition $E^\prime \rightarrow K^\prime \rightarrow E^{\prime\prime}$ we get a distinguished triangle $E^\prime \rightarrow E^{\prime\prime} \rightarrow E \rightarrow E^\prime[1]$ for some object $E \in \langle \bigoplus_{i \in I^\prime \cup I^{\prime\prime}}E_i\rangle$ and morphisms of distinguished triangles
    $$\begin{tikzcd}
      E^\prime \arrow{r}\arrow{d} & K^\prime \arrow{r}\arrow{d} & K \arrow{r}\arrow[dashed]{d}{\exists} & E^\prime[1]\arrow{d}\\
      E^\prime \arrow{r}\arrow{d} & E^{\prime\prime} \arrow{r}\arrow{d} & E \arrow{r}\arrow[dashed]{d}{\exists} & E^\prime[1]\arrow{d}\\
      Y_{n-1} \arrow{r} & X_{n-1} \arrow{r} & X_n \arrow{r} & Y_{n-1}[1]
    \end{tikzcd}$$
    
    The composition of the induced morphisms gives a morphism $K \rightarrow X_n$, which need not be the original morphism, however the compositions with $X_n \rightarrow Y_{n-1}[1]$ agree by construction.
    We observe that because $h^{K}$ is a homological functor, the distinguished triangle $Y_{n-1} \rightarrow X_{n-1} \rightarrow X_{n} \rightarrow Y_{n-1}[1]$ gives rise to an exact sequence
    $$\begin{tikzcd}
      \Hom{\T}{K,X_{n-1}} \arrow{r} & \Hom{\T}{K,X_n} \arrow{r} & \Hom{\T}{K,Y_{n-1}[1]}
    \end{tikzcd}$$
    and, by the observation above, the difference of these two morphisms can be expressed as a morphism $K \rightarrow X_{n-1} \rightarrow X_n$.
    By induction, we have a factorization 
    $$\begin{tikzcd}
      K \arrow{rd}\arrow{rr} && X_{n-1}\\
      & E^{\prime\prime\prime}\arrow{ur}
    \end{tikzcd}$$
    through an object of $\langle \bigoplus_{i \in I^{\prime\prime\prime}} E_i\rangle$ with $I^{\prime\prime\prime} \subseteq I$ finite.
    Then the composition of the induced morphisms
    $$\begin{tikzcd}
      & K\arrow[dashed]{d}{\exists !}\arrow{ld}\arrow{rd}\\
      E\arrow{rd} & E \oplus E^{\prime\prime\prime} \arrow[dashed]{d}{\exists!}\arrow{r}\arrow{l} & E^{\prime\prime\prime}\arrow{d}\arrow{ld}\\
      & X_n & \arrow{l}X_{n-1}
    \end{tikzcd}$$
    gives the desired morphism with $E \oplus E^{\prime\prime\prime} \in \langle \bigoplus_{i \in I^\prime \cup I^{\prime\prime} \cup I^{\prime\prime\prime}} E_i\rangle.$
  \end{proof}
\end{lem}

\begin{defn}
  Let $\T$ be a triangulated category with coproducts.
  We say $\T$ is compactly generated if there exists a family $\{E_i\}_{i \in I}$ of compact objects such that $\bigoplus_{i \in I} E_i$ generates $\T$.
\end{defn}

\begin{prop}\label{GeneratorIFFClassicAndCompactlyGenerated}
  Let $\T$ be a triangulated category with coproducts.
  If $E$ is a compact object of $\T$, then
  $E$ is a classical generator for $\T_c$ ($\T_c = \langle E \rangle$) and $\T$ is compactly generated 
  if and only if
  $E$ is a generator for $\T$.
  
  \begin{proof}
    First assume that $E$ is a classical generator for $\T_c$, and $\{E_i\}_{i \in I} \subseteq \T_c$ generates $\T$.
    By Lemma~\ref{classicalgenisgen}, $E$ is a generator for $\T_c$, and $\bigoplus_{i \in I}E_i \in \langle E \rangle = \T_c$.
    Hence, given an object $X$ of $\T$ such that $\Hom{\T}{E,X[n]} = 0$ for all $n$, it follows from Lemma~\ref{classicalgenhom} that
    $$\Hom{\T}{\bigoplus_{i \in I}E_i, X} = 0$$
    and thus $X = 0$.
    Therefore $E$ is a generator for $\T$.
    
    Conversely, assume that $E$ is a generator for $\T$.
    By definition, $\T$ is compactly generated, so it remains to show that $\T_c = \langle E \rangle$.
    Towards that end, let $X$ be an object of $\T_c$.
    By applying Lemma~\ref{hocolim} with $I = \{1\}$ and $E_1 = E$, using the same notation, we can write $X = \operatorname{hocolim}X_n$.
    By Lemma~\ref{homhocolimiscolim}, there exists a factorization
    $$\begin{tikzcd}
      X \arrow{rd}\arrow{rr}{\sim} & & \operatorname{hocolim} X_n\\
      & X_n\arrow{ur}
    \end{tikzcd}$$
    for some $n$.
    Using Proposition~\ref{isotriangle}, (TR1), and (TR4), we obtain a commutative diagram 
    $$\begin{tikzcd}
      X \arrow{r}\arrow{d} & X_n \arrow{r}\arrow{d} & A \arrow{d}\arrow{r} & X[1]\arrow{d}\\
      X \arrow{d}\arrow{r} & \operatorname{hocolim}X_n \arrow{d}\arrow{r} & 0 \arrow{d}\arrow{r} & X[1]\arrow{d}\\
      X_n \arrow{d}\arrow{r} & \operatorname{hocolim}X_n \arrow{d}\arrow{r} & B \arrow{r}\arrow{d} & X_n[1]\arrow{d}\\
      A \arrow{r} & 0 \arrow{r} & B \arrow{r} & A[1]
    \end{tikzcd}$$
    with rows distinguished triangles, showing that $A \rightarrow X[1]$ is the zero morphism.
    
    Now, by Lemma~\ref{hocolimfactorization}, we obtain a factorization
    $$\begin{tikzcd}
      X \arrow{rr}\arrow{rd} & & X_n\\
      & E^\prime \arrow{ur}
    \end{tikzcd}$$
    with $E^\prime$ an object of $\langle E \rangle$.
    Applying (TR1) and (TR4) yields a commutative diagram
    $$\begin{tikzcd}
      X \arrow{d}\arrow{r} & E^\prime \arrow{r}\arrow{d} & A^\prime \arrow{r}\arrow{d} & X[1]\arrow{d}\\
      X \arrow{r}\arrow{d} & X_n \arrow{r}\arrow{d} & A \arrow{r}{0}\arrow{d} & X[1]\arrow{d}\\
      E^\prime \arrow{r}\arrow{d} & X_n \arrow{r}\arrow{d} & B^\prime \arrow{r}\arrow{d} & E^\prime[1]\arrow{d}\\
      A^\prime \arrow{r} & A \arrow{r} & B^\prime \arrow{r} & A^\prime[1]
    \end{tikzcd}$$
    with rows distinguished triangles, which implies the morphism $A^\prime \rightarrow X[1]$ is the zero morphism.
    Therefore $E^\prime \cong X \oplus A^\prime$ by Corollary~\ref{splittriangle}, and $X$ is an object of $\langle E \rangle$, as desired.
  \end{proof}
\end{prop}

\begin{defn}
  We say that a triangulated functor $\F$ is {\it continuous} if $\F$ commutes with coproducts.
\end{defn}

\begin{prop}[{\cite[Tag 0A8E]{stacks-project}}]\label{brownrep}
  Let $\T$ be a triangulated category with coproducts which is compactly generated.
  If $\F \colon \T \to \T^\prime$ is a continuous triangulated functor, then $\F$ has a triangulated right adjoint.
\end{prop}

%TODO: Should probably include a proof at some point.
\begin{prop}\label{naturalisocontinuousfunctors}
  Let $\T, \T^\prime$ be triangulated categories with coproducts and assume that $\T$ is compactly generated by $\{E_i\}_I$.
  If $\F, \G \colon \T \to \T^\prime$ are continuous triangulated functors and $\eta \colon \F \to \G$ is a natural transformation such that for each $i \in I$ 
  $$\eta(E_i) \colon \F(E_i) \overset{\sim}\to \G(E_i)$$
  is an isomorphism, then $\eta$ is a natural isomorphism.
\end{prop}

\begin{cor}
  Let $\T, \T^\prime$ be triangulated categories with coproducts.
  Assume that $\T$ and $\T^\prime$ are compactly generated by $\{E_i\}_I$ and $\{F_j\}_J$, respectively, and let $E = \oplus_{i \in I}$, $F = \oplus_{j \in J} F_j$.
  If $\F \colon \T \to \T^\prime$ is a continuous triangulated functor satisfying
  \begin{enumerate}
  \item
    there exists an isomorphism $\alpha \in \T^\prime(\F(E), F)$, and
  \item
    the morphism
    $$\F(E,E) \colon \T(A,A) \to \T^\prime(\F E, \F E)$$
    is an isomorphism,
  \end{enumerate}
  then $\F$ is an equivalence of categories.

  \begin{proof}
    First note that it suffices to assume that the collections $\{E_i\}_I$ and $\{F_j\}_J$ are closed under shifts, and hence $E$ and $F$ are fixed by shifts.
    By Proposition~\ref{brownrep} there exists a right adjoint, $\G \colon \T^\prime \to \T$ and hence we have natural transformations
    $$\varepsilon \colon \id_{\T} \to \G \circ \F \ \text{and}\ \eta \colon \F \circ \G \to \id_{\T^\prime}.$$
    Since $\F$ being an equivalence of categories is equivalent to both $\varepsilon$ and $\eta$ being isomorphisms, it suffices by Proposition~\ref{naturalisocontinuousfunctors} to show that $\varepsilon_E$ and $\eta_F$ are isomorphisms.

    First we show $\varepsilon_E$ is an isomorphism.
    By (TR1) we can choose a distinguished triangle of $\T$
    $$\begin{tikzcd}
      E \arrow{rr}{\G(\alpha) \circ \varepsilon_E} && GF \arrow{r}{u} & C \arrow{r}{v} & E
    \end{tikzcd}$$
    and we have a distinguished triangle
    $$\begin{tikzcd}
      \F E \arrow{r}{\id_{\F E}} & F \arrow{r} & 0 \arrow{r} & \F A
    \end{tikzcd}$$
    Applying the homological functors $\T(E, -)$ and $\T^\prime(\F A, - )$ to the we obtain a diagram
    $$\begin{tikzcd}
      \T(E,E) \arrow{rr}{h^{\E}(\G(\alpha) \circ \varepsilon_E)}\arrow{dd}{\F(E,E)} && \T(E,\G F) \arrow{r}{h^E(u)}\arrow{dd}[rotate=90,yshift=-.5ex]{\sim} & \T(E,C) \arrow{r}{h^E(v)}\arrow{dd} & \T(E,E) \arrow{rr}{h^{\E}(\G(\alpha) \circ \varepsilon_E)}\arrow{dd}{\F(E,E)} && \T(E, \G F)\arrow{dd}[rotate=90,yshift=-.5ex]{\sim}\\
      \\
      \T^\prime(\F E, \F E) \arrow{rr}{h^{\F E}(\alpha)} && \T^\prime(\F E, F) \arrow{r} & 0 \arrow{r} & \T^\prime(\F E, \F E) \arrow{rr}{h^{\F E}(\alpha)} && \T^\prime(\F E, F)
    \end{tikzcd}$$
    with exact rows and unlabeled isomorphisms the isomorphism of adjunction.
    We note that it suffices to show that this diagram commutes, for then the 5-Lemma implies $\T(E,C) \cong 0$ and, since $E$ generates $\T$, it follows that $C \cong 0$ and thus $\G(\alpha) \circ \varepsilon_E$ is an isomorphism.
    But $\alpha$ is an isomorphism, so it follows that $\varepsilon_E$ is also an isomorphism.

    For the outermost squares, we first note that the image of $\G(\alpha) \circ \varepsilon_E$ under the isomorphism of adjunction is
    $$\eta_F \circ \F\G(\alpha) \circ \F(\varepsilon_E) = \alpha \circ \eta_{\F E} \circ \F(\varepsilon_E) = \alpha$$
    and commutativity follows.

    For the remaining square, it suffices to show that $\F(v) = 0$, for then
    $$\F(E,E) \circ h^E(v) = F(v \circ -) = h^{\F E}(\F(v)) = 0.$$
    This follows from the following observation: because $\F$ is triangulated, we obtain a morphism of triangles in $\T^\prime$ by (TR3)
    $$\begin{tikzcd}
      \F E \arrow{rr}{\F\G(\alpha) \circ \F(\varepsilon)} \arrow{dd}{\id_{\F E}} &&
      \F\G F \arrow{dd} \arrow{r}{\F(u)}\arrow{dd}{\alpha^{-1} \circ \eta_{F}} &
      \F C \arrow{r}{\F(v)} \arrow[dashed]{dd} &
      \F E \arrow{dd}{\id_{\F E}}\\\\
      \F E \arrow{rr}{\id_{\F E}} && \F E \arrow{r} & 0 \arrow{r} & \F E
    \end{tikzcd}$$
    and so $\F(v) = 0$.
    This establishes that $\F$ is fully faithful.

    For essentially surjective, we show that $\eta_F$ is also an isomorphism.
  \end{proof}
\end{cor}
\end{document}
