\documentclass[dissertation.tex]{subfiles}
\begin{document}

\subsection{Basic Properties of Triangulated Categories}
\begin{defn}
  Let $\T$ be an additive category equipped with autoequivalences of categories, $[n] : \CC \rightarrow \CC$ for $n \in \Z$, and a class of distinguished triangles.
  We say $\T$ is {\it triangulated} if the distinguished triangles satisfy the following axioms:
  \begin{description}[style=nextline]
    \item[TR1]\label{TR1}
      \begin{itemize}
      \item
        Every morphism $X \rightarrow Y$ can be embedded in a distinguished triangle $X \rightarrow Y \rightarrow Z \rightarrow X[1]$.
      \item
        The triangle \begin{tikzcd} X \arrow{r}{\id_X} & X \arrow{r} & 0 \arrow{r} & X[1]\end{tikzcd} is distinguished.
      \item
        Any triangle isomorphic to a distinguished triangle is also distinguished.
      \end{itemize}
    \item[TR2]\label{TR2}
      If 
      \begin{tikzcd}
        X \arrow{r}{f} & Y \arrow{r}{g} & Z \arrow{r}{h} & X[1]
      \end{tikzcd} 
      is a distinguished triangle, then the two rotated triangles
      $$\begin{tikzcd}
        Y \arrow{r}{g} & Z \arrow{r}{h} & X[1] \arrow{r}{-f[1]} & Y[1]
      \end{tikzcd}$$
      and
      $$\begin{tikzcd}
        Z[-1] \arrow{r}{-h[1]} & X \arrow{r}{f} & Y \arrow{r}{g} & Z
      \end{tikzcd}$$
      are also distinguished.
    \item[TR3]\label{TR3}
      Given distinguished triangles
      $$\begin{tikzcd}
        X_i \arrow{r}{f_i} & Y_i\arrow{r}{g_i} & Z_i \arrow{r}{h_i} & Z_i
      \end{tikzcd}$$
      for $i = 1, 2$ and a diagram
      $$\begin{tikzcd}
        X_1 \arrow{d}{\alpha}\arrow{r}{f_1} & Y_1 \arrow{d}{\beta}\arrow{r}{g_1} & Z_1 \arrow[dashed]{d}{\exists \gamma}\arrow{r}{h_1} & X_1[1]\arrow{d}{\alpha[1]}\\
        X_2 {\alpha}\arrow{r}{f_2} & Y_2 \arrow{r}{g_2} & Z_2 \arrow{r}{h_2} & X_2[1]
      \end{tikzcd}$$
      such that the left-hand square commutes, the dashed arrow can be filled in with a morphism $\gamma$, which is not necessarily unique.
      
    \item[TR4]\label{TR4}
      Given three distinguished triangles
      $$\begin{tikzcd}
        X \arrow{r}{f} & Y \arrow{r}{i} & X^\prime \arrow{r}{\ell} & X[1]\\
        X \arrow{r}{g \circ f} & Z \arrow{r}{j} & Y^\prime \arrow{r}{m} & X[1]\\
        Y \arrow{r}{g} & Z \arrow{r}{k} & Z^\prime \arrow{r}{n} & Y[1]
      \end{tikzcd}$$
      then there exists a distinguished triangle
      $$\begin{tikzcd}
        X^\prime \arrow{r}{u} & Y^\prime \arrow{r}{v} & Z^\prime \arrow{r}{i[1]\circ n} & X^\prime[1]
      \end{tikzcd}$$
      making the diagram
      $$\begin{tikzcd}
        X\arrow{d}{\id_X}\arrow{r}{f} & Y\arrow{r}{i}\arrow{d}{g} & X^\prime\arrow{r}{\ell}\arrow{d}{u} & X[1]\arrow{d}{id_{X[1]}}\\
        X \arrow{d}{f}\arrow{r}{g \circ f} & Z\arrow{r}{j}\arrow{d}{\id_Z} & Y^\prime\arrow{r}{m}\arrow{d}{v} & X[1]\arrow{d}{f[1]}\\
        Y\arrow{r}{g}\arrow{d}{i} & Z \arrow{r}{k}\arrow{d}{j} & Z^\prime \arrow{r}{n}\arrow{d}{\id_{Z^\prime}} & Y[1]\arrow{d}{i[1]}\\
        X^\prime \arrow{r}{u} & Y^\prime \arrow{r}{v} & Z^\prime \arrow{r}{i[1] \circ n} & X^\prime[1]
      \end{tikzcd}$$
  \end{description}
\end{defn}
commute.

\begin{defn}
  Let $\T$ be a triangulated category.
  \begin{itemize}
  \item
    If $T^\prime$ is any other triangulated category, a {\it triangulated functor}, $\F \colon \T^\prime \rightarrow \T$, is an additive functor that preserves distinguished triangles.

    Formally, $\F$ is equipped with natural isomorphisms
    $$\F(X[1]) \cong \F(X)[1]$$
    such that for any distinguished triangle 
    $$\begin{tikzcd}
      X \arrow{r}{u} & Y \arrow{r}{v} & Z \arrow{r}{w} & X[1]
    \end{tikzcd}$$
    of $T^\prime$, the triangle
    $$\begin{tikzcd}
      \F(X) \arrow{r}{\F(u)} & \F(Y) \arrow{r}{\F(v)} & \F(Z) \arrow{r}{\F(w)} & F(X)[1]
    \end{tikzcd}$$
    is a distinguished triangle of $\T$.
  \item
    We say a subcategory $\T^\prime$ of $\T$ is a {\it triangulated subcategory} if $\T^\prime$ is triangulated and the inclusion functor, $\T^\prime \rightarrow \T$, is triangulated.
  \item
    We say a full triangulated subcategory $\T^\prime$ of $\T$ is {\it saturated} if whenever $X \oplus Y$ is isomorphic to an object of $T^\prime$, both $X$ and $Y$ are isomorphic to objects of $T^\prime$.
  \end{itemize}
\end{defn}

\begin{prop}
  Let $\T, \T^\prime$ be triangulated categories and $\F \colon \T^\prime \rightarrow \T$ a triangulated functor.
  Given any object $Z$ of $\T^\prime$, $\F(Z) \cong 0$ if and only if for any distinguished triangle
  $$\begin{tikzcd}
    X \arrow{r}{f} & Y \arrow{r}{g} & Z \arrow{r}{h} & X[1]
  \end{tikzcd}$$
  there is an isomorphism of distinguished triangles of $\T$
  $$\begin{tikzcd}
    \F(X) \arrow{r}{\F(f)}\arrow{d}{\id_{\F(X)}} & \F(Y) \arrow{r}{\F(g)}\arrow{d} & \F(Z) \arrow{r}{\F(h)}\arrow{d} & \F(X)[1]\arrow{d}{\id_{\F(X)}}\\
      \F(X) \arrow{r}{\id_{F(X)}} & \F(X) \arrow{r} & 0 \arrow{r} & \F(X)[1]
  \end{tikzcd}$$
  
  \begin{proof}
    The converse is obvious from the definition of isomorphism of triangles.
    Assume that $\F(Z) \cong 0$ and note that this immediately implies $\F(h) = 0$.    
    By rotating the diagram and applying (TR3) we get the morphism of distinguished triangles
    $$\begin{tikzcd}
      \F(Z)[-1] \arrow{r}{-\F(h)}\arrow{d} & \F(X) \arrow{r}{\F(f)}\arrow{d}{\id_{\F(X)}} & \F(Y) \arrow{r}{\F(g)}\arrow[dashed]{d}{\exists} & \F(Z)\arrow{d}\\
      0 \arrow{r} & \F(X) \arrow{r}{\id_{\F(X)}} & \F(X) \arrow{r} & 0
    \end{tikzcd}$$
    and this is an isomorphism by the 5 Lemma.
  \end{proof}
\end{prop}

\begin{prop}\label{fulltriangles}
  Let $\T$ be a triangulated category, let $\T^\prime$ be a full triangulated subcategory, and let 
  $$D\colon 
  \begin{tikzcd}
    X \arrow{r}{f} & Y \arrow{r}{g} & Z \arrow{r}{h} & X[1] 
  \end{tikzcd}$$
  be a distinguished triangle of $\T$.
  \begin{enumerate}
  \item
    If $D$ is distinguished in $\T$ and two out of three of $X,Y,Z$ are objects of $\T^\prime$, then the third is isomorphic to an object of $T^\prime$.
    Moreover, if $T^\prime$ is strictly full, then all three are objects of $T^\prime$.
  \item
    If $D$ is a triangle in $T^\prime$, then it is distinguished in $\T^\prime$.
  \end{enumerate}

  \begin{proof}
    For the first assertian, note that because we may rotate, it suffices to assume that $X$ and $Y$ are objects of $T^\prime$.
    We observe that by (TR1) we obtain a distinguished triangle of $T^\prime$ 
    $$D^\prime : 
    \begin{tikzcd}
      X \arrow{r}{f} & Y\arrow{r}{u} & Z^\prime \arrow{r}{v} & X[1]
    \end{tikzcd},$$
    which is also distinguished in $\T$.
    By (TR3) we have a morphism of distinguished triangles of $\T$ to $D$
    $$\begin{tikzcd}
      D^\prime: & X \arrow{d}{\id_X}\arrow{r}{f} & Y \arrow{d}{\id_Y}\arrow{r}{u} & Z^\prime \arrow[dashed]{d}{\exists \gamma}\arrow{r}{v} & X[1]\arrow{d}{\id_{X[1]}}\\
      D: & X \arrow{r}{f} & Y \arrow{r}{g} & Z \arrow{r}{h} & X[1]
    \end{tikzcd}$$
    which is an isomorphism by the 5 Lemma.
    This proves the first assertian.
    
    For the second, we note that when $Z$ is an object of $\T^\prime$, we have $\Hom{\T^\prime}{Z,Z^\prime} = \Hom{\T}{Z,Z^\prime}$, so that the isomorphism above is an isomorphism of triangles in $T^\prime$.
  \end{proof}
\end{prop}

\begin{prop}
  Let $\T$ be a triangulated category.
  If 
  \begin{tikzcd} X \arrow{r}{f} & Y \arrow{r}{g} & Z \arrow{r}{h} & X[1]\end{tikzcd} 
  is a distinguished triangle, then so is
  $$\begin{tikzcd}X[n] \arrow{r}{f[n]} & Y[n] \arrow{r}{g[n]} & Z[n] \arrow{r}{h[n]} & X[n+1]\end{tikzcd}.$$

  \begin{proof}
    We note that it suffices to show that the result holds for $n = 1$ since $[n] = [n - 1] \circ [1]$.
    We have by (TR2) the successive rotations
    $$\begin{tikzcd}
      Y \arrow{r}{g} & Z\arrow{r}{h} & X[1]\arrow{r}{-f[1]} & Y[1]\\
      Z \arrow{r}{h}& X[1]\arrow{r}{-f[1]} & Y[1]\arrow{r}{-g[1]} & Z[1]\\
      X[1] \arrow{r}{-f[1]} & Y[1] \arrow{r}{-g[1]} & Z[1]\arrow{r}{-h[1]} & X[2],
    \end{tikzcd}$$
    all of which are distinguished triangles.
    We then have an isomorphism of triangles
    $$\begin{tikzcd}
      X[1] \arrow{d}{\id_{X[1]}}\arrow{r}{f[1]} & Y[1] \arrow{d}{-\id_{Y[1]}}\arrow{r}{g[1]} & Z[1]\arrow{d}{\id_{Z[1]}}\arrow{r}{h[1]} & X[2]\arrow{d}{-\id_{X[2]}}\\
      X[1] \arrow{r}{-f[1]} & Y[1] \arrow{r}{-g[1]} & Z[1]\arrow{r}{-h[1]} & X[2].
    \end{tikzcd}$$
    Therefore 
    $$\begin{tikzcd}
      X[1] \arrow{r}{f[1]} & Y[1] \arrow{r}{g[1]} & Z[1]\arrow{r}{h[1]} & X[2]
    \end{tikzcd}$$
    is distinguished, as desired.
  \end{proof}
\end{prop}

\begin{defn}
  Let $\T$ be a triangulated category and let $\A$ be an abelian category.
  An additive functor $H : \T \rightarrow \A$ (resp. $H : \T^\text{op}\rightarrow A$) is called {\it homological} (resp. {\it cohomological}) if for every distinguished triangle 
  \begin{tikzcd}
    X \arrow{r}{f} & Y \arrow{r}{g} & Z \arrow{r}{h} & X[1]
  \end{tikzcd}
  the sequence
  $$\begin{tikzcd}
    HX \arrow{r}{H(f)} & HY \arrow{r}{H(g)} & HZ
  \end{tikzcd}$$
  (resp. $\begin{tikzcd}
    HZ \arrow{r}{H(g)} & HY \arrow{r}{H(f)} & HX
  \end{tikzcd}$)
  is exact in $\A$ 
\end{defn}

\begin{prop}
  Let $\T$ be a triangulated category.
  If 
  \begin{tikzcd}
    X \arrow{r}{f} & Y \arrow{r}{g} & Z \arrow{r}{h} & X[1]
  \end{tikzcd} 
  is a distinguished triangle, then $g \circ f = 0$ and $h \circ g = 0$.

  \begin{proof}
    By (TR3) we obtain a morphism of distinguished triangles
    $$\begin{tikzcd}
      X \arrow{d}{\id_X}\arrow{r}{\id_X} & X \arrow{d}{f}\arrow{r} & 0\arrow[dashed]{d}{\exists !0} \arrow{r} & X[1]\arrow{d}{\id_{X[1]}}\\
      X \arrow{r}{f} & Y \arrow{r}{g} & Z \arrow{r}{h} & X[1]
    \end{tikzcd}$$
    and thus $g \circ f = 0$.

    That $h \circ g = 0$ follows from the same argument applied to the distinguished triangle
    $$\begin{tikzcd}
      Y \arrow{r}{g} & Z \arrow{r}{h} & X[1] \arrow{r}{-g[1]} & Y[1].
    \end{tikzcd}$$
  \end{proof}
\end{prop}

\begin{prop}
  Let $\T$ be a triangulated category.
  For any object $X$ of $\T$, the functor $h^X = \Hom{\T}{X,\_}$ (resp. $h_X = \Hom{\T}{\_,X}$) is homological (resp. cohomological).
  
  \begin{proof}
    Given a distinguished triangle \begin{tikzcd}A \arrow{r}{f} & B \arrow{r}{g} & C \arrow{r}{h} & A[1]\end{tikzcd}, we obtain a complex
    $$\begin{tikzcd}
      \Hom{\T}{X,A} \arrow{r}{f_*} &\Hom{\T}{X,B} \arrow{r}{g_*} & \Hom{\T}{X,C}.
    \end{tikzcd}$$
    since for any morphism $\varphi : X \rightarrow A$ 
    $$g_* \circ f_*(\varphi) = g \circ f \circ \varphi = 0.$$
    We need only show that $f_* = \ker g_*$.
    
    Given a morphism $\psi : X \rightarrow B$ such that $g_*(\psi) = \psi \circ g= 0$, we have a morphism of distinguished triangles by (TR1)
    $$\begin{tikzcd}
      X \arrow{r}\arrow{d}{\psi} & 0 \arrow{r}\arrow{d} & X[1] \arrow[dashed]{d}{\exists \gamma}\arrow{r}{\id_{X[1]}} & X[1]\arrow{d}{\psi[1]}\\
      B \arrow{r}{g} & C \arrow{r}{h} & A[1] \arrow{r}{-f[1]} & B[1]
    \end{tikzcd},$$
    which yields $f_*(-\gamma[-1]) = -f \circ \gamma[-1] = \psi$.
    Therefore $f_* = \ker g_*$, as desired.
    
    By replacing $\T$ by $\T^\text{op}$, we see that $h_X$ is cohomological.
  \end{proof}
\end{prop}


\begin{prop}\label{isotriangle}
  Let $\T$ be a triangulated category.
  If 
  \begin{tikzcd}
    X \arrow{r}{f} & Y \arrow{r} & Z \arrow{r} & X[1]
  \end{tikzcd} is a distinguished triangle, then $f$ is an isomorphism if and only if $Z \cong 0$.
  
  \begin{proof}
    Assume that $f$ is an isomorphism.
    We have by (TR3) a morphism of triangles
    $$\begin{tikzcd}
      X \arrow{d}{\id_X}\arrow{r}{f} & Y \arrow{d}{f^{-1}}\arrow{r} & Z \arrow[dashed]{d}{\exists}\arrow{r} & X[1]\arrow{d}{\id_{X[1]}}\\
      X \arrow{r}{\id_X} & X \arrow{r} & 0 \arrow{r} & X[1]
    \end{tikzcd}$$
    which is an isomorphism of triangles by the 5 Lemma.
    Hence $Z \cong 0$.
    
    Conversely, if $Z \cong 0$, then we have morphisms of distinguished triangles by (TR3)
    $$\begin{tikzcd}
      Z \arrow{d}\arrow{r} & X[1] \arrow{d}{\id_{X[1]}}\arrow{r}{f[1]} & Y[1]\arrow[dashed]{d}{\exists g} \arrow{r} & Z[1]\arrow{d}\\
      0 \arrow{r} & X[1] \arrow{r}{\id_{X[1]}} & X[1] \arrow{r} & 0
    \end{tikzcd}$$
    and the induced morphism is an isomorphism by the 5 Lemma.
    We also have a morphism of distinguished triangles by (TR3)
    $$\begin{tikzcd}
      Z \arrow{d}\arrow{r} & X[1] \arrow{d}{g^{-1}}\arrow{r}{f[1]} & Y[1]\arrow[dashed]{d}{\exists h} \arrow{r} & Z[1]\arrow{d}\\
      0 \arrow{r} & Y[1] \arrow{r}{\id_{Y[1]}} & Y[1] \arrow{r} & 0
    \end{tikzcd}$$
    We see that $g^{-1} = h \circ f[1]$ and thus
    $$\id_{Y[1]} = g^{-1} \circ g = h \circ (f[1] \circ g) = h$$
    yields $f[1]$ an isomorphism with inverse $g$.
    Moreover $f$ is also an isomorphism because $[1]$ is an equivalence of categories and hence reflects isomorphisms.
    Therefore $f$ is an isomorphism if and only if $Z \cong 0$.
  \end{proof}
\end{prop}

\begin{prop}\label{sumtriangles}
  Let $\T$ be a triangulated category admitting coproducts indexed by a set $I$.
  If 
  $$\left\{D_i : \begin{tikzcd}X_1 \arrow{r}{f_i} & Y_i \arrow{r}{g_i} & Z_i\arrow{r}{h_i} & X_i[1]\end{tikzcd}\right\}_{i\in I}$$ 
  is a collection of distinguished triangles, then the triangle 
  $$\bigoplus_{i \in I} D_i : \begin{tikzcd} \bigoplus_{i \in I} X_i \arrow{r}{\oplus f_i} & \bigoplus_{i \in I}Y_i \arrow{r}{\oplus g_i} & \bigoplus_{i \in I} Z_i \arrow{r}{\oplus h_i} & \bigoplus_{i \in I} X_i[1]\end{tikzcd}$$
  is distinguished.

  \begin{proof}
    We first note that because $[1]$ is an autoequivalence it is left adjoint to $[-1]$ and hence necessarily commutes with coproducts, so that there is a unique isomorphism $\left(\bigoplus_{i \in I}X_i\right)[1] \cong \bigoplus_{i \in I}X_i[1]$.
    By (TR1) and (TR3) we have for each $i$ a morphism of distinguished triangles
    $$\begin{tikzcd} 
      X_i \arrow{r}{f_i}\arrow{d} & Y_i\arrow{d}\arrow{r}{g_i} & Z_i \arrow[dashed]{d}{\exists \gamma_i}\arrow{r}{h_i} & X_i[1] \arrow{d}\\
      \bigoplus_{i \in I} X_i \arrow{r}{\oplus f_i} & \bigoplus_{i \in I}Y_i \arrow{r}{g} & Z \arrow{r}{h} & \bigoplus_{i \in I} X_i[1]
    \end{tikzcd}.$$
    For any object $Z^\prime$ of $\T$, applying the cohomological functor $h_{Z^\prime}$ to the triangle $D_i$ gives a long exact sequence
    $$h_{Z^\prime}(D_i) : 
    \begin{tikzcd} 
      \cdots \arrow{r} & h_{Z^\prime}(Y_i[1]) \arrow{r} & h_{Z^\prime}(X_i[1]) \arrow{r} & h_{Z^\prime}(Z_i) \arrow{r} & h_{Z^\prime}(Y_i) \arrow{r} & h_{Z^\prime}(X_i) \arrow{r} & \cdots
    \end{tikzcd},$$
    and similarly gives a long exact sequence
    $$\begin{tikzcd}
      \cdots \arrow{r} & h_{Z^\prime}\left(\bigoplus_{i \in I}Y_i[1]\right) \arrow{r} & h_{Z^\prime}\left(\bigoplus_{i \in I}X_i[1]\right) \arrow{r} & h_{Z^\prime}(Z) \arrow{r} & h_{Z^\prime}\left(\bigoplus_{i \in I}Y_i\right) \arrow{r} & h_{Z^\prime}\left(\bigoplus_{i \in I} X_i\right) \arrow{r} & \cdots
    \end{tikzcd}$$
    We note that cohomology commutes with direct products, so we obtain the commutative diagram
    $$\begin{tikzcd} 
      \prod_{i \in I}h_{Z^\prime}(Y_i[1]) \arrow{r}\arrow{d}{\alpha_1} & \prod_{i \in I}h_{Z^\prime}(X_i[1]) \arrow{r}\arrow{d}{\alpha_2} & \prod_{i \in I}h_{Z^\prime}(Z_i) \arrow{r}\arrow{d}{\alpha_3} & \prod_{i \in I}h_{Z^\prime}(Y_i) \arrow{r}\arrow{d}{\alpha_4} & \prod_{i \in I}h_{Z^\prime}(X_i)\arrow{d}{\alpha_5}\\
      h_{Z^\prime}\left(\bigoplus_{i \in I}Y_i[1]\right) \arrow{r} & h_{Z^\prime}\left(\bigoplus_{i \in I}X_i[1]\right) \arrow{r} & h_{Z^\prime}(Z) \arrow{r} & h_{Z^\prime}\left(\bigoplus_{i \in I}Y_i\right) \arrow{r} & h_{Z^\prime}\left(\bigoplus_{i \in I} X_i\right)
    \end{tikzcd}$$
    with exact rows and $\alpha_1, \alpha_2, \alpha_4, \alpha_5$ all isomorphisms.
    By the 5 Lemma we see that $\alpha_3$ is an isomorphism, and thus by the Yoneda Lemma
    $$h^Z(Z^\prime) = h_{Z^\prime}(Z) \cong \prod_{i \in I}h_{Z^\prime}(Z_i) \cong h_{Z^\prime}\left(\bigoplus_{i \in I}Z_i\right) = h^{\bigoplus_{i \in I} Z_i}(Z^\prime).$$
    implies there is an isomorphism $\gamma : \bigoplus_{i \in I} Z_i \rightarrow Z$ which yields an isomorphism of triangles
    $$\begin{tikzcd}
      \bigoplus_{i \in I} X_i \arrow{d}{\id}\arrow{r}{\oplus f_i} & \bigoplus_{i \in I}Y_i \arrow{d}{\id}\arrow{r}{\oplus g_i} & \bigoplus_{i \in I} Z_i \arrow{d}{\gamma}\arrow{r}{\oplus h_i} & \bigoplus_{i \in I} X_i[1]\arrow{d}{\id}\\
      \bigoplus_{i \in I} X_i \arrow{r}{\oplus f_i} & \bigoplus_{i \in I}Y_i \arrow{r}{g} & Z \arrow{r}{h} & \bigoplus_{i \in I} X_i[1]
    \end{tikzcd}.$$
    Therefore $\bigoplus_{i \in I}D_i$ is distinguished, as desired.
  \end{proof}
\end{prop}

\begin{cor}\label{corsumtriangles}
  Let $\T$ be a triangulated category and let $X,Y$ be objects of $\T$.
  The triangle
  $$\begin{tikzcd}
    X \arrow{r} & X \oplus Y \arrow{r} & Y \arrow{r}{0} & X[1]
  \end{tikzcd}$$
  is distinguished.
  
  \begin{proof}
    Realize 
    $$\begin{tikzcd}
      X \arrow{r} & X \oplus Y \arrow{r} & Y \arrow{r}{0} & X[1]
    \end{tikzcd}$$
    as the coproduct of the distinguished triangles
    $$\begin{tikzcd}
      X \arrow{r} & X \arrow{r} & 0\arrow{r} & X[1]
    \end{tikzcd}
    \text{and}
    \begin{tikzcd}
      0 \arrow{r} & Y \arrow{r} & Y \arrow{r} & 0.
    \end{tikzcd}$$
  \end{proof}
\end{cor}

\begin{cor}\label{splittriangle}
  Let $\T$ be a triangulated category.
  If 
  \begin{tikzcd}
    X \arrow{r} & Y \arrow{r} & Z \arrow{r}{0} & X[1]
  \end{tikzcd}
  is a distinguished triangle, then $Y \cong X \oplus Z$.
  
  Moreover, if $\imath_X \colon X \rightarrow Y$, $\imath_Z \colon \rightarrow Y$, $\pi_X \colon Y \rightarrow X$, and $\pi_Z \colon Y \rightarrow Z$ are the coproduct injections and product projections, then 
  $$\begin{tikzcd}
    Y \arrow{rr}{\imath_X \circ p_X + \imath_Z \circ p_Z} & & Y \arrow{r} & 0 \arrow{r} & Y[1]
  \end{tikzcd}$$
  is a distinguished triangle.
  
  \begin{proof}
    By taking rotates, we obtain a morphism of distinguished triangles
    $$\begin{tikzcd}
      Z[-1] \arrow{d}{\id_{Z[-1]}}\arrow{r}{0} & X \arrow{d}{\id_X}\arrow{r} & Y \arrow{r}\arrow[dashed]{d}{\exists} & Z\arrow{d}{\id_{Z}}\\
      Z[-1] \arrow{r}{0} & X \arrow{r} & X \oplus Z \arrow{r} & X[1]
    \end{tikzcd}$$
    and the induced morphism is an isomorphism by the 5 Lemma.
    
    For the second assertion, we note that by Corollary~\ref{splittriangle} we have $p_Z \circ i_X = 0$ and $p_X \circ i_Z = 0$, hence if we let $\varphi = \imath_X \circ p_X + \imath_Z \circ p_Z$, then the diagram
    %$$\left(\imath_X \circ p_X + \imath_Z \circ p_Z\right)\imath_X = \imath_X \circ p_X \circ \imath_X + \imath_Z \circ p_Z \circ \imath_X = \imath_X$$
    %and 
    %$$\left(\imath_X \circ p_X + \imath_Z \circ p_Z\right)\imath_Z = \imath_X \circ p_X \circ \imath_Z + \imath_Z \circ p_Z \circ \imath_Z = \imath_Z$$
    $$\begin{tikzcd}
      X \arrow[bend right]{rdd}{i_X}\arrow{rd}{\imath_X} & & Z \arrow{ld}{\imath_Z}\arrow[bend left]{ldd}{i_Z}\\
      & Y \arrow{d}{\varphi}\\
      & Y
    \end{tikzcd}$$
    commutes, and by unicity $\varphi = \id_Y$.
  \end{proof}
\end{cor}

\begin{prop}\label{extendsquare}
  Let $\T$ be a triangulated category.
  Any commutative diagram
  $$\begin{tikzcd}
    A \arrow{r}{i}\arrow{d}{u}& B\arrow{d}{u^\prime}\\
    A^\prime \arrow{r}{i^\prime}& B^\prime
  \end{tikzcd}$$
  may be extended to a diagram
  $$\begin{tikzcd}
    A \arrow{r}{i}\arrow{d}{u}& B\arrow{d}{u^\prime}\arrow{r}{j} & C \arrow{r}{k}\arrow{d}{u^{\prime\prime}} & A[1]\arrow{d}{u[1]}\\
    A^\prime \arrow{r}{i^\prime}\arrow{d}{v}& B^\prime\arrow{r}{j^\prime}\arrow{d}{v^\prime} & C^\prime \arrow{r}{k^\prime}\arrow{d}{v^{\prime\prime}} & A^\prime[1]\arrow{d}{v[1]}\\
    A^{\prime\prime} \arrow{r}{i^{\prime\prime}}\arrow{d}{w} & B^{\prime\prime} \arrow{r}{j^{\prime\prime}}\arrow{d}{w^\prime} & C^{\prime\prime} \arrow{r}{k^{\prime\prime}}\arrow{d}{w^{\prime\prime}} & A^{\prime\prime}[1]\arrow{d}{-w[1]}\\
    A[1] \arrow{r}{i[1]} & B[1] \arrow{r}{j[1]} & C[1] \arrow{r}{-k[1]} & A[2]
  \end{tikzcd}$$
  where all rows and columns are distinguished triangles, and all squares are commutative but the bottom right, which is anti-commutative,
  $w[1] \circ k^{\prime\prime} = - k[1] \circ w^{\prime\prime}$.
  
  \begin{proof}
    We use (TR1) to produce the first two rows and columns, and the distinguished triangle 
    $$\begin{tikzcd}
      A \arrow{rr}{i^\prime \circ u = u^\prime \circ i} & & B^\prime \arrow{r}{g} & D \arrow{r}{h} & A[1].
    \end{tikzcd}$$
    Next we apply (TR4) to get the octohedral diagrams 
    $$\begin{tikzcd}
      A \arrow{r}{i}\arrow{d}{\id_A} & B\arrow{r}{j}\arrow{d}{u^\prime} & C\arrow{r}{k}\arrow{d}{\alpha} & A[1]\arrow{d}{\id_{A[1]}}\\
      A \arrow{r}{u^\prime \circ i}\arrow{d}{i} & B^\prime \arrow{r}{g}\arrow{d}{\id_{B^\prime}} & D \arrow{r}{h}\arrow{d}{\beta} & A[1]\arrow{d}{i[1]}\\
      B \arrow{r}{u^\prime}\arrow{d}{j} & B^\prime \arrow{r}{v^\prime}\arrow{d}{g} & B^{\prime\prime} \arrow{r}{w^\prime}\arrow{d}{\id_{B{\prime\prime}}} & B[1]\arrow{d}{j[1]}\\
      C \arrow{r}{\alpha} & D \arrow{r}{\beta} & B^{\prime\prime}\arrow{r}{j[1] \circ w^\prime} & C[1]
    \end{tikzcd}
    \ \text{and}\ 
    \begin{tikzcd}
      A \arrow{r}{u}\arrow{d}{\id_A} & A^\prime \arrow{r}{v}\arrow{d}{i^\prime} & A^{\prime\prime}\arrow{r}{w}\arrow{d}{\alpha^\prime} & A[1]\arrow{d}{\id_{A[1]}}\\
      A\arrow{r}{i^\prime \circ u}\arrow{d}{u} & B^\prime \arrow{r}{g}\arrow{d}{\id_{B^\prime}} & D \arrow{r}{h}\arrow{d}{\beta^\prime} & A[1]\arrow{d}{u[1]}\\
      A^\prime \arrow{r}{i^\prime}\arrow{d}{v} & B^\prime \arrow{r}{j^\prime}\arrow{d}{g} & C^\prime \arrow{r}{k^\prime}\arrow{d}{\id_{C^\prime}} & A^\prime[1]\arrow{d}{v[1]}\\
      A^{\prime\prime} \arrow{r}{\alpha^\prime} & D \arrow{r}{\beta^\prime} & C^\prime \arrow{r}{v[1] \circ k^\prime} & A^{\prime\prime}[1]
    \end{tikzcd}$$
    with distinguished rows, from which we obtain the morphisms of triangles
    %which yields the distinguished triangles
%    $$\begin{tikzcd}
%      C \arrow{r}{\alpha} & D \arrow{r}{\beta} & B^{\prime\prime} \arrow{r}{j[1] \circ w^\prime} & C[1]
%    \end{tikzcd}
%    \ \text{and}\ 
%    \begin{tikzcd}
%      A^{\prime\prime} \arrow{r}{\alpha^\prime} & D \arrow{r}{\beta^\prime} & C^\prime \arrow{r}{v[1] \circ k^\prime} & A^{\prime\prime}[1]
%    \end{tikzcd}$$
    $$\begin{tikzcd}
      A \arrow{r}{u}\arrow{d}{\id_A} & A^\prime \arrow{r}{v}\arrow{d}{i^\prime} & A^{\prime\prime} \arrow{r}{w}\arrow{d}{\alpha^\prime} & A[1] \arrow{d}{\id_{A[1]}}\\
      A \arrow{r}{i^\prime \circ u} \arrow{d}{i} & B^\prime \arrow{r}{g}\arrow{d}{\id_{B^\prime}} & D \arrow{r}{h}\arrow{d}{\beta} & A[1]\arrow{d}{i[1]}\\
      B \arrow{r}{u^\prime} & B^\prime \arrow{r}{v^\prime} & B^{\prime\prime} \arrow{r}{w^\prime} & B[1]
    \end{tikzcd}
    \ \text{and}\ 
    \begin{tikzcd}
      A \arrow{r}{i} \arrow{d}{\id_A} & B \arrow{r}{j}\arrow{d}{u^\prime} & C \arrow{r}{k}\arrow{d}{\alpha} & A[1]\arrow{d}{\id_{A[1]}}\\
      A \arrow{r}{u^\prime \circ i}\arrow{d}{u} & B^\prime \arrow{r}{g}\arrow{d}{\id_{B^\prime}} & D \arrow{r}{h}\arrow{d}{\beta^\prime} & A[1]\arrow{d}{u[1]}\\
      A^\prime \arrow{r}{i^\prime} & B^\prime \arrow{r}{j^\prime} & C^\prime \arrow{r}{k^\prime} & A^\prime[1].
    \end{tikzcd}$$

    Define $i^{\prime\prime} = \beta \circ \alpha^\prime$, $u^{\prime\prime} = \beta^\prime \circ \alpha$.
    Apply (TR1) to get the third row
    $$\begin{tikzcd}
      A^{\prime\prime} \arrow{r}{i^{\prime\prime}} & B^{\prime\prime} \arrow{r}{j^{\prime\prime}} & C^{\prime\prime} \arrow{r}{k^{\prime\prime}} & A^{\prime\prime}[1]
    \end{tikzcd}$$
    and then apply (TR4) to get the octohedral diagram
    $$\begin{tikzcd}
      A^{\prime\prime} \arrow{r}{\alpha^\prime}\arrow{d}{\id_{A^{\prime\prime}}} & D \arrow{r}{\beta^\prime}\arrow{d}{\beta} & C^\prime \arrow{r}{v[1] \circ k^\prime}\arrow{d}{v^{{\prime\prime}}} & A^{\prime\prime}[1]\arrow{d}{\id_{A^{\prime\prime}}}\\
      A^{\prime\prime} \arrow{r}{i^{\prime\prime}}\arrow{d}{\alpha^\prime} & B^{\prime\prime} \arrow{r}{j^{\prime\prime}}\arrow{d}{\id_{B^{\prime\prime}}} & C^{\prime\prime} \arrow{r}{k^{\prime\prime}}\arrow{d}{w^{\prime\prime}} & A^{\prime\prime}[1]\arrow{d}{\alpha^\prime[1]}\\
      D \arrow{r}{\beta}\arrow{d}{\beta^\prime} & B^{\prime\prime} \arrow{r}{j[1] \circ w^\prime}\arrow{d}{j^{\prime\prime}} & C[1] \arrow{r}{-\alpha[1]}\arrow{d}{\id_{C[1]}} & D[1]\arrow{d}{\beta^\prime[1]}\\
      C^\prime \arrow{r}{v^{\prime\prime}} & C^{\prime\prime} \arrow{r}{w^{\prime\prime}} & C[1] \arrow{r}{-u^{\prime\prime}[1]} & C^\prime[1]
    \end{tikzcd}$$
    with distinguished rows,
    %to produce a distinguished triangle
    %$$\begin{tikzcd}
    %  C^\prime \arrow{r}{v^{\prime\prime}} & C^{\prime\prime} \arrow{r}{w^{\prime\prime}} & C[1] \arrow{rrr}{-\beta^\prime[1] \circ \alpha[1] = -u^{\prime\prime}[1]} & & & C^\prime[1],
    %\end{tikzcd}$$
    then take the rotate of bottow row of the diagram to get the distinguished triangle
    $$\begin{tikzcd}
      C \arrow{r}{u^{\prime\prime}} & C^\prime \arrow{r}{v^{\prime\prime}} & C^{\prime\prime} \arrow{r}{w^{\prime\prime}} & C[1].
    \end{tikzcd}$$
    This gives the third column, and we note here that
    $$u^{\prime\prime} \circ j = \beta^\prime \circ \alpha \circ j = \beta^\prime \circ g \circ u^\prime = j^\prime \circ u^\prime$$
    and 
    $$k^\prime \circ u^{\prime\prime} = \beta^\prime \circ \alpha \circ k^\prime = u[1] \circ h \circ \alpha = u[1] \circ k,$$
    so the top row of the diagram is commutative.
    
    For the second row of the diagram, we note that the first square commutes by construction,
    %$$i^{\prime\prime} \circ v = \beta \circ \alpha^\prime \circ v = \beta \circ g \circ i^\prime = v^\prime \circ i^\prime,$$
    $j^\prime = \beta^\prime \circ g$ by the second octohedral diagram, whence
    $$v^{\prime\prime} \circ j^\prime = v^{\prime\prime} \circ \beta^\prime \circ g = j^{\prime\prime} \circ \beta \circ g = j^{\prime\prime} \circ \beta \circ g = j^{\prime\prime} \circ v^\prime,$$
    and finally $k^{\prime\prime} \circ v^{\prime\prime} = v[1] \circ k^\prime$ follows from the last octohedral diagram.
    Hence the second row of the diagram is commutative.
    
    Similarly, the first square of the third row commutes by construction and the last octohodreal diagram gives $w^{\prime\prime} \circ j^{\prime\prime} = j[1] \circ w$.
    Finally, the first octohedral diagram gives $k = h \circ \alpha$ and the last octohedral diagram gives $-\alpha[1] \circ w^{\prime\prime} = \alpha^\prime[1] \circ k^{\prime\prime}$, and hence    
    $$-k[1] \circ w^{\prime\prime} = -(h \circ \alpha)[1] \circ w^{\prime\prime} = h[1] \circ -\alpha[1] \circ w^{\prime\prime} = h[1] \circ \alpha^\prime[1] \circ k^{\prime\prime} = w[1] \circ k^{\prime\prime},$$
    as desired.
  \end{proof}
  %TODO: Maybe rewrite the TR4 steps to actually include the octohedral diagrams.
\end{prop}

\subsection{Homotopy}
%I have no idea where this is supposed to fit right now. 
%Just need (some) of these things for Verdier Localization.
%Will figure out placement later.
\begin{defn}
  Let $\T$ be a triangulated category.
  Given a morphism, $f$, of triangles
  $$\begin{tikzcd}
    X_1 \arrow{r}{u}\arrow{d}{f_1} & X_2 \arrow{r}{v}\arrow{d}{f_2} & X_3 \arrow{r}{w}\arrow{d}{f_3} & X_1[1]\arrow{d}{f_1[1]}\\
    Y_1 \arrow{r}{u^\prime} & Y_2 \arrow{r}{v^\prime} & Y_3 \arrow{r}{w^\prime} & Y_1[1]
  \end{tikzcd}$$
  we can form the {\it mapping cone} of $f$, 
  $$\begin{tikzcd}
    X_2 \oplus Y_1 \arrow{rr}{-v \oplus 0 + u^\prime \oplus f_2} & & X_3 \oplus Y_2 \arrow{rr}{-w \oplus 0 + v^\prime \oplus f_2} & & X_1[1] \oplus Y_3 \arrow{rrr}{-u[1] \oplus 0 + w^\prime \oplus f_1[1]} & & & X_2[1] \oplus Y_1[1]
  \end{tikzcd}$$
\end{defn}
\begin{defn}
  Two morphisms of triangles, $f$ and $g$,
  $$\begin{tikzcd}
    X_1 \arrow[shift left]{d}{g_1}\arrow[shift right,swap]{d}{f_1}\arrow{r}{u}
    & X_2 \arrow[shift left]{d}{g_2}\arrow[shift right,swap]{d}{f_2}\arrow{r}{dv} 
    & X_3 \arrow[shift left]{d}{g_3}\arrow[shift right,swap]{d}{f_3}\arrow{r}{w} 
    & X_1[1]\arrow[shift left]{d}{g_1[1]}\arrow[shift right,swap]{d}{f_1[1]}\\
    Y_1 \arrow{r}{u^\prime}& Y_2\arrow{r}{v^\prime} & Y_3\arrow{r}{w^\prime} & Y_3[1]
  \end{tikzcd}$$
  are said to be {\it homotopic} if there exist morphisms $s_i$ as in the diagram below
  $$\begin{tikzcd}
    X_1 \arrow{r}{u} & X_2 \arrow{r}{v}\arrow[swap]{ld}{s_2} & X_3 \arrow[swap]{ld}{s_3}\arrow{r}{w} & X_1[1] \arrow[swap]{ld}{s_1[1]}\\
    Y_1 \arrow{r}{u^\prime} & Y_2 \arrow{r}{v^\prime} & Y_3 \arrow{r}{w^\prime} & Y_1[1]
  \end{tikzcd}$$
  such that 
  \begin{eqnarray*}
    f_1 - g_1 &=& s_2 \circ u + w^\prime[-1] \circ s_1\\
    f_2 - g_2 &=& s_3 \circ v + u^\prime \circ s_2,\,\text{and}\\
    f_3 - g_3 &=& s_1[1] \circ w + v^\prime \circ s_3.
  \end{eqnarray*}
\end{defn}

\begin{defn}
  We say that a sequence
  $$\begin{tikzcd}
    X \arrow{r} & Y \arrow{r} & Z \arrow{r} & X[1]
  \end{tikzcd}$$
  is {\it contractible} if 
  if the morphism
  $$\begin{tikzcd}
    X \arrow{d}{id_X}\arrow{r} & Y \arrow{r}\arrow{d}{\id_Y} & Z \arrow{r}\arrow{d}{\id_Z} & X[1] \arrow{d}{\id_{X[1]}}\\
    X \arrow{r} & Y \arrow{r} & Z \arrow{r} & X[1]
  \end{tikzcd}$$
  is null homotopic (i.e. homotopic to the zero morphism).
\end{defn}

\begin{prop}
  If 
  $\begin{tikzcd}
    X \arrow{r} & Y \arrow{r} & Z \arrow{r} & X[1]
  \end{tikzcd}$
  is contractible, then it is distinguished.
\end{prop}

\begin{lem}
  The mapping cone on the zero morphism between distinguished triangles is a distinguished triangle.
\end{lem}

\begin{cor}
  Given a morphism of distinguished triangles
  $$\begin{tikzcd}
    X \arrow{r}\arrow{d} & Y \arrow{r}\arrow{d} & Z \arrow{r}\arrow{d} & X[1]\arrow{d}\\
    X^\prime \arrow{r} & Y^\prime \arrow{r} & Z^\prime \arrow{r} & X^\prime[1]
  \end{tikzcd}$$
  if the morphism is null homotopic, then the mapping cone is a distinguished triangle.
\end{cor}
\begin{cor}
  Given a commutative diagram
  $$\begin{tikzcd}
    X \arrow{r}\arrow{d} & Y \arrow{r}\arrow{d} & Z \arrow{r}\arrow{d} & X[1]\arrow{d}\\
    X^\prime \arrow{r} & Y^\prime \arrow{r} & Z^\prime \arrow{r} & X^\prime[1]
  \end{tikzcd}$$
  if one of the rows is distinguished and the other is null homotopic, then the cone of the morphism is a distinguished triangle.
\end{cor}

%From Neeman, Some new axioms for triangulated categories.
\begin{prop}
  Given a commutative diagram
  $$\begin{tikzcd}
    X \arrow{r}{f}\arrow{d}{u} & Y\arrow{d}{u^\prime}\\
    X^\prime \arrow{r}{f^\prime} & Y^\prime
  \end{tikzcd}$$
  by (TR1) and (TR3) we obtain a morphism, $u$, of distinguished triangles
  $$\begin{tikzcd}
      X \arrow{r}{f}\arrow{d}{u} & Y\arrow{d}{u^\prime}\arrow{r}{g} & Z \arrow{r}{h}\arrow[dashed]{d}{\exists u^{\prime\prime}} & X[1]\arrow{d}{u[1]}\\
    X^\prime \arrow{r}{f^\prime} & Y^\prime \arrow{r}{g^\prime} & Z^\prime \arrow{r}{h^\prime} & X^\prime[1]
  \end{tikzcd}$$
  We may choose $h$ such that the mapping cone of $u$ is a distinguished triangle.

\begin{proof}
  First we show that we have an octohedral diagram
  $$\begin{tikzcd}
    X^\prime \oplus X \arrow{r}{\alpha}\arrow{d}{\id_{X^\prime \oplus X}} & X^\prime \oplus Y^\prime \oplus Y \arrow{r}{\beta}\arrow{d}{\delta} & Y^\prime \oplus Z \arrow{r}{\gamma}\arrow{d}{\varphi} & X^\prime[1] \oplus X[1]\arrow{d}{\id_{X^\prime \oplus X}[1]}\\
    X^\prime \oplus X \arrow{r}{\delta \circ \alpha}\arrow{d}{\alpha} & Y^\prime \arrow{r}{\beta^\prime}\arrow{d}{\id_{Y^\prime}} & Z^\prime \oplus X[1] \arrow{r}{\gamma^\prime}\arrow{d}{\psi} & X^\prime[1] \oplus X[1]\arrow{d}{\alpha[1]}\\
    X^\prime \oplus Y^\prime \oplus Y \arrow{r}{\delta}\arrow{d}{\beta} & Y^\prime \arrow{r}{\beta^{\prime\prime}}\arrow{d}{\beta^\prime} & X^\prime[1] \oplus Y[1] \arrow{r}{\gamma^{\prime\prime}}\arrow{d}{\id_{X^\prime \oplus Y}[1]} & X^\prime[1] \oplus Y[1] \oplus Y^\prime[1]\arrow{d}{\beta[1]}\\
    Y\prime \oplus Z \arrow{r}{\varphi} & Z^\prime \oplus X[1] \arrow{r}{\psi} & X^\prime[1] \oplus Y[1] \arrow{r}{\beta[1] \circ \gamma^{\prime\prime}} & Y^\prime[1] \oplus Z[1]
  \end{tikzcd}$$
  
  We first construct the first row.
  We have a distinguished triangle
  $$\begin{tikzcd}
    X \arrow{r}{f} & Y \arrow{r}{g} & Z \arrow{r}{h} & X[1]
  \end{tikzcd}$$
  and we form the morphisms
  $$\begin{tikzcd}
    & & X^\prime \arrow[swap]{lldd}{\id_{X^\prime}}\arrow{rrdd}{f^\prime}\arrow[dashed,yshift=-1.5ex]{dd}{\exists !\id_{X^\prime} \times f^\prime}\\
    \\
    X^\prime & & X^\prime \oplus Y^\prime \arrow[swap]{ll}{p_{X^\prime}}\arrow{rr}{p_{Y^\prime}} & & Y^\prime
  \end{tikzcd}
  \ \text{and}\ 
  \begin{tikzcd}
    X^\prime \arrow{rrd}{\imath_{X^\prime}}\arrow[swap,bend right]{rrdd}{-f^\prime} & & & & Y^\prime \arrow[swap]{lld}{\imath_{Y^\prime}}\arrow[bend left]{lldd}{\id_{Y^\prime}}\\
    & & X^\prime \oplus Y^\prime \arrow[dashed]{d}{\exists ! -f \oplus \id_{Y^\prime}}\\
    & & Y^\prime
  \end{tikzcd}$$
  then note that by construction we have
  $$(f^\prime \circ p_{X^\prime} + (-f \oplus \id_{Y^\prime})) \circ \imath_{X^\prime} = f^\prime - f^\prime = 0$$
  as well as
  $$(f^\prime \circ p_{X^\prime} + (-f \oplus \id_{Y^\prime})) \circ \imath_{Y^\prime} = 0 + \id_{Y^\prime}$$
  which implies 
  $$p_{Y^\prime} = f^\prime \circ p_{X^\prime} + (-f \oplus \id_{Y^\prime}).$$
  Taking the sequence
  $$\begin{tikzcd}
    X^\prime \arrow{rr}{\id_{X^\prime} \times f^\prime} & &
    X^\prime \oplus Y^\prime \arrow{rr}{-f^\prime \oplus \id_{Y^\prime}} & &
    Y^\prime \arrow{r}{0} & 
    X^\prime[1].
  \end{tikzcd}$$
  we see it is contractible with homotopy given by
  $$\begin{tikzcd}
    X^\prime \arrow{rr}{\id_{X^\prime} \times f^\prime} & &
    X^\prime \oplus Y^\prime \arrow{rr}{-f^\prime \oplus \id_{Y^\prime}}\arrow[swap]{lld}{p_{X^\prime}} & &
    Y^\prime \arrow{r}{0}\arrow[swap]{lld}{\imath_{Y^\prime}} &
    X^\prime[1] \arrow[swap]{ld}{0}\\
    X^\prime \arrow{rr}{\id_{X^\prime} \oplus f^\prime} & &
    X^\prime \oplus Y^\prime \arrow{rr}{-f^\prime \oplus \id_{Y^\prime}} & &
    Y^\prime \arrow{r}{0} & 
    X^\prime[1]\\
  \end{tikzcd},$$
  since if we let $\varphi = \id_{X^\prime} \times f^\prime \circ p_{X^\prime} + \imath_{Y^\prime} \circ -f^\prime \oplus \id_{Y^\prime}$, then
  we get a commutative diagram
  $$\begin{tikzcd}
    & X^\prime \oplus Y^\prime \arrow{ld}{p_{X^\prime}}\arrow[swap]{rd}{p_{Y^\prime}}\arrow{d}{\varphi}\\
    X^\prime & X^\prime \oplus Y^\prime \arrow[swap]{l}{p_{X^\prime}}\arrow{r}{p_{Y^\prime}}& Y^\prime
  \end{tikzcd}$$
  which implies by unicity that $\varphi = \id_{X^\prime \oplus Y^\prime}$.
  We now obtain the first row of the diagram by taking the coproduct of these triangles.
\end{proof}
\end{prop}

\begin{defn}
  Let $\T$ be a triangulated category.
  A commutative square
  $$\begin{tikzcd}
    X \arrow{r}{f}\arrow{d}{g} & Y\arrow{d}{g^\prime}\\
    Z \arrow{r}{f^\prime} & W
  \end{tikzcd}$$
  is called {\it homotopy cartesian} if there is a distinguished triangle
  $$\begin{tikzcd}
    X \arrow{r}{f \times g} & Y \oplus Z \arrow{r}{-f^\prime \oplus g^\prime} & W \arrow{r} & X[1].
  \end{tikzcd}$$
  We call $X$ the {\it homotopy pullback} and $W$ the {\it homotopy pushout}.
\end{defn}

\begin{prop}
  Homotopy pullbacks and pushouts exist and are unique up to (not necessarily unique) isomorphism.
  
  \begin{proof}
    First we prove the result for pushouts and note that pullbacks follow from an analogous argument with the arrows turned around.
    Given a diagram
    $$\begin{tikzcd}
      X \arrow{r}{f}\arrow{d}{g} & Y\\
      Z
    \end{tikzcd}$$
    we produce a triangle 
    $$\begin{tikzcd}
      X \arrow{r}{f \times g} & Y \oplus Z \arrow{r}{u} & W \arrow{r} & X[1]
    \end{tikzcd}$$
    by (TR1), then define
    $$\begin{tikzcd}
      Z \arrow{r}{-\imath_Z}\arrow[swap]{rd}{f^\prime} & Y \oplus Z\arrow{d}{u}\\
      & W
    \end{tikzcd}
    \ \text{and}\ 
    \begin{tikzcd}
      Y \arrow{r}{\imath_Y}\arrow[swap]{rd}{g^\prime} & Y \oplus Z\arrow{d}{u}\\
      & W
    \end{tikzcd}$$
    Now we see that
    \begin{eqnarray*}
      g^\prime \circ f - f^\prime \circ g &=& (u \circ \imath_Y) \circ (\pi_Y \circ f \times g) + (u \circ \imath_Z) \circ (\pi_Z \circ f \times g)\\
      &=& u \circ (\imath_Y \circ \pi_Y + \imath_Z \circ \pi_Z) \circ f \times g\\
      &=& u \circ f \times g\\
      &=& 0
    \end{eqnarray*}
    gives the homotopy cartesian diagram
    $$\begin{tikzcd}
      X \arrow{r}{f}\arrow{d}{g} & Y\arrow{d}{g^\prime}\\
      Z \arrow{r}{f^\prime} & W
    \end{tikzcd}$$
    
    For uniqueness, we see that if we have another homotopy pushout, $W^\prime$, then we get an isomorphism of triangles
    $$\begin{tikzcd}
      X \arrow{r}{f \times g}\arrow{d}{\id_X} & Y \oplus Z \arrow{r}{u}\arrow{d}{\id_{X \oplus Y}} & W \arrow{r}\arrow[dashed]{d}{\exists} & X[1]\arrow{d}{\id_{X[1]}}\\
      X \arrow{r}{f \times g} & Y \oplus Z \arrow{r}{u^\prime} & W^\prime \arrow{r} & X[1]
    \end{tikzcd}$$
    by (TR3) and the 5 Lemma.
    Therefore $W \cong W^\prime$.
  \end{proof}
\end{prop}

%From Neeman, Triangulated Categories.
\subsection{Verdier Localization}
\begin{defn}
  Let $\T^\prime, \T$ be triangulated categories and let $\F \colon \T^\prime \rightarrow \T$ be a triangulated functor.
  The kernel of $\F$ is the inclusion of the full subcategory $\mathscr{K}$ of $\T^\prime$ with objects $K$ such that $\F(K) \cong 0$.
\end{defn}

\begin{lem}
  Let $\T^\prime, \T$ be  triangulated categories, $\F \colon \T^\prime \rightarrow \T$ a triangulated functor, and $\ker{\F} \colon \mathscr{K} \rightarrow \T^\prime$ its kernel.  
  Then $\mathscr{K}$ is a strictly full, saturated triangulated subcategory of $\T^\prime$.
  
  \begin{proof}
    First we note that $\mathscr{K}$ is strictly full by definition.
    Indeed, if $X$ is an object of $\T^\prime$ isomorphic to an object $K$ of $\mathscr{K}$, then 
    $$\F(X) \cong \F(K) \cong 0$$
    implies $X$ is an object of $\mathscr{K}$.
    
    That $\mathscr{K}$ is triangulated now follows from observing that if we have a morphism 
    \begin{tikzcd}K^\prime \arrow{r}{f} & K\end{tikzcd}
      of $\mathscr{K}$, then by (TR1) we may embed $f$ into a distinguished triangle of $\T^\prime$,
      $$\begin{tikzcd}
        K^\prime \arrow{r}{f} & K \arrow{r}{g} & Z \arrow{r}{h} & K^\prime[1]
      \end{tikzcd}$$
      which gives an isomorphism of distinguished triangles in $\T$
      $$\begin{tikzcd}
        \F(K^\prime) \arrow{r}{\F(f)}\arrow{d} & \F(K) \arrow{r}{\F(g)}\arrow{d} & \F(Z) \arrow{r}{\F(h)}\arrow{d} & \F(K^\prime)[1]\arrow{d}\\
        0 \arrow{r} & 0 \arrow{r} & \F(Z) \arrow{r} & 0
      \end{tikzcd}$$
      because $\F$ is triangulated.
      It then follows from Proposition~\ref{isotriangle} that $Z$ is an object of $\mathscr{K}$, so we take the distinguished triangles of $\mathscr{K}$ to be the distinguished triangles
      $$\begin{tikzcd}
        K^\prime \arrow{r} & K \arrow{r} & K^{\prime\prime} \arrow{r} & K^\prime[1]
      \end{tikzcd}$$
      of $\T^\prime$ with $K,K^\prime, K^{\prime\prime}$ objects of $\mathscr{K}$.
      
      Finally, we see $\mathscr{K}$ is saturated, for if $X \oplus Y$ is isomorphic to some object $K$ of $\mathscr{K}$, then 
      $$\F(X) \oplus \F(Y) \cong \F(X \oplus Y) \cong \F(K) \cong 0$$
      implies $F(X) \cong 0$ and $F(Y) \cong 0$.
      Therefore $X$ and $Y$ are objects of $\mathscr{K}$.
  \end{proof}
\end{lem}

\begin{defn}
  Let $\T$ be a triangulated category and let $\T^\prime$ be a strictly full triangulated subcategory.
  For any two objects $X$ and $Y$ of $\T$, define the collection of morphisms $\operatorname{Mor}_{\T^\prime}(X,Y)$ to be the morphisms $f \in \T(X,Y)$ such that there exists some object $Z$ of $\T^\prime$ and a distinguished triangle
  $$\begin{tikzcd}
    X \arrow{r}{f} & Y \arrow{r}{g} & Z \arrow{r}{h} & X[1].
  \end{tikzcd}$$
\end{defn}

\begin{defn}
  Let $\T$ be a triangulated category and let $\T^\prime$ be a strictly full triangulated subcategory.
  Define the subcategory $\operatorname{Mor}_{\T^\prime}$ of $\T$ to be the category with objects those of $\T$ and morphisms $\operatorname{Mor}_{\T^\prime}(X,Y)$ for objects $X$ and $Y$ of $\T$.
\end{defn}

\begin{lem}\label{moriscat}
  Let $T^\prime$ be a triangulated subcategory of a triangulated category $\T$.
  Then
  \begin{enumerate}
    \item
      Every isomorphism $f : X \rightarrow Y$ of $\T$ is in $\operatorname{Mor}_{\T^\prime}(X,Y)$.
    \item
      Let $f : X \rightarrow Y$ and $g : Y \rightarrow Z$ be morphisms of $\T$.
      If any two of 
      $f \in \operatorname{Mor}_{\T^\prime}(X,Y)$,
      $g \in \operatorname{Mor}_{\T^\prime}(Y,Z)$,
      $g \circ f \in \operatorname{Mor}_{\T^\prime}(X,Z)$
      hold, then so does the third. 
  \end{enumerate}
  
  \begin{proof}
    \begin{enumerate}
    \item
      If $f$ is an isomorphism, then by Proposition~\ref{isotriangle}
      $$\begin{tikzcd}
        X \arrow{r}{f} & Y \arrow{r} & 0 \arrow{r} & X[1]
      \end{tikzcd}$$
      is a distinguished triangle, and since $\T^\prime$ is an additive subcategory of $\T$, $0$ is an object of $\T^\prime$.
      Thus $f \in \operatorname{Mor}_{\T^\prime}(X,Y)$ and, in particular, $id_X \in \operatorname{Mor}_{T^\prime}(X,X)$.
    \item
      By (TR1) and (TR4) we have a commutative octohedral diagram
      $$\begin{tikzcd}
        X \arrow{r}{f}\arrow{d}{\id_X} & Y \arrow{r}{i}\arrow{d}{g} & C^\prime \arrow{r}{\ell}\arrow{d}{u} & X[1] \arrow{d}{\id_{X[1]}}\\
        X \arrow{r}{g \circ f}\arrow{d}{f} & Z\arrow{r}{j}\arrow{d}{\id_{Z}} & \arrow{r} C \arrow{r}{m}\arrow{d}{v} & X[1]\arrow{d}{f[1]}\\
        Y \arrow{r}{g}\arrow{d}{i} & Z \arrow{r}{k}\arrow{d}{j} & C^{\prime\prime}\arrow{d}{\id_{C^{\prime\prime}}} \arrow{r}{n} & Y[1]\arrow{d}{i[1]}\\
        C^\prime \arrow{r}{u} & C \arrow{r}{v} & C^{\prime\prime}\arrow{r}{i[1] \circ m} & C^\prime[1]
      \end{tikzcd}$$
      with distinguished rows.
      To say that any two of 
      $f \in \operatorname{Mor}_{\T^\prime}(X,Y)$,
      $g \in \operatorname{Mor}_{\T^\prime}(Y,Z)$,
      $g \circ f \in \operatorname{Mor}_{\T^\prime}(X,Z)$
      hold is to say that any two of $C^\prime, C, C^{\prime\prime}$ are objects of $\T^\prime$.
      Since $\T^\prime$ is a strictly full triangulated subcategory, the last row of the diagram implies by Proposition~\ref{fulltriangles} that if any two of $C^\prime, C, C^{\prime\prime}$ are objects of $\T^\prime$, then so is the third.
      Therefore if any two of 
      $f \in \operatorname{Mor}_{\T^\prime}(X,Y)$,
      $g \in \operatorname{Mor}_{\T^\prime}(Y,Z)$,
      $g \circ f \in \operatorname{Mor}_{\T^\prime}(X,Z)$
      hold, then so does the third.
    \end{enumerate}
  \end{proof}
\end{lem}

\begin{defn}
  By the previous Lemma, we are justified in defining the subcategory $\operatorname{Mor}_{\T^\prime}$ to be the category with objects those of $\T$ and morphisms $\operatorname{Mor}_{\T^\prime}(X,Y)$ for any pair of objects $X$ and $Y$ of $\T$.
\end{defn}

\begin{defn}
  Let $\T$ be a triangulated category and let $\T^\prime$ be a strictly full triangulated subcategory.
  Fix objects $X$ and $Y$ of $\T$.
  Define a relation on diagrams $(Z,f,g)$ of the form 
  $$\begin{tikzcd}
    & Z\arrow{rd}{f}\arrow{ld}{g}\\
    X & & Y
  \end{tikzcd}$$
  with $f \in \operatorname{Mor}_{T^\prime}(Z,X)$,
  $$(Z,f,g) \sim (Z^\prime, f^\prime, g^\prime)$$
  if and only if there exists a third diagram $(Z^{\prime\prime}, f^{\prime\prime}, g^{\prime\prime})$ and a commutative diagram
  $$\begin{tikzcd}
    & Z\arrow[swap]{ld}{f}\arrow{rd}{g}\\
    X & Z^{\prime\prime} \arrow[swap]{l}{f^{\prime\prime}}\arrow{r}{g^{\prime\prime}}\arrow{u}{u}\arrow[swap]{d}{v} & Y\\
    & Z^\prime\arrow{ul}{f^\prime}\arrow[swap]{ur}{g^\prime}
  \end{tikzcd}$$
\end{defn}

\begin{lem}
  The relation defined above is an equivalence relation.
  
  \begin{proof}
    It's clear that the relation is reflexive and symmetric.
    To see that it's transitive, assume that we have the relations $(Z_1,f_1,g_1) \sim (Z_2, f_2,g_2) \sim (Z_3,f_3,g_3)$
    given by the commutative diagrams
    $$\begin{tikzcd}
      & Z_1\arrow[swap]{ld}{f_1}\arrow{rd}{g_1}\\
      X & Z \arrow[swap]{l}{f}\arrow{r}{g}\arrow{u}{u}\arrow[swap]{d}{v} & Y\\
      & Z_2\arrow{ul}{f_2}\arrow[swap]{ur}{g_2}
    \end{tikzcd}
    \ \text{and}\ 
    \begin{tikzcd}
      & Z_2\arrow[swap]{ld}{f_2}\arrow{rd}{g_2}\\
      X & Z^\prime \arrow[swap]{l}{f^\prime}\arrow{r}{g^\prime}\arrow{u}{u^\prime}\arrow[swap]{d}{v^\prime} & Y\\
      & Z_3\arrow{ul}{f_3}\arrow[swap]{ur}{g_3}
    \end{tikzcd}$$
    Take the homotopy pullback
    $$\begin{tikzcd}
      Z^{\prime\prime} \arrow{r}\arrow{d} & Z\arrow{d}{v}\\
      Z^\prime \arrow{r}{u^\prime} & Z_2.
    \end{tikzcd}$$
    which induces a commutative diagram
    $$\begin{tikzcd}
      & & Z_1 \arrow{lldd}{f_1}\arrow{rrdd}{g_1}\\
      & & Z\arrow{u}{u}\arrow{ld}{v}\arrow{rd}{v}\\
      X & Z_2\arrow{l}{f_2} & Z^{\prime\prime}\arrow{u}\arrow{d} & Z_2\arrow{r}{g_2} & Y\\
      & & Z^\prime\arrow{lu}{u^\prime}\arrow{ru}{u^\prime}\arrow{d}{v^\prime}\\
      & & Z_3\arrow{lluu}{f_3} \arrow{rruu}{g_3}
    \end{tikzcd}$$
    By Lemma~\ref{moriscat} that $u \in \Mor{\T^\prime}(Z,Z_1)$, $u^\prime \in \Mor{\T^\prime}(Z^\prime, Z_2)$, $v \in \Mor{\T^\prime}(Z,Z_2)$, and $V^\prime \in \Mor{T^\prime}(Z^\prime,Z_3)$ because $f \in \Mor{\T^\prime}(Z,X)$, $f^\prime \in \Mor{\T^\prime}(Z^\prime,X)$, $g \in \Mor{\T^\prime}(Z,Y)$ and $g^\prime \in \Mor{\T^\prime}(Z^\prime,Y)$.
    
  \end{proof}
\end{lem}

\begin{thm}
%  Consider the category with objects triangulated categories and morphisms triangulated functors.
%  If $\T$ is a triangulated category and $\T^\prime$ is a strictly full subcategory, then the inclusion functor $\T^\prime \rightarrow \T$ has a cokernel.
  Let $\T$ be a triangulated category.
  If $\T^\prime$ a strictly full triangulated subcategory of $\T$, then there exists a triangulated category $\T/\T^\prime$ and a universal triangulated functor $\pi \colon \T \rightarrow \T/\T^\prime$ such that if $K \rightarrow \T$ is the kernel of $\pi$, then $\T^\prime$ is a subcategory of $K$.

  That is, for any triangulated functor $\F : \T \rightarrow \T^{\prime\prime}$ making the diagram
  $$\begin{tikzcd}
    \T^\prime \arrow{r}\arrow[bend left]{rr}{0} & \T \arrow{d}{\pi}\arrow{r}{\F} & \T^{\prime\prime}\\
    & \T/\T^\prime \arrow[dashed]{ur}{\exists !\overline{\F}}
  \end{tikzcd}$$
  commute, there exists a unique triangulated functor $\overline{\F}$ such that $\F = \overline{\F} \circ \pi$.
  
  The category $\T/\T^\prime$ is called the Verdier quotient of $\T$ by $\T^\prime$ and $\pi : \T \rightarrow \T^\prime$ is called the Verdier localisation.
\end{thm}

\begin{rmk}
  Essentially, this says that in the category with objects triangulated categories and morphisms triangulated functors, the inclusion functor of a strictly full subcategory has a cokernel.
\end{rmk}
\end{document}
