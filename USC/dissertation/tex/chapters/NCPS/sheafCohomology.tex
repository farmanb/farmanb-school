The funtor \(Q\) admits a more geometrically pleasing interpretation, which will serve to help interpret the somewhat onerous conditions in the sequel.
We will often refer to the image of \(A\) in \(\QGr{A}\) as \(\mathcal{O}_X\), thinking of this as the structure sheaf on the noncommutative projective scheme \(X\).
Following \parencite{AZ94}, one defines sheaf cohomology of a quasi-coherent sheaf \(\mathcal{M} = \pi{M}\) to be
\[\underline{H}^i(\mathcal{M}) := \EXT^i_{\QGr{A}}(\mathcal{O}_X, \mathcal{M})\]
and the un-graded sheaf cohomology by
\[H^i(\mathcal{M}) := \underline{H}^i(\mathcal{M})_0.\]

For the Ext-computations, generally one takes an injective resolution \(I\) of \(\omega\mathcal{M}\) in \(\Gr{A}\) then computes
\[\underline{H}^i(\mathcal{M}) = H^i\QGR{A}(\mathcal{O}_X, \pi{I}) \cong H^i\GR{A}(A, QI) \cong H^i(QI) \cong \mathbf{R}^iQ(M).\]
In some sense, the functor \(Q\) should therefore be like the usual global sections functor.

On the other hand, one can also give more explicit descriptions of \(Q\) and \(\tau\). 

\begin{proposition}\label{prop: explicit Q and tau}
  Let \(A\) be a finitely generated connected graded \(k\)-algebra and let \(M\) be a graded \(A\)-module. Then 
  \begin{align*}
    \tau M & = \op{colim}_n \GR{A}(A/A_{\geq n}, M) \\
    Q M & = \op{colim}_n \GR{A}(A_{\geq n}, M).
  \end{align*}
\end{proposition}

\begin{proof}
  This is standard localization theory, see \parencite{Stenstrom75}.
\end{proof}
