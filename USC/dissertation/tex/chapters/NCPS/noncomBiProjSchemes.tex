In studying questions of kernels and bimodules, we will have to move outside the realm of \(\Z\)-gradings. While one can generally treat \(G\)-graded \(k\)-algebras in our analysis, we limit the scope a bit and only consider \(\Z^2\)-gradings of the following form.

\begin{definition}
  Let \(A\) and \(B\) be connected graded \(k\)-algebras. The tensor product \(A \otimes_k B\) will be equipped with its natural bi-grading 
  \begin{displaymath}
    (A \otimes_k B)_{n_1,n_2} = A_{n_1} \otimes_k B_{n_2}. 
  \end{displaymath}
  A \textbf{bi-bi module} for the pair \((A,B)\) is a \(\Z^2\)-graded \(A \otimes_k B\) module. 
\end{definition}

\begin{remark}
  As noted in the remarks above \parencite[Lemma 4.1]{VdB}, the notion of \(A\)-torsion and \(B\)-torsion bi-bi modules is well-defined provided that \(A\) and \(B\) are finitely generated as \(k\)-algebras.
  From this point on, unless stated otherwise, all of our \(k\)-algebras will be assumed to be finitely generated.
\end{remark}

%We make note of the following special \(\Z\)-graded subring of \(A \otimes_k B\)
%\begin{definition}\label{def: segre product}
%  Let \(A\) and \(B\) be connected graded \(k\)-algebras.
%  The \textbf{Segre product} of \(A\) and \(B\) is the graded \(k\)-algebra
%  \[ A \times_k B = \bigoplus_{0 \leq i} A_i \otimes_k B_i.\]
%\end{definition}

%\begin{proposition}\label{proposition: segre product of generated in degree 1 is generated in degree 1}
%  If \(A\) and \(B\) are connected graded \(k\)-algebras that are finitely generated in degree one, then \(A \times_k B\) is finitely generated in degree one.
%\end{proposition}

%\begin{proof}
%  Let \(S = \{x_i\}_{i = 1}^r \subseteq A_1\) and \(T = \{y_i\}_{i = 1}^s \subseteq B_1\) be generators.
%  Take a homogenous element \(a \otimes b \in A_d \otimes_k B_d\).
%  We can write
%  \[a = \sum_{i = 1}^m \alpha_i X_1^{(i)} \cdots X_d^{(i)}\ \text{and}\ b = \sum_{j = 1}^n \beta_j Y_1^{(j)} \cdots Y_d^{(j)}\]
%  for \(\alpha_i,\beta_j \in k\), \((X_1^{(i)}, \ldots, X_d^{(i))} \in \prod_{i = 1}^d S\), and \((Y_1^{(j)}, \ldots, Y_d^{(j)}) \in \prod_{i = 1}^d T\).
  %\[a = \sum_{i = 1}^m \alpha_i x_1^{e_{1,i}} \cdots x_r^{e_{r,i}}\ \text{and}\ b = \sum_{j = 1}^n \beta_j y_1^{f_{1,i}} \cdots y_r^{f_{r,i}}\]
  %where for each \(i\) and each \(j\) we have \(\alpha_i, \beta_j \in k\), and
  %\[\sum_{\ell=1}^r e_{\ell,i} = d\ \text{and}\ \sum_{\ell=1}^s f_{\ell,j} = d.\]
%  Hence we have
%  \begin{eqnarray*}
%    a \otimes b &=& \left(\sum_{i=1}^m \alpha_i X_1^{(i)} \cdots X_d^{(i)}\right) \otimes \left(\sum_{j = 1}^n \beta_j Y_1^{(j)} \cdots Y_d^{(j)}\right)\\
%    &=& \sum_{i = 1}^m\left(\alpha_i X_1^{(i)} \cdots X_d^{(i)} \otimes \left(\sum_{j = 1}^n \beta_j Y_1^{(j)} \cdots Y_d^{(j)}\right)\right)\\
%    &=& \sum_{i = 1}^m\left(\sum_{j = 1}^n\left(\alpha_i  X_1^{(i)} \cdots X_d^{(i)} \otimes \beta_j Y_1^{(j)} \cdots Y_d^{(j)}\right)\right)\\
%    &=& \sum_{i,j} \alpha_i\beta_j (X_1^{(i)} \otimes Y_1^{(j)}) \cdots (X_d^{(i)} \otimes Y_d^{(j)})
%  \end{eqnarray*}
%    Therefore \(A \times_k B\) is finitely generated in degree one by \(\{x_i \otimes y_j\}_{i,j}\).
  %\begin{eqnarray*}
  %a \otimes b &=& \sum_{i = 1}^m \alpha_i x_1^{e_{1,i}} \cdots x_r^{e_{r,i}} \otimes \sum_{j = 1}^n \beta_j y_1^{f_{1,i}} \cdots y_r^{f_{r,i}}\\
  %&=& \sum_{i = 1}^m \left(\alpha_i x_1^{e_{1,i}} \cdots x_r^{e_{r,i}} \otimes \sum_{j = 1}^n \beta_j y_1^{f_{1,i}} \cdots y_r^{f_{r,i}}\right)\\
  %&=& \sum_{i = 1}^m \left(\sum_{j = 1}^n \alpha_i\beta_j \left(x_1^{e_{1,i}} \cdots x_r^{e_{r,i}} \otimes y_1^{f_{1,i}} \cdots y_r^{f_{r,i}}\right)\right)
%\end{eqnarray*}

%  Now, fix \(i\) and \(j\).
 % We can rewrite
 % \[x_1^{e_{1,i}} \cdots x_r^{e_{r,i}} = X_1^{(i)} \cdots X_d^{(i)}\ \text{and}\ y_1^{f_{1,i}} \cdots y_r^{f_{r,i}} = Y_1^{(j)} \cdots Y_d^{(j)}\]
 % with each of the \(X\)'s a generator of \(A\) and each of the \(Y\)'s a generator of \(B\).
 % Then we have
  %\[x_1^{e_{1,i}} \cdots x_r^{e_{r,i}} \otimes y_1^{f_{1,i}} \cdots y_r^{f_{r,i}} = (X_1^{(i)} \otimes Y_1^{(j)}) \cdots (X_d^{(i)} \otimes Y_d^{(j)})\]
%\end{proof}

There are a few notions of torsion for a bi-bi module that one could use, but we take the following.

\begin{definition}
  Let \(A\) and \(B\) be finitely generated, connected graded \(k\)-algebras, and let \(M\) be a bi-bi \(A\)-\(B\) module. We say that \(M\) is \textbf{torsion} if it lies in the smallest Serre subcategory containing \(A\)-torsion bi-bi modules and \(B\)-torsion bi-bi modules.
\end{definition}

\begin{lemma} \label{lemma: alternate char of bibi torsion}
    Let \(A\) and \(B\) be finitely generated, connected graded \(k\)-algebras.
  A bi-bi module \(M\) is torsion if and only if there exists \(n_1,n_2\) such that 
  \begin{displaymath}
    (A \otimes B)_{\geq n_1, \geq n_2} m = 0
  \end{displaymath}
  for all \(m \in M\).
\end{lemma}

\begin{proof}
  For necessity, note that if \(M\) is \(A\)-torsion, then \((A \otimes B)_{\geq n, \geq 0} m = 0\) for some \(n\) for each \(m \in M\).
  Similarly if \(M\) is \(B\)-torsion then \((A \otimes B)_{\geq 0,\geq n} M = 0\) for some \(n\).
  So it suffices to show that if
  \begin{displaymath}
    (A \otimes B)_{\geq n_1, \geq n_2} m = 0 , \forall m \in M
  \end{displaymath}
  then it lies in the Serre category generated by \(A\) and \(B\) torsion. Let \(\tau_B M\) be the \(B\)-torsion submodule of \(M\) and consider the quotient \(M/\tau_B M\). For \(m \in M\), we have \(A_{\geq n_1}m\) is \(B\)-torsion, so its image in the quotient \(M/\tau_B M\) is \(A\)-torsion. Consequently, \(M / \tau_B M\) is \(A\)-torsion itself and \(M\) is an extension of \(B\)-torsion and \(A\)-torsion. 
\end{proof}

One can form the quotient category
\begin{displaymath}
  \QGr{A \otimes_k B} := \Gr{A \otimes_k B} / \Tors{A \otimes_k B}.
\end{displaymath}

\begin{lemma} \label{lemma: biQ and bQGr}
  The quotient functor 
  \begin{displaymath}
    \pi : \Gr{A \otimes_k B} \to \QGr{A \otimes_k B}
  \end{displaymath}
  has a fully faithful right adjoint 
  \begin{displaymath}
    \omega : \QGr{A \otimes_k B} \to \Gr{A \otimes_k B}
  \end{displaymath}
  with 
  \begin{displaymath}
    QM := \omega \pi M = \op{colim}_{n_1,n_2} \GR{(A \otimes_k B)}( A_{\geq n_1} \otimes_k B_{\geq n_2} , M)
  \end{displaymath}
\end{lemma}

\begin{proof}
  %Apply Proposition~\ref{prop: existence of serre functor} and Proposition~\ref{prop: explicit Q and tau}.
  This is just an application of \cite[Cor. 1, III.3]{DCA62}.
\end{proof}

\begin{corollary}\label{corollary: relation on Qs}
  We have an isomorphism 
  \begin{displaymath}
    Q_{A \otimes_k B} \cong Q_A \circ Q_B \cong Q_B \circ Q_A
  \end{displaymath}
\end{corollary}

\begin{proof}
  This follows from Lemma~\ref{lemma: biQ and bQGr} using tensor-Hom adjunction. 
\end{proof}

We also have the following standard triangles of derived functors. 

\begin{lemma} \label{lemma: exact triangles}
    Let \(A\) and \(B\) be finitely generated connected graded algebras. Then, we have natural transformations 
  \begin{displaymath}
    \mathbf{R} \tau \to \op{Id} \to \mathbf{R} Q 
  \end{displaymath}
  which when applied to any graded module \(M\) gives an exact triangle 
  \begin{displaymath}
    \mathbf{R} \tau M \to M \to \mathbf{R} Q M.
  \end{displaymath}
\end{lemma}

\begin{proof}
  Before we begin the proof, we clarify the statement. The conclusions hold for graded \(A\) (or \(B\)) modules and for bi-bi modules. Due to the formal properties, it is economical to keep the wording of the theorem as so since any reasonable interpretation yields a true statement. 
  
  For the case of graded \(A\) modules, this is well-known, see \parencite[Property 4.6]{BVdB}. For the case of bi-bi \(A \otimes_k B\) modules, the natural transformations are obvious. For each \(M\), the sequence 
  \begin{displaymath}
    0 \to \tau M \to M \to Q M
  \end{displaymath}
  is exact. It suffices to prove that if \(M = I\) is injective, then the whole sequence is actually exact. Here one can use the system of exact sequences
  \begin{displaymath}
    0 \to A_{\geq n_1} \otimes_k B_{\geq n_2} \to A \otimes_k B \to (A \otimes_k B) / A_{\geq n_1} \otimes_k B_{\geq n_2} \to 0
  \end{displaymath}
  and exactness of \(\op{Hom}(-,I)\) plus Lemma~\ref{lemma: alternate char of bibi torsion} to get exactness. 
\end{proof}
