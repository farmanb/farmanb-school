\begin{definition}\label{def: segre product}
  Let \(A\) and \(B\) be connected graded \(k\)-algebras.
  The \textbf{Segre product} of \(A\) and \(B\) is the graded \(k\)-algebra
  \[ A \times_k B = \bigoplus_{0 \leq i} A_i \otimes_k B_i.\]
\end{definition}

\begin{proposition}\label{proposition: segre product of generated in degree 1 is generated in degree 1}
  If \(A\) and \(B\) are connected graded \(k\)-algebras that are finitely generated in degree one, then \(A \times_k B\) is finitely generated in degree one.
\end{proposition}

\begin{proof}
  Let \(S = \{x_i\}_{i = 1}^r \subseteq A_1\) and \(T = \{y_i\}_{i = 1}^s \subseteq B_1\) be generators.
  Take a homogenous element \(a \otimes b \in A_d \otimes_k B_d\).
  We can write
  \[a = \sum_{i = 1}^m \alpha_i X_1^{(i)} \cdots X_d^{(i)}\ \text{and}\ b = \sum_{j = 1}^n \beta_j Y_1^{(j)} \cdots Y_d^{(j)}\]
  for \(\alpha_i,\beta_j \in k\), \((X_1^{(i)}, \ldots, X_d^{(i)}) \in \prod_{i = 1}^d S\), and \((Y_1^{(j)}, \ldots, Y_d^{(j)}) \in \prod_{i = 1}^d T\).
  Hence we have
  \begin{eqnarray*}
    a \otimes b &=& \left(\sum_{i=1}^m \alpha_i X_1^{(i)} \cdots X_d^{(i)}\right) \otimes \left(\sum_{j = 1}^n \beta_j Y_1^{(j)} \cdots Y_d^{(j)}\right)\\
    &=& \sum_{i = 1}^m\left(\alpha_i X_1^{(i)} \cdots X_d^{(i)} \otimes \left(\sum_{j = 1}^n \beta_j Y_1^{(j)} \cdots Y_d^{(j)}\right)\right)\\
    &=& \sum_{i = 1}^m\left(\sum_{j = 1}^n\left(\alpha_i  X_1^{(i)} \cdots X_d^{(i)} \otimes \beta_j Y_1^{(j)} \cdots Y_d^{(j)}\right)\right)\\
    &=& \sum_{i,j} \alpha_i\beta_j (X_1^{(i)} \otimes Y_1^{(j)}) \cdots (X_d^{(i)} \otimes Y_d^{(j)})
  \end{eqnarray*}
    Therefore \(A \times_k B\) is finitely generated in degree one by \(\{x_i \otimes y_j\}_{i,j}\).
\end{proof}

As a nice corollary, we can relax the conditions on \textcite[Theorem 2.4]{VR96} to avoid the Noetherian conditions on the Segre and tensor products.

\begin{theorem}[{\textcite[Theorem 2.4]{VR96}}]\label{theorem: Van Rompay}
    Let \(A\) and \(B\) be finitely generated, connected graded \(k\)-algebras, and let \(S = A \times_k B\), \(T = A \otimes_k B\).
    If \(A\) and \(B\) are both generated in degree one, then there is an equivalence of categories
    \[\begin{tikzcd}[row sep=tiny]
    \mathbb{V} \colon \QGr{S} \arrow{r} & \QGr{T}\\
    E \arrow[mapsto]{r} & \pi_{T}\left(T \otimes_S \omega_{S}E\right)
    \end{tikzcd}\]
\end{theorem}

\begin{proof}
  As noted in Van Rompay's comments preceding the Theorem, the hypothesis is necessary only to ensure that \(\QGr{S}\) and \(\QGr{T}\) are well-defined.
  Thanks to Proposition~\ref{proposition: f.g. torsion} and Lemma~\ref{lemma: alternate char of bibi torsion}, the equivalence follows by running the same argument.
\end{proof}
