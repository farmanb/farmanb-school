\chapter{Differential Graded Categories}\label{section: background on dgcats}
Recall that a \textbf{dg-category}, \(\A\), over \(k\) is a category enriched over the category of chain complexes, \(\Ch{k}\), a \textbf{dg-functor}, \(F \colon \A \ra \B\) is a \(\Ch{k}\)-enriched functor, a \textbf{morphism of dg-functors of degree \(n\)}, \(\eta \colon F \ra G\), is a \(\Ch{k}\)-enriched natural transformation such that \(\eta(A) \in \B\left(FA, GA\right)^n\) for all objects \(A\) of \(\mathcal \A\), and a \textbf{morphism of dg-functors} is a degree zero, closed morphism of dg-functors.
We will denote by \(\dgcat{k}\) the 2-category of small \(\Ch{k}\)-enriched categories, and by \(\DGCAT{k}(\A,\B)\) the dg-category of dg-functors from \(\A\) to \(\B\).

Recall also that for \(\A\) and \(\B\) small dg categories, we may define a dg-category \(\A \otimes \B\) with objects \(\op{ob}(\A) \times \op{ob}(\B)\) and morphisms
\[(\A \otimes \B)\left((X,Y), (X^\prime,Y^\prime)\right) = \A(X,X^\prime) \otimes_k \B(Y, Y^\prime).\]
It is well known that there is an isomorphism
\[\dgcat{k}(\A \otimes \B, \C) \cong \dgcat{k}(\A, \DGCAT{k}(\B, \C)),\]
endowing \(\dgcat{k}\) with the structure of a symmetric monoidal closed category.

For any dg-category, \(\A\), we denote by \(Z^0(\A)\) the category with objects those of \(\A\) and morphisms
\[Z^0(\A)(A_1, A_2) := Z^0(\A(A_1,A_2)).\]
By \(H^0(\A)\) we denote the category with objects those of \(\A\) and morphisms
\[H^0(\A)(A_1, A_2) := H^0(\A(A_1,A_2)).\]
Following \parencite{CS15}, we say that two objects \(A_1\), \(A_2\) of a dg-category, \(\A\), are \textbf{dg-isomorphic} (respectively, \textbf{homotopy equivalent}) if there is a morphism \(f \in Z^0(\A)(A_1, A_2)\) such that \(f\) (respectively, the image of \(f\) in \(H^0(\A)(A_1, A_2)\)) is an isomorphism.
In such a case, we say that \(f\) is a \textbf{dg-isomorphism} (respectively, \textbf{homotopy equivalence}).

