In this section, we recall the definition of pretriangulated differential graded categories and provide a useful tool for proving that a dg-functor is a quasi-equivalence.

\begin{definition}
  We say that a dg-category, \(\A\), is pretriangulated if
  \begin{enumerate}[(i)]
  \item
    for all objects \(A\) of \(\A\) and for all integers \(n\) there exists an object \(A[n]\) representing the functor \(h_A[n]\), and
  \item
    for each morphism \(f \in Z^0(\A(A_1, A_2))\) there exists an object \(\op{cone}(f)\) representing the pointwise cone functor
    \[\op{cone}(f_\ast)(A) = \op{cone}\left(\A(A,A_1) \overset{f_\ast(A)}\to \A(A,A_2)\right)\]
  \end{enumerate}
  In this case, the Yoneda embedding descends to a triangulated functor
  \[H^0(Y_\A) \colon H^0(\A) \to H^0(\dgMod{A}).\]
\end{definition}

The following result will prove remarkably useful throughout.
\begin{lemma}[{\textcite[Lemma 2.2.1]{SchwedeShipley}}]
  Let \(\D\) be a triangulated category with coproducts and let \(\mathcal{K}\) be a set of compact objects.
  Then the following are equivalent:
  \begin{enumerate}[(i)]
  \item
    the smallest triangulated subcategory of \(\D\) containing \(\mathcal{K}\) that is closed under coproducts is \(\D\) itself,
  \item
    An object \(D\) of \(\D\) is trivial if and only if \(\D(K, X[n]) = 0\) for all objects \(K\) of \(\mathcal{K}\) and all integers \(n\).
  \end{enumerate}
\end{lemma}

As a first application, we record a handy proposition.  It is suspected that this is well known, but satisfactory references in the literature seem difficult to find.

\begin{proposition} \label{proposition: quasi-equivalence on gens is quasi-equivalence}
  Let \(\A\) and \(\B\) be pretriangulated dg-categories.
  Assume that \(H^0(\A)\) and \(H^0(\B)\) each have a set of compact generators, \(\{A_i\}_I\) and \(\{B_j\}_J\).
  If \(F \colon \A \to \B\) is a continuous dg-functor satisfying \(F(\{A_i\}_I) = \{B_j\}_J\) and the structure morphism
  \[F_{A_{i_1}, A_{i_2}} \colon \A(A_{i_1}, A_{i_2}) \to \B(FA_{i_1}, FA_{i_2})\]
  is a quasi-isomorphism for all \(i_1, i_2 \in I\), then \(F\) is a quasi-equivalence.
\end{proposition}

\begin{proof}
  We observe that it suffices to show that \(F\) is quasi-fully faithful.
  Indeed, if \(F\) is quasi-fully faithful, then the essential image of \(H^0(\A)\) under \(H^0(F)\) is a triangulated subcategory of \(H^0(\B)\) that is closed under coproducts and contains the generators \(\{B_j\}_J\) by assumption.
  Since \(H^0(\B)\) is the smallest such triangulated subcategory, it follows that the essential image of \(H^0(\A)\) is all of \(H^0(\B)\) and thus \(F\) is quasi-essentially surjective.

  We break the argument into two pieces.
  The proof of each case is similar in style to the proof that \(F\) is quasi-essentially surjective above.
  In the first case, we show that the full dg-subcategory, \(\C\), of \(\A\) consisting of objects \(C\) such that
  \[F_{A_i,C} \colon \A(A_i, C) \to \B(FA_i, FC)\]
  is a quasi-isomorphism for all \(i \in I\) satisfies \(H^0(\C) = H^0(\A)\), so that, being a full dg-subcategory of \(\A\) with the same objects as \(H^0(\C)\), \(\C = \A\).
  Having established this, we obtain a non-trivial full dg-subcategory, \(\D\), of \(\A\) consisting of objects \(D\) such that
  \[F_{D,X} \colon \A(D,X) \to \B(FD, FX)\]
  is a quasi-isomorphism for all objects \(X\) of \(\A\).
  Once again we show that the subcategory \(H^0(\D) = H^0(\A)\).
  By the same argument, mutatis mutandis, this implies that \(\D = \A\) and \(F\) is quasi-fully faithful.

  Towards the first goal, we note that it suffices to show \(H^0(\C)\) is triangulated, closed under coproducts, and contains \(\{A_i\}_I\).
  The latter condition is guaranteed by hypothesis.
  That \(H^0(\C)\) is closed under translation follows from the pretriangulated structure.  Indeed, for any integer \(n\), any \(i \in I\), and any object \(C\) of \(\C\) we have the isomorphisms
  \[H^0(\A(A_i, C[n])) \cong H^0(\A(A_i, C)[n]) \cong H^n(\A(A_i, C))\]
  and, similarly, \(H^0(\B(FA_i, FC[n])) \cong H^n(\B(FA_i, FC))\).
  Now, for any distinguished triangle \(C_1 \to C_2 \to X \to C_1[1]\) of \(H^0(\A)\) with \(C_1, C_2\) objects of \(\C\) we see that \(X\) is an object of \(\C\) by applying the Five Lemma to the morphism of long exact sequences induced by the homological functors \(h^0_{A_i}(-) := H^0(\A)(A_i, -)\) and \(h^0_{FA_i}(-) := H^0(\B)(FA_i, -)\)
  \[\begin{tikzcd}[column sep=tiny]
  \cdots \arrow{r} & h^0_{A_i}(C_1) \arrow{r}\arrow{d}{H^0(F_{A_i,C_1})} & h^0_{A_i}(C_2) \arrow{r}\arrow{d}{H^0(F_{A_i,C_2})} & h^0_{A_i}(X) \arrow{r}\arrow{d}{H^0(F_{A_i,X})} & h^1_{A_i}(C_1) \arrow{r}\arrow{d}{H^1(F_{A_i, C_1})} & h^1_{A_i}(C_2) \arrow{r}\arrow{d}{H^1(F_{A_i,C_2})} & \cdots\\
  \cdots \arrow{r} & h^0_{FA_i}(FC_1) \arrow{r} & h^0_{FA_i}(FC_2) \arrow{r} & h^0_{FA_i}(FX) \arrow{r} & h^1_{FA_i}(FC_1) \arrow{r}& h^1_{FA_i}(FC_2)\arrow{r}&\cdots
  \end{tikzcd}\]
  for each \(i \in I\).
  Hence by equipping \(H^0(\C)\) with the distinguished triangles from \(H^0(\A)\) of the form \(C_1 \to C_2 \to C_3 \to C_1[1]\) with the \(C_i\) objects of \(\C\), \(H^0(\C)\) inherits the structure of a triangulated subcategory.
  Finally we note that because \(A_i\) and \(FA_i \in \{B_j\}_J\) are compact, and the induced functor \(H^0(F)\) commutes with direct sums, we have for any set, \(\{C_\alpha\}\), of objects of \(\C\) the isomorphism
  \begin{gather*}
    H^0\left(\A\left(A_i, \bigoplus_\alpha C_\alpha\right)\right) \cong \bigoplus_\alpha H^0\left(\A\left(A_i, C_\alpha\right)\right) \cong \bigoplus_\alpha H^0\left(\B\left(FA_i, C_\alpha\right)\right)\\
    \cong H^0\left(\B\left(FA_i, \bigoplus_\alpha FC_\alpha\right)\right) \cong H^0\left(\B\left(FA_i, F\left(\bigoplus_\alpha C_\alpha\right)\right)\right)
  \end{gather*}
  implies that \(H^0(\C)\) is closed under coproducts.

  To see that \(\D = \A\), we again observe that it suffices to show \(H^0(\D)\) is triangulated, closed under coproducts, and contains the generators, \(\{A_i\}_I\).
  The latter condition follows from the fact that the category \(\C\) contains \(\{A_i\}_I\).
  For any object \(D\) of \(\D\) and any object \(X\) of \(\A\), the fact that translation is an auto-equivalence yields the natural isomorphisms
  \[\A(D[n], X) \cong \A(D, X[-n])\ \text{and}\ \B(FD[n], FX) \cong \B(FD, FX[-n])\]
  from which we obtain the isomorphism
  \[H^0(\A(D[n], X)) \cong H^{-n}(\A(D,X)) \cong H^{-n}(\B(FD, FX)) \cong H^0(\B(FD[n], FX))\]
  for all \(n\).
  Hence \(H^0(\D)\) is closed under translations.
  Next we see that for any set of objects \(\{D_\alpha\}\) of \(\D\) and any object \(X\) of \(\A\) we have the isomorphism
  \begin{gather*}
    H^0\left(\A\left(\bigoplus_\alpha D_\alpha, X\right)\right)
  \cong \prod_\alpha H^0\left(\A\left(D_\alpha, X\right)\right)
    \cong \prod_\alpha H^0\left(\B\left(FD_\alpha, X\right)\right)\\
    \cong H^0\left(\B\left(\bigoplus_\alpha FD_\alpha, X\right)\right)
    \cong H^0\left(\B\left(F\left(\bigoplus_\alpha D_\alpha\right), X\right)\right)
  \end{gather*}
  which implies that \(H^0(\D)\) is closed under coproducts.
  Finally, for any distinguished triangle \(D_1 \to D_2 \to Z \to D_1[1]\) of \(H^0(\A)\) with \(D_1, D_2\) objects of \(\D\) we see that \(Z\) is an object of \(\D\) by applying the Five Lemma to the morphism of long exact sequences induced by the cohomological functors \(h^X_0(-) := H^0(\A)(-, X)\) and \(h^{FX}_0(-) := H^0(\B)(-, FX)\)
  \[\begin{tikzcd}[column sep=tiny]
  \cdots \arrow{r} & h^X_0(D_2) \arrow{r}\arrow{d}{H^0(F_{D_2,X})} & h^X_0(D_1) \arrow{r}\arrow{d}{H^0(F_{D_1,X})} & h^X_1(Z) \arrow{r}\arrow{d}{H^1(F_{Z,X})} & h^X_1(D_2) \arrow{r}\arrow{d}{H^1(F_{D_2,X})} & h^X_1(D_1) \arrow{d}{H^1(F_{D_1,X})}\arrow{r}& \cdots\\
  \cdots \arrow{r} & h^{FX}_0(FD_X) \arrow{r} & h^{FX}_0(FD_1) \arrow{r} & h^{FX}_1(FZ) \arrow{r} & h^{FX}_1(FD_2) \arrow{r} & h^{FX}_1(FD_1)\arrow{r} & \cdots
  \end{tikzcd}\]
  %\[\begin{tikzcd}[column sep=tiny]
  %\cdots \arrow{r} & H^0(\A(D_2, X)) \arrow{r}\arrow{d}{H^0(F_{D_2,X})} & H^0(\A(D_1, X)) \arrow{r}\arrow{d}{H^0(F_{D_1,X})} & H^1(\A(Z, X)) \arrow{r}\arrow{d}{H^1(F_{Z,X})} & H^1(\A(D_2, X)) \arrow{r}\arrow{d}{H^1(F_{D_2,X})} & \cdots\\
  %\cdots \arrow{r} & H^0(\B(FD_2, FX)) \arrow{r} & H^0(\B(FD_1, FX)) \arrow{r} & H^1(\B(FZ, FX)) \arrow{r} & H^1(\B(FD_2, FX)) \arrow{r}& \cdots\\
  %\end{tikzcd}\]
  for each \(i \in I\).
  Hence by equipping \(H^0(\D)\) with the distinguished triangles
  \[D_1 \to D_2 \to D_3 \to D_1[1]\]
  of \(H^0(\A)\), where the \(D_i\) are objects of \(\D\), inherits the structure of a triangulated subcategory.
  Therefore \(\D = \A\) and \(F\) is quasi-fully faithful, as desired.
\end{proof}
