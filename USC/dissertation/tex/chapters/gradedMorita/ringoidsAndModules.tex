Though these results are stated in fuller generality, in the sequel we will generally be concerned only with the groups \(\Z\) and \(\Z^2\).
We begin our adaptation with our notion of ringoids with multiple objects.

\begin{definition}
  To a \(G\)-graded \(k\)-algebra, \(A\), associate the category \(\A\) with objects the group \(G\), morphisms given by
  \[\A(g_1, g_2) = A_{g_2 - g_1},\]
  and composition defined by the multiplication \(A_{g_2 - g_1}A_{g_3 - g_2} \subseteq A_{g_3 - g_1}\).
\end{definition}

The category $\A$ is naturally enriched over $\Mod{k}$.
However, since we wish to deal with chain complexes, we will upgrade our enriching category to the category of chain complexes by viewing modules as chain complexes concentrated in degree zero.
In particular, we regard \(\A\) as a dg-category by considering the \(k\)-module of morphisms as the complex
$$\A(g_1,g_2)^n = \left\{
\begin{array}{ll}
  A_{g_2 - g_1} & \text{if}\ n = 0,\\
  0 & \text{else}.
\end{array}
\right.$$
with zero differential.
From this point on, whenever we speak of modules, we will mean the full subcategory of the functor category $\op{Fun}(\A^\opp,\CH{k})$ consisting of $\Ch{k}$-enriched functors, which we denote by $\dgMod{\A}$.

As an unfortunate side effect of considering chain complexes of graded modules, there will be many instances where there are two simultaneous gradings on an object: homological degree and homogenous degree. 
We avoid the latter term, preferring weight, and use degree solely when referring to homological degree.

For clarity, consider the example of a complex of \(G\)-graded left \(A\)-modules, \(M\).
The degree \(n\) piece of \(M\) is the \(G\)-graded left \(A\)-module \(M^n\).
The weight \(g\) piece of the graded module \(M^n\) is the \(A_0\)-module of homogenous elements of (graded) degree \(g\), \(M^n_g\).
Note that in this terminology, the usual morphisms of graded modules are the weight zero morphisms.

As mentioned above, we have a natural enrichment of the category of chain complexs of graded modules over a graded ring.

\begin{definition}
  Denote by \(\ldgGrMod{A}\) the dg-category with objects chain complexes of \(G\)-graded left \(A\)-modules and morphisms defined as follows.
  
  We say that a morphism \(f \colon M \to N\) of degree \(p\) is a collection of morphisms
  \[f^n \colon M^n \to N^{n+p}\]
  of weight zero.
  We denote by \(\ldgGrMod{A}\left(M,N\right)^p\) the collection of all such morphisms, which we equip with the differential
  \[d(f) = d_N \circ f + (-1)^{p+1}f \circ d_M\]
  and define \(\ldgGrMod{A}(M,N)\) to be the resulting chain complex.
  Composition is the usual composition of graded morphisms.

  We denote by \(\rdgGrMod{A}\) the same construction with \(G\)-graded right \(A\)-modules, which are equivalently left modules over the opposite ring, \(A^\opp\).
\end{definition}

\begin{remark}
  One should note that the closed morphisms are precisely the morphisms of complexes \(M \to N[p]\) and, in particular, the closed degree zero morphisms are precisely the usual morphisms of complexes.
\end{remark}

The following lemma illustrates that modules and chain complexes are one and the same.

\begin{lemma}\label{lem:GrModAsMod}
  Let \(G\) be an abelian group.
  If \(A\) is a \(G\)-graded algebra over \(k\) and \(\A\) the associated dg-category, then there is an isomorphism of dg-categories
  \[\ldgGrMod{A} \cong \dgMod{\A}.\]

  \begin{proof}
    We first construct a dg-functor \(F \colon \ldgGrMod{A} \to \dgMod{\A}\).
    For each \(g \in G\), denote by \(A(g)[0]\) the complex with \(A(g)\) in degree zero and consider the full subcategory of \(\ldgGrMod{A}\) of all such complexes.
    We see that a morphism \[f \in \ldgGrMod{A}(A(g)[0],M)^n\] is just the data of a morphism \(f^0 \colon A(g) \to M^n\) which gives
    \[\ldgGrMod{A}(A(g)[0],M)^n \cong \Gr{A}(A(g), M^n) \cong M^n_{-g}\]
    and hence \(M_{-g} := \ldgGrMod{A}(A(g)[0], M)\) is the complex with \(M^n_{-g}\) in degree \(n\).
    In particular, when \(M = A(h)[0]\), we have
    %\[\ldgGrMod{A}(A(g)[0],A(h)[0]) \cong \Gr{A}(A(g), A(h)) \cong A_{h - g} = \A(g,h),\]
    \[\ldgGrMod{A}(A(g)[0],A(h)[0]) := A(h)[0]_{-g} = \A(g,h),\]
    which allows us to identify this subcategory with \(\A\) via the enriched Yoneda embedding, \(A(h)[0]\) corresponding to the representable functor \(\A(-,h)\).
    Using this identification, we can define the image of \(M\) in \(\dgMod{\A}\) to be the dg-functor that takes an object \(g \in G\) to
    \[M_{-g} = \ldgGrMod{A}(A(g)[0], M)\]
    with structure morphism
    \[\A(g,h) \cong \ldgGrMod{A}(A(g)[0], A(h)[0]) \to \CH{k}(M_{-h}, M_{-g})\]
    induced by the representable functor \(\ldgGrMod{A}(-, M)\).
    We define the image of a morphism \(f \in \ldgGrMod{A}(M,N)\) to be the natural transformation given by the collection of morphisms
    \[h^{A(-g)[0]}(f) \colon \ldgGrMod{A}(A(-g)[0], M) \to \ldgGrMod{A}(A(-g)[0],N)\]
    indexed by \(G\).

    Conversely, we note that the data of a functor \(M \colon \A^\opp \to \CH{k}\) is a collection of chain complexes, \(M_g := M(g)\), indexed by \(G\) and morphisms of complexes
    \[\begin{tikzcd}
    \cdots \arrow{r} & A_{g - h} \arrow{r}\arrow{d} & 0 \arrow{r}\arrow{d} & \cdots\\
    \cdots \arrow{r} & \CH{k}(M_g,M_h)^0 \arrow{r} & \CH{k}(M_g, M_h)^1 \arrow{r} & \cdots
    \end{tikzcd}\]    The non-zero arrow factors through \(Z^0(\CH{k}(M_g, M_h))\), so the structure morphism is equivalent to giving a morphism
    \[A_{g - h} \to \CH{k}(M_g,M_h)\]
    and thus \(M\) determines a complex of graded \(A\)-modules
    \[\widetilde{M} = \bigoplus_{g \in G}M_{-g}.\]
    A morphism \(\eta \colon M \to N\) is simply a collection of natural transformations \(\eta^p\) such that for each \(g \in G\) we have \(\eta^p(g) \in \CH{k}(M_g,N_g)^p\) and the naturality implies that \(\eta^p(g)\) is \(A\)-linear.
    The natural transformation \(\eta^p\) thus determines a morphism
    \[\bigoplus_{g \in G} \eta^p(-g) \in \ldgGrMod{A}\left(\widetilde{M}, \widetilde{N}\right)^p,\]
    and hence \(\eta\) determines a morphism in \(\ldgGrMod{A}\left(\widetilde{M}, \widetilde{N}\right)\), which is the collection of all such homogenous components.
    This defines a dg-functor \(\dgMod{\A} \to \ldgGrMod{A}\) which is clearly the inverse of \(F\).
  \end{proof}
\end{lemma}

\begin{remark}
  It is worth noting that it is natural from the ringoid perspective to reverse the weighting on the opposite ring in that, formally,
  \[A^\opp_{g} = \A^\opp(0,g) = \A(g,0) = A_{-g}\]
  so that \(\A^\opp(-, h) = \A(h,-)\) is the representable functor corresponding to the left module \(A^\opp(h)\) by
  \[\bigoplus_{g \in G} \A^\opp(-g,h) = \bigoplus_{g \in G} \A(h,-g) = \bigoplus_{g \in G} A_{-(g + h)} = \bigoplus_{g \in G} A^\opp_{g + h} = A^\opp(h).\]
  With this convention, when considering right modules, one can dispense with the formality of the opposite ring by constructing from a complex, \(M\), the dg-functor \(\A \to \CH{k}\) mapping \(g\) to \(M_g := \rdgGrMod{A}(A(-g)[0], M)\).
  %  As such, the covariant representable functor corresponding to the left \(A^\opp\)-module, \(A(h)\), is
  %\[g \mapsto \ldgGrMod{A^\opp}(A(-g)[0], A(h)[0]) =: A(h)[0]_g = \A^\opp(g, -h) = \A(-h, g)\]
  %since
  %\[(A^\opp)_i =: \A^\opp(-i,0) = \A(0,-i) = A_{-i}.\]
\end{remark}

When \(G = \Z^2\), and \(A\), \(B\) are \(\Z\)-graded algebras over \(k\), we denote the dg-category of chain complexes of \(G\)-graded \(B\)-\(A\)-bimodules by \(\ldgGrMod{A^\opp \otimes_k B}\).
We associate to the \(\Z^2\)-graded \(k\)-algebra \(A^\opp \otimes_k B\) the tensor product of the associated dg-categories, \(\A^\opp \otimes \B\).
Note that, as in the remark above, in the identification
\[\ldgGrMod{\left(A^\opp \otimes_k B\right)} \cong \dgMod{\A^\opp \otimes \B}\]
the weighting coming from the \(A\)-module structure is reversed.
The representable functors in this case are
\[\A^\opp \otimes \B\left((-,-), (u,v)\right) := \A^\opp(-, u) \otimes_k \B(-,v)\]
and correspond to \((A^\opp \otimes_k B)(u,v) := A^\opp(u) \otimes_k B(v)\) by 
\[\bigoplus_{(x,y) \in \Z^2} \A^\opp \otimes \B\left((-x,-y),(u,v)\right) = \bigoplus_{(x,y) \in \Z^2} A^\opp_{u + x} \otimes_k B_{v + y} = A^\opp(u) \otimes_k B(v)\]


\begin{remark}\label{remark: tensor with twist}
  It is sometimes convenient to note the following.
  Let \(P\) be a chain complex of bi-bi \(A\)-modules.
  If \(\mathcal{P}(m,n) = P_{m,-n}\) is the corresponding dg \(\A\)-\(\A\)-bimodule, then by the construction of the tensor product, it's easy to see that for any \(u\) the tensor product 
  \[\A(-, u) \otimes_\A \mathcal{P} \cong \mathcal{P}(u, -)\]
  corresponds to the chain complex of left \(A\)-modules
  \[\bigoplus_{n \in \Z} \mathcal{P}(u,-n) = \bigoplus_{n \in \Z} P_{u,n} = P_{u, \ast}\]
  We will often identify \(P\) with \(\mathcal{P}\), as well as \(\A(-, u)\) with \(A(u)\), and, under this identification, write \({P \otimes_\A A(u)= P_{u,\ast}}.\)

  Similarly, for any \(v\), if we regard \(A(v)\) as a right \(A\)-module, we will often write \({A(v) \otimes_\A P = P_{\ast,-v}}\) for the chain complex of right \(A\)-modules.
  We remark that as an artifact of the reverse weighting, we can homogenize these formulas by thinking of \(P\) as a left \(A^\opp\)-module, make the formal identification \(A(-v) = A^\opp(v)\) and then
  \[A^\opp(v) \otimes_\A P = A(-v) \otimes_\A P = P_{\ast,v}.\]
\end{remark}
