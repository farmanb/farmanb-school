\chapter{Differential Graded Categories}\label{section: background on dgcats}
Recall that a \textbf{dg-category}, \(\A\), over \(k\) is a category enriched over the category of chain complexes, \(\Ch{k}\), a \textbf{dg-functor}, \(F \colon \A \ra \B\) is a \(\Ch{k}\)-enriched functor, a \textbf{morphism of dg-functors of degree \(n\)}, \(\eta \colon F \ra G\), is a \(\Ch{k}\)-enriched natural transformation such that \(\eta(A) \in \B\left(FA, GA\right)^n\) for all objects \(A\) of \(\mathcal \A\), and a \textbf{morphism of dg-functors} is a degree zero, closed morphism of dg-functors.
We will denote by \(\dgcat{k}\) the 2-category of small \(\Ch{k}\)-enriched categories, and by \(\DGCAT{k}(\A,\B)\) the dg-category of dg-functors from \(\A\) to \(\B\).

Recall also that for \(\A\) and \(\B\) small dg categories, we may define a dg-category \(\A \otimes \B\) with objects \(\op{ob}(\A) \times \op{ob}(\B)\) and morphisms
\[(\A \otimes \B)\left((X,Y), (X^\prime,Y^\prime)\right) = \A(X,X^\prime) \otimes_k \B(Y, Y^\prime).\]
It is well known that there is an isomorphism
\[\dgcat{k}(\A \otimes \B, \C) \cong \dgcat{k}(\A, \DGCAT{k}(\B, \C)),\]
endowing \(\dgcat{k}\) with the structure of a symmetric monoidal closed category.

For any dg-category, \(\A\), we denote by \(Z^0(\A)\) the category with objects those of \(\A\) and morphisms
\[Z^0(\A)(A_1, A_2) := Z^0(\A(A_1,A_2)).\]
By \(H^0(\A)\) we denote the category with objects those of \(\A\) and morphisms
\[H^0(\A)(A_1, A_2) := H^0(\A(A_1,A_2)).\]
Following \parencite{CS15}, we say that two objects \(A_1\), \(A_2\) of a dg-category, \(\A\), are \textbf{dg-isomorphic} (respectively, \textbf{homotopy equivalent}) if there is a morphism \(f \in Z^0(\A)(A_1, A_2)\) such that \(f\) (respectively, the image of \(f\) in \(H^0(\A)(A_1, A_2)\)) is an isomorphism.
In such a case, we say that \(f\) is a \textbf{dg-isomorphism} (respectively, \textbf{homotopy equivalence}).

\section{The Model Structure on DG-Categories}
We collect here some basic results on the model structure for \(\dgcat{k}\).
For any dg-functor \(F \colon \A \to \B\), we say that \(F\) is
\begin{enumerate}[(i)]
\item
  \textbf{quasi-fully faithful} if for any two objects \(A_1\), \(A_2\) of \(\A\) the morphism
  \[F(A_1, A_2) \colon \A(A_1, \A_2) \to \B(FA_1, FA_2)\]
  is a quasi-isomorphism of chain complexes,
\item
  \textbf{quasi-essentially surjective} if the induced functor \(H^0(F) \colon H^0(\A) \to H^0(\B)\) is essentially surjective,
\item
  a \textbf{quasi-equivalence} if \(F\) is quasi-fully faithful and quasi-essentially surjective,
\item
  a \textbf{fibration} if \(F\) satisfies the following two conditions:
  \begin{enumerate}[(a)]
  \item
    for all objects \(A_1, A_2\) of \(\A\), the morphism \(F(A_1,A_2)\) is a degree-wise surjective morphism of complexes, and
  \item
    for any object \(A\) of \(\A\) and any isomorphism \(\eta \in H^0(\B)(H^0(F)A, B)\), there exists an isomorphism \(\nu \in H^0(\C)(A,A^\prime)\) such that \(H^0(F)(\nu) = \eta\).
  \end{enumerate}
\end{enumerate}
In \parencite{Tabuada05} it is shown that taking the class of fibrations defined above and the class of weak equivalences to be the quasi-equivalences, \(\dgcat{k}\) becomes a cofibrantly generated model category.
The localization of \(\dgcat{k}\) at the class of quasi-equivalences is the homotopy category, \(\Ho{\dgcat{k}}\).
We will denote by \([\A,\B]\) the morphisms of \(\Ho{\dgcat{k}}\).

A small dg-category \(\A\) is said to be \textbf{h-projective} if for all objects \(A_1, A_2\) of \(\A\) and any acyclic complex, \(C\), every morphism of complexes \(\A(A_1, A_2) \to C\) is null-homotopic.
In \parencite{CS15}, it is shown that there exists an h-projective category \(A^{\op{hp}}\) quasi-equivalent to \(\A\) and, as a result, the localization of the full subcategory of \(\dgcat{k}\) of h-projective dg-categories at the class of quasi-equivalences is equivalent to \(\Ho{\dgcat{k}}\).
In particular, when \(k\) is a field, every dg-category is h-projective and hence one can compute the derived tensor product by
\[\A \otimes^\mathbf{L} \B = \A^{\op{hp}} \otimes \B = \A \otimes \B.\]

\section{Differential Graded Modules}\label{subsection: dg modules}
Before making the relevant definitions, we pause for a brief justification of the use of the word module.
To a ring $A$, one can associate the $\Ab$-enriched category, \(\A\), with one object, endomorphisms the abelian group $A$, and composition given by multiplication.
We will refer to the category $\A$ as the \textbf{ringoid associated to $A$}.
As one is wont to do in mathematics, we shift perspective by invoking enriched category theory and abstract away to the 2-category, $\op{Ab-cat}$, of all small $\Ab$-enriched categories.
Indeed, it is an easy exercise in translation that one recovers the classical category of $A$-modules as the $\Ab$-enriched category of $\Ab$-enriched functors, $\op{Ab-cat}(\A, \Ab)$.

More generally, for any $\Ab$-enriched category, $\A$, one could reasonably call $\op{Ab-cat}(\A,\Ab)$ the category of $\Ab$-modules over $\A$; the classical $A$-modules could then be regarded as $\Ab$-modules over the ringoid $\A$.
%In this way, one could reasonbly regard the classical notion of modules over a ring as a special case of what one might call $\Ab$-modules over $\A$.
Since these constructions really only rely on the fact that $\Ab$ is a symmetric monoidal closed category, one is naturally led to think about mimicing this construction with another category, $\V$, of the same type.
This of course leads to $\V$-modules over a $\V$-category, $\A$.
As dg-categories are just $\V$-enriched categories for $\V = \Ch{k}$, we adopt the name dg-module.

For any small dg-category, \(\A\), denote
\[\dgMod{\A} := \DGCAT{k}\left(\A^{\opp},\CH{k}\right),\]
the dg-category of dg-functors, where \(\CH{k}\) denotes the dg-category of chain complexes equipped with the internal Hom from its symmetric monoidal closed structure.
The objects of \(\dgMod{\A}\) will be called \textbf{dg \(\A\)-modules}.
Since one may view the dg \(\A^{\opp}\)-modules as what should reasonably be called left dg \(\A\)-modules, the terms right and left will be dropped in favor of dg \(\A\)-modules and dg \(\A^{\opp}\)-modules, respectively.
We note here that the somewhat vexing choice of terminology is such that we can view objects of \(\A\) as dg \(\A\)-modules by way of the enriched Yoneda embedding
\[Y_\A \colon \A \to \dgMod{\A}.\]

Just as one usually calls an abelian group with compatible left $A$-action and right $B$-action an $A$-$B$-module, we define for any two small dg-categories, \(\A\) and \(\B\), the category of dg \(\A\)-\(\B\)-bimodules to be \(\dgMod{\A^{\opp} \otimes \B}\).
We note here that the symmetric monoidal closed structure on \(\dgcat{k}\) allows us to view bimodules as morphisms of dg-categories by the isomorphism
\begin{eqnarray*}
  \dgMod{\A^{\opp} \otimes \B} &=& \DGCAT{k}\left(\A \otimes \B^{\opp}, \CH{k}\right)\\
  &\cong& \DGCAT{k}(\A, \DGCAT{k}\left(\B^{\opp}, \CH{k}\right))\\
  &=& \DGCAT{k}\left(\A, \dgMod{\B}\right).
\end{eqnarray*}
The image of a dg \(\A\)-\(\B\)-bimodule, \(E\), is the dg-functor \(\Phi_E(A) = E(A,-)\).

As a final note, we draw a connection between chain complexes and dg-modules over a ringoid that parallels the discussion of $A$-modules and the so-called $\Ab$-modules above.
Let $A$ be a $k$-algebra and consider the category of chain complexes, $\Ch{A}$.
One can can construct (see, e.g., \parencite{Weibel94}) for any two chain complexes a chain complex of morphisms
\[\CH{A}(C,D)^n = \prod_{m \in \Z}\Mod{A}\left(C^m, D^{m + n}\right)\]
with differential given by
\[d(f) = d_D \circ f + (-1)^{n+1} f \circ d_C.\]
Denoting by $\CH{A}$ the category with objects chain complexes of $A$-modules and morphisms given by this complex, a similar translation shows that this is equivalent to the dg-category $\dgMod{\A}$.

\section{h-Projective DG-Modules}
We say that a dg \(\A\)-module, \(N\), is \textbf{acyclic} if \(N(A)\) is an acyclic chain complex for all objects \(A\) of \(\A\).
A dg \(\A\)-module, \(M\), is said to be \textbf{h-projective} if
\[H^0(\dgMod{\A})(M,N) := H^0(\dgMod{A}(M,N)) = 0\]
for every acyclic dg \(\A\)-module, \(N\).
The full dg-subcategory of \(\dgMod{\A}\) consisting of h-projectives will be called \(\hproj{\A}\).

We always have a special class of h-projectives given by the representables, \(h_A = \A(-, A)\) for if \(M\) is acyclic, then from the enriched Yoneda Lemma we have
\[H^0(\dgMod{A})(h_A,M) := H^0(\dgMod{\A}(h_A, M)) \cong H^0(M(A)) = 0.\]
Noting that closure of \(\hproj{\A}\) under homotopy equivalence follows immediately from the Yoneda Lemma applied to \(H^0(\dgMod{A})\), we define \(\overline{\A}\) to be the full dg-subcategory of \(\hproj{\A}\) consisting of the dg \(\A\)-modules homotopy equivalent to representables.

We will say an h-projective dg \(\A\)-\(\B\)-bimodule, \(E\), is \textbf{right quasi-representable} if for every object \(A\) of \(\A\) the dg \(\B\)-module \(\Phi_E(A)\) is an object of \(\overline{\B}\), and we will denote by \(\hproj{\A^\opp \otimes \B}^\rqr\) the full subcategory of \(\hproj{\A^\opp \otimes \B}\) consisting of all right quasi-representables.

\section{The Derived Category of a DG-Category}
By definition, a degree zero closed morphism
\[\eta \in Z^0(\dgMod{\A})(M,N)\]
satisfies
\[\eta(A) \in Z^0(\CH{k}(M(A), N(A))) = \Ch{k}(M(A), N(A))\]
for all objects \(A\) of \(\A\).
Hence we are justified in the following definitions:
\begin{enumerate}[(i)]
\item
  \(\eta\) is a \textbf{quasi-isomorphism} if \(\eta(A)\) is a quasi-isomorphism of chain complexes for all objects \(A\) of \(\A\), and
\item
  \(\eta\) is a \textbf{fibration} if \(\eta(A)\) is a degree-wise surjective morphism of complexes for all objects \(A\) of \(\A\).
\end{enumerate}
Equipping \(\Ch{k}\) with the standard projective model structure (see \parencite[Section 2.3]{Hovey99}), these definitions endow \(Z^0(\dgMod{\A})\) with the structure of a particularly nice cofibrantly generated model category (see \parencite[Section 3]{Toen07}).
In analogy with the definition of the derived category of modules for a ring \(A\), the \textbf{derived category of \(\A\)} is defined to be the model category theoretic homotopy category,
\[\mathrm{D}(\A) = \Ho{Z^0(\dgMod{\A})} = Z^0(\dgMod{\A})[\mathcal{W}^{-1}]\]
obtained from localizing \(Z^0(\dgMod{\A})\) at the class, \(\mathcal{W}\), of quasi-isomorphisms.

It can be shown (see \parencite[Section 3.5]{Keller95}) that for every dg \(\A\)-module, \(M\), there exists an h-projective, \(N\), and a quasi-isomorphism \(N \to M\), which one calls an \textbf{h-projective resolution of \(M\)}.
Moreover, it is not difficult to see that any quasi-isomorphism between h-projective objects is in fact a homotopy equivalence.
It follows that there is an equivalence of categories between \(H^0(\hproj{\A})\) and \(\mathrm{D}(\A)\) for any small dg-category, \(\A\).

It should be noted that this generalizes the notion of derived categories of modules over a $k$-algebra, $A$.
Making the identification of $\CH{A}$ and $\dgMod{\A}$ as at the end of Section~\ref{subsection: dg modules}, where $\A$ is the ringoid associated to $A$, it is easy to recognize the categories \(Z^0(\dgMod{\A})\), \(H^0(\dgMod{\A})\), and \(\mathrm{D}(\A)\), as the categories \(\Ch{A}\), \(K(A)\), the usual category up to homotopy, and the derived category of \(\Mod{A}\), respectively.
In the language of \parencite{Lunts-Orlov}, \(\hproj{\A}\) is a dg-enhancement of \(\mathrm{D}(\Mod{A})\).


\section{Tensor Products of DG-Modules}

Let \(M\) be a dg \(\A\)-module, let \(N\) be a dg \(\A^{\opp}\)-module, and let \(A, B\) be objects of \(\A\).
For ease of notation, we drop the functor notation \(M(A)\) in favor of \(M_A\) and write \(\A_{A,B}\) for the morphisms \(\A(A, B)\).
We have structure morphisms
\[M_{A,B} \in \CH{k}\left(\A_{A,B}, \CH{k}(M_B, M_A)\right) \cong \CH{k}\left(M_B \otimes_k \A_{A,B}, M_A\right)\]
and
\[N_{A,B} \in \CH{k}\left(\A_{A,B}, \CH{k}(N_A, N_B)\right) \cong \CH{k}\left(\A_{A,B} \otimes_k N_A, N_B\right),\]
which give rise to a unique morphism
\[M_B \otimes_k {A}_{A,B} \otimes_k N_A \ra M_A \otimes_k N_A \oplus M_B \otimes_k N_B\]
induced by the universal properties of the biproduct.
The two collections of morphisms given by projecting onto each factor induced morphisms 
\[\Xi_1, \Xi_2 \colon \bigoplus_{A,B \in \op{Ob}(\A)} M_B \otimes_k \A_{A,B} \otimes_k N_A \ra \bigoplus_{\C \in \op{Ob}(\A)} M_C \otimes_k N_C,\]
and we define the tensor product of \(M\) and \(N\) to be the coequalizer in \(\Ch{k}\)
\[\begin{tikzcd}
\bigoplus_{(i,j) \in \Z^2} M_j \otimes_k \A_{A,B} \otimes_k N_A \arrow[shift left]{r}{\Xi_1} \arrow[shift right,swap]{r}{\Xi_2} & \bigoplus_{\ell \in \Z} M_\ell \otimes_k N_\ell\arrow{r} & M \otimes_\A N
\end{tikzcd}.\]It is routine to check that a morphism \(M \ra M^\prime\) of right dg \(\A\)-modules induces by the universal property for coequalizers a unique morphism
\[M \otimes_\A N \ra M^\prime \otimes_\A N\]
yielding a functor
\[- \otimes_\A N \colon \dgMod{\A} \ra \CH{k}.\]

One extends this construction to bimodules as follows.
Given objects \(E\) of \(\dgMod{\A \otimes \B}\) and \(F\) of \(\dgMod{\B^{\opp} \otimes \C}\), we recall that we have associated to each a dg-functor \(\Phi_E \colon \A^{\opp} \ra \dgMod{\B}\) and \(\Phi_F \colon \C^{\opp} \ra \dgMod{\B^{\opp}}\) by the symmetric monoidal closed structure on \(\dgcat{k}\).
For each pair of objects \(A\) of \(\A\) and \(C\) of \(\C\), we obtain dg-modules
\[\Phi_E(A) = E(A, -) \colon \B^{\opp} \ra \CH{k}\ \text{and}\ \Phi_F(C) = F( -, C) \colon \B \ra \CH{k}\]
and hence one may define the object \(E \otimes_\B F\) of \(\dgMod{\A \otimes \C}\) by
\[\left(E \otimes_\B F\right)(A, C) = \Phi_E(A) \otimes_\B \Phi_F(C).\]
One can show that by a similar argument to the original that a morphism \(E \to E^\prime\) of \(\dgMod{\A \otimes \B}\) induces a morphism \(E \otimes_\B F \to E^\prime \otimes_\B F\) of \(\dgMod{\A \otimes \C}\), and a morphism \(F \to F^\prime\) of \(\dgMod{\B^{\opp} \otimes \C}\) induces a morphism \(E \otimes_\B F \to E \otimes_\B F^\prime\) of \(\dgMod{\A \otimes \C}\).

\begin{remark}\label{rem: tensor over k}
  Denote by \(\K\) the dg-category with one object, \(\ast\), and morphisms given by the chain complex
  \[\K(\ast,\ast)^n =
  \left\{ \begin{matrix}
    k & n = 0\\
    0 & n \neq 0
  \end{matrix}\right.\]  with zero differential.
  This category serves as the unit of the symmetric monoidal structure on \(\dgcat{k}\), so for small dg-categories, \(\A\) and \(\C\), we can always identify \(\A\) with \(\A \otimes \K\) and \(\C\) with \(\K^\opp \otimes \C\).
  With this identification in hand, we obtain from taking \(\B = \K\) in the latter construction a special case:
  Given a dg \(\A^\opp\)-module, \(E\), and a dg \(\C\)-module, \(F\), we have a dg \(\A\)-\(\C\)-bimodule defined by the tensor product
  \[\left(E \otimes F\right)(A,C) := \left(E \otimes_{\K} F\right)(A,C) = E(A) \otimes_k F(C).\]
\end{remark}

\section{Bimodules as Morphisms of Module Categories}
Let \(E\) be a dg \(\A\)-\(\B\)-bimodule.
Following \parencite[Section 3]{CS15}, we can extend the associated functor \(\Phi_E\) to a dg-functor
\[\widehat{\Phi_M} \colon \dgMod{\A} \to \dgMod{\B}\]
defined by \(\widehat{\Phi_E}(M) = M \otimes_\A E\).
Similarly, we have a dg-functor in the opposite direction
\[\widetilde{\Phi_M} \colon \dgMod{\B} \to \dgMod{\A}\]
defined by \(\widetilde{\Phi_M}(N) = \dgMod{\B}(\Phi_M( - ), N)\).

For any dg-functor \(G \colon \A \to \B\) we denote by \(\Ind{G}\) the extension of the dg-functor
\[A \to \B \overset{Y_\B}\to \dgMod{\B}\]
and its right adjoint by \(\Res{G}\).
By way of the enriched Yoneda Lemma we see that for any object \(A\) of \(\A\) and any dg \(\B\)-module, \(N\), 
\[\Res{G}(N)(A) = \dgMod{\B}(h_{GA}, N) \cong N(GA).\]

We record here some useful propositions regarding extensions of dg-functors.

\begin{proposition}[{\parencite[Prop 3.2]{CS15}}]
  Let \(\A\) and \(\B\) be small dg-categories.
  Let \(F \colon \A \to \dgMod{\B}\) and \(G \colon \A \to \B\) be dg-functors.
  \begin{enumerate}[(i)]
  \item
    \(\widehat{F}\) is left adjoint to \(\widetilde{F}\) (hence \(\Ind{G}\) is left adjoint to \(\Res{G}\)),
  \item
    \(\widehat{F} \circ Y_\A\) is dg-isomorphic to \(F\) and \(H^0(\widehat{F})\) is continuous (hence \(\Ind{G} \circ Y_\A\) is dg-isomorphic to \(Y_\B \circ G\) and \(H^0(\Ind{G})\) is continuous),
  \item
    \(\widehat{F}(\hproj{\A}) \subseteq \hproj{\B}\) if and only if \(F(A) \subseteq \hproj{B}\) (hence \(\Ind{G}(\hproj{\A}) \subseteq \hproj{B}\)),
  \item
    \(\Res{G}(\hproj{\B}) \subseteq \hproj{\A}\) if and only if \(\Res{G}(\bar{B}) \subseteq \hproj{\A}\); moreover,\\ \(H^0(\Res{G})\) is always continuous,
  \item
    \(\Ind{G} \colon \hproj{\A} \to \hproj{\B}\) is a quasi-equivalence if \(G\) is a quasi-equivalence.
  \end{enumerate}
\end{proposition}

\begin{remark}\label{rem: tensoring with reps}
  \begin{enumerate}
  \item
    We note that for dg \(\A\)- and \(\A^\opp\)-modules, \(M\) and \(N\), part \((i)\) implies that the dg-functors
    \[- \otimes_\A N \colon \dgMod{\A} \to \CH{k}\ \text{and}\ M \otimes_\A - \colon \dgMod{\A^\opp} \to \CH{k}\]
    have right adjoints
    \[\widetilde{N}(C) = \CH{k}(N( - ), C)\ \text{and}\ \widetilde{M}(C) = \CH{k}(M( - ), C),\]
    respectively.
    As an immediate consequence of the enriched Yoneda Lemma
    \[h_A \otimes_\A N \cong N(A)\ \text{and}\ M \otimes_\A h^A \cong M(A)\]
    holds for any object \(A\) of \(\A\).
  \item
    We denote by \(\Delta_\A\) the dg \(\A\)-\(\A\)-bimodule corresponding to the Yoneda embedding, \(Y_\A\), under the isomorphism
    \[\dgMod{\A^\opp \otimes \A} \cong \DGCAT{k}\left(\A, \dgMod{\A}\right).\]
    It's clear that we have a dg-functor
    \[\Delta_\A \otimes_\A - \colon \dgMod{\A^\opp \otimes \A} \to \dgMod{\A^\opp \otimes \A}\]
    and for any dg \(\A\)-\(\A\)-bimodule, \(E\), we see that
    \[(\Delta_\A \otimes_\A E)(A,A^\prime) = h_A \otimes_\A E(- , A^\prime) \cong E(A,A^\prime)\]
    implies that \(\Delta_\A \otimes_\A E \cong E\).
  \end{enumerate}
\end{remark}

When starting with an h-projective we have a very nice extension of dg-functors:
\begin{proposition}[{\parencite[Lemma 3.4]{CS15}}]
  For any h-projective dg \(\A\)-\(\B\)-bimodule, \(E\), the associated functor
  \[\Phi_E \colon \A \to \dgMod{\B}\]
  factors through \(\hproj{\B}\).
\end{proposition}

As a direct consequence of the penultimate proposition, this means that we can view the extension of \(\Phi_E\) as a dg-functor
\[\widehat{\Phi_E} = - \otimes_\A E \colon \hproj{A} \to \hproj{B}.\]
Put another way, tensoring with an h-projective \(\A\)-\(\B\)-bimodule preserves h-projectives.

One essential result about \(\dgcat{k}\) comes from T\"oen's result on the existence, and description of, the internal Hom in its homotopy category. 

\begin{theorem}[{\parencite[Theorem 1.1]{Toen07}}, {\parencite[4.1]{CS15}}] \label{theorem: Toen}
  Let \(\A\), \(\B\), and \(\C\) be objects of \(\dgcat{k}\).
  There exists a natural bijection
  \[[\A,\C] \overset{1:1}\longleftrightarrow \Iso{H^0(\hproj{\A^{\opp} \otimes \C}^\rqr)}\]
  Moreover, the dg-category \(\RHom{\B,\C} := \hproj{\B^\opp \otimes \C}^\rqr\) yields a natural bijection
  \[[\A \otimes \B, \C] \overset{1:1}\longleftrightarrow [\A, \RHom{\B,\C}]\]
  proving that the symmetric monoidal category \(\Ho{\dgcat{k}}\) is closed.
\end{theorem}

\begin{corollary}[{\parencite[7.2]{Toen07}},{\parencite[Cor. 4.2]{CS15}}] \label{corollary: Toen}
  Given two dg categories \(\A\) and \(\B\), \(\RHom{\A, \hproj{\B}}\) and \(\hproj{\A^\opp \otimes \B}\) are isomorphic in \(\Ho{\dgcat{k}}\).
  Moreover, there exists a quasi-equivalence
  \[\RHomc{\hproj{\A},\hproj{\B}} \to \RHom{\A, \hproj{B}}.\]
\end{corollary}

To get a sense of the value of this result, let us recall one application from \parencite[Section 8.3]{Toen07}. Let \(X\) and \(Y\) be quasi-compact and separated schemes over \(\op{Spec} k\). 
Recall the dg-model for \(\mathrm{D}(\op{Qcoh} X)\), \(\mathcal L_{\op{qcoh}}(X)\), is the \(\mathcal C(k)\)-enriched subcategory of fibrant and cofibrant objects in the injective model structure on \(C(\op{Qcoh} X)\).

\begin{theorem}[{\parencite[Theorem 8.3]{Toen07}}]
  Let \(X\) and \(Y\) be quasi-compact, quasi-separated schemes over \(k\). Then there exists an isomorphism in \(\Ho{\dgcat{k}}\)
  \begin{displaymath}
    \RHomc{\mathcal L_{\op{qcoh}} X,\mathcal L_{\op{qcoh}} Y} \cong \mathcal L_{\op{qcoh}} (X \times_k Y)
  \end{displaymath}
  which takes a complex \(E \in \mathcal L_{\op{qcoh}} (X \times_k Y)\) to the exact functor on the homotopy categories
  \begin{align*}
    \Phi_E : \mathrm{D}(\Qcoh{X}) & \to \mathrm{D}(\Qcoh{Y}) \\
    M & \mapsto \mathbf{R}\pi_{2 \ast} \left( E \overset{\mathbf{L}}{\otimes} \mathbf{L}\pi_1^\ast M \right)
  \end{align*}
\end{theorem}

\begin{proof}
  The first part of the statement is exactly as in \parencite{Toen07}. The second part is implicit. 
\end{proof}
