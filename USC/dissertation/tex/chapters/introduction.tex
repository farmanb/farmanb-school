\chapter*{Introduction}
\subsection*{Fourier-Mukai Kernels for Noncommutative Projective Schemes}

In light of their absense in noncommutative projective geometry, the natural question to ask is what these kernels should be.
To\"en's derived Morita theory \parencite{Toen07} gives an overarching framework to attack such a problem by abstracting to the higher categorical structure of differential graded (dg) categories.
Working within the homotopy category of the 2-category of all small dg-categories over a commutative ring, To\"en is able to provide an incredibly elegant reformulation of Fourier-Mukai functors at the level of pre-triangulated dg-categories via the dg-subcategory, \(\mathbf{R}\underline{\operatorname{Hom}}_c\), of the internal Hom.
Indeed, using this machinery, kernels have been recovered for schemes in \parencite{Toen07}, and obtained for higher derived stacks in \parencite{BFN10} and for categories of matrix factorizations in \parencite{Dyckerhoff11,PV12,BFK14}.
In each case, the work lies in the identification of the internal Hom object obtained from this machinery within the theory from which the input dg-categories originate, for even if they arise geometrically, the resulting Hom is often quite abstract.

The obvious first step in such work is to identify the possible input dg-categories for the machinery of derived Morita theory.
In the situation of interest, one considers the noncommutative projective scheme, \(\operatorname{QGr} A\), associated to the connected graded algebra, \(A\), over a field, \(k\). 
The natural choice of dg-category is the dg-enhancement, \(\mathcal{D}(\operatorname{QGr} A)\), of the derived category \(\operatorname{D}(\operatorname{QGr} A)\), in the sense of \parencite{Lunts-Orlov}, which is unique up to equivalence in the homotopy category of dg-categories.
One must then identify the dg-category \(\mathbf{R}\underline{\operatorname{Hom}}_c(\mathcal{D}(\operatorname{QGr} A), \mathcal{D}(\operatorname{QGr} B))\) noncommutative geometrically.

Generally, care must be taken to ensure good behavior of \(\operatorname{QGr} A\), but one may exert some control by imposing cohomological conditions on the ring, \(A\).
Two such common conditions are the Ext-finite condition of \parencite{BVdB} and the condition \(\chi^\circ(M)\) of \parencite{AZ94}.
One can interpret these conditions geometrically as imposing Serre vanishing for the noncommutative twisting sheaves together with a local finite dimensionality over the ground field, \(k\).
Specifically, one can force good behavior with respect to To\"en's derived Morita theory by requiring that two connected graded algebras, \(A\) and \(B\), over a field, \(k\), are both left and right Noetherian, Ext-finite, and satisfy the condition \(\chi^\circ(M)\) for the left/right \(A\)-modules \(M = A, A^{\operatorname{op}}\), and the left/right \(B\)-modules \(M = B, B^{\operatorname{op}}\).
We call such a pair of algebras a \textit{delightful couple}.

In this work we establish the identification
\[\mathbf{R}\underline{\operatorname{Hom}}_c(\mathcal{D}(\operatorname{QGr} A), \mathcal{D}(\operatorname{QGr} B)) \cong \mathcal{D}(\operatorname{QGr}(A^{\operatorname{op}} \otimes_k B))\]
in the homotopy category of dg-categories under these hypotheses.

As an easy corollary of the main result, one has the following statement.
\begin{theorem}
  Let \(X\) and \(Y\) be noncommutative projective schemes associated to a delightful couple of over a field \(k\), both of which are generated in degree one.
  Then for any equivalence \(\mathrm{D}(\Qcoh{X}) \to \mathrm{D}(\Qcoh{Y})\), there exists an object \(P\) of \(\mathrm{D}(\Qcoh{X \times_k Y})\) whose associated integral transform is an equivalence of Fourier-Mukai type.
\end{theorem}

\noindent
The interested reader can see Corollary~\ref{corollary: NCP morita degree 1} for a more careful statement of this result.

\section*{Conventions}
The ring \(k\) will always be at least Noetherian and commutative, though often will be a field.
Often, for ease of notation, \(\C(X,Y)\) will be used to refer to the morphims, \(\op{Hom}_\C(X,Y)\), between objects \(X\) and \(Y\) of a category \(\C\), though we shall also use an undecorated \(\op{Hom}\) depending on the complexity of the notation, provided the meaning is clear from context. 
Whenever \(\C\) has a natural enrichment over a category, \(\mathcal{V}\), we will denote by \(\underline{\C}(X,Y)\) the \(\mathcal{V}\)-object of morphisms.

For example, the category of complexes of \(k\)-vector spaces, \(\Ch{k}\), can be endowed with the the structure of a \(\Ch{k}\)-enriched category using the hom total complex, \(\CH{k}(C,D) := \underline{C}(k)(C,D)\) which has in degree \(n\) the \(k\)-vector space
\[\CH{k}(C,D)^n = \prod_{m \in \Z}\Mod{k}\left(C^m, D^{m + n}\right)\]
and differential
\[d(f) = d_D \circ f + (-1)^{n+1} f \circ d_C.\]
It should be noted that \(Z^0(\CH{k}(C,D)) = \Ch{k}(C,D)\).

