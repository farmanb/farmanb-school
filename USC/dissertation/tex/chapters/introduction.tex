\chapter*{Introduction}
One of the most useful and pleasing statements in algebra is the Morita Theorem \parencite{Morita}.

\begin{theorem}
  Let \(A\) and \(B\) be rings and assume we have an equivalence of module categories 
  \begin{displaymath}
    F : \Mod{A} \to \Mod{B}.
  \end{displaymath}
  Then there exists an \(A\)-\(B\) bimodule which is projective as an \(A\)-module and isomorphic to \(B\) as a \(B\)-module. Consequently, \(P \otimes_A -\) is an equivalence. 
\end{theorem}

For noncommutative rings, one can easily find non-isomorphic examples of Morita equivalent rings.
One such easy example is the equivalence between the modules over a ring \(A\) and the modules over its matrix ring, \(\op{M}_n(A)\), given by the bimodule \(A^n\).
For commutative rings, two rings are Morita equivalent if and only if they are isomorphic. One can try to slacken the relationship by considering the derived version of Morita for rings \parencite{Rickard}.

\begin{theorem}
  Let \(A\) and \(B\) be rings and assume we have an equivalence of derived categories of modules
  \begin{displaymath}
    F : \mathrm{D}(\Mod{A}) \to \mathrm{D}(\Mod{B}).
  \end{displaymath}
  Then there exists a complex of \(A\)-\(B\) bimodules which is perfect as a complex of \(A\)-modules and whose derived endomorphisms as \(B\)-modules are just \(B\) in degree \(0\). Consequently, \(P \overset{\mathbf{L}}{\otimes}_A -\) is an equivalence. 
\end{theorem}

Even still, here two commutative rings are derived Morita equivalent if and only if they are isomorphic. To get a strictly weaker equivalence relation, we should globalize the notion of a commutative ring by passing to schemes. 

\begin{theorem}
  Let \(X\) and \(Y\) be quasi-compact, quasi-separated schemes over a field, \(k\), and assume we have an equivalence 
  \begin{displaymath}
    F: \mathrm{D}(\Qcoh{X}) \to \mathrm{D}(\Qcoh{Y}).
  \end{displaymath}
  Then there exists an object \(P \in \mathrm{D}(\Qcoh{X \times Y})\) such that
  \begin{displaymath}
    M \mapsto \Phi_P (M) := \mathbf{R}\pi_{Y \ast} \left( P \overset{\mathbf{L}}{\otimes} \mathbf{L}\pi_X^\ast M \right) 
  \end{displaymath}
  gives an equivalence. 
\end{theorem}

Here \(X \overset{\pi_X}\longleftarrow X \times Y \overset{\pi_Y}\longrightarrow Y\) are the projections. This result, as stated, follows immediately from results of T\"oen and Lunts-Orlov \parencite{Toen07,Lunts-Orlov}. A more specific version for smooth projective schemes predates this \parencite{Orlov97}. In this generality, we unlock a deep and subtle equivalence relationship. Indeed, understanding these Fourier-Mukai partnerships is a central problem in the field of derived categories in algebraic geometry. 
Given the richness of the globalized Morita statement for commutative rings, one is led to think about a globalized version for noncommutative rings. A framework for phrasing such a question is due to Artin and Zhang \parencite{AZ94} and goes by the name Noncommutative Projective Geometry. A graded Morita theorem for noncommutative rings is due to Zhang \parencite{Zhang96}.  


Following the line of thought above, one wonders about derived Morita theory for noncommutative projective schemes \(X\) and \(Y\) which are associated to connected graded \(k\)-algebras \(A\) and \(B\).
As an application of the main result of this article, we have the following statement. One says that \(A\) and \(B\) form a delightful couple if they are both Ext-finite in the sense of \parencite{VdB}, both are left and right Noetherian, and both satisfy \(\chi^\circ(R)\) for \(R=A,A^\opp\) for \(A\) and \(R = B, B^\opp\) for \(B\). One can think of this requirement as Serre vanishing for the twisting sheaves plus some finite-dimensionality over \(k\).

\begin{theorem}
  Let \(X\) and \(Y\) be noncommutative projective schemes associated to a delightful couple of over a field \(k\), both of which are generated in degree one.
  Then for any equivalence \(\mathrm{D}(\Qcoh{X}) \to \mathrm{D}(\Qcoh{Y})\), there exists an object \(P\) of \(\mathrm{D}(\Qcoh{X \times_k Y})\) whose associated integral transform is an equivalence of Fourier-Mukai type.
\end{theorem}

\noindent
The interested reader can see Corollary~\ref{corollary: NCP morita degree 1} for a more careful statement of this result.

This result is, as the notation suggests, an application of a more general result. Work of T\"oen provides a good bicategorical structure for dg-categories up to quasi-equivalence \parencite{Toen07} - an internal Hom. The construction of \(\RHomc{\mathcal C, \mathcal D}\) is abstract even if the dg-categories \(\mathcal C\) and \(\mathcal D\) arise geometrically (or noncommutative geometrically). To unleash the power of T\"oen's work, one needs to identify the internal Hom more ``internally.'' Indeed, this is done for schemes in \parencite{Toen07}, for higher derived stacks (using machinery of Lurie in place of T\"oen) in \parencite{BFN10}, and matrix factorizations in \parencite{Dyckerhoff11,PV12,BFK14}. The main result here is the identification of this internal Hom in Noncommutative Projective Geometry, see Theorem~\ref{theorem: derived morita for NCP}. 

%TODO: Try to fit the stuff down here and the stuff above together, coherently.
\subsection*{Fourier-Mukai Kernels for Noncommutative Projective Schemes}

In light of their absense in noncommutative projective geometry, the natural question to ask is what these kernels should be.
To\"en's derived Morita theory \parencite{Toen07} gives an overarching framework to attack such a problem by abstracting to the higher categorical structure of differential graded (dg) categories.
Working within the homotopy category of the 2-category of all small dg-categories over a commutative ring, To\"en is able to provide an incredibly elegant reformulation of Fourier-Mukai functors at the level of pre-triangulated dg-categories via the dg-subcategory, \(\mathbb{R}\underline{\operatorname{Hom}}_c\), of the internal Hom.
Indeed, using this machinery, kernels have been recovered for schemes in \parencite{Toen07}, and obtained for higher derived stacks in \parencite{BFN10} and for categories of matrix factorizations in \parencite{Dyckerhoff11,PV12,BFK14}.
In each case, the work lies in the identification of the internal Hom object obtained from this machinery within the theory from which the input dg-categories originate, for even if they arise geometrically, the resulting Hom is often quite abstract.

The obvious first step in such work is to identify the possible input dg-categories for the machinery of derived Morita theory.
In the situation of interest, one considers the noncommutative projective scheme, \(\operatorname{QGr} A\), associated to the connected graded algebra, \(A\), over a field, \(k\). 
The natural choice of dg-category is the dg-enhancement, \(\mathcal{D}(\operatorname{QGr} A)\), of the derived category \(\operatorname{D}(\operatorname{QGr} A)\), in the sense of \parencite{Lunts-Orlov}, which is unique up to equivalence in the homotopy category of dg-categories.
One must then identify the dg-category \(\mathbb{R}\underline{\operatorname{Hom}}_c(\mathcal{D}(\operatorname{QGr} A), \mathcal{D}(\operatorname{QGr} B))\) noncommutative geometrically.

Generally, care must be taken to ensure good behavior of \(\operatorname{QGr} A\), but one may exert some control by imposing cohomological conditions on the ring, \(A\).
Two such common conditions are the Ext-finite condition of \parencite{BVdB} and the condition \(\chi^\circ(M)\) of \parencite{AZ94}.
One can interpret these conditions geometrically as imposing Serre vanishing for the noncommutative twisting sheaves together with a local finite dimensionality over the ground field, \(k\).
Specifically, one can force good behavior with respect to To\"en's derived Morita theory by requiring that two connected graded algebras, \(A\) and \(B\), over a field, \(k\), are both left and right Noetherian, Ext-finite, and satisfy the condition \(\chi^\circ(M)\) for the left/right \(A\)-modules \(M = A, A^{\operatorname{op}}\), and the left/right \(B\)-modules \(M = B, B^{\operatorname{op}}\).

In this work we establish the identification
\[\mathbb{R}\underline{\operatorname{Hom}}_c(\mathcal{D}(\operatorname{QGr} A), \mathcal{D}(\operatorname{QGr} B)) \cong \mathcal{D}(\operatorname{QGr}(A^{\operatorname{op}} \otimes_k B))\]
in the homotopy category of dg-categories under these hypotheses.
As an easy corollary, a pleasing derived Morita statement is obtained.
\begin{theorem}
  If there exists an equivalence
  \(f \colon \operatorname{D}(\operatorname{QGr} A) \to \operatorname{D}(\operatorname{QGr} B)\),
  then there exists an object \(P\) of \(\operatorname{D}\left(\operatorname{QGr} \left(A^{\operatorname{op}} \otimes_k B\right)\right)\) and a Fourier-Mukai transform
  \(\Phi_P \colon \operatorname{D}(\operatorname{QGr} A) \to \operatorname{D}(\operatorname{QGr} B)\)
  that is an equivalence.
\end{theorem}

\section*{Conventions}
The ring \(k\) will always be Noetherian and commutative.
Often, for ease of notation, \(\C(X,Y)\) will be used to refer to the morphims, \(\op{Hom}_\C(X,Y)\), between objects \(X\) and \(Y\) of a category \(\C\). We shall also use an undecorated \(\op{Hom}\) depending on the complexity of the notation. 
Whenever \(\C\) has a natural enrichment over a category, \(\mathcal{V}\), we will denote by \(\underline{\C}(X,Y)\) the \(\mathcal{V}\)-object of morphisms.

For example, the category of complexes of \(k\)-vector spaces, \(\Ch{k}\), can be endowed with the the structure of a \(\Ch{k}\)-enriched category using the hom total complex, \(\CH{k}(C,D) := \underline{C}(k)(C,D)\) which has in degree \(n\) the \(k\)-vector space
\[\CH{k}(C,D)^n = \prod_{m \in \Z}\Mod{k}\left(C^m, D^{m + n}\right)\]
and differential
\[d(f) = d_D \circ f + (-1)^{n+1} f \circ d_C.\]
It should be noted that \(Z^0(\CH{k}(C,D)) = \Ch{k}(C,D)\).
