One can regard the bimodule \(\mathbf{R}Q_{A \otimes_k A^\opp}\Delta\) as a sum of \(A\)-modules
\[\mathbf{R}Q_{A \otimes_k A^\opp}\Delta = \bigoplus_{x} (\mathbf{R}Q_{A \otimes_k A^\opp}\Delta)_{\ast,x}\]
and define for any object, \(M\), of \(\ldgGrMod{A}\) the object
\[\mathbf{R}\underline{\op{Hom}}_A(M, \mathbf{R}Q_{A \otimes_k A^\opp}\Delta) = \bigoplus_x \mathbf{R}\op{Hom}_A(M, (\mathbf{R}Q_{A \otimes_k A^\opp}\Delta)_{\ast,x})\]
of \(\rdgGrMod{A}\).
Consider the functor
\begin{align*}
  (-)^{\vee} : \ldgGrMod{A}^\opp & \to \rdgGrMod{A} \\
  M & \mapsto \mathbf{R}\underline{\op{Hom}}_A \left( M , \mathbf{R}Q_{A \otimes_k A^\opp} \Delta \right)
\end{align*}

\begin{lemma} \label{lemma: duality really is a duality}
  Assume \(A\) is delightful. Then the natural map 
  \begin{displaymath}
    \id \to (-)^{\vee \vee}
  \end{displaymath}
  given by evaluation is a quasi-isomorphism for \(\mathbf{R}Q_A A(x)\), for all \(x\). Furthermore, there are quasi-isomorphisms
  \begin{displaymath}
    \left( \mathbf{R}Q_A A(x) \right)^\vee \cong \mathbf{R}Q_{A^\opp} A(-x).
  \end{displaymath}
\end{lemma}

\begin{proof}
  %We observe that because \(\mathbf{R}Q_A\) and \(\mathbf{R}Q_{A^\opp}\) both commute with coproducts we obtain from Proposition~\ref{proposition: when beta is an isomorphism} two decompositions of \(\mathbf{R}Q_{A \otimes_k A^\opp} \Delta\) as an \(A\)-module
  %\begin{gather*}
  %\mathbf{R}Q_{A \otimes_k A^\opp} \Delta \cong \mathbf{R}Q_{A} \circ \mathbf{R}Q_{A^\opp} \Delta
  %= \mathbf{R}Q_{A} \circ \mathbf{R}Q_{A^\opp} \left(\bigoplus_j \Delta_{\ast,j}\right)\\
  %\cong \bigoplus_j \mathbf{R}Q_A \circ \mathbf{R}Q_{A^\opp} \left(\Delta_{\ast,j}\right) \cong \bigoplus_j \mathbf{R}Q_A (\Delta_{\ast,j}) = \bigoplus_j \mathbf{R}Q_A A(j)
%\end{gather*}
  %and as an \(A^\opp\)-module
  %\begin{gather*}
  %\mathbf{R}Q_{A \otimes_k A^\opp} \Delta \cong \mathbf{R}Q_{A^\opp} \circ \mathbf{R}Q_A \Delta
  %= \mathbf{R}Q_{A^\opp} \circ \mathbf{R}Q_A \left(\bigoplus_i \Delta_{i,\ast}\right)\\
  %\cong \bigoplus_i \mathbf{R}Q_{A^\opp} \circ \mathbf{R}Q_A (\Delta_{i,\ast})
  %\cong \bigoplus_i \mathbf{R}Q_{A^\opp} (\Delta_{i,\ast})
  %\cong \bigoplus_i \mathbf{R}Q_{A^\opp} A(i).
  %\end{gather*}
  We first exhibit the latter quasi-isomorphisms.  
%  By Proposition~\ref{proposition: bi-torsion is a composition}, we have quasi-isomorphisms
 % \[\mathbf{R}Q_{A \otimes_k A^\opp} \Delta \cong \mathbf{R}Q_{A} \circ \mathbf{R}Q_{A^\opp} \Delta\ \text{and}\ \mathbf{R}Q_{A \otimes_k A^\opp} \Delta \cong \mathbf{R}Q_{A^\opp} \circ \mathbf{R}Q_A \Delta.\]
  Using the quasi-isomorphisms of Proposition~\ref{proposition: when beta is an isomorphism}, we obtain two quasi-isomorphic decompositions of \(\mathbf{R}Q_{A \otimes_k A^\opp} \Delta\) as a sum of \(A\)-modules
  %we have a quasi-isomorphism \(\mathbf{R}Q_{A \otimes_k A^\opp} \Delta \cong \mathbf{R}Q_A \Delta\) and hence
  \[(\mathbf{R}Q_{A \otimes_k A^\opp} \Delta)_{\ast,j} \cong (\mathbf{R}Q_{A} \Delta)_{\ast,j} = \bigoplus_i (\mathbf{R}Q_{A} \Delta)_{i,j} = \bigoplus_i \mathbf{R}Q_A(\Delta_{\ast,j})_i = \mathbf{R}Q_A A(j)\]
  and as a sum of \(A^\opp\)-modules
  \[(\mathbf{R}Q_{A \otimes_k A^\opp} \Delta)_{i,\ast} \cong (\mathbf{R}Q_{A^\opp} \Delta)_{i,\ast} = \bigoplus_j (\mathbf{R}Q_{A^\opp} \Delta)_{i,j} = \bigoplus_j \mathbf{R}Q_{A^\opp}(\Delta_{i,\ast})_j = \mathbf{R}Q_{A^\opp} A(i).\]
  The first implies that \((\mathbf{R}Q_{A \otimes_k A^\opp} \Delta)_{\ast,j}\) is right orthogonal to \(\tau_A\)-torsion, hence by applying \(\mathbf{R}\underline{\op{Hom}}_A(-, \mathbf{R}Q_{A \otimes_k A^\opp} \Delta)\) to the triangle
  \[\mathbf{R}\tau_A A(x) \to A(x) \to \mathbf{R}Q_A A(x)\]
  we obtain a triangle %(\mathbf{R}Q_{A \otimes_k A^\opp} \Delta)_{-x,\ast} \cong
  \[(\mathbf{R}\tau_A A(x))^\vee \cong 0 \to A(x)^\vee \overset{\sim}\to (\mathbf{R}Q_A A(x))^\vee.\]
  Moreover, since \(A(x)\) is compact, we also obtain a quasi-isomorphism
  \begin{gather*}
    (\mathbf{R}Q_A A(x))^\vee
    \cong A(x)^\vee
    = \bigoplus_j \mathbf{R}\op{Hom}_A(A(x), (\mathbf{R}Q_{A \otimes_k A^\opp}\Delta)_{\ast,j})\\
    \cong \mathbf{R}\op{Hom}_A\left(A(x), \bigoplus_j(\mathbf{R}Q_{A \otimes_k A^\opp}\Delta)_{\ast,j}\right)
    = \mathbf{R}\op{Hom}_A(A(x), \mathbf{R}Q_{A \otimes_k A^\opp} \Delta) \\
    \cong (\mathbf{R}Q_{A \otimes_k A^\opp} \Delta)_{-x,\ast}
  \end{gather*}
  and the second decomposition yields
  \begin{displaymath}
    \left(\mathbf{R}Q_A A(x)\right)^\vee \cong (\mathbf{R}Q_{A \otimes_k A^\opp} \Delta)_{-x,\ast} \cong \mathbf{R}Q_{A^\opp} A(-x).
  \end{displaymath}
  Applying this twice, we get 
  \begin{displaymath}
    (\mathbf{R}Q_A A(x))^{\vee \vee} \cong \mathbf{R}Q_A A(x).
  \end{displaymath}
  We need only check that the natural map \(\nu : 1 \to (-)^{\vee \vee}\) induces the identity after this quasi-isomorphism.

%  Using the fact that \(\pi_A A(x)\) is a compact object and the derived adjunction \(\pi_A \dashv \mathbf{R}\omega_A\), we observe also that there is a quasi-isomorphism
%  \begin{gather*}
%    \mathbf{R}\op{Hom}_A(\mathbf{R}Q_A A(x), \mathbf{R}Q_{A \otimes_k A^\opp}\Delta) \cong \mathbf{R}\op{Hom}_{\QGr{A}}(\pi_A A(x), \pi_A \mathbf{R}Q_{A^\opp} \Delta) \\\cong \bigoplus_j \mathbf{R}\op{Hom}_{\QGr{A}}(\pi_A A(x), (\pi_A \mathbf{R}Q_{A^\opp} \Delta)_{\ast,j}) \cong \bigoplus_j \mathbf{R}\op{Hom}_A(\mathbf{R}Q_A A(x), (\mathbf{R}Q_{A \otimes_k A^\opp} \Delta)_{\ast,j})\\
%    = (\mathbf{R}Q_A A(x))^\vee
%  \end{gather*}

%  We have 
%  \begin{displaymath}
%    (\mathbf{R}Q_A A(x))^\vee \cong \mathbf{R}\op{Hom}_A (A(x), \mathbf{R}Q_{A \otimes_k A^\opp} \Delta) \cong (\mathbf{R}Q_{A \otimes_k A^\opp} \Delta)_{-x,\ast}
%  \end{displaymath}

  Note that we found a map
  \begin{displaymath}
    A(-x) \to \mathbf{R}Q_{A^\opp} A(-x) \to ( \mathbf{R}Q_A A(x) )^\vee 
  \end{displaymath}
  inducing the quasi-isomorphism \((\mathbf{R}Q_A A(x))^{\vee \vee} \cong \mathbf{R}Q_A A(x)\).
  If \(\alpha\) is the image of \(1\) in \(\mathbf{R}Q_{A^\opp} A(-x)\), denote by \(\alpha^\vee\) the image in \((\mathbf{R}Q_A A(x))^\vee\).
  Since \(1 \in A(-x)_x\), one can identify \(\alpha^\vee\) as a morphism
  \[\mathbf{R}Q_A A(x) \to (\mathbf{R}Q_{A \otimes_k A^\opp} \Delta)_{\ast,x} \cong \mathbf{R}Q_A A(x)\]
  %\begin{displaymath}
  %\mathbf{R}Q_A A(x) \to \mathbf{R}Q_{A \otimes_k A^\opp} \Delta (0,x) \cong \mathbf{R}Q_A ( \Delta (x,0) )
%\end{displaymath}
  which is the natural inclusion.
  For any \(a \in \mathbf{R}Q_A A(x)\) we obtain a morphism
  \[\op{ev}_a \colon (\mathbf{R}Q_A A(x))^\vee \to \mathbf{R}Q_{A \otimes A^\opp} \Delta \]
  and hence
  \[\op{ev}_a(\alpha^\vee) = \alpha^\vee(a) = a\]
  %Evaluating this at \(a \in \mathbf{R}Q_A A(x)\) gives \(a\) back.
  Thus, we see that \(\nu\) is quasi-fully faithful on \(\mathbf{R}Q_A A(x)\) for all \(x\). 
\end{proof}

\begin{definition}
  Let \(Q\A\) be the full dg-subcategory of \(\CH{\Gr{A}}\) with objects given by \(Q_A\) applied to injective resolutions of \(A(x)\) for all \(x\). 
\end{definition}

\begin{corollary} \label{corollary: duality is a duality}
  Assume that \(A\) is delightful. The functor \((-)^\vee\) induces a quasi-equivalence \((Q\A)^\opp \cong Q(\A^\opp)\). 
\end{corollary}

\begin{proof}
  From Lemma~\ref{lemma: duality really is a duality}, we see that \((-)^\vee\) is quasi-fully faithful on \(Q\A\) and has quasi-essential image \(Q(\A^\opp)\). 
\end{proof}

\begin{lemma} \label{lemma: trace map}
  Assume that \(A\) is delightful. There is a natural map 
  \begin{displaymath}
    \eta : M^\vee \overset{\mathbf{L}}{\otimes}_{\mathcal A} N \to \op{Hom}_A(M,N).
  \end{displaymath}
  Then \(\eta\) is a quasi-isomorphism for any \(M\) and any \(N \cong \mathbf{R}Q_A N\). 
\end{lemma}

\begin{proof}
  First, note that we have the natural map 
  \begin{displaymath}
    M^\vee \overset{\mathbf{L}}{\otimes}_{\mathcal A} N \to \mathbf{R}\underline{\op{Hom}}_A(M, \mathbf{R}Q_{A \otimes_k A^\opp} \Delta \overset{\mathbf{L}}{\otimes}_{\mathcal A} N).
  \end{displaymath}
  For \(M = A(x)\), we see this map is a quasi-isomorphism using the fact that \(A\) satisfies \(\bigstar\) from Proposition~\ref{proposition: vanishing of tensor}. 
  Since \(A\) satisfies \(\bigstar\), the map \(N \cong \Delta \otimes_{\mathcal A} N \to \mathbf{R}Q_{A \otimes_k A^\opp} \Delta \overset{\mathbf{L}}{\otimes}_{\mathcal A} N\) is a quasi-isomorphism. So the map  
  \begin{displaymath}
    \mathbf{R}\underline{\op{Hom}}_A(M, N) \to \mathbf{R}\underline{\op{Hom}}_A(M, \mathbf{R}Q_{A \otimes_k A^\opp} \Delta \overset{\mathbf{L}}{\otimes}_{\mathcal A} N)
  \end{displaymath}
  is also a quasi-isomorphism. Combining the two gives the desired quasi-isomorphism for \(M = A(x)\). But the condition \(\eta\) is a quasi-isomorphism is RTJ in \(M\) so is true for all \(M\) by Proposition~\ref{proposition: RTJ properties}
\end{proof}

