We recall a particularly nice type of property of objects in the setting of compactly generated triangulated categories.
In the sequel, many of our properties will be of this type, so we give this little gem a name.

\begin{definition} \label{definition: run the jewels}
  Let \(\D\) be a compactly generated triangulated category.
  Let \(\mathtt{P}\) be a property of objects of \(\D\). 
  We say that \(\mathtt{P}\) is \textbf{RTJ} if it satisifies the following three conditions.
  \begin{itemize}
  \item Whenever \(A \to B \to C\) is a triangle in \(\D\) and \(\mathtt{P}\) holds for \(A\) and \(B\), then \(\mathtt{P}\) holds for \(C\). 
  \item If \(\mathtt{P}\) holds for \(A\), then \(\mathtt{P}\) holds for the translate \(A[1]\).
  \item Let \(I\) be a set and \(A_i\) be objects of \(\D\) for each \(i \in I\). If \(\mathtt{P}\) holds for each \(A_i\), then \(\mathtt{P}\) holds for \(\bigoplus_{i \in I} A_i\). 
  \end{itemize}
\end{definition}

\begin{proposition} \label{proposition: RTJ properties}
  Let \(\mathtt{P}\) be an RTJ property that holds for a set of compact generators of \(\D\). Then \(\mathtt{P}\) holds for all objects of \(\D\).
\end{proposition}

\begin{proof}
  Let \(P\) be the full triangulated subcategory of objects for which \(\mathtt{P}\) holds. Then \(P\)
  \begin{itemize}
  \item contains a set of compact generators,
  \item is triangulated, and
  \item is closed under formation of coproducts. 
  \end{itemize}
  Thus, \(P\) is all of \(\D\). 
\end{proof}


\begin{definition} \label{definition: tensor vanishing}
  Let \(M\) be a complex of left graded \(A\)-modules and let \(N\) be a complex of right graded \(A\)-modules. We say that the pair satisfies \(\bigstar(M,N)\) if we have vanishing of the tensor product
  \begin{displaymath}
    \mathbf{R} \tau_{A^\opp} N {\otimes}^{\mathbf{L}}_{\mathcal A} \mathbf{R} Q_A M = 0.
  \end{displaymath}
  If \(\bigstar(M,N)\) holds for all \(M\) and \(N\), then we say that \(A\) satisfies \(\bigstar\). 
\end{definition}

\begin{proposition} \label{proposition: big star condition}
  Let \(A\) be a finitely generated, connected graded \(k\)-algebra.
  Assume that \(\mathbf{R}\tau_A\) and \(\mathbf{R}\tau_{A^\opp}\) commute with coproducts. Then \(A\) satisfies \(\bigstar\) if and only \(\bigstar( A(u), A(v) )\) holds for each \(u,v \in \Z\).
\end{proposition}

\begin{proof}
  The necessity is clear, so assume that \(\bigstar(A(u), A(v))\) holds for each \(u,v \in \Z\).
  First, we consider the property \(\bigstar(M, A(v))\) of objects, \(M\), of \(\mathrm{D}(\Gr{A})\).
  It's clear that this is an RTJ property that holds, by assumption, for the set of compact generators, \(\{A(u)\}_{u \in \Z}\).
  Hence \(\bigstar(M,A(v))\) holds for all \(M\) by Proposition~\ref{proposition: RTJ properties}.

  Now fix any object \(M\) of \(\mathrm{D}(\Gr{A})\) and consider the property \(\bigstar(M, N)\) of objects, \(N\), of \(\mathrm{D}(\Gr{A^\opp})\).
  This is again an RTJ property for which \(\bigstar(M,A(v))\) holds for all \(v \in \Z\).
  By Proposition~\ref{proposition: RTJ properties}, \(\bigstar(M,N)\) holds for all N .
  Since the choice of \(M\) was arbitrary, it follows that \(\bigstar(M,N)\) holds for all \(M\) and for all \(N\).
  Therefore \(A\) satisfies \(\bigstar\).
%  The necessity is clear, so assume that \(\bigstar( A(u), A(v) )\) holds for each \(u,v \in \Z\). Note that \(\bigstar(M , A(v))\) holds for all \(v\) is an RTJ property of \(M\) that holds for a set of compact generators \(A(u), u \in \Z\). Thus, by Proposition~\ref{proposition: RTJ properties}, \(\bigstar(M , A(v))\) holds for all \(v\) holds for all \(M\) in \(\mathrm{D}(\Gr{A})\). Similarly, we can consider the property of \(N\) in \(\mathrm{D}(\Gr{(A^\opp)})\): \(\bigstar(M , N)\) holds for all objects \(M\) of \(\mathrm{D}(\Gr{A})\). This is also RTJ so \(\bigstar(M,N)\) holds for all \(M\) and \(N\).
\end{proof}

There are various types of projection formulas.
We record here two which will be useful in the sequel.

\begin{proposition} \label{proposition: projection formula}
  Let \(A\) be a finitely generated, connected graded \(k\)-algebra.
  Let \(P\) be a complex of bi-bi \(A\)-modules and let \(M\) be a complex of left graded \(A\)-modules. Assume \(\mathbf{R} \tau_A\) commutes with coproducts. There is a natural quasi-isomorphism
  \begin{displaymath}
    ( \mathbf{R} \tau_A P ) \overset{\mathbf{L}}{\otimes}_{\mathcal A} M \to \mathbf{R} \tau_A \left( P \overset{\mathbf{L}}{\otimes}_{\mathcal A} M \right).
  \end{displaymath}
  Assume \(\mathbf{R} Q_A\) commutes with coproducts. There is a natural quasi-isomorphism
  \begin{displaymath}
    ( \mathbf{R} Q_A P ) \overset{\mathbf{L}}{\otimes}_{\mathcal A} M \to \mathbf{R} Q_A \left( P \overset{\mathbf{L}}{\otimes}_{\mathcal A} M \right).
  \end{displaymath}
\end{proposition}

\begin{proof}
  We treat the \(\tau\) projection formula. The \(Q\) projection formula is analogous. By Corollary~\ref{corollary: Q preserves bimodules}, we see that the tensor product is well-defined. It suffices to exhibit a natural transformation for the underived functors applied to modules to generate the desired natural transformation. Given
  \begin{displaymath}
    \psi \otimes_{\mathcal A} m \in \GR{A}(A/A_{\geq m}, P) \otimes_{\mathcal A} M 
  \end{displaymath}
  we naturally get 
  \begin{align*}
    \widetilde{\psi} : A/A_{\geq m} & \to P \otimes_{\mathcal A} M \\
    a & \mapsto \psi(a) \otimes_{\mathcal A} m. 
  \end{align*}
  Taking the colimit gives the natural transformation.
  
  Let us look at the natural transformation in the case that \(P = A(u) \otimes_k A(v)\) and \(M = A(w)\).
  Recall from Remark~\ref{remark: tensor with twist} that
  \[
  \mathbf{R}\tau_A(P) \otimes_\A^\mathbf{L} A(w) \cong \mathbf{R}\tau_A(P) \otimes_\A A(w) \cong \mathbf{R}\tau_A(P)_{\ast,w} 
  := \bigoplus_{x \in \Z} \mathbf{R}\tau_A(P_{\ast,w})_x
  = \mathbf{R}\tau_A(P_{\ast,w})
  \]
  which is compatible with the natural transformation.
  The property that the natural transformation is a quasi-isomorphism is RTJ in each entry.
  Thus, it holds for all \(P\) and \(M\) by Proposition~\ref{proposition: RTJ properties}.
%  Recall that 
%  \begin{gather*}
%    \left( A(u) \otimes_k A(v) \right) \overset{\mathbf{L}}{\otimes}_{\mathcal A} A(w) \cong \left( A(u) \otimes_k A(v) \right) \otimes_{\mathcal A} A(w) \\ \cong \left( A(u) \otimes_k A(v) \right)_{\ast,w} \cong A(u) \otimes_k A_{v+w}. 
%  \end{gather*}
%  So
%  \begin{displaymath}
%    \mathbf{R}\tau_A \left( \left( A(u) \otimes_k A(v) \right) \overset{\mathbf{L}}{\otimes}_{\mathcal A} A(w) \right) \cong \mathbf{R} \tau_A \left( A(u) \otimes_k A_{v+w} \right)
%  \end{displaymath}
%  while 
%  \begin{gather*}
%    \mathbf{R}\tau_A \left( A(u) \otimes_k A(v) \right) \overset{\mathbf{L}}{\otimes}_{\mathcal A} A(w) \cong \mathbf{R}\tau_A \left( A(u) \otimes_k A(v) \right)_w \\ \cong \mathbf{R}\tau_A \left( (A(u) \otimes_k A(v))_{\ast,w} \right) \cong \mathbf{R}\tau_A \left( A(u) \otimes_k A_{v+w} \right)
%  \end{gather*}
  %which are compatible with the natural transformation. The property that the natural transformation is a quasi-isomorphism is RTJ in each entry. Thus, it holds for all \(P\) and \(N\) by Proposition~\ref{proposition: RTJ properties}. 
\end{proof}

For the hypothesis, recall Definition~\ref{definition: delightful couple}. 

\begin{proposition} \label{proposition: vanishing of tensor}
  Assume \(A\) is delightful. Then \(\bigstar\) holds for \(A\).
\end{proposition}

\begin{proof}
  By Proposition~\ref{proposition: big star condition}, it suffices to check \(\bigstar(M,A(v))\) for each \(v\). This is equivalent to \(\bigstar(M,\bigoplus_v A(v))\). Equipping the sum with a bi-bi structure as \(\Delta\), we reduce to checking \(\bigstar(M,\Delta)\). Using Proposition~\ref{proposition: when beta is an isomorphism} and Lemma~\ref{lemma: exact triangles} for \(A\) and \(A^\opp\), we have a natural quasi-isomorphism
  \[\mathbf{R}\tau_{A^\opp} \Delta \overset{\mathbf{L}}{\otimes}_{\mathcal A} \mathbf{R}Q_A M \cong \mathbf{R}\tau_{A} \Delta \overset{\mathbf{L}}{\otimes}_{\mathcal A} \mathbf{R}Q_{A} M.\]
  Using Proposition~\ref{proposition: projection formula}, we have a natural quasi-isomorphism
  \[\mathbf{R}\tau_{A} \Delta \overset{\mathbf{L}}{\otimes}_{\mathcal A} \mathbf{R}Q_{A} M \cong \mathbf{R}\tau_{A} \left( \Delta \overset{\mathbf{L}}{\otimes}_{\mathcal A} \mathbf{R}Q_{A} M  \right) \cong \mathbf{R}\tau_{A} \left( \mathbf{R}Q_{A} M \right) = 0.\]
  
%  By Proposition~\ref{proposition: big star condition}, it suffices to check \(\bigstar(A(u),N)\) for each \(u\). This is equivalent to \(\bigstar(\bigoplus_u A(u),N)\). Equipping the sum with a bi-bi structure as \(\Delta\), we reduce to checking \(\bigstar(\Delta,N)\). Using Proposition~\ref{proposition: when beta is an isomorphism} for \(A\) and \(A^\opp\), we have a natural quasi-isomorphism
%  \begin{displaymath}
%    \mathbf{R}\tau_{A^\opp} N \overset{\mathbf{L}}{\otimes}_{\mathcal A} \mathbf{R}Q_A \Delta \cong \mathbf{R}\tau_{A^\opp} N \overset{\mathbf{L}}{\otimes}_{\mathcal A} \mathbf{R}Q_{A^\opp} \Delta.
%  \end{displaymath}
%  Using Proposition~\ref{proposition: projection formula}, we have a natural quasi-isomorphism
%  \begin{displaymath}
%    \mathbf{R}\tau_{A^\opp} N \overset{\mathbf{L}}{\otimes}_{\mathcal A} \mathbf{R}Q_{A^\opp} \Delta \cong \mathbf{R}Q_{A^\opp} \left( \mathbf{R}\tau_{A^\opp} N \overset{\mathbf{L}}{\otimes}_{\mathcal A} \Delta  \right) \cong \mathbf{R}Q_{A^\opp} \left( \mathbf{R}\tau_{A^\opp} N \right) = 0.
%  \end{displaymath}
\end{proof}

