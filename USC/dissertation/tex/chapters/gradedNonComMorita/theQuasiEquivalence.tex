Now we turn to the main result. 

\begin{theorem} \label{theorem: derived morita for NCP}
  Let \(k\) be a field. Let \(A\) and \(B\) be connected graded \(k\)-algebras. If \(A\) and \(B\) form a delightful couple, then there is a natural quasi-equivalence 
  \begin{displaymath}
    F : \hinj{\QGr{A^\opp \otimes_k B}} \to \RHomc{ \hinj{\QGr{A}}, \hinj{\QGr{B}} }
  \end{displaymath}
  such that for an object \(P\) of \(\mathrm{D}(\QGr{A^\opp \otimes_k B})\), the exact functor \(H^0(F(P))\) is isomorphic to 
  \begin{displaymath}
    \Phi_P(M) :=  \pi_B \left( \mathbf{R}\omega_{A^\opp \otimes_k B} P \overset{\mathbf{L}}{\otimes}_{\mathcal A} \mathbf{R}\omega_A M \right).
  \end{displaymath}
\end{theorem}

\begin{proof}
  Applying Corollary~\ref{corollary: Toen}, it suffices to provide a quasi-equivalence
  \begin{displaymath}
    G : \hinj{\QGr{A^\opp \otimes_k B}} \to \hproj{ (Q \mathcal A)^\opp \otimes_k Q \mathcal B}
  \end{displaymath}
  Using Corollary~\ref{corollary: duality is a duality}, we have a quasi-equivalence
  \begin{displaymath}
    \hproj{ (Q \mathcal A)^\opp \otimes_k Q \mathcal B} \cong \hproj{ Q \mathcal A^\opp \otimes_k Q \mathcal B}. 
  \end{displaymath}
  From Lemma~\ref{lemma: another model for QA otimes QB} we have a quasi-fully faithful functor 
  \begin{displaymath}
    Q \mathcal A^\opp \otimes_k Q \mathcal B \to \hinj{\QGr{A^\opp \otimes_k B}}. 
  \end{displaymath}
  This gives a functor 
  \begin{displaymath}
    \hinj{\QGr{A^\opp \otimes_k B}} \to \hproj{Q \mathcal A^\opp \otimes_k Q \mathcal B}
  \end{displaymath}
  which is a quasi-equivalence by a standard argument, see e.g. \parencite[Theorem 5.1]{Dyckerhoff11}.
  
  Tracing out the quasi-equivalences, one just needs to manipulate 
  \begin{align*}
    \op{Hom} (\mathbf{R}Q_A A(x)^\vee \otimes_k \mathbf{R}Q_B B(y), P) & \cong \op{Hom} ( \mathbf{R}Q_B B(y) , \op{Hom}( \mathbf{R}Q_A A(x)^\vee, \mathbf{R}\omega_{A^\opp \otimes_k B} P)) \\
    & \cong \op{Hom} ( \mathbf{R}Q_B B(y) , \mathbf{R}\omega_{A^\opp \otimes_k B} P \overset{\mathbf{L}}{\otimes}_{\mathcal A} \mathbf{R}Q_A A(x) ) 
  \end{align*}
  using Propostion~\ref{proposition: vanishing of tensor} and Lemma~\ref{lemma: trace map}. This says that the induced continuous functor is
  \begin{displaymath}
    M \mapsto \pi_B \left( \mathbf{R}\omega_{A^\opp \otimes_k B} P \overset{\mathbf{L}}{\otimes}_{\mathcal A} \mathbf{R}\omega_A M \right). 
  \end{displaymath}
\end{proof}

The following statement is now a simple application of Theorem~\ref{theorem: derived morita for NCP} and results of \parencite{Lunts-Orlov}. 

\begin{corollary} \label{corollary: NCP morita}
  Let \(A\) and \(B\) be a delightful couple of connected graded \(k\)-algebras with \(k\) a field. Assume that there exists an equivalence
  \begin{displaymath}
    f : \mathrm{D} (\QGr{A}) \to \mathrm{D} (\QGr{B}).
  \end{displaymath}
  Then there exists an object \(P \in D ( \QGr{A^\opp \otimes_k B} )\) such that 
  \begin{displaymath}
    \Phi_P : \mathrm{D} ( \QGr{A}) \to \mathrm{D} (\QGr{B} )
  \end{displaymath}
  is an equivalence.
\end{corollary}

\begin{proof}
  Applying \parencite[Theorem 1]{Lunts-Orlov} we know there is a quasi-equivalence between the unique enhancements, i.e. there is an \( F \in [ \hinj{ \QGr{A}}, \hinj{ \QGr{B}} ]\) giving an equivalence
  \begin{displaymath}
    H^0(F) : H^0(\hinj{ \QGr{A} }) = \mathrm{D}(\QGr{A}) \to H^0(\hinj{ \QGr{B} }) = \mathrm{D}(\QGr{B}).
  \end{displaymath}
  Then, by Theorem~\ref{theorem: derived morita for NCP}, there exists a \(P \in \mathrm{D}(\QGr{A^\opp \otimes_k B})\) such that \(\Phi_P = H^0(F)\). 
\end{proof}

We wish to identify the kernels as objects of the derived category of an honest noncommutative projective scheme.
In general, one can only hope that kernels obtained as above are objects of the derived category of a noncommutative (bi)projective scheme.
However, we have the following special case in which we can collapse the \(\Z^2\)-grading to a \(\Z\)-grading.

\begin{corollary} \label{corollary: NCP morita degree 1}
  Let \(A\) and \(B\) be a delightful couple of connected graded \(k\)-algebras with \(k\) a field that are both generated in degree one.
  Assume that there exists an equivalence
  \begin{displaymath}
    f : \mathrm{D} (\QGr{A}) \to \mathrm{D} (\QGr{B}).
  \end{displaymath}
  Then there exists an object \(P \in D ( \QGr{A^\opp \times_k B} )\) that induces an equivalence
  \[\begin{tikzcd}[row sep=tiny]
  \mathrm{D}(\QGr{A}) \arrow{r}& \mathrm{D}(\QGr{B})\\
  M \arrow[mapsto]{r} & \pi_B\left(\mathbb{V}_\text{dg}(P) \otimes^\mathbf{L} \mathbf{R}\omega_A M\right)
  \end{tikzcd}\]
\end{corollary}

\begin{proof}
  The equivalence \(\mathbb{V}\) of Theorem~\ref{theorem: Van Rompay} extends naturally to a quasi-equivalence
  \[\mathbb{V}_\text{dg} \colon \hinj{\QGr{S}} \to \hinj{\QGr{T}}.\]
  Now choose \(P\) such that \(\mathbb{V}_\text{dg}(P)\) is homotopy equivalent to the kernel obtained by an application of Corollary~\ref{corollary: NCP morita}, so the desired equivalence is \(\Phi_{\mathbb{V}_\text{dg}(P)}\).
\end{proof}
