\section{The quasi-equivalence}

Now we turn to the main result. 

\begin{theorem} \label{theorem: derived morita for NCP}
  Let \(k\) be a field. Let \(A\) and \(B\) be connected graded \(k\)-algebras. If \(A\) and \(B\) form a tasty pair, then there is a natural quasi-equivalence 
  \begin{displaymath}
    F : \hinj{\QGr{A^\opp \otimes_k B}} \to \RHomc{ \hinj{\QGr{A}}, \hinj{\QGr{B}} }
  \end{displaymath}
  such that for an object \(P\) of \(\mathrm{D}(\QGr{A^\opp \otimes_k B})\), the exact functor \(H^0(F(P))\) is isomorphic to 
  \begin{displaymath}
    \Phi_P(M) :=  \pi_B \left( \mathbf{R}\omega_{A^\opp \otimes_k B} P \overset{\mathbf{L}}{\otimes}_{\mathcal A} \mathbf{R}\omega_A M \right).
  \end{displaymath}
\end{theorem}

\begin{proof}
  Applying Corollary~\ref{corollary: Toen}, it suffices to provide a quasi-equivalence
  \begin{displaymath}
    G : \hinj{\QGr{A^\opp \otimes_k B}} \to \hproj{ (Q \mathcal A)^\opp \otimes_k Q \mathcal B}
  \end{displaymath}
  Using Corollary~\ref{corollary: duality is a duality}, we have a quasi-equivalence
  \begin{displaymath}
    \hproj{ (Q \mathcal A)^\opp \otimes_k Q \mathcal B} \cong \hproj{ Q \mathcal A^\opp \otimes_k Q \mathcal B}. 
  \end{displaymath}
  From Lemma~\ref{lemma: another model for QA otimes QB} we have a quasi-fully faithful functor 
  \begin{displaymath}
    Q \mathcal A^\opp \otimes_k Q \mathcal B \to \hinj{\QGr{A^\opp \otimes_k B}}. 
  \end{displaymath}
  This gives a functor 
  \begin{displaymath}
    \hinj{\QGr{A^\opp \otimes_k B}} \to \hproj{Q \mathcal A^\opp \otimes_k Q \mathcal B}
  \end{displaymath}
  which is a quasi-equivalence by a standard argument, see e.g. \parencite[Theorem 5.1]{Dyckerhoff11}.
  
  Tracing out the quasi-equivalences, one just needs to manipulate 
  \begin{align*}
    \op{Hom} (\mathbf{R}Q_A A(x)^\vee \otimes_k \mathbf{R}Q_B B(y), P) & \cong \op{Hom} ( \mathbf{R}Q_B B(y) , \op{Hom}( \mathbf{R}Q_A A(x)^\vee, \mathbf{R}\omega_{A^\opp \otimes_k B} P)) \\
    & \cong \op{Hom} ( \mathbf{R}Q_B B(y) , \mathbf{R}\omega_{A^\opp \otimes_k B} P \overset{\mathbf{L}}{\otimes}_{\mathcal A} \mathbf{R}Q_A A(x) ) 
  \end{align*}
  using Propostion~\ref{proposition: vanishing of tensor} and Lemma~\ref{lemma: trace map}. This says that the induced continuous functor is
  \begin{displaymath}
    M \mapsto \pi_B \left( \mathbf{R}\omega_{A^\opp \otimes_k B} P \overset{\mathbf{L}}{\otimes}_{\mathcal A} \mathbf{R}\omega_A M \right). 
  \end{displaymath}
\end{proof}

The following statement is now a simple application of Theorem~\ref{theorem: derived morita for NCP} and results of \parencite{Lunts-Orlov}. 

\begin{corollary} \label{corollary: NCP morita}
  Let \(A\) and \(B\) be a tasty pair of connected graded \(k\)-algebras with \(k\) a field. Assume that there exists an equivalence
  \begin{displaymath}
    f : \mathrm{D} (\QGr{A}) \to \mathrm{D} (\QGr{B}).
  \end{displaymath}
  Then there exists an object \(P \in D ( \QGr{A^\opp \otimes_k B} )\) such that 
  \begin{displaymath}
    \Phi_P : \mathrm{D} ( \QGr{A}) \to \mathrm{D} (\QGr{B} )
  \end{displaymath}
  is an equivalence.
\end{corollary}

\begin{proof}
  Applying \parencite[Theorem 1]{Lunts-Orlov} we know there is a quasi-equivalence between the unique enhancements, i.e. there is an \( F \in [ \hinj{ \QGr{A}}, \hinj{ \QGr{B}} ]\) giving an equivalence
  \begin{displaymath}
    H^0(F) : H^0(\hinj{ \QGr{A} }) = \mathrm{D}(\QGr{A}) \to H^0(\hinj{ \QGr{B} }) = \mathrm{D}(\QGr{B}).
  \end{displaymath}
  Then, by Theorem~\ref{theorem: derived morita for NCP}, there exists a \(P \in \mathrm{D}(\QGr{A^\opp \otimes_k B})\) such that \(\Phi_P = H^0(F)\). 
\end{proof}

%TODO: Determine if this gets yanked into Section 2 or not.
We wish to identify the kernels as objects of the derived category of an honest noncommutative projective scheme.  Towards this end, we define a graded ring associated to a pair of graded \(k\)-algebras.
\begin{definition}\label{def: segre product}
  Let \(A\) and \(B\) be connected graded \(k\)-algebras.
  The \textbf{Segre product} of \(A\) and \(B\) is the graded \(k\)-algebra
  \[ A \times_k B = \bigoplus_{0 \leq i} A_i \otimes_k B_i.\]
\end{definition}

In general, one can only hope that kernels obtained as above are objects of the derived category of a noncommutative (bi)projective scheme.
However, we have the following special case in which we can collapse the \(\Z^2\)-grading to a \(\Z\)-grading.
Denote by \(\A \times \B\) the dg-category with objects \(\Z\) and morphisms \(\A \times \B(i,j)\) the chain complex with \(A_i \otimes_k B_i\) in degree zero.

\begin{lemma}\label{lemma: Q(AxB) is Q(A tensor B)}
  Assume \(A\) and \(B\) satisfy (insert hypotheses here).
  Then there is a quasi-equivalence
  \[Q(\A \times \B) \to Q(\A \otimes \B)\]
\end{lemma}

\begin{proof}
  We have an obvious functor \(\Delta \colon \A \times \B \to \A \otimes \B\) defined by \(\Delta(i) = (i,i)\) on objects with structure morphism the identity.
  %\[ \A \times \B(i,j) = A_{j-i} \otimes_k B_{j-i} \to A_{j-i} \otimes_k B_{j-i}= \A \otimes \B((i,i), (j,j)).\]
  This yields an adjoint pair of dg-functors \(\Ind{\Delta} \colon \dgMod{\A \times \B} \to \dgMod{\A \otimes \B}\) and \(\Res{\Delta} \colon \dgMod{\A \otimes \B} \to \dgMod{\A \times \B}\).
  Moreover, it is easy to check that \(\Ind{\Delta}\) is induced by the object \(P\) of \(\dgMod{\left(\A \times \B\right)^\opp \otimes \left(\A \otimes \B \right)}\) defined by
  \[P(i,(m,n)) = \left(\A \otimes \B\right)^\opp\left((i,i),(m,n)\right) = A_{i- m} \otimes_k B_{i - n}\]
  in the sense that \(\Ind{\Delta} \cong \widehat{\Phi_P} = - \otimes_{\A \times \B} P\).
\end{proof}
\begin{corollary} \label{corollary: NCP morita degree 1}
  Let \(A\) and \(B\) be a tasty pair of connected graded \(k\)-algebras with \(k\) a field that are both generated in degree one.
  Assume that there exists an equivalence
  \begin{displaymath}
    f : \mathrm{D} (\QGr{A}) \to \mathrm{D} (\QGr{B}).
  \end{displaymath}
  Then there exists an object \(P \in D ( \QGr{A^\opp \times_k B} )\) such that 
  \begin{displaymath}
    \Phi_P : \mathrm{D} ( \QGr{A}) \to \mathrm{D} (\QGr{B} )
  \end{displaymath}
  is an equivalence.
\end{corollary}

\begin{proof}
\end{proof}
