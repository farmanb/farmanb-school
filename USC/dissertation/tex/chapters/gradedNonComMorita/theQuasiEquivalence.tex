
\begin{lemma} \label{lemma: quasi-equivalence on gens is quasi-equivalence}
  Let \(\A\) and \(\B\) be dg-categories.
  Assume that \(H^0(\A)\) and \(H^0(\B)\) are both triangulated categories, each with a set of compact generators \(\{A_i\}_I\) and \(\{B_j\}_J\).
  If \(F \colon \A \to \B\) is a continuous dg-functor satisfying \(F(\{A_i\}_I) = \{B_j\}_J\) and the structure morphism
  \[F_{A_{i_1}, A_{i_2}} \colon \A(A_{i_1}, A_{i_2}) \to \B(FA_{i_1}, FA_{i_2})\]
  is a quasi-isomorphism for all \(i_1, i_2 \in I\), then \(F\) is a quasi-equivalence.
\end{lemma}

\begin{proof}
  We observe that it suffices to show that \(F\) is quasi-fully faithful.
  Indeed, if \(F\) is quasi-fully faithful, then the essential image of \(H^0(\A)\) under \(H^0(F)\) is a triangulated subcategory of \(H^0(\B)\) that is closed under coproducts and contains the generators \(\{B_j\}_J\) by assumption.
  Since \(H^0(\B)\) is the smallest such triangulated subcategory, it follows that the essential image of \(H^0(\A)\) is all of \(H^0(\B)\) and thus \(F\) is quasi-essentially surjective.

  We break the argument into two pieces.
  %First we show that \(F_{A_i, X} \colon \A(A_i, X) \to \B(FA_i, FX)\) is a quasi-isomorphism for all objects \(X\) of \(A\).
  %With this established, we are able to extend the result to all pairs of objects of \(A\).
  The proof of each case is similar in style to the proof that \(F\) is quasi-essentially surjective above.
  In the first case, we show that the full dg-subcategory, \(\C\), of \(\A\) consisting of objects \(C\) such that
  \[F_{A_i,C} \colon \A(A_i, C) \to \B(FA_i, FC)\]
  is a quasi-isomorphism for all \(i \in I\) satisfies \(H^0(\C) = H^0(\A)\), so that, being a full dg-subcategory of \(\A\) with the same objects as \(H^0(\C)\), \(\C = \A\).
  Having established this, we obtain a non-trivial full dg-subcategory, \(\D\), of \(\A\) consisting of objects \(D\) such that
  \[F_{D,X} \colon \A(D,X) \to \B(FD, FX)\]
  is a quasi-isomorphism for all objects \(X\) of \(\A\).
  Once again we show that \(H^0(\D) = H^0(\A)\).
  By the same argument, mutatis mutandis, this implies that \(\D = \A\) and \(F\) is quasi-fully faithful.
  
  %Consider the full dg-subcategory, \(\C\), of \(\A\) consisting of objects \(C\) such that
  %\[F_{A_i,C} \colon \A(A_i, C) \to \B(FA_i, FC)\]
  %is a quasi-isomorphism for all \(i \in I\).
  %First we show that \(\C = \A\), which is equivalent to showing that \(H^0(\C) = H^0(\A)\) because the objects of \(H^0(\C)\) coincide with the objects of \(\C\) and the objects of \(H^0(\A)\) coincide with the objects of \(\A\).
  %We reduce further to showing that \(H^0(\C)\) is a triangulated subcategory of \(H^0(\A)\) that is closed under coproducts and contains \(\{A_i\}_I\), since \(H^0(\A)\) is the smallest triangulated subcategory satisfying this property.
  Towards the first goal, we note that it suffices to show \(H^0(\C)\) is triangulated, closed under coproducts, and contains \(\{A_i\}_I\).
  The latter condition is guaranteed by hypothesis.
  That \(H^0(\C)\) is closed under translation follows from the fact that if \(C\) is an object of \(\C\), then for all \(i\) and for all \(n\)
  \[H^0(\A(A_i, C[n])) \cong H^{n}(\A(A_i, C)) \cong H^{n}(\B(FA_i, FC)) \cong H^0(\B(FA_i, FC[n])).\]
  Now, for any distinguished triangle \(C_1 \to C_2 \to X \to C_1[1]\) of \(H^0(\A)\) with \(C_1, C_2\) objects of \(\C\) we see that \(X\) is an object of \(\C\) by applying the 5 Lemma to the morphism of long exact sequences induced by the homological functors \(H^0(\A)(A_i, -)\) and \(H^0(\B)(FA_i, -)\)
  \[\begin{tikzcd}[column sep=tiny]
  \cdots \arrow{r} & H^0(\A(A_i, C_1)) \arrow{r}\arrow{d}{H^0(F_{A_i,C_1})} & H^0(\A(A_i, C_2)) \arrow{r}\arrow{d}{H^0(F_{A_i,C_2})} & H^0(\A(A_i, X)) \arrow{r}\arrow{d}{H^0(F_{A_i,X})} & H^1(\A(A_i, C_1)) \arrow{r}\arrow{d}{H^1(F_{A_i, C})} & \cdots\\
  \cdots \arrow{r} & H^0(\B(FA_i, FC_1)) \arrow{r} & H^0(\B(FA_i, FC_2)) \arrow{r} & H^0(\B(F_i, FC)) \arrow{r} & H^1(\B(F_i, FC_1)) \arrow{r}& \cdots\\
  \end{tikzcd}\]
  for each \(i \in I\).
  Hence by equipping \(H^0(\C)\) with the distinguished triangles from \(H^0(\A)\) of the form \(C_1 \to C_2 \to C_3 \to C_1[1]\) with the \(C_i\) objects of \(\C\), \(H^0(\C)\) inherits the structure of a triangulated subcategory.
  Finally we note that because \(A_i\) and \(FA_i \in \{B_j\}_J\) are compact, and the induced functor \(H^0(F)\) commutes with direct sums, we have for any set, \(\{C_\alpha\}\), of objects of \(\C\) the isomorphism
  \begin{gather*}
    H^0\left(\A\left(A_i, \bigoplus_\alpha C_\alpha\right)\right) \cong \bigoplus_\alpha H^0\left(\A\left(A_i, C_\alpha\right)\right) \cong \bigoplus_\alpha H^0\left(\B\left(FA_i, C_\alpha\right)\right)\\
    \cong H^0\left(\B\left(FA_i, \bigoplus_\alpha FC_\alpha\right)\right) \cong H^0\left(\B\left(FA_i, F\left(\bigoplus_\alpha C_\alpha\right)\right)\right)
  \end{gather*}
  implies that \(H^0(\C)\) is closed under coproducts.
  %As \(H^0(\A)\) is the smallest triangulated subcategory of itself containing \(\{A_i\}_I\) and closed under coproducts, it follows that \(H^0(\C) = H^0(\A)\) and thus \(\C = \A\).

  %Now consider the full dg-subcategory, \(\D\), of \(\A\) consisting of objects \(D\) such that
  %\[F_{D,X} \colon \A(D,X) \to \B(FD, FX)\]
  %is a quasi-isomorphism for all objects \(X\) of \(\A\), which we have established is non-trivial since it contains \(\{A_i\}_I\).
  To see that \(\D = \A\), we again observe that it suffices to show \(H^0(\D)\) is triangulated, closed under coproducts, and contains the generators, \(\{A_i\}_I\).
  The latter condition follows from the fact that the category \(\C\) contains \(\{A_i\}_I\).
  For any object \(D\) of \(\D\) and any object \(X\) of \(\A\), the fact that translation is an auto-equivalence yields the natural isomorphism
  \[\A(D[n], X) \cong \A(D, X[-n])\]
  from which we obtain the isomorphism
  \[H^0(\A(D[n], X)) \cong H^{-n}(\A(D,X)) \cong H^{-n}(\B(FD, FX)) \cong H^0(\B(FD[n], FX))\]
  for all \(n\).
  Hence \(H^0(\D)\) is closed under translations.
  Next we see that for any set of objects \(\{D_\alpha\}\) of \(\D\) and any object \(X\) of \(\A\) we have the isomorphism
  \begin{gather*}
    H^0\left(\A\left(\bigoplus_\alpha D_\alpha, X\right)\right)
  \cong \prod_\alpha H^0\left(\A\left(D_\alpha, X\right)\right)
    \cong \prod_\alpha H^0\left(\B\left(FD_\alpha, X\right)\right)\\
    \cong H^0\left(\B\left(\bigoplus_\alpha FD_\alpha, X\right)\right)
    \cong H^0\left(\B\left(F\left(\bigoplus_\alpha D_\alpha\right), X\right)\right)
  \end{gather*}
  which implies that \(H^0(\D)\) is closed under coproducts.
  Finally, for any distinguished triangle \(D_1 \to D_2 \to Z \to D_1[1]\) of \(H^0(\A)\) with \(D_1, D_2\) objects of \(\D\) we see that \(Z\) is an object of \(\D\) by applying the 5 Lemma to the morphism of long exact sequences induced by the cohomological functors \(H^0(\A)(-, X)\) and \(H^0(\B)(-, FX)\)
  \[\begin{tikzcd}[column sep=tiny]
  \cdots \arrow{r} & H^0(\A(D_2, X)) \arrow{r}\arrow{d}{H^0(F_{D_2,X})} & H^0(\A(D_1, X)) \arrow{r}\arrow{d}{H^0(F_{D_1,X})} & H^1(\A(Z, X)) \arrow{r}\arrow{d}{H^1(F_{Z,X})} & H^1(\A(D_2, X)) \arrow{r}\arrow{d}{H^1(F_{D_2,X})} & \cdots\\
  \cdots \arrow{r} & H^0(\B(FD_2, FX)) \arrow{r} & H^0(\B(FD_1, FX)) \arrow{r} & H^1(\B(FZ, FX)) \arrow{r} & H^1(\B(FD_2, FX)) \arrow{r}& \cdots\\
  \end{tikzcd}\]
  for each \(i \in I\).
  Hence by equipping \(H^0(\D)\) with the distinguished triangles \(D_1 \to D_2 \to D_3 \to D_1[1]\) of \(H^0(\A)\), where the \(D_i\) are objects of \(\D\), inherits the structure of a triangulated subcategory.
  Therefore \(D = \A\) and \(F\) is quasi-fully faithful, as desired.
  %As \(H^0(\A)\) is the smallest triangulated subcategory of \(H^0(\A)\) containing \(\{A_i\}_I\) and closed under coproducts, we see that \(H^0(\C) = H^0(\A)\).
  %Since the objects of \(H^0(\A)\) coincide with those of \(\A\), it follows that \(F\) is quasi-fully faithful.

  %The quasi-essential image of \(F\) is therefore a triangulated subcategory of \(H^0(\B)\) that contains \(\{B_j\}_J\) and is closed under coproducts.
  %Therefore \(F\) is quasi-essentially surjective, as desired.
\end{proof}


We now turn to the main result.

\begin{theorem} \label{theorem: derived morita for NCP}
  Let \(k\) be a field. Let \(A\) and \(B\) be connected graded \(k\)-algebras. If \(A\) and \(B\) form a delightful couple, then there is a natural quasi-equivalence 
  \begin{displaymath}
    F : \hinj{\QGr{A^\opp \otimes_k B}} \to \RHomc{ \hinj{\QGr{A}}, \hinj{\QGr{B}} }
  \end{displaymath}
  such that for an object \(P\) of \(\mathrm{D}(\QGr{A^\opp \otimes_k B})\), the exact functor \(H^0(F(P))\) is isomorphic to 
  \begin{displaymath}
    \Phi_P(M) :=  \pi_B \left( \mathbf{R}\omega_{A^\opp \otimes_k B} P \overset{\mathbf{L}}{\otimes}_{\mathcal A} \mathbf{R}\omega_A M \right).
  \end{displaymath}
\end{theorem}

\begin{proof}
  Applying Corollary~\ref{corollary: Toen}, it suffices to provide a quasi-equivalence
  \begin{displaymath}
    G : \hinj{\QGr{A^\opp \otimes_k B}} \to \hproj{ (Q \mathcal A)^\opp \otimes_k Q \mathcal B},
  \end{displaymath}
  Using Corollary~\ref{corollary: duality is a duality}, we have a quasi-equivalence
  \begin{displaymath}
    \hproj{ (Q \mathcal A)^\opp \otimes_k Q \mathcal B} \cong \hproj{ Q \mathcal A^\opp \otimes_k Q \mathcal B}. 
  \end{displaymath}
  From Lemma~\ref{lemma: another model for QA otimes QB} we have a quasi-fully faithful functor 
  \begin{displaymath}
    \imath \colon Q \mathcal A^\opp \otimes_k Q \mathcal B \to \hinj{\QGr{A^\opp \otimes_k B}}
  \end{displaymath}
  which induces a dg-functor
  \begin{displaymath}
    \imath^\ast \colon \hinj{\QGr{A^\opp \otimes_k B}} \to \dgMod{Q \mathcal A^\opp \otimes_k Q \mathcal B}
  \end{displaymath}
  mapping an object \(P\) of \(\hinj{\QGr{A^\opp \otimes_k B}}\) to the dg-functor
  \[\begin{tikzcd}[row sep=tiny]
  \left(Q\A^\opp \otimes Q\B\right)^\opp \arrow{r} & \CH{k}\\
  E \arrow[mapsto]{r} & \hinj{\QGr{A^\opp \otimes_k B}}(iE, P).
  \end{tikzcd}\]
  We first note that, because the image of objects of \(Q\A^\opp \otimes Q\B\) are compact in \(\hinj{\QGr{A^\opp \otimes_k B}}\), for any set \(J\) the natural map
  \[\imath^\ast\left(\bigoplus_{j \in J} P_j\right)\to \bigoplus_{j \in J} \imath^\ast(P_j)\]
    is a weak equivalence, so \(\imath^\ast\) is continuous.
    By identifying \(P_{u,v} = \pi_{A^\opp \otimes_k B}(A(u) \otimes_k B(v))\) as an object of \(\dgMod{Q\A^\opp \otimes Q\B}\), we obtain the weak equivalence
    \[\imath^\ast(P) = \hinj{\QGr{A^\opp \otimes_k B}}(\imath(-), P) \cong \dgMod{Q\A^\opp \otimes_k Q\B}(-, P)\]
    and, consequently, the weak equivalence
    \begin{eqnarray*}
      \dgMod{Q\A^\opp \otimes Q\B}(i^\ast(P_{u,v}), \imath^\ast(P_{u^\prime, v^\prime}))
      &\cong& \imath^\ast(P_{u^\prime, v^\prime})(P_{u,v}) \\
      &=& \hinj{\QGr{A^\opp \otimes_k B}}(P_{u,v}, P_{u^\prime, v^\prime}).
    \end{eqnarray*}
    Since the collection \(\{P_{u,v}\}_{\Z^2}\) are compact generators for \(\hinj{\QGr{A^\opp \otimes_k B}}\), it follows that \(\imath^\ast\) is a quasi-equivalence between \(\hinj{\QGr{A^\opp \otimes_k B}}\) and \(\hproj{Q\A^\opp \otimes Q\B}\), the full dg-subcategory of compact objects of \(\dgMod{Q\A^\opp \otimes Q\B}\), by Lemma~\ref{lemma: quasi-equivalence on gens is quasi-equivalence}

  Tracing out the quasi-equivalences, one just needs to manipulate 
  \begin{align*}
    \op{Hom} (\mathbf{R}Q_A A(x)^\vee \otimes_k \mathbf{R}Q_B B(y), P) & \cong \op{Hom} ( \mathbf{R}Q_B B(y) , \op{Hom}( \mathbf{R}Q_A A(x)^\vee, \mathbf{R}\omega_{A^\opp \otimes_k B} P)) \\
    & \cong \op{Hom} ( \mathbf{R}Q_B B(y) , \mathbf{R}\omega_{A^\opp \otimes_k B} P \overset{\mathbf{L}}{\otimes}_{\mathcal A} \mathbf{R}Q_A A(x) ) 
  \end{align*}
  using Propostion~\ref{proposition: vanishing of tensor} and Lemma~\ref{lemma: trace map}. This says that the induced continuous functor is
  \begin{displaymath}
    M \mapsto \pi_B \left( \mathbf{R}\omega_{A^\opp \otimes_k B} P \overset{\mathbf{L}}{\otimes}_{\mathcal A} \mathbf{R}\omega_A M \right). 
  \end{displaymath}
\end{proof}

The following statement is now a simple application of Theorem~\ref{theorem: derived morita for NCP} and results of \parencite{Lunts-Orlov}. 

\begin{corollary} \label{corollary: NCP morita}
  Let \(A\) and \(B\) be a delightful couple of connected graded \(k\)-algebras with \(k\) a field. Assume that there exists an equivalence
  \begin{displaymath}
    f : \mathrm{D} (\QGr{A}) \to \mathrm{D} (\QGr{B}).
  \end{displaymath}
  Then there exists an object \(P \in D ( \QGr{A^\opp \otimes_k B} )\) such that 
  \begin{displaymath}
    \Phi_P : \mathrm{D} ( \QGr{A}) \to \mathrm{D} (\QGr{B} )
  \end{displaymath}
  is an equivalence.
\end{corollary}

\begin{proof}
  Applying \parencite[Theorem 1]{Lunts-Orlov} we know there is a quasi-equivalence between the unique enhancements, i.e. there is an \( F \in [ \hinj{ \QGr{A}}, \hinj{ \QGr{B}} ]\) giving an equivalence
  \begin{displaymath}
    H^0(F) : H^0(\hinj{ \QGr{A} }) = \mathrm{D}(\QGr{A}) \to H^0(\hinj{ \QGr{B} }) = \mathrm{D}(\QGr{B}).
  \end{displaymath}
  Then, by Theorem~\ref{theorem: derived morita for NCP}, there exists a \(P \in \mathrm{D}(\QGr{A^\opp \otimes_k B})\) such that \(\Phi_P = H^0(F)\). 
\end{proof}

We wish to identify the kernels as objects of the derived category of an honest noncommutative projective scheme.
In general, one can only hope that kernels obtained as above are objects of the derived category of a noncommutative (bi)projective scheme.
However, we have the following special case in which we can collapse the \(\Z^2\)-grading to a \(\Z\)-grading.

\begin{corollary} \label{corollary: NCP morita degree 1}
  Let \(A\) and \(B\) be a delightful couple of connected graded \(k\)-algebras with \(k\) a field that are both generated in degree one.
  Assume that there exists an equivalence
  \begin{displaymath}
    f : \mathrm{D} (\QGr{A}) \to \mathrm{D} (\QGr{B}).
  \end{displaymath}
  Then there exists an object \(P \in D ( \QGr{A^\opp \times_k B} )\) that induces an equivalence
  \[\begin{tikzcd}[row sep=tiny]
  \mathrm{D}(\QGr{A}) \arrow{r}& \mathrm{D}(\QGr{B})\\
  M \arrow[mapsto]{r} & \pi_B\left(\mathbb{V}_\text{dg}(P) \otimes^\mathbf{L} \mathbf{R}\omega_A M\right)
  \end{tikzcd}\]
\end{corollary}

\begin{proof}
  The equivalence \(\mathbb{V}\) of Theorem~\ref{theorem: Van Rompay} extends naturally to a quasi-equivalence
  \[\mathbb{V}_\text{dg} \colon \hinj{\QGr{S}} \to \hinj{\QGr{T}}.\]
  Now choose \(P\) such that \(\mathbb{V}_\text{dg}(P)\) is homotopy equivalent to the kernel obtained by an application of Corollary~\ref{corollary: NCP morita}, so the desired equivalence is \(\Phi_{\mathbb{V}_\text{dg}(P)}\).
\end{proof}
