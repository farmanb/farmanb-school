\documentclass[10pt]{amsart}
\usepackage{amsmath,amsthm,amssymb,amsfonts,enumerate,mymath,mathtools,tikz-cd}
\openup 5pt
\author{Blake Farman\\University of South Carolina}
\title{Math 747:\\Homework 03}
\date{September 20, 2013}
\pdfpagewidth 8.5in
\pdfpageheight 11in
\usepackage[margin=1in]{geometry}

\begin{document}
\maketitle

\providecommand{\p}{\mathfrak{p}}
\providecommand{\m}{\mathfrak{m}}

\newtheorem{thm}{}
\newtheorem{lem}{Lemma}
\newtheorem{prop}{Proposition}
\theoremstyle{definition}
\newtheorem{defn}{Definition}[thm]

\newcommand{\A}{\mathbb{A}}

\begin{thm}
  Define smooth matps $m \colon S^3 \times S^3 \rightarrow S^3$ and $i \colon S^3 \rightarrow S^3$ so that $S^3$ is a group with multiplication map $m$ and inversion map $i$.
  \begin{proof}
    Define the following maps for $x = (x_1, x_2, x_3, x_4, y_1, y_2, y_3, y_4) \in \R^8$
    \begin{align*}
      m_1 \colon \R^8 &\rightarrow \R\\
      x &\mapsto x_1y_1 - x_2y_2 - x_3y_3 - x_4y_4,
    \end{align*}
    
    \begin{align*}
      m_2 \colon \R^8 &\rightarrow \R\\
      x &\mapsto x_1y_2 + x_3y_4 - x_4y_3 + x_2y_1,
    \end{align*}

    \begin{align*}
      m_3 \colon \R^8 &\rightarrow \R\\
      x &\mapsto x_1y_3 + x_4y_2 - x_2y_4 + x_3y_1,
    \end{align*}
    and
    \begin{align*}
      m_4 \colon \R^8 &\rightarrow \R\\
      x &\mapsto x_1y_4 + x_2y_3 - x_3y_2 + x_4y_1.
    \end{align*}
    Let $m = m_1 \times m_2 \times m_3 \times m_4 \colon \R^8 \rightarrow \R^4$.
    It is not difficult, but rather lengthy, to check that $m$ is associative.
    If $x \in S^3 \times S^3$, then with some routine algebra, we have 
    $$\sum_{i=1}^4 m_i(x)^2 = \sum_{i=1}^4\sum_{j=1}^4 (x_iy_j)^2 = \sum_{i=1}^4 x_i^2 \sum_{j=1}^4 y_j^2 = \sum_{i=1}^4 x_i^2 = 1.$$
    Hence $m$ maps elements of $S^3 \times S^3$ to elements of $S^3$.
    
    Observe that these are polynomials, so each of the partials exist and $m$ is a smooth map.
    Identify $S^3 \times S^3$ as the subspace 
    $$\left\{(x_1, x_2, x_3, x_4, y_1, y_2, y_3, y_4) \in \R^8 \;\middle\vert\; x_1^2 + x_2^2 + x_3^2 + x_4^2 = y_1^2 + y_2^2 + y_3^2 + y_4^2 = 1\right\} \subseteq \R^8.$$
    Then any open set of $S^3\times S^3$ is obtained as $U \cap S^3 \times S^3$ with $U$ open in $R^8$ and so, since $m$ is smooth at every point of $U$, it follows that $m$ is smooth on $S^3 \times S^3$.
    
    For $x = (x_1, x_2, x_3, x_4)$, define the map
    \begin{align*}
      i \colon \R^4 &\rightarrow \R^4\\
      x &\mapsto \frac{1}{x_1^2 + x_2^2 + x_3^2 + x_4^2} (x_1, -x_2, -x_3, -x_4).
    \end{align*}
    With some routine algebra, it's easy to see that $m_1(x,i(x)) m_1(i(x), x) = x_1^2 + x_2^2 + x_3^2 x_4^2$ and $m_i(x,i(x)) = m_i(i(x), x) = 0$ for $2 \leq i \leq 4$.
    Hence $m(i(x), x) = m(x, i(x)) = (1,0,0,0)$.
    Moreover, $m(x, (1,0,0,0)) = m((1,0,0,0), x) = x$ implies that $i(x)$ is the inverse of $x$.
    By a similar argument, as for $m$, this map is also smooth on $S^3$.
  \end{proof}
\end{thm}

\begin{thm}
  Construct a smooth atlas on $\mathbb{P}^1_\C$.
  Then prove that the map $p \colon S^3 \rightarrow \mathbb{P}^1_\C$ defined by $p(z) = [z]$ is smooth.
  What are the fibers?

  \begin{proof}
    Take as charts the maps
    \begin{align*}
      \varphi_1 \colon \mathbb{P}^1 \setminus \left\{[1 : 0]\right\} &\rightarrow \C & \text{and} & & \varphi_2 \colon \mathbb{P}^1 \setminus \left\{[0 : 1]\right\} & \rightarrow \C\\
             [z : w] & \mapsto \frac{z}{w} & & & [z : w] & \mapsto \frac{w}{z}
    \end{align*}
    It is easy to see that the inverses of these maps are
    \begin{align*}
      \varphi_1^{-1} \colon \C &\rightarrow \mathbb{P}^1 \setminus \left\{[1 : 0]\right\} & \text{and} & & \varphi_2^{-1} \colon \C & \rightarrow \mathbb{P}^1 \setminus \left\{[0 : 1]\right\}\\
      z & \mapsto \left[z : 1 \right] & & & z & \mapsto [1 : z].
    \end{align*}
    Namely, for $w \in \C^\times$ we have, 
    $$[z : w] \overset{\phi_1}{\mapsto} \frac{z}{w} \overset{\phi_1^{-1}}{\mapsto} [z/w : 1] \sim [z : w]\ 
    \text{and}\ z \overset{\phi_1^{-1}}{\mapsto} [z : 1] \overset{\phi_1}{\mapsto} z.$$
    For $z \in \C^\times$, we have
    $$[z : w] \overset{\phi_2}{\mapsto} \frac{w}{z} \overset{\phi_2^{-1}}{\mapsto} [1 : w/z] \sim [z : w]\ 
    \text{and}\ z \overset{\phi_2^{-1}}{\mapsto} [1 : z] \overset{\phi_2}{\mapsto} z.$$
    Moreover
    \begin{align*}
      \varphi_2 \circ \varphi_1^{-1}  = \varphi_1 \circ \varphi_2^{-1} \colon \C^\times &\rightarrow \C^\times\\
      z &\mapsto \frac{1}{z}
    \end{align*}
    is holomorphic, hence smooth.

    For the charts on $S^3$, take the usual stereographic projections from the north and south poles
    \begin{align*}
      \psi_1 \colon S^3 \setminus \left\{(0, 1)\right\} &\rightarrow \C \times \R& \text{and} & & \psi_2 \colon S^1 \setminus \left\{(0, -1)\right\} & \rightarrow \C \times \R\\
             (z, w) & \mapsto \frac{1}{1 - \Im{w}}(z, \Re{w}) & & & (z, w) & \mapsto \frac{1}{1 + \Im{w}}(z, \Re{w}).
    \end{align*}
    %Define the map
    %\begin{align*}
    %  t \colon \C \times \R &\rightarrow \C \times \R\\
    %  (z,x) &\mapsto \frac{2}{1 + \abs{z}^2 + x^2}
    %\end{align*}
    %so that
    The inverses are
    \begin{align*}
      \psi_1^{-1} \colon \C \times \R &\rightarrow S^3 \setminus \left\{(0,1)\right\} & \text{and} & & \psi_2^{-1} \colon \C \times \R & \rightarrow S^3 \setminus \left\{(0,-1)\right\}\\
      (z,x) & \mapsto \left(\frac{2z}{1 + \abs{z}^2 + x^2}, \frac{x + i(x^2 + \abs{z}^2 - 1)}{1 + \abs{z}^2 + x^2}\right) & & & (z,x) & \mapsto \left(\frac{2z}{1 + \abs{z}^2 + x^2}, \frac{x + i(1 - x^2 - \abs{z}^2)}{1 + \abs{z}^2 + x^2}\right).%(t(z,x)z, t(z,x)x + i(t(z,x) - 1)).
    \end{align*}

    Let $(z,x) \in \C \times \R$ be given.
    Then 
    \begin{eqnarray*}
      \varphi_1 \circ p \circ \psi_1^{-1}(z,x) &=& \varphi_1\left(\left[\frac{2z}{1 + \abs{z}^2 + x^2} : \frac{2x}{1 + \abs{z}^2 + x^2} + i\frac{x^2 + \abs{z}^2 - 1}{1 + \abs{z}^2 + x^2}\right]\right)\\
      &=& \frac{2z}{2x + i(\abs{z}^2 + x^2 - 1)}
    \end{eqnarray*}
    is smooth, provided that $\abs{z^2} \neq 1$.
    Similarly,
    $$\varphi_2 \circ p \circ \psi_1^{-1}(z,x) = \frac{2x + i(\abs{z}^2 + x^2 - 1)}{2z}$$
    is smooth, provided $z \neq 0$.
    The other two maps are
    \begin{eqnarray*}
      \varphi_1 \circ p \circ \psi_2^{-1}(z,x) &=& \varphi_1\left(\left[\frac{2z}{1 + \abs{z}^2 + x^2} : \frac{2x}{1 + \abs{z}^2 + x^2} + i\frac{1 - x^2 - \abs{z}^2}{1 + \abs{z}^2 + x^2}\right]\right)\\
      &=& \frac{2z}{2x + i(1 - \abs{z}^2 - x^2)}
    \end{eqnarray*}
    and
    $$\varphi_2 \circ p \circ \psi_2^{-1}(z,x) = \frac{2x + i(1 - \abs{z}^2 - x^2)}{2z}.$$
    The first is smooth provided $\abs{z}^2 \neq 1$ and the second is smooth provided $z \neq 0$.
    
    The fibers of this map are the sets $p^{-1}\left([z : w]\right) = \left\{\lambda(z,w) \;\middle\vert\; \lambda \in \C^\times, \abs{\lambda}^2 = 1\right\}$.
    This follows from observing that if $p(z,w) = [z : w] \sim [z^\prime : w^\prime] = p(z^\prime, w^\prime)$, then $(z^\prime, w^\prime) = (\lambda z, \lambda w)$ for some $\lambda \in C^\times$.
    Since $(z,w)$ and $(z^\prime, w^\prime)$ both lie on $S^3$, it follows that 
    $$1 = \abs{z^\prime}^2 + \abs{w^\prime}^2 = \abs{\lambda z}^2 + \abs{\lambda w}^2 = \abs{\lambda}^2\left(\abs{z}^2 + \abs{w}^2\right) = \abs{\lambda}^2.$$
  \end{proof}
\end{thm}

\begin{thm}
  Let $X$ be a smooth, compact, connected $n$-manifold with $n \geq 1$.
  Check that the set of $\mathcal{E}(X)$ of smooth complex-valued functions on $X$ is a ring under point-wise addition and multiplication.
  Is this ring Noetherian?

  \begin{proof}
    Let $f, g \in \mathcal{E}(X)$ be given and let $\varphi$ be a chart.
    Let $u_1$, $v_1$ be the component functions of $f$ and $u_2, v_2$ those of $g$.
    Since each of $f, g$ are smooth, all the partials of $u_1 \circ \varphi^{-1}, u_2 \circ \varphi^{-1}, v_1 \circ \varphi^{-1}, v_2 \circ \varphi^{-1}$ exist and are continuous, hence under point-wise addition and multiplication
    $$fg \circ \varphi^{-1} = (u_1 \circ \varphi^{-1} \cdot u_2 \circ  \varphi^{-1}, v_1 \circ \varphi^{-1} \cdot v_2 \circ  \varphi^{-1})$$
    and
    $$f + g = (u_1 \circ \varphi^{-1} + u_2 \circ \varphi^{-1}, v_1 \circ \varphi^{-1} + v_2 \circ \varphi^{-1})$$
    are also smooth.

    This ring is not Noetherian in general.
    Take $X = [-1,1]$ and consider the family of opens $U_n = (-1/n, 1/n)$ for $n \in \N$.
    Observe that $U_1 \supset U_2 \supset U_3 \supset \ldots$. and there is a natural contravariant inclusion of the ideals of functions that vanish on each $U_i$, $\mathfrak{a}_i = \left\{ f \in \mathcal{E}(X) \;\middle\vert\; f\mid_{U_i} = 0 \right\}$,
    \begin{equation}\label{3.1}
      \mathfrak{a}_1 \subseteq \mathfrak{a}_2 \subseteq \mathfrak{a}_3 \subseteq \ldots.
    \end{equation}
    %$$ \mathcal{E}(U_0) \subseteq \mathcal{E}(U_1) \subseteq \mathcal{E}(U_2) \subseteq \mathcal{E}(U_3) \subseteq \ldots $$
    %given by restriction.
    The functions
    $$f_n(x) = \left\{
    \begin{array}{ll}
      e^{\frac{-1}{x^2 - 1/n^2}} & \text{if}\ x \not \in U_n\\
      0 & x \in U_n.
    \end{array}
    \right.
    $$
    are such that $f_n \in \mathfrak{a}_n$ but $f_n \not \in \mathfrak{a}_{n+1}$, and hence the inclusions in \ref{3.1} are strict.
    Therefore $\mathcal{E}(X)$ does not satisfy the A.C.C. and is not Noetherian.
  \end{proof}
\end{thm}

\begin{thm}\label{ex4}
  Suppose that $x \in X$ is a point on a smooth $n$-manifold.
  
  \begin{defn}
    Define an equivalence relation $\sim$ on the set of smooth complex-valued functions on $X$ by making two functions $f$ and $g$ equivalen, $f \sim g$, if $f\mid_U = g\mid_U$ for some open neighbourhood $U$ of $x$.
    The quotient set is called the {\bf ring of germs of smooth functions at} $x$ and written $\mathcal{E}_x$.

    The equivalence class of $f \in \mathcal{E}(X)$ is written $[f] \in \mathcal{E}_x$.\\

    Check that $\mathcal{E}_x$ has a ring structure such that $f \mapsto [f]$ is a ring homomorphism.
    Is $\mathcal{E}_x$ Noetherian?
  \end{defn}

  \begin{proof}
    Let $f, g \in \mathcal{E}(X)$ be given.
    Define pointwise addition and multiplication in $\mathcal{E}_x$ by $[f] + [g] = [f + g]$ and $[f][g] = [fg]$.
    It remains to show that this is independent of choice of represntative.
    Towards this end, suppose that $f \sim f^\prime$ and $g \sim g^\prime$.
    There exist neighbourhoods $U \subseteq X$ and $V \subseteq X$ of $x$ such that $f\mid_U = f^\prime\mid_U$ and $g\mid_U = g^\prime\mid_U$, so we have
    $$(f + g)\mid_{U \cap V} = (f^\prime + g^\prime)\mid_{U \cap V}\ \text{and}\ (fg)\mid_{U \cap V} = (f^\prime g^\prime)\mid_{U \cap V}.$$
    Therefore $f + g \sim f^\prime + g^\prime$ and $fg \sim f^\prime g^\prime$ imply $[f + g] = [f^\prime + g^\prime]$ and $[fg] = [f^\prime g^\prime]$, as desired.		

    This ring is not Noetherian in general.
    Let $X = [-1,1]$ and consider the smooth functions
    \begin{align*}
      f_n \colon X &\rightarrow \R\\
      x &\mapsto \left\{
      \begin{array}{ll}
	\frac{1}{x^n}e^\frac{-1}{x^2} & \text{if}\ x \neq 0,\\ 
	0 & \text{if}\ x = 0
      \end{array}
      \right.
    \end{align*}
    for $n \geq 0$.
    First note that there is no neighbourhood of $0$ for which $f_n = 0$.
    Now observe that for $0 \leq m < n$ we have $x^{n-m}f_n(x) = f_m(x)$.
    Suppose that for some $g \in \mathcal{E}(X)$, $f_n(x) = gf_m(x)$ on some neighbourhood $U$ of $0$.
    Then it follows that for $0 \neq x \in U$
    $$\frac{1}{x^m}e^\frac{-1}{x^2}\left(\frac{1}{x^{n-m}} - g(x)\right) = 0$$
    implies that $g(x) = 1/x^{n-m}$, which is absurd.
    Hence in $\mathcal{E}_0$ we have the ascending chain
    $$([f_0]) \subset ([f_1]) \subset ([f_2]) \subset \ldots$$
    which does not stabilize.
  \end{proof}
\end{thm}

\begin{thm}
  Let $\mathcal{E}_x$ be as in the previous problem.
  Fix a chart $\phi \colon U \rightarrow V$ on $X$ with the property that $x \in U$ and $\phi(x) = 0$.
  \begin{enumerate}
  \item\label{5.1}
    Is the map $\mathcal{E}_x \rightarrow \R$ defined by
    $$f \mapsto f \circ \phi^{-1}(0)$$
    well-defined?
    If well-defined, does the map depend on the choice of $\phi$?
  \item
    Is the subset $I_1 = \left\{ [f] \in \mathcal{E}_x \;\middle\vert\; f \circ \phi^{-1}(0) = 0 \right\}$ an ideal?
    If an ideal, is $I_1$ independent of the choice of $\phi$?
    Is $I_1$ a prime ideal?
  \item
    Is the map $\mathcal{E}_x \rightarrow[t_1, \ldots, t_n]/(t^2)$ defined by
    $$[f] \mapsto f\circ \phi^{-1}(0) + \sum_{i=1}^n \frac{\partial f \circ \phi^{-1}}{\partial x_i}(0) \cdot t_i$$
    well-defined?
    If well-defined, does the map depend on the choice of $\phi$?
  \item
    Is the subset $I_2 = \left\{ [f] \in \mathcal{E}_x \;\middle\vert\; f \circ \phi(0) = \frac{\partial f \circ \phi^{-1}}{\partial x_1}(0) = \frac{\partial f \circ \phi^{-1}}{\partial x_n}(0) = 0\right\}$ an ideal?
    If an ideal, is $I_2$ independent of the choice of $\phi$?
    Is $I_2$ a prime ideal?
  \end{enumerate}
  
  \begin{proof}
    \begin{enumerate}
    \item
      The map $f \mapsto f \circ \phi^{-1}(0)$ is well-defined and independent of the choice of chart, so long as $\phi(x) = 0$.
      If $f \sim f^\prime$, we have that there exists some neighbourhood $U^\prime$ of $x$ with $f\mid_{U^\prime} = f^\prime\mid_{U^\prime}$.
      Hence $f\mid_{U \cap U^\prime} = f^\prime\mid_{U \cap U^\prime}$ and so in particular at $x = \phi^{-1}(0)$, we have
      $$f \circ \phi^{-1}(0) = f(x) = f^\prime(x) = f^\prime \circ \phi^{-1}(0).$$
    \item
      This is a maximal, hence prime, ideal independent of the choice of $\phi$, provided $\phi(0) = x$.
      Observe that the map from part~(\ref{5.1}) is actually a surjective ring homomorphism with kernel $I_1$.
      For any pair $[f], [g] \in \mathcal{E}_x$
      $$(f + g)\circ \phi^{-1}(0) = (f + g)(x) = f(x) + g(x) = f \circ \phi^{-1}(0) + g \circ \phi^{-1}(0)$$
      and 
      $$(fg) \circ \phi^{-1}(0) = fg(x) = f(x)g(x) = (f \circ \phi^{-1}(0))(g \circ \phi^{-1}(0)).$$
      Surjectivity follows from the fact that constant functions are smooth.
      Therefore $\mathcal{E}_x / I_1 \cong \R$ gives maximality, as desired.
    \item
      The map $[f] \mapsto f\circ \phi^{-1}(0) + \sum_{i=1}^n \frac{\partial f \circ \phi^{-1}}{\partial x_i}(0) \cdot t_i$ is well-defined and independent of the choice of chart, for essentially the same reason as in part~(\ref{5.1}).
      If $f \sim f^\prime$, then there exists some neighbourhood $U^\prime$ of $x$ with $f\mid_{U^\prime} = f^\prime\mid_{U^\prime}$.
      %Restrict the domain of $\phi^{-1}$ to $\phi(U \cap U^\prime)$, which it should be noted includes $0$.
      Note that $\phi(U \cap U^\prime)$ is a neighbourhood of $0$.
      For any ball, $B$, about $0$ of sufficiently small radius such that $B \subseteq \phi(U \cap U^\prime)$, we have that $f\mid_{\phi^{-1}(B)} = f^\prime\mid_{\phi^{-1}(B)}$.
      Hence $f\circ\phi^{-1}\mid_B = f^\prime\circ\phi^{-1}\mid_B$ holds. 
      Since the behavior of the partials of $f \circ \phi^{-1}$ depend only on the limit of the difference quotient (in the appropriate direction) in balls about $0$, it follows that the partials agree.
      %For two smooth functions $f \sim f^\prime$ we have $f\mid_{U \cap U^\prime} = f^\prime\mid_{U \cap U^\prime}$, which ensures that the secant lines are identical in a small neighbourhoods of $x$.
    \item
      This set is an ideal contained in $I_1$.
      If $[f], [g] \in I_2$ and $[h] \in \mathcal{E}_x$, then
      $$\frac{\partial}{\partial x_i}((f + g) \circ \phi^{-1})(0) = \frac{\partial}{\partial x_i}(f\circ \phi^{-1} + g \circ \phi^{-1})(0) = \frac{\partial}{\partial x_i}(f \circ \phi^{-1})(0) + \frac{\partial}{\partial x_i}(g\circ \phi^{-1})(0) = 0$$
      and
      $$\frac{\partial}{\partial x_i} (fh\circ\phi^{-1})(0) = \frac{\partial}{\partial x_i} (f\circ \phi^{-1})(0) \cdot (h \circ\phi^{-1}(0)) + (f\circ \phi^{-1})(0) \cdot \frac{\partial}{\partial x_i} (h \circ\phi^{-1})(0) = 0.$$
      This ideal is not, in general, prime.
      Take two elements of $[f], [g] \in I_1 \setminus I_2$.
      Then by the above, it follows that 
      $$\frac{\partial}{\partial x_i} (fg \circ \phi^{-1})(0) = 0$$
      holds for each $i = 1, \ldots, n$.

      Take for instance $X$ as in the proof of part~(\ref{ex4}).
      Let $f(x) = g(x) = x \in \mathcal{E}_0$.
      Since $f(0) = g(0) = 0$ and $df/dx = dg/dx = 1$, it follows that $[f], [g] \in I_1 \setminus I_2$.
      However, $fg(0) = 0$ and $d(fg)/dx(0)= 0$ imply that $fg \in I_2$.
    \end{enumerate}
  \end{proof}
\end{thm}
\end{document}
