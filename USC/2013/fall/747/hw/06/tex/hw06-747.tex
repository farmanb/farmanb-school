\documentclass[10pt]{amsart}
\usepackage{amsmath,amsthm,amssymb,amsfonts,enumerate,mymath,mathtools,tikz-cd}
\openup 5pt
\author{Blake Farman\\University of South Carolina}
\title{Math 747:\\Homework 03}
\date{September 27, 2013}
\pdfpagewidth 8.5in
\pdfpageheight 11in
\usepackage[margin=1in]{geometry}

\begin{document}
\maketitle

\providecommand{\p}{\mathfrak{p}}
\providecommand{\m}{\mathfrak{m}}

\newtheorem{thm}{}
\newtheorem{lem}{Lemma}
\newtheorem{prop}{Proposition}
\theoremstyle{definition}
\newtheorem{defn}{Definition}[thm]

\newcommand{\A}{\mathbb{A}}

\begin{thm}
  Let $X = \C/(\Z + i\Z)$, $Y = \C$, and 
  \begin{align*}
    q \colon Y & \rightarrow X\\
    z & \mapsto z \pmod{\Z + i\Z},
  \end{align*}
  the quotient map.
  
  Consider the following curve in $X$
  \begin{align*}
    u \colon [0,1] & \rightarrow X\\
    t & \mapsto t \pmod{\Z + i\Z}.
  \end{align*}
  Construct a lifting $\tilde{u} \colon [0,1] \rightarrow Y$ of $u$ with $\tilde{u}(0) = 0$.
  What is $\tilde{u}(1)$?

  \begin{proof}
    Define the map
    \begin{align*}
      \tilde{u} \colon [0,1] & \rightarrow Y\\
      t & \mapsto t.
    \end{align*}
    Observe that $\tilde{u}$ is continuous and $q \circ \tilde{u}(t) = q(t) = t \pmod{\Z + i\Z} = u(t)$.
    Therefore $\tilde{u}$ is a lifting of $u$ and $\tilde{u}(1) = 1$.
  \end{proof}
\end{thm}

\begin{thm}
  Let $X = \C/(\Z + i\Z)$, $Y = \C$, and 
  \begin{align*}
    p \colon Y & \rightarrow X\\
    z & \mapsto 2 \cdot z \pmod{\Z + i\Z},
  \end{align*}
  the multiplication by $2$ map.
  
  Consider the following curve in $X$
  \begin{align*}
    u \colon [0,1] & \rightarrow X\\
    t & \mapsto t \cdot i \pmod{\Z + i\Z}.
  \end{align*}
  Construct a lifting $\tilde{u} \colon [0,1] \rightarrow Y$ of $u$ with $\tilde{u}(0) = 0$.
  What is $\tilde{u}(1)$?
  
  \begin{proof}
    Define the map
    \begin{align*}
      \tilde{u} \colon [0,1] & \rightarrow Y\\
      t & \mapsto \frac{1}{2}it.
    \end{align*}
    Observe that $\tilde{u}$ is continuous and 
    $$p \circ \tilde{u}(t) = p\left(\frac{1}{2}it\right) = it \pmod{\Z + i\Z} = u(t).$$
    Therefore $\tilde{u}$ is a lifting of $u$ and $\tilde{u}(1) = \frac{1}{2}i$.
    
  \end{proof}
\end{thm}

\begin{thm}
  Read Theorem 4.10 in Forster.
  Assume the maps $p$ and $q$ satisfy the curve lifting property.
  Use Problems 1 and 2 to prove that $\pi_1(X, 0)$ is non-trivial.
  
  Guess what group $\pi_1(X, 0)$ is.
  Can you prove your guess is correct?
  
  \begin{proof}
    Observe that the curves from (1) and (2) are both closed loops about $0 \pmod{\Z + i\Z}$ since $1, i \in \Z + i\Z$.
    Suppose to the contrary that $H \colon [0,1] \times [0,1] \rightarrow X$ is a homotopy with $u_0 = H(t,0)$ the curve from (1) and $u_1 = H(t,1)$ the curve from (2).
    Since $q$ has the curve lifting property, we note that each curve $u_s(t) = H(t,s)$ lifts to $\tilde{u}_s \colon [0,1] \rightarrow Y$.
    Now since we have a lift $u_1$ to 
    \begin{align*}
      \tilde{u}_1 \colon [0,1] & \rightarrow Y\\
      t & \mapsto t \cdot i
    \end{align*}
    since $q \circ \tilde{u}_1(t) = q(ti) = t \cdot i \pmod{\Z + i\Z} = u_1(t)$, and this is lift is unique, this must be the lift given by the lifting property.
    However, $u_0(1) = 1 \neq i = u_1(1)$, contrary to Theorem 4.10.
    Therefore no such homotopy exists and the fundamental group of $X$ is not trivial.
  \end{proof}
\end{thm}
\end{document}
