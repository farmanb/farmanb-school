\documentclass[10pt]{amsart}
\usepackage{amsmath,amsthm,amssymb,amsfonts,enumerate,mymath,mathtools,tikz-cd}
\openup 5pt
\author{Blake Farman\\University of South Carolina}
\title{Math 747:\\Homework 09}
\date{November 1, 2013}
\pdfpagewidth 8.5in
\pdfpageheight 11in
\usepackage[margin=1in]{geometry}

\begin{document}
\maketitle

\providecommand{\p}{\mathfrak{p}}
\providecommand{\m}{\mathfrak{m}}
\providecommand{\Deck}[1]{\operatorname{Deck}\left(#1\right)}

\newtheorem{thm}{}
\newtheorem{lem}{Lemma}
\newtheorem{prop}{Proposition}
\theoremstyle{definition}
\newtheorem{defn}{Definition}[thm]

\newcommand{\A}{\mathbb{A}}

\begin{thm}
  Let $X$ be a connected manifold and $p \colon \tilde{X} \rightarrow X$ be its universal covering.
  Let $G \subset \Deck{\tilde{X}/X}$ be a subgroup, $Y = \tilde{X}/G$ be the quotient of $\tilde{X}$ by the equivalence relation defined in 5.8 and $q \colon Y \rightarrow X$ be the map induced by $p$.
  Show that $q$ is a covering map which is Galois if and only if $G$ is a normal subgroup of $\Deck{\tilde{X}/X}$.
  In the latter case,
  $$\Deck{Y/X} \cong \Deck{\tilde{X}/X}/G.$$
  
  \begin{proof}
    Assume that $q \colon Y \rightarrow X$ is a Galois covering map.
    By Theorem 5.9, we may regard the canonical projection $\pi \colon \tilde{X} \rightarrow Y$ as the universal covering of $Y$, and $G$ as $\Deck{\tilde{X}/Y} \subset \Deck{\tilde{X}/X}$.
    Let $\sigma \in \Deck{\tilde{X}/X}$ and $\tilde{x} \in \tilde{X}$ be given.
    Observe that
    $$q \circ \pi (\tilde{x}) = p(\tilde{x}) = p \circ \sigma(\tilde{x}) = q \circ \pi \circ \sigma(\tilde{x}).$$
    Since $q$ is Galois, there is an induced map $\tilde{\sigma} \colon Y \rightarrow Y$ such that the diagram
    \begin{center}
      \begin{tikzcd}
	\tilde{X} \arrow{r}{\sigma} \arrow[swap]{d}{\pi} & \tilde{X} \arrow{d}{\pi}\\
	Y \arrow{r}{\tilde{\sigma}}& Y
      \end{tikzcd}
    \end{center}
    commutes.
    This induces a map $\Deck{\tilde{X}/X} \rightarrow \Deck{Y/Y}$.
    It is clear from the definition of $Y$ that $G \leq \ker\left({\Deck{\tilde{X}/X} \rightarrow \Deck{Y/Y}}\right)$.
    For the reverse inclusion, let $\sigma \in \ker\left({\Deck{\tilde{X}/X} \rightarrow \Deck{Y/Y}}\right)$ be given.
    It follows that $\tilde{\sigma} \colon Y \rightarrow Y$ is the identity on $Y$ so we have the commutative diagram
    \begin{center}
      \begin{tikzcd}
	\tilde{X} \arrow{r}{\sigma} \arrow{d}{\pi} & \tilde{X} \arrow{d}{\pi}\\
	Y \arrow[equal]{r} & Y
      \end{tikzcd}
    \end{center}
    and so $\tilde{\sigma} \in \Deck{\tilde{X}/Y}$.
    This gives the short exact sequence
    $$0 \rightarrow \Deck{\tilde{X}/Y} \rightarrow \Deck{\tilde{X}/X} \rightarrow \Deck{Y/X} \rightarrow 0$$
    from which it follows that $\Deck{\tilde{X}/Y}$ is normal and 
    $$\Deck{\tilde{X}/X}/\Deck{\tilde{X}/Y} \cong \Deck{Y/X}$$
    
    Convserely, assume that $G \leq \Deck{\tilde{X}/X}$.
    Let $y_0, y_1 \in Y$ be such that $q(y_0) = q(y_1)$.
    Choose $\tilde{x}_0, \tilde{x}_1 \in \tilde{X}$ such that $\pi(\tilde{x}_0) = y_0$ and $\pi(\tilde{x}_1) = y_1.$
    Observe that
    $$p(\tilde{x}_0) = q \circ \pi (\tilde{x}_0) = q(y_0) = q(y_1) = q \circ \pi(\tilde{x}_1) = p(\tilde{x}_1).$$
    Since $p$ is Galois, there exists some $\sigma \in \Deck{\tilde{X}/X}$ with $\sigma(\tilde{x}_0) = \tilde{x}_1$.
    Let $g \in G$ be given and observe that
    $$\sigma \circ g(\tilde{x}_0) = \sigma \circ g \circ \sigma^{-1}(\tilde{x}_1) = h (\tilde{x_1})$$
    for some $h \in G$.
    Therefore the map $\tilde{\sigma} \in \Deck{\tilde{X}/X}/G$ is such that $\tilde{\sigma}(y_0) = y_1$, and so $q$ is Galois.
  \end{proof}
  
\end{thm}
\end{document}
