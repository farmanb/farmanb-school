\documentclass[10pt]{amsart}
\usepackage{amsmath,amsthm,amssymb,amsfonts,enumerate,mymath,mathtools,tikz-cd}
\openup 5pt
\author{Blake Farman\\University of South Carolina}
\title{Math 747:\\Homework 08}
\date{September 25, 2013}
\pdfpagewidth 8.5in
\pdfpageheight 11in
\usepackage[margin=1in]{geometry}

\begin{document}
\maketitle

\providecommand{\p}{\mathfrak{p}}
\providecommand{\m}{\mathfrak{m}}

\newtheorem{thm}{}
\newtheorem{lem}{Lemma}
\newtheorem{prop}{Proposition}
\theoremstyle{definition}
\newtheorem{defn}{Definition}[thm]

\newcommand{\A}{\mathbb{A}}

\begin{thm}
  Let $X = \C \setminus \{\pm 1\}$ and $Y = \C \setminus \left\{\pi/2 + k\pi  \;\middle\vert\; k \in \Z \right\}$.
  Prove that $\Deck{Y \xrightarrow{\sin} X}$ consists of the following transformations
  \begin{enumerate}[(i)]
  \item
    $f_k(z) = z + 2k\pi, k \in \Z$,
  \item
    $g_k(z) = -z + (2k + 1)\pi, k \in \Z$.
  \end{enumerate}
  Calculate the products $f_k \circ f_\ell$, $f_k \circ g_\ell$, $g_l \circ f_k$, $g_k \circ g_\ell$.

  \begin{proof}
    By Exercise 4.1 it is clear that $\left\{f_k \;\middle\vert\; k \in \Z \right\} \cup \left\{g_k \;\middle\vert\; k \in \Z \right\} \subseteq \Deck{Y \xrightarrow{\sin} X}$.
    To see the reverse inclusion, let $f \in \Deck{Y \xrightarrow{\sin} X}$ be given.
    From the commutative diagram
    \begin{center}
      \begin{tikzcd}
        Y \arrow{r}{f} \arrow[swap]{rd}{\sin} & Y\arrow{d}{\sin}\\
        & X
      \end{tikzcd}
    \end{center}
    it follows that
    $$e^{if(z)} - e^{iz} = -\frac{e^{if(z)} - e^{iz}}{e^{i(f(z) + z)}},$$
    from which it follows that either $e^{i(f(z) + z)} = -1$ or $e^{if(z)} = e^{iz}$.
    In the first case it follows that
    $$f(z) = (2k + 1)\pi - z = g_k(z)$$
    for some $k \in \Z$.
    In the second case, we obtain 
    $$f(z) = z + 2k\pi = f_k(z)$$
    for some $k \in \Z$.
  \end{proof}
\end{thm}

\begin{thm}
  Determine the covering transformations of
  $$\tan \colon \C \rightarrow \mathbb{P}^1 \setminus \{\pm i\}.$$
  
  \begin{proof}
    From Exercise 4.4 it is clear that $\left\{ z \mapsto z + k\pi \;\middle\vert\; k \in \Z \right\} \subseteq \Deck{\C \xrightarrow{\tan} \mathbb{P}^1 \setminus \{\pm i\}}$.
    To see the reverse inclusion, let $f \in \Deck{\C \xrightarrow{\tan} \mathbb{P}^1 \setminus \{\pm i\}}$ be given.
    From the commutative diagram
    \begin{center}
      \begin{tikzcd}
        \C \arrow{r}{f} \arrow[swap]{rd}{\tan} & \C\arrow{d}{\tan}\\
        & \mathbb{P}^1 \setminus \{\pm i\}
      \end{tikzcd}
    \end{center}
    it follows that
    $$e^{i2f(z)} - e^{i2z} = -(e^{i2f(z)} - e^{i2z}),$$
    whence $e^{i2f(z)} = e^{i2z}$.
    Therefore $f(z) = z + \pi$.
  \end{proof}
\end{thm}

\begin{thm}
  Let $\Gamma$, $\Gamma^\prime \subset \C$ be lattices and let
  $$f \colon \C / \Gamma \rightarrow \C / \Gamma^\prime$$
  be a non-constant holomorphic map with $f(0) = 0$.
  Show that there exists a unique $\alpha \in \C^\times$ such that $\alpha\Gamma \subset \Gamma^\prime$ and the following diagram is commutative
  \begin{center}
    \begin{tikzcd}
      \C \arrow[swap]{d}{\pi} \arrow{r}{F} & \C \arrow{d}{\pi^\prime}\\
      \C/\Gamma \arrow{r}{f} & \C/\Gamma^\prime
    \end{tikzcd}
  \end{center}
  where $F(z) = \alpha z$, and $\pi$ and $\pi^\prime$ are the canonical projections.
  Prove that $f$ is an unbranched covering map and
  $$\Deck{\C/\Gamma \xrightarrow{f} \C/\Gamma^\prime} \cong \Gamma^\prime / \alpha\Gamma.$$

  \begin{proof}
  \end{proof}
\end{thm}
\end{document}
