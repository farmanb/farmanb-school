\documentclass[10pt]{amsart}
\usepackage{amsmath,amsthm,amssymb,amsfonts,enumerate,mymath,mathtools,tikz-cd}
\openup 5pt
\author{Blake Farman\\University of South Carolina}
\title{Math 747:\\Homework 08}
\date{September 25, 2013}
\pdfpagewidth 8.5in
\pdfpageheight 11in
\usepackage[margin=1in]{geometry}

\begin{document}
\maketitle

\providecommand{\p}{\mathfrak{p}}
\providecommand{\m}{\mathfrak{m}}

\newtheorem{thm}{}
\newtheorem{lem}{Lemma}
\newtheorem{prop}{Proposition}
\theoremstyle{definition}
\newtheorem{defn}{Definition}[thm]

\newcommand{\A}{\mathbb{A}}

\begin{thm}
  \begin{enumerate}[(a)]
  \item
    Show that $\tan(\C) = \mathbb{P}^1 \setminus \{\pm i\}$ and
    $$\tan \colon \C \rightarrow \mathbb{P}^1 \setminus \{\pm i\}$$
    is a covering map.
  \item
    Let $X = \C \setminus \left\{it \;\middle\vert\; t \in \R,\, \abs{t} \geq 1\right\}$.
    Show that for $k \in \Z$ there exists a unique holomorphic $\arctan_k \colon X \rightarrow \C$ with
    $$\tan \circ \arctan_k = \operatorname{id}_X$$
    and
    $$\arctan_k(0) = k$$
    (the $k^\text{th}$ branch of $\arctan$).
  \end{enumerate}
  
  \begin{proof}
    \begin{enumerate}[(a)]
    \item
      To see that $\tan(\C) = \mathbb{P}^1$, first write
      \begin{equation}\label{1.1}
        \tan{z} = \frac{\sin{z}}{\cos{z}} = \frac{e^{iz} - e^{-iz}}{e^{iz} + e^{-iz}} = -i\frac{e^{i2z} - 1}{e^{i2z} + 1}.
      \end{equation}
      Observe that we may then rewrite $\tan$ as the composition
      $$z \mapsto e^{i2z} \mapsto \frac{e^{i2z} - 1}{e^{i2z} + 1} \mapsto -i\left(\frac{e^{i2z} - 1}{e^{i2z} + 1}\right).$$
      Then since the map
      \begin{align*}
        \C\setminus\{-1,0\} &\rightarrow \C\setminus\{\pm 1\}\\
        z &\mapsto \frac{z - 1}{z + 1}
      \end{align*}
      is a bijection, with inverse $z \mapsto \frac{z + 1}{1 - z}$ it follows that $\tan(\C) = \mathbb{P}^1 \setminus \{\pm i\}$.
      
      Suppose $\tan(z_1) = \tan(z_2)$.
      Using \eqref{1.1}, some routine algebra yields 
      $$e^{i2z_1} - e^{i2z_2} = -(e^{i2z_1} - e^{i2z_2}),$$
      whence $e^{i2z_1} = e^{i2z_2}$ implies $z_1 = z_2 + \pi$.
      For a point $y \in \mathbb{P}^1 \setminus \{\pm i\}$, choose a point $x \in \tan^{-1}(y)$.
      Using the fact that $\tan$ is a local homeomorphism, there exist neighbourhoods $V_0$ of $x$ and $U$ of $y$ with $\tan \colon V_0 \rightarrow U$ a homeomorphism.
      Letting $V_k = V_0 + k\pi$ we have
      $$\tan^{-1}(U) = \coprod_{k \in \Z} V_k.$$
      Therefore $\tan$ is a covering map, as desired.
    \item
      Regard $\mathbb{P}^1$ as $\C \cup \{\infty\}$.
      Fix $k \in \Z$ and let 
      $$L_k = \left\{z \in \C \;\middle\vert\; \Re{z} = \pi/2 + k\pi\right\} \setminus \{\pi/2 + k\pi\}.$$
      For $z = (\pi/2 + k\pi) + iy \in L_k$ we have
      $$\tan(s) = -i\frac{-e^{-2y} - 1}{-e^{-2y} + 1} = i \frac{e^{2y} + 1}{e^{2y} - 1} \in \left\{it \;\middle\vert\; t \in \R,\, \abs{t} \geq 1\right\}$$
      and these are the only such points of $\C$.
      Let
      $$S_k = \left\{z \in \C \;\middle\vert\; -\pi/2 + k\pi < \Re{z} < \pi/2 + k\pi\right\},$$
      so by the observation above $\tan \mid_{S_k} \colon S_k \rightarrow X$.
      Note that $\operatorname{id}_X(0) = 0 = \tan(k\pi)$ and that $X$ is simply connected.
      By Theorem 4.17 there exists a unique lift of $\operatorname{id}_X \colon X \rightarrow \mathbb{P}^1$
      \begin{center}
        \begin{tikzcd}
          X \arrow[dotted]{r}{\tilde{\operatorname{id}}_X} \arrow[swap]{rd}{\operatorname{id}_X} & \C \arrow{d}{\tan}\\
          & \mathbb{P}^1
        \end{tikzcd}
      \end{center}
      with $\tilde{\operatorname{id}_X}(0) = k\pi$.
      Since $\tilde{\operatorname{id}}_X \circ \tan = \operatorname{id}_X$, take $\arctan_k = \tilde{\operatorname{id}}_X$.
    \end{enumerate}
  \end{proof}
\end{thm}

\begin{thm}
\end{thm}

\begin{thm}
    Consider the following polynomials:
    \begin{align*}
        x^3 + x + t,\, &x^3 - x + t\\
        x^3 + tx + 1,\, &x^3 - 2tx + t\\
        x^3 - x - t,\, &x^3 + t^2x - t^3.
    \end{align*}
    If $f(x,y) \in \C[x,t]$ is one of these polynomials, then define
    $$X = \left\{ (x,t) \in \C^2 \;\middle\vert\; f(x,t) = 0\right\}$$
    and
    \begin{align*}
        p \colon X &\rightarrow \C\\
        (x,t) & \mapsto t.
    \end{align*}
    For at least two of the polynomials, answer the following questions:
    \begin{enumerate}
        \item
	Is $X$ a Riemann surface?
	(Note: You can assume $X$ is connected.  Why does this make the problem easier?)
        \item
	If $X$ is a Riemann surface, compute the ramification points of $p$?
    \end{enumerate}

    \begin{proof}
    \end{proof}
\end{thm}
\end{document}
