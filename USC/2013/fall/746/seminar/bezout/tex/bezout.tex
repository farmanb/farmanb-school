\documentclass[10pt]{amsart}
\usepackage{amsmath,amsthm,amssymb,amsfonts,enumerate,mymath,mathtools,tikz-cd}
\openup 5pt
\author{Blake Farman\\University of South Carolina}
\title{Math 746:\\B\'{e}zout's Theorem}
\date{September 18, 2013}
\pdfpagewidth 8.5in
\pdfpageheight 11in
\usepackage[margin=1in]{geometry}

\begin{document}
\maketitle

\providecommand{\p}{\mathfrak{p}}
\providecommand{\m}{\mathfrak{m}}

\newtheorem{thm}{Theorem}
\newtheorem{lem}{Lemma}
\newtheorem{prop}{Proposition}
\theoremstyle{definition}
\newtheorem{defn}{Definition}

\newcommand{\A}{\mathbb{A}}

\begin{defn}\label{curve}
	For a field $k$, an \it{algebraic (projective) curve of degree $d$}, $C$, is the set of solutions in $\mathbb{P}^2_k$
		$$C \colon F(X,Y,Z) = 0$$
	where $F(X,Y,Z) \in k[x,y,z]$ is a homogenous, degree $d$ polynomial.
	The \it{affine part} of $C$ is the set of solutions in $\A_k^2$ $$f(x,y) = F(x,y,1) = 0$$.
\end{defn}


\begin{thm}[B\'{e}zout] 
	Let $k$ be an algebraically closed field and $C_1$ and $C_2$ be projective curves with no common components.
	Then 
		$$\sum_{P \in C_1 \cap C_2} I(C_1 \cap C_2, P) = (\deg{C_1})(\deg{C_2})$$
	where the sum is over all points of $C_1 \cap C_2$ having coordinates in $k$.
	In particular, if $C_1$ and $C_2$ are smooth curves with only transversal intersections, then $\#\left\{C_1 \cap C_2\right\} \leq (\deg{C_1})(\deg{C_2})$; and in all cases there is an inequality
		$$\#\left\{C_1 \cap C_2 \right\} \leq (\deg{C_1})(\deg{C_2}).$$
\end{thm}

Throughout, we will work with $A = k[x,y]$, $F_1$ and $F_2$ homogenous polynomials as in Definition~\ref{def1}, and $f_1$, $f_2$ be the affine parts.
Let $n_1 = \deg{f_1}$ and $n_2 = \deg{f_2}$.

\begin{prop}\label{prop1}
	Then the number of intersection points of $C_1$ and $C_2$ in $\A^2$ satisfies
		$$\#(C_1 \cap C_2 \cap \A^2) \leq \dim A /(f_1,f_2) \leq n_1n_2.$$
	Here, the dimension is as a $k$-vector space.

	\begin{proof}
		Let $P_1, P_2, \ldots, P_m$ be $m$ distinct points in $\A^2$.
		Choose lines $L_i$ such that $L_i(P_i) = 0$ and $\alpha_j^i = L_i(P_j) \neq 0$ for all $i \neq j$.
		For each $1 \leq i \leq m$ define $$\beta_i = \prod_{j \neq i} \alpha_j^i$$
		Now define the polynomials $h_i = L_i/\beta_i$ and observe that $h_i(P_i) = 1$.
		Suppose that the $P_i$ lie in $C_1 \cap C_2$ and that there is a linear dependence modulo $(f_1, f_2)$
			$$c_1h_1 + \ldots + c_mh_k = g_1f_1 + g_2f_2.$$
		Observe that $f_1(P_i) = f_2(P_i) = 0$ and $h_i(P_i) = 1$ imply necessarily that $c_i = 0$ for each $1 \leq i \leq m$.
		This establishes our lower bound.

		For each integer $d$, let $A_d$ be the $k$-vector of polynomials of degree at most $d$ and let 
			$$W_d = A_{d - n_1}f_1 + A_{d - n_2}f_2.$$
		Observe that $W_d \subseteq (f_1, f_2)$ and, if $d < n_1 \vee n_2$, then $W_d = 0$.
		Since the monomials $x^iy^j$ with $k = i + j \leq d$ span $A_d$, we note that for each $k \leq d$ the number of monomials of degree $k$ is
			$$\#\left\{0 + k, 1 + (k-1), \ldots, (k-1) + 1, k + 0\right\} = k + 1.$$
		Hence we we have
			$$\dim{A_d} = \sum_{k=0}^d (k + 1) = \frac{(d + 1)(d + 2)}{2}.$$
		If $d \geq n_1 + n_2$, it should be clear from counting degrees that 
			$$A_{d - n_1 - n_2}f_1f_2 \subseteq A_{d - n_1}f_1 \cap A_{d - n_2}f_2.$$
		For the reverse containment, we note that $f_1$ and $f_2$ are supposed not to have any common components, so every element of $A_{d - n_1}f_1 \cap A_{d - n_2}f_2$ is necessarily divisible by $f_1f_2$.
		Thus we have 
		\begin{equation}\label{eq1.1}
			A_{d - n_1 - n_2}f_1f_2 = A_{d - n_1}f_1 \cap A_{d - n_2}f_2.
		\end{equation}

		Now for $f \neq 0$, define the map
		\begin{align*}
			\psi \colon A_{d-j} &\rightarrow A_{d-j}f\\
			g &\mapsto fg,
		\end{align*}
		and note that this a surjective map with trivial kernel, hence an isomorphism, which gives 
			$$\dim{A_{d-j}}f = \dim{A_{d-j}} = (d-j + 1)(d - j + 2)/2.$$
		Hence by linear algebra we have
		\begin{eqnarray*}
			\dim{W_d} &=& \dim_{A_{d - n_1}f_1} + \dim_{A_{d - n_1}f_1} - \dim{A_{d - n_1}f_1 \cap A_{d - n_2}f_2}\\
			&=& \dim_{A_{d - n_1}f_1} + \dim_{A_{d - n_1}f_1} - \dim{A_{d - n_1 - n_2}f_1f_2}\\
			&=& \frac{(d - n_1 + 1)(d - n_1 + 2)}{2} + \frac{(d - n_2 + 1)(d - n_2 + 2)}{2} - \frac{(d - n_1 - n_2 + 1)(d - n_1 - n_2 + 2)}{2}\\
			&=& \frac{(d + 1)(d + 2)}{2} - n_1n_2\\
			&=& \dim{A_d} - n_1n_2.
		\end{eqnarray*}
		Hence $\dim{A_d/W_d} = n_1n_2$.

		Now for any set of $n_1n_2 + 1$ polynomials from $A$, take $d$ sufficiently large so that $g_i \in A_d$ and $d \geq n_ 1 + n_2$.
		By the previous results, we have that the $g_i$ are linearly dependent modulo $W_d \subseteq (f_1, f_2)$.
		Therefore we have established
			$$\#(C_1 \cap C_2 \cap \A^2) \leq \dim A /(f_1,f_2) \leq n_1n_2.$$

	\end{proof}
\end{prop}

\begin{prop}\label{prop2}
	If $C_1$ and $C_2$ do not meet at infinity, then
		$$\dim A /(f_1,f_2) = n_1n_2.$$

	\begin{proof}
		First note that we need only establish $\dim A/(f_1, f_2) \geq n_1n_2$.
		Let $f(x) = \sum_{i,j} c_{i,j}x^iy^j$ with $c_{i,j} \in k$ be a given polynomial of degree $n$ in $A$.
		Let 
			$$F(X,Y,Z) = \sum_{i,j}c_{i,j}X^iY^jZ^{n - i - j}$$
		be the homogenous polynomial with affine part $f(x,y) = F(x,y,1)$.
		Define $f^*(x,y) = F(x,y,0)$, the homogenous part of highest degree.
		Since $f^*(x,y)$ is homogenous, we can write $f(x,y) = y^nf^*(x/y, 1)$.
		Regarding the latter as an element of $k\left[x/y\right]$, we have the factorization 
			$$f^*\left(\frac{x}{y}, 1\right) = \left(a_1\frac{x}{y} + b_1\right)\left(a_2\frac{x}{y} + b_2\right)\cdots\left(a_r\frac{x}{y} + b_r\right),\, r = \deg{f^*\left(\frac{x}{y}, 1\right)} \leq n$$		
		since $k$ is algebraically closed.
		Hence we have the factorization
			$$f^*(x,y) = y^{n-r}(a_1x + b_1y)(a_2x + b_2y)\cdots(a_rx + b_ry)$$
		over $A$.
		We note that the points at infinity of $f(x,y) = 0$ are precisely the points satisfying $F(X,Y,0) = 0$ are precisely the points satisfying $f^*(x,y) = 0$, namely the set of points given by the homogenous coordinates $[b_i : -a_i : 0]$.
		
		Since $C_1$ and $C_2$ do not meet at infinity, it is then clear that $f_1^*$ and $f_2^*$ do not share a common factor.
		Moreover, in this case $(f_1, f_2) \cap A_d = W_d$.
		The containment $W_d \subseteq A_d \cap (f_1, f_2)$ is clear.
		For the reverse containment, let $g_1$ and $g_2$ have minimal degree amongst elements $g_1f_1 + g_2f_2 \in A_d$.
		If $\deg{g_1} > d - n_1$ then it follows that
			$$0 = (g_1f_1)^* + (g_2f_2)^* = g_1^*f_1^* + g_2^*f_2^*,$$
		where the right-hand equality follows from the construction of the homogenous part of highest degree.
		Since $f_1^*$ and $f_2^*$ are relatively prime, it follows that for some $h$, $g_1^* = f_2^*h$ and $g_2^* = -f_1^*h$ must hold.
		Hence for some $r_i \in A$ with $\deg{r_i} < \deg{g_i^*}$ for $i = 1,2$ we can write
		$$g_1^* = f_2^*h + r_1\ \text{and}\ g_2^* = -f_1^*h + r_2,$$
		from which it follows easily that we can write 
			$$g_1 = f_2h + r_1^\prime\ \text{and}\ g_2 = -f_1h + r_2^\prime$$
		for some $r_i^\prime \in A$ with $\deg{r_i^\prime} < \deg{g_i}$
		But then
			$$f_1g_1 + f_2g_2 = f_1f_2h - f_2f_1h + f_1r_1^\prime + f_2r_2^\prime = f_1r_1^\prime + f_2r_2^\prime,$$
		contradicting minimality, which establishes the reverse containment.

		Now note that from the proof of Proposition~\ref{prop1} we have that there are $n_1n_2$ elements of $A_d$ that are linearly independent modulo $W_d$.
		Since $(f_1, f_2) \cap A_d = W_d$, they are also linearly independent modulo $(f_1, f_2)$, which implies
			$$\dim A/(f_1,f_2) \geq n_1n_2,$$
		as desired.
	\end{proof}
\end{prop}

We will now strengthen the inequality from Proposition~\ref{prop1}.  

\begin{defn}\label{localringpoint}
	Let $P$ be a point in $\A^2$.
	Define the {\it local ring of P}, $(\mathcal{O}_P, \mathfrak{m}_P)$, to be the localization of $A$ at the complement of the ideal of $P$, $I(P) = \left\{f \in A \;\middle\vert\; f(p) = 0 \right\}$.
\end{defn}

\begin{defn}\label{intmult}
	Define the {\it intersection multiplicity (intersection index) of $C_1$ and $C_2$ at $P$} to be
		$$I(C_1 \cap C_2, P) = \dim \mathcal{O}_P/(f_1, f_2)_P,$$
	where $(f_1, f_2)_P$ denotes the ideal of $\mathcal{O}_P$ generated by $f_1$ and $f_2$.
\end{defn}

\begin{prop}\label{prop3}
	For a suitably defined \it{intersection multiplicity} of $C_1$ and $C_2$ at $P$, $I(C_1 \cap C_2, P)$, we have
		$$\sum_{P \in C_1 \cap C_2 \cap \A^2} I(C_1 \cap C_2, P) \leq \dim A / (f_1, f_2).$$

	\begin{proof}
		First observe that $\mathcal{O}_P$ is easily identified as a subring of $k(x,y)$ containing $k$, hence has a natural $k$-vector space structure.
		Since vector spaces are projective modules, we have the exact sequence
		\begin{center}
			\begin{tikzcd}
				0 \arrow{r} & \mathfrak{m}_P \arrow{r} & \mathcal{O}_P \arrow{r}{\epsilon_P} &  k \arrow{r} & 0
			\end{tikzcd}
		\end{center}
		where $\epsilon_P$ is evaluation at $P$, which is split, giving $\mathcal{O}_P \cong k \oplus \mathfrak{m}_P$.
		
		Observe that any finite set of elements of $\mathcal{O}_P$ can be written over a common denominator.
		Suppose that $g_1/h, \ldots, g_r/h \in \mathcal{O}_P$, $g_i \in A$, are linearly independent modulo $(f_1, f_2)_P$.
		Consider the relation
			$$\alpha_1g_1 + \ldots + \alpha_rg_r = \beta_1 f_1 + \beta_2 f_2$$
		for some $\alpha_i \in k$ and $\beta_1, \beta_2 \in A.$
		Then in $\mathcal{O_P}$ we have
			$$\alpha_1(g_1/h) + \ldots + \alpha_r(g_r/h) = (\beta_1/h)f_1 + (\beta_2/h) f_2,$$
		which implies that $\alpha_1 = \ldots = \alpha_r = 0$.
		Hence $g_1, \ldots, g_r \in A$ are linearly independent modulo $(f_1, f_2)$, which in turn implies that the intersection multiplicity is finite; namely
			$$\dim \mathcal{O}_P/(f_1, f_2)_P \leq \dim A/(f_1, f_2) \leq n_1n_2$$
		
		%HERE

		Now observe that if $P \not \in C_1 \cap C_2$, then one of $f_i(P) \neq 0$ and so at least one of $f_i \in \mathcal{O}_P^\times$.
		Hence $I(C_1 \cap C_2, P) = 0$ in this case.
		If $P \in C_1 \cap C_2$, both lie in $\mathfrak{m}_P$ and thus $(f_1, f_2) \subseteq \mathfrak{m}_P$ holds.
		Viewing $\mathcal{O}_P = k \oplus \mathfrak{m}_P$ as a coproduct, it is clear that the map $1 \oplus \pi \colon k \oplus \mathfrak{m}_P \rightarrow k \oplus \mathfrak{m}_P/(f_1,f_2)_P$, where $\pi$ is the canonical projection homomorphism, is an isomorphism.
		Hence we have
			$$I(C_1 \cap C_2, P) = \dim{k} + \dim{\mathfrak{m}_P/(f_1, f_2)} = 1 + \dim{\mathfrak{m}_P/(f_1, f_2)_P}$$
		and $I(C_1 \cap C_2, P) \geq 1$ with equality if and only if $\mathfrak{m}_P = (f_1, f_2)_P$.
	
		Let $r$ be such that $r \geq \dim{\mathcal{O}_P/(f_1, f_2)_P}$.
		We aim to show that $\mathfrak{m}_P^r \subseteq (f_1, f_2)_P$.
		Given $r$ elements $t_1, t_2, \ldots, t_r \in \mathfrak{m}_P$, consider the ideals $J_0 = \mathfrak{m}_P$, $J_i = t_1t_2 \cdots t_i\mathcal{O}_P + (f_1, f_2)_P$ of $\mathcal{O}_P$, $J_{r + 1} = (f_1, f_2)_P.$
		Observe that we have a chain of ideals
		$$J_{r+1} \subseteq J_{r} \subseteq \ldots \subseteq J_2 \subseteq J_1 \subseteq J_0.$$
		If for each $1 \leq i \leq r$, $\dim{J_i/J_{i+1}} = \dim{J_i} - \dim{J_{i+1}} \geq 1$ then we have
			$$r < \sum_{i = 0}^r \dim{J_i/J_{i+1}}= \sum_{i = 0}^r \left(\dim{J_i} - \dim{J_{i+1}}\right) = \dim{m_P} - \dim{(f_1, f_2)_P} = \dim{m_P/(f_1, f_2)_P},$$
		which is absurd.
		Hence for some $i$, $J_i = J_{i+1}$.
		If $i = r$, then $t_1t_2 \cdots t_r \in (f_1, f_2)_P$ implies that $\mathfrak{m}_P^r \subseteq (f_1, f_2)_P$.
		If $i < r$, then for some $f \in \mathcal{O}_P$ and $g \in (f_1, f_2)_P$ we have
			$$t_1 t_2 \cdots t_i = t_1 t_2 \cdots t_{i+1}f + g,$$
		and so $t_1 t_2 \cdots t_i(1 - t_{i+1}f) = g \in (f_1, f_2)_P$.
		However, $\mathcal{O}_P$ is local, so $1 - t_{i+1}f$ is a unit and thus
			$$t_1 t_2 \cdots t_i = (1 - t_{i+1}f)^{-1}g \in (f_1, f_2)_P,$$
		which gives $t_1 t_2 \cdots t_r \in (f_1, f_2)_P$.
		Hence $\mathfrak{m}_P^r \subseteq (f_1, f_2)_P$.

		We have determined that there are at most $n_1n_2$ points in $C_1 \cap C_2 \cap \A^2$.
		As in Proposition~\ref{prop1}, we can construct a polynomial $h \in A$ such that $h(P) = 1$ and $h(Q) = 0$ for all $P \neq Q \in (C_1 \cap C_2 \cap A^2)$.
		Hence $h \in \mathcal{O}_P^\times$ and $h \in \mathfrak{m}_Q$ for each $Q \neq P$, and so for $r \geq \dim{\mathcal{O}_Q / (f_1, f_2)_Q}$, we have $h^r \in (f_1, f_2)_Q$ for $Q \neq P$.

		Suppose that the polynomials $g_1/h, \ldots, g_m/h$ span $\mathcal{O}_P/(f_1, f_2)_P$.
		Observe that for any element $f \in \mathcal{O}_P$, there exist $\alpha_1, \ldots, \alpha_r \in k$ such that
			$$f/h - (\alpha_1(g_1/h) + \ldots + \alpha_m(g_m/h)) \in (f_1, f_2)_P$$
		and so we have
			$$f - (\alpha_1g_1 + \ldots + \alpha_mg_m) \in (f_1, f_2)_P.$$
		In particular, we can find an element $g$ of $A$ that is congruent to $f  h^{-r}$ modulo $(f_1, f_2)_P$.
		Then $gh^r$ is congruent to $f$ modulo $(f_1, f_2)_P$ and congruent to 0 modulo $(f_1, f_2)_Q$.
		This shows that the map
		\begin{align*}
			\varphi \colon A &\rightarrow \prod_{P \in C_1 \cap C_2 \cap \A^2} \mathcal{O}_P/(f_1, f_2)_P\\
			f &\mapsto (\ldots, f \mod{ (f_1, f_2)_P}, \ldots)
		\end{align*}
		is a surjection.
		Finally, we observe that $(f_1, f_2) \subseteq \ker\varphi$ implies
			$$\dim{A/\ker\varphi} = \sum_{P} \dim{\mathcal{O}_P/(f_1, f_2)_P} = \sum_{P} I(C_1 \cap C_2, P) \leq \dim{A/(f_1, f_2)}.$$	
	\end{proof}
\end{prop}

\end{document}
