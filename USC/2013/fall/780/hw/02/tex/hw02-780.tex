\documentclass[10pt]{amsart}
\usepackage{amsmath,amsthm,amssymb,amsfonts,enumerate,mymath,mathtools}
\openup 5pt
\author{Blake Farman\\University of South Carolina}
\title{Math 780:\\Homework 01}
\date{September 4, 2013}
\pdfpagewidth 8.5in
\pdfpageheight 11in
\usepackage[margin=1in]{geometry}

\begin{document}
\maketitle

\providecommand{\p}{\mathfrak{p}}
\providecommand{\m}{\mathfrak{m}}

\newtheorem{thm}{}
\newtheorem{lem}{Lemma}

\begin{thm}\label{ex1}
  Let $a$, $b$, $c$, and $d$ denote positive integers.
  Prove each of the following:
  \begin{enumerate}[(a)]
  \item
    $a \mid b$ and $b \mid c$ implies $a \mid c$.
  \item
    $ac \mid bc$ implies $a \mid b$.
  \item
    $a \mid b$ and $c \mid d$ implies $ac \mid bd$.
  \end{enumerate}
  
  \begin{proof}
    \begin{enumerate}[(a)]
    \item
      Since $b \mid c$ and $a \mid b$ we have $c = bm$ and $b = an$ for some integerse $m$ and $n$.
      Therefore $c = bm = anm$ implies $a \mid c$.
    \item
      Write $bc = (ac)m = (am)c$ for some integer $m$.
      Subtracting $(am)c$ from both sides we obtain
      $$0 = bc - (am)c = (b - am)c$$
      which implies $b = am$ since $c$ is not zero.
      Therefore $a \mid b$.
    \item
      For some integers $m$ and $n$, write $b = am$ and $d = cn$.
      Then $bd = (ac)mn$, which implies $ac \mid bd$.
    \end{enumerate}
  \end{proof}
\end{thm}

\begin{thm}\label{ex2}
  Prove that if $n$ is an integer at least as large as $2$ which is composite, then $n$ has a prime divisor which is at most $\sqrt{n}$.

  \begin{proof}
    Suppose to the contrary that all the prime factors of $n$ are strictly larger as $\sqrt{n}$.
    Since $n$ is composite, there exist at least two primes $p$ and $q$, not necessarily distinct, dividing $n$.
    Hence for some integer $m \geq 1$ we have
    $$n = pqm > (\sqrt{n})^2 = n$$
    a contradiction.
    Therefore $n$ has a prime divisor which is at most $\sqrt{n}$.
  \end{proof}
\end{thm}

\begin{thm}\label{ex3}
  Observe that $n^4 + 4^n$ is prime for $n = 1$.
  Prove that $n^4 + 4^n$ is composite if $n$ is an integer strictly larger than 1.

  \begin{proof}
    First observe that if $n$ is even, then $n^4 + 4^n \neq 2$ and $2 \mid n^4 + 4^n$ imply that $n^4 + 4^n$ is not prime.
    Assume that $n$ is an odd integer at least as large as 3, and note that $(n+1)/2$ is thus an integer, whence
    $$n^4 + 4^n = (n^2)^2 + (2^n)^2 = (n^2 + 2^n)^2 - (n 2^{\frac{n + 1}{2}})^2 = (n^2 + 2^n - n 2^{\frac{n + 1}{2}})(n^2 + 2^n + n 2^{\frac{n + 1}{2}})$$
    is a factorization of $n^4 + 4^n$ in $\Z$.
    %\begin{eqnarray*}
    %  n^4 + 4^n &=& (n^2)^2 + (2^n)^2\\
      %&=& (n^2 + 2^n)^2 - 2(n^2)2^n\\
      %&=& (n^2 + 2^n)^2 - n^2 2^{n+1}\\
     % &=& (n^2 + 2^n)^2 - (n 2^{\frac{n + 1}{2}})^2\\
      %&=& (n^2 + 2^n - n 2^{\frac{n + 1}{2}})(n^2 + 2^n + n 2^{\frac{n + 1}{2}}).
    %\end{eqnarray*}
    It remains to show that neither factor is a unit.
    Observe that since $n \geq 3$ we have $n^2 + 2^n + n 2^{\frac{n + 1}{2}} \geq 29$, so it suffices to show that $n^2 + 2^n - n 2^{\frac{n + 1}{2}} \neq 1$.
    Suppose to the contrary that 
    $n^2 + 2^n - n 2^{\frac{n + 1}{2}} = 1$
    so that
    $$n^4 + 4^n = n^2 + 2^n + n 2^{\frac{n + 1}{2}} = (n^2 + 2^n - n 2^{\frac{n + 1}{2}}) + 2n2^{\frac{n + 1}{2}}= 1 + 2n2^{\frac{n + 1}{2}}.$$
    With some elementary calculus, it is easy to see that $n \leq 2^{\frac{n+1}{2}}$ holds for $n \geq 3$.
    Combined with the observation that $2^{n-1} \geq 4$ we have
    $$n^4 + 4^n =  1 + 2n2^{\frac{n + 1}{2}} \leq 1 + 2(2^{n+1}) \leq 1 + (2^{n-1})(2^{n+1}) = 1 + 4^n,$$
    %\begin{eqnarray*}
     % n^4 + 4^n &=&  1 + 2n2^{\frac{n + 1}{2}} \leq& 1 + 2(2^{n+1}) \leq 1 + 2^{n-1}2^{n+1} = 1 + 4^n
      %&=& 1 + (2^{\frac{n + 1}{2}})^3\\
      %&\leq& 1 + (2^{\frac{n + 1}{2}})^4\\
      %&=& 1 + 4(4^n)\\
  %\end{eqnarray*}
    which implies $n^4 \leq 1$, a contradiction.
    Therefore $n^4 + 4^n$ is composite for $n > 1$.
%    \begin{eqnarray*}
%      1 &=& (n^2 + 2^n - n 2^{\frac{n + 1}{2}})^2\\
%      &=& (n^2 + 2^n)^2 - 2(n^2 + 2^n)n2^{\frac{n + 1}{2}} + (n2^{\frac{n + 1}{2}})^2\\
%      &=& (n^2 + 2^n)^2 - (n2^{\frac{n + 1}{2}})^2 - 2(n^2 + 2^n)n2^{\frac{n + 1}{2}} + 2(n2^{\frac{n + 1}{2}})^2\\
%      &=& n^4 + 4^n - 2n2^{\frac{n + 1}{2}}((n^2 + 2^n) - n2^{\frac{n + 1}{2}})\\
%      &=& n^4 + 4^n - 2n2^{\frac{n + 1}{2}}
%    \end{eqnarray*}
%    from which it follows that $n^4 + 4^n = 1 + 2n2^{\frac{n + 1}{2}} < n^2 + 2^n + n 2^{\frac{n + 1}{2}}$.
  \end{proof}
\end{thm}

\begin{thm}\label{ex4}
  Find the complete set of integer solutions in $x$ and $y$ to
  $$821x + 1997y = 24047.$$

  \begin{proof}
    Using the Euclidean algorithm we have
    \begin{eqnarray*}
      1997 &=& 821(2) + 355\\
      821 &=& 355(2) + 111\\
      355 &=& 111(3) + 22\\
      111 &=& 22(5) + 1
    \end{eqnarray*}
    Then we have the equations $1 = 111 - 5(22)$, $22 = 355 - 3(111)$, $111 = 821 - 2(355)$, and $355 = 1997 - 2(821)$.
    Substituting accordingly, we arrive at
    \begin{eqnarray*}
      1 &=& 111 - 5(22)\\
      &=& 16(111) - 5(355)\\
      &=& 16(821) - 37(355)\\
      &=& 90(821) - 37(1997).
    \end{eqnarray*}
    Hence multiplying both sides by $24047$ we arrive at the solution 
    $$2164230(821) - 889739(1997) = 24047.$$
    The solution set is then $\left\{(2164230 + 1997k, 889739 + 821k) \;\middle\vert\; k \in \Z\right\}$.
  \end{proof}
\end{thm}

\begin{thm}\label{ex5}
  Prove that for every non-constant polynomial $f(x)$ with integer coefficients, there is an integer $m$ such that $f(m)$ is composite.
  
  \begin{proof}
    Write $f(x) = a_nx^n + a_{n-1}x^{n-1} + \ldots + a_0$, $a_i \in \Z$. 
    Observe that we may assume that $f(m) = p$ is prime for some integer $m$.
    Since $\abs{f}$ is an increasing function, it follows that $\abs{f(m + kp)} \rightarrow \infty$ as $k \rightarrow \infty$, so for sufficiently large $k$ we have $\abs{f(m + kp)} > p$.
    By expanding we have
    \begin{eqnarray*}
      f(m + kp) &=& \sum_{j=0}^{n}a_{n-j}\sum_{i=0}^{n-j} {n-j \choose i}m^{n-j-i}(kp)^i\\
      %&=& a_n\sum_{i=0}^n{n \choose i}m^{n-i}(kp)^{i} + a_{n-1}\sum_{i=0}^{n-1}{n-1 \choose i}m^{n-1-i}(kp)^{i} + \ldots + a_1m + a_1(kp)+ a_0\\
      &=& \sum_{i=0}^n a_{n-i}m^{n-i} + \sum_{j=0}^{n-1}a_{n-j}\sum_{i=1}^{n-j} {n-j \choose i}m^{n-j-i}(kp)^i\\
      &=& p + \sum_{j=0}^{n}a_{n-j}\sum_{i=1}^{n-j} {n-j \choose i}m^{n-j-i}(kp)^i \equiv 0 \pmod{p}
      \end{eqnarray*}
    is thus composite, as desired.
  \end{proof}
\end{thm}

\begin{thm}\label{ex6}
  A large furniture store sells 6 kinds of dining room suites, whose prices are $\$231$, $\$273$, $\$429$, $\$600.60$, $\$1001$, and $\$1501.50$, respectively.
  Once a South American buyer came, purchased some suites, paid the total amount due, $\$13519.90$, and sailed for South America.
  The manager lost the duplicate bill of sale and had no other memorandum of each kind of suite purchased.
  Help him by determining the exact number of suites of each kind the South American buyer bought.
  (Don't forget to show that your solution is unique).

  \begin{proof}
    The number of suites of each kind can be modeled as integer solutions to the equation
    $$2310x + 2730y + 4290z + 6006s + 10010t + 15015u = 135199.$$
    The table below gives the factorizations of each, with the entry in the table indicating the power of the prime present in the factorization.
    \begin{center}
      \begin{tabular}{|c|c|c|c|c|c|c|}
        \hline
        & 2 & 3 & 5 & 7 & 11 & 13\\
        \hline
        2310 & 1 & 1 & 1 & 1 & 1 & 0\\
        \hline
        2730 & 1 & 1 & 1 & 1 & 0 & 1\\
        \hline
        4290 & 1 & 1 & 1 & 0 & 1 & 1\\
        \hline
        6006 & 1 & 1 & 0 & 1 & 1 & 1\\
        \hline
        10010 & 1 & 0 & 1 & 1 & 1 & 1\\
        \hline
        15015 & 0 & 1 & 1 & 1 & 1 & 1\\
        \hline
      \end{tabular}
    \end{center}
    Observe that from the zeroes down the diagonal, we can construct the following congruences
    \begin{eqnarray*}    
      9x &\equiv& 12 \pmod{13}\\
      2y &\equiv& 9 \pmod{11}\\
      6z &\equiv& 1 \pmod{7}\\
      s &\equiv& 4 \pmod{5}\\
      2t &\equiv& 1 \pmod{3}\\
      u &\equiv& 1 \pmod{2}.\\
    \end{eqnarray*}
    The solutions to these congruences are $x \equiv 10 \pmod{13}$, $y \equiv 10 \pmod{11}$, $z \equiv 6 \pmod{7}$, $s \equiv 4 \pmod{5}$, $t \equiv 2 \pmod{3}$, and $u \equiv 1 \pmod{2}$.
    Checking the least representatives for each, we have
    $$2310(10) + 2730(10) + 4290(6) + 6006(4) + 10010(2) + 15015 = 135199.$$
    Uniqueness follows directly from the CRT.
  \end{proof}
\end{thm}
\end{document}
