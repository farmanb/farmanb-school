\documentclass[10pt]{amsart}
\usepackage{amsmath,amsthm,amssymb,amsfonts,enumerate,mymath,mathtools}
\openup 5pt
\author{Blake Farman\\University of South Carolina}
\title{Math 780:\\Homework 03}
\date{September 25, 2013}
\pdfpagewidth 8.5in
\pdfpageheight 11in
\usepackage[margin=1in]{geometry}

\begin{document}
\maketitle

\providecommand{\p}{\mathfrak{p}}
\providecommand{\m}{\mathfrak{m}}

\newtheorem{thm}{}
\newtheorem{lem}{Lemma}

\begin{thm}\label{ex1}
  Prove that 1105 and 1729 are absolute pseudo-primes.
  
  \begin{proof}
    Observe that $1105 = 5 \cdot 13 \cdot 17$, $1104 = 2^4 \cdot 3 \cdot 23$, $1729 = 7 \cdot 13 \cdot 19$, and $1728 = 3^3 \cdot 2^6$.
    Take $a \in \Z$ such that $(a, 1105) = 1$.
    Then we observe that $4$, $12$, and $16$ divide $1104$ and so by Fermat's Little Theorem
    \begin{eqnarray*}
      a^{1104} &\equiv 1 \pmod{5}\\
      a^{1104} &\equiv 1 \pmod{13}\\
      a^{1104} &\equiv 1 \pmod{17}.
    \end{eqnarray*}
    Therefore by the CRT, $a^{1104} \equiv 1 \pmod{1105}$, and $1105$ is an absolute pseudo-prime.
    
    Take $a \in \Z$ such that $(a, 1729) = 1$.
    Again we observe that $6$, $12$, and $18$ all divide $1728$ and so it follows from Fermat's Little Theorem that
    \begin{eqnarray*}
      a^{1104} &\equiv 1 \pmod{7}\\
      a^{1104} &\equiv 1 \pmod{13}\\
      a^{1104} &\equiv 1 \pmod{19}.
    \end{eqnarray*}
    Therefore by the CRT, $a^{1728} \equiv 1 \pmod{1729}$, and $1729$ is an absolute pseudo-prime.
  \end{proof}
\end{thm}

\begin{thm}\label{ex2}
  Prove that if $n$ is a pseudo-prime, then $2^n - 1$ is a pseudo-prime.

  \begin{proof}
    Let $N = 2^n - 1$ and observe that $N \equiv 1 \pmod{n}$, hence $N = kn + 1$ for some integer $k$.
    Then
    $$2^{N} - 1 = 2^{nk + 1} - 1 = 2(N+1)^k - 1 =\sum_{i=0}^{k-1} {k \choose i}N^{k-i} + 1 \equiv 1 \pmod{N}.$$
    Therefore $2^{N} \equiv 2 \pmod{N}$, as desired.
  \end{proof}
\end{thm}

\begin{thm}\label{ex1}
  Prove the converse of Wilson's Theorem.
  More specifically, prove that $n$ is an integer strictly larger than 1 for which $(n-1)! \equiv -1 \pmod{n}$, then $n$ is prime.
  
  \begin{proof}
    %Suppose to the contrary that $n$ is composite and $(n-1)! \equiv -1 \pmod{n}$.
    %When $n = 4$, we have $3! \equiv 2 \pmod{4}$.
    %Assume that $n > 4$.
    %Observe that for every proper divisor, $d$, of $n$, $a \mid (n-1)!$.
    %Hence 
    Suppose that we have a factorisation $n = kd$.
    Observe that $d \mid (n-1)!$, hence $(n-1)! = md$ for some integer $m$.
    By assumption, there exists an integer $\ell$ such that
    $$(n-1)! = \ell n - 1 = \ell(kd) - 1 = md,$$
    whence $1 = d(\ell k - m)$.
    Therefore $d$ is a unit and $n$ is irreducible, hence prime, as desired.
  \end{proof}
\end{thm}

\begin{thm}
	Find the smallest positive integer $n > 2$ such that $2$ divides $n$, $3$ divides $n + 1$, 4 divides $n + 2$, 5 divides $n + 3$, and $6$ divides $n + 4$.
	Prove your answer is the least such $n$.

	\begin{proof}
		First observe that we have the congruences
		\begin{eqnarray*}
			n & \equiv &  0 \pmod{2}\\
			n & \equiv & 2 \pmod{3}\\
			n & \equiv & 2 \pmod{4}\\
			n & \equiv & 2 \pmod{5}\\
			n & \equiv & 2 \pmod{6}.
		\end{eqnarray*}
		Note that $n \equiv 2 \pmod{4}$ immediately implies $n \equiv 0 \pmod{2}$, and the congruences $n \equiv 0 \pmod{2}$ and $n \equiv 2 \pmod{3}$ immediately implies $n \equiv 2 \pmod{6}$, for if $n = 3k + 2$, then $3k + 2\equiv k \pmod{2}$ implies $k \equiv 0 \pmod{2}$.
		Hence we can reduce to the congruences
		\begin{eqnarray*}
			n &\equiv& 2 \pmod{3}\\
			n &\equiv& 2 \pmod{4}\\
			n &\equiv& 2 \pmod{5}.
		\end{eqnarray*}
		By the CRT, there exists a unique class modulo 60 satisfying these congruences, namely the class of $2$.
		Therefore the smallest $n > 2$ satisfying the congruences above is $62$.
	\end{proof}	 
\end{thm}

\begin{thm}
	Let $k$ be an arbitrary positive integer.
	Prove that there is a positive integer $m$ such that $m + 1$, $m + 2$, \ldots, $m + k$ are each not square-free.
	
	\begin{proof}
		Fix some positive integer $k$.
		Since there are infinitely many prime numbers, we may choose $k$ distinct primes, $p_1, p_2, \ldots, p_k$.
		For $i \neq j$, $(p_i^2, p_j^2) = 1$, hence by the CRT there exists a soltion to the congruences
		\begin{eqnarray*}
			m &\equiv& -1 \pmod{p_1^2}\\
			m &\equiv& -2 \pmod{p_2^2}\\
			& \vdots & \\
			m &\equiv& -k \pmod{p_k^2}.
		\end{eqnarray*}
		Therefore $m$ is as desired.
	\end{proof}
\end{thm}

\begin{thm}
	Calculate the remainder when the number $123456789101112 \ldots 19781979$ is divided by $1980$.
	
	\begin{proof}
		Let $x = 123456789101112 \ldots 19781979$.
		First recognize that there are 980 four digit numbers, 900 three digit numbers, 90 two digit numbers, and 9 single digit numbers.
		Hence
		\begin{eqnarray*}
			x &=& \sum_{k = 0}^{979} (1979 - k) \cdot 10^{4k} + \sum_{k = 0}^{899} (999 - k) \cdot 10^{3k + 3920} + \\
			&+& \sum_{k = 0}^{89} (99 - k) \cdot 10^{2k + 6620} + \sum_{k = 0}^9 (9 - k) \cdot 10^{k + 6800}.
		\end{eqnarray*}
		Then it's obvious that $x \equiv 1979 \equiv 3 \pmod{4}$ and $x \equiv 1979 \equiv 4 \pmod{5}$.
		With a simple loop in, say, Sage, it's easy to compute $ x \equiv 0 \pmod{9}$ and $x \equiv 2 \pmod{11}$.	
		By the CRT we can use the extended Euclidean algorithm we can compute solutions to the equations
		$$\begin{array}{lll}
			4u_1 + 5v_1=1 & 5u_4 + 9v_4 = 1 & 9u_6 + 11v_6 = 1\\
			4u_2 + 9v_2 = 1& 5u_5 + 11v_5 = 1\\
			4u_3 + 11v_1 = 1
		\end{array}$$
		to obtain a solution to the congruences modulo 1980,
		$$n = 3(5 v_1) (9v_2)(11v_3) + 4(4u_1)(9v_4)(11v_5) + 2(4u_3)(5u_5)(9u_6).$$
		One such set of solutions, in the same order as the equations, are the $(u,v)$ pairs
		$$\begin{array}{lll}
			\left(-1, 1\right) & \left(2, -1\right) & \left(5, -4\right)\\
			\left(-2, 1\right) & \left(-2, 1\right)\\
			\left(3, -1\right)
		\end{array}$$
		which give $n = -10701 \equiv 1179 \pmod{1980}.$
		Therefore $x \equiv 1179 \pmod{1980}$.
	\end{proof}
\end{thm}
\end{document}
