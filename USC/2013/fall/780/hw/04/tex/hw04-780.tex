\documentclass[10pt]{amsart}
\usepackage{amsmath,amsthm,amssymb,amsfonts,enumerate,mymath,mathtools}
\openup 5pt
\author{Blake Farman\\University of South Carolina}
\title{Math 780:\\Homework 04}
\date{October 7, 2013}
\pdfpagewidth 8.5in
\pdfpageheight 11in
\usepackage[margin=1in]{geometry}

\begin{document}
\maketitle

\providecommand{\p}{\mathfrak{p}}
\providecommand{\m}{\mathfrak{m}}

\newtheorem{thm}{}
\newtheorem{lem}{Lemma}

\begin{thm}\label{ex1}
  Calculate $\phi(180)$ and $\phi(1323)$.
  
  \begin{proof}
    First factor $180 = 2^2 \cdot 3^2 \cdot 5$ and $1323 = 3^3 \cdot 7^2$.
    Then 
    $$\phi(180) = \phi(2^2)\phi(3^2)\phi(5) = 2 \cdot 6 \cdot 4 = 48$$
    and
    $$\phi(1323) = \phi(3^3)\phi(7^2) = 18 \cdot 42 = 756$$
  \end{proof}
\end{thm}

\begin{thm}\label{ex1}
  Prove that if $n$ is a positive integer such that $\phi(m) \neq \phi(n)$ for every positive integer $m$ different from $n$, then $n > 10^{30}$.
   
  \begin{proof}
  \end{proof}
\end{thm}

\begin{thm}\label{ex1}
  Let $\alpha_1$, $\alpha_2$, and $\alpha_3$ be the roots of $x^3 + x + 1$.
  Calculate 
  $$S_k = \sum_{j=1}^3 \alpha_j^k,\ k = 1, 2, \ldots, 10.$$
  
  \begin{proof}
    First observe that 
    $$x^3 + x + 1 = (x - \alpha_1)(x - \alpha_2)(x - \alpha_3) = x^3 - \sigma_1 x^2 + \sigma_2 x - \sigma_3.$$
    Hence we have $S_1 = \sigma_1 = 0$.
    Then we observe that
    $$S_2 = \sigma_1^2 - 2\sigma_2 = -2.$$
    Since we are given $\alpha_i^3 = -(\alpha_i + 1)$ we have
    $$S_3 = \sum_{i=1}^3 -(\alpha_i + 1) = -(\sigma_1 + 3) = -3.$$
    Using the same equality, we have $\alpha_i^4 = -(\alpha_i^2 + \alpha_i)$ and so
    $$S_4 = -(S_2 + S_1) = 2.$$
    From $\alpha_i^5 = \alpha_i^2\alpha_i^3 = -\alpha_i^2(\alpha_i + 1) = -(\alpha_i^3 + \alpha_i^2)$, it follows that
    $$S_5 = -(S_3 + S_2) = 5$$
    From $\alpha_i^6 = (\alpha_i^3)^2 = \alpha_i^2 + 2\alpha_i + 1$ it follows that
    $$S_6 = S_2 + 2S_1 + 3 = 1.$$
    From $\alpha_i^7 = \alpha_i^3 + 2\alpha_i^2 + \alpha_i = 2\alpha_i^2 - 1$ it follows that
    $$S_7 = 2S_2 - 3 = -7.$$
    From $\alpha_i^8 = 2\alpha_i^3 - \alpha_i$ it follows that
    $$S_8 = 2S_3 - S_1 = -6.$$
    From $\alpha_i^9 = 2\alpha_i^4 - \alpha_i^2$ it follows that
    $$S_9 = 2S_4 - S_2 = 6.$$
    From $\alpha_i^{10} = 2\alpha_i^5 - \alpha_i^3$ it follows that
    $$S_{10} = 2S_5 - S_3 = 13.$$
  \end{proof}
\end{thm}

\begin{thm}\label{ex1}
  Determine whether $x^4 + 1$ is a factor of $x^{25} + 2x^{23} + x^{17} + x^{13} + x^7 + x^3 + 1$ using arithmetic modulo $x^4 + 1$.
  
  \begin{proof}
    The polynomial $x^4 + 1$ is not a factor of $x^{25} + 2x^{23} + x^{17} + x^{13} + x^7 + x^3 + 1$.
    We have the following equalities
    \begin{eqnarray*}
      x^{25} &=& (x^4 + 1)(x^{21} - x^{17} + x^{13} - x^9 + x^5 - x) + x\\
      2x^{23} &=& 2(x^4 + 1)(x^{19} - x^{15} + x^{11} - x^7 + x^3) - 2x^3\\
      x^{17} &=& (x^4 + 1)(x^{13} - x^9 + x^5 - x) + x\\
      x^{13} &=& (x^4 + 1)(x^9 - x^5 + x) - x\\
      x^7 &=& (x^4 + 1)(x^3) - x^3
    \end{eqnarray*}
    Therefore 
    \begin{eqnarray*}
      x^{25} + 2x^{23} + x^17 + x^13 + x^7 + x^3 + 1 &\equiv& x - 2x^3 + x - x - x^3 + x^3 + 1\\
      &\equiv& -2x^3 + x + 1 \pmod{x^4 + 1}.
    \end{eqnarray*}
  \end{proof}
\end{thm}

\begin{thm}\label{ex1}
  Consider all lines which meet the graph on $y = 2x^4 + 7x^3 + 3x - 5$ in four distinct points, say $(x_i, y_i)$, $i = 1, 2, 3, 4$.
  Show that $(x_1 + x_2 + x_3 + x_4)/4$ is independent of the line and find its value.
  
  \begin{proof}
    Suppose $y = mx + b$ is a line satisfying the hypotheses above.
    Then we have
    $$2x^4 + 7x^3 + (3 - m)x - (5 + b) = (x - x_1)(x - x_2)(x - x_3)(x - x_4).$$ 
    Expanding the right-hand side, we obtain
    $$(x - x_1)(x - x_2)(x - x_3)(x - x_4) = x^4 - \sigma_1 x^3 + \sigma_2 x^2 - \sigma_3 x + \sigma_4$$
    Therefore by comparing coefficients we have
    $$\frac{\sigma_1}{4} = \frac{(x_1 + x_2 + x_3 + x_4)}{4} = -\frac{7}{4},$$
    as desired.
  \end{proof}
\end{thm}
\end{document}
