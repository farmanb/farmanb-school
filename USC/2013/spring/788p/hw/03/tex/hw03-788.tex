\documentclass[10pt]{amsart}
\usepackage{amsmath,amsthm,amssymb,amsfonts,enumerate,mymath,tikz-cd}
\openup 5pt
\author{Blake Farman\\University of South Carolina}
\title{Math 788p:\\Homework 03}
\date{February 11, 2013}
\pdfpagewidth 8.5in
\pdfpageheight 11in
\usepackage[margin=1in]{geometry}

\begin{document}
\maketitle

\providecommand{\p}{\mathfrak{p}}
\providecommand{\m}{\mathfrak{m}}

\theoremstyle{plain}
\newtheorem{thm}{}
\newtheorem{lem}{Lemma}
\theoremstyle{definition}
\newtheorem{defn}{Definition}

\setcounter{thm}{2}

\begin{defn}[Limit]\label{def1}
  Let $\mathcal{C}$ and $I$ be categories and let $\mathcal{C}^{I}$ be the category with objects functors from $I$ to $\mathcal{C}$ and arrows natural transformations between functors.
  Define the diagonal functor $\Delta \colon \mathcal{C} \rightarrow \mathcal{C}^{I}$ by the object map that sends $c$ to the
  constant functor\footnote{That is, $\Delta c$ is $c$ at each object $i$ of $I$ and $1_c$ at each arrow of $I$.  
    Here, the functor category amounts to the 'diagrams of shape $I$' in $\mathcal{C}$; a functor $F \colon I \rightarrow \mathcal{C}$ is essentially a choice of $I$-indexed objects and morphisms of $\mathcal{C}$ between these objects.
    $I$ is usually small or finite.} 
  $\Delta c: I \rightarrow \mathcal{C}$ and the arrow map that sends an arrow $f: c \rightarrow c^\prime$ of $\mathcal{C}$ to the natural 
  transformation $\Delta f: \Delta c \dot{\rightarrow} \Delta c^\prime$, which has the value $f$ at each object $i$ of $I$.

  Fix a functor $F \colon I \rightarrow \mathcal{C}$ of $\mathcal{C}^{I}$ and consider a natural transformation $\tau \colon \Delta c \dot{\rightarrow} F$.
  Since $\Delta c$ is the constant functor, the natural transformation $\tau$ is the collection of morphisms $\tau_i \colon c \rightarrow F_i$ of $\mathcal{C}$, for each object $i$ of $I$, such that for every arrow $u \colon i \rightarrow j$ of $I$, the diagram 
  \begin{center}
    \begin{tikzcd}
      & \arrow{ld}[description]{\tau_i}c\arrow{rd}[description]{\tau_i} & & i\arrow{d}[description]{u}\\
      F_i\arrow{rr}[description]{Fu} & & F_j & j
    \end{tikzcd}
  \end{center}
  %$\tau_j = Fu \circ \tau_i$
  commutes.\footnote{Essentially, by identifying the top two vertices of the naturality square, it collapses to a triangle.}
  This is a 'cone to the base $F$ from the vertex $c$'.
  
  A limit for $F$ is the data of an object $\varprojlim_{i \in I}F_i$ of $\mathcal{C}$ and a natural transformation $\nu \colon \Delta \varprojlim_{i \in I}F_i \dot{\rightarrow} F$ that is universal amongst natural transformations $\tau \colon \Delta c \dot{\rightarrow} F$, for objects $c$ of $\mathcal{C}$.
  The natural transformation $\nu$ is a universal cone to the base $F$.
  Pictorially, $\varprojlim_{i \in I}F_i$ is the unique object (up to isomorphism) of $\mathcal{C}$ such that for every $u \colon i \rightarrow j$ of $I$, and every natural transformation $\tau \colon \Delta c \dot{\rightarrow} F$ there exists a unique morphism $f \colon c \rightarrow \varprojlim_{i \in I}F_i$ such that the diagram
  \begin{center}
    \begin{tikzcd}
      & \arrow[bend right]{ldd}{\tau_i}\arrow[dotted]{d}{\exists !h} c\arrow[bend left]{rdd}{\tau_i} &\\
      & \arrow{ld}{\nu_i}\varprojlim_{i \in I}F_i\arrow{rd}{\nu_i} &\\
      F_i\arrow{rr}{Fu} & & F_j
    \end{tikzcd}
  \end{center}
  commutes.
\end{defn}

%As an example, regard $J = \N$ as a discrete category; the objects are natural numbers $n$ and the only morphisms are the identity morphisms, $n \xrightarrow{\operatorname{id}_n} n$.
%If $\mathcal{C} = {\bf Rng}$, then the objects of the functor category ${\bf Rng}^{\N}$ can be safely identified as collections of rings, $\mathcal{R} = \left\{G_n \;\middle\vert\; n \in \N\right\}$.
%For two collections $\mathcal{R}$ and $\mathcal{S}$, a natural transformation $\eta \colon \mathcal{R} \dot{\rightarrow} \mathcal{S}$ is simply a collection of indexed ring homomorphisms $\eta_n \colon R_n \rightarrow S_n$, for each $n \in \N$.
%Since $\N$ is taken to be a discrete category, naturality is trivial (there are no ring homomorphisms $R_n \rightarrow S_m$ within a diagram if $n \neq m$).

%The diagonal functor $\Delta \colon {\bf Rng} \rightarrow {\bf Rng}^{\N}$ sends a ring $R$ to the constant collection $\mathcal{R} = \left\{R_n = R \;\middle\vert\; n \in \N\right\}$.
%A universal arrow from a collection $\mathcal{R}$ to $\Delta$ is an infinite product consisting of the object $\prod_{k \in \N} R_k$ of ${\bf Rng}$ and an indexed collection of ring homomorphism $\pi_n \colon \prod_{k \in \N} R_k \rightarrow R_n$ such that for every other element $\mathcal{S}$ equipped with ring homomorphisms $f_n \colon \mathcal{S} \rightarrow R_n$ for each $n \in \N$, there exists a unique ring homomorphism $f \colon \prod_{k \in \N} R_k \rightarrow \mathcal{S}$ and for each $n, m \in \N$, the diagram 
%\begin{center}
%  \begin{tikzcd}
%    &\mathcal{S}\arrow[bend right]{ldd}[description]{f_n}\arrow[dotted]{d}{\exists ! f}\arrow[bend left]{rdd}[description]{f_m}\\
%    & \arrow{rd}[description]{\pi_n}\prod_{k \in \N}R_k \arrow{ld}[description]{\pi_m} &\\
%    R_n &  & R_m
%  \end{tikzcd}
%\end{center}
%commutes.

\begin{thm}\label{ex3}
  $\Z_p$ is the inverse limit of the rings $\Z/(p^n)$.
  \begin{proof}
    Regard $\N$ as a category with respect to the relation $\leq$ by taking the objects natural numbers and a unique arrow $n \rightarrow m$ if and only if $n \leq m$.
    Take $\mathcal{C} = {\bf Rng}$, the category with objects rings and arrows ring homomorphisms.
    Consider the functor $F \colon \N^\text{op} \rightarrow {\bf Rng}$, where $\N^\text{op}$ is the category $\N$ with the arrows reversed (i.e. the domain and codomain are switched), with object map that sends $n$ to $\Z/p^n\Z$ and
    arrow map that sends $n \rightarrow m$ for every $n \geq m$ to the projection 
    \begin{align*}
      \varphi_{n,m} \colon \Z/(p^n) &\twoheadrightarrow \Z/(p^m)\\
      \alpha &\mapsto \alpha \pmod{p^m}.
    \end{align*}
    This defines the inverse system
    \begin{center}
      \begin{tikzcd}
        \ldots \arrow{r}{\varphi_{5,4}} & \Z/(p^4) \arrow{r}{\varphi_{4,3}} & \Z/(p^3) \arrow{r}{\varphi_{3,2}} & \Z/(p^2)\arrow{r}{\varphi_{2,1}} & \Z/(p).
      \end{tikzcd}
    \end{center}
    Note that by defining the functor with domain the opposite category, we have a covariant, rather than contravariant, functor.
    
    Consider $\Z_p$ as the subring
    $$\left\{(x_1, x_2, \ldots) \in \prod_{n=1}^\infty \Z/(p^n) \;\middle\vert\; \varphi_{n+1,n}(x_{n+1}) = x_n \right\} \subseteq \prod_{n=1}^\infty \Z/(p^n)$$
    equipped with the morphisms 
    \begin{align*}
      \pi_n \colon \Z_p &\rightarrow \Z/(p^n)\\
      (x_1, x_2, \ldots) & \mapsto x_n.
    \end{align*}
    This forms the cone
    \begin{center}
      \begin{tikzcd}
        &&&\Z_p \arrow[dashed]{lld}{\ldots} \arrow{ld}[description]{\pi_4} \arrow{d}[description]{\pi_3} \arrow{rd}[description]{\pi_2} \arrow{rrd}[description]{\pi_1}\\
        &\ldots\arrow{r}[below]{\varphi_{4,5}} & \Z/(p^4) \arrow{r}[below]{\varphi_{3,4}}& \Z/(p^3) \arrow{r}[below]{\varphi_{2,3}}& \Z/(p^2)\arrow{r}[below]{\varphi_{1,2}} & \Z/(p).
      \end{tikzcd}
    \end{center}
    Let $R$ be the vertex of any other cone, equipped with morphisms $f_n \colon R \rightarrow \Z/(p^n)$.
    Then the map 
    \begin{align*}
      f \colon R &\rightarrow \Z_p\\
      r &\mapsto (f_1(r), f_2(r), f_3(r), \ldots).
    \end{align*}
    is forced by the commutativity of the diagram
    \begin{center}
      \begin{tikzcd}
        &R\arrow[bend right]{ldd}[description]{f_n}\arrow[dotted]{d}{\exists ! f}\arrow[bend left]{rdd}[description]{f_m}\\
        & \arrow{ld}[description]{\pi_n}\Z_p \arrow{rd}[description]{\pi_m} &\\
        \Z/(p^n)\arrow{rr}[description]{\varphi_{m,n}} &  & \Z/(p^m).
      \end{tikzcd}
    \end{center}    
    To see that this is actually a morphism, take $r_1, r_2 \in R$.
    Then we have 
    $$f(r_1 + r_2) = (f_1(r_1 + r_2), f_2(r_1 + r_2), \ldots) = (f_1(r_1) + f_1(r_2), f_2(r_1) + f_2(r_2), \ldots),$$
    since each $f_i$ is a morphism.
    Similarly, 
    $$f(r_1 r_2) = (f_1(r_1 r_2), f_2(r_1 r_2), \ldots) = (f_1(r_1) f_1(r_2), f_2(r_1) f_2(r_2), \ldots).$$
  \end{proof}
\end{thm}

\setcounter{thm}{6}
\begin{thm}
  $\Z_p$ is compact.

  \begin{proof}
    Equip $\Z/(p^n)$ with the discrete topology and consider the infinite product $X = \prod_{n=1}^\infty \Z/(p^n)$ equipped with the usual product topology.
    Since each of $\Z/(p^n)$ is finite, each is also compact.
    Hence Tychonoff's Theorem guarantees that $X$ is a compact topological space.
    Note that we have implicity constructed a limit in {\bf Top}.
    In particular, $X = \varprojlim_{i \in I}F_i$ where, 
    regarding $\N$ as a discrete category (the only arrows are the identity maps $n \rightarrow n$ for each $n \in N$), 
    the functor $F \colon \N \rightarrow {\bf Top}$ sends the object $n$ to $\Z/(p^n)$ 
    and the arrow $n \rightarrow n$ to $\Z/(p^n) \rightarrow \Z/(p^n)$ in the obvious way.
    
    Equip $\Z_p$ with the maps
    \begin{align*}
      f_n \colon \Z_p  &\rightarrow \Z/(p^n)\\
      x &\mapsto x \pmod{p^n}.
    \end{align*}
    To see that these are continuous, let $U \subseteq \Z/(p^n)$ be an open set and write $U$ as the union of singletons, $U = \bigcup_{u \in U}\left\{u\right\}$.
    Observe that the basis for the topology on $\Z_p$ consists of the sets
    $$B_k(\alpha) = \left\{x \in \Z_p \;\middle\vert\; x \equiv \alpha \pmod{p^k}\right\} = f_k^{-1}(\alpha).$$
    Continuity of $f_n$ now follows from $$f_n^{-1}(U) = f_n^{-1}\left(\bigcup_{u \in U} u\right) = \bigcup_{u \in U} f_n^{-1}(u).$$
    By the universal property of limits, there exists a unique continuous function $f \colon \Z_p \rightarrow X$ such that the diagram 
    \begin{center}
      \begin{tikzcd}
        &\Z_p\arrow[bend right]{ldd}[description]{f_n}\arrow[dotted]{d}{\exists ! f}\arrow[bend left]{rdd}[description]{f_m}\\
        & \arrow{ld}[description]{\pi_n}X \arrow{rd}[description]{\pi_m} &\\
        \Z/(p^n) &  & \Z/(p^m)
      \end{tikzcd}
    \end{center}  
    commutes.

    Consider the map     
    \begin{align*}
      f \colon \Z_p &\hookrightarrow X\\
      x &\mapsto (f_1(x), f_2(x), f_3(x), \ldots).
    \end{align*}
    The structure on $\Z_p$ dictates that this map satisfies the commutativity conditions.
    It remains to show that $f$ is continuous.
    Towards that end, define the collections $\mathcal{S}_n = \left\{\pi_n^{-1}(U) \;\middle\vert\; U \subseteq \Z/(p^i) \right\}$ for each $n \in \N$.
    Since each $\pi_n$ is continuous, each element $S \in \mathcal{S}_n$ is open in $X$ and we can write $X = \bigcup_{n=1}^\infty \mathcal{S}_n$.
    Hence the collection $\mathcal{S} = \bigcup_{n=1}^\infty \mathcal{S}_n$ is a subbasis for $X$ by definition and so it suffices to show that for any $U \subseteq \Z/(p^n)$, the preimage $\pi_n^{-1}(U)$ is open in $\Z_p$.
    This follows from the commutativity conditions; namely 
    $$f^{-1}(\pi_n^{-1}(U)) = (f \circ \pi_n)^{-1}(U) = f_n^{-1}(U)$$
    is open because $f_n$ is continuous.
    Thus we may identify $\Z_p$ as a subspace of $X$, and it suffices to show that $\Z_p$ is closed in the compact topological space $X$.
    
    Let $x = (x_1, x_2, x_3, \ldots) \in X \setminus \Z_p$ be given.
    There exists some $n \in \N$ such that $x_n \not \equiv x_{n+1} \pmod{p^n}$.
    Let $\alpha = \pi_n(x_n)$ and $\beta = \pi_{n+1}(x_{n+1})$ and consider the open neighbourhood $U = \pi_n^{-1}(\alpha) \cap \pi_{n+1}^{-1}(\beta)$ of $x$.
    For any $y \in U$, we have by construction that $\pi_n^{-1}(y) = \alpha \not \equiv \beta = \pi_{n+1}^{-1}(y) \pmod{p^n}$ and thus $y \not \in \Z_p$.
    Hence $U \subseteq X \setminus \Z_p$ and so $X \setminus \Z_p$ is open.
    Therefore $\Z_p$ is closed, as desired.
    
    %, there exists a unique continuous function, $f$, such that the diagram
    %    \begin{center}
    %      \begin{tikzcd}
    %        &\Z_p\arrow[bend right]{ldd}[description]{f_n}\arrow[dotted]{d}{\exists ! f}\arrow[bend left]{rdd}[description]{f_m}\\
    %        & \arrow{ld}[description]{\pi_n}\prod_{n=1}^\infty \Z/(p^n) \arrow{rd}[description]{\pi_m} &\\
    %        \Z/(p^n) &  & \Z/(p^m)
    %      \end{tikzcd}
    %    \end{center}    
    %    commutes.
  \end{proof}
\end{thm}
\end{document}
