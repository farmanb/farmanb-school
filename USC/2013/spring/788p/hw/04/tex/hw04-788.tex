\documentclass[10pt]{amsart}
\usepackage{amsmath,amsthm,amssymb,amsfonts,enumerate,mymath,tikz-cd,pbox}
\openup 5pt
\author{Blake Farman\\University of South Carolina}
\title{Math 788p:\\Homework 04}
\date{February 27, 2013}
\pdfpagewidth 8.5in
\pdfpageheight 11in
\usepackage[margin=1in]{geometry}

\begin{document}
\maketitle

\providecommand{\p}{\mathfrak{p}}
\providecommand{\m}{\mathfrak{m}}

\theoremstyle{plain}
\newtheorem{thm}{}
\newtheorem{lem}{Lemma}
\theoremstyle{definition}
\newtheorem{defn}{Definition}

\setcounter{thm}{3}

\begin{thm}
  If you would like to learn PARI/GP or SAGE, here are a few easy warmups:
  Get an installation of PARI/GP, SAGE, or other software working, and compute:
  \begin{enumerate}[(a)]
  \item
    The sum of the first hundred primes;
  \item
    The average value of all the quadratic residues mod 97;
  \item
    The number of roots of $x^{101} + x + 1$ mod 103
  \end{enumerate}
  
  \begin{proof}
    \begin{enumerate}[(a)]
    \item
      The commmand $\operatorname{reduce}(\operatorname{lambda} p,q: p + q, \operatorname{primes\_first\_n}(100));$ yields 24133.
    \item
      The commands
      \begin{align*}
        &\operatorname{res} = \operatorname{quadratic\_residues}(97);\\
        &\operatorname{reduce}(\operatorname{lambda}\ p,q: p + q, \operatorname{res})/\operatorname{len}(\operatorname{res});
      \end{align*}
      yield 2328/49.
    \item
      The commands
      \begin{align*}
        &\operatorname{A}.\left<t\right> = \operatorname{PolynomialRing}(\operatorname{GF}(103));\\
        &\operatorname{len}((t\text{\^{ }} 101 + t + 1).\operatorname{roots}());
      \end{align*}
      yield the count of the two roots $\left[\left(56, 1\right), \left(46, 1\right)\right]$.
    \end{enumerate}
  \end{proof}
\end{thm}

\begin{thm}
  \newcommand{\fp}{\mathfrak{p}}
  Suppose $K/\Q$ is a sextic field (i.e., it has degree 6 over $\Q$). Determine all possible splittings of primes $p$ in $\mathcal{O}_K$.
  For example, you might have $(p) = \fp_1^2\fp_2^2\fp_3$, where $f(\fp_1 \mid p) = f(\fp_2 \mid p) = 1$ and $f(\fp_3 \mid p) = 1$. 
  You could abbreviate this to $(1^21^21)$, where each exponent indicates the ramification index, and each base indicates the inertial degree.
  \begin{proof}
    These are listed in the table for exercise 7.
  \end{proof}
\end{thm}

\begin{thm}
  Let $K$ be the cubic field of discriminant -31 given by $x^3 + x - 1 = 0$. 
  Either using a computer or by hand, find primes $p$ which are inert (still prime), totally split (type $(111)$), partially split (type $(21)$), and ramified (type $(1^21)$).
  \begin{proof}
    Let $\theta$ be such that $\theta^3 + \theta - 1 = 0$.
    Then the following primes have the desired properties:\\
    $2 : \left(2\right) \\
    47 : \left(47, \theta + 34\right) \cdot \left(47, \theta - 22\right) \cdot \left(47, \theta - 12\right)\\
    3 : \left(3, \theta^{2} - \theta - 1\right) \cdot \left(3, \theta + 1\right)\\
    31 : \left(3, \theta^{2} - \theta - 1\right) \cdot \left(3, \theta + 1\right)$
  \end{proof}
\end{thm}

\begin{thm}
  In a previous exercise you enumerated all possible splitting types for sextic fields.
  Do they all actually occur? The e-f-g theorem rules everything out, but it doesn’t tell you
  that every possible splitting type occurs.

  Determine, as best as you can, which of the splitting types occur in actual sextic fields. This
  is likely to require extensive trial and error using a computer program.

  In addition, observe which splitting types are the most common.
  
  \begin{proof}
    \begin{center}
      \begin{tabular}{|c|c|c|c|}
        \hline
        Prime & Field Polynomial & Factors & Type\\
        \hline
        2 & $x^6 - x^3 + 1$ & (2) & $6^1$\\
        \hline
        2 & $x^6 + 5x^4 - 2x^3 + 4x^2 + 4x + 1$ & $(2, a^3 + a^2 + 1)^2$ & $3^2$\\
        \hline
        3 & $x^6 - 5$ & $(3, a^2 + 1)^3$ & $2^3$\\
        \hline
        5 & $x^6 - 5$ & $(5, a)^6$ & $1^6$\\
        \hline
        103 & $x^6 - 69x^4 - 206x^3 + 1587x^2 - 14214x - 1558$ &  $(103, a + 34)^3 \cdot (103, a - 34)^3$ & $1^31^3$\\
        \hline
        3 & $x^6 + 23x^5 + 23x^4 + 23x^3 + 23x^2 + 23x + 667$ & \pbox{20cm}{$(3, -2/3a^5 - 15a^4 - 22/3a^3 -$\\$ 2/3a^2 - 16a - 34/3)^2\cdot(3, a + 1)^4$} & $1^21^4$\\
        \hline
        17 & $x^6 + 6x + 3$ & $(17, a + 3) \cdot (17, a^5 - 3a^4 - 8a^3 + 7a^2 - 4a + 1)$ & $1^11^5$\\
        \hline
        2 & $x^6 - 3$ & $(2, a + 1)^2 \cdot (2, a^2 + a + 1)^2$ & $1^22^2$\\
        \hline
        2 & $x^6 + 12317x^4 + 12317x^2 + 1391821$ & \pbox{20cm}{$(2, -1/113a^5 - 5/678a^4 - 109a^3 - 91a^2 -$\\$109a - 4469/6)^4\cdot(2, 1/678a^4 + 18a^2 - 4025/6)$} & $1^42^1$\\
        \hline
        2 &$x^6 + 11x^3 + 143x + 1859$ & $(2, a + 1)^3 \cdot (2, a^3 + a^2 + 1)$ & $1^33^1$\\
        \hline
        7 & $x^6 + 5x + 5$ & $(7, a - 3)^2 \cdot (7, a^4 - a^3 - a^2 + 3a - 1)$ & $1^24^1$\\
        \hline
        3 & $x^6 + 16x^5 + 14x^4 + 12x^3 + 6x^2 + 4x + 2$ & $(3, a + 4) \cdot (3, a^5 - a^3 + a^2 - a - 1)$ & $1^15^1$\\
        \hline
        23 & $x^6 + 91x^4 - 2x^3 + 2884x^2 + 188x + 31745$ & $(23, a^2 + 3a - 7)^2\cdot(23, a^2 - 6a - 6)$ & $2^22^1$\\
        \hline
        13 & $x^6 + 6x + 3$ & $(13, a^2 + 3a - 6) \cdot (13, a^4 - 3a^3 + 2a^2 + 2a + 6)$& $2^14^1$\\
        \hline
        2 & $x^6 + x^5 + x^4 + x^3 + x^2 + x + 1$ & $(2, a^3 + a + 1) \cdot (2, a^3 + a^2 + 1)$ & $3^13^1$\\
        \hline
        19 & $x^6 + 13x^5 + 13x^4 + 221x^3 + 247x^2 + 4199$  & $(19, a + 8) \cdot (19, a - 7)^2 \cdot (19, a)^3$ & $1^11^21^3$\\
        \hline
        23 & $x^6 - 14x^4 + 20x^3 + 49x^2 - 140x + 307$ & $(23, a + 10)^2 \cdot (23, a - 4)^2\cdot (23, a - 6)^2$& $1^21^21^2$\\
        \hline
        3 & $x^6 + 53x^5 + 53x^4 + 53x^3 + 53x^2 + 53x + 3127$ & \pbox{20cm}{$(3, -1/3a^5 - 17a^4 + 52/3a^3 -$\\
          $1/3a^2 - 17a + 4/3) \cdot$\\
          $(3, 1/3a^5 + 17a^4 - 52/3a^3 +$\\
          $1/3a^2 + 18a + 23/3) \cdot$\\
          $(3, a + 1)^4$} & $1^11^11^4$\\
        \hline
        5 & $x^6 + 2x^5 + 2x^4 + 6x^3 + 10x^2 + 30$ & $(5, a + 1) \cdot (5, a)^3 \cdot (5, a^2 + a + 1)$ & $1^11^32^1$\\
        \hline
        83 & $x^6 + 12317x^4 + 12317x^2 + 1391821$ & $(83, a + 38)^2 \cdot (83, a - 38)^2 \cdot (83, a^2 + 16)$ & $1^2,1^2,2^1$\\
        \hline
        23 & $x^6 + 53x^5 + 53x^4 + 3127x^3 + 3233x^2 + 190747$ & $(23, a + 1) \cdot (23, a + 3) \cdot (23, a^2 - 10a - 8)^2$ & $1^11^12^2$\\
        \hline
        5 & $x^6 + 47x^5 + 47x^4 + 47x^3 + 47x^2 + 47x + 2491$ & $(5, a + 1)^2 \cdot (5, a^2 + a + 1) \cdot (5, a^2 - a + 1)$ & $1^2,2^1,2^1$\\
        \hline
        7 & $x^6 + 6x + 3$ & $(7, a + 3) \cdot (7, a^2 - 2a - 2) \cdot (7, a^3 - a^2 + 2a + 3)$ & $1^12^13^1$\\
        \hline
        17 & $x^6 - 40x^3 + 1372$ & \pbox{20cm}{$(17, a^2 + 8a + 6) \cdot (17, a^2 - 2a + 6) \cdot$\\ $(17, a^2 - 6a + 6)$} & $2^12^12^1$\\
        \hline
        5 & $x^6 + 6x + 3$ & $(5, a + 2) \cdot (5, a - 1) \cdot (5, a^4 - a^3 - 2a^2 + 1)$ & $1^1,1^1,4^1$\\
        \hline
        3 & $x^6 + 2x^5 + 2x^4 + 6x^3 + 10x^2 + 30$ & $(3, a - 1) \cdot (3, a)^2 \cdot (3, a^3 - a - 1)$ & $1^1,1^2,3^1$\\
        \hline
        23 & $x^6 + 55x^4 - 2x^3 + 1084x^2 + 116x + 7601$ & $(23, a - 1) \cdot (23, a - 5) \cdot (23, a - 8)^2 \cdot (23, a + 11)^2 $& $1^1 1^1 1^2 1^2$\\
        \hline
        19 & $x^6 + 11x^5 + 11x^4 + 143x^3 + 3553x^2 + 46189$ & $(19, a + 8) \cdot (19, a - 7) \cdot (19, a - 9) \cdot (19, a)^3$ & $1^1 1^1 1^1 1^3$\\
        \hline
        5 & $x^6 - 33x^4 - 22x^3 + 363x^2 - 726x - 1210$ & \pbox{20cm}{$(5, a + 3) \cdot (5, a) \cdot$\\
          $ (5, a^2 - 2a - 2) \cdot (5, a^2 - a + 1)$} & $1^1,1^1,2^1,2^1$\\
        \hline
        59 & $x^6 + 5x + 5$ & \pbox{20cm}{$(59, a + 2) \cdot (59, a + 23) \cdot (59, a - 12) \cdot$\\
          $ (59, a^3 - 13a^2 + 10a + 11)$} & $1^1 1^1 1^1 3^1$\\
        \hline
        7 & $x^6 + 11x^5 + 11x^4 + 11x^3 + 11x^2 + 11x + 143$ & \pbox{20cm}{$(7, a + 1) \cdot (7, a + 2) \cdot$\\
          $(7, a - 2) \cdot (7, a - 3) \cdot (7, a + 3)^2$} & $1^1 1^1 1^1 1^1 1^2$\\
        \hline
        127 & $x^6 + 16x^5 + 14x^4 + 12x^3 + 6x^2 + 4x + 2$ & \pbox{20cm}{$(127, a + 11) \cdot (127, a + 19) \cdot$\\
          $(127, a - 11) \cdot (127, a - 56) \cdot$ \\ 
          $(127, a^2 + 53a - 15)$} & $1^1 1^1 1^1 1^1 2^1$\\
        \hline
        73 & $x^6 + 3$ & \pbox{20cm}{$(73, a + 11) \cdot (73, a + 15) \cdot (73, a + 26)$ \\ 
        $(73, a - 11) \cdot (73, a - 15) \cdot (73, a - 26)$} & $1^1 1^1 1^1 1^1 1^1 1^1$\\
        \hline
      \end{tabular}
    \end{center}
    The types that were most easily produced were the types without ramification.
    Na\"ively factoring the first thousand primes over a handful of sextic fields yielded all the splitting types without ramification.
    The most elusive types were $1^1 1^1 1^4$ and $1^4 2^1$.
  \end{proof}
\end{thm}

\begin{thm}
  Take a ton of cubic fields, sorted by discriminant.
  \begin{enumerate}[(a)]
  \item
    Pick your favorite prime, not 2 or 3.
    Then, find cubic fields in which this prime is totally split, inert, partially split, and ramified.
  \item
    Pick any cubic field which does not have a square discriminant.
    Count the number of primes $p \leq X$, where $X$ is some reasonably big number, for which $p$ is totally split, inert, partially split, and ramified.
    What do the proportions look like?
  \item
    Now pick a cubic field which {\it does} have a square discriminant.
    How does your calculation change?
    Can you explain why?
  \item
    One simple thing you can do with these tables is to count all the cubic fields $K$ with $\abs{\operatorname{Disc}(K)} < X$.
    Davenport and Heilbronn (1971) proved that the number of such fields is asymptotic to $\frac{1}{3\zeta(3)}X$, where $\zeta(3) = 1.202\ldots$.
    Does this match your data well?
  \item
    Once again, picking some prime larger than 3, compute counts of cubic fields with $\abs{\operatorname{Disc}(K)} < X$ where this prime is totally split, inert, partially split, and ramified.
    Describe your data.
    Can you guess the relative proportions?
  \end{enumerate}

  \begin{proof}
    For my ton of cubic fields, I took all cubics with discriminant bounded in absolute value by $10^6$, of which there are 237017.
    \begin{enumerate}[(a)]
    \item
      Using the prime $p = 73$, the following table gives field polynomials and splitting types.
      \begin{center}
        \begin{tabular}{|c|c|}
          \hline
          Splitting Type & Field Polynomial\\
          \hline
          totally split & $x^3 - 3x - 1$\\ 
          \hline
          inert & $x^3 - x^2 + 1$\\ 
          \hline
          partially split & $x^3 + x - 1$\\
          \hline
          ramified & $x^3 - x^2 - 24x + 27$\\
          \hline
        \end{tabular}
      \end{center}
    \item
      Using the field with polynomial $x^3 - x^2 + 1$ and discriminant $-23$, the counts and proportions for each of the splitting types of the first 1000 primes are as follows:
      \begin{center}
        \begin{tabular}{|c|c|c|}
          \hline
          Splitting Type & Count &Proportion\\
          \hline
          totally split & 157 & $157/1000$ (nearly $1/6$)\\ 
          \hline
          inert & 334 &  $167/500$ (nearly $1/3$)\\ 
          \hline
          partially split & 508 & $127/250$ (nearly $1/2)$\\
          \hline
          ramified & 1 & $1/1000$\\
          \hline
        \end{tabular}
      \end{center}
      \item
        Using the field with polynomial $x^3 - x^2 - 2x + 1$ and discriminant $49$, the counts and proportions for each of the splitting types of the first 1000 primes are as follows:
        \begin{center}
        \begin{tabular}{|c|c|c|}
          \hline
          Splitting Type & Count & Proportion\\
          \hline
          totally split & 331 & $331/1000$ (nearly $1/3$)\\
          \hline
          inert & 668 & $167/250$ (nearly $2/3$)\\ 
          \hline
          partially split & 0 & $0/1000$\\
          \hline
          ramified & 1 & $1/1000$\\
          \hline
        \end{tabular}
      \end{center}
        The major difference here is that there are no partially split primes.
        Of the 237017 fields selected, the 159 with square discriminants are all Galois.
        As was mentioned in class, when the field is Galois, the e-f-g Theorem gives $[K : Q] = efg$, i.e. $e_1 = e_2 = \ldots = e_g$ and $f_1 = f_2 = \ldots = f_g$, which precludes partial splitting.
      \item
        Counting the number of fields with discriminant less than 1000, the result of Davenport and Heilbronn suggests roughly 277 fields, while there are 154 such fields in my data set.
        Checking against the 237017 fields with discriminant less than $10^6$, the table below gives the ratios 
        $$r_n = \frac{\abs{\left\{K \;\middle\vert\; \abs{\operatorname{disc}(K)} < 10^n \right\}}}{\frac{10^n}{3\zeta(3)}},\, 2 \leq n \leq 6.$$ 
        \begin{center}
          \begin{tabular}{|c|c|}
            \hline
            $n$ & $r_n$\\
            \hline
            2 & 0.324555363853090\\
            \hline
            3 & 0.555350289259733\\
            \hline
            4 & 0.685893668942865\\
            \hline
            5 & 0.787767991485640\\
            \hline
            6 & 0.854723763048533\\
            \hline
          \end{tabular}
        \end{center}
        which appears to be tending towards 1, as expected. 
      \item
        Using the prime $p = 179$, the counts of the splitting types in number of fields with discriminant less than 1000 are given by the table below.
        \begin{center}
          \begin{tabular}{|c|c|}
            \hline
            Splitting Type & Count\\
            \hline
            totally split & 26\\ 
            \hline
            inert & 45\\ 
            \hline
            partially split & 82 \\
            \hline
            ramified & 1\\
            \hline
          \end{tabular}
        \end{center}
    \end{enumerate}
  \end{proof}
\end{thm}
\end{document}
