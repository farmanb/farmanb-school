\documentclass[10pt]{amsart}
\usepackage{amsmath,amsthm,amssymb,amsfonts,enumerate,mymath,tikz-cd,pbox,mathtools, cite}
\openup 5pt
\author{Blake Farman\\University of South Carolina}
\title{Math 788p:\\Homework 07}
\date{May 8, 2013}
\pdfpagewidth 8.5in
\pdfpageheight 11in
\usepackage[margin=1in]{geometry}

\begin{document}
\maketitle

\providecommand{\p}{\mathfrak{p}}
\providecommand{\m}{\mathfrak{m}}
\newcommand{\legendre}[2]{\left(\frac{#1}{#2}\right)}
\theoremstyle{plain}
\newtheorem{thm}{}
\newtheorem{lem}{Lemma}
\theoremstyle{definition}
\newtheorem{defn}{Definition}
\newtheorem{prop}{Proposition}
\newtheorem{cor}{Corollary}

\setcounter{thm}{6}

\begin{thm}
  Describe, in around two to three pages, something interesting you have learned this semester, about number theory or a related topic, outside of this course and your other coursework.
\end{thm}

The bulk of my extra-curricular learning this semester has been devoted in large part to developing the basic machinery for algebraic geometry, particularly the language of category theory.  
One result I found particularly interesting is Yoneda's Lemma, which is a somewhat similar result to the embedding of an algebra in its ring of endomorphisms.
Namely, a category, $\mathcal{C}$, can be identified as a subcategory of the category of presheaves (i.e. the category of covariant functors $\mathcal{C}^\text{op} \rightarrow {\bf Set}$), $[\mathcal{C}^\text{op}, {\bf Set}]$.
Morally speaking, this means that you can recover information about an object in a category simply by studying the morphisms into it.
The development here mostly follows that found in \cite{lane1998categories}\footnote{The development here is (not unexpectedly) rather terse.  Ravi Vakil's notes provide a better 'feel' for Yoneda, but, at least in my opinion, are a bit looser when it comes down to doing the work.} with (most) of the omitted details filled in.

The proof of Yoneda's Lemma is not particularly involved once the language is unwound.  Most of the unwinding is done in the form of a proposition, from which Yoneda follows quickly.  This requires the following
\begin{defn}{Univseral Arrow}
  If $S: \mathcal{D} \rightarrow \mathcal{C}$ is a functor and $c$ an object of $\mathcal{C}$, a universal arrow from $c$ to $S$ is a pair $\left<r, u\right>$ consisting of an object $r$ of $\mathcal{D}$ and an arrow $u \colon c \rightarrow Sr$ of $C$, such that every pair $\left<d,f\right>$ with $d$ an object of $\mathcal{D}$ and $f \colon c \rightarrow Sd$, there is a unique arrow $f^\prime \colon r \rightarrow d$ of $\mathcal{D}$ such that the diagram
  \begin{center}
    \begin{tikzcd}
      c \arrow[equals]{d}\arrow{r}{u} & Sr \arrow[dashed]{d}{Sf^\prime} & r \arrow[dashed]{d}{\exists! f^\prime}\\
      c \arrow{r} & Sd & d
    \end{tikzcd}
  \end{center}
  commutes.
\end{defn}
This definition makes nearly trivial the following
\begin{prop}
  For a functor $S \colon \mathcal{D} \rightarrow \mathcal{C}$ a pair $\left<r, u \colon c \rightarrow Sr\right>$ is a universal from $c$ to $S$ if and only if the functor sending each $f^\prime \colon r \rightarrow d$ into $Sf^\prime \circ u \colon c \rightarrow Sd$ is a bijection of hom-sets
  \begin{equation}\label{eq1}
    \mathcal{D}(r,d) \cong \mathcal{C}(S,d),
  \end{equation}
  where $D$ and $C$ are the usual covariant hom-functors.
  This bijection is natural in $d$.
  Conversely, any natural isomorphism \eqref{eq1} is determined in this way by a unique arrow $u \colon c \rightarrow Sr$ such that $\left<r,u\right>$ is universal from $c$ to $S$.
  
  \begin{proof}
    The bijection follows directly from the definition of the universal.
    Namely, for every arrow $f \colon c \rightarrow Sd$, there is a unique arrow $f^\prime \colon r \rightarrow d$ with $f = Sf^\prime \circ u$.
    Naturality is just the statement $S(g^\prime \circ f^\prime) \circ u = Sg^\prime \circ Sf^\prime \circ u$, which follows directly from functoriality of $S$ and associativity of composition.

    For the converse, let $\varphi \colon \mathcal{D}(r,\_) \rightarrow \mathcal{C}(c, {S\_})$ be a natural transformation and let $u \colon c \rightarrow Sr$ to be the image of the identity on $r$ under the component $\varphi_r$.
    For every morphism $f^\prime \colon r \rightarrow d$, we have a morphism
    \begin{align*}
      f^\prime \circ \_ \colon \mathcal{D}(r,r) &\rightarrow \mathcal{D}(r,d)\\
      g^\prime &\mapsto f^\prime \circ g^\prime.
    \end{align*}
    In particular, when $g^\prime = 1_r$, from the bijection and the naturality square we obtain
    %we have by naturality $Sf^\prime \circ \varphi_r(1_r) = Sf^\prime \circ u = \varphi_d(f^\prime \circ 1_r) = \varphi_d(f^\prime)$, which is precisely the statement that $\left<r,u\right>$ is universal.
    \begin{center}
      \begin{tikzcd}
        r\arrow{ddd}[left]{f^\prime} & \mathcal{D}(r,r)\arrow{ddd} \arrow{rrr}{\varphi_r}\arrow{ddd}[left]{f^\prime \circ \_}  & & & \mathcal{C}(c, Sr)\arrow{ddd}{Sf^\prime \circ \_}\\
        & & 1_r \arrow[mapsto]{d}\arrow[mapsto]{r} & u := \varphi_r(1_r) \arrow[mapsto]{d}\\
        & & f^\prime \circ 1_r = f^\prime \arrow[mapsto]{r} & f := \varphi_d(f^\prime) = Sf^\prime \circ u\\
        d & \mathcal{D}(r,d) \arrow{rrr}[below]{\varphi_d} & & & \mathcal{C}(c, Sd)
      \end{tikzcd}
    \end{center}
    which is precisely the statement that $\left<r, u\right>$ is universal.
    This idea will be the crux of the proof of Yoneda.
  \end{proof}
\end{prop}
From this we see that whenever $\mathcal{C}$ and $\mathcal{D}$ have small hom-sets\footnote{That is, for all $c, c^\prime$ objects of $\mathcal{C}$ and $d, d^\prime$ objects of $\mathcal{D}$, 
  $\operatorname{Hom}_\mathcal{C}(c, c^\prime)$ and $\operatorname{Hom}_\mathcal{D}(d, d^\prime)$ are actually {\it sets}, not proper classes.},
and $S \colon \mathcal{D} \rightarrow \mathcal{C}$ is a functor, we obtain a natural isomorphism between the functors $\mathcal{D}(r,\_) \cong \mathcal{C}(c, S\_)$.
Note that when we write $\mathcal{C}(c, S\_)$ we mean the functor $\mathcal{C}(c, S\_) \colon \mathcal{D} \rightarrow {\bf Set}$ with object function $d \mapsto \operatorname{Hom}_\mathcal{C}(c, Sd)$ and arrow function that sends $f \colon d \rightarrow d^\prime$ to the set function 
\begin{align*}
  Sf \circ \_ \colon \operatorname{Hom}_\mathcal{C}(c, Sd) &\rightarrow \operatorname{Hom}_\mathcal{C}(c, Sd^\prime)\\
  g &\mapsto Sf \circ g.
\end{align*}
In fact, this is true in general of {\it any} set valued functor, such as a pre-sheaf (or a sheaf) on a topological space $X$ viewed as a preorder under inclusion maps.
In this case we say
  \begin{defn}
    Let $\mathcal{D}$ have small hom-sets.
    A representation of a functor $K \colon \mathcal{D} \rightarrow {\bf Set}$ is a pair $\left<r, \psi\right>$, $r$ an object of $\mathcal{D}$, and
    $$\psi \colon \mathcal{D}(r, \_) \cong K$$
    a natural isomorphism.
    The object $r$ is called the representing object.
    The functor $K$ is said to be representable when such a representation exists.
  \end{defn}

  That this holds in general (not just for the case in Proposition 1) is the
  \begin{lem}[Yoneda]
    If $K \colon \mathcal{D} \rightarrow {\bf Set}$ and $r$ is an object of $\mathcal{D}$ having small hom-sets, there is a bijection
    $$y \colon \operatorname{Nat}(D(r, \_), K) \cong Kr,$$
    which sends each natural transformation $\alpha \colon D(r, \_) \rightarrow K$ to $\alpha_r(1_r)$, the image of the identity $1_r$ on $r$, to an element of $Kr$.
    \begin{proof}
      The proof follows from the same techniques in Proposition 1 and the naturality square
      \begin{center}
        \begin{tikzcd}
          \mathcal{D}(r, r) \arrow{d}[left]{f \circ \_}\arrow{r}{\alpha_r} & Kr \arrow{d}{Kf} & r \arrow{d}{f}\\
          \mathcal{D}(r, d) \arrow{r}{\alpha_d} & Kd & d.
        \end{tikzcd}
      \end{center}
    \end{proof}
  \end{lem}
  This immediately yields a useful
  \begin{cor}
    For objects $r, s \in \mathcal{D}$, each natural transformation $\mathcal{D}(r, \_) \rightarrow \mathcal{D}(s, \_)$ has the form
    \begin{align*}
      \_ \circ h \colon \mathcal{D}(r, \_) &\rightarrow \mathcal{D}(s, \_)\\
      f & \mapsto f \circ h 
    \end{align*}
    for a unique arrow $h \colon s \rightarrow r$
  \end{cor}
  For with this Corollary, it is not difficult to show that the map $y$ above is natural in $K$ and $r$.
  This requires viewing $y$ as a natural isomorphism between the bifunctors $N, E \colon [\mathcal{D}, {\bf Set}] \times \mathcal{D} \rightarrow {\bf Set}$.
  Here $E$ is the evaluation functor that sends a pair $\left<K, r\right>$ to the set $Kr$ and $N$ is the functor that takes a pair $\left<K, r\right>$ to the set of natural transformations $\operatorname{Nat}(\mathcal{D}(r, \_), K)$.
  For a natural transformation $F \colon K \rightarrow K^\prime$ and an arrow $f \colon r \rightarrow r^\prime$ of $\mathcal{D}$, $F \times f \colon \left<K, r\right> \rightarrow \left<K^\prime, r^\prime\right>$ is a morphism of $[\mathcal{D}, {\bf Set}] \times \mathcal{D}$ in the evident way and the naturality square is
  \begin{center}
    \begin{tikzcd}
      \left<K, r\right>\arrow{d}[left]{F \times f} & N(K,r) \arrow{r}{y_{\left<K,r\right>}}\arrow{d}[left]{N(F \times f)} & E(K,r)\arrow{d}{E(F \times f)}\\
      \left<K^\prime, r^\prime\right> & N(K,r^\prime) \arrow{r} & E(k, r^\prime)
  \end{tikzcd}
  \end{center}
  
  We may now define the Yoneda functor $Y \colon D^\text{op} \rightarrow [\mathcal{D}, {\bf Set}]$ with object function $r \mapsto \mathcal{D}(r, \_)$ and arrow function that sends $f \colon s \rightarrow r$ to the arrow
  \begin{align*}
    \_ \circ f \colon \mathcal{D}(r, \_) &\rightarrow \mathcal{D}(s, \_)\\
   g & \mapsto g \circ f.
  \end{align*}
  Similarly (i.e. reverse all the arrows), there is the dual functor $Y^\prime \colon \mathcal{D} \rightarrow [\mathcal{D}^\text{op}, {\bf Set}]$ with object function $r \mapsto \mathcal{D}(\_, r)$ and arrow function that sends $f \colon s \rightarrow r$ to the arrow
  \begin{align*}
    f \circ \_  \colon \mathcal{D}(\_, s)  &\rightarrow \mathcal{D}(\_, r)\\
   g & \mapsto f \circ g.
  \end{align*}
  The latter is of particular interest in algebraic geometry.
  Indeed, both of these functors are full and faithful, which says that the embedding given by the dual Yoneda functor allows us to view, in a natural way, the original category as a subcategory of its category of presheaves.
  
  \bibliographystyle{plain}
  \bibliography{hw07-788.bib}{}

\end{document}



