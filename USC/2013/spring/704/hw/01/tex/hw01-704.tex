\documentclass[10pt]{amsart}
\usepackage{amsmath,amsthm,amssymb,amsfonts,enumerate,mymath}
\openup 5pt
\author{Blake Farman\\University of South Carolina}
\title{Math 704:\\Homework 01}
\date{February 28, 2013}
\pdfpagewidth 8.5in
\pdfpageheight 11in
\usepackage[margin=1in]{geometry}

\begin{document}
\maketitle

\providecommand{\p}{\mathfrak{p}}
\providecommand{\m}{\mathfrak{m}}

\newtheorem{thm}{}
\newtheorem{lem}{Lemma}

\setcounter{thm}{1}

\begin{thm}\label{ex1}
  Let $(X, \mathcal{A}, \mu)$ be a measure space.
  Let $f, f_n \colon X \rightarrow \R^*$ be integrable for all $n \in \N$.
  Let 
  $$\int_{X} \abs{f_n - f} d\mu \rightarrow 0\ \text{as}\ n \rightarrow \infty.$$
  Show that $K = \left\{f_n\right\}_{n \in \N}$ is uniformly integrable.
  That is, for every $\varepsilon > 0$ there exists a $\delta > 0$ such that $s = \sup \left\{ \int_E \abs{f_n} d\mu \;\middle\vert\; n \in N\right\} < \varepsilon$ for every $E \in \mathcal{A}$ with $\mu(E) < \delta$.
  \begin{proof}
    Let $\varepsilon > 0$ be given.
    Choose $N \in \N$ such that 
    $$\int_X \abs{f_n - f} d\mu < \frac{\varepsilon}{2}$$
    whenever $n \geq N$.
    Observe that we have $\abs{f_n} - \abs{f} \leq \abs{f_n  - f}$ by the triangle inequality and so 
    \begin{equation}\label{1.1}
      \int_X \abs{f_n} - \abs{f} d\mu = \int_X \abs{f_n} d\mu - \int_X \abs{f} d\mu \leq \int_X \abs{f_n - f} d\mu < \frac{\varepsilon}{2}
    \end{equation}
    follows from Propositions 5.16 and 5.26.
    Since the $\abs{f_n}$ are integrable, it follows from Corollary 5.24 that there exist positive $\delta_1, \delta_2, \ldots, \delta_{N-1}$ such that if $k < N$, then
    \begin{equation}\label{1.2}
      \int_E \abs{f_k} d\mu < \varepsilon
    \end{equation}
    holds for every measurable $E$ with $\mu(E) < \delta_k$.
    Similarly, there exists $\delta_f > 0$ such that 
    \begin{equation}\label{1.3}
      \int_E \abs{f} d\mu < \frac{\varepsilon}{2}
    \end{equation}
    holds for every measurable $E \subseteq X$ with $\mu(E) < \delta_f$.
    Let $\delta = \min\left\{\delta_f, \delta_1, \ldots, \delta_{N-1}\right\}$ and let $E \subseteq X$ be a measurable set with $\mu(E) < \delta$.
    By \eqref{1.1}, \eqref{1.3}, and Proposition 5.15 (b) we have that 
    $$\int_E \abs{f_n} d\mu = \int_E \abs{f_n - f} d\mu + \int_E \abs{f} d\mu < \frac{\varepsilon}{2} + \frac{\varepsilon}{2} = \varepsilon$$
    whenever $n \geq N$.
    Then by \eqref{1.2} we have that
    $$\int_E \abs{f_k} d\mu < \varepsilon$$
    holds whenever $k \leq N$.
    Since $s$ is the least upper bound on the set $\left\{\int_E \abs{f_n} d\mu \;\middle\vert\; n \in N\right\}$, it follows that $s < \varepsilon$, as desired.
  \end{proof}
\end{thm}
  
\end{document}
