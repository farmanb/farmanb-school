\documentclass[10pt]{amsart}
\usepackage{amsmath,amsthm,amssymb,amsfonts,enumerate}
\openup 5pt
\author{Blake Farman\\University of South Carolina}
\title{Math 702:\\Homework 09}
\date{April 17, 2013}
\pdfpagewidth 8.5in
\pdfpageheight 11in
\usepackage[margin=1in]{geometry}

%Field names
\newcommand{\Z}{\mathbb{Z}}
\newcommand{\R}{\mathbb{R}}
\newcommand{\Q}{\mathbb{Q}}
\newcommand{\C}{\mathbb{C}}
\newcommand{\F}{\mathbb{F}}
\newcommand{\N}{\mathbb{N}}
\newcommand{\uhp}{\mathfrak{h}}

%Operator names
\newcommand{\ord}{\operatorname{ord}}
\newcommand{\Det}{\operatorname{Det}}
\newcommand{\Gal}{\operatorname{Gal}}
\newcommand{\Inn}[1]{\operatorname{Inn}\left(#1\right)}
\newcommand{\Aut}[1]{\operatorname{Aut}\left(#1\right)}
\newcommand{\real}[1]{\operatorname{\mathfrak{Re}}\left(#1\right)}
\newcommand{\imag}[1]{\operatorname{\mathfrak{Im}}\left(#1\right)}
\newcommand{\Syl}[2]{\operatorname{Syl}_{#1}\left(#2\right)}
\newcommand{\SL}[2]{\operatorname{SL}_#1\left(#2\right)}
\newcommand{\GL}[2]{\operatorname{GL}_#1\left(#2\right)}
\newcommand{\M}[2]{\operatorname{M}_#1\left(#2\right)}
\newcommand{\PSL}[2]{\operatorname{PSL}_#1\left(#2\right)}
\newcommand{\Mat}[2]{\operatorname{Mat}_{#1}\left(#2\right)}
\providecommand{\norm}[1]{\lVert#1\rVert}
\newcommand{\dist}[2]{\operatorname{dist}\left(#1,#2\right)}
\newcommand{\cntr}[1]{\mathbf{Z}\left(#1\right)}
\newcommand{\abs}[1]{\left| #1 \right|}
\newcommand{\orbit}[1]{\mathcal{O}_{#1}}
\newcommand{\card}[1]{\operatorname{card}#1}
\newcommand{\Log}[1]{\operatorname{Log}\left(#1\right)}
\newcommand{\Arg}[1]{\operatorname{Arg}\left(#1\right)}
\newcommand{\Tor}[2]{\operatorname{Tor}_{#1}\left(#2\right)}
\newcommand{\Ann}[2]{\operatorname{Ann}_{#1}\left(#2\right)}
\newcommand{\Hom}[2]{\operatorname{Hom}_{#1}\left(#2\right)}
\newcommand{\End}[2]{\operatorname{End}_{#1}\left(#2\right)}

\renewcommand{\qedsymbol}{\(\blacksquare\)}
\renewcommand{\epsilon}{\varepsilon}

\begin{document}
\maketitle

\providecommand{\p}{\mathfrak{p}}
\providecommand{\m}{\mathfrak{m}}

\newtheorem{thm}{}
\newtheorem{lem}{Lemma}

\begin{thm}
	Let $R$ be a ring, let $M$ be a (left) $R$-module, and let $\Tor{R}{M}$ denote the set of torsion elements of $M$.
	\begin{enumerate}[(a)]
		\item
			Suppose that $R$ is an integral domain.
			Prove that $\Tor{R}{M}$ is a submodule of $M$, the torsion submodule of $M$.
		\item
			Suppose that $R$ is commutative.
			Give an example of an $R$-module $M$ in which $\Tor{R}{M}$ is not a submodule of $M$.
	\end{enumerate}
	
	\begin{proof}
		\begin{enumerate}[(a)]
			\item
				Since $0 \in \Tor{R}{M}$, it is non-empty.
				Let $x,y \in \Tor{R}{M}$ and $0 \neq r \in R$ be given.
				There exist elements $r_x, r_y$ of $R$ such that $r_x x = r_y y = 0$.
				Hence, since $R$ is an integral domain,
				$$r_x r_y(x + ry) = r_y(r_x x) + r_x r(r_y y) = 0$$
				with $r_x r_y \neq 0$.
				Therefore $x + ry \in \Tor{R}{M}$ and $\Tor{R}{M}$ is a submodule, as desired.
			\item
				Take the $\Z$-module $\Z/6\Z$.
				The elements 2 and 3 are both torsion elements, $3 \cdot (2) = 0$ and $2 \cdot (3) = 0$.
				However, $5 = 2 + 3 \not \in \Tor{R}{M}$.
		\end{enumerate}
	\end{proof}
\end{thm}

\begin{thm}
	Let $R$ be a ring, let $M$ be a (left) $R$-module, and let $\Ann{R}{N}$ denote the annihilator of $N$ in $R$.
	Prove that $\Ann{R}{N}$ is an ideal (two-sided).
	
	\begin{proof}
		First observe that $0 \in \Ann{R}{N}$, and so it is not empty.
		Let $a_1, a_2 \in \Ann{R}{N}$ be given.
		For any $n \in N$, it follows from
			$$(a_1 + a_2)n = a_1 n + a_2 n = 0$$
		that $a_1 + a_2 \in \Ann{R}{N}$.
		Then for any $r \in R$, by the definition of the $R$-action on $N$, 
		$$(a_1r)n = a_1(rn) = 0\ \text{and}\ (ra_1)n = r(a_1n) = 0.$$
		Therefore $\Ann{R}{N}$ is a two-sided ideal.
	\end{proof}
\end{thm}

\begin{thm}
Let $R$ be a commutative ring, let $M$, $A$, $B$ be $R$-modules, and let $F$ be a free $R$-module.
\begin{enumerate}[(a)]
	\item
		Prove the following.
		\begin{enumerate}
			\item
				$\Hom{R}{A,B}$ is an $R$-module.
			\item
				$\Hom{R}{R,M} \cong M$ as $R$-modules.
			\item
				Prove that $\End{R}{M} = \Hom{R}{M,M}$ is a ring, the endomorphism ring of $M$.
		\end{enumerate}
	\item
		Prove one of the following $R$-module isomorphisms.
		\begin{enumerate}
			\item
				$\Hom{R}{A \times B, M} \cong \Hom{R}{A,M} \times \Hom{R}{B,M}$.
			\item
				$\Hom{R}{M, A \times B} \cong \Hom{R}{M,A} \times \Hom{R}{M,B}$.
		\end{enumerate}
	\item
		Suppose that $\operatorname{rank}_R(F) = n$ is finite.
		\begin{enumerate}
			\item
				Prove that $\Hom{R}{F,R} \cong F$ as $R$-modules.
			\item
				Prove that $\Hom{R}{F,M} \cong M \times \ldots M$ ($n$ times).
		\end{enumerate}
\end{enumerate}

\begin{proof}
	\begin{enumerate}[(a)]
		\item
		\begin{enumerate}[(i)]
			\item
				Let $\varphi, \psi \in \Hom{R}{A,B}$ be given.
				Under pointwise addition, $\Hom{R}{A,B}$ is an abelian group:
				\begin{enumerate}
					\item
						The unique map $0 \colon A \rightarrow 0 \hookrightarrow B$ is the additive identity,
					\item
						$(\varphi + \psi)(a) = \varphi(a) + \psi(a) = \psi(a) + \varphi(a) = (\psi + \varphi)(a)$, for each $a \in A$
					\item
						$(\varphi - \varphi)(a) = \varphi(a) - \varphi(a) = 0$, for each $a \in A$.
				\end{enumerate}
				Define an $R$-action
				\begin{align*}
					R \times \Hom{R}{A,B} &\rightarrow \Hom{R}{A,B}\\
					(r, \varphi) &\mapsto r\varphi.
				\end{align*}
				Since $R$ is a commutative ring, this action is well defined.  Namely $r\varphi$ is a homomorphism,
				$$r\varphi(a + sb) = r(\varphi(a) + s\varphi(b)) = r\varphi(a) + rs\varphi(b) = (r\varphi)(a) + s(r\varphi)(b).$$
				For any $a \in A$ and $r, s \in R$, using the $R$-action on $B$, this action satisfies the module axioms
				\begin{enumerate}
					\item
						$((r + s)\varphi)(a) = (r + s)\varphi(a) = r\varphi(a) + s\varphi(a) = (r\varphi)(a) + (s\varphi)(a)$,
					\item
						$((rs)\varphi)(a) = (rs)\varphi(a) = r(s\varphi(a)) = r(s\varphi)(a)$,
					\item
						$r(\varphi + \psi)(a) = r(\varphi(a) + \psi(a)) = r\varphi(a) + r\psi(a) = (r\varphi)(a) + (r\psi)(a)$, and
					\item
						$(1\varphi)(a) = 1 \varphi(a) = \varphi(a)$.
				\end{enumerate}
				Therefore $\Hom{R}{A,B}$ is an $R$-module.
			\item
				Let $\varphi \in \Hom{R}{M}$ be given.
				Since $R$ is cyclically generated by $1$ as an $R$-module, for $r \in R$, $\varphi(r) = r\varphi(1)$.
				Observe that the choice of target for $1$ is equivalent to defining the morphism $\varphi \colon r \mapsto rm$.
				Namely, $\varphi(r) = r\varphi(1) = rm$ implies by 
		\end{enumerate}
	\end{enumerate}
\end{proof}

\end{thm}

\begin{thm}[Schur's Lemma]
	let $R$ be a ring.
	An $R$-module $M$ is {\it simple} if and only if it is nonzero and its only submodules are $\left\{0\right\}$ and $M$.
	Suppose that $M$ is a simple $R$-module.
	Prove that $\End{R}{M}$ is a division ring.
	
	\begin{proof}
		Let $\varphi \in \End{R}{M}$ be given.
		Observe that $\varphi(0) = \varphi(0 + 0) = \varphi(0) + \varphi(0)$ implies by cancellation that $\varphi(0) = 0 \in \varphi(M)$.
		For any two elements $a, b \in M$ and any ring element $r$,
		$$\varphi(a) + r\varphi(b) = \varphi(a) + \varphi(rb) = \varphi(a + rb) \in \varphi(M),$$
		and so $\varphi(M)$ is a submodule of $M$.
		Hence $\varphi$ is either the zero morphism, or an automorphism.
		Since by part iii of 3 (a) $\End{R}{M}$ is a ring, it follows that $\End{R}{M}^\times = \operatorname{Aut}_R(M)$.
		Therefore $\End{R}{M}$ is a (not necessarily commutative) division ring.
	\end{proof}
\end{thm}  
\end{document}
