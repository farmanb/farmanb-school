\documentclass[10pt]{amsart}
\usepackage{amsmath,amsthm,amssymb,amsfonts,enumerate,mymath,mathtools,tikz}
\usetikzlibrary{shapes}

\openup 5pt
\author{Blake Farman\\University of South Carolina}
\title{Math 702:\\Homework 05}
\date{March 1, 2013}
\pdfpagewidth 8.5in
\pdfpageheight 11in
\usepackage[margin=1in]{geometry}

\begin{document}
\maketitle

\providecommand{\p}{\mathfrak{p}}
\providecommand{\m}{\mathfrak{m}}

\newtheorem{thm}{}
\newtheorem{lem}{Lemma}

\begin{thm}
  Let $F$ be a field.
  Suppose that $x$ is transcendental over $F$ and suppose that $K$ is a field with $F \subset K \subseteq F(x)$.
  Show that $x$ is algebraic over $K$, and conclude that $[F(x) : K]$ is finite.
  
  \begin{proof}
    Let $t \in K \setminus F$ be given.
    Since $K \subseteq F(x)$, there exist co-prime $f, g \neq 0 \in F[x]$ such that $t = f(x) / g(x)$.
    Observe that since $t \not \in F$, it must be the case that $\max\left\{\deg{f(x)}, \deg{g(x)}\right\} \geq 1$.
    Considering $f(y) - tg(y) \in K[y]$, it follows that $x$ is a root.
    Hence $x$ is algebraic over $K$ and there exists some minimal irreducible polynomial, $m_{x,K}(y) \in K[y]$, dividing $f(y) - tg(y)$.
    Therefore it follows from the fact that $\deg{f(y) - tg(y)}$ is finite that $F(x)/K$ is finite and algebraic.
  \end{proof}
\end{thm}

\begin{thm}
  Let $K$ be a field, and let $p(x), q(x) \neq 0 \in K[x]$ be relatively prime with $t = p(x)/q(x) \in K(x)$.
  \begin{enumerate}[(a)]
  \item
    Let $r(y) = p(y) - tq(y) \in K(t)[y]$.
    Show that $r(y)$ is irreducible in $K(t)[y]$ and has $x$ as a root.
  \item
    Show that $r(y)$ has degree in $K(t)[y]$ equal to $\max\{\deg{p}, \deg{q}\}$.
  \item
    Conclude that $[K(x) : K(t)] = [K(x) : K(p(x)/q(x))] = \max\{\deg{p}, \deg{q}\}$.
  \end{enumerate}
  \begin{proof}
    \begin{enumerate}[(a)]
    \item\label{2.a}
      First observe that if $x \in K$, then the results hold trivially.
      Assume $x \not \in K$.
      It then follows directly that if $r(y)$ were to have $x$ as a root, it must be the case that $t \not \in K$.
      Consider $r$ in the polynomial ring $(K[y])[t]$.
      Since $r$ is linear in $t$ and $p,q$ are co-prime, it follows that if $r = a \cdot b$ for some $a,b \in (K[y])[t]$, one of $a,b \in K[y]^\times = (K[y])[t]^\times = K^\times$.
      Therefore $r$ is irreducible in $(K[y])[t] = (K[t])[y]$ and, by Gauss' Lemma, irreducible in $K(t)[y]$.
      Finally observe that if $\epsilon_x \colon (K[t])[y] \rightarrow K[t]$ is the usual evaluation homomorphism, it follows from $t = p(x)/q(x)$ that $r \in \ker \epsilon_x$.
    \item\label{2.b}
      Since $t \not \in K$ and $p,q \in K[y]$, it follows that there can be no cancellation in the difference $p(y) - tq(y)$.
      Therefore, considering $p(y) - tq(y)$ as a polynomial in $y$ with coefficients in the field $K(t)$, $\deg{p(y) - tq(y)} = \max\{\deg{p}, \deg{q}\}$.
    \item
      By part \eqref{2.a}, $r(y)$ is the minimal polynomial for $x$ over $K(t)$.
      Therefore by part \eqref{2.b}, $[K(x) : K(t)] = [K(x) : K(p(x)/q(x))] = \deg{r} = \max\{\deg{p}, \deg{q}\}$.
    \end{enumerate}
  \end{proof}
\end{thm}

\begin{thm}
  Let $F$ be a field with characteristic $p > 0$.
  Suppose that $K/F$ is finite and that $p \nmid [K : F]$.
  Show that $K/F$ is separable.
  \begin{proof}
    Since $K$ is finite, there exist $\alpha_1, \ldots, \alpha_n$ such that $K = F(\alpha_1, \ldots, \alpha_n)$.
    It suffices to show that each $\alpha_i$ is separable.
    Towards that end, construct the tower 
    $$F_0 = F \subseteq F_1 = F_0(\alpha_1) \subseteq \ldots F_n = F_{n-1}(\alpha_{n}) = K.$$
    Since indices are multiplicative in towers, we have $[K:F] = \prod_{i=0}^{n-1} [F_{i+1} : F_i]$.
    Let $f_i = m_{\alpha_i, F}$ be the minimal polynomial for $\alpha_i$.
    If any $\alpha_i$ is inseparable, then by the theorem from class, we have $f_i(x) = f_{i, \text{sep}}(x^{p^{k_{i}}})$ and
    $$[F_{i+1} : F_i]_i = \frac{[F_{i+1} : F_{i}]}{[F_{i+1} : F_i]_s} = p^{k_{i+1}}.$$
    However, since $p \nmid [K : F]$ and $[F_{i+1} : F_i] \mid [K : F]$ for each $0 \leq i \leq n-1$, it follows that $p \nmid [F_{i+1} : F_i]$.
    Therefore $k_{i+1} = 0$, $[F_{i+1} : F_i] = [F_{i+1} : F_i]_s$ and each $\alpha_i$ is separable, as desired.
  \end{proof}
\end{thm}

\begin{thm}
  Let $p$ be prime, let $x$ and $y$ be independent transcendentals over $\F_p$, let $K = \F_p(x,y)$, and let $F = \F_p(x^p - x, y^p - x)$.
  \begin{enumerate}[(a)]
  \item
    Prove $[K : F] = p^2$ and that the separable and inseparable degrees of $K/F$ are both equal to $p$.
  \item
    Prove that there is a field $E$ with $F \subseteq E \subseteq K$ which is purely inseparable over $F$ of degree $p$.
  \end{enumerate}

  \begin{proof}
    \begin{enumerate}[(a)]
    \item
      First observe that $x$ is a root of $t^p - t - x^p + x \in F[t]$ and so $m_{x, F}(t) \mid t^p - t - x^p + x$.
      By an argument similar to Exercise 2 on Homework 3, replacing $\F_p[x]$ by $F[t]$ and letting $a = x^p + x$, $m_{x, F}(t) = t^p - t - x^p - x$.
      Then $D_x(m_{x,F}(t) = -1$, which is co-prime to $m_{x,F}(t)$.
      Therefore $F(x)/F$ is a separable extension of degree $p$.
      
      Now we have that $(y^p - x) + x = y^p \in F(x)$ and so $y$ is a root of $t^p - y^p = (t - y)^p \in F(x)[t]$.
      By unique factorization in the Euclidean domain $F(x)[t]$, we have that $m_{y, F(x)}(t) = (t - y)^n$ for some $n \leq p$.
      Suppose to the contrary that $n < p$ and observe that $y^n \in F(x)$.
      Since $\gcd(p,n) = 1$ there exist integers $a$ and $b$ such that $an + bp = 1$, and thus $(y^n)^a(y^p)^b = y^{an + bp} = y \in F(x)$, a contradiction.
      Therefore $F(x,y)/F(x)$ is an inseparable extension of degree $p$. 
    \item
      Let $u = x^p - x$ and $v = y^p - x$.
      Then $u - v = x^p - y^p = (x - y)^p$ which is a root of $t^p - (x - y)^p = (t - (x - y))^p \in F[t]$.
      Hence by unique factorization in the Euclidean domain $F[t]$, it follows that $m_{(x-y), F}(t) = (t - (x - y))^n$ for some $n \leq p$ and $E = F(x - y) \subset K$.
      Observe then that if $x - y \in F$, then $x - (x - y) = y \in F(x)$ and $F(x,y)$, a degree $p^2$ extension, is contained in the degree $p$ extension $\subseteq F(x)$, which is absurd.
      Therefore $E$ is a purely inseparable extension of $F$.
    \item
      Observe that it suffices to show for each $i, \ell$ that 
    \end{enumerate}
  \end{proof}
\end{thm}
\end{document}
