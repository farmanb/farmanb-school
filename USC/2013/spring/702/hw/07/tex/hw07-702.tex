\documentclass[10pt]{amsart}
\usepackage{amsmath,amsthm,amssymb,amsfonts,enumerate,mymath,mathtools,tikz}
\usetikzlibrary{shapes}

\openup 5pt
\author{Blake Farman\\University of South Carolina}
\title{Math 702:\\Homework 05}
\date{March 1, 2013}
\pdfpagewidth 8.5in
\pdfpageheight 11in
\usepackage[margin=1in]{geometry}

\begin{document}
\maketitle

\providecommand{\p}{\mathfrak{p}}
\providecommand{\m}{\mathfrak{m}}

\newtheorem{thm}{}
\newtheorem{lem}{Lemma}

\newcommand{\End}[2]{\operatorname{End}_{#1}\left(#2\right)}
\newcommand{\Hom}[2]{\operatorname{Hom}_{#1}\left(#2\right)}

\begin{thm}
  Let $F$ be a field of characteristic $p > 0$.
  Suppose that $f(x) = x^p - x - a \in F[x]$ has no root in $F$.
  Prove that $f$ has splitting field $K/F$, Galois with $\Gal{K/F} \cong \Z/p\Z$.

  \begin{proof}
    First observe that $D_x(f) = -1$ implies that $f$ is separable and thus has $p$ distinct roots.
    Let $\alpha$ be any root of $f$ in $K/F$.
    Since the prime subfield of $F$ is $\F_p$, it follows that for $1 \leq i \leq p-1$
    $$(\alpha + i)^p - (\alpha + i) - a = \alpha^p - i - (\alpha + i) - a = \alpha^p - \alpha - a = 0.$$
    Hence the $p$ roots of $f$ are $\alpha + i$ for $0 \leq i \leq p-1$.
    Then $F(\alpha)/F \cong F[x]/(f)$ is a degree $p$ extension of $F$ containing all the roots of $f$, and is the splitting field of $f$ over $F$.
    Hence $K/F = F(\alpha)/F$ is Galois of degree $p$.
    Therefore $\Gal{K/F}$ is cyclic of order $p$, as desired.
  \end{proof}
\end{thm}
\end{document}
