\documentclass[dissertation.tex]{subfiles}
\begin{document}
\subsection{Tensored and Cotensored Categories}

\begin{defn}[\cite{Borceux2}]
  Let $\V$ be a symmetric monoidal closed category, $\C$ a $\V$-category, $X$ an object of $\C$, and $V$ an object of $\V$.
  \begin{itemize}
  \item
    The {\it tensor of $V$ and $X$} exists if there is an object $V \otimes X$ of $\C$ together with isomorphisms
    $$\C(V \otimes X, Y) \cong [V, \C(X,Y)]_{\V}$$
    of $\V$, $\V$-natural in the object $Y$ of $\C$.
    We say that $\C$ is tensored if $V \otimes X$ exists for all objects $V$ of $V$ and $X$ of $\C$.
  \item
    The {\it cotensor of $V$ and $X$} exists if there is an object $[V,X]$ of $\C$ together with isomorphisms
    $$\C(Y, [V,X]) \cong [V, \C(Y,X)]_{\V}$$
    of $\V$, $\V$-natural in the object $Y$ of $\C$.
    We say that $\C$ is cotensored if $[V,X]$ exists for all objects $V$ of $V$ and $X$ of $\C$.
  \end{itemize}
\end{defn}

\begin{prop}[\cite{Borceux2}]
  If $\V$ is a symmetric monoidal closed category, then $\V$ is both a tensored and cotensored $\V$-category, with tensor $\otimes$ and cotensor $[- , -]_{\V}$
\end{prop}

\begin{prop}[\cite{Borceux2}, 6.5.5]
  Let $\V$ be a symmetric monoidal closed category and let $V$ be an object of $\V$.
  If $\C$ is a tensored (resp. cotensored) $\V$-category, then the correspondence
  $$V \otimes - \colon \C \to \C, X \mapsto V \otimes X$$
  (resp. $[V, -] \colon \C \to \C, X \mapsto [V, X]$)
  induced a $\V$-functor.
\end{prop}

\begin{prop}[\cite{Borceux2}, 6.5.6]
  Let $\V$ be a symmetric monoidal closed category.
  If $\C$ is a tensored and cotensored $\V$-category, then for every object $V$ of $\V$ the two $\V$-functors above induce isomorphisms
  $$\C(V \otimes X, Y) \cong \C(X, [V,Y]),$$
  $\V$-natural in $X$ and $Y$.
\end{prop}

\begin{prop}[\cite{Borceux2}, 6.5.7]\label{modulesaretensored}
  Let $\V$ be a complete symmetric monoidal closed category.
  For every small $\V$-category $\C$, the $\V$-category of $\V$-functors from $\C$ to $\V$ and $\V$-natural transformations is both tensored and cotensored.
\end{prop}
\end{document}
