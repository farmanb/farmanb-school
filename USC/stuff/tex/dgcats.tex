\documentclass[dissertation.tex]{subfiles}
\begin{document}
\section{Differential Graded Modules and Algebras}
{\noindent Throughout, let $k$ be a commutative ring.}
        
\begin{defn}
  A {\it differential graded $k$-module}, $M$, is a complex of $k$-modules
  $$M \colon \begin{tikzcd}
    \cdots \arrow{r}{d_M^{n-2}} & M^{n-1} \arrow{r}{d_M^{n-1}} & M^n \arrow{r}{d_M^n} & M^{n+1} \arrow{r}{d_M^{n+1}} & \cdots
  \end{tikzcd},$$
  or, equivalently, a $\Z$-graded module over the graded ring $k$, concentrated in degree 0, equipped with a degree one morphism $d_M : M \rightarrow M[1]$ such that $d_M^2 = 0$.
  A {\it morphism of differential graded $k$-modules} is a morphism of chain complexes.
  
  The {\it shift} of a dg $k$-module is the shifted complex
  $$M[1] \colon \begin{tikzcd}
    \cdots \arrow{r}{-d_M^{n-1}} & M^{n} \arrow{r}{-d_M^{n}} & M^{n+1} \arrow{r}{-d_M^{n+1}} & M^{n+2} \arrow{r}{-d_M^{n+2}} & \cdots
  \end{tikzcd}$$
  
  The tensor product of two dg $k$-modules is the usual tensor product in $\Ch{k}$, the category of chain complexes of $k$-modules.
\end{defn}

\begin{defn}
  We say a category, $\C$, is a {\it $k$-category} if
  \begin{enumerate}
  \item
    for each pair of objects $X$ and $Y$ of $\C$, $\Hom{\C}{X,Y}$ is a $k$-module, and
  \item
    composition 
    $$\Hom{\C}{X,Y} \otimes_k \Hom{\C}{Y,Z} \rightarrow \Hom{\C}{X,Z}$$
    in $\C$ is $k$-linear, associative, admitting units $\id_X \in \Hom{\C}{X,X}$.
  \end{enumerate}
\end{defn}

\section{The Category of Differential Graded Categories}

\begin{defn}
  A {\it differential graded} or {\it dg category} is a $k$-category, $\C$, satisfying
  \begin{enumerate}
  \item
    $\Hom{\C}{X,Y}$ is a dg $k$-module for all objects $X,Y$ of $\C$, and
  \item
    composition 
    $$\begin{tikzcd}
      \Hom{\C}{X,Y} \otimes_k \Hom{\C}{Y,Z} \arrow{r} & \Hom{\C}{X,Z}
    \end{tikzcd}$$
    in $\C$ is a morphism of dg $k$-modules;
    that is a morphism of $\Ch{k}$.
  \end{enumerate}
  
  A {\it morphism of dg categories}, or {\it dg functor}, $\mathscr{F} \colon \C \rightarrow \D$, is a functor such that
  $$F(X,Y) \colon 
  \begin{tikzcd}
    \Hom{\C}{X,Y} \arrow{r} & \Hom{\D}{FX,FY}
  \end{tikzcd}$$
  is a morphism of dg $k$-modules.
  Denote by $\op{dgcat}_k$ the category with objects small dg categories and morphisms dg functors.
\end{defn}

\begin{eg}
  \begin{description}[style=nextline]
  \item[dg category with one object]
    Let $A$ be a dg $k$-algebra; that is, a graded $k$-algebra equipped with a differential
    $$d(fg) = d(f)g + (-1)^nfd(g),\ f \in A^n, g \in A.$$
    Define $\mathscr{A}$ to be the category with one object, $\ast$, and morphisms $\mathscr{A}(\ast, \ast) = A$, with composition defined by the multiplication in $A$.
  \item[dg $k$-modules]
    Define the category $\cdg{k}$ to have objects chain complexes of $k$-modules, and morphisms $\cdg{k}(C,D) = [C,D]_{\Ch{k}}$, the internal hom for $\Ch{k}$ as in Definition~\ref{ChInternalHomDefn}.
    The composition of morphisms
    $$\begin{tikzcd}
      \cdg{k}(C,D) \otimes_k \cdg{k}(D,E) \arrow{r} & \cdg{k}(C,E)
    \end{tikzcd}$$
    is the composition morphism of Remark~\ref{MorphismsOfDGModules} (\ref{MorphismsOfDGModules.composition}).
    
    As an immediate consequence of Proposition~\ref{ChHomTensorAdjunction} we have the following observation:
    \begin{prop}
      %The category $\mathcal{C}_\text{dg}(k)$ is both a tensored and cotensored $\Ch{k}$-category.
      For any three chain complexes, $C$, $D$, and $E$, there is a canonical isomorphism of chain complexes
      $$\cdg{k}(C \otimes_k E, D) \cong \cdg{k}(C,\cdg{k}(E,D)).$$
      
      \begin{proof}
        Let $X$ be any object of $\Ch{k}$.
        Using the adjunction of Proposition~\ref{ChHomTensorAdjunction} we have
        \begin{eqnarray*}
          \Ch{k}(X,\cdg{k}(C \otimes_k E, D) &\cong& 
          \Ch{k}(X \otimes_k (C \otimes_k E), D)\\
          &\cong& \Ch{k}((X \otimes_k C) \otimes_k E, D)\\
          &\cong& \Ch{k}(X \otimes_k C, \cdg{k}(E,D))\\
          &\cong& \Ch{k}(X, \cdg{k}(C, \cdg{k}(E,D))).
        \end{eqnarray*}
        Therefore by the Yoneda Lemma there is a unique isomorphism 
        $$\cdg{k}(C \otimes_k E, D) \cong \cdg{k}(C, \cdg{k}(E,D)).$$
      \end{proof}
    \end{prop}
    %Note that $[C,D]$ is, equivalently, the graded module of graded morphisms; that is, a morphism $f \in \mathcal{C}_\text{dg}(k)(C, D)^m$ is a collection of morphisms $f^n \colon C^n \rightarrow D^{n+m}$.
    %Equip $\mathcal{C}_\text{dg}(k)(C, D)$ with the differential
    %$$\begin{tikzcd}
    %  \mathcal{C}_\text{dg}(k)(C, D)^n \arrow{r} & \mathcal{C}_\text{dg}(k)(C, D)^{n+1}\\
    %  f \arrow[mapsto]{r} & d_D \circ f + (-1)^{n+1}f \circ d_C
    %\end{tikzcd}$$
    %and define composition of morphisms by the tensor product in $\mathcal{C}(k)$,
    %$$\begin{tikzcd}
    %  \mathcal{C}_\text{dg}(k)(D, E) \otimes_k \mathcal{C}_\text{dg}(k)(C, D) \arrow{r} & \mathcal{C}_\text{dg}(k)(C, E).
    %\end{tikzcd}$$
  \end{description}
\end{eg}

\begin{defn}
  Let $\C$ and $\D$ be objects of $\op{dgcat}_k$.
  The {\it tensor product}, $\C \otimes \D$, is the dg category with objects $\op{ob}\C \times \op{ob}\D$ and
  morphisms
  $$\begin{tikzcd}
    \Hom{\C \otimes \D}{(X, Y), (X^\prime, Y^\prime)} = \Hom{\C}{X,X^\prime} \otimes_k \Hom{\D}{Y,Y^\prime}.
  \end{tikzcd}$$
  For ease of notation, we will denote by $X \otimes Y$ the object $(X,Y)$ of $\C \otimes \D$.
\end{defn}

\begin{defn}
  Given two dg functors $\mathscr{F},\mathscr{G} \colon \C \rightarrow \D$, define $\SHom{\mathscr{F},\mathscr{G}}^n$ to be the $k$-module of degree $n$ natural transformations.
  That is, morphisms of functors, $\eta \colon \mathscr{F} \rightarrow \mathscr{G}$ such that for each object $X$ of $\C$, $\eta(X) \in \Hom{\D}{\mathscr{F}(X),\mathscr{G}(X)}^n$.
\end{defn}

\begin{prop}
  Given a degree $n$ natural transformation $\eta \colon \mathscr{F} \rightarrow \mathscr{G}$, the collection of morphisms
  $$d^n_{\Hom{\D}{\mathscr{F}X, \mathscr{G}X}}\left( \eta(X) \right) \in \Hom{\D}{\mathscr{F}X, \mathscr{G}X}^{n+1}$$
  define a natural transformation of degree $n + 1$ and hence endow $\SHom{\mathscr{F},\mathscr{G}}$ with the structure of a dg $k$-module, where the differential, $d_{\SHom{\mathscr{F},\mathscr{G}}}^n$ sends $\eta$ to this natural transformation.
  
  \begin{proof}
    It's clear that so long as the collection of $d^n_{\D(\F X, \G X)}(\eta(X))$ defines a natural transformation, the resulting sequence
    $$\begin{tikzcd}
      \cdots \arrow{r} & \SHom{\F, \G}^n \arrow{r} & \SHom{\F,\G}^{n+1} \arrow{r} & \cdots
    \end{tikzcd}$$
    will be a complex.
    
    First we note that, by definition, any dg functor necessarily commutes with the differentials:
    $$d_{\D(\mathscr{F}X, \mathscr{F}X^\prime)}\left(\mathscr{F}(f) \right) = \mathscr{F}\left(d_{\C(X,X^\prime)}(f)\right)$$
    and composition is a morphism of dg $k$-modules, so we have the commutative diagrams
    $$\begin{tikzcd}
      \D(\F X^\prime, \G X^\prime) \otimes_k \D(\F X, \F X^\prime) \arrow{r}\arrow[swap]{d}{d \otimes 1 + 1 \otimes d}& \D(\F X, \G X^\prime)\arrow{d}{d}\\
      \D(\F X^\prime, \G X^\prime) \otimes_k \D(\F X, \F X^\prime)[1] \arrow{r}& \D(\F X, \G X^\prime)[1]
    \end{tikzcd}$$
    and
    $$\begin{tikzcd}
      \D(\G X, \G X^\prime) \otimes_k \D(\F X, \G X) \arrow{r}\arrow[swap]{d}{d \otimes 1 + 1 \otimes d}& \D(\F X, \G X^\prime)\arrow{d}{d}\\
      \D(\G X, \G X^\prime) \otimes_k \D(\F X, \G X)[1] \arrow{r}& \D(\F X, \G X^\prime)[1]
    \end{tikzcd}$$
    For a morphism $f \in \C(X,X^\prime)$, chasing $\eta(X^\prime) \otimes \F(f)$ and $\G(f) \otimes \eta(X)$ through the diagram gives
    \begin{eqnarray*}
      d(\eta(X^\prime)) \circ \F(f) + \eta(X^\prime) \circ d(\F(f)) 
      &=& d(\eta(X^\prime) \circ \F(f))\\
      &=& d(\G(f) \circ \eta(X))\\
      &=& d(\G(f)) \circ \eta(X) + \G(f) \circ d(\eta(X)).
    \end{eqnarray*}
    By the fact that $\F$,$\G$ commute with differentials and $\eta$ is a natural tranformation, we see
    $$\eta(X^\prime) \circ d(\F(f)) = \eta(X^\prime) \circ \F(d(f)) = \G(d(f)) \circ \eta(X) = d(\G(f)) \circ \eta(X).$$
    Therefore
    $$d(\eta(X^\prime)) \circ \F(f) = \G(f) \circ d(\eta(X)),$$
    as desired.
  \end{proof}
\end{prop}

\begin{defn}
  For two objects $\C$ and $\D$ of $\op{dgcat}_k$, define the object $\SHom{\C,\D}$ of $\op{dgcat}_k$ to be the category with objects dg functors $\mathscr{F} \colon \C \rightarrow \D$ and morphisms $\SHom{\mathscr{F},\mathscr{G}}$.
  
  Given two dg functors $\F,\G \colon \C \rightarrow \D$, define a {\it morphism of dg functors}, $\eta \colon \F \rightarrow \G$ to be a closed, degree zero natural transformation.
  That is, $\eta \in \SHom{\F, \G}^0$ and its image in $\SHom{\F,\G}^1$ under the differential is zero.
\end{defn}

\begin{rmk}
  There is a natural isomorphism of bifunctors
  $$\Hom{\op{dgcat}_k}{\C \otimes \D, \mathscr{E}} \cong \Hom{\op{dgcat}_k}{\C, \SHom{\D,\mathscr{E}}},$$
  which endows $\op{dgcat}_k$ with a symmetric monoidal closed structure.

  For $X,\A, \B, \C$ dg-categories, we have by abstract nonsense
  \begin{eqnarray*}
    \dgcat{k}\left(X, \SHom{\A \otimes \B, \C}\right) &\cong& \dgcat{k}\left(X \otimes \A \otimes \B, \C\right)\\
    &\cong& \dgcat{k}\left(X \otimes \A, \SHom{\B, \C}\right)\\
    &\cong& \dgcat{k}\left(X, \SHom{\A,\SHom{\B, \C}}\right)
  \end{eqnarray*}
  and hence $\SHom{\A \otimes \B, \C} \cong \SHom{\A, \SHom{\B, \C}}$ follows from the enriched Yoneda Lemma.
  For $\C = \cdg{k}$, we can write down an explicit isomorphism between dg-bimodules and $\Ch{k}$-enriched Hom sets,
  $$\cdg{\A \otimes \B^\text{op}} = \SHom{\A \otimes \B^\text{op}, \cdg{k}} \cong \SHom{\A, \SHom{\B^\opp, \cdg{k}}} = \SHom{\A,\cdg{\B^\opp}}.$$
  
  Given a bimodule $\funct{P}$, we obtain a dg-functor $\Phi_{\funct{P}} \colon \A \to \cdg{\B^\text{op}}$ which takes objects to the right dg $\B$-module 
  $$\Phi_{\funct{P}}(X)(Y) = \funct{P}(X,Y)$$
  with structure morphism
  $$\begin{tikzcd}
    \B(Y^\prime, Y) \cong k \otimes_k \B(Y^\prime, Y) \arrow{rr}{u^\A_X \otimes 1} && 
    \A(X,X) \otimes_k \B(Y^\prime, Y) \arrow{rr}{\funct{P}(X \otimes Y^\prime, X \otimes Y)} &&
    \cdg{k}(\funct{P}(X,Y), \funct{P}(X,Y^\prime))
  \end{tikzcd}$$
  For a morphism $f \in \A(X,X^\prime)$ we define $\Phi_\funct{P}(f) \colon \Phi_\funct{P}(X) \to \Phi_\funct{P}(X^\prime)$ by the collection of morphisms
  $$\begin{tikzcd}
    \A(X,X^\prime) \cong \A(X,X^\prime) \otimes_k k \arrow{rr}{1 \otimes u^\B_Y} && \A(X,X^\prime) \otimes \B(Y,Y) \arrow{rr}{\funct{P}(X \otimes Y, X^\prime \otimes Y)} && \cdg{k}(\funct{P}(X,Y), \funct{P}(X^\prime, Y)).
  \end{tikzcd}$$

  We have commutative diagrams
  $$
  \begin{tikzcd}
    k \otimes_k \A(X,X^\prime) \otimes_k \B(Y^\prime, Y) \arrow{r}{\sim}\arrow{dd}{u_{X \otimes Y^\prime} \otimes_k 1} & \A \otimes \B(X \otimes Y^\prime,X^\prime \otimes Y) \arrow[equals]{dd}\\
    \\
    \A(X,X) \otimes_k \B(Y^\prime, Y^\prime) \otimes_k \A(X,X^\prime) \otimes_k \B(Y^\prime, Y)\arrow{r} \arrow{dd}[rotate=90,yshift=-1ex]{\sim} & \A \otimes \B(X \otimes Y^\prime,X^\prime \otimes Y) \arrow{dddd}{\funct{P}(X \otimes Y^\prime, X^\prime \otimes Y)}\\
    \\
    \A(X,X) \otimes_k \B(Y^\prime, Y) \otimes_k \A(X,X^\prime) \otimes_k \B(Y^\prime, Y^\prime)\arrow{dd}{\funct{P}(X \otimes Y^\prime, X \otimes Y) \otimes_k \funct{P}(X \otimes Y^\prime, X^\prime \otimes Y^\prime)} \\
    \\
    \cdg{k}(\funct{P}(X,Y), \funct{P}(X,Y^\prime)) \otimes_k \cdg{k}(\funct{P}(X,Y^\prime),\funct{P}(X^\prime, Y^\prime))\arrow{r} & \cdg{k}(\funct{P}(X,Y), \funct{P}(X^\prime, Y^\prime)) 
  \end{tikzcd}
  $$
  and
  $$
  \begin{tikzcd}
    \A \otimes \B(X \otimes Y^\prime,X^\prime \otimes Y) \arrow[equals]{dd} & \A(X,X^\prime) \otimes_k \B(Y^\prime, Y)  \otimes_k k\arrow[swap]{l}{\sim}\arrow{dd}{1 \otimes_k u_{X^\prime \otimes Y}}\\
    \\
    \A \otimes \B(X \otimes Y^\prime,X^\prime \otimes Y) \arrow{dddd}{\funct{P}(X \otimes Y^\prime, X^\prime \otimes Y)}  & \A(X,X^\prime) \otimes_k \B(Y^\prime, Y) \otimes_k \A(X^\prime, X^\prime) \otimes_k \B(Y,Y)\arrow{l}\arrow{dd}[rotate=90,yshift=-1ex]{\sim}\\
    \\
    & \A(X,X^\prime) \otimes_k \B(Y,Y) \otimes_k \A(X^\prime, X^\prime) \otimes_k \B(Y^\prime, Y)\arrow[swap]{dd}{\funct{P}(X \otimes Y, X^\prime \otimes Y) \otimes_k \funct{P}(X^\prime \otimes Y^\prime, X^\prime \otimes Y)}\\
    \\
    \cdg{k}(\funct{P}(X,Y), \funct{P}(X^\prime, Y^\prime)) & \cdg{k}(\funct{P}(X,Y), \funct{P}(X^\prime, Y)) \otimes_k \cdg{k}(\funct{P}(X^\prime,Y), \funct{P}(X^\prime, Y^\prime))\arrow{l}
  \end{tikzcd}
  $$
  with top horizontal morphisms the canonical isomorphisms, right isomorphisms the associativity isomorphisms, and horizontal morphisms composition.
  The top squares commute because $\A \otimes \B^\opp$ is a dg-category, and the bottom squares commute because $\funct{P}$ is dg-functor.
  We then see that the naturality squares
  $$\begin{tikzcd}
    X \arrow{dd}{f} & Y^\prime \arrow{dd}{g} & \funct{P}(X,Y) \arrow{rr}{\Phi_{\funct{P}}(f)(Y^\prime)}\arrow{dd}{\Phi_{\funct{P}}(g)} && \funct{P}(X^\prime, Y)\arrow{dd}{\Phi_{\funct{P}}(X^\prime)(g)}\\
    \\
    X^\prime & Y & \funct{P}(X,Y^\prime) \arrow{rr}{\Phi_{\funct{P}}(f)(Y)} && \funct{P}(X^\prime,Y^\prime)
  \end{tikzcd}$$
  commute by chasing these diagrams.
  This makes $\Phi_\funct{P}$ a dg-functor.
  A morphism $\eta \colon \funct{P} \to \funct{P}^\prime$ is a collection of morphisms $\eta(X \otimes Y) \colon \funct{P}(X \otimes Y) \to \funct{P}^\prime(X \otimes Y)$, which we can view as a morphism of functors $\Phi(\eta) \colon \Phi_{\funct{P}} \to \Phi_{\funct{P}^\prime}$ by $\Phi(\eta)(X)(Y) = \eta(X \otimes Y)$.

  Conversely, if we have a dg-functor $\F \colon \A \to \cdg{\B^\opp}$, we can construct a dg $\A \otimes \B^\opp$-module as follows.
  Consider the two natural functors
  $$\begin{tikzcd}
    \B^\opp \arrow{r}{h_-} & \cdg{\B^\opp} \arrow{rrr}{\cdg{\B^\opp}(-, \F X)} &&& \cdg{k}
  \end{tikzcd}
  \ \text{and}\ 
  \begin{tikzcd}
    \A \arrow{r}{\F} & \cdg{\B^\opp} \arrow{rrr}{\cdg{\B^\opp}(h_Y, -)} &&& \cdg{k}
  \end{tikzcd}$$
  for fixed $X$ and $Y$.
  We obtain a chain complex for each pair of objects $X$ of $\A$ and $Y$ of $\B$ from
  $$\cdg{\B^\opp}(h_Y, \F(X)) \cong \F(X)(Y)$$
  and a structure morphism
  $$\begin{tikzcd}
    \A(X,X^\prime) \otimes_k \B(Y^\prime, Y)\arrow{d}{\F(X,X^\prime) \otimes_k h_-(Y^\prime, Y)}\\
    \cdg{\B^\opp}(\F(X), \F(X^\prime)) \otimes_k \cdg{\B^\opp}(h_{Y^\prime}, h_Y)\arrow{d}{\cdg{\B^\opp}(h_Y, -)(\F(X),\F(X^\prime)) \otimes_k \cdg{\B^\opp}(-, \F(X^\prime))(h_{Y^\prime}, h_Y)}\\
    \cdg{k}(\F(X)(Y), \F(X^\prime)(Y)) \otimes_k \cdg{k}(\F(X^\prime)(Y), \F(X^\prime)(Y^\prime))\arrow{d}{c_{\F(X)(Y), \F(X^\prime)(Y), \F(X^\prime)(Y^\prime)}}\\
    \cdg{k}(\F(X)(Y), \F(X^\prime)(Y^\prime))
  \end{tikzcd}$$
  If we denote by $\mathbbm{k}$ the unit of $\dgcat{k}$, which has one object and endomorphisms the complex with $k$ in degree zero, then a morphism $\eta \colon \F \to \G$ is simply a collection of morphisms $\eta(X) \colon \mathbbm{k} \to \cdg{\B^\opp}(\F(X),\G(X))$ as in Definition~\ref{VNatDefn}, which is mapped to the collection of morphisms $\eta(X)(Y) \colon \cdg{k}(\F(X)(Y),\G(X)(Y))$ which clearly make the naturality diagrams
  $$\begin{tikzcd}
    X \otimes Y^\prime \arrow{d} & \F(X)(Y^\prime) \arrow{r}{\eta(X)(Y)}\arrow{d} & \G(X)(Y^\prime)\arrow{d}\\
    X^\prime \otimes Y & \F(X^\prime)(Y) \arrow{r}{\eta(X^\prime)(Y^\prime)} & \G(X^\prime)(Y)
  \end{tikzcd}$$
  commute.
\end{rmk}

\begin{defn}
  Let $\C$ be a dg category.
  Define
  \begin{enumerate}
  \item
    the category $Z^0(\C)$ to be the category with objects those of $\C$ and morphisms
    $$\Hom{Z^0(\C)}{X,Y} = Z^0\left(\Hom{\C}{X,Y}\right),$$
  \item
    the category $H^0(\C)$ to be the category with objects those of $\C$ and morphisms
    $$\Hom{H^0(\C)}{X,Y} = H^0\left(\Hom{\C}{X,Y}\right),$$
  \item
    the {\it homology category}, $H^\ast(\C)$, to be the category with objects those of $\C$ and morphisms 
    $$H^\ast\C(X,Y) = \bigoplus H^n\C(X,Y).$$
  \end{enumerate}
\end{defn}

\begin{rmk}\label{DGInducedHomology}
  Note that given a dg functor, $\mathscr{F} \colon \C \rightarrow \D$, for $X$ and $Y$ objects of $\C$, 
  $$\mathscr{F}(X,Y) \colon \C(X,Y) \rightarrow \D(\mathscr{F}X, \mathscr{F}Y)$$
  is a morphism of $\mathcal{C}(k)$ and this immediately implies the diagram
  $$\begin{tikzcd}
    B^n(\C(X,Y))\arrow{r}{\imath^n}\arrow[bend left]{rr}{\im d^{n-1}}\arrow[dashed]{d}{\exists !B^n(\F(X,Y))} & Z^n(\C(X,Y)) \arrow{r}{\ker d^n}\arrow[dashed]{d}{\exists !Z^n(\F(X,Y))} & \C(X,Y)^{n} \arrow{d}{\F(X,Y)}\\
    B^n(\D(\F X,\F Y) \arrow{r}{\imath^n}\arrow[bend right]{rr}{\im d^{n-1}} & Z^n(\D(\F X,\F Y))\arrow{r}{\ker d^n} & \D(\F X,\F Y)^{n}
  \end{tikzcd}$$
  commutes for each $n$.
  Hence $H^n$ induces a functor
  $H^n(\mathscr{F}) \colon H^n(\C) \rightarrow H^n(\D)$
  with object map\\ $H^n(\mathscr{F})(X) = \mathscr{F}(X)$ and induced arrow map 
  $$\begin{tikzcd}
    0 \arrow{r} & B^n(\C(X,Y))\arrow{r}{\imath^n}\arrow{d}{B^n(\F(X,Y))} & Z^n(\C(X,Y)) \arrow{r}{\coker \imath^n}\arrow{d}{Z^n(\F(X,Y))} & H^n(\C(X,Y)) \arrow[dashed]{d}{\exists !H^n(\F(X,Y))}\arrow{r} & 0 \\
    0 \arrow{r} & B^n(\D(\F X,\F Y) \arrow{r}{\imath^n} & Z^n(\D(\F X,\F Y))\arrow{r}{\coker \imath^n} & H^n(\D(\F X,\F Y)) \arrow{r} & 0
  \end{tikzcd}$$
\end{rmk}

\section{Modules over a Differential Graded Category}

\begin{defn}
  Let $\C$ be a small dg category and let $\mathscr{M} : \C \rightarrow \cdg{k}$ be a dg functor.
  \begin{enumerate}
  \item
    We say that $\mathscr{M}$ is a right (resp. left) dg $\C$-module if $\mathscr{M}$ is contravariant (resp. covariant).
  \item
    The {\it homology of a dg $\C$-module}, $\mathscr{M}$, is the induced functor
    $$\begin{tikzcd}
      H^\ast(\C) \arrow{r}{H^\ast(\mathscr{M})}& \Gr{k}\\
      X \arrow[mapsto]{r} & H^\ast(\mathscr{M}(X))
      %H^\ast(\C(Y,X)) \arrow[mapsto]{r} & H^\ast(\mathcal{C}_\text{dg}(\mathscr{M}(X),\mathscr{M}(Y))
    \end{tikzcd}$$
    where $\Gr{k}$ denotes the category of graded modules.
  \item
    Define the category of dg $\C$-modules $\cdg{\C} = \SHom{\C, \cdg{k}}$, and the category of right $\C$-modules by
    $$\mathcal{C}(\C) = Z^0 \cdg{\C}.$$
    Note that the morphisms of this category are just $\Ch{k}$-natural transformations and this endows $\mathcal{C}(\C)$ with the structure of an abelian category.
    %TODO: This should probably be justified at some point; either in here, or somewhere else.  It's not immediately clear at the moment.
  \item
    A morphism $\eta \colon \mathscr{L} \rightarrow \mathscr{M}$ of $\C$-modules is the data of a collection of morphisms $\eta(X) \in \Ch{k}(\funct{L}X,\funct{M}X)$, natural in $X$.
    As such, each $\eta(X)$ induces a morphism
    $$H^\ast(\eta(X)) \colon H^\ast(\funct{L}X) \to H^\ast(\funct{M}X)$$
    and naturality of $\eta$ ensures naturality of $H^\ast(\eta)$, giving a morphism
    $$H^\ast(\eta) \colon H^\ast(\mathscr{L}) \rightarrow H^\ast(\mathscr{M}).$$
    We say $\eta$ is a {\it quasi-isomorphism} if $H^\ast(\eta)$ is an isomorphism.
  \end{enumerate}
\end{defn}

\subsection{The Category up to Homotopy}
\begin{defn}
  Let $\C$ be a small dg category.
  The {\it category up to homotopy of dg $\C$-modules} is defined to be $\mathcal{H}(\C) = H^0(\cdg{\C})$.
  %Note that $\mathcal{H}(\C)$ is a triangulated category.
  %TODO: Is this obviously just the same proof for K(A), where A is an abelian category?
\end{defn}

Given a dg $\C$-module, $\mathscr{M} \colon \C^\text{op} \rightarrow \cdg{k}$, by Remark~\ref{DGInducedHomology} we have a well-defined map
$$Z^0(\mathscr{M}(Y,X)) \colon
\begin{tikzcd}
  Z^0(\C(Y,X)) \arrow{r} & Z^0(\cdg{k}(\mathscr{M}X,\mathscr{M}Y)) = \mathcal{C}(k)(X,Y)
\end{tikzcd}$$
which allows us to view the image of $\mathscr{M}$ in $\mathcal{C}(\C)$ as a presheaf $\mathscr{M} \colon Z^0(\C) \to \mathcal{C}(k)$.

\begin{defn}
  Given a morphism $\eta \in \mathcal{C}(\C)(\mathscr{L}, \mathscr{M})$, we note that for each object $X$ of $\C$, $\eta(X)$ is a morphism of chain complexes.
  Define the dg functor $\cone{\eta} \colon \C^\text{op} \rightarrow \cdg{k}$ as follows.
  For each object $X$ of $\C$ we have the chain complex $\cone{\eta}(X) = \cone{\eta(X)}$ equipped with the differential
  $$d_{\cone{\eta}(X)} = \left(\begin{matrix}
    d_{\mathscr{L}(X)[1]} & 0\\
    \\
    -\eta(X)[1] & d_{\mathscr{M}(X)}
  \end{matrix}\right).$$

  Given a morphism $f \in \C(Y,X)^n$, we define $\cone{\eta}(f)$ by
  $$\begin{tikzcd}[ampersand replacement=\&]
    \cone{\eta}(X) \arrow{rrrr}{\left(\begin{matrix}(-1)^n\mathscr{L}(f)[1] & 0\\0 & \mathscr{M}(f)\end{matrix}\right)} \&\&\&\& \cone{\eta}(Y)[n]
  \end{tikzcd}.$$
  Since both $\mathscr{L}$ and $\mathscr{M}$ are dg functors and $\eta$ is natural, it is straightforward to see that this gives a morphism of complexes
  $$\begin{tikzcd}
    \C(Y,X) \arrow{r} & \mathcal{C}(k)(\cone{\eta(X)}, \cone{\eta(Y)})
  \end{tikzcd}$$
\end{defn}

\begin{rmk}
  \begin{enumerate}
  \item
    Note that if we are given a morphism $f \in Z^0(\C(Y,X))$, then $\mathscr{L}(f)$ and $\mathscr{M}(f)$ are both morphisms of complexes and $\eta$ is natural, so we obtain a morphism of complexes, $\cone{\eta}(f)$,
    $$\begin{tikzcd}[ampersand replacement=\&]
      \cone{\eta}(X) \arrow{rrr}{\left(\begin{matrix}\mathscr{L}(f)[1] & 0\\0 & \mathscr{M}(f)\end{matrix}\right)} \&\&\& \cone{\eta}(Y)
    \end{tikzcd}.$$
  \item
    We also note that we have a short exact sequence of $\C$-modules
    $$\begin{tikzcd}
      0 \arrow{r} & \mathscr{M} \arrow{r}{\left(\begin{matrix}0\\1\end{matrix}\right)} & \cone{\eta} \arrow{r}{\left(-1\ 0\right)} & \mathscr{L}[1] \arrow{r} & 0
    \end{tikzcd}$$
  \end{enumerate}
\end{rmk}

\begin{prop}
  The category $\mathcal{H}(\C)$ with distinguished triangles those triangles that are isomorphic to the image of
  $$\begin{tikzcd}
    \mathscr{L} \arrow{r}{\eta} & \mathscr{M} \arrow{r} & \cone{\eta} \arrow{r} & \mathscr{L}[1] 
  \end{tikzcd}$$
  in $\mathcal{H}(\C)$ is triangulated.

  \begin{proof}
    First, suppose that we are given two morphisms $\eta, \nu \in \mathcal{C}(\C)(\mathscr{L},\mathscr{M})$ having the same image in $\mathcal{H}(\C)(\mathscr{L},\mathscr{M})$.
    By definition, there exists a morphism $h \in \cdg{\C}(\mathscr{L},\mathscr{M})^{-1}$ such that 
    $$\eta - \nu = d_{\cdg{\C}(\mathscr{L},\mathscr{M})}^{-1}(h) = d_\mathscr{M} \circ h + h \circ d_\mathscr{L}.$$
    It's straightforward to show that
    $$\begin{tikzcd}[ampersand replacement=\&]
      \cone{\eta} \arrow{r}{\left(\begin{matrix}1 & 0\\-h & 1\end{matrix}\right)} \& \cone{\nu}
    \end{tikzcd}
    \ \text{and}\ 
    \begin{tikzcd}[ampersand replacement=\&]
      \cone{\nu} \arrow{r}{\left(\begin{matrix}1 & 0\\h & 1\end{matrix}\right)} \& \cone{\eta}
    \end{tikzcd}$$
    give mutually inverse morphisms of $\mathcal{C}(\C)$, so the triangles are well-defined.
    
    For (TR1) we see that given any morphism $\overline{\eta} \in \mathcal{H}(\C)(\mathscr{L},\mathscr{M})$, we can fit $\overline{\eta}$ into a distinguished triangle by lifting to $\mathcal{C}(\C)$ and taking the image of
    $$\begin{tikzcd}
      \mathscr{L} \arrow{r}{\eta} & \mathscr{M} \arrow{r} & \cone{\eta} \arrow{r} & \mathscr{L}[1]
    \end{tikzcd}$$
    in $\mathcal{H}(\C)$.
  \end{proof}
\end{prop}
\section{The Model Structure on $\mathcal{C}(\C)$}


\begin{defn}[\cite{Toen}]
  Let $\C$ be a small dg category and $\eta \colon \funct{L} \to \funct{M}$ a morphism of $\mathcal{C}(\C)$.
  Equipping $\Ch{K}$ with the projective model structure, we say the morphism $\eta$ is a weak-equivalence (resp. fibration) if for each object $X$ of $\C$ the morphism
  $$\eta(X) \in \Ch{k}(\funct{L}X, \funct{M}X)$$
  is a weak-equivalence (resp. fibration).
  This endows $\mathcal{C}(\C)$ with the structure of a cofibrantly generated model category in the sense of \cite{Hovey98}, 2.1.
  %From Keller
  %A dg $\C$-module, $\mathscr{M}$, is {\it cofibrant} if for every epimorphic quasi-isomorphism $\mathscr{L} \rightarrow \mathscr{N}$, every morphism $\mathscr{M} \rightarrow \mathscr{L}$ factors through $\mathscr{L}$,
  %$$\begin{tikzcd}
  %  & \mathscr{M}\arrow{d}\arrow[dashed,swap]{ld}{\exists}\\
  %  \mathscr{L} \arrow{r} & \mathscr{N} \arrow{r} & 0
  %\end{tikzcd}$$
\end{defn}

\begin{rmk}
  \ \\
  \begin{itemize}
  \item
    We note that the representables are all cofibrant by the Yoneda Lemma.
    Namely, given an object $X$ of $\C$, a trivial cofibration (i.e. a level-wise epimorphic quasi-isomorphism) 
    $\eta \in \mathcal{C}(\C)(\funct{L},\funct{M})$,
    and a morphism 
    $\nu \in \mathcal{C}(\C)(h_X, \funct{M})$$ \cong \funct{M}(X)$ 
    we can pull the image of $\nu$ back along $\eta(X)$ to some\\
    $\xi \in \mathcal{C}(\C)(h_X, \funct{L}) \cong \funct{L}(X),$
    giving the desired lift
    $$\begin{tikzcd}
      0 \arrow{r}\arrow{d} & \funct{L}\arrow{d}{\eta}\\
      h_X \arrow{r}{\nu}\arrow[dashed]{ur}{\exists \xi} & \funct{M}
    \end{tikzcd}$$
  \item
    Regarding $\dgcat{k}$ as the 2-category of $\Ch{k}$-enriched categories, it's clear from Proposition~\ref{modulesaretensored} that $\cdg{\C}$ is a tensored and cotensored category, which gives rise to bifunctors
    $$\otimes: \cdg{k} \times \cdg{\C} \to \cdg{\C}\ \text{and}\ \cdg{k}(-,-) \colon \cdg{k}^\text{op} \times \cdg{\C} \to \cdg{\C}$$
    and isomorphisms 
    $$\cdg{k}(C, \cdg{\C}(\F, \G)) 
    \cong \cdg{\C}(C \otimes \F, \G)
    \cong \cdg{\C}(\F, \cdg{k}(C,\G)).$$
    for each object $C$ of $\cdg{k}$, $\Ch{k}$-natural in $\F$ and $\G$.
    This is an adjunction of two variables in the sense of \cite{Hovey98} 4.1.12 and the bifunctor endows $\cdg{\C}$ with the structure of a $\Ch{k}$-module in the sense of \cite{Hovey98}, 4.2.18.
    Applying $Z^0$ gives rise to an adjunction of two variables
    $$\mathcal{C}(k)(C, \cdg{\C}(\F,\G))
    \cong \mathcal{C}(\C)(C \otimes \F, \G)
    \cong \mathcal{C}(\C)(\F, \cdg{k}(C,\G)).$$
    We show that this endows $\mathcal{C}(\C)$ with a $\mathcal{C}(k)$-model structure.
    
    Since $\mathcal{C}(k)$ is a symmetric monoidal model category (see \cite{Hovey98}, 4.2.13) with cofibrant unit $k$, the chain complex consisting of $k$ in degree zero, it suffices to show that the adjunction is Quillen.
    Given cofibrations $f \colon C \to D$ of $\mathcal{C}(k)$ and $\eta \colon \F \to \G$ of $\mathcal{C}(\C)$,
    we have the pushout in $\mathcal{C}(\C)$
    $$\begin{tikzcd}
      C \otimes \F \arrow{rr}{1 \otimes \eta}\arrow{dd}{f \otimes 1} && C \otimes \G \arrow{dd}{u_1}\arrow[bend left]{rdddd}{f \otimes 1}\\
      \\
      D \otimes \F \arrow{rr}{u_2}\arrow[bend right]{rrrdd}{1 \otimes \eta} && D \otimes \F \coprod_{C \otimes \F} C \otimes \G\arrow[dashed]{rdd}{\exists !f \square \eta}\\
      \\
      & & & D \otimes \G
    \end{tikzcd}$$
    and we must show that $f \square \eta$ is a cofibration that is trivial whenever either one of $f$ and $\eta$ is.
    Since $f \square \eta$ is just the data of morphisms $f \square \eta(X)$
    %  $$\begin{tikzcd}
    %    C \otimes \F(X) \arrow{rr}{1 \otimes \eta(X)}\arrow{dd}{f \otimes 1} && C \otimes \G(X) \arrow{dd}{u_1(X)}\arrow[bend left]{rdddd}{f \otimes 1}\\
    %    \\
    %    D \otimes \F(X) \arrow{rr}{u_2(X)}\arrow[bend right]{rrrdd}{1 \otimes \eta(X)} && D \otimes \F(X) \coprod_{C \otimes \F(X)} C \otimes \G(X)\arrow[dashed]{rdd}{\exists !(f \square \eta)(X)}\\
    %    \\
    %    & & & D \otimes \G(X)
    %  \end{tikzcd}$$
    indexed by the objects of $\C$, we reduce to the case that $(f \square \eta)(X)$ is a cofibration that is trivial provided either one $\eta(X)$ or $f$ is.
    By definition of the model structure on $\mathcal{C}(\C$, $\eta(X)$ is a cofibration, so this follows directly from the fact that $\mathcal{C}(k)$ is a symmetric monoidal model category.
    We note that when we restrict to the subcategory of $\cdg{\C}$ with cofibrant and fibrant objects, then by \cite{Hovey98} 4.3.1 this induces a Quillen adjunction 
    $$[C,\mathbb{R}\Hom{\cdg{\C}}{\F,\G}]_{\mathcal{C}(k)}
    \cong [C \otimes^\mathbb{L} \F, \G]_{\mathcal{C}(\C)}
    \cong [\F, \mathbb{R}\Hom{\cdg{k}}{C,\G}]_{\mathcal{C}(\C)},$$
    where $[-,-]$ denotes morphisms of the homotopy category, $\mathbb{R}\op{Hom}$ and $\otimes^\mathbb{L}$ are the right and left derived functors, respectively.
    In particular, denoting by $\mathcal{I}$ the dg category with one object and endomorphisms the complex with $k$ concentrated in degree zero, we note that $\mathcal{C}(\C)$ is obtained from
    \begin{eqnarray*}
      \dgcat{k}(\mathcal{I}, \cdg{\C}) &=& \dgcat{k}(\mathcal{I}, \SHom{\C^\text{op}, \cdg{k}}\\
      &\cong& \dgcat{k}(\mathcal{I} \otimes_k \C^\text{op}, \cdg{k})\\
      &\cong& \dgcat{k}(\C^\text{op}, \cdg{k})\\
      &=& Z^0 \SHom{\C^{op}, \cdg{k}}\\
      &=& Z^0\cdg{\C}\\
      &=:& \mathcal{C}(\C).
    \end{eqnarray*}
    Denoting by $\cdg{\C}_\text{cf}$ the subcategory of cofibrant and fibrant objects, we obtain an equivalence between $H^0\cdg{\C}_\text{cf}$ and $\mathcal{D}(\C) = \op{Ho}(\mathcal{C}(\C))$ by the isomorphism
    $$H^0(\mathbb{R}\Hom{\cdg{C}}{\F,\G}) \cong [k,\mathbb{R}\Hom{\cdg{\C}}{\F,\G}]_{\mathcal{C}(k)} \cong [k \otimes^\mathbb{L} \F, \G]_{\mathcal{C}(\C)} \cong [\F,\G]_{\mathcal{C}(\C)}.$$
  \end{itemize}
\end{rmk}

%From Keller
%\begin{thm}
%  The category $\mathcal{C}(\C)$ admits a projective model structure with weak equivalences quasi-isomorphisms and fibrations epimorphisms.
%  For this structure, each object is fibrant and the cofibrant objects are the cofibrant dg $\C$-modules.
%\end{thm}

\section{The Derived Category of a Differential Graded Category}

\begin{defn}
  Let $\C$ be a small dg category.
  The derived category $\mathcal{D}(\C)$ is the localization of $\mathcal{C}(\C)$ at the class of quasi-isomorphisms, $\op{Ho}(\mathcal{C}(\C))$.
  Equivalently, this is the Verdier quotient of $\mathcal{H}(\C)$ by acyclics.
\end{defn}

\section{Generators of Differential Graded Categories}
\subsection{Pretriangulated Differential Graded Categories}

\begin{defn}
  Let $\C$ be a small dg category.
  We have the Yoneda embedding 
  $$\begin{tikzcd}
    Z^0(\C) \arrow{r}{h_{\_}} & \mathcal{C}(\C)\\
    X \arrow[mapsto]{r} & h_X.
  \end{tikzcd}$$
  We say that $\mathscr{M} \colon \C^\text{op} \rightarrow \cdg{k}$ is {\it quasi-representable} if there is an object $X$ of $\C$ such that $h_X$ is quasi-isomorphic to $\mathscr{M}$
  
  We say that $\C$ is {\it pretriangulated} if
  \begin{enumerate}
  \item
    for each $n \in \Z$ and each object $X$ of $\C$, $h_X[n]$ is quasi-representable, and
  \item
    for all $f \in Z^0(\C)(X,Y)$, $\op{cone}(f_*)$ is quasi-representable, where $f_* : h_X \rightarrow h_Y$ is the image of $f$ under the Yoneda embedding.
  \end{enumerate}
\end{defn}

\begin{defn}
  Let $\C$ be a pretriangulated dg category and let $E$ be an object of $\C$.
  By abuse of notation, 
  \begin{enumerate}
  \item
    We say $E$ is a {\it classical generator} of $\C$ if $h_E$ is a classical generator of $H^0(\C)$,
  \item
    We say $E$ is a {\it strong generator} of $\C$ if $h_E$ is a strong generator of $H^0(\C)$,
  \item
    We say $E$ is a {\it (weak) generator} of $\C$ if $h_E$ is a (weak) generator of $H^0(\C)$, and
  \item
    We say $\C$ is compactly generated if $H^0(\C)$ is.
  \end{enumerate}
\end{defn}

%\begin{lem}[Brown representability]
%  Let $\T$ be a triangulated category with direct sums which is compactly generated.
%  Let $H$ be a cohomological functor on $\T$ which transforms direct sums into direct products.
%  Then $H$ is representable.
%\end{lem}

\begin{defn}
  A dg functor $\mathscr{F} \colon \C \rightarrow \D$ is called a {\it quasi-equivalence} if 
  \begin{enumerate}
  \item
    for all objects $X$ and $Y$ of $\C$, the morphism
    $$\mathscr{F}(X,Y) \colon
    \begin{tikzcd}
      \C(X,Y) \arrow{r} & \D(\mathscr{F}X,\mathscr{F}Y)
    \end{tikzcd}$$
    of $\mathcal{C}(k)$ is a quasi-isomorphism, and
  \item
    the induced functor $H^0(\mathscr{F}) \colon H^0(\C) \rightarrow H^0(\D)$ is an equivalence.
  \end{enumerate}
\end{defn}

\begin{lem}
  Let $k$ be a field, and let $\C$, $\D$ be objects of $\dgcat{k}$.
  Given an object $\mathscr{L}$ of $\cdg{\C}$ (resp. an object $\mathscr{M}$ of $\cdg{\D}$), there is a well-defined dg functor $\mathscr{L} \otimes - \colon \cdg{\D} \to \cdg{\C \otimes \D}$ (resp. $- \otimes \mathscr{M} \colon \cdg{\C} \to \cdg{\C \otimes \D}$).
  

  \begin{proof}
    For an object $\mathscr{M}$ of $\D$, define 
    $$\mathscr{L} \otimes \mathscr{M}(X \otimes Y) = \mathscr{L}(X) \otimes_k \mathscr{M}(X).$$
    Since $\mathscr{L}$ and $\mathscr{M}$ are both dg-functors, we have morphisms of $\mathcal{C}(k)$
    $$\C(X^\prime, X) \to [\mathscr{L}(X), \mathscr{L}(X^\prime)]\ \text{and}\ \D(Y^\prime, Y) \to [\mathscr{M}(Y), \mathscr{M}(Y^\prime)],$$
    which correspond under the adjunction of Proposition~\ref{ChHomTensorAdjunction} to morphisms
    $$\C(X^\prime, X) \otimes_k \mathscr{L}(X) \to \mathscr{L}(X^\prime)
    \ \text{and}\ 
    \D(Y^\prime, Y) \otimes_k \mathscr{M}(Y) \to \mathscr{M}(Y^\prime),$$
    respectively.
    Tensoring the left-hand morphism by $\D(Y^\prime,Y) \otimes_k \mathscr{M}(Y)$ and the right hand-morphism by $\mathscr{L^\prime}(X^\prime)$ we obtain a morphism
    $$\C(X^\prime, X) \otimes_k \D(Y^\prime, Y) \otimes_k \mathscr{L}(X) \otimes_k \mathscr{M}(Y) \to 
    \mathscr{L}(X) \otimes_k \D(Y^\prime,Y) \otimes_k \mathscr{M}(Y) \to
    \mathscr{L}(X^\prime) \otimes_k \mathscr{M}(Y^\prime)$$
    which corresponds under the adjunction to a morphism
    $$\C \otimes \D(X^\prime \otimes Y^\prime, X \otimes Y) \to
    [\mathscr{L}(X) \otimes_k \mathscr{M}(Y), \mathscr{L}(X^\prime) \otimes_k \mathscr{M}(Y^\prime)].$$
    Using the adjunction of Proposition~\ref{ChHomTensorAdjunction} we obtain the morphism
    $$\begin{tikzcd}
      \Ch{k}(\cdg{\C}, [\funct{L}(X), \funct{L}(X^\prime)]) \arrow{d}[rotate=90,xshift=-1ex,yshift=1ex]{\sim}\\
      \Ch{k}(\cdg{\C} \otimes_k \funct{L}(X), \funct{L}(X^\prime))\arrow{d}{- \otimes_k \left(\D(Y^\prime, Y) \otimes_k \funct{M}(Y)\right)}\\
      \Ch{k}(\cdg{\C \otimes \D}(X^\prime \otimes Y^\prime, X \otimes Y) \otimes_k \funct{L}(X) \otimes \funct{M}(Y), \funct{L}(X^\prime) \otimes_k \funct{M}(Y^\prime) \otimes_k \D(Y^\prime, Y))
      \arrow{d}{\funct{L}(X^\prime) \otimes_k \D(Y^\prime,Y) \otimes_k M(Y) \to \funct{L}(X^\prime) \otimes_k \funct{M}(Y^\prime) \circ -}\\
      \Ch{k}(\cdg{\C \otimes \D}(X^\prime \otimes Y^\prime, X \otimes Y) \otimes_k \funct{L}(X) \otimes_k \funct{M}(Y), \funct{L}(X^\prime) \otimes_k \funct{M}(Y^\prime))\arrow{d}[rotate=90,xshift=-1ex,yshift=1ex]{\sim}\\
      \Ch{k}(\cdg{\C \otimes \D}(X^\prime \otimes Y^\prime), [\funct{L}(X) \otimes_k \funct{M}(X), \funct{L}(X^\prime) \otimes_k \funct{M}(Y^\prime)])
    \end{tikzcd}$$
  \end{proof}
\end{lem}

\begin{lem}\label{TriangulatedFunctorsPreserveGens}
  If $\F \colon \T^\prime \to \T$ is a triangulated functor, then for any object $X$ of $T^\prime$, 
  $$\F\left(\langle X \rangle_n\right) \subseteq \langle \F(X) \rangle_n.$$

  \begin{proof}
    We proceed by induction on $n$.
    When $n = 1$, for any object $Y$ of $\langle X \rangle_1$, there exists, by definition, an isomorphism $X \cong Y \oplus Z$ for some object $Z$ of $\T^\prime$.
    Since $\F$ is additive we have
    $$\F(Y) \oplus \F(Z) \cong \F(Y \oplus Z) \cong F(X)$$
    and hence $\F(Y)$ is an object of $\langle \F(X) \rangle_1$.

    Assume the result holds up to $n$.
    Given an object $Y$ of $\langle X \rangle_n$, choose a distinguished triangle
    $$A \to X^\prime \to B \to A[1]$$
    with $A$ an object of $\langle X \rangle_1$, $B$ an object of $\langle X \rangle_{n-1}$, and $X^\prime \cong Y \oplus Z$ for some object $Z$ of $\T^\prime$.
    Since $\F$ is triangulated we have a distinguished triangle
    $$\F(A) \to \F(X^\prime) \to \F(B) \to \F(A)[1]$$
    of $\T$ with $\F(A)$ an object of $\langle \F(X) \rangle_1$ and $\F(B)$ an object of $\langle \F(X) \rangle_{n-1}$ by induction.
    Since $\F$ is additive we have
    $$\F(Y) \oplus \F(Z) \cong \F(Y \oplus Z) \cong \F(X^\prime)$$
    and hence $\F(Y)$ is an object of $\langle \F(X) \rangle_n$.
    Therefore $\F\left(\langle X \rangle_n\right) \subseteq \langle \F(X) \rangle_n$.
  \end{proof}
\end{lem}

%TODO: Make the statement more precise.
%Fit it to context.
\begin{lem}\label{AdditiveFunctorsCommuteWithCones}
  If $\F \colon \C \rightarrow \D$ is an additive functor, then for any morphism, $f$, of complexes of $\C$ we have
  $\F\left(\cone{f}\right) \cong \cone{\F(f)}$.
\end{lem}

\begin{lem}\label{RepresentablesGenerateSmallDG}
  If $\C$ is a small dg-category, then $\mathcal{D}(\C)$ is compactly generated.

  \begin{proof}
    We view $\mathcal{D}(\C)$ as the Verdier quotient of $\mathcal{H}(\C)$ by level-wise acyclic modules.
    Let $H = \oplus_{X \in \ob{\C}} h_X$.
    Given an object $\mathcal{M}$ of $\mathcal{D}(\C)$ such that for all objects $X$ of $\C$ and all $n$
    $$0 \cong \mathcal{D}(\C)(H, \mathcal{M}[n]) \cong \prod_{X \in \ob{\C}} H^n\mathcal{M}(X)$$
    then $\mathcal{M}$ is level-wise acyclic, and hence $\mathcal{M} \cong 0$ in $\mathcal{D}(\C)$, as desired.
    Therefore $\mathcal{D}(\C)$ is compactly generated by the representables.
  \end{proof}
\end{lem}

\begin{prop}
  Let $\A$ and $\B$ be dg-categories with $H^0(\A)$, $H^0(\B)$ compactly generated triangulated categories with coproducts, generated by $\{E_i\}_I$, $\{F_j\}_J$, respectively, and let $\F \colon \A \to\B$ a dg-functor such that the induced functor $H^0(\F) \colon H^0(\A) \to H^0(\B)$ commutes with coproducts.
  If $H^0(\F)$ satisfies
  \begin{enumerate}
  \item
    for each $i \in I$ there exists an isomorphism $\alpha_{i,j} \colon H^0(\F)(E_i) \cong F_j$ for some $j \in J$,
  \item
    For each $j \in J$ there exists $i \in I$ such that $F_j \cong H^0(\F)(E_i)$,
  \item
    for each pair $(i_1, i_2) \in I \times I$,
    $$H^0(\F)(E_{i_1}, E_{i_2}) \colon H^0(\A)(E_{i_1}, E_{i_2}) \cong H^0(B)(\F(E_{i_1}), \F(E_{i_2}))$$
  \end{enumerate}
  then $F$ is a quasi-equivalence.

  \begin{proof}
    Replace $\{E_i\}_I$ by $\cup_{n \in \Z}\{E_i[n]\}$ and $\{F_j\}_J$ by $\cup_{m \in \Z}\{F_j[m]\}$, then take $E = \oplus_I E_i$, $F = \oplus_J F_j$, and apply Corollary~\ref{fullyfaithfulongensisequivalence}.
  \end{proof}
\end{prop}

\begin{cor}
  Let $\C$ be as above and $H$ be as above.
  The object $H$ clasically generates the compact subcategory, $\Perf{\C}$, of $D(\C)$,
  $$\Perf{\C} = \langle H \rangle.$$

  \begin{proof}
    Apply Proposition~\ref{GeneratorIFFClassicAndCompactlyGenerated}.
  \end{proof}
\end{cor}

\begin{thm}
  Let $k$ be a field, and let $\C$ and $\mathscr{D}$ be pretriangulated dg-categories.
  %If $E_i$ is a family of generators for $\C$ and $F_i$ is a family of generators for $\mathscr{D}$, then the $E_i \otimes F_j$ are a family of generators for $\mathcal{H}\left(\C \otimes \mathscr{D}\right)$.
  If $G$ is a classical generator for $\C$ and $H$ is a classical generator for $\mathscr{D}$, then $h_G \otimes h_H$ is a generator for $\mathcal{D}\left(\C \otimes \mathscr{D}\right)$.
  
  \begin{proof}
    First we note that the images of the representables are all compact, and by Lemma~\ref{RepresentablesCompactlyGenerateSmallDG} they generate $\mathcal{D}(\C \otimes \D)$, hence also $\mathcal{D}(\C \otimes \D)_c$.
    It then follows from Lemma~\ref{GeneratorIFFClassicAndCompactlyGenerated} that the representables clasically generate $\mathcal{D}(\C \otimes \D)_c$.
    By another application of Lemma~\ref{GeneratorIFFClassicAndCompactlyGenerated}, it suffices to show that $h_G \otimes h_H$ classically generates the representables, for then we have that $\mathcal{D}(\C \otimes \D)$ is compactly generated and 
    $$\langle h_G \otimes h_H \rangle = \mathcal{D}(\C \otimes \D)_c$$
    from which it follows that $h_G \otimes h_H$ generates $\mathcal{D}(\C \otimes \D)$.
    
    We observe that because $\C$ and $\D$ are pre-triangulated categories, the Yoneda functors
    $$H^0(\C) \to \mathcal{H}(\C)\ \text{and}\ H^0(\D) \to \mathcal{H}(\D)$$
    are both triangulated.
    Identifying $H^0(\C)$ and $H^0(\D)$ with their images we have
    $$H^0(\C) \subseteq \langle h_G \rangle\ \text{and}\ H^0(\D) \subseteq \langle h_H \rangle$$
    by Lemma~\ref{TriangulatedFunctorsPreserveGens}.
    
    Fix an object $X$ of $\C$ and view $h_X$ as an object of $\mathcal{D}(\C)$.
    Consider the additive functor $h_X \otimes -: \mathcal{D}(\D) \to \mathcal{D}\left(\C \otimes \mathscr{D}\right)$. %TODO: Is this really additive?  Does it make sense?
    It follows from Lemma~\ref{AdditiveFunctorsCommuteWithCones} that $\F$ is triangulated and hence 
    $$h_X \otimes \langle h_H \rangle \subseteq \langle h_X \otimes h_H \rangle$$
    by Lemma~\ref{TriangulatedFunctorsPreserveGens}.
    For any object $Y$ of $\D$, there exists an $n$ for which $h_Y \in \langle h_H \rangle_n$, whence
    $$h_X \otimes h_Y \in h_X \otimes \langle h_H\rangle_n \subseteq \langle h_X \otimes h_H\rangle_n.$$
    There exists some $m$ for which $h_X \in \langle h_G \rangle_m$ and thus
    $$h_X \otimes h_H \in \langle h_G \rangle_m \otimes h_H \subseteq \langle h_G \otimes h_H \rangle_m$$
    %$$\langle h_X \otimes h_H \rangle \subseteq \langle h_G \otimes h_H\rangle_m$$
    follows from a symmetric argument with $- \otimes h_G$.
    Therefore for any object $X$ of $\C$ and any object $Y$ of $\D$,
    $$h_X \otimes h_Y \in \langle h_X \otimes h_H \rangle \subseteq \langle h_G \otimes h_H\rangle$$
    follows by minimality.
  \end{proof}
\end{thm}

\end{document}
