\documentclass[10pt,draft]{book}
\usepackage{mystyle}
\usepackage{style}

\openup 5pt
\author{Blake Farman\\University of South Carolina}
\title{
  {Geometry of Derived Categories on Noncommutative Projective Schemes}\\
  {University of South Carolina}
}
\date{October 3, 2016}
\pdfpagewidth 8.5in
\pdfpageheight 11in


\begin{document}
\maketitle

\tableofcontents

\newtheorem{thm}{Theorem}[section]
\newtheorem{lem}[thm]{Lemma}
\newtheorem{cor}[thm]{Corollary}
\newtheorem{prop}[thm]{Proposition}

\theoremstyle{definition}
\newtheorem{defn}[thm]{Definition}
\newtheorem{rmk}[thm]{Remark}
\newtheorem{eg}[thm]{Example}

\chapter{Introduction}
\chapter{Preliminaries}
\section{Categories}
%\subfile{categories.tex}
%TODO: How should categories be defined?  What sort of set theoretic precautions need be taken here...
%\subsection{The Yoneda Embedding}
%\subfile{yoneda.tex} %Is this necessary?  Trim down?
%\subsection{Fiber Products in $\hat{\mathscr{C}}$} %Maybe necessary?
%\subfile{fiberproducts.tex}
%\section{Enriched Categories}
%\subfile{monoidalcats.tex}
%\subfile{enrichedcats.tex}
%\subfile{tensorcats.tex}
%\section{Some Homological Algebra}
%\subfile{homalg.tex}
%\section{Triangulated Categories}
%\subfile{triangulatedcats.tex}
%\subfile{verdier.tex}
%\subsection{Generators of Triangulated Categories}
%\subfile{triangulatedgens.tex}
%\chapter{Differential Graded Categories}
%\subfile{dgcats.tex}
%\section{Graded Algebra}
%\subfile{gradedalgebra.tex}
\chapter{Non Commutative Projective Schemes}
\subfile{non.com.geom.tex}
%\chapter{Fourier-Mukai Kernels for Noncommutative Projective Schemes}
%\subfile{KNCP.tex}

%\bibliographystyle{amsalpha}
%\bibliography{biblio}
\end{document}

