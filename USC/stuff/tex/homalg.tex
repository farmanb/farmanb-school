\documentclass[dissertation.tex]{subfiles}
\begin{document}
Let $A$ be a commutative ring.

\subsection{Tensor Products of Chain Complexes}
Given two complexes, 
$$C : \cdots \to C^{n-1} \overset{d^{n-1}_C}\to C^n \overset{d^n_C}\to C^{n+1} \to \cdots$$
and 
$$D : \cdots \to D^{n-1} \overset{d^{n-1}_D}\to D^n \overset{d^n_D}\to D^{n+1} \to \cdots,$$
of $A$-modules,
consider the double complex
$$\begin{tikzcd}
  & & \vdots & \vdots & \vdots & \\
  C \otimes_A D^{y + 1}: & \cdots \arrow{r}& C^{x - 1} \otimes_A D^{y + 1}\arrow{u}\arrow{r} & C^x \otimes_A D^{y + 1}\arrow{u}\arrow{r} & C^{x + 1} \otimes_A D^{y + 1}\arrow{u}\arrow{r} & \cdots\\
  C \otimes_A D^{y} & \cdots \arrow{r} & C^{x - 1} \otimes_A D^{y}\arrow{u}\arrow{r} & C^x \otimes_A D^{y}\arrow{u}\arrow{r} & C^{x + 1} \otimes_A D^{y}\arrow{u}\arrow{r} & \cdots\\
  C \otimes_A D^{y - 1} & \cdots \arrow{r} & C^{x - 1} \otimes_A D^{y - 1}\arrow{u}\arrow{r} & C^x \otimes_A D^{y - 1}\arrow{u}\arrow{r} & C^{x + 1} \otimes_A D^{y - 1}\arrow{u}\arrow{r} & \cdots\\
  & & \vdots\arrow{u} & \vdots\arrow{u} & \vdots\arrow{u} & \\
  & & C^{x-1} \otimes_A D & C^x \otimes_A D & C^{x + 1} \otimes_A D
\end{tikzcd}$$
with horizontal differentials
$$\begin{tikzcd}
  C^x \times D^y \arrow{r}{\otimes}\arrow{d}{d^x_C \times D^y} & C^x \otimes_A D^y \arrow[dashed]{d}{\exists ! d^x_C \otimes_A D^y}\\
  C^{x + 1} \times D^y \arrow{r}{\otimes} & C^{x + 1}\otimes_A D^y
\end{tikzcd}$$
and vertical differentials
$$\begin{tikzcd}
  C^x \times D^y \arrow{r}{\otimes}\arrow{d}{(-1)^x \times d_D^y} & C^x \otimes_A D^y \arrow[dashed]{d}{\exists ! (-1)^x \otimes_A d_D^y}\\
  C^{x} \times D^{y+1} \arrow{r}{\otimes} & C^{x + 1}\otimes_A D^{y + 1}
\end{tikzcd}$$
induced by the universal property for the tensor product in $\operatorname{Mod}-{A}$.
For fixed $n$ the differentials
$$\begin{tikzcd}
  & \ddots\\
  & C^{x - 1} \otimes_A D^{y + 1} \arrow{r} & C^x \otimes_A D^{y + 1}\\
  & & C^x \otimes_A D^{y} \arrow{r} & C^{x + 1} \otimes_A D^{y}\\
  & & & C^{x + 1} \otimes_A D^{y - 1} \arrow{r} & C^{x + 2} \otimes D^{y - 1}\\
  & & & & \ddots &
\end{tikzcd}$$
and
$$\begin{tikzcd}
  \ddots
  & C^{x - 1} \otimes_A D^{y + 2}\\
  & C^{x - 1} \otimes_A D^{y + 1} \arrow{u}& C^x \otimes_A D^{y + 1}\\
  & & C^x \otimes_A D^y \arrow{u} & C^{x + 1} \otimes_A D^y\\
  & & & C^{x + 1} \otimes_A D^{y - 1}\arrow{u} & \ddots\\
\end{tikzcd}$$
induce morphisms
$$d^n_c \otimes_A 1 \colon \bigoplus_{x + y = n} C^x \otimes_A D^y \to \bigoplus_{x + y = n} C^{x+1} \otimes_A D^{y}$$
and
$$(-1)^x \otimes_A d^n_D \colon \bigoplus_{x + y = n} C^x \otimes_A D^y \to \bigoplus{x + y = n} C^{x} \otimes_A D^{y + 1}.$$
\begin{defn}
  For two chain complexes $C$ and $D$, define the tensor product of chain complexes to be the object
  $$\left(C \otimes_A D\right)^n = \bigoplus_{x + y = n}C^x \otimes_A D^y$$
  equipped with the differential
  $$d^n_C \otimes_A 1 + (-1)^x \otimes_A d^n_D.$$
\end{defn}

Fix a morphism of chain complexes $f \colon C \to D$ and a chain complex $E$.
For each $x$ and $y$, there is an induced morphism
$$\begin{tikzcd}
  C^x \times E^y \arrow{r}\arrow{d}{f^x \times 1} & C^x \otimes_A E^y \arrow[dashed]{d}{\exists ! f^x \otimes E^y}\\
  D^x \times E^y \arrow{r} & D^x \otimes_A E^y
\end{tikzcd}$$
and hence a unique morphism
$$\left(f \otimes_A E\right)^n \colon \left(C \otimes_A E\right)^n \to \left(D \otimes_A E\right)^n$$
defined by the collection of commutative diagrams
$$\begin{tikzcd}
  C^x \otimes_A E^y \arrow{r}\arrow{d}{f^x \otimes_A E^y} & \left(C \otimes_A E\right)^n \arrow{d}{(f \otimes_A E)^n}\\
  D^x \otimes_A E^y \arrow{r} & \left(D^x \otimes_A E^y\right)^n
\end{tikzcd}$$
indexed by $x + y = n$.

\begin{lem}
  The morphism $f \otimes_A E = \left(\left(f \otimes_A E\right)^n\right)$ is a morphism of complexes.
  
  \begin{proof}
    Since the differential on the tensor product is the sum of the horiztonal and vertical differentials, it suffices to check that $f \otimes_A E$ commutes with each separately.
    %For let $d^n_{h,C \otimes_A E}$, $d^n_{v,C \otimes_A E}$, $d^n_{h, D \otimes_A E}$, and $d^n_{v, D \otimes_A E}$ be the horizontal and vertical differentials, then
    %\begin{eqnarray*}
    %  \left(f \otimes_A E\right)^{n+1} \circ d_{C \otimes_A E}^n - 
    %  d_{D \otimes_A E}^n \circ \left(f \otimes_A E\right)^n &=& 
    %  \left(f \otimes_A E\right)^{n+1} \circ d^n_{h, C \otimes_A E} -
    %  d^n_{h,D \otimes_A E} \circ \left(f \otimes_A E\right)^{n}\\
    %  &+&
    %  \left(f \otimes_A E\right)^{n+1} \circ d^n_{v, C \otimes_A E}
    %  -
    %  d^n_{v,D \otimes_A E} \circ \left(f \otimes_A E\right)^{n}\\
    %\end{eqnarray*}
    We exhibit the horizontal differential and then observe that the argument for the vertical differential is a simpler version of the same argument, mutatis mutandis.
    
    First note that the diagram
    $$\begin{tikzcd}
      C^x \times E^y \arrow{rr}{d^x_c \times E^y} \arrow{d}{f^x \times E^y} &&C^{x + 1} \times E^y\arrow{d}{f^{x + 1} \times E^y}\\
      D^x \times E^y \arrow{rr}{d_D^x \times E^y} && D^{x+1} \times E^y
    \end{tikzcd}$$
    commutes because each composition makes both diagrams
    $$\begin{tikzcd}
      C^x \arrow{rd}\arrow{dd}{f^x}\arrow[bend right,swap]{dddd}{f^{x+1} \circ d^x_C} & & E^y \arrow{ld}\arrow{dd}{1} & & C^x \arrow{rd}\arrow{dd}{d_C^x}\arrow[bend right,swap]{dddd}{d_D^x \circ f^x} & & E^y \arrow{ld}\arrow{dd}{1}\\
      & C^x \times E^y\arrow[swap]{dd}{f^x \times E^y}\arrow[dashed,bend left]{dddd}{\exists!} & & & & C^x \times E^y \arrow[swap]{dd}{d_C^x \times E^y}\arrow[dashed,bend left]{dddd}{\exists !}\\
      D^x \arrow{rd}\arrow{dd}{d_D^x} & & E^y \arrow{ld}\arrow{dd}{1} & & C^{x+ 1} \arrow{rd}\arrow{dd}{f^{x+1}}& & E^y\arrow{ld}\arrow{dd}{1}\\
      & D^x \times E^y\arrow[swap]{dd}{d_D^x \times E^y} & & & & C^{x+1} \times E^y\arrow[swap]{dd}{f^{x+1} \times E^y}\\
      D^{x+1} \arrow{rd} & & E^y\arrow{ld} & & D^x \arrow{rd} & & E^y \arrow{ld}\\
      & D^{x+1} \times E^y & & & & D^{x+1} \times E^y
    \end{tikzcd}$$
    commute, hence they must be the same by unicity.
    Since $(C \otimes_A E)^n$ is a coproduct, the moprhism
    $$h = (f \otimes_A E)^{n+1} \circ d^n_{h,C \otimes_A E} - d^n_{h,D \otimes_A E} \circ (f \otimes_A E)^n$$
    is determined by the diagrams
    \begin{equation}\label{triangle}
      \begin{tikzcd}
        C^x \times E^y \arrow{r}\arrow{rd} & (C \otimes_A E)^n \arrow{d}{h}\\
        & (D \otimes_A E)^{n+1}
      \end{tikzcd}
    \end{equation}
    indexed by $x + y = n$, with diagonal morphism the composition.
    For all $x + y = n$, all faces other than the front face of the cube with diagonal morphisms the canonical inclusions into the coproducts
    $$\begin{tikzcd}
      C^x \times E^y \arrow{rd}\arrow{rr}{d^x_C \times E^y}\arrow{dd}{f^x \times E^y} & & C^{x+1} \times E^y\arrow[pos=0.75,dashed]{dd}{f^{x + 1} \times E^y}\arrow{rd}\\
      & (C \otimes_A E)^n \arrow[pos=0.15]{rr}{d^n_{h,C \otimes_A E}}\arrow[pos=0.75]{dd}{(f \otimes_A E)^n} & & (C \otimes E)^{n+1}\arrow{dd}{(f \otimes_A E)^{n+1}}\\
      D^x \times E^y \arrow[pos=0.75,dashed]{rr}{d^x_D \times E^y}\arrow{rd}& & D^{x + 1} \times E^y\arrow[dashed]{rd}\\
      & (D \otimes_A E)^n \arrow{rr}{d^n_{h,D \otimes_A E}} & & (D \otimes_A E)^{n+1}
    \end{tikzcd}$$
    commute.
    Hence we deduce that the diagonal morphism
    $$C^x \times E^y \to (D \otimes_A E)^{n+1}$$
    in \ref{triangle}
    is the zero morphism and
    $$(f \otimes_A E)^{n+1} \circ d^n_{h,C \otimes_A E} = d^n_{h,D \otimes_A E} \circ (f \otimes_A E)^n$$
    holds by unicity.
    Therefore $f \otimes_A E$ is a morphism of complexes, as desired.
  \end{proof}
\end{lem}

An immediate consequence of the Lemma is
\begin{cor}
  For a fixed chain complex $E$, we have functors
  $$- \otimes_A E, E \otimes_A - \colon \Ch{A} \to \Ch{A}$$
  that take a chain complex, $C$, to $C \otimes_A E$ (resp. $E \otimes_A C$) and a morphism $f \in \Ch{A}(C,D)$ to $f \otimes_A E$ (resp. $E \otimes_A f$).
\end{cor}

\subsection{Hom Total Complex}
Given two complexes, $C$ and $D$, form the double complex
$$\begin{tikzcd}
  & & \vdots & \vdots & \vdots\\
  h_{D^{y+1}}\left(C\right) & \cdots \arrow{r} & h_{D^{y+1}}\left(C^{x+1}\right)\arrow{u}\arrow{r} & h_{D^{y+1}}\left(C^x\right)\arrow{u}\arrow{r} & h_{D^{y+1}}\left(C^{x-1}\right) \arrow{u}\arrow{r} & \cdots\\
  h_{D^{y}}\left(C\right) & \cdots \arrow{r} & h_{D^{y}}\left(C^{x+1}\right)\arrow{u}\arrow{r} & h_{D^{y}}\left(C^x\right)\arrow{u}\arrow{r} & h_{D^{y}}\left(C^{x-1}\right) \arrow{u}\arrow{r} & \cdots\\
  h_{D^{y-1}}\left(C\right) & \cdots \arrow{r} & h_{D^{y-1}}\left(C^{x+1}\right)\arrow{u}\arrow{r} & h_{D^{y-1}}\left(C^x\right)\arrow{u}\arrow{r} & h_{D^{y-1}}\left(C^{x-1}\right) \arrow{u}\arrow{r} & \cdots\\
  & & \vdots\arrow{u} & \vdots\arrow{u} & \vdots\arrow{u}\\
  & & h^{C^{x+1}}\left(D\right) & h^{C^x}\left(D\right) & h^{C^{x-1}}\left(D\right)
\end{tikzcd}$$
with horizontal morphisms $h_{D^{y}}\left(d_C^x\right)$ and vertical morphisms $(-1)^{y+x+1}h^{C^x}\left(d_D^y\right)$.
\begin{defn}\label{ChInternalHomDefn}
  Define 
  $$[C,D]^n = \prod_{y - x = n} \Hom{A}{C^x,D^y} = \prod_{x \in \Z} \Hom{A}{C^x,D^{x+n}}$$
  with differential the sum of the induced morphisms
  $$\begin{tikzcd}
    \left[C,D\right]^n = \prod_{x \in \Z} \Hom{A}{C^x,D^{x + n}} \arrow{rrr}{\prod_{x \in \Z} h_{D^{x + n}}\left(d_C^{x - 1}\right)} &&& \prod_{x \in \Z} \Hom{A}{C^{x-1},D^{x + n}} = \left[C,D\right]^{n+1}
  \end{tikzcd}$$
  and
  $$\begin{tikzcd}
    \left[C,D\right]^n = \prod_{x \in \Z} \Hom{A}{C^x,D^{x + n}} \arrow{rrr}{\prod_{x \in \Z} (-1)^{n + 1}h^{C^{x}}\left(d_D^{x + n}\right)} &&& \prod_{x \in \Z} \Hom{A}{C^{x-1},D^{x + n + 1}} = \left[C,D\right]^{n+1}
  \end{tikzcd}$$
\end{defn}

Fix a morphism of complexes $f : C \to D$ and a complex $E$.
For each $x$ and $n$, there is a morphism
$$h^{E^x}(f^{x+n}) \colon \Hom{A}{E^x,C^{x + n}} \to \Hom{A}{E^x,D^{x+n}}.$$
Fixing $n$, this collection of morphisms indexed by $x \in \Z$ induces a unique morphism\\ $[E,f]^n \colon [E,C]^n \to [E,D]^n$ determined by the commutative diagrams 
$$\begin{tikzcd}
  \left[E,C\right]^n \arrow{r}\arrow{d}{[E,f]^n} & \Hom{A}{E^x,C^{x+n}}\arrow{d}{h^{E^x}\left(f^{x+n}\right)}\\
  \left[E,D\right]^n \arrow{r} & \Hom{A}{E^x,D^{x+n}}
\end{tikzcd}$$
with horizontal morphisms the canonical projections from the product.

\begin{lem}
  The morphism $[E,f] = ([E,f]^n) \colon [E,C] \to [E,D]$ defines a morphism of complexes.

  \begin{proof}
    As in the case of the tensor product, it suffices to show that $[E,f]^n$ commutes with the horizontal and vertical differentials independently.
    Here we exhibit the vertical case and note that the horizontal case follows from a simpler version of the same argument, mutatis mutandis.

    Fix an integer $n$.
    For the vertical morphisms, we observe first that the diagram
    $$\begin{tikzcd}
      \Hom{A}{E^x,C^{x+n}} \arrow{d}{h^{E^x}\left(f^{x+n}\right)}\arrow{rrr}{(-1)^{n+1}h^{E^x}\left(d^{x+n}_C\right)} &&& \Hom{A}{E^x,C^{x+n+1}}\arrow{d}{h^{E^x}\left(f^{x+n+1}\right)}\\
      \Hom{A}{E^x,D^{x+n}} \arrow{rrr}{(-1)^{n+1}h^{E^x}\left(d^{x+n}_D\right)} &&& \Hom{A}{E^x,D^{x+n+1}}
    \end{tikzcd}$$
    commutes since
    \begin{eqnarray*}
      (-1)^{n+1} h^{E^x}(f^{x+n+1}) \circ h^{E^x}(d^{x+n}_C) &=& (-1)^{n+1}h^{E^x}(f^{x+n+1} \circ d^{x+n}_C)\\
      &=& (-1)^{n+1}h^{E^x}(d^{x+n}_D \circ f^{x+n})\\
      &=& (-1)^{n+1} h^{E^x}(d^{x+n}_D) \circ h^{E^x}(f^{x+n}).
    \end{eqnarray*}
    Since $[E,D]$ is a product, we note that the morphism
    $$h = [E,f]^{n+1} \circ \prod_{x \in \Z} h^{E^{x}}(d^{x+n}_C) - \prod_{x \in \Z} h^{E^{x}}(d^{x+n}_D) \circ [E,f]^n \colon [E,C]^n \to [E,D]^{n+1}$$
    is determined by the diagrams
    \begin{equation}\label{homtriangle}
      \begin{tikzcd}
        \left[E,C\right]^n \arrow{r}{h}\arrow{rd} & \left[E,D\right]^{n+1}\arrow{d}\\
        & \Hom{A}{E^x,D^{x + n+1}}
      \end{tikzcd}
    \end{equation}
    indexed by $x$, with the vertical morphism the canonical projection.
    Using the square above, we obtain for all $x$ a cube with diagonal morphisms the canonical projections
    $$\begin{tikzcd}
      \left[E,C\right]^n \arrow{rd}\arrow{rr}{\prod_{x \in \Z} (-1)^{n+1}h^{E^x}(d^{x+n}_C)}\arrow{dd}{[E,f]^n} & & \left[E,C\right]^{n+1} \arrow{rd}\arrow[dashed,pos=0.75]{dd}{[E,f]^{n+1}}\\
      & \Hom{A}{E^x,C^{x+n}} \arrow[pos=0.4]{rr}{(-1)^{n+1}h^{E^x}(d^{x+n}_C)}\arrow[pos=0.75]{dd}{h^{E^x}(f^{x+n})} & & \Hom{A}{E^x,C^{x+n+1}} \arrow{dd}{h^{E^x}(f^{x+n})}\\
      \left[E,D\right]^n \arrow{rd}\arrow[pos=0.46,dashed]{rr}{\prod_{x \in \Z} (-1)^{n+1}h^{E^x}(d^{x+n}_D)} & & \left[E,D\right]^{n+1}\arrow[dashed]{rd}\\
      & \Hom{A}{E^x,D^{x+n}} \arrow{rr}{(-1)^{n+1}h^{E^x}(d^{x+n}_D)} & & \Hom{A}{E^x,D^{x+n+1}}
    \end{tikzcd}$$
    with all faces except the back face commutative.
    By chasing through the cube, we deduce that the diagonal arrow
    $$[E,C]^n \to \Hom{A}{E^x,D^{x+n+1}}$$
    of \ref{homtriangle} must be zero.
    Hence by unicity we have
    $$[E,f]^{n+1} \circ \prod_{x \in \Z} h_{C^{x+n}}(d^{x-1}(E)) = \prod_{x \in \Z} h_{D^{x+n}}(d^{x-1}(E)) \circ [E,f]^n.$$
    Therefore $[E,f]$ is a morphism of complexes, as desired.
  \end{proof}
\end{lem}

As an immediate consequence of the Lemma, we have

\begin{cor}
  For a fixed chain complex $E$, we have functors
  $$[E,-], [-,E] \colon \Ch{A} \to \Ch{A}$$
  that take a chain complex, $C$, to $[E,C]$ (resp. $[C,E]$) and a morphism $f \in \Ch{A}(C,D)$ to $[E,f]$ (resp. $[f,E] \colon [D,E] \to [C,E]$).
\end{cor}

\begin{prop}\label{ChHomTensorAdjunction}
  The functor $[E,-]$ is right adjoint to the functor $E \otimes_A -$.
\end{prop}

\begin{cor}
  The category $\Ch{A}$ is a complete symmetric monoidal closed category.
  %TODO: Maybe some justification for complete?
\end{cor}

\begin{rmk}\label{MorphismsOfDGModules}
  \begin{enumerate}
    The adjunction above gives rise to the following morphisms in $\Ch{A}$:
  \item
    For any pair of chain complexes, $C$ and $D$, an evaluation morphism
    $$\begin{tikzcd}
      \Ch{A}\left([C,D],[C,D]\right) \arrow{r}{\sim} & \Ch{A}\left([C,D] \otimes_A C, D\right)\\
      \id_{[C,D]} \arrow[mapsto]{r} & \operatorname{ev}_{C,D}
    \end{tikzcd}$$
  \item\label{MorphismsOfDGModules.composition}
    For any three chain complexes, $C$, $D$, and $E$, an associative composition 
    $$\begin{tikzcd}
      \left[C,D\right] \otimes_A \left[D,E\right] \arrow{r} & \left[C,E\right]
    \end{tikzcd}$$
    given by the image of the morphism
    $$\begin{tikzcd}
      \left[C,D\right] \otimes_A \left[D,E\right] \otimes_A C \arrow{r}{\sim} &
      \left[C,D\right] \otimes_A C \otimes_A \left[D,E\right] \arrow[swap]{ldd}{\operatorname{ev}_{C,D} \otimes_A [D,E]}\\
      \\
      D \otimes_A \left[D,E\right] \arrow{r}{\sim} & 
      \left[D,E\right] \otimes_A D \arrow{r}{\operatorname{ev}_{D,E}} &
      E
    \end{tikzcd}$$
    under the adjunction
    $$\Ch{A}\left([C,D] \otimes_A [D,E] \otimes_A C, E\right) \cong 
    \Ch{A}\left([C,D] \otimes [D,E], [C,E]\right)$$
  \item
    Regarding $A$ as the complex with $A$ in degree 0 and 0 elsewhere, for every chain complex, $C$, a unit morphism $A \to [C,C]$
    $$\begin{tikzcd}
      \Ch{A}(C,C) \arrow{r}{\sim} & \Ch{A}(A \otimes_A C, C) \arrow{r}{\sim} & \Ch{A}\left(A, [C,C]\right)\\
      \id_C \arrow[mapsto]{rr} & & u_C
    \end{tikzcd}$$
    making the diagram
    $$\begin{tikzcd}
      \left[C,D\right] \otimes_A A \arrow[swap]{d}{[C,D] \otimes_A u_D}\arrow{rd}{\sim} & & A \otimes_A \left[C,D\right]\arrow[swap]{ld}{\sim}\arrow{d}{u_C \otimes_A [C,D]}\\
      \left[C,D\right] \otimes_A \left[D,D\right]\arrow{r} & \left[C,D\right] &  \left[C,C\right] \otimes_A \left[C,D\right]\arrow{l}
    \end{tikzcd}$$
    with horizontal morphisms composition commute.
  \item
    Again regarding $A$ as a complex concentrated in degree 0, an isomorphism for all chain complexes $C$
    $$\begin{tikzcd}
      \Ch{A}\left(C,C\right) \arrow{r}{\sim} & 
      \Ch{A}\left(C \otimes_A,C\right) \arrow{r}{\sim}&
      \Ch{A}\left(C,\left[A,C\right]\right)\\
      \id_C \arrow{rr} & & \imath_C
    \end{tikzcd}$$
    with inverse
    $$\begin{tikzcd}
      \left[A,C\right] \arrow{r}{\sim} & \left[A,C\right] \otimes_A A \arrow{r}{\operatorname{ev}_{A,C}} & C.
    \end{tikzcd}$$
  \end{enumerate}
\end{rmk}

%Stuff written down so I don't forget the damn proof.
%Probably not relevant; may need to be (re)moved.
\subsection{Miscellaneous}

\begin{prop}
  Let $\A$ be an abelian category and let
  $$0 \to A \overset{i}\to B \overset{p}\to C \to 0$$
  be a short exact sequence.
  The following are equivalent:
  \begin{enumerate}[(i)]
  \item
    There exists a section, $s \in \A(C,B)$, of $p$,
  \item
    There exists a retract, $r \in \A(B,A)$, of $i$,
  \item
    There exists a unique isomorphism of complexes
    $$\begin{tikzcd}[ampersand replacement=\&]
      0 \arrow{r} \& A \arrow{r}{i}\arrow{dd}{1} \& B \arrow{r}{p}\arrow{dd}{\left(\begin{matrix}r\\p\end{matrix}\right)} \& C \arrow{r}\arrow{dd}{1} \& 0\\
      \\
      0 \arrow{r} \& A \arrow{r}{\left(\begin{matrix}1\\0\end{matrix}\right)} \& A \oplus C \arrow{r}{\left(\begin{matrix}0 & 1\end{matrix}\right)} \& C \arrow{r} \& 0
    \end{tikzcd}$$
  \end{enumerate}

  \begin{proof}
    We first show that $(i)$ and $(ii)$ are equivalent, then show the equivalence of $(i)$ and $(iii)$.

    Assume that we have a section, $s$, of $p$.
    We construct a retract from the universal property for kernels:
    $$\begin{tikzcd}
      B \arrow{rr}{1 - s \circ p}\arrow[bend left]{rrr}{0}\arrow[dashed]{rrd}{\exists ! r} && B \arrow{r}{p} & C\\
      && A \arrow[swap]{u}{i}
    \end{tikzcd}$$
    We note that because $i$ is monic
    $$i \circ r \circ i = (1 - s \circ p) \circ i = i = i \circ \id_A$$
    implies that $r \circ i = \id_A$.

    Assuming we have a retract, $r$, of $i$, we obtain the section $s$ of $p$ by a formally dual argument.
    Namely, from the universal property for cokernels:
    $$\begin{tikzcd}
      A \arrow{r}{i}\arrow[bend left]{rrr}{0} & B\arrow{d}{p} \arrow{rr}{1 - i \circ r} && B\\
      & C \arrow[dashed]{urr}{\exists !s}
    \end{tikzcd}$$
    Here we note that because $p$ is an epimorphism we have
    $$p \circ s \circ p = p \circ (1 - i \circ r) = p = \id_C \circ p$$
    and hence  $s \circ p = \id_C$.

    It's clear that $(iii)$ implies $(ii)$, which is equivalent to $(i)$, with the retact given by the component $r$ of the isomorphism.
    Assuming $(i)$, we observe that
    $$i \circ r \circ s = (1 - s \circ p) \circ s = s - s = 0$$
    implies $r \circ s = 0$ and, by construction,
    $$i \circ r + s \circ p = \id_B.$$
    Taking advantage of the biproduct structure in abelian categories we have induced morphisms
    $$\begin{tikzcd}[ampersand replacement=\&]
      A \arrow[swap,bend right]{rrddd}{i}\arrow{rrd}{\left(\begin{matrix}1\\0\end{matrix}\right)}\&\& \&\& C\arrow[bend left]{llddd}{s}\arrow[swap]{lld}{\left(\begin{matrix}0\\1\end{matrix}\right)}\\
      \& \& A \oplus C\arrow[dashed]{dd}{\exists !\left(\begin{matrix}i & s\end{matrix}\right)}\\
      \\
      \& \& B
    \end{tikzcd}$$
    and
    $$\begin{tikzcd}[ampersand replacement=\&]
      \& \& B \arrow[dashed]{dd}{\exists !\left(\begin{matrix}r\\p\end{matrix}\right)}\arrow[bend right]{lldd}{r}\arrow[bend left]{rrdd}{p}\\
      \\
      A \& \& A \oplus C \arrow{ll}{\left(\begin{matrix}1 & 0\end{matrix}\right)} \arrow{rr}{\left(\begin{matrix}0 & 1\end{matrix}\right)}\&\& C
    \end{tikzcd}$$
    Their compositions give the desired identities:
    $$\left(\begin{matrix}r\\p\end{matrix}\right)\left(\begin{matrix}i & s\end{matrix}\right) = \left(\begin{matrix}r \circ i & r \circ s\\p \circ i & p \circ s\end{matrix}\right) = \left(\begin{matrix}1 & 0\\0 & 1\end{matrix}\right)$$
    and
            $$\left(\begin{matrix}i & s\end{matrix}\right)\left(\begin{matrix}r\\p\end{matrix}\right) = i \circ r + s \circ p = \id_B,$$
   whence the isomorphism of complexes.
  \end{proof}
\end{prop}
\end{document}
