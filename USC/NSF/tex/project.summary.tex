\documentclass[11pt]{article}
\renewcommand{\familydefault}{\rmdefault}

\usepackage{bibentry}
\usepackage{enumerate}
\usepackage[margin=1in]{geometry}
%\openup 5pt

\title{MSPRF: Derived Categories in Noncommutative Projective Algebraic Geometry\\\small{Project Summary}}
\author{Blake A. Farman}
\date{}
\begin{document}
\maketitle

\noindent\textbf{Overview.}
Over the past three decades, Derived Categories have become an increasingly important object of study in Algebraic Geometry.
One can trace this importance largely, as suggested by Bridgeland in his 2006 ICM address \cite{Bridgeland06} to three primary applications.
The first is the connection between Algebraic Geometry and String Theory via Homological Mirror Symmetry.
The second is the fact that often times the Derived Category encodes information hidden away from the traditional geometry, unlocking otherwise undetectable deep relationships between varieties.
The third is that Derived Categories may well serve as the dictionary by which one translates the standard techniques of Projective Algebraic Geometry to the setting of noncommutative graded algebra.

To the last point, the methods of birational geometry and moduli spaces have recently been exported to the setting of Noncommutative Algebraic Geometry by way of Artin and Zhang's Noncommutative Projective Schemes, to much acclaim.
Indeed, classifications of noncommutative curves and planes have been achieved.
The Noncommutative Projective Scheme does not, in general, admit a traditional space on which to do geometry, however, the analogue of the Derived Category does exist.
If one is guided by intuition from the commutative case, the main result of \cite{Orlov1997} says that equivalences between the derived category of smooth projective varieties arise geometrically.
The main question the PI intends to tackle is, loosely, ``How good an invariant is the Derived Category of a Noncommutative Projective Scheme?''

The success of Derived Categories in providing classifications in low dimension suggests that, wielded appropriately, derived categories of Noncommutative Projective Schemes should yield results of equal import.

\noindent\textbf{Intellection Merit.}
In the commutative case, the main result of \cite{Orlov1997} says that equivalences between the derived category of smooth projective varieties arise geometrically.
The PI intends to bring to bear his recent work on Fourier-Mukai Kernels for Noncommutative Projective Schemes to provide a tractable approach to settle the classification of noncommutative surfaces up to birational equivalence, as set forth by Stafford in his 2002 ICM address \cite{Stafford02}.

\noindent\textbf{Broader Impacts.}
\begin{enumerate}[Step 1.]
\item
  Collect results on Derived Categories in Noncommutative Algebraic Geometry.
\item
  ?
\item
  PROFIT!
\end{enumerate}
\newpage
\bibliographystyle{alpha}
\nobibliography{biblio}
\end{document}
