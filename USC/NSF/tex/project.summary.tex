\documentclass[11pt]{article}
\renewcommand{\familydefault}{\rmdefault}

\usepackage{bibentry}
\usepackage{enumerate}
\usepackage[margin=1in]{geometry}
%\openup 5pt

\title{MSPRF: Derived Categories in Noncommutative Projective Algebraic Geometry\\\small{Project Summary}}
\author{Blake A. Farman}
\date{}
\begin{document}
\maketitle

\noindent\textbf{Overview.}
Over the past three decades, Derived Categories have become an increasingly important object of study in Algebraic Geometry.
One can trace this importance largely, as suggested by Bridgeland in his 2006 ICM address to three primary applications.
The first is the connection between Algebraic Geometry and String Theory via Homological Mirror Symmetry.
The second is the fact that often times the Derived Category encodes information hidden away from the traditional geometry, unlocking otherwise undetectable deep relationships between varieties.
The third is that Derived Categories may well serve as the dictionary by which one translates the standard techniques of Projective Algebraic Geometry to the setting of noncommutative graded algebra.

To the last point, the methods of birational geometry and moduli spaces have recently been exported to the setting of Noncommutative Algebraic Geometry by way of Artin and Zhang's Noncommutative Projective Schemes, to much acclaim.
Indeed, classifications of noncommutative curves and planes have been achieved, though general surfaces remain open.
Though the noncommutative projective scheme does not, in general, admit a traditional space on which to do geometry, it does admit a direct analogue of the Derived Category.
If, as suggested by Artin and Zhang's introduction of noncommutative projective schemes, one's intuition should be guided by the commutative case, the notion of Fourier-Mukai kernels should play an integral role in this notion of noncommutative projective geometry.
Indeed, the success in studying derived categories in projective algebraic geometry suggests that, wielded appropriately, derived categories of Noncommutative Projective Schemes should yield results of equal import.
%Orlov demonstrates that equivalences between the derived category of smooth projective varieties arise geometrically.
%The main question the author intends to tackle is, loosely, ``How good an invariant is the Derived Category of a Noncommutative Projective Scheme?''



\noindent\textbf{Intellection Merit.}
%In the commutative case, Orlov has shown that equivalences between the derived category of smooth projective varieties arise geometrically.
Derived categories may be the purest bridge between the commutative and noncommutative worlds of algebraic geometry.
The author intends to use his recent work on Fourier-Mukai Kernels for Noncommutative Projective Schemes to provide a tractable approach to adapting methods from projective algebraic geometry to the study of noncommutative algebra.

Development of these derived categories also promise benefits to the physics community.
The Heisenberg uncertainty principle tells us that noncommutative algebra is essential to a description of the physical world.
More precisely, noncommutative spacetimes are an essential part of modern theoretical physics, e.g. \cite{DoNe01}, and already appear in homological mirror symmetry \cite{AKO08}.
Moduli spaces and birational geometry are essential tools for noncommutative projective schemes and derived categories will serve to strengthen and extend their reach.

\noindent\textbf{Broader Impacts.}

\bibliographystyle{alpha}
\nobibliography{biblio}
\end{document}
