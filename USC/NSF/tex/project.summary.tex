\documentclass[11pt]{article}
\renewcommand{\familydefault}{\rmdefault}
\openup 2pt
\usepackage{bibentry}
\usepackage{enumerate}
\usepackage{enumitem}
\setlist{nosep}
\usepackage[margin=1in]{geometry}
%\openup 5pt

\title{MSPRF: Derived Categories in Noncommutative Projective Algebraic Geometry\\\small{Project Summary}}
\author{Blake A. Farman}
\date{}
\begin{document}
\maketitle

\noindent\textbf{Overview.}
Over the past three decades, derived categories have become a central object of study in algebraic geometry.
One can trace this importance largely, as suggested by Bridgeland in his 2006 ICM address, to three primary applications: the connection between algebraic geometry and string theory via homological mirror symmetry; the derived category encodes information hidden away from the traditional geometry, unlocking deep relationships between varieties that are otherwise undetectable;
derived categories may well serve as the dictionary by which one translates the standard techniques of projective algebraic geometry to the setting of noncommutative graded algebra.

\noindent\emph{Research Overview}
To the last point, this propsosal focuses on translating seminal results for derived categories in algebraic geometry to the noncommutative setting.
The author intends to do so using his recent work on Fourier-Mukai kernels for noncommutative projective schemes.
The main research objectives are
\begin{itemize}

\item
  Identify how far away from isomorphic derived equivalent noncommutative projective schemes are,
\item
  Determine whether moduli spaces and birational geometry of noncommutative varieties be encoded in their derived categories.
%  Determine whether moduli spaces of sheaves on a noncommutative variety can be encoded in the derived category,
\item
  Determine to what extent Bridgeland stability can be ported over from the commutative case,
\item
  As a special case of 1, establish an analogue of Bondal-Orlov reconstruction.
\end{itemize}
%To the last point, the methods of birational geometry and moduli spaces have recently been exported to the setting of Noncommutative Algebraic Geometry by way of Artin and Zhang's Noncommutative Projective Schemes, to much acclaim.
%Indeed, classifications of noncommutative curves and planes have been achieved, though general surfaces remain open.
%Though the noncommutative projective scheme does not, in general, admit a traditional space on which to do geometry, it does admit a direct analogue of the Derived Category.
%If, as suggested by Artin and Zhang's noncommutative projective schemes, one's intuition should be guided by the commutative case, the notion of Fourier-Mukai kernels should play an integral role in this notion of noncommutative projective geometry.
%Indeed, the success in studying derived categories in projective algebraic geometry suggests that, wielded appropriately, derived categories of Noncommutative Projective Schemes should yield results of equal import.

%Orlov demonstrates that equivalences between the derived category of smooth projective varieties arise geometrically.
%The main question the author intends to tackle is, loosely, ``How good an invariant is the Derived Category of a Noncommutative Projective Scheme?''



\noindent\textbf{Intellectual Merit.}
%In the commutative case, Orlov has shown that equivalences between the derived category of smooth projective varieties arise geometrically.
Derived categories may be the purest bridge between the commutative and noncommutative worlds of algebraic geometry.
The author intends to use his recent work on Fourier-Mukai kernels for noncommutative projective schemes to provide a tractable approach to adapting methods from projective algebraic geometry to the study of noncommutative algebra.
The seminal results for derived categories in algebraic geometry should yield deep consequences in the noncommutative case.

Development of methods of derived categories also promises benefits to the physics community.
The Heisenberg uncertainty principle tells us that noncommutative algebra is essential to a description of the physical world.
More precisely, noncommutative spacetimes are an essential part of modern theoretical physics, e.g. \cite{DoNe01}, and already appear in homological mirror symmetry \cite{AKO08}.
Moduli spaces and birational geometry are essential tools for noncommutative projective schemes and derived categories will serve to strengthen and extend their reach.

\noindent\textbf{Broader Impacts.}
The proposed activities will have a profound impact on the advancement of scientific knowledge, both in mathematics and physics, as well as the mentoring of an early career researcher.
Beyond these, work at the University of Toronto will help foster international connections.
Additionally, the author has participated in the development of K-12 students in mathematics through the University of South Carolina's high school math contest, and receipt of the MSPRF will help ensure that the author is in a position to seek out further opportunities of this kind in the future.
\bibliographystyle{alpha}
\nobibliography{biblio}
\end{document}
