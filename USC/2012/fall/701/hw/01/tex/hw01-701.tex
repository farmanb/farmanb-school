\documentclass[10pt]{amsart}
\usepackage{../../../../../../../tex/mymath/mymath}
\usepackage{amsmath,amsthm,amssymb,amsfonts,enumerate}
\openup 5pt
\author{Blake Farman\\University of South Carolina}
\title{Math 701:\\Homework 01}
\date{September 5, 2012}
\pdfpagewidth 8.5in
\pdfpageheight 11in
\usepackage[margin=1in]{geometry}
\begin{document}
\maketitle

\newtheorem{thm}{}
\begin{thm}
	Suppose that $G$ has subgroups $H$ and $K$, and that $G = H \cup K$.  Prove that either $H = G$ or $K = G$.
	\begin{proof}
		Observe that if $H \subseteq K$ or $K \subseteq H$, then the result follows directly.
		Hence it suffices to assume the existence of elements $h \in H$ and $k \in K$ such that $h \not\in K$ and $k \not\in H.$
		
		Consider the product $hk \in G = H \cup G$ and observe that $h^{-1}(hk) = k$ and $(hk)k^{-1} = h$ imply $hk \not\in H$ and $hk \not \in K$, respectively.  Hence $hk \not \in H \cup K = G,$ a contradiction.  Therefore either $H = G$ or $K = G,$ as desired.
	\end{proof}
\end{thm}

\begin{thm}
	Suppose that $H$ is finite, non-empty subset of $G.$  Show that $H$ is a subgroup if and only if $H$ is closed under the  binary operation $G.$
	\begin{proof}
	Assume $H \leq G$ and note that $H$ is closed by definition.  
	Conversely, assume $H$ is closed under the operation in $G.$
	Observe that if $H = \left< 1 \right> ,$ then $H \leq G.$
	Hence it suffices to assume the existence of $1 \not = h \in H.$  
	Since $H$ is finite, there exist $0< i < j$ such that $h^i = h^j.$  
	Multiplying both sides by $(h^{-1})^i$ on the left we obtain
		$$1 = h^{j-i} = h^{j-i-1}h.$$
	Observe that by the choice of $h$ we have $0 < j-i-1$, from which $h^{j-i-1} = h^{-1}$ follows.  
	Moreover, since $H$ is closed we have $1, h^{-1} \in H.$  
	Therefore $H \leq G,$ as desired.
	\end{proof}
\end{thm}

\begin{thm}
	Suppose that for all $g \in G$, we have $g^2 = 1_G.$  Show that $G$ is abelian.
	\begin{proof}
	Let $g,h \in G$ be given and observe that $gh \in G$ implies $(gh)^2 = 1.$  
	Multiplying both sides by $(gh)^{-1} = hg$ yields $gh = hg.$ 
	Therefore $G$ is abelian.
	\end{proof}
\end{thm}

\begin{thm}
	Suppose that $G$ is finite with even order.  Show that there exists $g \in G$ such that $o(g) = 2.$
	\begin{proof}
	Let $2n$ be the order of $G$ and consider the set of pairs of elements with order strictly larger than two, 
		$$A = \left\{ (g,g^{-1}) \mid g \not = g^{-1} \in G \right\}.$$
	Since $G$ must have an identity, this set contains at most $n - 1$ distinct elements. 
	Hence there are at most $2n-2$ elements of order strictly larger than two.
	In particular, the remaining non-identity element, say $h,$ must satisfy $h = h^{-1}$.
	Therefore $G$ contains an element of order two, as desired.
	\end{proof}
\end{thm}

\begin{thm}
  Let $\displaystyle{GL_2(\mathbb{Q}) = \left\{ \left(
  \begin{array}{cc}
    a & c\\
    c & d\\
    \end{array}
  \right) \in \text{Mat}_{2\times 2}\mid ad - bc \not = 0\right\}},$ the group of non-singular $2 \times 2$ matrices with rational entries.  Let $S = \displaystyle{\left(
    \begin{array}{cc}
      0 & -1\\
      1 & 0
    \end{array}
    \right)}$ and $R = \displaystyle{\left(    
    \begin{array}{cc}
      0 & 1\\
      -1 & -1
    \end{array}\right)}$.
  \begin{enumerate}[(a)]
    \item
      Compute the orders of $S$ and $R$.
    \item 
      Compute the order of $SR$.
  \end{enumerate}
  \begin{proof}
    \begin{enumerate}[(a)]
    \item
      First compute 
      $$S^2 = \left(\begin{array}{cc}
        -1 & 0\\
        0 & -1\\
        \end{array}\right)$$
      and observe that $S^2 = (S^2)^{-1}$.  Hence, $o(S) = 4$.
      Next, compute
      $$R^2 = \left(\begin{array}{cc}
        -1 & -1\\
        1 & 0 \\
        \end{array}\right)$$
      and, similarly, observer $R^{-1} = R^2.$  Hence $o(R) = 3$.
    \item
      Compute 
      $$SR = \left(\begin{array}{cc}
        1 & 1\\
        0 & 1 \\
        \end{array}\right)
      \text{ and }
      SR^2 = \left(\begin{array}{cc}
        1 & 2\\
        0 & 1 \\
        \end{array}\right).$$
      Arguing inductively for $n \geq 2$, we see that $$(SR)^{n+1} = (SR)(SR)^{n} = \left(\begin{array}{cc}
        1 & 1\\
        0 & 1 \\
        \end{array}\right)
      \left(\begin{array}{cc}
        1 & n\\
        0 & 1 \\
        \end{array}\right) = 
      \left(\begin{array}{cc}
        1 & n+1\\
        0 & 1 \\
        \end{array}\right).$$
    \end{enumerate}
    Therefore $o(SR) = \infty.$
  \end{proof}
\end{thm}

\begin{thm}
	Let $a,b \in G$, and suppose that there exists $r \geq 1$ with $bab^{-1} = a^r.$
	Show, for all $j \geq 1$, that $b^jab^{-j} = a^{r^j}$.
	\begin{proof}
          First note that it suffices to show that if $bab^{-1} = a^r$, then $b^j a = a^{r^j} b^j$.
          Rearranging the equation $bab^{-1} = a^r,$ we obtain $a = b^{-1}a^rb$. 
          Fix $n \geq 0$ and consider $a^{r^n} = (b^{-1}a^rb)(b^{-1}a^rb)\ldots(b^{-1}a^rb)$.
          Observe that the right hand side collapses to $b^{-1}(a^r)^{r^n}b$, whence the commutation relation 
          \begin{equation}\label{comm}
            ba^{r^n} = a^{r^{n+1}}b.
          \end{equation}
          It is then clear that, for fixed $j \geq 1,$ the identity $b^ja = a^{r^j}b^j$ follows from $j$ applications of \eqref{comm} to the left hand side, where $n$ ranges from zero to $j - 1$.
	\end{proof}
\end{thm}

\begin{thm}
	Suppose that $g \in G$ has finite order $n$.
	\begin{enumerate}[(a)]
			\item
                          Let $t \in \mathbb{Z}$.  Show that $o(g^t) = \displaystyle{\frac{n}{\gcd(n,t)}}.$
			\item
                          Suppose that $0 \leq r \leq n-1$ has $\gcd(r,n) = 1$.  Show that $o(g^r) = n$.
			\item 
                          Let $0 < d \mid n$.  Show that $g^{n/d}$ has order $d$ in $G$.
	\end{enumerate}
	\begin{proof}
          \begin{enumerate}[(a)]
            \item
              Let $d = \gcd(n,t)$ and observe that $(g^t)^{n/d} = (g^n)^\alpha = 1,$ for some $\alpha \in \mathbb{Z}$.
              Consider any other integer $s$ such that $(g^t)^s = 1$.
              Since $n$ is the order of $g$ in $G$, it follows that $st$ must be a multiple of $n$.
              Minimality of $n/d$ then follows directly from the fact that $t(n/d) = \text{lcm}(n,t).$
            \item
              That $o(g^r) = n$ follows directly from part (a) with $t = r$.
            \item
              Apply part (a) with $t = n/d$ to obtain $$o(g^{n/d}) = \frac{n}{\gcd(n/d,n)} = n\left(\frac{d}{n}\right) = d.$$
        \end{enumerate}
	\end{proof}
\end{thm}

\begin{thm}
	Suppose that $x,y \in G$ commute: $xy = yx$.  Suppose also that $o(x) = m, o(y) = n$, and $o(xk) = k$ are finite.
	\begin{enumerate}[(a)]
		\item
                  Show that $k \mid \text{lcm}(m,n).$
		\item
                  Suppose that $\left< x \right> \cap \left< y \right> = \left< 1_G \right>.$  Show that $k = \text{lcm}(m,n).$
                \item
                  Suppose that $\gcd(m,n) = 1.$
		\begin{enumerate}[i.]
			\item 
                          Show that $\left< x \right> \cap \left< y \right> = \left< 1_G \right>.$
			\item
                          Now, suppose you know for all $a,b \in \mathbb{Z}$, that $\text{lcm}(a,b) = \displaystyle{\frac{ab}{\gcd(a,b)}.}$  Use this fact, together with (b) and (c,i) to show that $k = mn.$
		\end{enumerate}
	\end{enumerate}
	\begin{proof}
          \begin{enumerate}[(a)]
          \item
            Let $\ell = \text{lcm}(m,n)$ and consider $(xy)^\ell = x^\ell y^\ell.$
            Since there exist integers $i,j$ such that $\ell = ni$ and $\ell = mj,$ we have $(xy)^\ell = (x^m)^i(y^n)^j = 1_G.$  
            Therefore $k \mid \ell$, as desired.
          \item
            Consider the identity $x^ky^k = 1_G.$  Since the intersection of $\left< x \right>$ and $\left< y \right>$ was assumed to be trivial, it is clear that $x^k = y^k = 1_G.$
            Hence $k$ must be a common multiple of $m$ and $n$.
            By the minimality of $k$ and $\ell$, we have $k = \ell,$ as desired.
          \item
            \begin{enumerate}[i.]
              \item
                Suppose that $x^\alpha \in \left< y \right>$ for some integer $0 < \alpha$.
                Then $x^{\alpha n} = 1$ implies $m \mid \alpha n$.
                Since $m$ and $n$ were assumed to be relatively prime,  $m$ must divide $\alpha$.
                Given that $x^\alpha \in \left< y \right>$ and $m \mid \alpha$, we have $x^\alpha = 1.$  
                Therefore $\left< x \right> \cap \left< y \right> = \left< 1_G \right>$.
                \footnote[1]{Note that, were you to choose an element of $\left< y \right>$, the same argument holds, mutatis mutandis.}
              \item
                By assumption, $\gcd(m,n) = 1$ and thus by (c,i) $\left< x \right> \cap \left< y \right> = <1_G>.$  
                Hence by (b), $$k = \text{lcm}(m,n) = mn/\gcd(m,n) = mn.$$
            \end{enumerate}
          \end{enumerate}
	\end{proof}
\end{thm}

\begin{thm}
	Suppose that $G \not = \left< 1_G \right>$ is finite and has no proper subgroups.  Show that there exists a prime $p$ such that $G$ is cyclic and $|G| = p$.
	\begin{proof}
          Let $1 \not = g \in G$ be given and observe that since $\left< g \right> \leq G$ and $G$ has no proper subgroups, $\left< g \right> = G.$  
          Hence $G$ is cyclic.
          Moreover, since $G$ is cyclic with no proper subgroups, and cyclic groups have subgroups of all orders dividing the order of the group, $G$ must have prime order.
	\end{proof}
\end{thm}
\end{document}
