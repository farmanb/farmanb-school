\documentclass[10pt]{amsart}
\usepackage{amsmath,amsthm,amssymb,amsfonts,enumerate,mymath}
\openup 5pt
\author{Blake Farman\\University of South Carolina}
\title{Math 701:\\Homework 03}
\date{September 19, 2012}
\pdfpagewidth 8.5in
\pdfpageheight 11in
\usepackage[margin=1in]{geometry}

\begin{document}
\maketitle

\newcommand{\Inn}[1]{\operatorname{Inn}\left(#1\right)}
\newcommand{\Aut}[1]{\operatorname{Aut}\left(#1\right)}
\newcommand{\cntr}[1]{\mathbf{Z}\left(#1\right)}
\newcommand{\abs}[1]{\left| #1 \right|}
\newcommand{\SL}[2]{\operatorname{SL}_#1\left(#2\right)}
\newcommand{\Mat}[2]{\operatorname{Mat}_{#1}\left(#2\right)}
\renewcommand{\qedsymbol}{\ensuremath{\blacksquare}}

\newtheorem{thm}{}
\newtheorem{lem}{Lemma}

\begin{thm}
  Suppose that $H$ and $N$ are subgroups of $G$ with $N \lhd G$.
  Show that $H \cap N \lhd H$.
  \begin{proof}
    Let $n \in H \cap N$ and $h \in H$ be given and observe that $n \in N$.
    Therefore, by normality of $N$ in $G$, $hnh^{-1} \in N$ implies $H \cap N \lhd H$.
  \end{proof}
\end{thm}

\begin{thm}
  Define $\SL{2}{\mathbb{Z}} = \left\{ M \in \Mat{2 \times 2}{\mathbb{Z}} \mid \det(M) = 1 \right\}$.
  It is a group, a fact that you do not have to prove.
  Let $N$ be a positive integer and define
  \begin{align*}
    \Gamma_0(N) & := \left\{ \left(
    \begin{array}{cc}
      a & b\\
      c & d
    \end{array}
    \right) \in \SL{2}{\mathbb{Z}}
    \;\middle\vert\; c \equiv 0\, (\text{mod } N)
    \right\}\\
    \Gamma_1(N) & := \left\{ \left(
    \begin{array}{cc}
      a & b\\
      c & d
    \end{array}
    \right) \in \SL{2}{\mathbb{Z}}
    \;\middle\vert\; a \equiv d \equiv 1,\, c \equiv 0\, (\text{mod } N)
    \right\}\\
  \end{align*}
  Observe that $\Gamma_1(N) \subseteq \Gamma_0(N) \SL{2}{\mathbb{Z}}$.
  \begin{enumerate}[(a)]
  \item
    Prove that $\Gamma_0(N)$ is a subgroup of $\SL{2}{\mathbb{Z}}$.
  \item
    Prove that $\Gamma_1(N) \lhd \Gamma_0(N)$ and that $\Gamma_0(N) / \Gamma_1(N) \cong (\mathbb{Z}/N\mathbb{Z})^{\times}$.
  \item
    Let $S = \left(
    \begin{array}{cc}
      0 & -1\\
      1 & 0
    \end{array}
    \right)$,
    $T = \left(
    \begin{array}{cc}
      1 & 1\\
      0 & 1
    \end{array}
    \right)$.
    Show that $\SL{2}{\mathbb{Z}} = \left<S,T\right>$.
  \end{enumerate}
  
  \begin{proof}
    \begin{enumerate}[(a)]
    \item
      Let $A = \left(
      \begin{array}{cc}
	a & b\\
	c & d
      \end{array}
      \right)$,
      $B = \left(
      \begin{array}{cc}
	e & f\\
	g & h
      \end{array}
      \right) \in \Gamma_0(N)$ be given.
      Observe that, modulo $N$, we have
      $$AB^{-1} = 
      \left(
      \begin{array}{cc}
	a & b\\
	0 & d
      \end{array}
      \right)
      \left(
      \begin{array}{cc}
	h & -f\\
	0 & e
      \end{array}
      \right)
      =
      \left(
      \begin{array}{cc}
	ah & be - af\\
	0 & de
      \end{array}
      \right) \in \Gamma_0(N).$$
      Therefore $\Gamma_0(N) \leq \SL{2}{\mathbb{Z}}$.
    \item
      Let $A = \left(
      \begin{array}{cc}
	a & b\\
	c & d
      \end{array}
      \right)$,
      $B = \left(
      \begin{array}{cc}
	e & f\\
	g & h
      \end{array}
      \right) \in \Gamma_1(N)$ be given.
      Observe that, modulo $N$, we have
      $$AB^{-1} = 
      \left(
      \begin{array}{cc}
	1 & b\\
	0 & 1
      \end{array}
      \right)
      \left(
      \begin{array}{cc}
	1 & -f\\
	0 & 1
      \end{array}
      \right)
      =
      \left(
      \begin{array}{cc}
	1 & b - f\\
	0 & 1
      \end{array}
      \right) \in \Gamma_1(N).$$
      Therefore $\Gamma_1(N) \leq \Gamma_0(N)$.
      
      To see that $\Gamma_1(N)$ is normal in $\Gamma_0(N)$, let $A = \left(
      \begin{array}{cc}
	a & b\\
	c & d
      \end{array}
      \right) \in \Gamma_0(N)$, and
      $B = \left(
      \begin{array}{cc}
	e & f\\
	g & h
      \end{array}
      \right) \in \Gamma_1(N)$
      be given.
      Then, modulo N, we have
      $$ABA^{-1} = \left(
      \begin{array}{cc}
	a & b\\
	0 & d
      \end{array}
      \right)
      \left(
      \begin{array}{cc}
	1 & f\\
	0 & 1
      \end{array}
      \right)
      \left(
      \begin{array}{cc}
	d & -b\\
	0 & a
      \end{array}
      \right) = 
      \left(
      \begin{array}{cc}
	ad & a^2f\\
	0 & ad
      \end{array}
      \right)
      $$
      Since $\det(A) = ad - bc$ and $bc \equiv 0 (\text{mod }\, N)$, it follows that $ad \equiv 1 (\text{mod }\, N)$.
      Therefore $ABA^{-1} \in \Gamma_1(N)$ implies that $\Gamma_1(N) \lhd \Gamma_0(N)$.
      
      Now observe that by the previous arguments, the cosets of $\Gamma_1(N)$ in $\Gamma_0(N)$ are determined by the upper left entry of the representative.  Namely,
      for any two matrices in $\Gamma_0(N)$ reduced modulo $N$, $A = \left(
      \begin{array}{cc}
	a & b\\
	0 & d
      \end{array}\right)$ and $B = \left(
      \begin{array}{cc}
	e & f\\
	0 & h
      \end{array}\right)$, the cosets they represent have the form
      \begin{align*}
      \left(
      \begin{array}{cc}
	a & b\\
	0 & d
      \end{array}\right)\Gamma_1(N) &= \left\{
      \left(
      \begin{array}{cc}
	a & n\\
	0 & d
      \end{array}\right) \middle\vert n \in \Z/N\Z, a \in (\Z/N\Z)^{\times} \text{ and } d = a^{-1} \right\} \text{ and } \\
      \left(
      \begin{array}{cc}
	e & f\\
	0 & h
      \end{array}\right)\Gamma_1(N) &= \left\{\left(
      \begin{array}{cc}
	e & n\\
	0 & h
      \end{array}\right) \middle\vert n \in \Z/N\Z, e \in (\Z/N\Z)^{\times} \text{ and } h = a^{-1} \right\},
      \end{align*}
      which agree if and only if $a = e$.
%      Furthermore, the matrix representing a coset's inverse simply swaps the elements on the diagonal.  
%      Namely, $$\left(
%      \begin{array}{cc}
%	a & *\\
%	0 & d
%      \end{array}\right)\left(
%      \begin{array}{cc}
%	d & *\\
%	0 & a
%      \end{array}\right) = \left(
%      \begin{array}{cc}
%	ad & *\\
%	0 & ad
%      \end{array}\right),$$
%      where $ad \equiv 1 (\text{mod } N)$ implies the right hand side is the form of the coset $\Gamma_1(N)$.
      Thus we may define the injective map 
      \begin{align*}
        \varphi \colon \Gamma_0(N) / \Gamma_1(N) &\longrightarrow (\Z/N\Z)^{\times}\\
        \left(
      \begin{array}{cc}
	a & *\\
	0 & d
      \end{array}
      \right) & \longmapsto a.
      \end{align*}
      By the observation above, there are $\phi(N)$ cosets in the domain and $\phi(N)$ elements in the target space, hence it suffices to show that $\varphi$ is a homomorphism.
      First consider $\varphi(A\Gamma_1(N)B\Gamma_1(N))$.
      Since $A\Gamma_1(N)B\Gamma_1(N) = AB\Gamma_1(N)$, we have $$\varphi(A\Gamma_1(N)B\Gamma_1(N)) = ae = \varphi(A\Gamma_1(N))\varphi(B\Gamma_1(N)),$$
      which shows $\varphi$ is a homomorphism.
      Therefore $\Gamma_0(N) / \Gamma_1(N) \cong (\mathbb{Z}/N\mathbb{Z})^{\times}$.
      
    \item

    \end{enumerate}
  \end{proof}
\end{thm}

\begin{thm}
  Recall that $\cntr{Z} = \left\{ g \in G \mid \forall\, x \in G, \, xg = gx \right\}$.
  \begin{enumerate}[(a)]
  \item
    Suppose that $G/\cntr{G}$ is cyclic.  Show that $G$ is abelian.
  \item
    Show that $G/\cntr{Z} \cong \Inn{G}$.  Conclude that $\Inn{G}$ cannot be cyclic unless it is the trivial group, $\left<1\right>$.
  \item
    Suppose that there exist primes $p,q$ (not necessarily distinct) such that $\abs{G} = pq$.
    Prove that $G$ is either abelian or $\cntr{G} = 1$.
  \end{enumerate}
  \begin{proof}
    \begin{enumerate}[(a)]
    \item
      Let $\pi: G \rightarrow G/\cntr{G}$ be the canonical projection homomorphism.
      Since $\pi$ is surjective, we may assume the existence of an element $g_0$ of $G$ such that $\left< \pi(g_0) \right> = G/\cntr{G}$.
      Let $g,h \in G$ be given.
      By assumption there exist integers $\alpha, \beta$ such that $\pi(g) = \pi(g_0^{\alpha})$ and $\pi(h) = \pi(g_0^{\beta})$.
      Hence there exist elements $z, z^{\prime} \in \cntr{G}$ such that $g = g_0^{\alpha}z$ and $h = g_0^{\beta}z^{\prime}$.
      Therefore
      $$gh = (g_0^{\alpha}z)(g_0^{\beta}z^{\prime})
      = g_0^{\alpha}g_0^{\beta}zz^{\prime}
      = g_0^{\beta}g_0^{\alpha}zz^{\prime}
      = (g_0^{\beta}z^{\prime})(g_0^{\alpha}z)
      = hg.$$
    \item
      Define the map
      \begin{align*}
	\varphi \colon G &\rightarrow \Inn{G}\\
	g & \mapsto gxg^{-1} = \theta_g.
      \end{align*}
      Let $g,h \in G$ be given and observe that $\varphi(gh) = ghxh^{-1}g^{-1} = \theta_g\theta_h = \varphi(g)\varphi(h)$ implies $\varphi$ is a homomorphism.
      Furthermore, observe that $\cntr{G} \subseteq \ker \varphi$.
      Fix $k \in \ker\varphi$, hence $kxk^{-1} = x$ holds for all $x \in G$.
      This shows the reverse containment and so, by the First Isomorphism Theorem, we have $G/\cntr{G} \cong \Inn{G}$.
      
      It now follows that $\Inn{G}$ is cyclic if and only if $G/\cntr{G}$ is cyclic.
      Moreover, if it were cyclic, then by part (a) we have $G = \cntr{G}$ and thus $\Inn{G}$ is trivial.
    \item
      By Lagrange's Theorem, the order of $\cntr{G}$ divides either $1$, $pq$, $p$, or $q$.
      The former two cases imply that either $\cntr{G}$ is trivial or $G$ is abelian, so it suffces to show $\cntr{G}$ does not have prime order.
      
      Suppose $\cntr{G}$ were to have prime order.
      Then $G/\cntr{G}$ would have prime order, and thus would be cyclic.
      Moreover, by part (a), $G$ would be abelian.
      However, $\cntr{G}$ was supposed to have order strictly smaller than that of $G$, a contradiction.
      Therefore either $G$ is abelian or $\cntr{G}$ is trivial.
      
    \end{enumerate}
  \end{proof}
\end{thm}
\begin{thm}
  Let $G = \left\{f \colon \mathbb{R} \rightarrow \mathbb{R} \mid \exists\, a \not = 0,\, b \in \mathbb{R} \text{ such that } \forall\, x\in \mathbb{R},\, f(x) = ax + b\right\}$.
  The set $G$ is the group of affine transformations of $\mathbb{R}$.  
  Show that $G$ has a subgroup $H$ and an element $g$ such that $H^{g}$ is a proper subgroup of $H$.
  \begin{proof}
    Let 
    $$H = \left\{ f \colon \mathbb{R} \rightarrow \mathbb{R} 
    \mid \exists\, b \in \mathbb{Z} \text{ such that } \forall\, x \in \mathbb{R},\, f(x) = x + b\right\}.$$ 
    Let $h_1(x) = x + b_1, h_2(x) = x + b_2 \in G$ be given.
    Then we have $h_1h_2^{-1}(x) = x + (b_2 - b_1)\in H$, which shows $H$ is a subgroup of $G$.
    
    Now let $g = 2x \in G$ and let $h = x + b \in H$ be given.
    With some minor computation, it is easy to see that $ghg^{-1}(x) = x + 2b$.
    Hence $H^{g} = \left\{ f \colon \mathbb{R} \rightarrow \mathbb{R} \mid \exists\, b \in 2\mathbb{Z} \text{ such that } \forall\, x \in \mathbb{R},\, f(x) = x+b\right\} < H$.
  \end{proof}
\end{thm}
\begin{thm}
  Let $G$ be finite, and let $H, K$ be subgroups of $G$.  Suppose that $\gcd(\abs{H}, \abs{K}) = 1$.
  Show that $H \cap K = \left<1\right>$.
  \begin{proof}
    By Lagrange's Theorem, $\abs{H \cap K}$ divides both $\abs{H}$ and $\abs{K}$, both of which are finite by assumption.
    Moreover, it was assumed that $\gcd(\abs{H}, \abs{K}) = 1$, hence $\abs{H \cap K} = 1$.
    Therefore $H \cap K = \left< 1_G\right>$.
  \end{proof}
\end{thm}

\begin{thm}
  Let $G$ be finite.
  \begin{enumerate}[(a)]
  \item
    Suppose that $H$ is a subgroup, that $N \lhd G$, and that $\gcd(\abs{H}, [G:N]) = 1$.
    Prove that $H \subseteq N$.
  \item
    Suppose that $N \lhd G$ and $\gcd(\abs{N}, [G:N]) = 1$.
    Prove that $N$ is the unique subgroup of order $\abs{N}$.
  \end{enumerate}
  \begin{proof}
    Observe that $NH$ is a subgroup by the normality of $N$ in $G$.
    Consider the index of $NH$ in $G$, $$[G : NH] = \frac{\abs{G}}{\abs{NH}} = \frac{[G:N]\abs{N \cap H}}{\abs{H}}.$$
    By assumption, $\abs{H}$ does not divide $[G:N]$, and so $\abs{H}$ must divide $\abs{H \cap N}$.
    Moreover, since $H \cap N \leq H$, Lagrange's Theorem implies $\abs{H \cap N}$ divides $\abs{H}$, hence $\abs{H} = \abs{H \cap N}$.
    Therefore $H = H \cap N$ implies $H \subseteq N$.
  \end{proof}
\end{thm}
\begin{thm}
  Let $G$ be finite, let $H$ be a subgroup, and let $\phi \colon G \rightarrow K$ be a homomorphism.
  \begin{enumerate}[(a)]
  \item
    Show that $[\phi(G) : \phi(H)]$ divides $[G : H]$.
  \item
    Show that $\abs{\phi(H)}$ divides $\abs{H}$.
  \end{enumerate}
  \begin{proof}
    \begin{enumerate}[(a)]
    \item
      Observe that by the First Isomorphism Theorem we have $\abs{G} = \abs{\phi(G)}\abs{ker\phi}$ and $\abs{H} = \abs{\phi(H)}\abs{\ker\phi|_H}$, all of which are finite.
      Hence $$[G : H] = \frac{\abs{G}}{\abs{H}} 
      = \frac{\abs{\phi(G)}}{\abs{\phi(H)}}\frac{\abs{\ker\phi}}{\abs{\ker\phi|_H}}
      = [\phi(G) : \phi(H)]\frac{\abs{\ker\phi}}{\abs{\ker\phi|_H}}.$$
      
      It remains only to show that $\abs{\ker\phi|_H}$ divides $\abs{\ker\phi}$.
      Let $h_1, h_2 \in \ker\phi|_H$ be given.
      Then we have 
      $$\phi(h_1h_2^{-1}) = \phi(h_1)\phi(h_2)^{-1} = 1_G,$$
      which implies $\ker\phi|_H \leq \ker\phi$, and the result follows from Lagrange's Theorem.  
    \item
      That $\abs{\phi(H)}$ divides $\abs{H}$ follows directly from the equation $\abs{H} = \abs{\phi(H)}\abs{\ker\phi|_H}$, obtained in part (a).
    \end{enumerate}
  \end{proof}
\end{thm}
\end{document}
