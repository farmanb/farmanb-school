\documentclass[10pt]{amsart}
\usepackage{amsmath,amsthm,amssymb,amsfonts,enumerate}
\openup 5pt
\author{Blake Farman\\University of South Carolina}
\title{Math 701:\\Homework 04}
\date{September 28, 2012}
\pdfpagewidth 8.5in
\pdfpageheight 11in
\usepackage[margin=1in]{geometry}

\begin{document}
\maketitle

\newcommand{\Inn}[1]{\operatorname{Inn}\left(#1\right)}
\newcommand{\Aut}[1]{\operatorname{Aut}\left(#1\right)}
\newcommand{\cntr}[1]{\mathbf{Z}\left(#1\right)}
\newcommand{\abs}[1]{\left| #1 \right|}
\newcommand{\SL}[2]{\operatorname{SL}_#1\left(#2\right)}
\newcommand{\Mat}[2]{\operatorname{Mat}_{#1}\left(#2\right)}
\newcommand{\orbit}[1]{\mathcal{O}_{#1}}
\newcommand{\real}[1]{\operatorname{\mathfrak{Re}}\left(#1\right)}
\newcommand{\imag}[1]{\operatorname{\mathfrak{Im}}\left(#1\right)}
\renewcommand{\qedsymbol}{\ensuremath{\blacksquare}}

\newtheorem{thm}{}
\newtheorem{lem}{Lemma}

\begin{thm}
  \begin{proof}
    \begin{enumerate}[(a)]
      \item
        Let $N_G(H)$ act on $H$ by conjugation.
        Since $N_G(H)$ normalizes $H$, the action is well defined and that the axioms for the action are satisfied follows from the group structure in $G$.
        The kernel of this action is, by definition, 
        $$K = \left\{n \in N_G(H) \mid \forall h \in H, nhn^{-1} = h\right\}.$$ 
        Observing that $nhn^{-1} = h$ implies $nh = hn$, it is clear that $C_G(H) = K \unlhd N_G(H)$.
      \item
        Observe that conjugation by the elements of $N_G(H)$ are precisely the inner automorphisms of $H$.
        Hence we may define the map
        \begin{align*}
          \varphi \colon N_G(H) &\rightarrow \Aut{H}\\
          n &\mapsto \theta_n.
        \end{align*}
        To see that $\varphi$ is a homomorphism, let $n_1,n_2 \in N_G(H)$ and $h \in H$ be given.
        Observe that by associativity, $$\theta_{n_1n_2}(h) = (n_1n_2)h(n_1n_2)^{-1} = (n_1n_2) h (n_2^{-1}n_1^{-1}) = n_1(n_2 h n_2^{-1})n_1^{-1} = \theta_{n_1}\theta_{n_2}(h).$$
        Hence $\varphi(n_1n_2) = \varphi(n_1)\varphi(n_2)$ implies $\varphi$ is a homomorphism.
        By the argument in part (a), $\ker\varphi = C_G(H)$.
        Therefore, by the First Isomorphism Theorem, $N_G(H)/C_G(H) \cong \varphi(N_G(H)) \leq \Aut{H}$.
    \end{enumerate}
  \end{proof}
\end{thm}

\begin{thm}
  \begin{proof}
    
  \end{proof}
\end{thm}


\begin{thm}
  \begin{proof}
    Let $\mathcal{L}$ be the left cosets of $K$ in $G$ and define the action
    \begin{align*}
      H \times \mathcal{L} &\rightarrow \mathcal{L}\\
      (h,gK) &\mapsto hgK.
    \end{align*}
    Observe that for fixed $g$, the orbit of $gK$ is $\orbit{gK} = \left\{hgK \;\middle\vert\; h \in H\right\} = HgK.$
    Hence we may write $$HgK = \bigcup_{hgK \in \orbit{gK}} hgK,$$
    and thus $\abs{HgK} = \sum_{hgK \in \orbit{gK}} \abs{hgK} = \abs{\orbit{gK}}\abs{gK} = \abs{\orbit{gK}}\abs{K}.$
    Rearranging, we obtain 
    \begin{equation}\label{crap}
          \abs{\orbit{gK}} = \frac{\abs{HgK}}{\abs{K}}.
    \end{equation}
    
    Counting in another way, we observe that $H_{gK} = \left\{h \in H \;\middle\vert\; hgK = gK\right\}$.
    Then for any such $h$, there exists an element $k$ of $K$ such that $hg = gk$, from which it follows that $h = gkg^{-1}$.
    Hence, by the Orbit Stabilizer Theorem, 
    \begin{equation}\label{morecrap}
      \abs{\orbit{gK}} = [H : G_{gK}] = \frac{\abs{H}}{\abs{H \cap gKg^{-1}}}.
    \end{equation}
    Equating \eqref{crap} and \eqref{morecrap}, we obtain $\abs{HgK} = \frac{\abs{K}\abs{H}}{\abs{H \cap gKg^{-1}}}$.
    Note that we obtain the usual order for the set $HK$ whenever any of $g \in K$, $g \in N_G(K)$, or $K \unlhd G$ hold.
  \end{proof}
\end{thm}

\begin{thm}
  \begin{proof}
    Let $G$ act on itself by conjugation.
% and let $\chi$ be the map
%    \begin{align*}
%      \chi \colon G &\rightarrow \mathbb{N}\cup\left\{0\right\}\\
%      g & \mapsto \abs{\left\{ G_g \right\}}
%    \end{align*}
    For each $g \in G$, the probability that $g$ is chosen is $1/\abs{G}$.
    The probability that the next element chosen commutes with $g$ is $\chi(g)/\abs{G}$.
    Then by the Cauchy-Frobenius Theorem, we have the probability that two randomly selected elements commute is given by 
    $$\sum_{g \in G} \frac{1}{\abs{G}} \frac{\chi(g)}{\abs{G}} = \frac{1}{\abs{G}} \left(\frac{1}{\abs{G}} \sum_{g \in G} \chi(g) \right) = \frac{k}{\abs{G}},$$
    where $k$ is the number of conjugacy classes of $G$.
  \end{proof}
\end{thm}

\begin{thm}
  
  \begin{proof}
    \begin{enumerate}[(a)]
      \item\label{5a}
        Observe that if $\alpha \in \Omega^P$, then $\orbit{\alpha} = \left\{\alpha\right\}$.
        Since the orbits partition $\Omega$, we have by the Orbit Stabilizer Theorem $\abs{\Omega} = \abs{\Omega^P} + \sum_{\alpha \not \in \Omega^P} [P : P_\alpha]$.
        By Lagrange's Theorem, $\sum_{\alpha \not \in \Omega^P} [P : P_\alpha] \equiv 0\, (\text{mod }P)$.
        Therefore $\abs{\Omega} \equiv \abs{\Omega^P}\, (\text{mod } P)$.
      \item\label{5b}
        Let $P$ act on itself by conjugation and observe that $P^P = \cntr{P}$.
        It is immediate from part (\ref{5a}) that $\abs{\cntr{P}} \equiv p^a \equiv 0 \, (\text{mod } p)$.
        Therefore $\cntr{P}$ is not trivial.
      \item\label{5c}
        Apply the result of problem 3(c) on Homework 3 with $q = p$ to see that either $\cntr{P} = 1$ or $P$ is abelian.
        The former does not hold by part (\ref{5b}), hence $P$ is abelian.
      \item\label{5d}
        Observe that $\mathbb{Z}(P)$ is normal in $P$ and, by part (\ref{5b}), non-trivial.
        Since $P$ is simple, it must be the case that $\cntr{P} = P$ and thus abelian.
        Now observe that by Cauchy's Theorem, $P$ has an element of order $p$, say $\rho$, and thus a subgroup $\left<\rho\right> \cong \mathbb{Z}/p\mathbb{Z}$.
        Since $P$ is abelian, $\rho \unlhd P$.
        Moreover, since $P$ is simple and $\mathbb{Z}/p\mathbb{Z}$ is non-trivial, it follows that  $P = \left<\rho\right> \cong \mathbb{Z}/p\mathbb{Z}$.
      \item\label{5e}
        Let $H$ be a proper subgroup of $P$ of maximal order.
        Since $P$ is a $p$-group, $H < N_P(H)$.
        Moreover, by the maximality of $H$, we have $N_P(H) = P$.
        Consider the quotient, $P/H$.
        The Third Isomorphism Theorem implies that the quotient has no non-trivial subgroups and thus is simple.
        However, by Lagrange's Theorem and the fact that $H$ is a proper subgroup, $P/H$ must be a $p$-group.
        It then follows from part (\ref{5d}) that $P/H \cong \mathbb{Z}/p\mathbb{Z}$, hence $[P:H] = p$, as desired. 
    \end{enumerate}
  \end{proof}
\end{thm}

\end{document}
