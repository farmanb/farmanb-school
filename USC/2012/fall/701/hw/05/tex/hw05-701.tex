\documentclass[10pt]{amsart}
\usepackage{amsmath,amsthm,amssymb,amsfonts,enumerate,mymath,mathabx}
\openup 5pt
\author{Blake Farman\\University of South Carolina}
\title{Math 701:\\Homework 05}
\date{October 5, 2012}
\pdfpagewidth 8.5in
\pdfpageheight 11in
\usepackage[margin=1in]{geometry}

\begin{document}
\maketitle

\newtheorem{thm}{}
\newtheorem{lem}{Lemma}

\begin{thm}
  The commutator subgroup of $G$ is $G^{\prime} = [G,G] = \left\{[g_1,g_2] = g_1^{-1}g_2^{-1}g_1g_2 \mid g_1,g_2 \in G\right\}$.
  \begin{enumerate}[(a)]
  \item
    Let $N \unlhd G$.  Prove that $G/N$ is abelian if and only if $G^\prime \subseteq N$.
  \item
    Suppose that $P$ is a non-abelian group with $\abs{P} = p^3$.
    Prove that $\cntr{P} = P^\prime$.
  \end{enumerate}
  \begin{proof}
    \begin{enumerate}[(a)]
    \item\label{1a}
      Let $\overline{G} = G/N$ and let $\overline{g_1}, \overline{g_2} \in \overline{G}$ be given.
      Assume $\overline{G}$ is abelian.
      By assumption and the definition of multiplication in the quotient group, $$[g_1,g_2]N = \overline{g_1}^{-1}\overline{g_2}^{-1}\overline{g_1}\,\overline{g_2} = \overline{g_1}^{-1}\overline{g_1}\,\overline{g_2}^{-1}\overline{g_2} = \overline{1} = N.$$
      Since the choice of $g_1$ and $g_2$ was arbitrary, $[g_1,g_2] \in N$ implies $[G,G] \subseteq N$.
      
      Conversely, assume $[G,G] \subseteq N$.
      Then by assumption and the definition of multiplication in the quotient group, $$\overline{g_1}^{-1}\overline{g_2}^{-1}\overline{g_1}\,\overline{g_2} = [g_1,g_2]N = N = \overline{1}.$$
      Multiplying both sides by $\overline{g_2}\,\overline{g_1}$ on the left we obtain $\overline{g_1}\,\overline{g_2} = \overline{g_2}\,\overline{g_1}$.
      Hence $\overline{G}$ is abelian.
      Therefore $G/N$ is abelian if and only if $[G,G] \subseteq N$.
    \item
      First observe that by the previous homework set, $\cntr{P}$ is non-trivial. 
      By the assumption that $P$ is non-abelian, $P^\prime$ is not trivial and $\cntr{P}$ has order strictly smaller than $p^3$.
      Hence it follows from Lagrange's Theorem that $\cntr{P}$ has order either $p$ or $p^2$.
      Furthermore, if $\cntr{P}$ were to have order $p^2$, then $P/\cntr{P}$ would be cyclic and, by the third homework set, $P$ would be abelian.
      Since this would contradict the assumption that $P$ is non-abelian, $\cntr{P}$ has order $p$.
      
      Now consider the quotient of $P$ by $\cntr{P}$, which, by the argument above, has order $p^2$.
      By the previous homework set, all such $p$-groups are abelian.
      It now follows from (\ref{1a}) that $P^\prime$ is a non-trivial subgroup of a group of order $p$.
      Therefore $\cntr{P} = P^\prime$ follows from Lagrange's Theorem.
    \end{enumerate}
  \end{proof}
\end{thm}

\begin{thm}
  \begin{enumerate}[(a)]
    \item
      Suppose that $\abs{G} = 148$.  Prove that $G$ has a normal Sylow 37-subgroup.
    \item
      Suppose that $\abs{G} = 1452$.  Prove that $G$ is not simple.
    \item
      Suppose that $\abs{G} = 280$.  Prove that $G$ has a normal Sylow subgroup.
  \end{enumerate}
  \begin{proof}
    \begin{enumerate}[(a)]
    \item
      Observe that $148 = 2^2\cdot37$ implies $n_{37}(G)$ divides 4 and is congruent to 1 modulo 37.
      Hence, of possible choices $1$, $2$, or $4$, only $n_{37}(G) = 1$ satisfies both conditions.
      Therefore $G$ has a normal Sylow 37-subgroup.
    \item
      Observe that $1452 = 2^2\cdot3\cdot11^2$.
      Then $n_{11}(G)$ is one of $1, 2,3,4,6,$ and $12$, of which only $1$ and $12$ are congruent to 1 modulo 11.
      The choices for $n_2(G)$ and $n_3(G)$ are 1, 3, 11, 33, 121, 363 and 1, 2, 4, 11, 22, 44, 121, 242, and 484, respectively.
      Of the choices for $n_3(G)$, only 1, 4, 22, 121, and 484 are congruent to 1 modulo 3.
      If $n_{11}(G) = 1$, then the result follows immediately.  Hence it suffices to show that if $n_{11}(G) = 12$, then one of $n_2(G)$ and $n_3(G)$ is one.

      Suppose, by way of contradiction, that $n_{11}(G) = 12$ and $1 < n_2(G), n_3(G)$.
      Observe that any given Sylow 2-, 3-, or 11-subgroup has 3, 2, and 120 non-identity elements, respectively, and the Sylow $p$-subgroups are pairwise disjoint.
      Counting elements we have at least 9 non-identity elements in the Sylow 2-subgroups, at least 8 elements in the Sylow 3-subgroups, and 1440 non-identity elements in the Sylow 11-subgroups.
      However, this implies there are at least 1458 total elements in a group of order 1452, a contradiction.
      Hence at least one of $n_2(G)$ or $n_3(G)$ is 1.
      Therefore $G$ has a normal Sylow subgroup and is not simple.
    \item
      Observe that $280 = 2^3\cdot5\cdot7$.
      The possible choices for $n_5(G)$ are 1, 2, 4, 7, 8, 14, 28, and 56, of which only 1 and 56 are congruent to 1 modulo 5.
      The possible choices for $n_7(G)$ are 1, 2, 4, 5, 8, 10, 20, and 40, of which only 1 and 8 are congruent to 1 modulo 7.
      The possible choices for $n_2(G)$ are 1, 5, 7, and 35.
      If $n_5(G) = 1$, then the result follows immediately.
      Hence it suffices to show that if $n_5(G) = 56$, then one of $n_2(G)$ and $n_7(G)$ is one.
      
      Suppose, by way of contradiction, that $n_5(G) = 56$ and $1 < n_2(G), n_7(G)$.
      Observe that any given Sylow 2-, 5-, or 7-subgroup has 7, 4, and 6 non-identity elements, respectively, and the Sylow $p$-subgroups are pairwise disjoint.
      Counting elements we observe that there are at least 35 non-identity elements in the Sylow 2-subgroups, 224 non-identity elements in the Sylow 5-subgroups, and at least 48 non-identity elements in the Sylow 7-subgroups.
      This implies there are at least 308 total elements in a group of order 280, a contradiction.
      Hence one of $n_2(G)$ or $n_7(G)$ must be 1.
      Therefore $G$ has a normal Sylow $p$-subgroup.
    \end{enumerate}
  \end{proof}
\end{thm}
\begin{thm}
  Suppose that $p$ is prime with $p \divides \abs{G}$, let $P \in \Syl{p}{G}$, and let $N \unlhd G$.
  Prove that $P \cap N \in \Syl{p}{N}$.
  Deduce that $PN/N \in \Syl{p}{G/N}$.
  \begin{proof}
    Let $p^{\alpha}n = \abs{G}$, $p^\alpha = \abs{P}$, and $p^\beta m = \abs{N}$, where $n$ is some $p$-free integer, $m$ is a divisor of $n$ and $\beta \leq \alpha$.
    Since $P \cap N$ is a subgroup of $P$, Lagrange's Theorem implies $\abs{P \cap N}$ divides $p^\alpha$.
    Hence $\abs{P \cap N} = p^\gamma$ for some integer $\gamma \leq \beta$.
    Now observe that the normality of $N$ implies $PN$ is a subgroup of $G$ of order $$\abs{PN} = \frac{\abs{P}\abs{N}}{\abs{P \cap N}} = p^{\alpha + \beta - \gamma}m.$$
    Moreover, since $p^\alpha$ is the maximal $p$-power divisor of $\abs{G}$ and $\abs{PN}$ must divide $\abs{G}$, $\gamma = \beta$ must hold.
    Therefore $P \cap N$ is a Sylow $p$-subgroup of $N$.

    Finally, consider the quotient $G/N$, which is of order $p^{\alpha-\beta}n/m$.
    By the Diamond Isomorphism Theorem, we have $PN/N \cong  P/(P\cap N)$ and thus $\abs{PN/N} =  \abs{P/(P\cap N)} = p^{\alpha-\beta}$.
    Therefore $PN/N$ is a Sylow $p$-subgroup of $G/N$.
  \end{proof}
\end{thm}

\begin{thm}
  Suppose that $p$ is prime with $p \divides \abs{G}$, and suppose that $P \in \Syl{p}{G}$.
  Show that $N_G(N_G(P)) = N_G(P)$.
  \begin{proof}
    Let $H = N_G(N_G(P))$ and let $N = N_G(P)$.
    Observe that $P \leq N \leq H$ implies $\abs{P} \divides \abs{H}$.
    Furthermore, since $H \leq G$ and $\abs{P}$ is the maximal $p$-power dividing $\abs{G}$, $\abs{H} = \abs{P}m$, where $m$ is $p$-free.
    It follows from the definition that $P \in \Syl{p}{H}$.
    Let $P^\prime \in \Syl{p}{H}$ be given and observe that by exercise 3, $P^\prime \cap N \in \Syl{p}{N}$.
    However, $P \unlhd N$ implies $P$ is the only element of $\Syl{p}{N}$.
    Hence $P^\prime = P$ implies $\Syl{p}{H} = \left\{P\right\}$.
    Therefore $P \unlhd H$ implies $N = H$, as desired.
  \end{proof}
\end{thm}

\begin{thm}
  Suppose that $p$ is prime with $p \divides \abs{G}$. 
  Suppose that $P \in \Syl{p}{G}$, that $N$ is a subgroup of $G$, and that $P \subseteq N \unlhd G$.
  Prove that $n_p(N) = n_p(G)$.
  \begin{proof}
    First observe that $\abs{P}$ is the maximal $p$-power dividing $\abs{G}$ and $P \leq N$ implies, by Lagrange's Theorem, that $\abs{P} \divides \abs{N}$.
    Hence $\Syl{p}{N} \subseteq \Syl{p}{G}$, and thus it suffices to show the reverse containment.
    
    Towards that end, let $P^\prime \in \Syl{p}{G}$ be given.
    By exercise 3, $P^\prime \cap N \in \Syl{p}{N}$.
    By the observation above, every element of $\Syl{p}{N}$ has order equal to that of $P$.
    Hence $\abs{P^\prime \cap N}$ is a subgroup of $P^\prime$ of identical order, whence $P^\prime = P \cap N \leq N$, completing the reverse containment.
    %= \abs{P} = \abs{P^\prime}$ and $P^\prime \cap N \leq P^\prime$ implies $P^{\prime} = P^\prime \cap N$.
    Therefore $n_p(N) = n_p(G)$.

  \end{proof}
\end{thm}

\begin{thm}
  Suppose that $p$, $q$, and $r$ are primes with $p < q < r$ and $\abs{G} = pqr$.
  \begin{enumerate}[(a)]
    \item
      Let $\in \Syl{r}{G}$, and let $Q \in \Syl{q}{G}$.
      Prove that at least one of $R$ and $Q$ is normal in $G$.
    \item
      Conclude that $QR$ is a normal subgroup of $G$.
    \item
      Use the result of Problem 5 to conclude that $R$ is a normal subgroup of $G$.
    \item
      Suppose that $q \notdivides  r - 1$.
      \begin{enumerate}[(i)]
      \item
        Use the result of Problem 5 to show that $Q$ is a normal subgroup of $G$.
      \item
        Conclude also that $QR \cong \Z/qr\Z$.
      \end{enumerate}
  \end{enumerate}
  \begin{proof}
    \begin{enumerate}[(a)]
    \item
      Suppose $1 < n_1(G), n_r(G)$.
      Observe that by the Sylow Counting Theorem, $n_r(G) \in \left\{1, p, r, pr\right\}$ and $n_r(G) \in \left\{1, p, q, pq\right\}$.
      However, $p < q < r$ implies $q \notdivides p - 1$ and $r \notdivides q-1$.
      Hence $r \leq n_q(G)$ and $pq \leq n_r(G)$.
      Now observe that elements of $\Syl{q}{G}$ have $q-1$ elements of order $q$ and elements of $\Syl{r}{G}$ have $r - 1$ elements of order $r$.
      Counting these elements, we have at least $r(q-1) + pq(r-1)$ non-identity elements in $G$.
      Using that $q - 1 \geq p$ and $pr - pq > 0$, we have $$r(q-1) + pq(r-1) \geq pqr + pr - pq > prq,$$
      a contradiction.
      Therefore, one of $Q$ and $R$ must be normal in $G$.
    \item
      Observe that because one of $Q$ and $R$ is normal in $G$, $QR$ is a subgroup of $G$.
      Let $t$ be either $q$ or $r$, according to whichever is normal, and let $N$ be the normal Sylow $t$-subgroup. 
      Now consider the quotient $G/N$, which has order $p(qr / t)$.
      By exercise 3, $QR/N \in \Syl{qr/t}{G/N}$ and, as was shown in class, $qr/t > p$ implies $n_{qr/t}(G/N) = 1$.
      Hence $QR/N \unlhd G/N$, and the result follows from the Lattice Isomorphism Theorem.
    \item
      Observe that since $QR$ is a group of order $qr$, the same result cited above implies $n_r(QR) = 1$.
      Then by exercise 5, $n_r(G) = n_r(QR) = 1$ implies $R$ is normal in $G$.
    \item
      \begin{enumerate}[(i)]
        \item
          By the Sylow Counting Theorem, $n_q(QR) \in \left\{1,r\right\}$.
          However, since $q$ divides $n_q(QR) - 1$, but does not divide $r-1$, we have $n_q(QR) = 1$.
          Hence the result of exercise 5 implies $n_q(G) = n_q(QR) = 1$.
          Therefore $Q \unlhd G$.
        \item
          The result follows directly from the Theorem proven in class.  Namely, if $q < r$ are primes, $\abs{QR} = qr$, and $q \notdivides r-1$, then $QR \cong \Z/qr\Z$.
      \end{enumerate}
    \end{enumerate}
  \end{proof}
\end{thm}

\end{document}
