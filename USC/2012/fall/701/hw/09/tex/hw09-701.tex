\documentclass[10pt]{amsart}
\usepackage{../../../../../../../tex/mymath/mymath}
\usepackage{amsmath,amsthm,amssymb,amsfonts,enumerate}
\openup 5pt
\author{Blake Farman\\University of South Carolina}
\title{Math 701:\\Homework 09}
\date{November 16, 2012}
\pdfpagewidth 8.5in
\pdfpageheight 11in
\usepackage[margin=1in]{geometry}

\begin{document}
\maketitle

%Field names
\newcommand{\Z}{\mathbb{Z}}
\newcommand{\R}{\mathbb{R}}
\newcommand{\Q}{\mathbb{Q}}
\newcommand{\C}{\mathbb{C}}
\newcommand{\F}{\mathbb{F}}
\newcommand{\N}{\mathbb{N}}
\newcommand{\uhp}{\mathfrak{h}}
\newcommand{\quat}{\mathbb{H}}

%Operator names
\newcommand{\ord}{\operatorname{ord}}
\newcommand{\Det}{\operatorname{Det}}
\newcommand{\Gal}{\operatorname{Gal}}
\newcommand{\Inn}[1]{\operatorname{Inn}\left(#1\right)}
\newcommand{\Aut}[1]{\operatorname{Aut}\left(#1\right)}
\newcommand{\real}[1]{\operatorname{\mathfrak{Re}}\left(#1\right)}
\newcommand{\imag}[1]{\operatorname{\mathfrak{Im}}\left(#1\right)}
\newcommand{\Syl}[2]{\operatorname{Syl}_{#1}\left(#2\right)}
\newcommand{\SL}[2]{\operatorname{SL}_#1\left(#2\right)}
\newcommand{\GL}[2]{\operatorname{GL}_#1\left(#2\right)}
\newcommand{\M}[2]{\operatorname{M}_#1\left(#2\right)}
\newcommand{\PSL}[2]{\operatorname{PSL}_#1\left(#2\right)}
\newcommand{\Mat}[2]{\operatorname{Mat}_{#1}\left(#2\right)}
\providecommand{\norm}[1]{\lVert#1\rVert}
\newcommand{\dist}[2]{\operatorname{dist}\left(#1,#2\right)}
\newcommand{\cntr}[1]{\mathbf{Z}\left(#1\right)}
\newcommand{\abs}[1]{\left| #1 \right|}
\newcommand{\orbit}[1]{\mathcal{O}_{#1}}
\newcommand{\card}[1]{\operatorname{card}#1}

%\renewcommand{\phi}{\varphi}
\renewcommand{\qedsymbol}{\(\blacksquare\)}
\renewcommand{\epsilon}{\varepsilon}

\newtheorem{thm}{}
\newtheorem{lem}{Lemma}

%\begin{lem}\label{units}
%  Let $R$ be a ring.  If $u$ is a unit, then $u$ is not a zero divisor.
%  \begin{proof}
%    Suppose for some $r \in R$ that $ur = 0$.
%    Then by associativity we have 
%    $$r = (u^{-1}u)r = u^{-1}(ur) = u^{-1}\cdot 0 = 0.$$
%    Similarly, if $ru = 0$, then
%    $$r = r(uu^{-1}) = (ru)u^{-1} = 0 \cdot u^{-1} = 0.$$
%    Therefore $u$ is not a zero divisor, as desired.
%  \end{proof}
%\end{lem}
\begin{thm}
  Let $R$ be a commutative ring.
  Define the set $R[[x]]$ of {\bf formal power series} in the indeterminate $x$ with coefficients from $R$ to be all formal infinite sums
  $$\sum_{n=0}^\infty a_nx^n = a_0 + a_1x + a_2 x^2 + a_3x^3 + \ldots.$$
  Define addition and multiplication in $R[[x]]$ by
  $$\sum_{n=0}^\infty a_nx^n + \sum_{n=0}^\infty b_mx^m = \sum_{k=0}^\infty (a_k + b_k)x^k,\; \left(\sum_{n=0}^\infty a_nx^n\right) \cdot \left(\sum_{m=0}^\infty b_mx^m\right) = \sum_{j=0}^\infty\left(\sum_{n=0}^j a_nb_{j-n}\right)x^j.$$
  \begin{enumerate}[(a)]
  \item
    Prove that $R[[x]]$ is a commutative ring.
    (Associativity of multiplication is a bit tedious, so you may assume it.)
  \item
    Show that $1-x$ is a unit in $R[[x]]$ with inverse $1 + x + x^2 + \ldots$.
  \item
    Prove that $\sum_{n=0}^\infty a_nx^n$ is a unit in $R[[x]]$ if and only if $a_0$ is a unit in $R$.
  \item
    Prove that if $R$ is an integral domain, then so is $R[[x]]$.
  \end{enumerate}

  \begin{proof}
    \begin{enumerate}[(a)]
    \item
      Let $\sum_{i=0}^\infty a_ix^i$, $\sum_{j=0}^\infty b_jx^j$, $\sum_{k=0}^\infty c_kx^k \in R[[x]]$ be given.
      To see that $(R[[x]], +)$ is a group, first note that closure under addition and
      $$\sum_{i=0}^\infty a_ix^i + \sum_{j=0}^\infty b_jx^j = \sum_{k=0}^\infty (a_k + b_k)x^k = \sum_{k=0}^\infty (b_k + a_k)x^k = \sum_{j=0}^\infty b_jx^j + \sum_{i=0}^\infty a_ix^i,$$
      both follow from the fact that $(R, +)$ is an additive abelian group.
      Define the series $0 = \sum_{n=0}^\infty 0_R$.
      Then it follows from
      $$\sum_{i=0}^\infty a_ix^i + \sum_{j=0}^\infty 0_R\cdot x^i = \sum_{i=0}^\infty (a_i + 0_R)x^i = \sum_{i=0}^\infty a_ix^i = \sum_{i=0}^\infty (0_R + a_i)x^i = \sum_{j=0}^\infty 0_R\cdot x^j + \sum_{i=0}^\infty a_ix^i.$$
      that $0$ is the additive identity.
      Moreover, $R[[x]]$ is closed under additive inverses with
      $$\sum_{i=0}^\infty a_ix^i + \sum_{i=0}^\infty -a_ix^i = \sum_{i=0}^\infty (a_i - a_i)x^i = \sum_{i=0}^\infty 0\cdot x^i = 0.$$
      To see that $R[[x]]$ is associative, consider 
      \begin{equation}\label{1.0}
        \sum_{i=0}^\infty a_ix^i + \left(\sum_{j=0}^\infty b_jx^j + \sum_{k=0}^\infty c_kx^k\right)  = \sum_{i=0}^\infty a_ix^i + \sum_{j=0}^\infty (b_j + c_j)x^j = \sum_{i=0}^\infty \left(a_i + (b_i + c_i)\right)x^i
      \end{equation}
      Since $(R,+)$ is an additive group, $a_i + (b_i + c_i) = (a_i + b_i) + c_i$ holds for all $i$.
      Therefore by \eqref{1.0}
      $$\sum_{i=0}^\infty a_ix^i + \left(\sum_{j=0}^\infty b_jx^j + \sum_{k=0}^\infty c_kx^k\right) = \sum_{i=0}^\infty \left((a_i + b_i) + c_i\right)x^i= \left(\sum_{i=0}^\infty a_ix^i + \sum_{j=0}^\infty b_jx^j\right) + \sum_{k=0}^\infty c_kx^k.$$
      
      Define the series $1 = \sum_{n=0}^\infty e_n$ where $e_0 = 1_R$ and $e_n = 0$ for all $n \geq 2$.  
      Consider the finite sums 
      $$\sum_{j=0}^i a_j e_{i-j}\quad \text{and}\quad \sum_{j=0}^i e_j a_{i-j}.$$
      It follows from the definition of $1$ that their values are  $a_ie_{i-i} = a_i$ and $e_0a_{i-0} = a_i$, respectively.
      Hence 
      $$\sum_{i=0}^\infty a_ix^i \cdot 1 = \sum_{i=0}^\infty \left(\sum_{j=0}^i a_j e_{i-j}\right)x^i  = \sum_{i=0}^\infty a_i =\sum_{i=0}^\infty \left(\sum_{j=0}^i e_j a_{i-j}\right)x^i = 1 \cdot \sum_{i=0}^\infty a_ix^i$$
      implies that $1$ is the multiplicative identity in $R[[x]]$.
      Finally, consider the product
      \begin{equation}\label{1.1}
        \sum_{i=0}^\infty a_ix^i \cdot \left(\sum_{j=0}^\infty b_jx^j + \sum_{k=0}^\infty c_kx^k \right) = \sum_{i=0}^\infty a_ix^i \cdot \sum_{j=0}^\infty (b_j + c_j)x^j = \sum_{i=0}^\infty \left(\sum_{j=0}^i a_j(b_{i-j} + c_{i-j})\right)x^i.
      \end{equation}
      Since $R$ is a ring, $a_i(b_{i-j} + c_{i-j}) = a_ib_{i-j} + a_ic_{i-j}$ holds for all $i,j$ and so it follows from \eqref{1.1} that
      \begin{equation}\label{1.2}
        \sum_{i=0}^\infty a_ix^i \cdot \left(\sum_{j=0}^\infty b_jx^j + \sum_{k=0}^\infty c_kx^k \right) = \sum_{i=0}^\infty \left(\sum_{j=0}^i a_jb_{i-j} +  \sum_{k=0}^i a_kc_{i-k}\right)x^i
      \end{equation}
      Then we have from the definition of addition and multiplication
      \begin{eqnarray}
        \sum_{i=0}^\infty a_ix^i \cdot \sum_{j=0}^\infty b_jx^j + \sum_{i=0}^\infty a_ix^i \cdot \sum_{k=0}^\infty c_kx^k &=& \sum_{i=0}^\infty \left(\sum_{j=0}^i a_jb_{i-j}\right)x^i +  \sum_{i=0}^\infty \left(\sum_{k=0}^i a_{k}c_{i-k}\right)x^i\\\label{1.3}
        &=& \sum_{i=0}^\infty \left(\sum_{j=0}^i a_jb_{i-j} +  \sum_{k=0}^i a_kc_{i-k}\right)x^i.\label{1.4}
      \end{eqnarray}
      From \eqref{1.2} and \eqref{1.4} it follows that
      $$\sum_{i=0}^\infty a_ix^i \cdot \left(\sum_{j=0}^\infty b_jx^j + \sum_{k=0}^\infty c_kx^k \right) = \sum_{i=0}^\infty a_ix^i \cdot \sum_{j=0}^\infty b_jx^j + \sum_{i=0}^\infty a_ix^i \cdot \sum_{k=0}^\infty c_kx^k.$$
      Since $R$ is commutative, distribution on the right follows by the same argument, mutatis mutandis.
      Moreover, observing that since $R$ is commutative, $a_i(b_{i-j} + c_{i-j}) = (b_{i-j} + c_{i-j})a_i$ and hence
      $$\sum_{i=0}^\infty a_ix^i \cdot \left(\sum_{j=0}^\infty b_jx^j + \sum_{k=0}^\infty c_kx^k \right) = \left(\sum_{j=0}^\infty b_jx^j + \sum_{k=0}^\infty c_kx^k \right) \cdot \sum_{i=0}^\infty a_ix^i.$$
      Therefore $R[[x]]$ is a commutative ring.
    \item
      First consider $1 - x$ as the series $\sum_{n=0}^\infty a_n$ where $a_0 = 1_R$, $a_1 = -1_R$ and $a_n = 0$ for all $n \geq 2$.
      %Consider the finite sums 
      %$$\sum_{i=0}^n a_i$$
      Then 
      $$\sum_{i=0}^\infty a_i \cdot \sum_{j = 0}^\infty x^j = \sum_{i=0}^\infty\left(\sum_{j=0}^i a_j\right)x^i = \sum_{i=0}^\infty\left(\sum_{j=0}^i a_{i-j}\right)x^i = \sum_{j = 0}^\infty x^j \cdot \sum_{i=0}^\infty a_i$$
       follows from the definition of multiplication and that $(R, +)$ is abelian.
      Observing that for $i > 0$
      \begin{equation*}
        \sum_{j=0}^i a_j = a_0 + a_1 = 0
      \end{equation*}
      it follows that $\sum_{i=0}^\infty\left(\sum_{j=0}^i a_j\right) = 1_R + 0\cdot x + 0 \cdot x^2 + \ldots = 1$.
      Therefore $$(1 - x) \cdot \sum_{n = 0}^\infty x^n = \sum_{n = 0}^\infty x^n \cdot (1 - x) = 1,$$ as desired.
    \item
      Let $\sum_{i=0}^\infty a_ix^i \in (R[[x]])^\times$ be given and let $\sum_{j=0}^\infty b_jx^j$ be its inverse.
      Then by the definition of multiplication it follows that
      $$\sum_{i=0}^\infty a_ix^i \cdot \sum_{j=0}^\infty b_jx^j = \sum_{i=0}^j\left(\sum_{j=0}^i a_jb_{i-j}\right) = 1_R + 0\cdot x + 0\cdot x^2 + \ldots = \sum_{i=0}^j\left(\sum_{j=0}^i b_j a_{i-j}\right) = \sum_{j=0}^\infty b_jx^j \cdot \sum_{i=0}^\infty a_ix^i.$$
      Therefore $1_R = a_0b_0$ implies $a_0$ and $b_0$ are units in $R$.
      
      Conversely, assume $\sum_{i=0}^\infty a_ix^i$ is a series with $a_0 \in R^\times$.
      Define the sequence $\left\{b_n\right\}_{n=0}^\infty$ by $b_0 = a_0^{-1}$, and $b_n = -a_0^{-1}\left(\sum_{i=0}^{n-1} b_ia_{n-i}\right)$ for each $n \geq 1$.
      Fix $i > 0$ and consider the finite sum
      \begin{eqnarray*}
        \sum_{j=0}^i b_ja_{i-j} = \sum_{j=0}^{i-1} b_ja_{j-i} + b_ia_0 &=& \sum_{j=0}^{i-1} b_ja_{i-j} + a_0(-a_0^{-1})\left(\sum_{j=0}^{i-1} b_ja_{i-j}\right)\\
        &=& \sum_{j=0}^{i-1} b_ja_{i-j} - \left(\sum_{j=0}^{i-1} b_ja_{i-j}\right)\\
        &=& 0.
      \end{eqnarray*}
      Note that when $i = 0$ we have $b_0a_0 = 1$ by construction.
      Hence 
      $$\sum_{j=0}^\infty a_jx^j\sum_{i=0}^\infty b_ix^i = \sum_{i=0}^\infty b_ix^i \sum_{j=0}^\infty a_jx^j = \sum_{i=0}^\infty \left(\sum_{j=0}^i b_ja_{i-j}\right)x^i = 1_R + 0\cdot x + 0\cdot x^2 + \ldots = 1.$$
      Therefore $\sum_{i=0}^\infty a_ix^i$ is invertible, as desired.
    \item
      We proceed by the contrapositive.
      Suppose that $R[[x]]$ is not an integral domain.
      Then there exist non-zero series $\sum_{i=0}^\infty a_ix^i$ and $\sum_{j=0}^\infty b_jx^j$ such that their product is zero.
      Let $k$ and $\ell$ be the smallest integers such that $a_k \neq 0$ and $b_\ell \neq 0$.
      Let $n = k + \ell$ and consider the coefficient of $x^n$.
      Observe that if $j < k$, then $a_j = 0$ and if $j > k$, then $n - j < \ell$ implies $b_{n-j} = 0$.
      Hence
      $$0 = \sum_{j=0}^{n} a_jb_{n-j} = \sum_{j=0}^{k-1} a_jb_{n-j} + a_kb_\ell + \sum_{j=k+1}^n a_jb_{n-j} = a_kb_\ell.$$
      Therefore, since both $a_k$ and $b_\ell$ are assumed to be non-zero elements of $R$, we have that $R$ is not an integral domain.
    \end{enumerate}
  \end{proof}
\end{thm}

\begin{thm}
  Let $K$ be a field.
  A {\bf discrete valuation} on $K$ is a function $v \colon K^\times \rightarrow \Z$ satisfying
  \begin{enumerate}[(i)]
  \item
    $v(ab) = v(a) + v(b)$.
    (i.e. $v$ is a homomorphism from the group $(K^\times, \times)$ to $(\Z, +)$),
  \item
    $v$ is surjective,
  \item
    $v(x + y) \geq \min \left\{v(x), v(y)\right\}$ for all $x,y \in K^\times$ with $x + y \neq 0$
  \end{enumerate}
  The set $R = \left\{x \in K^\times \;\middle\vert\; v(x) \geq 0\right\} \cup \left\{0\right\}$ is the {\bf valuation ring} of $v$.
  \begin{enumerate}[(a)]
  \item
    Prove that $R$ is a subring of $K$. (In general, a ring $R$ is called a {\bf discrete valuation ring} if there is some field $K$ and some discrete valuation $v$ on $K$ such that $R$ is the valuation ring of $v$.)
  \item
    Prove that for each non-zero element $x \in K$, either $x$ or $x^{-1}$ is in $R$.
  \item
    Prove that an element $x$ is a unit of $R$ if and only if $v(x) = 0$.
  \end{enumerate}
  \begin{proof}
    \begin{enumerate}[(a)]
    \item
      Let $x,y,z \in R$ be given.
      We first show that $(R, +)$ is an abelian group.
      It suffices to show that $x - y \in R$.
      If $x - y = 0$, then $x - y \in R$ holds by definition, so it suffices to assume $x - y \neq 0$.
      Also note that since $K$ is a field, $0 \neq x - y \in K^\times$.
      Observe for $a \in K$ that 
      $$v(a) = v(a \cdot 1_K) = v(a) + v(1_K) = v(1_K) + v(a)  = v(1_K \cdot a) = v(a)$$ 
      implies $v(1_K) = 0$.
      Furthermore that $0 = v(1) = v((-1)(-1)) = v(-1) + v(-1)$ implies $v(-1) = -v(-1)$ and thus $v(-1) = 0$.
      Hence
      \begin{equation}\label{2.1}
        v(-y) = v(-1 \cdot y) = v(-1) + v(y) = v(y).
      \end{equation}
      It now follows from \eqref{2.1} and the definition of $R$ that
      $$v(x + (-y)) \geq \min\left\{v(x), v(-y)\right\} = \min\left\{v(x), v(y)\right\} \geq 0$$
      and thus $(R, +)$ is a group.
      That $(R, +)$ is abelian follows from $K$ being a field.
      
      To see that $R$ is closed under multiplication, consider $z(x + y)$.
      Note that $R$ inherits commutativity, associativity and distribution from $K$.
      If either $z = 0$ or $(x+y) = 0$, then $z(x + y) = 0 \in R$ by definition.
      Assume both are non-zero.
      Since $K$ is a field, $z(x + y) \neq 0$.
      Since $(R, +)$ is a group, we have $(x + y) \in R$ and       
      $$v(zx + zy) = v(z(x+y)) = v(z) + v(x + y) \geq 0$$
      since $v(z), v(x + y) \geq 0$ holds by definition of $R$.
      Therefore $z(x + y) \in R$ and $R$ is a subring of $K$.
    \item
      Let $x \in K^\times$ be given.
      Observe that since $v$ is a homomorphism we have $v(x^{-1}) = -v(x)$.
      If $v(x) = 0$, then both $x \in R$ and $x^{-1} \in R$ hold.
      If $v(x) \neq 0$, then either $v(x)$ or $-v(x)$ is positive.
      Therefore either $x \in R$ or $x^{-1} \in R$.
    \item
      Assume that $u$ is a unit in $R$.
      Suppose by way of contradiction that $v(u) > 0$.
      Then $v(u^{-1}) = -v(u) < 0$ implies $u^{-1} \not \in R$, a contradiction.
      Therefore $v(u) = 0$.
      
      Conversely, assume $u \in R$ and $v(u) = 0$.
      Then $v(u^{-1}) = -v(0) = 0$ implies $u^{-1} \in R$.
      Therefore $u$ is a unit.
    \end{enumerate}
  \end{proof}
\end{thm}

\begin{thm}
  A specific example of a discrete valuation ring is obtained when $p$ is prime, $K = \Q$, and 
  $$v_p \colon \Q^\times \rightarrow \Z,\, \text{by}\  v_p\left(\frac{a}{b}\right) = \alpha,\, \text{where}\  \frac{a}{b} = p^\alpha\frac{c}{d},\, p \nmid c\ \text{and}\ p \nmid d.$$
  \begin{enumerate}[(a)]
  \item
    Prove that the corresponding valuation ring is 
    $$R = \left\{\frac{a}{b} \in \Q \;\middle\vert\; \gcd(a,b) = 1,\, \text{and}\ p\nmid b\right\}.$$
  \item
    Describe $R^\times$.
  \end{enumerate}
  \begin{proof}
    To see that $v_p$ is a discrete valuation on $K$, let $\frac{a}{b}, \frac{c}{d} \in \Q^\times$ be given.
    Write $a = p^{\alpha_1}m$, $b = p^{\alpha_2}n$, $c = p^{\alpha_3}k$, and $d = p^{\alpha_4}\ell$, where $m$, $n$, $k$, and $\ell$ are all $p$-free integers.
    Then 
    $$v_p\left(\frac{ac}{bd}\right) = v_p\left(p^{\alpha_1 - \alpha_2 + \alpha_3 - \alpha_4}\frac{mk}{n\ell}\right) = (\alpha_1 - \alpha_2) + (\alpha _3  - \alpha_4) = v_p\left(\frac{a}{b}\right) + v_p\left(\frac{c}{d}\right).$$
    Since $p^n \in \Q^\times$ for all $n \in \Z$, $v_p(p^n) = n$ implies $v_p$ is surjective.
    Now note that since $\Z/p\Z$ is a field and $n$, $m$, $k$, $\ell$ were all chosen to be $p$-free, the products $n\ell$, $km$, and $m\ell$ are not divisible by $p$.
    Hence if we let $M = \min\left\{\alpha_1 + \alpha_4, \alpha_2 + \alpha_3\right\}$, then
    \begin{equation}\label{3.1}
      \frac{a}{b} + \frac{c}{d} = \frac{ad + cb}{bd} = \frac{p^{\alpha_1 + \alpha_4}n\ell + p^{\alpha_2 + \alpha_3}km}{p^{\alpha_2 + \alpha_4}m\ell} = p^{M - \alpha_2 - \alpha_4}\left(\frac{p^{\alpha_1 + \alpha_4 - M}n\ell + p^{\alpha_2 + \alpha_3 - M}km}{m\ell}\right).
    \end{equation}
    Since one of $\alpha_1 + \alpha_4 - M$ and $\alpha_2 + \alpha_3 - M$ is zero, the numerator in the right-hand side of \eqref{3.1} is not divisible by $p$.
    Hence $v_p\left(\frac{a}{b} + \frac{c}{d}\right) = M - \alpha_2 - \alpha_3$.
    Moreover, note that 
    $$M - \alpha_2 - \alpha_3 = \left\{
    \begin{array}{ll}
      \alpha_1 - \alpha_2 & \text{if}\ M = \alpha_1 + \alpha_4,\\
      \alpha_3 - \alpha_4 & \text{if}\ M = \alpha_2 + \alpha_3\\
    \end{array}
    \right.$$ implies $v_p\left(\frac{a}{b} + \frac{c}{d}\right) \geq \min\left\{\alpha_1 - \alpha_2, \alpha_3 - \alpha_4\right\} = \min\left\{v_p\left(\frac{a}{b}\right), v_p\left(\frac{c}{d}\right)\right\}$.
    Therefore $v_p$ is a discrete valuation on $K$.
    \begin{enumerate}[(a)]
      \item
        By definition, $R = \left\{\frac{a}{b} \in \Q \;\middle\vert\; v_p\left(\frac{a}{b}\right) = p^\alpha,\, \alpha > 0\right\}$.
        Let $\frac{a}{b} \in R$, where $a,b \in \Z$, be given and write $a = p^{\alpha_1}m$, $b = p^{\alpha_2}n$, where $m$ and $n$ are $p$-free.
        Since $v_p(\frac{a}{b}) = \alpha_1 - \alpha_2 \geq 0$, we may let $d = \gcd(m,n)$, $\alpha = \alpha_1 - \alpha_2$, and rewrite, for some $k, \ell \in \Z$, 
        $$\frac{a}{b} = p^{\alpha}\frac{dk}{d\ell} = p^{\alpha}\frac{k}{\ell} = \frac{p^{\alpha}k}{\ell}.$$
        Then we have $\gcd(k,\ell) = 1$ and $p \nmid \ell$.
        Therefore $R = \left\{\frac{a}{b} \in \Q \;\middle\vert\; \gcd(a,b) = 1,\, \text{and}\ p\nmid b\right\}$, as desired.
      \item
        Let $\frac{a}{b} \in R^\times$ be given.
        By part (a) we have that $\frac{a}{b} = \frac{p^\alpha k}{\ell}$ for some $k, \ell \in \Z$.
        Moreover, by part (c) of Exercise 2, we have that $v(\frac{a}{b}) = \alpha = 0$
        Hence $R^\times = \left\{\frac{a}{b} \in \Q \;\middle\vert\; \gcd(a,b) = 1,\, \text{and}\ p \nmid a,\, p\nmid b \right\}$.
    \end{enumerate}
  \end{proof}
\end{thm}

\begin{thm}
  Let $F$ be a field, and define the ring $$F((x)) = \left\{ \sum_{n=N}^\infty a_nx^n \;\middle\vert\; a_n \in F\ \text{and}\ N \in \Z\right\}.$$
  (Every element of $F((x))$ is a power series in $x$ plus a polynomial, possibly 0, in $1/x$.  
  I.e., each element of $F((x))$ has only finitely many terms with negative powers of $x$.)
  \begin{enumerate}[(a)]
  \item
    Prove that $F((x))$ is a field.
  \item
    Define the map 
    $$v \colon F((x)) \rightarrow \Z\ \text{by}\ v\left(\sum_{n=N}^\infty a_nx^n\right) = N$$
    where $a_N$ is the first non-zero coefficient in the series (i.e., $N$ is the ``order of zero or pole of the series at 0'').
    Prove that $v$ is a discrete valuation on $F((x))$ whose discrete valuation ring is $F[[x]]$, the ring of formal power series.
  \end{enumerate}
  \begin{proof}
    \begin{enumerate}[(a)]
    \item
      %Let $\sum_{n=N}^\infty a_nx^n$ and $\sum_{n=M}^\infty b_nx^n$ be given.
      %Let $S = \min\left\{N, M, L\right\}$.
      % and rewrite the sums as
      %$$\sum_{n=S}^\infty a_nx^n = \sum_{n=S}^{N-1} 0_F + \sum_{n=N}^\infty a_nx^n,\,\sum_{n=S}^\infty b_nx^n = \sum_{n=S}^{M-1} 0_F + \sum_{n=M}^\infty b_nx^n,\,  \text{and}\ \sum_{n=S}^\infty c_nx^n = \sum_{n=S}^{L-1} 0_F + \sum_{n=L}^\infty c_nx^n.$$
      %Use the addition
      %$$\sum_{n=N}^\infty a_nx^n + \sum_{n=M}^\infty b_nx^n = \sum_{n=S}^\infty (a_n + b_n)x^n,$$ 
      %where $n < N$ and $n < M$ imply $a_n = 0$ and $b_n = 0$, respectively, and multiplication
      %$$\sum_{n=N}^\infty a_nx^n \sum_{n=M}^\infty b_nx^n = \sum_{n=N+M}^\infty \left(\sum_{i + j = n} a_ib_j\right)x^n.$$
      %It remains only to show that every element $0 \neq \sum_{n=N}^\infty a_nx^n$ of $F((x))$ is invertible.
      %Let $0 \neq \sum_{n=N}^\infty a_nx^n \in F((x))$ be given.
      Since $F((x))$ is a ring, it suffices to show that every element $0 \neq \sum_{n=N}^\infty a_nx^n$ of $F((x))$ is invertible.
      %Assume that $N < 0$.
      Consider $x^{-N}$ as a series with terms $b_n = 0$ for all $n \neq -N$ and $b_{-N} = 1$.
      Then by construction
      $$c_n = \sum_{i + j = n} a_ib_j = a_{n+N}b_{-N} = a_{n+N}.$$
      If $N \leq n < 0$, then it follows that $n + N < N$ and thus $c_n = 0$ for all $n < 0$.
      Similarly, if $n < 0 < N$, then it follows that $n + N < N$ and thus $c_n = 0$ for all $n < 0$.
      Hence we have $$x^{-N} \sum_{n=N}^\infty a_nx^n = \sum_{n=0}^\infty c_nx^n.$$
      Since $c_0 = a_N$ is a unit, the series $\sum_{n=0}^\infty c_nx^n$ is invertible using the same method as in part (c) of Exercise 1.
      Hence
      $$\left(\left(\sum_{n=0}^\infty c_nx^n\right)^{-1}x^{-N}\right) \sum_{n=N}^\infty a_nx^n = 1$$
      and $\sum_{n=N}^\infty a_nx^n$ is invertible, as desired.
      Therefore $F((x))$ is a field.
    \item
      Let $\sum_{n=N}^\infty a_nx^n$ and $\sum_{n=M}^\infty b_nx^n \in F((x))$ be given.
      First consider $$\sum_{n=N}^\infty a_nx^n \cdot \sum_{n=M}^\infty b_nx^n = \sum_{n=N+M}^\infty \left(\sum_{i + j = n} a_ib_j\right)x^n.$$
      Then we have
      \begin{eqnarray*}
        v\left(\sum_{n=N}^\infty a_nx^n \cdot \sum_{n=M}^\infty b_nx^n\right) &=& v\left(\sum_{n=N+M}^\infty \left(\sum_{i + j = n} a_ib_j\right)x^n\right)\\
        &=& N + M\\
        &=& v\left(\sum_{n=N}^\infty a_nx^n\right) + v\left(\sum_{n=M}^\infty b_nx^n\right)
      \end{eqnarray*}
      To see that $v$ is surjective, for each $N \in \Z$, construct a sequence $\sum_{n=N}^\infty a_nx^n$ with $a_N \neq 0_F$ and all other $a_n = 0_F$, so that $v(\sum_{n=N}^\infty a_nx^n) = N$.
      Finally, we have 
      \begin{eqnarray*}
        v\left(\sum_{n=N}^\infty a_nx^n + \sum_{n=M}^\infty b_nx^n\right) &=& v\left(\sum_{n=\min\{N,M\}}(a_n + b_n)x^n\right)\\
        &=& \min\{N,M\}\\
        &=& \min\left\{v\left(\sum_{n=N}^\infty a_nx^n\right), v\left(\sum_{n=M}^\infty b_nx^n \right)\right\}.
      \end{eqnarray*}
      Therefore $v$ is a discrete valuation on $F((x))$ with discrete valuation ring
      \begin{eqnarray*}
        R &=& \left\{\sum_{n=N}^\infty a_nx^n \in F^\times \;\middle\vert\; v\left(\sum_{n=N}^\infty a_nx^n\right) \geq 0 \right\} \cup \left\{0\right\}\\
        &=& \left\{\sum_{n=N}^\infty a_nx^n \in F^\times \;\middle\vert\; N \geq 0\right\} \cup \left\{0\right\}\\
        &=& \left\{\sum_{n=0}^\infty a_nx^n \;\middle\vert\; a_n \in F \right\}\\
        &=& F[[x]].
      \end{eqnarray*}
      
    \end{enumerate}
  \end{proof}
\end{thm}

\begin{thm}
  Let $\quat = \Z + \Z i + \Z j + \Z k$ be the ring of na\"{i}ve integral Hamilton Quaternions and define $N \colon S \rightarrow \Z$ by $N(a + bi + cj + dk) = a^2 + b^2 + c^2 + d^2$.
  (This map is called a {\it norm}.)
  \begin{enumerate}[(a)]
  \item
    Suppose that $\alpha = a + bi + cj + dk \in S$.
    Define $\overline{\alpha} = a - bi - cj - dk$.
    Prove that $N(\alpha) = \alpha \overline{\alpha}$.
  \item
    For all $\alpha, \beta \in S$ prove that $N(\alpha\beta) = N(\alpha)N(\beta)$.
    (We say that the norm is multiplicative.)
  \item
    Let $u \in S$.
    Prove that $u \in S^\times$ if and only if $N(u) = 1$.
    Show that $S^\times$ is a non-abelian group of order 8.
    (This is the ``quaternion group.''
    Note that the inverse of $a \neq 0$ in the ring of rational quaternions, $\quat(\Q)$ (replace $\Z$ by $\Q$ in the definition), is $\frac{\overline{\alpha}}{N(\alpha)}$.)
  \end{enumerate}
  \begin{proof}
    \begin{enumerate}[(a)]
    \item
      Using the identities $ij = -ji = k$, $jk = -kj = i$, $ki = -ik = j$, and $i^2 = j^2 = k^2 = -1$, distribute and collect like terms to obtain
      \begin{eqnarray*}
        \alpha\overline{\alpha} &=& (a^2 + b^2 + c^2 + d^2) + (-ab + ab - cd + dc)i + (-ac + ca -db + bd)j + (-ad + da -bc + cb)k\\
        &=& a^2 + b^2 + c^2 + d^2\\
        &=& N(\alpha).
      \end{eqnarray*}
      \item
        Let $\alpha = a + bi + cj + dk, \beta = e + fi + gj + hk \in \quat$ be given.
        Using the identities $ij = -ji = k$, $jk = -kj = i$, $ki = -ik = j$, and $i^2 = j^2 = k^2 = -1$, distribute and collect like terms to obtain
        \begin{equation}\label{5.1}
          \alpha\beta = (ae - bf - cg -dh) + (af + be + ch -dg) + (ag - bh + ce + df)j + (ah + bj - cf + de)k.
        \end{equation}
        Consider the product $\overline{\beta}\overline{\alpha}$.
        Using \eqref{5.1}, we have 
        \begin{eqnarray*}
          \overline{\beta}\overline{\alpha} &=& (ae - bf - cg - dh) + (-af -be -ch + dg)i + (-ag + bh - ce - df)j + (-ah -bg + cf -de)k\\
          &=& (ae - bf - cg - dh) -(af + be + ch - dg)i - (ag - bh + ce + df)j - (ah + bg - cf + de)k\\
          &=& \overline{\alpha\beta}.
        \end{eqnarray*}
        Noting that the integers commute with the elements of $\quat$, we have
        $$N(\alpha\beta) = \alpha \beta \overline{\alpha \beta} = \alpha \beta \overline{\beta} \overline{\alpha} = \alpha N(\beta) \overline{\alpha} = \alpha \overline{\alpha} N(\beta) = N(\alpha)N(\beta).$$
        Therefore the norm is multiplicative.
      \item
        First observe that for any element $\alpha \in \quat$ we have
        $$N(\alpha) = N(1_\quat \alpha) = N(1_\quat)N(\alpha).$$
        Hence $N(1_\quat) = 1$.
        Let $u \in \quat$ be a unit so that
        $$N(1_\quat) = N(uu^{-1}) = N(u)N(u^{-1}) = 1$$
        implies $N(u)$ is a unit in $\Z$.
        Since $N(u) = a^2 + b^2 + c^2 + d^2 > 0$ and $1$ is the only positive unit, it follows that $N(u) = 1$.
        
        Conversely, let $u \in \quat$ and assume $N(u) = 1$.
        Then $$N(u) = u\overline{u} = 1$$ implies $u^{-1} = \overline{u}$.
        Therefore $u \in \quat$ is a unit if and only if $N(u) = 1$.
        
        Let $u = a + bi + cj + dk \in \quat^\times$ be given.
        First observe that the quaternion group, $Q_8 = \left\{\pm 1, \pm i, \pm j, \pm k \right\}$, is isomorphic to a subgroup of $\quat^\times$ under the embedding
        \begin{align*}
          i \colon Q_8 &\hookrightarrow \quat^\times\\
          g &\mapsto g.
        \end{align*}
        Hence it suffices to show that $H^\times \subseteq Q_8$.
        Since $u\overline{u} = a^2 + b^2 + c^2 + d^2 = 1$, it follows that one of $a, b, c, d$ is 1 and the others are zero.
        Taking all such combinations, we have $\quat^\times \subseteq Q_8$.
        Therefore $\quat^\times \cong Q_8$.
    \end{enumerate}
  \end{proof}
\end{thm}

\begin{thm}
  An element $x$ in $R$ is nilpotent if and only if for some positive integer $m$, we have $x^m = 0$.
  Suppose that $R$ is commutative.
  \begin{enumerate}[(a)]
  \item
    Prove that $x$ is either zero or a zero divisor.
  \item
    Prove that for all $r \in R$, we have $rx$ nilpotent.
  \item
    Prove that $1 + x \in R^\times$.
  \item
    Deduce that the sum of a nilpotent element and a unit is a unit.
  \end{enumerate}
  \begin{proof}
    \begin{enumerate}[(a)]
    \item
      If $x = 0$, then the result holds for all positive integers.
      Assume $x \neq 0$.
      Let $2 \leq m \in \Z$ be the smallest integer such that $x^m = 0$.
      It follows from the choice of $m$ that $x^m = x(x^{m-1}) = 0$ with $x^{m-1} \neq 0$.
      Therefore $x$ is a zero divisor.
    \item
      Let $r \in R$ be given and let $0 < m \in \Z$ be such that $x^m = 0$.
      Since $R$ is commutative, we have 
      $$(rx)^m = r^mx^m = (r^m)0 = 0.$$
      Therefore $rx$ is nilpotent.
    \item
      Let $0 < m \in \Z$ be the smallest integer such that $x^m = 0$.
      Since $x^{i} \neq 0$ for $2 \leq i \leq m - 1$, it follows from the distributive axiom for rings that
      $$(1 + x)(1 - x + x^2 - x^3 + \ldots + (-1)^{m-1}x^{m-1}) = 1 + x^m = 1.$$
      Therefore $1+x$ is a unit.
    \item
      Let $u \in R^\times$ be given and let $x$ be a nilpotent element of $R$.
      It follows from parts (b) and (c) that $1 + u^{-1}x$ is a unit.
      Therefore $u(1 + u^{-1}x) = u + x$ is a unit by virtue of $R^\times$ being a multiplicative group.
    \end{enumerate}
  \end{proof}
\end{thm}
\end{document}
