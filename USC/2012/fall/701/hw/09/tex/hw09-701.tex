\documentclass[10pt]{amsart}
\usepackage{amsmath,amsthm,amssymb,amsfonts,enumerate,mymath}
\openup 5pt
\author{Blake Farman\\University of South Carolina}
\title{Math 701:\\Homework 09}
\date{November 16, 2012}
\pdfpagewidth 8.5in
\pdfpageheight 11in
\usepackage[margin=1in]{geometry}

\begin{document}
\maketitle

\newtheorem{thm}{}
\newtheorem{lem}{Lemma}

\begin{thm}
  Let $R$ be a commutative ring.
  Define the set $R[[x]]$ of {\bf formal power series} in the indeterminate $x$ with coefficients from $R$ to be all formal infinite sums
  $$\sum_{n=0}^\infty a_nx^n = a_0 + a_1x + a_2 x^2 + a_3x^3 + \ldots.$$
  Define addition and multiplication in $R[[x]]$ by
  $$\sum_{n=0}^\infty a_nx^n + \sum_{n=0}^\infty b_mx^m = \sum_{k=0}^\infty (a_k + b_k)x^k,\; \left(\sum_{n=0}^\infty a_nx^n\right) \cdot \left(\sum_{m=0}^\infty b_mx^m\right) = \sum_{j=0}^\infty\left(\sum_{n=0}^j a_nb_{j-n}\right)x^j.$$
  \begin{enumerate}[(a)]
  \item
    Prove that $R[[x]]$ is a commutative ring.
    (Associativity of multiplication is a bit tedious, so you may assume it.)
  \item
    Show that $1-x$ is a unit in $R[[x]]$ with inverse $1 + x + x^2 + \ldots$.
  \item
    Prove that $\sum_{n=0}^\infty a_nx^n$ is a unit in $R[[x]]$ if and only if $a_0$ is a unit in $R$.
  \item
    Prove that if $R$ is an integral domain, then so is $R[[x]]$.
  \end{enumerate}

  \begin{proof}
  \end{proof}
\end{thm}

\begin{thm}
  Let $K$ be a field.
  A {\bf discrete valuation} on $K$ is a function $v \colon K^\times \rightarrow \Z$ satisfying
  \begin{enumerate}[(i)]
  \item
    $v(ab) = v(a) + v(b)$.
    (i.e. $v$ is a homomorphism from the group $(K^\times, \times)$ to $(\Z, +)$),
  \item
    $v$ is surjective,
  \item
    $v(x + y) \geq \min \left\{v(x), v(y)\right\}$ for all $x,y \in K^\times$ with $x + y \neq 0$
  \end{enumerate}
  The set $R = \left\{x \in K^\times \;\middle\vert\; v(x) \geq 0\right\} \cup \left\{0\right\}$ is the {\bf valuation ring} of $v$.
  \begin{enumerate}[(a)]
  \item
    Prove that $R$ is a subring of $K$. (In general, a ring $R$ is called a {\bf discrete valuation ring} if there is some field $K$ and some discrete valuation $v$ on $K$ such that $R$ is the valuation ring of $v$.)
  \item
    Prove ethat for each non-zero element $x \in K$, either $x$ or $x^{-1}$ is in $R$.
  \item
    Prove that an element $x$ is a unit of $R$ if and only if $v(x) = 0$.
  \end{enumerate}
  \begin{proof}
  \end{proof}
\end{thm}

\begin{thm}
  A specific example of a discrete valuation ring is obtained when $p$ is prime, $K = \Q$, and 
  $$v_p \colon \Q^\times \rightarrow \Z,\, \text{by}\  v_p\left(\frac{a}{b}\right) = \alpha,\, \text{where}\  \frac{a}{b} = p^\alpha\frac{c}{d},\, p \nmid c\ \text{and}\ p \nmid d.$$
  \begin{enumerate}[(a)]
  \item
    Prove that the corresponding valuation ring is 
    $$R = \left\{\frac{a}{b} \in \Q \;\middle\vert\; \gcd(a,b) = 1,\, \text{and}\ p\nmid b\right\}.$$
  \item
    Describe $R^\times$.
  \end{enumerate}
  \begin{proof}
  \end{proof}
\end{thm}

\begin{thm}
  Let $F$ be a field, and define the ring $$F((x)) = \left\{ \sum_{n=N}^\infty a_nx^n \;\middle\vert\; a_n \in F\ \text{and}\ N \in \Z\right\}.$$
  (Every element of $F((x))$ is a power series in $x$ plus a polynomial, possibly 0, in $1/x$.  
  I.e., each element of $F((x))$ has only finitely many terms with negative powers of $x$.)
  \begin{enumerate}[(a)]
  \item
    Prove that $F((x))$ is a field.
  \item
    Define the map 
    $$v \colon F((x)) \rightarrow \Z\ \text{by}\ v\left(\sum_{n=N}^\infty a_nx^n\right) = N$$
    where $a_N$ is the first non-zero coefficient in the series (i.e., $N$ is the ``order of zero or pole of the series at 0'').
    Prove that $v$ is a discrete valuation on $F((x))$ whose discrete valuation ring is $F[[x]]$, the ring of formal power series.
  \end{enumerate}
  \begin{proof}
  \end{proof}
\end{thm}

\begin{thm}
  Let $\quat = \Z + \Z i + \Z j + \Z k$ be the ring of na\"{i}ve integral Hamilton Quaternions and define $N \colon S \rightarrow \Z$ by $N(a + bi + cj + dk) = a^2 + b^2 + c^2 + d^2$.
  (This map is called a {\it norm}.)
  \begin{enumerate}[(a)]
  \item
    Suppose that $\alpha = a + bi + cj + dk \in S$.
    Define $\overline{\alpha} = a - bi - cj - dk$.
    Prove that $N(\alpha) = \alpha \overline{\alpha}$.
  \item
    For all $\alpha, \beta \in S$ prove that $N(\alpha\beta) = N(\alpha)N(\beta)$.
    (We say that the norm is multiplicative.)
  \item
    Let $u \in S$.
    Prove that $u \in S^\times$ if and only if $N(u) = \pm 1$.
    Show that $S^\times$ is a non-abelian group of order 8.
    (This is the ``quaternion grouip.''
    Note that the inverse of $a \neq 0$ in the ring of rational quaternions, $\quat(\Q)$ (replace $\Z$ by $\Q$ in the definition), is $\frac{\overline{\alpha}}{N(\alpha)}$.)
  \end{enumerate}
  \begin{proof}
  \end{proof}
\end{thm}

\begin{thm}
  An element $x$ in $R$ is nilpotent if and only if for some positive integer $m$, we have $x^m = 0$.
  Suppose that $R$ is commutative.
  \begin{enumerate}[(a)]
  \item
    Prove that $x$ is either zero or a zero divisor.
  \item
    Prove that for all $r \in R$, we have $rx$ nilpotent.
  \item
    Prove that $1 + x \in R^\times$.
  \item
    Deduce that the sum of a nilpotent element and a unit is nilpotent.
  \end{enumerate}
  \begin{proof}
    \begin{enumerate}[(a)]
    \item
      If $x = 0$, then the result holds for all integers.
      Assume $x \neq 0$.
      Let $2 \leq m \in \Z^{>0}$ be the smallest integer such that $x^m = 0$.
      Since $2 \leq m$, it follows from the choice of $m$ that $x^m = x(x^{m-1}) = 0$ with $x^{m-1} \neq 0$.
      Therefore $x$ is a zero divisor.
    \item
      Let $r \in R$ be given and let $m \in \Z^{> 0}$ be such that $x^m = 0$.
      Since $R$ is commutative, we have $(rx)^m = r^mx^m = (r^m)0 = 0$.
      Therefore $rx$ is nilpotent.
    \item
      Let $m \in \Z^{>0}$ be the smallest integer such that $x^m = 0$.
    \item
      Let $u \in R^\times$ be given and for some nilpotent element $x$, let m $\in \Z^{>0}$ be the smallest integer such that $x^m = 0$
    \end{enumerate}
  \end{proof}
\end{thm}
\end{document}
