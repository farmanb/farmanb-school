\documentclass[10pt]{amsart}
\usepackage{../../../../../../../tex/mymath/mymath}
\usepackage{amsmath,amsthm,amssymb,amsfonts,enumerate,mymath}
\openup 5pt
\author{Blake Farman\\University of South Carolina}
\title{Math 701:\\Homework 08}
\date{November 9, 2012}
\pdfpagewidth 8.5in
\pdfpageheight 11in
\usepackage[margin=1in]{geometry}

\begin{document}
\maketitle

\newtheorem{thm}{}
\newtheorem{lem}{Lemma}

\begin{thm}
  Let $n \geq 5$.
  Show that $A_N = (S_n)^\prime = [S_n, S_n]$.
  \begin{proof}
    First consider the group homomorphism $\varepsilon \colon S_n \rightarrow \left\{-1,1\right\}$.
    Observe that the codomain is isomorphic to $\Z/2\Z$ and thus abelian.
    Then for any two elements $\sigma, \tau$ of $S_n$ we have 
    $$
    \varepsilon\left([\sigma,\tau]\right) = \varepsilon\left(\sigma\tau\sigma^{-1}\tau^{-1}\right) 
    = \varepsilon\left(\sigma\right)\varepsilon\left(\tau\right)\varepsilon\left(\sigma^{-1}\right)\varepsilon\left(\tau^{-1}\right) 
    = \varepsilon\left(\sigma\right)\varepsilon\left(\sigma\right)^{-1}\varepsilon\left(\tau\right)\varepsilon\left(\tau^{-1}\right) 
    = 1.
    $$
    Hence $(S_n)^\prime \leq A_n$.
    By Exercise 5 of Homework 6, $(S_n)^\prime$ contains a 3-cycle and hence is a non-trivial subgroup of $A_n$.
    Moreover, $(S_n)^\prime$ is normal in $S_n$ and thus normal in the simple group $A_n$.
    Therefore $A_n = (S_n)^\prime$.
  \end{proof}
\end{thm}

\begin{thm}
  \begin{enumerate}[(a)]
  \item
    Construct a composition series for $S_4$.
    Conclude that $S_4$ is solvable.
  \item
    Show that $S_n$ is not solvable for $n \geq 5$.
  \end{enumerate}
  \begin{proof}
    \begin{enumerate}[(a)]
    \item
      First consider the subgroup $\left<\left(1 \; 2\right)\left( 3 \; 4\right),  \left(1 \; 3\right)\left( 2 \; 4\right)\right>$ of $A_n$.
      Observe that this is an abelian group with four elements of order two (isomorphic to the Klein four-group), with fourth element $$\left(1 \; 2\right)\left( 3 \; 4\right)\left(1 \; 3\right)\left( 2 \; 4\right) = \left(1 \; 3\right)\left( 2 \; 4\right)\left(1 \; 2\right)\left( 3 \; 4\right) = \left(1 \; 4\right)\left( 2 \; 3\right).$$
      By Exercise 6 of Homework 6, this group is normal in $A_4$.
      Note also that the index of $\left<\left(1 \; 2\right)\left( 3 \; 4\right),  \left(1 \; 3\right)\left( 2 \; 4\right)\right>$ in $A_n$ is 3.
      
      Consider the composition series 
      $$\left<1\right> \unlhd \left<\left(1 \; 2\right)\left( 3 \; 4\right)\right> \unlhd \left<\left(1 \; 2\right)\left( 3 \; 4\right),  \left(1 \; 3\right)\left( 2 \; 4\right)\right> \unlhd A_4 \unlhd S_4.$$
      The factors are isomorphic to $\Z/2\Z$, $\Z/2\Z$, $\Z/3\Z$, and $\Z/2\Z$, respectively.
      Each of these are abelian, therefore $S_4$ is solvable.
    \item
      From the Corollary to the proof that $A_n$ is simple for $n \geq 5$, the only normal subgroups of $S_n$ are $\left< 1 \right>$, the simple group $A_n$, and $S_n$.
      Hence the only possible choices for series are $\left< 1 \right> \unlhd A_n \unlhd S_n$ and $\left< 1 \right> \unlhd S_n$.
      Since neither $A_n$ nor $S_n$ are abelian, $S_n$ is not solvable. 
    \end{enumerate}
  \end{proof}
\end{thm}

\begin{thm}
  Let $p$ be prime.
  Show that a finite $p$-group is solvable.
  \begin{proof}
    Let $P$ be a $p$-group of order $p^\alpha$ for some integer $\alpha \geq 1$.
    By Exercise 5 (e) of Homework 4, every finite $p$-group has a normal subgroup of index $p$.
    Since every such subgroup of a $p$-group is itself a $p$-group, we may construct a series
    $$\left<1\right> = P_\alpha \unlhd P_{\alpha-1} \unlhd \ldots \unlhd P_2 \unlhd P_1 \unlhd P_0 = P$$
    by choosing $P_{i+1}$ to be a normal subgroup of index $p$ in $P_i$.
    Since each factor $P_i / P_{i+1}$ has order $p$, it is isomorphic to the abelian group $\Z/p\Z$.
    Therefore $P$ is solvable.
  \end{proof}
\end{thm}

\begin{thm}
  A non-trivial abelian group $A$ is said to be divisible if and only if for each element $a \in A$ and for every $k \in \Z$, there exists $x \in A$ with $kx = a$.
  \begin{enumerate}[(a)]
  \item
    Prove that $\Q$, the additive group of rationals, is divisible.
  \item
    Prove that no finite abelian group is divisible.
  \item
    Prove that the quotient of a divisible group by a proper subgroup is divisible.
    Deduce that $\Q/\Z$ is divisible.
  \end{enumerate}
  \begin{proof}
    \begin{enumerate}[(a)]
    \item
      Let $a \in \Q$ and $k \in \Z$ be given.
      Since $a/k$ is an element of $\Q$, then $$k\left(\frac{a}{k}\right) = \underbrace{\frac{a}{k} + \frac{a}{k} + \ldots + \frac{a}{k}}_{k} = a$$
      shows that $\Q$ is divisible.
    \item
      Let $G$ be a finite abelian group of order $n > 1$.
      Let $0 \not = g \in G$ be given.
      Observe that for every element $g^\prime$ of $G$, $ng^\prime = 0 \not = g$.
      Hence there exists an element $g$ of $G$ and an integer $n$ such that for every element $g^\prime$ of $G$, $g \not = ng^\prime$.
      Therefore $G$ is not divisible.
    \item
      Let $G$ be a divisible group and let $H < G$ be given.
      Let $\pi \colon G \rightarrow G/H$ be the canonical projection homomorphism.
      Fix an element $g + H$ of $G/H$ and an integer $k$.
      Since $G$ is divisible, there exists an element $g^\prime$ of $G$ such that $g = kg^\prime$.
      Observe that $$\pi(kg^\prime) = \pi\left(\sum_{i=1}^k g^\prime\right) = \sum_{i=1}^k\pi\left(g^\prime\right) = k\pi\left(g^\prime\right) = k(g^\prime + H)$$ holds because $\pi$ is a homomorphism.
      Therefore
      $$k(g^\prime + H) = \pi(kg^\prime) =  \pi(g) = g + H$$
      shows $G/H$ is a divisible group.
      
      To see that $\Q/\Z$ is divisible, observe that $\Q$ is not cyclic by Exercise 5 of Homework 7.
      Therefore the containment $\left< 1 \right> = \Z \leq \Q$ must in fact be proper and the result follows directly from (a) and (c).
    \end{enumerate}
  \end{proof}
\end{thm}

\begin{lem}\label{normalizer}
  Let $G$ be a group and let $H,K \leq G$.
  If $H \leq N_G(K)$, then $HK$ is a subgroup.
  
  \begin{proof}
    Observe that by Exercise 4 of Homework 2 it suffices to show $HK = KH$.
    Towards that end, let $h \in H$ and $k \in K$ be given.
    Consider the element $hk \in HK$.
    Since $H \leq N_G(K)$, we have $hkh^{-1} \in K$.
    Hence $$hk = hk(h^{-1}h) = (hkh^{-1})h \in KH$$ implies $HK \subseteq KH$.
    
    To see the reverse containment, consider the element $kh \in KH$.
    Since $H \leq N_G(K)$, it follows that $h^{-1}kh \in K$.
    Hence $$kh = (hh^{-1})kh = h(h^{-1}kh) \in HK$$ implies $KH \subseteq HK$.
    Therefore $KH = HK$, as desired.
  \end{proof}
\end{lem}
\begin{thm}
  Let $G = \SL{2}{\Z/3\Z}$.
  \begin{enumerate}[(a)]
  \item
    Find $\abs{G}$.
  \item
    Give all Sylow 3-subgroups of $G$.
  \item
    Prove that the subgroup of $G$ generated by 
    $\left(\begin{array}{cc}
      0 & -1\\
      1 & 0
    \end{array}\right)$ and 
    $\left(\begin{array}{cc}
      1 & 1\\
      -1 & -1
    \end{array}\right)$ 
    is the unique 2-Sylow subgroup of $G$.
  \item
    Show that $\cntr{G} = \left\{\pm \left(\begin{array}{cc}
      1 & 0\\
      0 & 1\\
    \end{array}\right)\right\}$.
    Conclude that $\PSL{2}{\Z/3\Z} = G/\cntr{G} \cong A_4$.
  \end{enumerate}
  \begin{proof}
    \begin{enumerate}[(a)]
    \item
      First observe that there are $3^4 = 81$ matrices in $\M{2}{\Z/3\Z}$.
      We first count the number of matrices of determinant zero, thereby determining the number of matrices in $\GL{2}{\Z/3\Z}$.  
      Then since $\Det \colon \GL{2}{\Z/3\Z} \rightarrow (\Z/3\Z)^\times \cong \Z/2\Z$ is a surjective homomorphism with $\ker\Det = \SL{2}{\Z/3\Z}$, we will have $\abs{\SL{2}{\Z/3\Z}} = \abs{\GL{2}{\Z/3\Z}}/2$ by the First Isomorphism Theorem.

      Towards that end, observe that there are four possible choices of matrices with determinant zero: the zero matrix, matrices with three zeroes, matrices with two zeroes, and matrices with all non-zero entries.
      There are four possible arrangements of a matrix with three zeroes and two choices for the fourth entry.
      Hence there are eight such matrices.
      For matrices with two zeroes, there four possible arrangements of the zeroes and two choice for the remaining two entries.
      Hence there are 16 such matrices.
      Finally, for matrices with all non-zero entries, consider the equation $ad = bc$.
      There are two choices for $ad$--1 or 2--and two possible choices for the pair $(a,d)$: $(1,1)$, $(2,2)$, $(1,2)$, and $(2, 1)$.
      Similarly, for each pair $(a,d)$, there are two corresponding choices for the pair $(c,d)$.
      Hence there are eight such matrices.
      Therefore there are $16 + 8 + 8 + 1 = 33$ elements of $\M{2}{\Z/3\Z}$ with determinant zero, $81 - 33 = 48$ elements in $\GL{2}{\Z/3\Z}$, and 24 elements in $\SL{2}{\Z/3\Z}$.
    \item
      By the Sylow Theorems, the number of Sylow 2-subgroups is either 1 or 3 and the number of 3-subgroups is either 1 or 4.
      With some computation, one finds the following four matrices of order 3:\\
      \begin{center}
        \begin{tabular}{c|c}
          $A$ & $A^2 = A^{-1}$\\
          \hline
          $\left(\begin{array}{cc}
            1 & 1\\
            0 & 1\\
          \end{array}\right)$& $\left(\begin{array}{cc}
            1 & 2\\
            0 & 1\\
          \end{array}\right)$\\
          \hline
          $\left(\begin{array}{cc}
            1 & 0\\
            2 & 1\\
          \end{array}\right)$ & $\left(\begin{array}{cc}
            1 & 0\\
            1 & 1\\
          \end{array}\right)$\\
          \hline
          $\left(\begin{array}{cc}
            2 & 1\\
            2 & 0\\
          \end{array}\right)$ & $\left(\begin{array}{cc}
            0 & 2\\
            1 & 2\\
          \end{array}\right)$\\
          \hline
          $\left(\begin{array}{cc}
            0 & 1\\
            2 & 2\\
          \end{array}\right)$ & $\left(\begin{array}{cc}
            2 & 2\\
            1 & 0\\
          \end{array}\right)$\\
        \end{tabular}
      \end{center}
      Since $n_3$ is either 1 or 4, these must be the generators for the only Sylow 3-subgroups.
    \item
      Observe that the Sylow 2-subgroups each have 7 distinct non-identity elements.  
      Since there are four Sylow 3-subgroups, each with two distinct elements, if $n_2 = 3$, then there are at least $21 + 9 + 1 = 31$ elements in a group of order 24, a contradiction.
      Hence $n_2 = 1$ and so it suffices to show that the group generated by     
      $\left(\begin{array}{cc}
        0 & 2\\
        1 & 0
      \end{array}\right)$ and 
      $\left(\begin{array}{cc}
        1 & 1\\
        2 & 2
      \end{array}\right)$ is of order eight.
      
      Towards that end, it's readily checked that
      $$BAB^{-1} = \left(\begin{array}{cc}
        0 & 1\\
        2 & 0
      \end{array}\right) = A^2.$$
      It then follows from Lemma 1 of Homework 7 and closure under the group operation that $\left<B\right> \leq N_G(\left<A\right>)$.
      Then by Lemma~\ref{normalizer} $\left<A,B\right> = \left<A\right>\left<B\right> \leq G$.
      The elements of the groups generated by $A$ and $B$ are given in the table below:\\
      \begin{center}
        \begin{tabular}{c||c|c|c}
          $\cdot^1$ & $\cdot^2$ & $\cdot^3$ & $\cdot^4$\\
          \hline
          $\left(\begin{array}{cc}
            0 & 2\\
            1 & 0
          \end{array}\right)$ & 
          $\left(\begin{array}{cc}
            2 & 0\\
            0 & 2 
          \end{array}\right)$ & 
          $\left(\begin{array}{cc}
            0 & 1\\
            2 & 0
          \end{array}\right)$ &
          $\left(\begin{array}{cc}
            1 & 0\\
            0 & 1
          \end{array}\right)$\\
          \hline
          $\left(\begin{array}{cc}
            1 & 1\\
            2 & 2
          \end{array}\right)$ & 
          $\left(\begin{array}{cc}
            2 & 0\\
            0 & 2
          \end{array}\right)$ & 
          $\left(\begin{array}{cc}
            2 & 2\\
            2 & 1
          \end{array}\right)$ & 
          $\left(\begin{array}{cc}
            1 & 0\\
            0 & 1
          \end{array}\right)$.\\
        \end{tabular}
      \end{center}
      Therefore the order of the group generated by $A$ and $B$ is  $$\abs{\left<A,B\right>} = \abs{\left<A\right>\left<B\right>} = \frac{\abs{\left<A\right>}\abs{\left<B\right>}}{\abs{\left<A\right> \cap \left<B\right>}} = \frac{4\cdot4}{2} = 8,$$
      as desired.
    \item
      %Let $S_3 \in \Syl{3}{G}$ be given.
      %Observe that since $S_3 \cap \cntr{G} \leq \cntr{G}$, it follows that $S_3 \cap \cntr{G} \unlhd G$.
      %Moreover, by closure of subgroups under inverses, $S_3 \cap \cntr{G}$ must either be $S_3$ or $\left< 1 \right>$.
      %Since $n_3(G) \not = 1$, $S_3$ is not normal in $G$ and thus $S_3 \cap \cntr{G} = \left< 1 \right>$.
      %Then by Exercise 3 of Homework 5, $\left< 1 \right> \in \Syl{3}{\cntr{G}}$ and thus it follows from the contrapositive of the Sylow Existence Theorem that $3 \nmid \abs{\cntr{G}}$.
      %Hence $\cntr{G}$ is a 2-group and $\cntr{G} \leq S_2$, the unique Sylow 2-subgroup of $G$.
      Let $M = \left(\begin{array}{cc}
        a & b\\
        c & d
      \end{array}\right) \in \cntr{G}$ be given.
      Since $M$ is an element of the center, it must commute with the matrices $C = \left(\begin{array}{cc}
        1 & 1\\
        0 & 1
      \end{array}\right)$
      and
      $D = \left(\begin{array}{cc}
        1 & 0\\
        1 & 1
      \end{array}\right)$.
      Computing $MC$ and $CM$ we have
      $$MC = \left(\begin{array}{cc}
        a & b\\
        c & d
      \end{array}\right)
      \left(\begin{array}{cc}
        1 & 1\\
        0 & 1
      \end{array}\right) =
      \left(\begin{array}{cc}
        a & a + b\\
        c & c + d
      \end{array}\right) = 
      \left(\begin{array}{cc}
        a + c & b + d\\
        c & d
      \end{array}\right)
      =
      \left(\begin{array}{cc}
        1 & 1\\
        0 & 1
      \end{array}\right)
      \left(\begin{array}{cc}
        a & b\\
        c & d
      \end{array}\right)
      = CM.$$
    \end{enumerate}
    Since $a \equiv a + c \pmod{3}$ and $a + b \equiv b + d \pmod{3}$, it follows that $c \equiv 0 \pmod{3}$ and $a \equiv d \pmod{3}$.
    Rewriting $M = \left(\begin{array}{cc}
      a & b\\
      0 & a
    \end{array}\right)$ it is clear from $\Det{M} = a^2 = 1$ that $a \equiv \pm 1 \pmod{3}$.
    Computing $MD$ and $DM$ we then have 
    $$MD = 
    \left(\begin{array}{cc}
      a & b\\
      0 & a
    \end{array}\right)
    \left(\begin{array}{cc}
      1 & 0\\
      1 & 1
    \end{array}\right)
    =
    \left(\begin{array}{cc}
      a + b & b\\
      a & a
    \end{array}\right) = \left(\begin{array}{cc}
      a & b\\
      a & b + a
    \end{array}\right)
    = 
    \left(\begin{array}{cc}
      1 & 0\\
      1 & 1
    \end{array}\right)
    \left(\begin{array}{cc}
      a & b\\
      0 & a
    \end{array}\right)
    =DM.$$
    Then $a \equiv a + b \pmod{3}$ implies that $b \equiv 0 \pmod{3}$.
    Therefore $M = \pm I$ and $\cntr{G} = \left\{\pm I\right\}$, as desired.
    
    To see that $\PSL{2}{\Z/3\Z}$ is isomorphic to $A_4$, consider the additive group $\Z/3\Z \times \Z/3\Z$. 
    Let $\Omega = \left\{\left<(0,1)\right>, \left<(1,0)\right>, \left<(1,1)\right>, \left<(1,2)\right>\right\}$ and define the action 
    \begin{align*}
      \SL{2}{\Z/3\Z} \times \Omega &\longrightarrow \Omega\\
      \left(\left(\begin{array}{cc}
        a & b\\
        c & d
      \end{array}\right), \left< (x,y) \right>\right) &\longmapsto \left<(ax + by, cx + dy)\right>.
    \end{align*}
    To see that the action is well-defined let $\left<(x,y)\right> \in \Omega$ and $\left(\begin{array}{cc}
      a & b\\
      c & d
      \end{array}\right), \left(\begin{array}{cc}
      e & f\\
      g & h
      \end{array}\right) \in \SL{2}{\Z/3\Z}$ be given.
    First observe that the four subgroups that comprise $\Omega$ all have pairwise trivial intersections and account for all the elements of $\Z/3\Z \times \Z/3\Z$.
    Moreover, since the elements of $\SL{2}{\Z/3\Z}$ are invertible, hence injective, the action never results in the trivial group.
    Hence $\left<(ax + by, cx + dy)\right> \in \Z/3\Z$ generates an element of $\Omega$.
    Then observe that $(x,y) + (x,y) = 2(x, y) = (2x, 2y)$ and 
    $$\left(\begin{array}{cc}
      a & b\\
      c & d
      \end{array}\right) \cdot \left<(2x,2y)\right> = \left<(2ax + 2by, 2cx + 2cy)\right> = \left<(ax + by, cx + dy)\right>$$
    implies that the action is independent of the choice of generator.
    %preserves the subgroup structure and  therefore
    Similarly, we see that
    \begin{eqnarray*}
      \left(\begin{array}{cc}
      a & b\\
      c & d
      \end{array}\right)
    \left(\begin{array}{cc}
      e & f\\
      g & h
    \end{array}\right) \cdot \left<(x,y)\right> &=& \left<(a(ex + fy) + b(gx + hy), c(ex + fy) + d(gx + hy))\right>\\
    &=& \left<((ae + bg)x + (af + bh)y, (ce + dg)x + (cf + dh)y)\right>\\
    &=& \left(\begin{array}{cc}
        ae + bg & af + bh\\
        ce + dg & cf + gh
      \end{array}\right)\cdot\left<(x,y)\right>
    \end{eqnarray*}
    shows that the action is associative.
    Finally, note that $I$ is the identity of the action, namely $$I\cdot\left<(x,y)\right> = \left<(1x + 0y,0x + 1y)\right> = \left<(x,y)\right>.$$
    
    Consider any element $\left(\begin{array}{cc}
      a & b\\
      c & d
    \end{array}\right)$
    that fixes all elements $\left<(x,y)\right>$.
    Then $(ax + by, cx + dy) \in \left<(x,y)\right>$ for all $(x,y)$ implies that $(ax + by, cx + dy)$ is a scalar multiple of $(x,y)$.
    Hence the kernel of the action is the set of scalar matrices contained in $\SL{2}{\Z/3\Z}$, $\left\{\pm I\right\} = \cntr{\SL{2}{\Z/3\Z}}$.
    Let $\varphi \colon \SL{2}{\Z/3\Z} \rightarrow S_4$ be the homomorphism induced by the action.
    Since $\ker\varphi$ coincides with the kernel of the action, it follows from the First Isomorphism Theorem that $\SL{2}{\Z/3\Z}/\cntr{\SL{2}{\Z/3\Z}}$ is isomorphic to a subgroup of order 12 in $S_4$.
    Therefore $\SL{2}{\Z/3\Z}/\cntr{\SL{2}{\Z/3\Z}} \cong A_4$, the unique subgroup of order 12 in $S_4$.
  \end{proof}
\end{thm}

\begin{thm}
  \begin{enumerate}[(a)]
  \item
    Let $p$ be prime and let 
    $$U_n(\Z/p\Z) = \left\{(x_{ij}) \in \GL{n}{\Z/p\Z} \;\middle\vert\; x_{ij} = 0 \text{ for all } i > j;\; x_{ii} = 1 \text{ for all } i \right\}.$$
    Show that $U_n(\Z/p\Z)$ is a Sylow $p$-subgroup of $\GL{N}{\Z/p\Z}$.
    It may be helpful to know that 
    $$\abs{\GL{N}{\Z/p\Z}} = \prod_{i=0}^{n-1}(p^n - p^i).$$
  \item
    Prove that the number of Sylow $p$-subgroups of $\GL{2}{\Z/p\Z}$ is $p + 1$.
  \end{enumerate}
  \begin{proof}
    \begin{enumerate}[(a)]
    \item
      From Linear Algebra, the determinant of an upper triangular matrix is the product of the diagonal entries.
      Hence $U_n(\Z/p\Z) \subseteq \SL{n}{\Z/p\Z} \leq \GL{n}{\Z/p\Z}$.
      Since $\GL{n}{\Z/p\Z}$ is finite, to show $U_n(\Z/p\Z)$ is a subgroup it suffices to show that it is closed under the group operation, for then $U_n(\Z/p\Z)$ is closed under inverses by Lagrange's Theorem, associative, and contains the identity.
      Towards that end, let $A = \left(a_{i,j}\right), B = \left(b_{i,j}\right) \in U_n(\Z/p\Z)$ be given.
      Consider the product $AB = \left(c_{i,j}\right)$.
      Fix some $i > j$ and write $$c_{i,j} = \sum_{k=1}^n a_{i,k}b_{k,j} = \sum_{k=1}^{i-1}a_{i,k}b_{k,j} + \sum_{k=i}^n a_{i,k}b_{k,j}.$$
      When $k < i$ we have $a_{i,k} = 0$ and when $k \geq i > j$ we have $b_{k,j} = 0$.
      Hence $c_{i,j} = 0$.
      Now fix some $i$ and consider 
      $$c_{i,i} = \sum_{k=1}^n a_{i,k}b_{k,i} = \sum_{k=1}^{i-1} a_{i,k}b_{k,i} + a_{i,i}b_{i,i} + \sum_{k=i+1}^n a_{i,k}b_{k,i} = 1 + \sum_{k=1}^{i-1} a_{i,k}b_{k,i} + \sum_{k=i+1}^n a_{i,k}b_{k,i}.$$
      When $i > k$ we have $a_{i,k} = 0$ and when $k > i$ we have $b_{k,i} = 0$.
      Hence $c_{i,i} = 1$.
      Therefore $(c_{i,j}) \in U_n(\Z/p\Z)$ implies $U_n(\Z/p\Z)$ is closed under the group operation.
      
      To see that $U_n(\Z/p\Z)$ is a Sylow $p$-subgroup, first consider the size of $U_n(\Z/p\Z)$.
      The choice of upper off-diagonal entries, of which there are $n(n-1)/2$, does not affect the determinant, so it follows that there are $p^{n(n-1)/2}$ such matrices.
      It suffices to show that $p^{n(n-1)/2}$ exactly divides $\abs{\GL{n}{\Z/p\Z}}$.
      First write 
      $$\abs{\GL{n}{\Z/p\Z}} = \prod_{i=0}^{n-1} (p^n - p^i) = (p^n - 1)(p)(p^{n-1} - 1) \ldots (p^{n-1})(p - 1) = p^{n(n-1)/2}\prod_{i=0}^{n-1}(p^{n-i} - 1).$$
      Then  $\prod_{i=0}^{n-1}(p^{n-i} - 1)  \equiv (-1)^n \pmod{p}$ implies $p \nmid m$.
      Therefore $U_n(\Z/p\Z) \in \Syl{p}{\GL{n}{\Z/p\Z}}$, as desired.
    \item
      First observe that $\Det \colon \GL{2}{\Z/p\Z} \rightarrow (\Z/p\Z)^\times$ is a surjective\footnote[1]{Take the matrix $(a_{i,i})$ with $a_{1,1} = a \in (\Z/p\Z)^\times$ and, all other diagonal entries 1, and all off-diagonal entries 0.} homomorphism  with $\ker\Det = \SL{2}{\Z/p\Z}$.
      Hence by (a) and Exercise 5 of Homework 5 it suffices to show $n_p(\SL{2}{\Z/p\Z}) = p + 1$.
      Moreover, by the First Isomorphism Theorem,
      $$\abs{\SL{2}{\Z/p\Z}} = \frac{\abs{\GL{2}{\Z/p\Z}}}{p-1} = \frac{(p^2 - 1)(p^2 - p)}{p-1} = p(p+1)(p-1).$$
        
      Since $n_p(\SL{2}{\Z/p\Z}) \equiv 1 \pmod{p}$ and $n_p(\SL{2}{\Z/p\Z}) \mid (p+1)(p-1)$, it follows that $n_p(\SL{2}{\Z/p\Z})$ is either 1 or $p+1$.
      To see that $n_p(\SL{2}{\Z/p\Z}) \not = 1$, take the matrix $\left(\begin{array}{cc}
        1 & 1\\
        0 & 1
      \end{array}\right) \in U_2(\Z/p\Z)$ and conjugate it by the matrix $\left(\begin{array}{cc}
        1 & 1\\
        1 & 0
      \end{array}\right) \in \GL{2}{\Z/p\Z}$.
      We have
      $$\left(\begin{array}{cc}
        1 & 1\\
        0 & 1
      \end{array}\right)\left(\begin{array}{cc}
        1 & 1\\
        1 & 0
      \end{array}\right)
      \left(\begin{array}{cc}
        0 & 1\\
        1 & -1
      \end{array}\right) = \left(\begin{array}{cc}
        1 & 2\\
        1 & 1
      \end{array}\right)\left(\begin{array}{cc}
        0 & 1\\
        1 & -1
      \end{array}\right)
      =
      \left(\begin{array}{cc}
        2  & -1\\
        1 & 0
      \end{array}\right) \not \in U_2(\Z/p\Z).$$
      Therefore $U_2(\Z/p\Z)\in \Syl{p}{\GL{n}{\Z/p\Z}}$ is not normal and so $n_p(\GL{2}{\Z/p\Z})  = n_p(\SL{2}{\Z/p\Z}) = p + 1$.
    \end{enumerate}  
  \end{proof}
\end{thm}
\end{document}
