\documentclass[10pt]{amsart}
\usepackage{amsmath,amsthm,amssymb,amsfonts,enumerate}
\openup 5pt
\author{Blake Farman\\University of South Carolina}
\title{Math 784:\\Homework 06}
\date{November 9, 2012}
\pdfpagewidth 8.5in
\pdfpageheight 11in
\usepackage[margin=1in]{geometry}

\renewcommand{\qedsymbol}{\ensuremath{\blacksquare}}
\newcommand{\abs}[1]{\left| #1 \right|}
\newcommand{\GCD}[1]{\operatorname{GCD}\left( #1 \right)}

\begin{document}
\maketitle

\newtheorem{thm}{}

\begin{thm}
	Let $n$ be a positive rational integer.
	Define a greatest common divisor for $n$ ideals $A_1, \ldots, A_n$ in $R$ as an ideal $D$ in $R$ that satisfies
	\begin{enumerate}[(i)]
		\item\label{1.1}
			$D \mid A_i$ for each $1 \leq i \leq n$,
		\item\label{1.2}
			If $E$ is an ideal dividing each of $A_1, \ldots, A_n$, then $E \mid D$.
	\end{enumerate}
	Prove that such a greatest common divisor is unique.
	Denote it by $\GCD{A_1, \ldots, A_n}$, and furthermore prove that
		$$\GCD{A_1, \ldots, A_n} = A_1 + A_2 + \ldots A_n.$$
	\begin{proof}
	Suppose that $D_1$ and $D_2$ are both GCDs as defined above.
	Since $D_1$ divides $A_i$ for each $1 \leq i \leq n$, it follows from \eqref{1.2} that $D_1$ divides $D_2$.
	Hence $D_2 \subseteq D_1$ by Theorem 69.
	Similarly, since $D_2$ divides $A_i$ for each $1 \leq i \leq n$, $D_2$ divides $D_1$.
	Hence $D_1 \subseteq D_2$ by Theorem 69.
	Therefore $D_1 = D_2$.
	
	Let $D = \GCD{A_1, \ldots, A_2}$ and let $I = A_1 + A_2 + \ldots + A_n$.
	Since $A_i \subseteq I$ for each $1 \leq i \leq n$, it follows from \eqref{1.2} that $I$ divides $D$.
	Hence $D \subseteq I$ by Theorem 69.
	Moreover, by \eqref{1.1} and Theorem 69, $A_i \subseteq D$ holds for each $1 \leq i \leq n$.
	Since $D$ is an ideal, it is closed under addition and thus $I \subseteq D$.
	Therefore $I = D$, as desired.
	\end{proof}
\end{thm}

\end{document}