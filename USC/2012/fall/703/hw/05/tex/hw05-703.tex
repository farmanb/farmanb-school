\documentclass[12pt]{amsart}
\renewcommand{\baselinestretch}{1.5}
\usepackage{amsmath,amsthm,amssymb,amsfonts,enumerate,paralist,mymath}
\openup 5pt
\author{Blake Farman\\University of South Carolina}
\title{Math 703:\\Homework 05}
\date{October 30, 2012}
\pdfpagewidth 8.5in
\pdfpageheight 11in
\usepackage[margin=.75in]{geometry}
\begin{document}
\maketitle

\newtheorem{setup}{}
\setcounter{setup}{16}
\newtheorem{ex}{}[setup]
\newtheorem{lem}{Lemma}
\theoremstyle{definition}
\newtheorem{defn}{Definition}

\begin{lem}\label{min-radius}
  Let $A$ be an open, connected subset of $\C$ with $\card{A} \geq 2$.
  Let $\gamma \colon [a, b] \rightarrow A$ be a path connecting two distinct points, $a_1, a_2$, of $A$.
  Then there exists some radius $r > 0$ such that for each $x \in [a,b]$, the open disc of radius $r$ about $\gamma(x)$ lies in $A$.
  
  \begin{proof}
    First observe that $\gamma$ continuous and $[a,b]$  compact imply $\gamma([a,b])$ is compact hence totally bounded by Theorem 2.46 and bounded by exercise 15 on Homework 4.
    Thus there exists some $M \in \R$ such that $\gamma([a,b])$ is contained in an open disc, $D$, of radius $M$.
    If $A$ is not bounded, then consider the smaller open, connected set $D \cap A \subseteq A$.
    Clearly if there exists an $r \in \R$ such that for every $t \in [a,b]$, the disc of radius $r$ about $\gamma(t)$ lies in $D \cap A$, then that same disc lies in $A$.
    Hence it suffices to assume $A$ is bounded.
    
    Since $A$ is assumed to be bounded, there exists some open disc, $D$, of finite radius $M$ such that $A$ lies in the closure of $D$, $\overline{D}$.
    %Since $A$ is open in $X$ and $D \cap A = A$, $A$ is open in $D$.
    Hence $\overline{D} \cap (X \setminus A)$ is closed and bounded, thus compact.
    Then consider $$\dist{\overline{D} \cap (X \setminus{A})}{\gamma([a,b])} = \inf \left\{\abs{\gamma(t) - d} \;\middle\vert\; x \in [a,b], d \in \overline{D} \cap (X \setminus{A})\right\}.$$
    By Exercise 3 on page 542 (Trench), there exist elements $t_0 \in [a,b]$ and $d_0 \in \overline{D} \cap (X \setminus{A})$ such that $\abs{\gamma(t_0) - d_0} = \dist{\overline{D} \cap (X \setminus{A})}{\gamma([a,b])}$.
    Moreover, $\gamma(t_0) \in A$ and $d_0 \not \in A$ imply $d_0 \not = t_0$ and thus this distance is non-zero.
    Take $r = \dist{\overline{D} \cap (X \setminus{A})}{\gamma([a,b])}$.
  \end{proof}
\end{lem}

\begin{lem}\label{partition}
  Let $A$ be an open, connected subset of $\C$ with $\card{A} \geq 2$.
  If $\gamma \colon [a, b] \rightarrow A$ is a path connecting two distinct points, $a_1$ and $a_2$, of $A$, then there exists a partition $P = [t_1, t_2, \ldots, t_n]$ of $[a,b]$ with
  $$a = t_1 \leq t_2 \leq \ldots \leq t_n = b$$
  such that $\gamma([t_i, t_{i+1}])$ is contained in an open disc, $D_i$, with $D_i \subseteq A$.
  \begin{proof}
    %First observe that since $\gamma$ is continuous, $\gamma([a,b])$ is compact, and thus totally bounded.
    By Lemma~\ref{min-radius}, there exists some $r > 0$ such that for every $x \in [a,b]$, the disc of radius $r$ about $\gamma(t)$ lies in $A$.
    %Since $\gamma([a,b])$ is totally bounded, there exist finitely many discs of radius $r$, $D_1, D_2, \ldots, D_n$, such that 
    %$$\gamma([a,b]) \subseteq \bigcup_{i=1}^n D_i.$$
    Since $[a,b]$ is compact, $\gamma$ is uniformly continuous on $[a,b]$.
    Hence there exists some $\delta$ such that for any $t_1, t_2 \in [a,b]$, $$\abs{\gamma(t_1) - \gamma(t_2)} \leq r$$ holds whenever $\abs{t_1 - t_2} \leq \delta$.
    Then take, for some $n$, a partition $a = t_1 \leq t_2 \leq \ldots \leq t_n = b$ such that $t_{i+1} - t_i < \delta$.
    Therefore $\gamma([t_i, t_{i+1}])$ lies in the open disc $D_i \subseteq A$ of radius $r$ centered about $\gamma(t_i)$ for each $1 \leq i \leq n$.
  \end{proof}
\end{lem}

\begin{defn}
  Let $A$ be an open, connected subset of $\C$ and let $\gamma_1 \colon [a_1,b_1] \rightarrow A, \gamma_2 \colon [a_2,b_2] \rightarrow A, \ldots, \gamma_n \colon [a_n,b_n] \rightarrow A$ be a finite collection of paths with $\gamma_i(b_i) = \gamma_{i+1}(a_i)$.
  Define $$\gamma = \gamma_1 + \gamma_2 + \ldots + \gamma_n$$ to be a parametrization that traces out $\gamma_1, \gamma_2, \ldots, \gamma_n$ in order.
\end{defn}

\begin{setup}
  Let $(X,d)$ be the usual metric on $\C$.
  Let $A$ be an open connected subset of $X$ with $\card{A} \geq 2$.
  Let $a_1, a_2 \in A$ such that $a_1 \not = a_2$.
  Show that $a_1$ and $a_2$ can be connected by a polygonal path that stays inside $A$ and has sides parallel to the axis.

  \begin{proof}
    %Observe that since $A$ is open and connected, $A$ is path connected.
    Let $\gamma \colon [a,b] \rightarrow A$ be a path connecting $a_1$ and $a_2$.
    Let $r$ be as in Lemma~\ref{min-radius} and let $P = [t_1, t_2, \ldots, t_n]$ be a partition of $[a,b]$ as in Lemma~\ref{partition}.
    Let $z_j = x_j + iy_j = \gamma(t_j)$ for each $1 \leq j \leq n$ and let $D_j \subseteq A$ be the disc of radius $r$ centered about $z_j$.
    For each pair $z_j, z_{j+1}$, let $\omega_j = x_{j+1} + iy_j$.
    Define the rectilinear path $\eta_j : [a_j, b_j] \rightarrow A$ which parametrizes the straight line from $z_j$ to $\omega_j$, then the straight line from $\omega_j$ to $z_{j+1}$.
    Let $z = x + iy \in \eta_j([a_j, b_j])$ and observe 
    %that $\abs{z_i - \omega_i} = \abs{x_i - x_{i+1}}$ and $\abs{z_{i+1} - \omega_i} = \abs{y_i - y_{i+1}}$
    $$\abs{z_j - z} = \sqrt{(x_j - x)^2 + (y_j - y)^2} \leq \sqrt{(x_j - x_{i+j})^2 + (y_j - y_{j+1})^2} = \abs{z_j - z_{j+1}} < r.$$
    %Then by the Pythagorean Theorem we have
    %\sqrt{\abs{z_i - \omega_i}^2 + \abs{\omega_i - z_{i+1}}^2} =
    %$$ \abs{z_i - z}\leq \sqrt{(x_i - x_{i+1})^2 + (y_i - y_{i+1})^2} = \abs{z_i - z_{i+1}} < r $$
    Hence $\eta_j([a_j, b_j])$ lies inside $D_j \subseteq A$.
    Therefore the curve $$\eta = \eta_1 + \eta_2 + \ldots + \eta_n$$ is a rectilinear path that stays inside $A$ with sides parallel to the axes.
    
  \end{proof}
\end{setup}

\begin{setup}
  Let $F \colon X \times Y \rightarrow Z$ be a function from the product of two metric spaces into a metric space.
  \begin{enumerate}[(a)]
  \item
    Suppose that $(x,y) \mapsto F(x,y)$ is continuous and that $Y$ is compact.
    Prove that $F(x,\cdot)$ tends to $F(x_0, \cdot)$ uniformly on $Y$ as $x$ tends to $x_0$.
  \item
    Conversely suppose $(x,y) \mapsto F(x,y)$ is continuous except possibly at points $(x,y) = (x_0, y)$, and suppose that $F(x, \cdot) \rightarrow F(x_0, \cdot)$ uniformly.
    Prove that $F$ is continuous everywhere.
  \end{enumerate}
  \begin{proof}
    Let $d_X$, $d_Y$, and $d_Z$ be the metrics on $X$, $Y$, and $Z$, respectively, and let $d$ be the product metric defined by $d((x_1,y_1), (x_2,y_2)) = \max \left\{d_X(x_1, x_2), d_Y(y_1, y_2)\right\}$.
    \begin{enumerate}[(a)]
    \item
      Let $\varepsilon > 0$ be given and fix some $x_0 \in X$.
      Since $F$ is continuous, there exists for each point $(x_0, y)$ a $\delta_y > 0$ such that 
      \begin{equation}\label{18.1}
        F\left(B_d \left( \left(x_0,y\right), \delta_y\right)\right)\subseteq B_{d_Z}\left(F \left(x_0, y\right), \frac{\varepsilon}{2}\right).
      \end{equation}
      Observe that the collection of open balls $\mathcal{B} = \left\{B_{d_Y}(y, \delta_y) \;\middle\vert\; y \in Y\right\}$ is an open cover of $Y$.
      Since $Y$ is compact, there exists a finite subcover 
      $$B_1 = B_{d_Y}\left(y_1, \delta_{y_1}\right), B_2 = B_{d_Y}\left(y_2, \delta_{y_2}\right), \ldots, B_k = B_{d_Y}\left(y_k, \delta_{y_k}\right),$$ 
      for some $y_1, y_2, \ldots, y_k \in Y$.

      Take $\delta = \min \left\{\delta_{y_1}, \delta_{y_2}, \ldots, \delta_{y_k}\right\}$.
      Observe that for each $y \in Y$,  $y \in B_i$ for some $1 \leq i \leq k$ and thus, so long as $d(x, x_0) < \delta \leq \delta_{y_i}$, both $(x, y)$ and $(x_0, y)$ lie in the open ball $B_d\left(\left(x_0, y_i\right), \delta_{y_i}\right)$.
      Hence by \eqref{18.1}, $d_X(x,x_0) < \delta$ implies
      \begin{align*}\label{inequalities}\tag{$\ast$}
        d_Z\left(F\left(x, y\right), F\left(x_0, y_i\right)\right) &< \frac{\varepsilon}{2} & \text{and}&& d_Z\left(F\left(x_0, y_i\right), F\left(x_0, y\right)\right) &< \frac{\varepsilon}{2}.
      \end{align*}
      Therefore if $d_X(x,x_0) < \delta$, then for all $y \in Y$ it follows from \eqref{inequalities} and the triangle inequality that
      $$d_Z(F(x, y), F(x_0, y)) < \frac{\varepsilon}{2} + \frac{\varepsilon}{2} = \varepsilon$$
      and the convergence is uniform, as desired.
    \item
      Let $\varepsilon > 0$ be given.
      Observe that since $F$ is continuous except possibly at $(x,y) = (x_0, y)$, the function $F(x_0, \cdot)$ is the uniform limit of continuous functions.
      Hence by Proposition 2.21, $F(x_0, \cdot)$ is continuous.
      By uniform convergence, there exists some $\delta_1 > 0$ such that for all $y \in Y$
      \begin{equation}\label{18.2}
        d_Z\left(F\left(x,y\right), F\left(x_0, y\right)\right) < \frac{\varepsilon}{2}
      \end{equation}
      holds whenever $d(x, x_0) < \delta_1$.
      By continuity, there exists some $\delta_2 > 0$ such that if $d_Y(y,y_0) < \delta_2$, then
      \begin{equation}\label{18.3}
        d_Z\left(F\left(x_0,y\right), F\left(x_0, y_0\right)\right) < \frac{\varepsilon}{2}
      \end{equation}

      Take $\delta = \min \left\{\delta_1, \delta_2\right\}$ and consider some $(x,y)$ in the open ball $B_d((x_0, y_0), \delta)$.
      It then follows from the definition of the metric that $d_X(x, x_0) < \delta \leq \delta_1$ and $d_Y(y,y_0) < \delta \leq \delta_2$.
      Hence by \eqref{18.2}, \eqref{18.3}, and the triangle inequality,
      $$d_z\left(F\left(x,y\right), F\left(x_0,y_0\right)\right) < \frac{\varepsilon}{2} + \frac{\varepsilon}{2} = \varepsilon,$$
      and $F$ is continuous on $X \times Y$, as desired.
    \end{enumerate}
  \end{proof}
\end{setup}

\begin{setup}
  Give an example of a continuous function between two metric spaces that fails to carry some Cauchy sequence to a Cauchy Sequence.
  \begin{proof}
    Take the metric space $(\mathbb{R}^{>0}, \abs{\cdot})$, where $\abs{\cdot}$ is the usual metric.
    The function $f(x) = 1/x$ is continuous, but the sequence $\left\{f(1/n)\right\}_{n=1}^\infty = \left\{n\right\}_{n=1}^\infty$ does not converge, hence is not Cauchy.
  \end{proof}
\end{setup}

\begin{setup}
  {\bf (Contraction mapping principle)} Let $(X,d)$ be a complete metric space, let $r$ be a number with $0 \leq r < 1$, and let $f \colon X \rightarrow X$ be a {\bf contraction mapping}, i.e., a function such that $d(f(x), f(y)) \leq rd(x,y)$ for all $x$ and $y$ in $X$.
  Prove that there exists a unique $x_0$ in $X$ such that $f(x_0) = x_0$.
  
  \begin{proof}
    First we show that $f$ is continuous.
    Let $\varepsilon > 0$ be given.
    Observe that if $r = 0$, then for $x,y \in X$, $d(f(x),f(y)) \leq rd(x,y) = 0 < \varepsilon$ implies $f$ is continuous.
    Hence we may assume $0 < r < 1$.
    Take $\delta = \varepsilon/r$ and then for $d(x,y) < \delta$, we have $$d(f(x), f(y)) \leq rd(x,y) < r\left(\frac{\varepsilon}{r}\right) = \varepsilon.$$
    Therefore $f$ is continuous.
    
    Fix some $x \in X$ and construct the sequence $y_1 = x$, $y_{n+1} = f(y_n)$.
    We now show $\left\{y_n\right\}_{n=1}^\infty$ is Cauchy.
    Towards that end, observe that if $n=2$, then 
    $$d(y_n, y_{n+1}) = d(y_2, y_3) \leq rd(y_1, y_2) = r^{n-1}d(x,f(x)).$$
    Assume that $d(y_n, y_{n+1}) \leq r^{n-1} d(x,f(x))$ holds up to some $n$.
    Then we have $$d\left(y_{n+1}, y_{n+2}\right) \leq r d\left(y_n, y_{n+1}\right) \leq r\left(r^{n-1}d\left(x, f\left(x\right)\right)\right) = r^n d\left(x, f\left(x\right)\right),$$
    and the inequality holds for all $n \in \N$.
    Now consider for $m < n \in \N$ the distance $d\left(y_m, y_n\right)$.
    Apply the triangle inequality $n-m$ times and rearrange using $d(y_n, y_{n+1}) \leq r^{n-1} d\left(x,f\left(x\right)\right)$ to obtain the inequality 
    $$d(y_m, y_n) \leq d(y_m, y_{m+1}) + d(y_{m+1}, y_{m+2}) + \ldots + d(y_{n-1}, y_n) \leq d(x,f(x)) \sum_{j=m-1}^{n-2}r^j$$
    Let $s = \sum_{j=m-1}^{n-2}r^j$ and observe that the difference $s - rs = s(1-r)$ telescopes to the closed form $s(1-r) = r^{m-1} - r^{n-1}$. 
    Since $0 \leq r < 1$, it follows that  $$s = \frac{r^{m-1} - r^{n-1}}{1-r} = r^{m-1}\left(\frac{1 - r^{n-m}}{1-r}\right) \leq r^{m-1}\left(\frac{1}{1-r}\right).$$ 
    Hence %one may use the usual argument for a geometric sum to obtain
    \begin{equation}\label{19.1}
      d(y_m, y_n) \leq d(x,f(x))s \leq r^{m-1}\left(\frac{d(x,f(x))}{1-r}\right).
      %\left(\frac{r^{m-1} - r^{n-1}}{1-r}\right) = d(x,f(x))r^{m-1}\left(\frac{1 - r^{n-m}}{1-r}\right) \leq r^{m-1}\left(\frac{d(x,f(x))}{1-r}\right).
    \end{equation}
    Now let $\varepsilon > 0$ be given.
    Observe that $r^n \rightarrow 0$ as $n \rightarrow \infty$ implies we may choose $N \in \N$ such that 
    $$r^{n} < \frac{1-r}{d(x,f(x))}\varepsilon$$
    holds for all $n \geq N$.
    Therefore, if $N < m < n$, then by \eqref{19.1}
    $$d(y_m, y_n) \leq r^{m-1}\left(\frac{d(x,f(x))}{1-r}\right) < \left(\frac{1-r}{d(x,f(x))}\right)\left(\frac{d(x,f(x))}{1-r}\right)\varepsilon = \varepsilon,$$
    implies $\left\{y_n\right\}_{n=1}^{\infty}$ is Cauchy.
    Moreover, since $(X,d)$ is complete, there exists some $y \in X$ such that $y = \lim_{n \rightarrow \infty} y_n$.
    
    To see that $f$ has a fixed point, let $\varepsilon > 0$ be given.
    Since $f$ is continuous, we may choose some $\delta > 0$ such that $d(f(x), f(y)) < \varepsilon/2$ holds for all $x \in X$ satisfying $d(x,y) < \delta$.
    If necessary, reduce $\delta$ so that $\delta \leq \varepsilon/2$.
    Choose $N \in \N$ such that $d(y_n, y) < \delta$ holds for all $n \geq N$ and consider the distance $d(y,f(y))$.
    By the triangle inequality, we have 
    $$d(y, f(y)) \leq d(y, y_{N+1}) + d(y_{N+1}, f(y)) = d(y, y_{N+1}) + d(f(y_N), f(y))< \frac{\varepsilon}{2} + \frac{\varepsilon}{2} = \varepsilon.$$
    Since the choice of $\varepsilon$ was arbitrary, it follows that $y = f(y)$.
    
    Finally, we show that the fixed point $y$ is unique.
    %Observe that the choice of $x$ in the original construction was arbitrary and thus such a construction works with any choice of $x$.
    Assume $y^\prime$ is another fixed point of $f$.
    Consider the distance $d(y, y^\prime)$.
    Since $y = f(y)$ and $y^\prime = f(y^\prime)$, it follows that 
    $$d(y, y^\prime) = d(f(y), f(y^\prime)) \leq rd(y,y^\prime).$$
    Subtracting $rd(y, y^\prime)$ from both sides of the inequality and rearranging yields
    $$d(y, y^\prime)(1-r) \leq 0.$$
    Noting that $0 < 1 - r \leq 1$ and $0 \leq d(y, y^\prime)$, divide both sides by $1-r$ to obtain $0 \leq d(y, y^\prime) \leq 0$.
    Therefore $d(y, y^\prime) = 0$ implies $y = y^\prime$ and $y$ is unique, as desired.
  \end{proof}
\end{setup}

\end{document}
