\documentclass[10pt]{amsart}
\renewcommand{\baselinestretch}{1.5}
\usepackage{amsmath,amsthm,amssymb,amsfonts,enumerate,paralist}
\openup 5pt
\author{Blake Farman\\University of South Carolina}
\title{Math 703:\\Homework 03}
\date{October 9, 2012}
\pdfpagewidth 8.5in
\pdfpageheight 11in
\usepackage[margin=1in]{geometry}
\begin{document}
\maketitle

\providecommand{\norm}[1]{\lVert#1\rVert}
\renewcommand{\qedsymbol}{\ensuremath{\blacksquare}}
\newcommand{\abs}[1]{\left| #1 \right|}
\newcommand{\dist}[2]{\operatorname{dist}\left(#1,#2\right)}

\newtheorem*{ex1}{p. 541, 1}
\newtheorem*{ex2a}{p. 542, 2a}
\newtheorem*{ex3}{p. 542, 3}
\newtheorem*{ex12}{p. 542, 12}
\newtheorem{lem}{Lemma}

\begin{ex1}
  Suppose that $T_1, T_2, \ldots, T_k$ are compact sets in a metric space $(A, \rho)$.
  Show that $\cup_{j=1}^k T_j$ is compact.
  \begin{proof}
    Let $\mathcal{U}$ be an open cover of $\cup_{j=1}^k T_j$.
    Since $T_i \subseteq \cup_{j=1}^k T_j$ for each $i \in \left\{1, \ldots, k\right\}$, there exist for each $i$ a sub-collection, $\mathcal{U}_i$, of $\mathcal{U}$ such that $\mathcal{U}_i$ covers $T_i$.
    Since each $T_i$ is compact, there exists a finite subcollection, $\tilde{\mathcal{U}}_i$, of $\mathcal{U}_i$ that covers $T_i$.
    Then $\cup_{j=1}^k \tilde{\mathcal{U}}_j$ is a finite union of a finite number of open sets from $\mathcal{U}$ that covers $\cup_{j=1}^k T_j$, hence a finite open subcover.
    Therefore $\cup_{j=1}^k T_j$ is compact.
  \end{proof}
\end{ex1}

\begin{ex2a}
  Show that a closed subset of a compact set is compact.
  \begin{proof}
    Let $F$ be a closed subset of the compact set $K$.
    Let $\left\{x_n\right\}_{n=1}^{\infty}$ be a sequence contained in $F$.
    Since $F \subseteq K$ and $K$ is compact, hence sequentially compact by Theorem 2.36 (Knapp), there exists a subsequence $\left\{x_{n_j}\right\}_{j=1}^{\infty}$ with a limit, say $x$, in $K$.
    However, $\left\{x_{n_j}\right\}_{j=1}^{\infty}$ is contained in $F$, a closed set.
    Hence by Corollary 2.23 (Knapp), we have that $x \in F$ and thus $F$ is sequentially compact.
    Therefore, by Theorem 2.36 (Knapp),  $F$ is compact.
  \end{proof}
\end{ex2a}

\begin{ex3}
  If $S$ and $T$ are non-empty subsets of a metric space $(A,\rho)$, we define the distance from $S$ to $T$ by
  $$ \dist{S}{T} = \inf \left\{\rho(s,t) \;\middle\vert\; s \in S, t \in T\right\}.$$
  Show that if $S$ and $T$ are compact, then $\dist{S}{T} = \rho(s,t)$ for some $s$ in $S$ and some $t$ in $T$.

  \begin{proof}
    Observe that if we define 
    \begin{align*}
      \rho_2: \left(A \times A\right) \times \left(A \times A\right) &\rightarrow \mathbb{R}^{\geq 0}\\
      ((a_1, a_2), (b_1, b_2)) &\mapsto \rho(a_1, b_1) + \rho(a_2, b_2)
    \end{align*}
    then $(A \times A, \rho_2)$ is a metric space.
    We first show that $\rho$ is continuous on $(A \times A, \rho_2)$.
    Towards that end, let $\varepsilon > 0$ be given, fix $a = (a_1, a_2) \in A \times A$ and let $b = (b_1,b_2) \in A \times A$ be such that $\rho_2(a,b) < \varepsilon$.
    Observe that after two applications of the triangle inequality for $\rho$ we have 
    $$\rho(b_1, b_2) \leq \rho(a_1, b_1) + \rho(a_1, a_2) + \rho(a_2, b_2) = \rho_2(a,b) + \rho(a) < \rho(a) + \varepsilon.$$
    By the same argument, mutatis mutandis, we find $\rho(a) < \rho(b) + \varepsilon$.
    Combining these two inequalities, we have $\rho(a) - \varepsilon < \rho(b) < \rho(a) + \varepsilon$.
    Hence $\rho_2(a,b) < \delta = \varepsilon$ implies $\abs{\rho(a) - \rho(b)} < \varepsilon$, and $\rho$ is continuous, as desired.
    
    Next, we show that $S \times T$ is compact.
    Let $\left\{ \left(s_n, t_n\right) \right\}_{n=1}^{\infty}$ be a sequence contained in $S \times T$.
    Since $S$ is compact, hence sequentially compact, we may produce a convergent subsequence $\left\{s_{n_j}\right\}_{j=1}^{\infty}$ of $\left\{s_n\right\}_{n=1}^{\infty}$ that converges to some point $s$ of $S$.
    Similarly, we may then consider $\left\{t_{n_j}\right\}_{j=1}^{\infty}$ as a sequence of $T$ with convergent subsequence, $\left\{s_{n_{j_k}}\right\}_{k=1}^{\infty}$, that converges to some point $t$ of $T$.
    Now, since $\left\{s_{n_j}\right\}_{j=1}^{\infty}$ converges to $s$, so too does $\left\{s_{n_{j_k}}\right\}_{k=1}^{\infty}$.
    Hence $\left\{ \left(s_{n_{j_k}}, t_{n_{j_k}}\right) \right\}_{k=1}^{\infty}$ is a convergent subsequence of $\left\{ \left(s_n, t_n\right) \right\}_{n=1}^{\infty}$, with limit $(s,t) \in S \times T$.
    Therefore $S \times T$ is sequentially compact, hence compact by Theorem 2.36 (Knapp).
    
    Finally, we observe that $\dist{S}{T} = \inf \rho(S \times T)$ and the result then follows from Corollary 2.39.
  \end{proof}
\end{ex3}

\begin{ex12}
  Let $\left\{T_n\right\}_{n=1}^{\infty}$ be a sequence of non-empty closed sets of a metric space such that 
  \begin{enumerate}[(a)]
    \item
      $T_1$ is compact,
    \item
      $T_{n+1} \subseteq T_n$ holds for $n \geq 1$, and
    \item
      $\lim_{n \rightarrow \infty} d(T_n) = 0$.
  \end{enumerate}
  Show that $\cap_{n=1}^{\infty} T_n$ contains exactly one member.
  \begin{proof}
    Observe that each $T_n$ is a closed subset of a compact set and that the collection $\left\{T_n \;\middle\vert\; n \in \mathbb{N}\right\}$ possesses the finite intersection property.
    It follows from Proposition 2.35 (Knapp) that $\cap_{n=1}^{\infty} T_n \not = \emptyset$.
    Let $t_1, t_2 \in \cap_{n=1}^{\infty} T_n$.
    By assumption (c), for every $\varepsilon > 0$, there exists an $N_\varepsilon \in \mathbb{N}$ such that $d(T_{N_\varepsilon}) < \varepsilon$.
    However, since $t_1, t_2 \in \cap_{n=1}^{\infty} T_n$, we have that $t_1, t_2 \in T_{N_\varepsilon}$. 
    Hence $\rho(t_1, t_2) \leq d(T_{N_\varepsilon}) < \varepsilon$ holds for all $\varepsilon$, which implies $t_1 = t_2$.
    Therefore $\cap_{n=1}^{\infty} T_n$ has only one element.
  \end{proof}
\end{ex12}

\end{document}
