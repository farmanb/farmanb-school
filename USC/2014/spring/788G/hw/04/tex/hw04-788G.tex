\documentclass[10pt]{amsart}
\usepackage{amsmath,amsthm,amssymb,amsfonts,enumerate,mymath,mathtools,tikz-cd,mathrsfs}
\openup 5pt
\author{Blake Farman\\University of South Carolina}
\title{Math 788G:\\Homework 04}
\date{April 28, 2014}
\pdfpagewidth 8.5in
\pdfpageheight 11in
\usepackage[margin=1in]{geometry}

\begin{document}
\maketitle

\providecommand{\p}{\mathfrak{p}}
\providecommand{\m}{\mathfrak{m}}
\providecommand{\Deck}[1]{\operatorname{Deck}\left(#1\right)}
%\newcommand{\Res}{\operatorname{Res}}
\newtheorem{thm}{}
\newtheorem{lem}{Lemma}
\newtheorem{prop}{Proposition}
\theoremstyle{definition}
\newtheorem{defn}{Definition}[thm]

\newcommand{\A}{\mathbb{A}}

\begin{thm}\label{Ex1}
  Verify directly, via brute force, that the discriminant of a binary cubic form is $\SL{2}{\Z}$-invariant.
  \begin{proof}
    Let $g = \left(\begin{array}{cc} \alpha & \beta \\ \gamma & \delta\end{array}\right) \in \GL{2}{\Z}$
%      \begin{eqnarray*}
%        f \cdot g(u,v) &=& \left(a\alpha^{3} 
%        + b \alpha^{2} \gamma 
%        + c \alpha  \gamma^{2} 
%        + d \gamma^{3} \right)u^{3}\\
%        &+& \left(3 a \alpha^{2} \beta 
%        + b \alpha^{2}  \delta 
%        + 2 b \alpha \beta \gamma 
%        + 2 c \alpha \gamma \delta 
%        + c \beta \gamma^{2} 
%        + 3 d \gamma^{2} \delta \right)u^{2} v\\
%        &+& \left(3 a \alpha\beta^{2} 
%        + 2 b \alpha \beta \delta 
%        + c \alpha \delta^{2} 
%        + b \beta^{2} \gamma 
%        + 2 c \beta \gamma \delta
%        + 3 d \gamma \delta^{2} \right)u v^{2} \\
%        &+& \left(a \beta^{3} 
%        + b \beta^{2} \delta 
%        + c \beta \delta^{2} 
%        + d \delta^{3} \right)v^{3}
%      \end{eqnarray*}
      Let
  \begin{eqnarray*}
    A &=& a\alpha^{3} 
        + b \alpha^{2} \gamma 
        + c \alpha  \gamma^{2} 
        + d \gamma^{3}\\ 
    B &=& 3 a \alpha^{2} \beta 
        + b \alpha^{2}  \delta 
        + 2 b \alpha \beta \gamma 
        + 2 c \alpha \gamma \delta 
        + c \beta \gamma^{2} 
        + 3 d \gamma^{2} \delta\\
    C &=& 3 a \alpha\beta^{2} 
        + 2 b \alpha \beta \delta 
        + c \alpha \delta^{2} 
        + b \beta^{2} \gamma 
        + 2 c \beta \gamma \delta
        + 3 d \gamma \delta^{2}\\
    D &=& a \beta^{3} 
        + b \beta^{2} \delta 
        + c \beta \delta^{2} 
        + d \delta^{3}\\
  \end{eqnarray*}
  so that $f \cdot g(u,v) = f(\alpha u + \beta v, \gamma u + \delta v) = Au^3 + Bu^2v + Cuv^2 + Du^3$.
  The discriminant for $f \cdot g$ is
  $$B^2C^2 + 18ABCD - 4AC^3 - 4B^3D - 27A^2D^2,$$ 
  which expands to
  \begin{eqnarray*}
    - 27 a^{2} d^{2}\left(\alpha^{6} \delta^{6} 
    - 6 \alpha^{5}\delta^{5} \beta\gamma 
    + 15 \alpha^{4}\delta^{4} \beta^{2}\gamma^{2} 
    - 20 \alpha^{3}\delta^{3} \beta^{3}\gamma^{3} 
    + 15 \alpha^{2}\delta^{2} \beta^{4}\gamma^{4} 
    - 6 \alpha\delta \beta^{5}\gamma^{5} 
    + \beta^{6} \gamma^{6}\right)\\
    + 18a b c d \left( \alpha^{6} \delta^{6} 
    - 6 \alpha^{5}\delta^{5} \beta\gamma 
    + 15 \alpha^{4}\delta^{4} \beta^{2}\gamma^{2} 
    - 20 \alpha^{3}\delta^{3} \beta^{3}\gamma^{3} 
    + 15 \alpha^{2}\delta^{2} \beta^{4}\gamma^{4} 
    - 6 \alpha\delta \beta^{5}\gamma^{5} 
    + \beta^{6}\gamma^{6}\right)\\
    - 4 a c^{3} \left( \alpha^{6}\delta^{6} 
    - 6 \alpha^{5}\delta^{5} \beta\gamma 
    + 15 \alpha^{4}\delta^{4} \beta^{2}\gamma^{2} 
    - 20 \alpha^{3}\delta^{3} \beta^{3}\gamma^{3} 
    + 15 \alpha^{2}\delta^{2} \beta^{4}\gamma^{4} 
    - 6 \alpha\delta \beta^{5}\gamma^{5} 
    + \beta^{6}\gamma^{6}\right)\\
    - 4b^{3} d \left(\alpha^{6} \delta^{6} 
    - 6 \alpha^{5}\delta^{5} \beta\gamma 
    + 15 \alpha^{4}\delta^{4} \beta^{2}\gamma^{2} 
    - 20 \alpha^{3}\delta^{3} \beta^{3}\gamma^{3} 
    + 15 \alpha^{2}\delta^{2} \beta^{4}\gamma^{4} 
    - 6 \alpha\delta \beta^{5}\gamma^{5} 
    + 1 \beta^{6}\gamma^{6}\right)\\
    + b^{2} c^{2} \left(\alpha^{6}\delta^{6} 
    - 6 \alpha^{5}\delta^{5} \beta\gamma 
    + 15 \alpha^{4}\delta^{4} \beta^{2}\gamma^{2} 
    - 20 \alpha^{3}\delta^{3} \beta^{3}\gamma^{3} 
    + 15 \alpha^{2}\delta^{2} \beta^{4}\gamma^{4} 
    - 6 b^{2} c^{2}\alpha\delta \beta^{5}\gamma^{5} 
    + \beta^{6} \gamma^{6}\right).
  \end{eqnarray*}
  This then reduces to 
  \begin{equation}\label{1.1}
    (\alpha\delta - \beta\gamma)^6 \left(b^{2}c^{2} + 18abcd - 4 ac^{3} - 4b^{3}d - 27 a^2 d^2 \right) = \det(g)^6\disc{f}.
  \end{equation}
  Since $g \in \GL{2}{\Z}$, we have $\det(g) \in \Z^\times = \{\pm 1\}$ and so the right hand side of \eqref{1.1} reduces to $(\pm 1)^6\disc{f} = \disc{f}$.
  We now note that $\SL{2}{\Z} \subseteq \GL{2}{\Z}$ and the result follows.
  \end{proof}
\end{thm}

\begin{thm}
  Let $f(u,v) = u^3 + a_2u^2v + a_3uv^2 + v^3$ be a binary cubic form.
  Prove that its discriminant is equal to the polynomial discriminant obtained by setting either $u$ or $v$ equal to 1.

  \begin{proof}
    Let $g(u) = f(u,1) = u^3 + a_2u^2 + a_3u + 1$ and $h(v) = f(1,v) = v^3 + a_3v^2 + a_2v + 1$.
    By the formula for the discriminants we have
    \begin{eqnarray*}
      \disc{f} &=& a_2^{2}a_3^{2} + 18a_2a_3 - 4 a_3^{3} - 4a_2^{3} - 27,\\
      \disc{g} &=& a_2^{2}a_3^{2} + 18a_2a_3 - 4 a_3^{3} - 4a_2^{3} - 27,\, \text{and}\\
      \disc{h} &=& a_3^{2}a_2^{2} + 18a_3a_2 - 4 a_2^{3} - 4a_3^{3} - 27.
    \end{eqnarray*}
    Provided the ring over which $f$ takes its coefficients is commutative, $\disc{g} = \disc{h} = \disc{f}$.
  \end{proof}
\end{thm}

\begin{thm}
  Let $f(u,v) = a_nu^n + a_{n-1}u^{n-1}v + \ldots + a_0v^n$.
  Prove
  \begin{enumerate}[(a)]
    \item
      If $a_n \neq 0$, then
      $$\disc{f} = a_n^{2n - 2}\disc{f(u,1)}$$
    \item
      If $a_n = 0$ and $a_{n-1} \neq 0$, then
      $$\disc{f} = a_{n-1}^2 \disc{a_{n-1}u^{n-1} + a_{n-2}u^{n-2}v + \ldots + a_0v^{n-1}}.$$
    \item
      If $a_n = 0$ and $a_{n-1} = 0$, then
      $$\disc{f} = 0.$$
  \end{enumerate}
  
  \begin{proof}
    \begin{enumerate}[(a)]
    \item\label{1.a}
      We observe that we may write 
      $$f(u,v) = \prod_{i = 1}^n (\alpha_iu - \beta_iv)$$
      and thus $a_n = \alpha_1\alpha_2 \cdots \alpha_n \neq 0$ implies $\alpha_i \neq 0$ holds for each $i$.
      The discriminant for $f$ is given by
      $$\disc{f} = \prod_{i < j} (\alpha_i\beta_j - \alpha_j\beta_i)^2 = \prod_{i < j} \alpha_i^2\alpha_j^2\left(\frac{\beta_j}{\alpha_j} - \frac{\beta_i}{\alpha_i}\right)^2.$$
      
      We first show that $\prod_{i < j} \alpha_i^2\alpha_j^2 = a_n^{2n - 2}$.
      We observe that the distinct elements in this product are the elements in the table below
      $$\begin{array}{c||cccccc}
        i \backslash j & 2 & 3 & 4 & 5 & \ldots &n\\
        \hline
        1 &  \alpha_1\alpha_2 & \alpha_1\alpha_3 & \alpha_1\alpha_4 & \alpha_1\alpha_5 & \ldots & \alpha_1,\alpha_n\\
        2 && \alpha_2\alpha_3 & \alpha_2\alpha_4 & \alpha_2\alpha_5 & \ldots & \alpha_2\alpha_n\\
        3 &&       & \alpha_3\alpha_4 & \alpha_3\alpha_5 & \ldots & \alpha_3\alpha_n\\
        4 &&       &       & \alpha_4,\alpha_5 & \ldots & \alpha_4\alpha_n\\
        \vdots &&       &       &       & \ddots & \vdots\\
        &&&&&&\alpha_{n-1},\alpha_n
      \end{array}$$
      We observe then that $\alpha_i$ appears $n - i$ times in the $i^\text{th}$ row, and once in each of the $i - 1$ rows before it, and so it follows that
      $$\prod_{i < j} \alpha_i^2\alpha_j^2 = \left[\prod_{i < j} \alpha_i\alpha_j\right]^2 = \left[\prod_{i < j} \alpha_i^{n-i + i + 1}\right]^2 = \prod_{i < j}(\alpha_i^{n-1})^2 = a_n^{2n-2}.$$
      Hence 
      $$\disc{f} = \prod_{i < j} (\alpha_i\beta_j - \alpha_j\beta_i)^2 = a_n^{2n-2}\prod_{i < j} \left(\frac{\beta_j}{\alpha_j} - \frac{\beta_i}{\alpha_i}\right)^2.$$
      
      %We now observe that if we assume $v \neq 0$, then by homogeneity
      %$$\frac{1}{v^n}f(u,v) = f\left(\frac{u}{v}, 1\right) = \prod_{i = 1}^n \left(\alpha_i \frac{u}{v} - \beta_i\right)$$
      We now obseve that when $v = 1$
      $$f(u,1) = \prod_{i = 1}^n \left(\alpha_iu - \beta_i\right)$$
      has as its roots the elements $\beta_i/\alpha_i$, for $i = 1, \ldots, n$.
      Therefore it follows from 
      $$\disc{f(u,1)} = \prod_{i < j} \left(\frac{\beta_j}{\alpha_j} - \frac{\beta_i}{\alpha_i}\right)^2$$
      that $\disc{f} = a_n^{2n - 2}\disc{f(u,1)}$.
    \item
      We observe first that $g(u,v) = a_{n-1}u^{n-1} + a_{n-2}u^{n-2}v + \ldots + a_0v^{n-1}$ is a binary $(n-1)$-ic form, so by Part~\ref{1.a}
      \begin{eqnarray*}
        a_{n-1}^2 \disc{g} &=& a_{n-1}^2 a_{n-1}^{2(n-1)-2} \disc{g(u,1)}\\
        &=& a_{n-1}^{2n-2} \disc{a_{n-1}u^{n-1} + a_{n-2}u^{n-2} + \ldots + a_0}
      \end{eqnarray*}
      If $a_n = 0$, then we may write
      \begin{eqnarray*}
        f(u,v) &=& v\left(a_{n-1}u^{n-1} + a_{n-2}u^{n-2}v + \ldots + a_0v^{n-1}\right)\\
        &=& (0u - v)\prod_{i = 1}^{n-1} (\alpha_i u - \beta_i v).
      \end{eqnarray*}
      Let $\alpha_n = 0$ and $\beta_n = 1$.
      %First we compute 
      %$$\prod_{i < j < n}(\alpha_i\beta_j - \alpha_j\beta_i)^2.$$
      Since $a_{n-1} \neq 0$ we may assume $\alpha_i \neq 0$ for $\alpha_i < n,$ and so by the same analysis, mutatis mutandis, as in Part~\ref{1.a}, we find that
      $$\prod_{i < j < n}(\alpha_i\beta_j - \alpha_j\beta_i)^2 = a_{n-1}^{2n - 4}\disc{g(u,1)}.$$
      To complete the computation, we observe that for $i < n$, we have
      $$\alpha_i\beta_n -\alpha_n\beta_i = \alpha_i.$$
      Therefore
      \begin{eqnarray*}
        \disc{f} &=& \prod_{i < j}(\alpha_i\beta_j - \alpha_j\beta_i)^2\\
        &=& \prod_{i < n} (\alpha_i\beta_n - \alpha_n\beta_i)^2\prod_{i < j < n}(\alpha_i\beta_j - \alpha_j\beta_i)^2\\
        &=& \prod_{i = 1}^{n-1} \alpha_i^2\prod_{i < j < n}(\alpha_i\beta_j - \alpha_j\beta_i)^2\\
        &=& a_{n-1}^2 a_{n-1}^{2n - 4}\disc{g(u,1)}\\
        &=& a_{n-1}^{2n - 2}\disc{g(u,1)}\\
        &=& a_{n-1}^2 \disc{a_{n-1}u^{n-1} + a_{n-2}u^{n-2}v + \ldots + a_0v^{n-1}}.
      \end{eqnarray*}
    \item
      If $a_n = a_{n-1} = 0$, then 
      $$f(u,v) = v^2\left(a_{n-2}u^{n-2} + a_{n-3}u^{n-3}v + \ldots + a_0v^{n-2}\right),$$
      has a repeated root.
      Therefore $\disc{f} = 0$.
      
    \end{enumerate}
  \end{proof}
\end{thm}

\begin{thm}
  Carry out the details of the computation given on p. 23.5 of the lecture notes.

  \begin{proof}
    Let $f(u,v) = u^3 + bu^2v + cuv^2 + dv^3$ and $g(u) = f(u,1) = (u - r)(u - s)(u - t)$.
    Let $v = u + b/3$ so that
    \begin{eqnarray*}
      g(u) &=& u^3 + bu^2 + 3\left(\frac{b}{3}\right)^2u + \left(\frac{b}{3}\right)^3 + \left(c - 3\left(\frac{b}{3}\right)^2\right)u  + \left(d - \left(\frac{b}{3}\right)^3\right)\\
      &=& \left(u + \frac{b}{3}\right)^3 + \left(c - 3\left(\frac{b}{3}\right)^2\right)u + \left(d - \left(\frac{b}{3}\right)^3\right)\\
      &=& v^3 + \left(c - 3\left(\frac{b}{3}\right)^2\right)\left(v - \frac{b}{3}\right) + \left(d - \left(\frac{b}{3}\right)^3\right)\\
      &=& v^3 + \left(c - \frac{b^2}{9}\right)v + \left(d - \frac{bc}{3} + 2\frac{b^3}{27}\right)\\
      &=& v^3 + Cv + D\\
      &=& (v - R)(v - S)(v - T).
    \end{eqnarray*}
    Expanding and equating coefficients, we see that $\sigma_1 = R + S + T = 0$, $\sigma_2 = RS + RT + ST = C$, and $\sigma_3 = RST = - D$.
    We can compute the discriminant for $f$ using the relations on the elementary symmetric functions and the formula
    \begin{eqnarray*}
      -\disc{f} &=& g^\prime(R)g^\prime(S)g^\prime(T) = (3R^2 + C)(3S^2 + C)(3T^2 + C)\\
      &=& 27R^2S^2T^2 + 9C(R^2S^2 + R^2T^2 + S^2T^2) + 3C^2(R^2 + S^2 + T^2) + C^3\\
      &=& 27\sigma_3^2 + 9C(\sigma_2^2 - 2\sigma_1\sigma_3) + 3C^2(\sigma_1^2 - 2\sigma_2) + C^3\\
      &=& 27(-D)^2 + 9C(C^2) + 3C^2(-2C) + C^3\\
      &=& 27D^2 + 4C^3.
    \end{eqnarray*}
    Therefore $\disc{f} = -27D^2 - 4C^3 = -4 \, b^{3} d + b^{2} c^{2} + 18 \, b c d - 4 \, c^{3} - 27 \, d^{2}$.
  \end{proof}
\end{thm}

\setcounter{thm}{5}
\begin{thm}
  Prove that there are $\frac{1}{3}(p^2 - 1)(p^2 - p)$ irreducible binary cubic forms over $\F_p$.

  \begin{proof}
    Let $X$ be the set of irreducible binary cubic forms.
    By Delone-Faddeev, the set of irreducible binary cubic forms modulo the twisted $\GL{2}{\F_p}$ action is in bijection with the cubic extensions of $\F_p$.
    Since there is a unique cubic extension $\F_{p^3}/\F_p$, the action is necessarily transitive and so for any $f \in X$,
    $$\abs{X} = \left[\GL{2}{\F_p} \colon \GL{2}{F_p}_f \right] = \frac{\abs{\GL{2}{F_p}}}{\abs{\GL{2}{F_p}_f}}$$
    
    We observe that an element $A \in \Mat{2}{\F_p}$ is invertible if and only if the columns of the matrix form a basis for $\F_p^2$.
    There are $p^2 - 1$ non-zero choices for the first column, and excluding the columns not in the span of the first, there $p^2 - p$ choices for the second column.
    Hence $\abs{\GL{2}{\F_p}} = (p^2 - 1)(p^2 - p)$.
    It remains to show that the stabilizer of an element $f \in X$ has order three.
    
    Fix an element $f \in X$ and write $f = \prod_{i=1}^n(r_iu - s_iv)$.
    Let $\theta = s_i/r_i$ and observe that $\F_{p^3} \cong \F_p(\theta) \cong \F_p(\theta^p) \cong \F_p(\theta^{p^2})$ and $\Gal{\F_{p^3}/\F_p} \cong \Z/3\Z$ with generator
    \begin{align*}
      \sigma \colon \F_{p^3} & \rightarrow \F_{p^3}\\
      \alpha &\mapsto \alpha^p.
    \end{align*}
    By the proof of Delone-Faddeev, there exists an element $g = \left(\begin{array}{cc}\alpha & \beta\\ \gamma & \delta\end{array}\right)$ of $\GL{2}{\F_p}$ such that
      $$\sigma(\theta) = \theta^p = \theta \cdot g = \frac{\alpha \theta + \beta}{\gamma \theta + \delta}.$$
      Now we observe that 
      $$\theta^{p^2} = \sigma(\theta^p) = \sigma(\theta \cdot g) = \frac{\sigma\alpha \sigma\theta + \sigma\beta}{\sigma\gamma\sigma\theta + \sigma\delta} = \frac{\alpha\theta^p + \beta}{\gamma\theta^p + \delta} = \theta^p \cdot g$$
      and, similarly,
      $$\theta = \sigma(\theta^{p^2}) = \sigma(\theta^p \cdot g) = \frac{\sigma\alpha \sigma\theta^p + \sigma\beta}{\sigma\gamma\sigma\theta^p + \sigma\delta} = \frac{\alpha\theta^{p^2} + \beta}{\gamma\theta^{p^2} + \delta} = \theta^{p^2} \cdot g.$$
      Then we note that $g$ is not the identity matrix, giving a non-trivial automorphism of $f$ since it fixes the roots.
      The third automorphism of $f$ is the matrix that sends $\theta$ to $\theta^2$.
  \end{proof}
\end{thm}

\setcounter{thm}{9}
\begin{thm}
  Is the following true or false?
  
  Consider the cubic ring $\Z[\alpha]$, where $\alpha^3 + b\alpha^2 + c\alpha + d = 0$.
  Then, the corresponding cubic form is $u^3 + bu^2v + cuv^2 + dv^3$.
  
  \begin{proof}
    This is false.
    Let $\omega = \alpha + b$ and $\omega = \alpha^2 + c$.
    This is a basis; for any $x \in \Z[\alpha]$, we see
    $$x = \ell + m\alpha + n\alpha^2 = (\ell - mb - nc) + m\omega + n\theta$$
    and if , then
    $$0 = \ell + m\omega + n\theta = (\ell + mb + nc) + m\alpha + n\alpha^2$$
    implies that $m = n = \ell = 0$.
    Moreover, from
    $$\omega\theta = \alpha^3 + b\alpha^2 + c\alpha + bc = -d + bc \in \Z$$
    we see that this is a normal basis.
    Then we can compute the corresponding form using Delone-Faddeev:
    $$\omega^2 = (-b^2 - c) + 2b(\alpha + b) + (\alpha^2 + c) = (-b^2 - c) - (-2b)\omega + \theta$$
    and
    \begin{eqnarray*}
      \theta^2 &=& \alpha^4 + 2c\alpha^2 + c^2\\
      &=& (bd + c^2) + bc\alpha + (b^2 + 2c)\alpha^2\\
      %&=& (bd + c^2) + bc\alpha + b^2c - b^2c + (b^2 + 2c)\alpha^2 + c(b^2+2c) - c(b^2 + 2c)\\
      &=& (bd + c^2 - c(b^2 + 2c) - b^2c) + bc(\alpha + b) + (b^2 + 2c)(\alpha^2 + c)\\
      &=& (bd - c^2 - 2b^2c) - (-bc)\omega + (b^2 + 2c)\theta,
    \end{eqnarray*}
    giving the form
    $$f(u,v) = u^3 + (-2b)u^2v + (-bc)uv^2 + (b^2 + 2c)v^3$$
    which is not the desired form.
    Indeed, we show that it is not, in general, $\GL{2}{\Z}$-equivalent to the desired form.
    
    Let $h(u,v) = u^3 + bu^2v + cuv^2 + dv^3$ and suppose there is an element $g = \left(\begin{array}{cc} \alpha & \beta\\ \gamma & \delta\end{array}\right) \in \GL{2}{\Z}$ such that 
      \begin{eqnarray}\label{10.1}
        f(u,v) &=& h \cdot g(u,v)\\
        &=& \frac{1}{\alpha\delta - \beta\gamma}h(\alpha u + \beta v, \gamma u + \delta v)\\ 
        &=& \pm h(\alpha u + \beta v, \gamma u + \delta v).\label{10.1.3}
      \end{eqnarray}
      If the right-hand side of \eqref{10.1.3} is negative, then by homogeneity we note that
      $$- h(\alpha u + \beta v, \gamma u + \delta v) = (-1)^3 h(\alpha u + \beta v, \gamma u + \delta v) = h(-\alpha u - \beta v, -\gamma u - \delta v)$$ and so by possibly replacing $g$ with $-g$ we may assume that $f(u,v) = h(\alpha u + \beta v, \gamma u - \delta v)$.
      Then we see that by Problem~\ref{Ex1}
      $$\disc{f} = \disc{h(\alpha u + \beta v, \gamma u + \delta v)} = (\alpha\delta - \beta\gamma)^6\disc{h} = (\pm 1)^6\disc{h} = \disc{h}.$$
      However,
      \begin{eqnarray*}
        \disc{f} - \disc{h} &=& (-4b^3 d + b^2c^2 + 18bcd - 4c^3 - 27d^2)
        - (b^2 c^2 - 4b^3 - 4c^3 + 18bc - 27)\\
        &=& 4b^3(1 - d) + 18bc(d - 1) + 27(1 - d^2),
      \end{eqnarray*}
      which is not zero unless $d = 1$ or 
      $$d = -\frac{4}{27}b^{3} + \frac{2}{3}bc - 1 = \frac{1}{27}(-4b^3 + 18bc - 27).$$
  \end{proof}
\end{thm}
\end{document}
