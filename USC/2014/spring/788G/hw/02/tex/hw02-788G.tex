\documentclass[10pt]{amsart}
\usepackage{amsmath,amsthm,amssymb,amsfonts,enumerate,mymath,mathtools,tikz-cd,mathrsfs}
\openup 5pt
\author{Blake Farman\\University of South Carolina}
\title{Math 788G:\\Homework 01}
\date{April 16, 2014}
\pdfpagewidth 8.5in
\pdfpageheight 11in
\usepackage[margin=1in]{geometry}

\begin{document}
\maketitle

\providecommand{\p}{\mathfrak{p}}
\providecommand{\m}{\mathfrak{m}}
\providecommand{\Deck}[1]{\operatorname{Deck}\left(#1\right)}
%\newcommand{\Res}{\operatorname{Res}}
\newtheorem{thm}{}
\newtheorem{lem}{Lemma}
\newtheorem{prop}{Proposition}
\theoremstyle{definition}
\newtheorem{defn}{Definition}[thm]

\newcommand{\A}{\mathbb{A}}

\begin{thm}\label{Ex1}
  Let $f(x,y) = ax^2 + bxy + cy^2$ be a quadratic form of discriminant $D \neq 0$.
  Prove that it is positive definite if and only if $D < 0$ and $a < 0$.
  In addition, describe what happens if $D = 0$.

  \begin{proof}
    Assume that $f$ is positive definite and let $g(x) = f(x,1)$.
    First we show that $a > 0$.
    Suppose that $a$ were not positive and observe that if $a = 0$, then $b \neq 0$ since $D \neq 0$.
    Hence one of $\lim_{x \rightarrow \infty}g(x) = -\infty$ or $\lim_{x \rightarrow -\infty}g(x) = -\infty$ must hold.
    Choose $N \in \Z$ sufficiently large so that $g(N) = f(N,1) < 0$, contrary to the assumption that $f$ is positive definite.
    Therefore $a > 0$.
    
    Suppose to the contrary that $D$ were positive, and note that there exist distinct real roots, 
    $$\alpha_1 <  -\frac{b}{2a} < \alpha_2,$$
    of $g$.
    It follows from the fact that $a > 0$ that the vertex of the parabola, $g$, lies below the $x$-axis, whence
    $$4a^2g\left(\frac{-b}{2a}\right) = a(b^2) + b(-b)(2a) + c(2a)^2 = f(-b, 2a) < 0,$$
    contradicting the assumption that $f$ is positive definite.
    Therefore $D < 0$, as desired.

    Conversely, assume that $a > 0$ and $D < 0$.		
    Then $f(x,y) > 0$ whenever $(x,y) \in \Z^2$ follows from
    \begin{equation}\label{1.1}
      4af(x,y) = \left((2ax)^2 + 2(2ax)by + b^2y^2\right) - b^2y^2 + 4acy^2 = (2ax + by)^2 - Dy^2 > 0.
    \end{equation}
    
    When $D = 0$, we observe that $g(x) = f(x,1)$ has a double root at $x = -b/2a$.
    If $f$ is positive definite, then it's clear that $a > 0$, for otherwise $g(x) = f(x,1) < 0$ holds for all $x \in \Z$, contrary to the assumption that $f$ is positive definite.
    Conversely, if $a > 0$ and $D = 0$, then for all $(x,y) \in \Z^2$ we have by Equation~\eqref{1.1}
    $$4af(x,y) = (2ax + by)^2 > 0$$
    and $f(x,y) > 0$ follows directly.
  \end{proof}
\end{thm}

\begin{thm}
  Can a quadratic form be indefinite over $\R$, but only represent positive integers when $x, y \in \Z$?

  \begin{proof}
    Suppose that $f(x,y) = ax^2 + bxy + cy^2$ were indefinite over $\R$ but positive definite over $\Z$.
    Let $g(x) = f(x,1)$.
    There exists some $[\alpha : \beta] \in \mathbb{P}_{\R}^1$ such that $f(\alpha,\beta) < 0$.
    If $\beta = 0$, then $f(\alpha,\beta) = a\alpha^2 < 0$ implies $a < 0$, contradicting Exercise~\ref{Ex1}.
    Assume that $\beta \neq 0$.
    Then we have
    $$g\left(\frac{\alpha}{\beta}\right) = \frac{1}{\beta^2}f(\alpha, \beta) < 0.$$
    Choose integers $m$ and $n$ such that $m < \alpha/\beta < n$ and observe that $g(m), g(n) > 0$ must hold since $f$ is supposed to be positive definite over $\Z$.
    By the Mean-Value Theorem, there exist real numbers $\alpha_1, \alpha_2$ such that 
    $$m < \alpha_1 < \frac{\alpha}{\beta} < \alpha_2 < n$$
    and $g(\alpha_1) = g(\alpha_2) = 0$.
    Hence 
    $$\operatorname{Disc}(f) = \operatorname{Disc}(g) = b^2 - 4ac> 0,$$
    contradicting Exercise~\ref{Ex1}.
    Therefore no such form exists.		
  \end{proof}
\end{thm}

\begin{thm}
  Prove that the action of $\GL{2}{\Z}$ defined in lecture does {\it not} definte a {\it left} action on binary quadratic forms.

  In other words, find $g, g^\prime$ and $f$ for which (if a left action was defined) we would have $g(g^\prime(f)) \neq (gg^\prime)(f)$.
  
  \begin{proof}
    Let $f(x,y) = x^2 + y^2$ and let
    $$g = \left(\begin{array}{cc}
      0 & -1\\
      1 & 0
    \end{array}\right)\ \text{and}\ 
    g^\prime = \left(\begin{array}{cc}
      1 & 1\\
      0 & 1
    \end{array}\right).$$
    We first note that 
    $$g^\prime \left(\begin{array}{c}
      x\\
      y
      \end{array}\right) = \left(\begin{array}{c}x + y\\y\end{array}\right)$$
    and thus
    $$g^\prime \cdot f(x,y) = f(x + y, y) = (x+y)^2 + y^2 = x^2 + 2xy + 2y^2.$$
    Then we note that
    $$g^\prime \left(\begin{array}{c}
      x\\
      y
      \end{array}\right) = \left(\begin{array}{c} -y \\ x \end{array}\right),$$
    from which it follows that
    $$g\cdot (g^\prime(x,y)) = g \cdot (x^2 + 2xy + 2y^2) = (-y)^2 + 2(-y)(x) + x^2 = y^2 - 2xy + x^2.$$
    However,
    $$gg^\prime \left(\begin{array}{c}
      x\\
      y
      \end{array}\right) = \left(\begin{array}{cc}
      0 & -1\\
      1 & 1
    \end{array}\right)\left(\begin{array}{c}
      x\\
      y
      \end{array}\right) = \left(\begin{array}{c}
      -y\\
      x + y
    \end{array}\right) 
    $$
    and thus
    $$(gg^\prime) \cdot f(x,y) = f(-y, x + y) = (-y)^2 + (x + y)^2 = x^2 + 2xy + 2y^2 \neq g \cdot (g^\prime \cdot f(x,y)),$$
    as desired.    
  \end{proof}
\end{thm}

\begin{thm}
  Prove {\it directly} (i.e. do not quote the reduction theorem) that the quadratic forms $x^2 + 5y^2$ and $2x^2 + 2xy + 3y^2$ are not equivalent.
  
  \begin{proof}
    Suppose to the contrary that there were integers $a,b,c,d$ with $ad - bc = 1$ such that
    $$(ax + by)^2 + 5(cx + dy)^2 = (a^2 + 5c^2)x^2 + (2ab + 10cd)xy + (b^2 + 5d^2)y^2 = 2x^2 + 2xy + 3y^2.$$
    Equating coefficients, we immediately see that $a^2 + 5c^2 = 2$ implies that $c = 0$ and so $a^2 = 2$.
    However, $a$ was supposed to be an integer, a contradiction.
    Therefore $x^2 + 5y^2$ and $2x^2 + 2xy + 3y^2$ are not equivalent.
  \end{proof}
\end{thm}

\setcounter{thm}{5}
\begin{thm}
  Compute $h(D)$ for $D = -7, -8, -43, -67, -163, \ldots$.
  \begin{proof}
    \begin{enumerate}[(a)]
    \item
      h(-7) = 1.
      
      We have that $\abs{b} \leq a \leq \sqrt{7/3} < 2$ and so $a = 1$.
      Since $-D = 7 \equiv 3 \pmod{4}$ and $$c = \frac{b^2 - D}{4a} \in \Z$$ it follows that $b \neq 0$.
      When $b = 1$ we have
      $$c = \frac{1 + 7}{4} = 2,$$ 
      giving the form $x^2 + xy + 2y^2$ and thus $h(-7) = 1$.
    \item
      h(-8) = 1.
      
      We have that $a \leq \sqrt{8/3} < 2$ and so $a = 1$, from which it follows that $b = 0$ or $b = 1$.
      When $b = 1$ we have $c = 9/4 \not \in \Z$.
      When $b = 0$ we have $c = 8/4 = 2$, giving the form $x^2 + 2y^2$.
      Therefore $h(-8) = 1$.
    \item
      h(-43) = 1

      We have that $\abs{b} \leq a \leq \sqrt{43/3} < 4$.
      Observe that $-D = 43 \equiv 1 \pmod{4}$, so it is necessary that $b^2 \equiv 1 \pmod 4$ in order to satisfy
      $$c = \frac{b^2 - D}{4a} \in \mathbb{Z}.$$
      Hence $b \not \in \{0, \pm 2\}$ must hold.
      When $a = 1$, this implies $b = 1$ immediately, giving the form $x^2 + xy + 11y^2$.
      If $a = 2$, then $b \in \{\pm 1\}$.
      However, $44 \equiv 3 \pmod{8}$ and so $a \neq 2$.
      If $a = 3$, $b \in \{\pm 1, 3\}$.
      Since $12 \nmid 44$ and $12 \nmid 52$, it follows that $a \neq 3$.
      Therefore $h(-43) = 1$.
    \item
      h(-67) = 1.
      
      We have $\abs{b} \leq a \leq \sqrt{67/3} < 5$.
      Observe that $-D = 67 \equiv 3 \pmod{4}$, so it is necessary that $b^2 \equiv 1 \pmod{4}$ and thus $b \not \in \{0, \pm 2, 4\}$.
      When $a = 1$ we have the form $x^2 + xy + 17y^2$.
      If $a = 2$, then $b \in \{\pm 1\}$, but $68 \equiv 4 \pmod{8}$, so $a \neq 2$. 
      If $a = 3$, then $b \in \{\pm 1, 3\}$.
      However $12 \nmid 68$ and $12 \nmid 76$, so $a \neq 3$.
      If $a = 4$, then $b \in \{\pm 1, 3\}$.
      However, $16 \nmid 68$ and $16 \nmid 76$.
      Therefore $h(-67) = 1$.
    \item
      h(-163) = 1.
      
      We have $\abs{b} \leq a \leq \sqrt{163/3} < 8$.
      Observe that $-D = 163 \equiv 3 \pmod{4}$, so it is necessary that $b^2 \equiv 1 \pmod{4}$ and thus $b \not \in \{0, \pm 2, \pm 4, 6\}$.
      We first check the possible values of $b$ for each value of $a$, then compute the residues of $b^2 - D \pmod {4a}$; we note that the only viable choices for $a$ and $b$ are those satisfying $0 = b^2 - D \pmod{4a}$ since
      $$c = \frac{b^2 - D}{4a} \in \Z.$$
      If $a = 1$, then $b = 1$.
      If $a = 2$, then $b \in \{\pm 1\}$, but $164 \equiv 4 \pmod{8}$, so $a \neq 2$. 
      If $a = 3$, then $b \in \{\pm 1, 3\}$.
      If $a = 4$, then $b \in \{\pm 1, \pm 3\}$.
      If $a = 5$, then $b \in \{\pm 1, \pm 3, 5\}$
      If $a = 6$ then $b \in \{\pm 1, \pm 3, \pm 5\}$.
      If $a = 7$, then $b \in \{\pm 1, \pm 3, \pm 5, 7\}$.
      The residue classes are collected in the following table; each column gives the residue class of $b^2 - D \pmod{4a}$, where appropriate.
      $$\begin{array}{c||cccccccc}
                        & 1 & 2 & 3 & 4 & 5 & 6 & 7\\
        \hline        
        (\pm 1)^2 + 163 & 0 & 4 & 8 & 4  & 4  & 20 & 24 & \pmod{4a}\\
        (\pm 3)^2 + 163 & - & - & 4 & 12 & 12 & 4  & 4  & \pmod{4a}\\
        (\pm 5)^2 + 163 & - & - & - & -  & 8  & 20 & 20 & \pmod{4a}\\
        7^2 + 163       & - & - & - & -  & -  & -  & 16 & \pmod{4a}
      \end{array}$$
      The only viable choice is $a = 1$, $b = 1$ and so we have the form $x^2 + xy + 41y^2$.
      Therefore $h(-163) = 1$.
    \end{enumerate}
  \end{proof}
\end{thm}

\begin{thm}
  Find some $D$ for which $h(D) > 5$.
  
  \begin{proof}
    h(-87) = 6.

    We have $\abs{b} \leq a \leq \sqrt{163/3} < 6$.
    Observe that $-D = 87 \equiv 3 \pmod{4}$, so it is necessary that $b^2 \equiv 1 \pmod{4}$ and thus $b \not \in \{0, \pm 2, \pm 4\}$.
    We first check the possible values of $b$ for each value of $a$, then compute the residues of $b^2 - D \pmod {4a}$; we note that the only viable choices for $a$ and $b$ are those satisfying $0 = b^2 - D \pmod{4a}$ since
    $$c = \frac{b^2 - D}{4a} \in \Z.$$
    If $a = 1$, then $b = 1$.
    If $a = 2$, then $b \in \{\pm 1\}$, but $164 \equiv 4 \pmod{8}$, so $a \neq 2$. 
    If $a = 3$, then $b \in \{\pm 1, 3\}$.
    If $a = 4$, then $b \in \{\pm 1, \pm 3\}$.
    If $a = 5$, then $b \in \{\pm 1, \pm 3, 5\}$
    The residue classes are collected in the following table; each column gives the residue class of $b^2 - D \pmod{4a}$, where appropriate.
    $$\begin{array}{c||cccccc}
      & 1 & 2 & 3 & 4 & 5\\
      \hline        
      (\pm 1)^2 + 87 & 0 & 0 & 4 & 8 & 8  & \pmod{4a}\\
      (\pm 3)^2 + 87 & - & - & 0 & 0 & 16 & \pmod{4a}\\
      (\pm 5)^2 + 87 & - & - & - & -  & 12  & \pmod{4a}\\
    \end{array}$$
    The possible forms we obtain are $x^2 + xy + 22y^2$, $2x^2 \pm xy + 11y^2$, $3x^2 + 3xy + 16y^2$, and $4x^2 \pm 3xy + 6y^2$.
    Therefore $h(-87) = 6$.
  \end{proof}
\end{thm}
\end{document}
