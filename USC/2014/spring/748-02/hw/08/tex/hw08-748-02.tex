\documentclass[10pt]{amsart}
\usepackage{amsmath,amsthm,amssymb,amsfonts,enumerate,mymath,mathtools,tikz-cd,mathrsfs}
\openup 5pt
\author{Blake Farman\\University of South Carolina}
\title{Math 748-02:\\Homework 08}
\date{April 11, 2014}
\pdfpagewidth 8.5in
\pdfpageheight 11in
\usepackage[margin=1in]{geometry}

\begin{document}
\maketitle

\providecommand{\p}{\mathfrak{p}}
\providecommand{\m}{\mathfrak{m}}
\providecommand{\Deck}[1]{\operatorname{Deck}\left(#1\right)}

\newtheorem{thm}{}
\newtheorem{lem}{Lemma}
\newtheorem{prop}{Proposition}
\theoremstyle{definition}
\newtheorem{defn}{Definition}[thm]

\newcommand{\A}{\mathbb{A}}

\begin{thm}
  Suppose $p = \exp \colon \C \rightarrow \C^\times$ is the universal covering of $\C^\times$ and $\omega$ is the holomorphic $1$-form $dz/z$ on $\C^\times$.
  Find $p^*\omega$.
  
  \begin{proof}
    Let $X = C^\times$, $Y = \C$, $U_1 = \C^\times \setminus \left\{x + iy \;\middle\vert\; x < 0, y = 0\right\}$ and $U_2 = \C^\times \setminus \left\{x + iy \;\middle\vert\; x > 0, y = 0\right\}$, 
    equipped with the natural inclusions into $\C$.
    We observe that there are logarithms $f_i = \log_{U_i} \in \mathcal{O}_X(U_i)$ and about a point $z_0 \in U_i$
    $$f_i(z) = f_i(z_0) + \frac{1}{z_0}(z - z_0) + O((z - z_0)^2) \equiv f_i(z_0) \frac{1}{z_0}(z - z_0) \pmod{m_{z_0}^2}.$$
    Hence on $U_i$,
    $$\omega = \frac{dz}{z} = df_i.$$
    Therefore $p^*\omega = p^*(df_i) = d(p^*f_i) = d(f_i \circ \exp) = dz$.
  \end{proof}
\end{thm}

\begin{thm}
  Prove that the holomorphic $1$-form
  $$\frac{dz}{1 + z^2},$$
  which is defined on $\C\setminus\{\pm i\}$, can be extended to a holomorphic $1$-form on $\mathbb{P}^1\setminus \{\pm i\}$.
  Let
  $$p = \tan \colon \C \rightarrow \mathbb{P}^1\setminus\{\pm i\}$$
  and find $p^*\omega$.
  
  \begin{proof}
    Let $X = \mathbb{P}^1\setminus\{\pm i\}$.
    Using the fact that
    $$\arctan(z) = \frac{i}{2} \log\left(\frac{i + z}{i - z}\right)$$
    then with the appropriate branch cuts we have, about $x \in X$,
    $$\arctan(z) = \arctan(x) + \frac{1}{1 + x^2}(z - x) + O((z - x)^2)$$
    and so it follows that 
    $$d(\arctan) = \frac{dz}{1 + z^2}.$$
    Therefore $p^*\omega = dp^*(\arctan) = dz$.
  \end{proof}
\end{thm}

\begin{thm}
  Suppose $p \colon Y \rightarrow X$ is a holomorphic mapping of Riemann surfaces, $a \in X$, $b \in p^{-1}(a)$, and $k$ is the multiplicity of $p$ at $b$.
  Given any holomorphic $1$-form $\omega$ on $X \setminus{a}$, show that 
  $$\Res_b(p^*\omega) = k\Res_a(\omega).$$
  
  \begin{proof}
    Choose charts $(U,z)$ on $Y$ centered about $b$ and $(U^\prime, w)$ on $X$ centered about $a$ such that if $P = w \circ p \circ z^{-1}$, then $P(z) = z^k$.
    Locally, we have $\omega = fdw$ for some $f \in \mathcal{O}_X(U^\prime)$ and about $w(a) = 0$,
    $$f \circ w^{-1} = \sum_{n = -\infty}^\infty c_n w^n.$$
    %Since $p^*\omega = p^*fd(p^*w)$, we have the expansion about $z(b) = 0$
    Expanding $p^*f \circ z^{-1}$ about $z(b) = 0$ we have
    $$p^*f \circ z^{-1} = f \circ p \circ z^{-1} = f \circ w^{-1} \circ (w \circ p \circ z^{-1}) = f \circ w^{-1} \circ P = \sum_{n = -\infty}^\infty c_n z^{kn}.$$
    We then observe that $p^*w = w \circ p = w \circ p \circ z^{-1} \circ z = P \circ z = z^{k}$, so we have $d(p^*w) = d(z^k)$ and for $y \in U$
    $$z^k = z(y)^k + kz^{k-1}(z - z(y)) + O((z - z(y))^2) \equiv z(y)^k + kz^{k-1}(z - z(y)) \pmod{\mathfrak{m}_y^2}$$
    from which it follows that $d(z^k) = kz^{k-1}dz$.
    Hence $p^*\omega = p^*f d (p^*z) = kz^{k-1}p^*fdz$ and about $z = 0$, 
    $$kz^{k-1}p^*f = kz^{k-1}\sum_{n = -\infty}^\infty c_n z^{nk} = \sum_{n = -\infty}^\infty kc_n z^{nk + k - 1}.$$
    When $n = -1$ we have 
    $$kc_nz^{nk + k - 1} = kc_{-1}z^{-k + k - 1} = kc_{- 1}z^{-1}.$$
    Therefore $\Res_b(p^*\omega) = kc_{-1} = k\Res_a(\omega)$, as desired.
  \end{proof}
\end{thm}
\end{document}
