\documentclass[10pt]{amsart}
\usepackage{amsmath,amsthm,amssymb,amsfonts,enumerate,mymath,mathtools,tikz-cd,mathrsfs}
\openup 5pt
\author{Blake Farman\\University of South Carolina}
\title{Math 748-02:\\Homework 02}
\date{February 3, 2014}
\pdfpagewidth 8.5in
\pdfpageheight 11in
\usepackage[margin=1in]{geometry}

\begin{document}
\maketitle

\providecommand{\p}{\mathfrak{p}}
\providecommand{\m}{\mathfrak{m}}
\providecommand{\Deck}[1]{\operatorname{Deck}\left(#1\right)}

\newtheorem{thm}{}
\newtheorem{lem}{Lemma}
\newtheorem{prop}{Proposition}
\theoremstyle{definition}
\newtheorem{defn}{Definition}[thm]

\newcommand{\A}{\mathbb{A}}

\begin{thm}
  In this problem, we will study some examples of B\'{e}zout's Theorem.
  Consider the following Riemann surfaces in $\mathbb{P}^2$.
  \begin{eqnarray*}
    R_1 &=& \left\{ [x : y : z] \in \mathbb{P}^2 \;\middle\vert\; x^2 + y^2 = 4z^2 \right\}\\
    R_2 &=& \left\{ [x : y : z] \in \mathbb{P}^2 \;\middle\vert\; (x - z)^2 + (y - z)^2 = z^2 \right\}\\
    S &=& \left\{ [x : y : z] \in \mathbb{P}^2 \;\middle\vert\; y^2z = x^3 - xz^2\right\}\\
    T_1 &=& \left\{ [x : y : z] \in \mathbb{P}^2 \;\middle\vert\; y = 0 \right\}\\
    T_2 &=& \left\{ [x : y : z] \in \mathbb{P}^2 \;\middle\vert\; x = 0 \right\}\\
  \end{eqnarray*}
  \begin{enumerate}[(a)]
  \item
    What are the points of $S \cap T_1$?
    For each point of intersection, compute the intersection multiplicity.
  \item
    What are the points of $S \cap T_2$?
    For each point of intersection, compute the intersection multiplicity.
  \item
    What are the points of $R_1 \cap R_2$?
    For each point of intersection, compute the intersection multiplicity.
  \end{enumerate}

  \begin{proof}
    \begin{enumerate}[(a)]
    \item
      The points of $S \cap T_1$ are
      $$S \cap T_1 = \left\{ [x : 0 : z] \;\middle\vert\; 0 = x(x - z)(x + z)\right\}.$$
      Since $z = 0$ implies $x = 0$, we may assume that $z = 1$ and so
      $$S \cap T_1 = \left\{ [0 : 0 : 1], [1 : 0 : 1], [-1 : 0 : 1] \right\}.$$
      
      For the intersection multiplicities we dehomogenizing with $z = 1$ and consider the resultant of the polynomials $f(x,y) = -x^{3} + y^{2} + x$ and $g(x,y) = y$ as polynomials in $\left(\C[x]\right)[y]$,
      \begin{eqnarray*}
        \operatorname{Res}(f,g) &=& \det \left(\begin{array}{ccc}
          1 & 0 & 1 \\
          0 & 1 & 0 \\
          -x^{3} + x & 0 & 0
        \end{array}\right)\\
        &=& x^3 - x\\
        &=& x(x+1)(x-1).
      \end{eqnarray*}
      Therefore the intersection multiplicities are all 1.
    \item
      The points of $S \cap T_2$ are
      $$S \cap T_2 = \left\{ [0 : y : z] \;\middle\vert\; y^2z = 0 \right\} = \left\{[0 : 1 : 0], [0 : 0 : 1]\right\}.$$
      
      For the point $[0 : 1 : 0]$, we dehomogenize with $y = 1$ and consider the resultant of the polynomials $f(x,z) = -x^{3} + x z^{2} + z$ and $g(x,z) = x$ as polynomials in $\left(\C[z]\right)[x]$,
      \begin{eqnarray*}
        \operatorname{Res}(f,g) &=& \det\left(\begin{array}{cccc}
          -1 & 1 & 0 & 0 \\
          0 & 0 & 1 & 0 \\
          z^{2} & 0 & 0 & 1 \\
          z & 0 & 0 & 0
        \end{array}\right)\\
        &=& -z.
      \end{eqnarray*}
      Hence the intersection multiplicity for $[0 : 1 : 0]$ is 1.
      
      For the point $[0 : 0 : 1]$, we dehomogenize with $z = 1$ and consider the resultant of the polynomials $f(x,y) = -x^{3} + y^{2} + x$ and $g(x,y) = x$ as polynomials in $\left(\C[y]\right)[x]$,
      \begin{eqnarray*}
        \operatorname{Res}(f,g) &=& \det \left(\begin{array}{cccc}
          -1 & 1 & 0 & 0 \\
          0 & 0 & 1 & 0 \\
          1 & 0 & 0 & 1 \\
          y^{2} & 0 & 0 & 0
        \end{array}\right)\\
        &=& -y^2.
      \end{eqnarray*}
      Hence the intersection multiplicity for $[0 : 0 : 1]$ is 2.
    \item
      Assume that $[x : y : z] \in R_1 \cap R_2$.
      Then we have
      $$z^2 = (x - z)^2 + (y - z)^2 = x^2 + y^2 + 2z^2 - 2z(x + y) = 4z^2 + 2z^2 - 2z(x + y)$$
      from which it follows that
      $$z^2(5z - 2(x + y)) = 0.$$
      
      Assume that $z = 0$ and note that both $x$ and $y$ are non-zero.
      From $x^2 + y^2 = (x + iy)(x - iy) = 0$ it follows that either $[x : y : z] = [x : -iy : 0] = [1 : -i : 0]$
      or $[x : y : z] = [x : ix : 0] = [1 : i : 0]$.
      
      When $z \neq 0$, then $[x : y : z] = [x/z : y/z : 1]$, and so we may assume that $z = 1$.
      We are then interested in pairs $(x,y) \in \C^2$ such that 
      \begin{equation}\label{1.1}
	x^2 + y^2 = 4
      \end{equation}
      and 
      \begin{equation}\label{1.2}
	(x - 1)^2 + (y-1)^2 = 1.
      \end{equation}
      Upon expanding \eqref{1.2} we obtain by using \eqref{1.1} and some routine algebra 
      $$x + y = 5/2.$$
      Taking $x = 5/2 - y$ in \eqref{1.1} we obtain
      $$8y^2 - 20y + 9 = 0.$$
      Using the quadratic formula, it follows that $y = (5 \pm \sqrt{7})/4$ and thus $x = 10/4 - y = \left(5 \mp \sqrt{7}\right)/4$.
      Hence
      $$R_1 \cap R_2 = \left\{[1 : -i : 0], [1 : i : 0], [5 \mp \sqrt{7} : 5 \pm \sqrt{7} : 4]\right\}.$$

      For the intersection multiplicites of $[1 : -i : 0]$ and $[1 : i : 0]$ we dehomogenize by taking $x = 1$ and consider the resultant of the polynomials $f(y,z) = -4z^2 + (y^2 + 1)$ and $g(y,z) = z^2 - 2(1 + y)z + (y^2 - 1)$ as elements of $\left(\C[y]\right)[z]$, 
      \begin{eqnarray*}
        \operatorname{Res}(f,g) &=& \det\left(\begin{array}{cccc}
          -4 & 0 & 1 & 0 \\
          0 & -4 & -2 \, y - 2 & 1 \\
          y^{2} + 1 & 0 & y^{2} + 1 & -2 \, y - 2 \\
          0 & y^{2} + 1 & 0 & y^{2} + 1
        \end{array}\right)\\
        &=& 9 \, y^{4} - 32 \, y^{3} + 18 \, y^{2} - 32 \, y + 9\\
        &=& {\left(y^{2} + 1\right)} {\left(9 \, y^{2} - 32 \, y + 9\right)}\\
        &=& {\left(y + i\right)} {\left(y - i\right)} {\left(9 \, y^{2} - 32 \, y + 9\right)}.
      \end{eqnarray*}
      Hence the intersection multiplicities of $[1 : -i : 0]$ and $[1 : i : 0]$ are both 1.
      
      For the intersection multiplicities of $[5 \mp \sqrt{7} : 5 \pm \sqrt{7} : 4]$, we dehomogenize by taking $z = 1$ and consider the resultant of the polynomials $f(x,y) = x^{2} + y^{2} - 4$ and $g(x,y) = {\left(y - 1\right)}^{2} + {\left(x - 1\right)}^{2} - 1$ as elements of $\left(\C[y]\right)[x]$,
      \begin{eqnarray*}
        \operatorname{Res}(f,g) &=& \det\left(\begin{array}{cccc}
          1 & 0 & 1 & 0 \\
          0 & 1 & -2 & 1 \\
          y^{2} - 4 & 0 & y^{2} - 2 \, y + 1 & -2 \\
          0 & y^{2} - 4 & 0 & y^{2} - 2 \, y + 1
        \end{array}\right) \\
        &=& 8 \, y^{2} - 20 \, y + 9.
      \end{eqnarray*}
      Observing that $\frac{5 \pm \sqrt{7}}{4}$ are both roots of this polynomial, it follows that the intersection multiplicities for $[5 \mp \sqrt{7} : 5 \pm \sqrt{7} : 4]$ are both 1.
    \end{enumerate}
  \end{proof}
\end{thm}

\begin{thm}
  Given a monic polynomial $f \in \C[x]$ of degree $n$, define the {\it discriminant} by
  $$\disc{f} = \operatorname{Res}_n(f, f^\prime) \in \C.$$
  Here $f^\prime$ is the derivative.
  
  Prove that $f$ has a multiple zero if and only if $\disc{f} = 0$.
  \begin{proof}
    Let $f(x) = a_n x^n + a_{n-1}x^{n-1} + \ldots + a_1 x + a_0$, so then $f^\prime(x) = n a_n x^{n-1} + (n-1)a_{n-1} x^{n-2} + \ldots + 2a_2 x + a_1$.
    Note that it suffices to show that $\disc{f} = 0$ if and only if $f$ and $f^\prime$ have a common zero.
    We observe that if we let 
    $$A = \left(
    \begin{array}{cccccccccc}
      a_n & 0 & 0 & \ldots & 0 & n a_n & 0 & 0 & \ldots & 0\\
      a_{n-1} & a_n & 0 & \ldots & 0 & (n-1) a_{n-1} & na_n & 0 & \ldots & 0\\
      a_{n-2} & a_{n-1} & a_n & \ldots & 0 & (n-2) a_{n-2} & (n-1)a_{n-1} & na_n & \ldots & 0\\
      \vdots & \vdots & \vdots & \ddots & \vdots & \vdots & \vdots & \vdots & \ddots & \vdots\\
      a_2 & a_3 & a_4 & \ldots & a_n & a_1 & 2a_2 & 3a_3 & \ldots & na_n\\
      a_1 & a_2 & a_3 & \ldots & a_{n-1} & 0 & a_1 & 2a_2 & \ldots & (n-1)a_{n-1}\\
      a_0 & a_1 & a_2 & \ldots & a_{n-2} & 0 & 0 & a_1 & \ldots & (n-2)a_{n-2}\\
      0 & a_0 & a_1 & \ldots & a_{n-3} & 0 & 0 & 0 & \ldots & (n-3)a_{n-3}\\
      \vdots & \vdots & \vdots & \ddots & \vdots & \vdots & \vdots & \vdots & \ddots & \vdots\\
      0 & 0 & 0 & \ldots & a_0 & 0 & 0 & 0 & \ldots & a_1\\
    \end{array}
    \right) \in \operatorname{Mat}_{2n - 1}(\C),$$
    then $\operatorname{Res}_n(f,f^\prime) = \det(A)$.
    Let $\alpha \in \C$ be such that $f(\alpha) = 0$ and take $x = (\alpha^{2n-2}, \alpha^{2n-3}, \ldots, \alpha, 1)$.
    Then we observe that 
    $$A^Tx^T = (\alpha^{n-2}f(\alpha), \alpha^{n-1}f(\alpha), \ldots, f(\alpha), \alpha^{n-1}f^\prime(\alpha), \alpha^{n-2}f^\prime(\alpha), \ldots, f^\prime(\alpha))^T = 0$$
    if and only if $f^\prime(\alpha) = 0$.
    But since $x \neq 0$, we note that $A^Tx^T = 0$ if and only if $\det(A^T) = \det(A) = 0$.
    Therefore $\operatorname{Res}_n(f,f^\prime) = 0$ if and only if $f$ has a multiple root.
  \end{proof}
\end{thm}

\begin{thm}
  \begin{enumerate}[(a)]
  \item
    Compute $\operatorname{Res}_{1,1}(f,g)$ where
    \begin{eqnarray*}
      f &=& v_1 x + v_0\\
      g &=& w_1x + w_0.
    \end{eqnarray*}
  \item
    Compute $\operatorname{Res}_{2,2}(f,g)$ where
    \begin{eqnarray*}
      f &=& v_2x^2 + v_1x + v_0\\
      g &=& w_2x^2 + w_1x + w_0.
    \end{eqnarray*}
  \item
    Compute $\disc{f}$ where
    $$f = v_wx^2 + v_1x + v_0.$$
  \end{enumerate}
  
  \begin{proof}
    \begin{enumerate}[(a)]
    \item
      $$\operatorname{Res}_{1,1}(f,g) = \det\left(\begin{array}{cc}
      v_1 & w_1\\
      v_0 & w_0
    \end{array}\right) = v_{1} w_{0} -v_{0} w_{1}.$$
    \item
      \begin{eqnarray*}
        \operatorname{Res}_{2,2}(f,g) &=& \det\left(\begin{array}{cccc}
          v_2 & 0 & w_1 & 0\\
          v_1 & v_2 & w_1 & w_2\\
          v_0 & v_1 & w_0 & w_1\\
          0 & v_0 & 0 & w_0
        \end{array}\right)\\
        &=& v_{0}^{2} w_{1} w_{2} - v_{0} v_{1} w_{1}^{2} - v_{0} v_{2} w_{0} w_{1}
        - v_{0} v_{2} w_{0} w_{2} + v_{0} v_{2} w_{1}^{2} + v_{1}^{2} w_{0}
        w_{1} - v_{1} v_{2} w_{0} w_{1} + v_{2}^{2} w_{0}^{2}.
      \end{eqnarray*}
    \item
      Since $f^\prime = 2v_2x + v_1$
      \begin{eqnarray*}
        \disc{f} &=& \operatorname{Res}(f,f^\prime)\\
        &=& \det \left(\begin{array}{ccc}
          v_2 & 2v_2 & 0\\
          v_1 & v_1 & 2v_2\\
          v_0 & 0 & v_1
        \end{array}\right)\\
        &=& 4 \, v_{0} v_{2}^{2} - v_{1}^{2} v_{2}.
      \end{eqnarray*}
    \end{enumerate}
  \end{proof}
\end{thm}
\end{document}
