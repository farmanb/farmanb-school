\documentclass[10pt]{amsart}
\usepackage{amsmath,amsthm,amssymb,amsfonts,enumerate,mymath,mathtools,tikz-cd,mathrsfs}
\openup 5pt
\author{Blake Farman\\University of South Carolina}
\title{Math 788:\\Comprehensive Exam}
\date{September 17, 2014}
\pdfpagewidth 8.5in
\pdfpageheight 11in
\usepackage[margin=1in]{geometry}

\begin{document}
\maketitle

\providecommand{\Tr}[2]{\operatorname{Tr}_{#1}\left(#2\right)}
\providecommand{\p}{\mathfrak{p}}
\providecommand{\m}{\mathfrak{m}}
\providecommand{\Deck}[1]{\operatorname{Deck}\left(#1\right)}
%\newcommand{\Res}{\operatorname{Res}}
\newtheorem{thm}{}
\newtheorem{lem}{Lemma}
\newtheorem{prop}{Proposition}
\theoremstyle{definition}
\newtheorem{defn}{Definition}[thm]

\newcommand{\A}{\mathbb{A}}

\begin{thm}\label{Ex1}
\end{thm}

\begin{thm}\label{Ex2}
\end{thm}

\begin{thm}\label{Ex3}
	\begin{enumerate}[(a)]
		\item
		Define the $p-$adic integers $\Z_p$.
		Determine, with proof, the maximal ideal $\mathfrak{m}$ and the residue class field $\Z_p/\mathfrak{m}$.
		\item
		Write out the $5$-adic expansions for $1/4$, $7$, and $\sqrt{6}$ in $\Q_5$. (First three digits for $\sqrt{6}$.
		\item
		Suppose you defined the $10$-adic integers in an analogous way to the $p$-adics.
		State your definition precisely, and then prove that you don't obtain an integeral domain.
		\item
		Like $\Z$, $\C[x]$ is also a principal ideal domain.
		Define what should be the analogue of $\Z_p$ for $\C[x]$, and describe the rings you obtain by your construction.
	\end{enumerate}
		
		\begin{proof}
		\begin{enumerate}[(a)]
		\item
		Define $\Z_p = \varprojlim{\Z/p^n\Z}$ and $\mathfrak{m} = p\Z_p$.
		Identify $\Z_p$ with a subring of $\prod_{i=1}^\infty \Z/p^n\Z$ subject to the compatability condition that the $n^\text{th}$ coordinate reduces to the $m^\text{th}$ coordinate modulo $p^m$ when $m \neq n$.
		It's clear that this ideal is maximal as
		$$0 \rightarrow \mathfrak{m} \rightarrow \Z_p \rightarrow \Z/p\Z \rightarrow 0$$
		is an exact sequence.
		To see that $\Z_p$ is local, we observe that it suffices to show that given any $z \in \Z_p$ and any $x \in \mathfrak{m}$, $1 + zx$ is a unit, for this characterizes local rings.
		
		Working in $\Z/p^n\Z$ we let $a_n$ be the $n^\text{th}$ coordinate of $1 + zx$ and let $\pi_n \colon \Z_p \rightarrow \Z/p^n\Z$ be the projections onto each factor.
		We see that since $x \in \mathfrak{m}$ implies $\pi_1(x) = 0$, we have
		$$a_1 = \pi_1(1 + zx) = \pi_1(1) + \pi_1(z)\pi_1(x) = \pi_1(1) = 1$$
		Now we note that $p \nmid a_n$ follows from the compatability conditions, for otherwise $a_n$ would reduce to zero modulo $p$.
		Hence $a_n$ is a unit in $\Z/p^n\Z$, so it has an inverse, say $a_nb_n \equiv 1 \pmod{p^n}$.
		We observe that $b = (b_1, b_2, \ldots)$ is the inverse of $1 + ax$ in $\prod_{n=1}^\infty \Z/p^n\Z$, so it remains only to show that $b \in \Z_p$.
		This follows directly from the fact that $a_nb_n \equiv 1 \pmod{p^n}$.
		Namely, whenever $m \leq n$ we have
		$$a_nb_n \equiv a_mb_n \pmod{p^m}$$
		and so by uniqueness of inverses we have $b_n \equiv b_m \pmod{p^m}$.
		\item
		We observe that $4$ is a unit in $\Z_p$, so we consider the image of $1/4$ in each of $\Z/5^n\Z$.
		For $n = 1$ the inverse of $4$ is $4$, for $n = 2$ we have $4(-6) \equiv 4(19) \equiv 1 \pmod 25$ and $19 = 4 + 3\cdot 5$.
		We show by induction that $1/4 = 4 + 3\sum_{n=1}^\infty 5^n$.
		Assume that the result holds up to $2 \leq n$.
		We write
		\begin{eqnarray*}
			4\left(4 + 3\sum_{k=1}^{n-1} 5^k\right) &=&  16 + 12\sum_{k=1}^{n-1} 5^k\\
			&=& 1 + 3\cdot 5 + (2 + 2\cdot 5)\sum_{k=1}^{n-1} 5^k\\
			&=& 1 + 3\cdot 5 + 2\sum_{k=1}^{n-1} 5^k + 2\sum_{k=2}^{n} 5^k\\
			&=& 1 + 3\cdot 5 + 2 \cdot 5 + 4\sum_{k=2}^{n-1} 5^k + 2 \cdot 5^n\\
			&=& 1 + 5^2 + 4\sum_{k=2}^{n-1} 5^k + 2 \cdot 5^n\\
			&=& 1 + 5^2 + 4\cdot 5^2 4\sum_{k=3}^{n-1} 5^k + 2\cdot 5^n\\
			&=& 1 + 5^3 + 4\sum_{k=3}^{n-1} 5^k + 2 \cdot 5^n\\
			&\vdots&\\
			&=& 1 + 5^{n-1} + 4 \cdot 5^{n-1} + 2 \cdot 5^n\\
			&=& 1 + 5^{n} + 2 \cdot 5^{n}\\
			&=& 1 + 3\cdot 5^n \equiv 1 \pmod{5^{n}}.
		\end{eqnarray*}
		Now we see that
		\begin{eqnarray*}
			4\left(4 + \sum_{k=1}^n 5^k\right) &=& 1 + 3\cdot 5^{n} + 12\cdot5^{n}\\
			&=& 1 + 3\cdot 5^{n} + 2\cdot 5^{n} + 2\cdot5^{n+1}\\
			&=& 1 + 5^{n+1} + 2\cdot 5^{n+1}\\
			&=& 1 + 3 \cdot 5^{n+1} \equiv 1 \pmod{5^{n+1}}.
		\end{eqnarray*}

		For $7$, we simply observe that $7 \equiv 2 \pmod{5}$ and for $n > 1$, $7 \equiv 7 \pmod{5^n}$, so
		$$7 = 2 + 1\cdot 5 + 0\cdot 5^2 + 0 \cdot 5^3 + \ldots = 2 + 5.$$
	
		For $\sqrt{6}$, we observe that $6 \equiv 1^2 \pmod{5}$, $6 \equiv 9^2 \pmod{25}$, $6 \equiv 16 \pmod{125}$, and $6 \equiv 109^2 \pmod{625}$.
		In order to satisfy the congruence conditions in $\Z_5$, we take
		$$\sqrt{6} = (1, -9, 16, -109, \ldots) = (1, 16, 16, 516, \ldots)$$
		so that
		$$\sqrt{6} = 1 + 3\cdot 5 + 0 \cdot 5^2 + 4 \cdot 5^3 + \ldots = 1 + 3\cdot 5 + 4\cdot5^3 + \ldots.$$
		\item
		Let $p$ and $q$ be distinct rational primes and define $\Z_{pq} = \varprojlim{\Z/(pq)^n\Z}$.
		We show in general that $\Z_{pq} \cong \Z_p \times \Z_q$.
		Since $\Z/(pq)^n\Z \cong \Z/p^n\Z \times \Z/p^m\Z$ we may choose, for each $n$, elements $x_n$ and $y_n$ of $\Z/(pq)^n\Z$ such that $x_n$ maps to $(1,0)$ and $y_n$ maps to $(0,1)$ under the isomorphism.
		Since the map 
		$$\Z/p^n\Z \times \Z/q^n\Z \rightarrow \Z/p^m\Z \times \Z/q^m\Z$$
		is, by the universal property for products, just reduction in each component when $m \leq n$
		we see that the elements $x = (x_1, x_2, \ldots)$ and $y = (y_1, y_2, \ldots)$ both satisfy the relevant compatability conditions, and thus are both elements of $\Z_{pq}$
		So we have a map
		\begin{align*}
			\varphi \colon \Z_{pq} &\rightarrow \Z_p \times \Z_q\\
			(x_1, x_2, \ldots) & \rightarrow \left((x_1 \pmod{p}, x_2 \pmod{p^2}, \ldots), (x_1 \pmod{q}, x_2 \pmod{q^2}, \ldots)\right)
		\end{align*}
		which is clearly a morphism of rings as the reduction in each component is a morphism of rings.
		Now we note that $\varphi(x) = (1,0)$ and $\varphi(y) = (0,1)$, so $\varphi$ is surjective.
		For injectivity, suppose $k = (k_1, k_2, \ldots)$ is an element of the kernel.
		Then we see that for each $n$, $k_n \equiv 0 \pmod{p^n}$ and $k_n \equiv 0 \pmod{q^n}$, so it follows that $k_n \equiv 0 \pmod{(pq)^n}$.
		Hence $k = 0$ and $\varphi$ is injective.
		Finally, we note that $x \not \in \ker\varphi$ and $y \not \in \ker\varphi$, but
		$$\varphi(xy) = \varphi(x)\varphi(y) = (1,0)\cdot(0,1) = (1\cdot 0, 0 \cdot 1) = (0,0) = 0$$
		implies $xy \in \ker\varphi = (0)$.
		Therefore $(0)$ is not prime, and $\Z_{pq}$ is not an integeral domain.
		\item
		The analogue of $\Z_p$ for $\C[x]$ is $\varprojlim{\C[x]/(x - a)^n}$, where $a \in \C$.
		These can be viewed as the ring of formal power series about $a$.
		\end{enumerate}
		\end{proof}
\end{thm}

\begin{thm}\label{Ex4}
	Compute the ring of integers and class group of $\Q(\sqrt{39})$.
	\begin{proof}
		Since $39 \equiv 3 \pmod{4}$, the ring of integers in $\Q(\sqrt{39})$ is $\Z[\sqrt{39}]$ and the discriminant is $4\cdot39 = 156$
		For the class group, we have that the Minkowski bound is
		$$B_k = \sqrt{39} < 7,$$
		so we consider the factorizations of $2$, $3$, and $5$.
		They are
		$(2) = (2, \sqrt{39} + 1)^2 = \mathfrak{p}_2^2$,
		$(3) = (3, \sqrt{39})^2 = \mathfrak{p}_3$, and
		$(5) = (5, \sqrt{39} - 1)(5, \sqrt{39} - 1) = \mathfrak{p}_5\mathfrak{p}_5^\prime$.
	
		Since $3 = -36 + 39 = (6 - \sqrt{39})(6 + \sqrt{39})$ we claim that $\mathfrak{p}_3 = (6 - \sqrt{39})$.
		The inclusion $(6 - \sqrt{39}) \subseteq \mathfrak{p}_3$ is clear.
		For the reverse inclusion, we note that since $3 \in (6 - \sqrt{39})$, we have $\sqrt{39} = 3\cdot2 - (6 - \sqrt{39}) \in (6 - \sqrt{39})$, so that $\mathfrak{p}_3 \subseteq (6 - \sqrt{39})$.
	
		For $\mathfrak{p}_2 = (2, \sqrt{39} + 1)$, we suppose to the contrary that $\mathfrak{p}_2$ is principal.
		Then for some $a + b\sqrt{39}$, $a, b \in \Z$, we have $N(a + b\sqrt{39}) = a^2 - 39b^2 = 2$.
		Reducing modulo $3$ we have
		$$a^2 \equiv 2 \pmod{3}$$
		a contradiction.

		Now suppose to the contrary that either $\mathfrak{p}_5$ or $\mathfrak{p}_5^\prime$ are principal.
		This implies that there exist integers $a$ and $b$ such that $a^2 - 39b^2 = 5$.
		Reducing modulo $13$ we see
		$$a^2 \equiv 5 \pmod{13},$$
		but
		$$\left(\frac{5}{13}\right) = \left(\frac{13}{5}\right) = \left(\frac{3}{5}\right) = -1$$
		a contradiction.
		Therefore $\mathfrak{p}_5$ and $\mathfrak{p}_5^\prime$ are not principal.

		Finally, we note that we have
		$$10 = 49 - 39 = (7 - \sqrt{39})(7 + \sqrt{39}) = \Norm{\Q(\sqrt{39})/\Q}{7 + \sqrt{39}}$$
		so that one of $(7 + \sqrt{39}) = \mathfrak{p}_2\mathfrak{p}_5$ or $(7 + \sqrt{39}) = \mathfrak{p}_2\mathfrak{p}_5^\prime$ holds.
		But then $\mathfrak{p}_2 \sim \mathfrak{p}_5$ or $\mathfrak{p}_2 \sim \mathfrak{p}_5^\prime$.
		This implies that either $\mathfrak{p}_5^2$ or $(\mathfrak{p}_5^\prime)^2$ are principal.	
		But since $(5) = \mathfrak{p}_5\mathfrak{p}_5^\prime$, we see that $\mathfrak{p}_2 \sim \mathfrak{p}_5 \sim \mathfrak{p}_5^\prime$ and the class group is thus isomorphic to $\Z/2\Z$.
	\end{proof}
\end{thm}

\begin{thm}\label{Ex5}
\end{thm}

\begin{thm}\label{Ex6}
	Let $L/K$ be an extension of number fields, let $\mathcal{O}_L$ and $\mathcal{O}_K$ be their rings ofintegers, and let $\mathfrak{p}$ be a prime of $\mathcal{O}_K$.
	We are interested in proving that $\mathcal{O}_L/\mathfrak{p}\mathcal{O}_L$ is an $\mathcal{O}_K/\mathfrak{p}$-vector space of dimension $n = [L : K]$.
	(This is an important step in the $efg$ theorem).
	\begin{enumerate}[(a)]
		\item
		Can we write the following, for some $\alpha_i \in \mathcal{O}_L$?
		\begin{equation}\label{6.1}
			\mathcal{O}_L = \mathcal{O}_K\alpha_1 \oplus \ldots \oplus \mathcal{O}_K\alpha_2.
		\end{equation}
		If not in general, by quoting a relevant theorem, give an example of a $K$ for which we can do this.
		\item
		Assume that \eqref{6.1} holds.
		Prove that the images of the $\alpha_i$ under the natural reduction map modulo $\mathfrak{p}$ form a basis for $\mathcal{O}_L/\mathfrak{p}\mathcal{O}_L$ as an $\mathcal{O}_K/\mathfrak{p}$-vector space, thereby obtaining the result.
		You may (and almost certainly will) give a proof that requires a condition on $\mathcal{O}_K$.
		This is allowed, if you state your condition explicitly and if it does not restrict your proof the only $K = \Q$.
	\end{enumerate}
	\begin{proof}
		\begin{lem}\label{L.6.1}
			Let $A$ be a domain, $K$ its field of fractions, and $L/K$ a finite extension.  
			If $B$ is the integral closure of $A$ in $L$, then every element $b \in L$ may be written as $\beta = b/a$, with $b \in B$ and $a \in A$.
		\begin{proof}
			Let $\beta \in L$ be given.
			Since $L/K$ is finite, it is necessarily algebraic, and so for some $m \mid n$ we have
			$$\alpha_0\beta^m + \alpha_1\beta^{m-1} + \ldots + \alpha_n = 0,\, \alpha_i \in K.$$
			Multiplying both sides by $\alpha_0^{m-1}$ we obtain an equation of integral dependence for $\alpha_0\beta$,
			$$\left(\alpha_0\beta\right)^m + \alpha_1(\alpha_0\beta)^{m-1} + \ldots + \alpha_n\alpha_0^{m-1} = 0$$
			which gives $\alpha_0\beta \in B$.
			Therefore there exists some $b \in B$ such that $\alpha_0\beta = b$, and so we have $\beta = b/\alpha_0$, as desired.
		\end{proof}
		\end{lem}

		\begin{lem}\label{L.6.2}
		 	Let $A$ be a domain, $K$ its field of fractions, and $L/K$ a finite, separable extension.
			If $\alpha_1, \ldots, \alpha_n$ is a basis, then the discriminant
			$$d(\alpha_1, \ldots, \alpha_n) \neq 0$$
			and
			\begin{align*}
				L \times L &\rightarrow K\\
				(x,y) &\mapsto \Tr{L/K}{xy}
			\end{align*}
			is a nondegenerate bilinear form on the $K$-vector space $L$.
			\begin{proof}
				Since $L/K$ is separable, there exists by the Primitive Element Theorem, an element $\theta \in L$ such that $L = K(\theta)$.
				If we let $n = [L : K]$, then we have a $K$-basis for $L$, $\left\{1, \theta, \ldots, \theta^{n-1}\right\}$
				Now we observe that if we have elements $x = \sum_{i=1}^na_i\theta^{i-1}$ and $y = \sum_{i=1}^nb_i\theta^{i-1}$, then
				$$\Tr{L/K}{xy} = \sum_{i=1}^n\sum_{j=1}^na_ib_j\theta^{i-1}\theta^{j-1}.$$
				If we define the matrix
				$$M = \left(\Tr{L/K}{\theta^{i-1}\theta^{j-1}}\right)_{i=1,j=1}^{n}$$
				and write $x = (a_1, \ldots, a_n)$ and $y = (b_1, \ldots, b_n)$, then
				$$\Tr{L/K}{xy} = x^TMy.$$
				If we let $\{\theta_i\}_{i=1}^n$ be the roots of the minimal polynomial for $\theta$, then because we've chosen a power basis for $L/K$,
				$$\det(M) = d(1, \theta, \ldots, \theta^{n-1}) = \prod_{i < j}(\theta_i - \theta_j)^2 \neq 0,$$
				as $L/K$ was assumed to be separable.

				Given another basis $\alpha_1, \ldots, \alpha_n$, we observe that the matrix  
				$$\left(\Tr{L/K}{\alpha^{i}\alpha^{j}}\right)_{i=1,j=1}^{n}$$
				can be written as $S^TMS$ for some non-singular matrix $S$, and so
				$$d(\alpha_1, \ldots, \alpha_n) = \det\left(\left(\Tr{L/K}{\theta^{i-1}\theta^{j-1}}\right)_{i=1,j=1}^{n}\right) = \det(S^TMS) \neq 0.$$
			\end{proof}
		\end{lem}
		
		\begin{lem}\label{L.6.3}
			Let $A$ be a domain, $K$ its field of fractions, and $L/K$ a finite, separable extension.
			Let $\alpha_1, \ldots, \alpha_n$ be a basis of $L/K$ which is contained in $B$, of discriminant $d = d(\alpha_1, \ldots, \alpha_n)$.
			Then
			$$dB \subseteq A\alpha_1 + \ldots + A\alpha_n.$$
	
			\begin{proof}
				Let $\alpha \in B$ be given.
				We may write $\alpha = a_1\alpha_1 + \ldots + a_n\alpha_n$, for some $a_i \in K$.
				First we observe that for $i = 1, \ldots, n$ we have
				$$\Tr{L/K}{\alpha_i\alpha} = \Tr{L/K}{\sum_{j=1}^n a_j\alpha_i\alpha_j} = \sum_{j=1}^na_j\Tr{L/K}{\alpha_i\alpha_j}.$$
				If we let $M = \left(\Tr{L/K}{\alpha_i\alpha_j}\right)$, 
				$x = (a_1, \ldots, a_n)^T$, and 
				$b = \left(\Tr{L/K}{\alpha_1\alpha}, \ldots, \Tr{L/K}{\alpha_n\alpha}\right)^T$ 
				then we have the system $Mx = b$.
				Multiplying on the left by the classical adjoint of $M$, $M^*$, we have
				$$M^*b = \det(M)I_nx = \left(\begin{array}{cccc}
							     	da_1 & & & \text{\huge0}\\
								& da_2 & \\
								%& & da_3\\
   								&  & \ddots\\
    								\text{\huge0} & & & da_n
							     \end{array}\right)$$
				Since the entries of $M^*b$ are elements of $A$, it follows that the $da_i$ are also elements of $A$.
				Therefore $$d\alpha = (da_1)\alpha_1 + \ldots + (da_n)\alpha_n \in A\alpha_1 + \ldots + A\alpha_n,$$
				 as desired.
			\end{proof}
		\end{lem}

		\begin{lem}\label{L.6.4}
			Let $A$ be a P.I.D., $K$ its field of fractions, and $L/K$ a finite, separable extension.
			If $B$ is the integral closure of $A$ in $L$, then $B$ is a free $A$-module of rank $n = [L : K]$.
			\begin{proof}
				That $B$ is an $A$-module is clear from the inclusion of $A$ in $B$.
				Let $\alpha_1, \ldots, \alpha_n$ be a basis for $L/K$.
				By Lemma~\ref{L.6.1} we may assume, after possibly multiplying each $\alpha_i$ by a suitable element of $A$, that $\alpha_i \in B$.
				If we let $d = d(\alpha_1, \ldots, \alpha_n)$, then by Lemma~\ref{L.6.2}, we have
				$$dB \subseteq A\alpha_1 + \ldots + A\alpha_n = M \subseteq B,$$
				and so it follows that
				$$B \subseteq A\frac{\alpha_1}{d} + \ldots + A\frac{\alpha_n}{d} = N.$$
				Since $N$ is a free $A$-module and $A$ is a PID, it follows that $B$ is a free $A$-module.
				For the rank of $B$ we observe that
				$$n = \operatorname{rank}_A(M) \leq \operatorname{rank}_A(B) \leq \operatorname{rank}_A(N) = n.$$
				%Hence $B$ is a finitely generated $A$-submodule of the free $A$-module, $N$, whence $B$ is a free $A$-module of rank at most $n$.
				%But also $B$ contains a free rank $n$ $A$-module, $M$, so the rank of $B$ is at least $n$.
				Therefore $B$ is a free rank $n$ $A$-module.
			\end{proof}
		\end{lem}

		\begin{lem}\label{L.6.5}
			If $K/\Q$ is a finite extension, then $K$ is the field of fractions of $\mathcal{O}_K$, the integral closure of $\Z$ in $K$.
			\begin{proof}
				Let $L$ be the field of fractions of $\mathcal{O}_K$.
				Since we have that $\mathcal{O}_K\setminus\{0\} \subseteq K^\times$, we have by the universal property for localisations the commutative diagram
				\begin{center}
				\begin{tikzcd}
				\mathcal{O}_K \arrow{r}\arrow{rd}& L\arrow[dotted]{d}{\exists !f}\\
				& K
				\end{tikzcd}
				\end{center}
				which implies that $f$ is not the zero morphism, hence is injective as $L$ and $K$ are both fields.
				By possibly replacing $L$ by its image under $f$, we may assume that $L$ is a subfield of $K$.
				
				Suppose to the contrary that there exists an element $\alpha \in K \setminus L$.
				By Lemma~\ref{L.6.1} there exists some non-zero integer $d$ such that $d\alpha \in \mathcal{O}_K \subseteq L$.
				But then since $Z \subseteq \mathcal{O}_K$ we have that $\frac{1}{d} \in L$ and so
				$$\frac{1}{d}(d\alpha) = \alpha \in L,$$
				a contradiction.
				Therefore $K = L$ is the field of fractions of $\mathcal{O}_K$.
			\end{proof}
		\end{lem}

		\begin{lem}\label{L.6.6}
			Let $L/K$ be an extension of number fields.
			If $\mathcal{O}_K$ is the integral closure of $\Z$ in $K$, and $\mathcal{O}_L$ is the integral closure of $\Z$ in $L$, then $\mathcal{O}_L$ is the integral closure of $\mathcal{O}_K$ in $L$.

			\begin{proof}
				Let $A$ be the integral closure of $\mathcal{O}_K$ in $L$.
				Since $A$ is integral over $\mathcal{O}_K$, by transitivity of integrality we have that $A$ is integral over $\Z$ and so $A \subseteq \mathcal{O}_L$.
				For the reverse containment, let $\alpha \in \mathcal{O}_L$ be given.
				Since $\mathcal{O}_L$ is integral over $\Z$, we have an equation of integral dependence
				$$\alpha^n + a_1\alpha^{n-1} + \ldots + a_n = 0,\, a_i \in \Z.$$
				Since $Z \subseteq \mathcal{O}_K$, the $a_i$ are elements of $\mathcal{O}_K$ and hence $\alpha$ is integral over $\mathcal{O}_K$.
				Therefore $\alpha \in A$, and $A = \mathcal{O}_L$.
			\end{proof}
		\end{lem}
	
		\begin{enumerate}[(a)]
			\item
			Assume that $K$ has class number 1 so that $\mathcal{O}_K$ is a PID.
			By Lemma~\ref{L.6.5} we have that the domain $A = \mathcal{O}_K$ has field of fractions $K$.
			Since $\Q$ is a field of characteristic 0, we see that $L/K$ is a separable extension, and by Lemma~\ref{L.6.6} we have that $B = \mathcal{O}_L$ is the integral closure of $\mathcal{O}_K$ in $L$.
			Apply Lemma~\ref{L.6.4} to see that $\mathcal{O}_L$ is a free $\mathcal{O}_K$-module of rank $[L : K]$.
		\end{enumerate}
	\end{proof}
\end{thm}

\end{document}
