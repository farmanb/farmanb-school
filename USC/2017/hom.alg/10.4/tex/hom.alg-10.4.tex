\documentclass[reqno, 12pt]{amsart}
\usepackage{style}

\title[Seminar]{Homological Algebra Seminar}
\author{Blake Farman}
\date{October 4, 2017}

\begin{document}
\maketitle

\section{\cite[Section 1.6]{Weibel95}: More on Abelian Categories}
\begin{lemma}
  Let $\C$ be a full subcategory of an abelian category, $\A$.
  \begin{enumerate}
  \item
    $\C$ is additive if and only if $\C$ contains the zero object and $\C$ is closed under biproducts.
  \item
    $\C$ is abelian and the inclusion functor $\imath \colon \C \to \A$ is exact if and only if $\C$ is additive and $\C$ is closed under taking kernels and cokernels.
  \end{enumerate}
\end{lemma}

\begin{remark}
  The usage of 'closed' in the second statement is ambiguous.
  There are two possible readings of closed.
  The first is the literal reading, which says that whenever $f \in \C(X,Y)$, taking the kernel, $\op{ker}(f) \colon K \to X$, in $\A$ implies that $K$ is actually an object of $\C$.
  This seems overly restrictive (if not false), for reasons that will become clear in the proof.

  Instead, we take the second possible reading, which seems more appropriate in the categorical setting.
  For any morphism $f \in \C(X,Y)$, denote by $\op{ker}(f)_{\A} \colon K \to X$ and by $\op{ker}(f)_{\C} \colon K^\prime \to X$ the kernels taken in $\A$ and $\C$, respectively.
  To say that $\C$ is closed under kernels is to say that there exists a unique isomorphism $h \in \A(K^\prime,K)$ making the diagram
  $$\begin{tikzcd}[column sep=huge]
      & K^\prime \arrow{d}{\op{ker}(f)_\A}\arrow[dashed,swap]{ld}{\exists !h}\\
      K \arrow{r}{\op{ker}(f)_\C} & X \arrow{r}{f} & Y
  \end{tikzcd}$$
  commute.
  Closure under cokernels is formally dual.

  It should be noted that this says that we can always \emph{choose} an object of $K$ of $\C$ and a morphism $K \to X$ that satisfies the universal property for kernels, not that all choices of a kernel have domain an object of $\C$.
  This is a subtle but somewhat important distinction.
\end{remark}
\begin{proof}
  \noindent
  \begin{enumerate}
  \item
    First assume that $\C$ is additive.
    By definition, $\C$ admits a zero object, say $Z$ and is closed under biproducts.
    Using that 0 is final and initial in $\A$, there exist unique morphisms $Z \to 0$ and $0 \to Z$ of $\A$.
    Composing
    $$0 \to Z \to 0$$
    is necessarily the identity since $\A(0,0)$ is a singleton.
    Similarly, the composition
    $$Z \to 0 \to Z$$
    must also be the identity because, by assumption,
    $$\A(Z,Z) = \C(Z,Z)$$
    is a singleton.
    Therefore $Z \cong 0$.

    Conversely, we note that $\C$ is an $\Ab$-category because it is a full subcategory of an abelian category; that is, for all objects $X$ and $Y$ of $\C$,
    $$\C(X,Y) = \A(X,Y)$$
    is an abelian group.
    The fact that $\C$ is additive now follows from the definition.

  \item
    If $\C$ is abelian, then, by definition, $\C$ is additive, has all kernels and cokernels, and is binormal (i.e. every monic is the kernel of its cokernel, and every epi is the cokernel of its kernel).
    The fact that the inclusion is exact says that taking the kernel in $\C$ and taking the kernel in $\A$ is the same (up to unique isomorphism) in the following sense.
    If we denote by $\op{ker}(f)_{\C} \colon K \to X$ the kernel of $f \in \C(X,Y) = \A(X,Y)$ in $\C$ and by $\op{ker}(f)_{\A} \colon K^\prime \to X$ the kernel of $f$ in $\A$, then we have a commutative diagram
    $$\begin{tikzcd}[column sep=huge]
      & K^\prime \arrow{d}{\op{ker}(f)_\A}\arrow[dashed,swap]{ld}{\exists !h}\\
      K \arrow{r}{\op{ker}(f)_\C} & X \arrow{r}{f} & Y
    \end{tikzcd}$$
    with $h \in \A(K^\prime, K)$ an isomorphism.
    This is our definition of closed under kernels in the remarks above.
    Passing to the opposite category, we obtain by duality that $\C$ is closed under cokernels.
    
    \begin{remark}
      We note that without further assumptions on $\C$, such as replete (meaning any object of $\A$ isomorphic to an object of $\C$ is itself an object of $\C$), it's not necessarily true that $K^\prime$ is an object of $\C$.
    \end{remark}

    
    Conversely, assume that $\C$ is additive and closed under taking kernels and cokernels.
    By assumption, the inclusion functor is exact: for any morphism $f \in \C(X,Y)$, any kernel $\op{ker}(f) \colon K \to X$ is again a kernel in for $f$ in $\A$.
    To see that $\C$ is abelian, we need only show that $\C$ is binormal.
    We show that monics are normal and note that the formally dual argument shows epis are also normal.
    
    Assume that $f \in \C(X,Y)$ is monic.
    We have a commutative diagram
    $$\begin{tikzcd}
      & K \arrow{d}{\op{im}(f)}\\
      X \arrow{rd}{0}\arrow{r}{f}\arrow[dashed]{ur}{\exists !h} & Y  \arrow{d}{\op{coker}(f)} \\& C
    \end{tikzcd}$$
    in $\C$ and include it into $\A$.
    By exactness of the inclusion functor and the fact that $\A$ is abelian, it follows that $h$ admits an inverse
    $$h^{-1} \in \A(K, X) = \C(K,X).$$
    Therefore $f$ is the kernel of $\op{coker}(f)$, as desired.
  \end{enumerate}
\end{proof}

\subsection{Functor Categories}
Let $\C$ and $\D$ be categories and let $F, G \colon \C \to \D$ be functors.
A natural transformation, $\eta \colon F \to G$, is a collection of morphisms, $\eta(X) \in \D(F(X), G(X))$, indexed by the objects of $\C$ subject to the condition that for all morphisms $f \in \C(X,Y)$ the diagram
$$\begin{tikzcd}
  X \arrow{d}{f} & F(X)\arrow{d}{F(f)}\arrow{r}{\eta(X)} & F(X)\arrow{d}{G(f)}\\
  Y & F(Y) \arrow{r}{\eta(Y)} & G(Y)
\end{tikzcd}$$
commutes.

The functor category, $\Fun(\C,\D)$, is the category with objects functors $F \colon \C \to \D$ and morphisms natural transformations, $\eta \colon F \to G$.

\begin{example}
  Let $A$ be a ring.
  Define the additive category $\A$ to be the category with one object, $\ast$, and morphisms $\A(\ast,\ast) = A$.
  The category $\A$ is called a ringoid (with one object).
  The full additive subcategory of $\Fun(\A, \Ab)$ is usually denoted by $\Mod{A}$, the category of left $A$-modules, and the full additive subcategory of $\Fun(\A^\opp, \Ab)$ is usually denoted by $\Mod{A^\opp}$, the category of right $A$-modules.
  Here, the ring $A^\opp$ is just the ring with its $A$-action reversed:
  $$a_1 \cdot a_2 := a_2a_1$$
  which makes no difference when $A$ is commutative.

  The equivalence of these categories is the following observation.
  The functor $F \colon \A \to \Ab$ identifies an abelian group, $M = F(\ast)$, and an $A$-action given by a morphism of abelian groups
  $$F(\ast,\ast) \colon \A(\ast,\ast) = A \to \Ab(M,M).$$
  Explicitly, for an element $a \in A = \A(\ast,\ast)$ and an element $m \in M$, the endomorphism $F(a)$ of the abelian group, $M$, defines the the action of $A$ on $M$ by
  $$\begin{tikzcd}[row sep=tiny]
    F(a) \colon M \arrow{r}& M\\
    m \arrow[mapsto]{r} & a \cdot m
  \end{tikzcd}$$
  This action is associative because $F$ is a functor
  $$(ab) \cdot m = F(ab)(m) = F(a) \circ F(b)(m) = F(a)(b \cdot m) = a \cdot( b \cdot m)$$
  and respects the addition on $M$ because $F$ is additive
  $$a \cdot (m_1 + m_2) = F(a)(m_1 + m_2) = F(a)(m_1) + F(a)(m_2) = a \cdot m_1 + a \cdot m_2.$$

  For two functors $F,G \colon \A \to \Ab$, let $M = F(\ast)$ and $N = G(\ast)$.
  A natural transformation $\eta \colon F \to G$ is the data of a morphism $\varphi = \eta(\ast) \colon M \to N$ such that for all $a \in A$ the diagram  
  $$\begin{tikzcd}
    \ast \arrow{d}{a} & M \arrow{r}{\varphi}\arrow{d}{a} & N\arrow{d}{a}\\
    \ast & M \arrow{r}{\varphi} & M
  \end{tikzcd}$$
  commutes.
  That is,
  $$a \cdot \varphi(m) = G(a) \circ \eta(\ast)(m)= \eta(\ast) \circ F(a)(m) = \varphi(a \cdot m).$$
\end{example}

\begin{example}
  Let $G = (G,+)$ be an abelian group.
  A $G$-graded ring, $A$, is a ring with a decomposition as abelian groups
  $$A = \bigoplus_{g \in G} A_g$$
  such that $A_g A_h \subseteq A_{g + h}$.
  The objects of the abelian group $A_g$ are called the \textbf{homogenous degree $g$ elements of $A$}.
  The canonical example is the $\Z$-graded ring $A[x_1, \ldots, x_n]$ where one defines for $n < 0$ the abelian groups $A_{-n} = 0$, $A_0 = A$, and $A_n$ the $A$-span of degree $n$ monomials (here, e.g., $x_1x_2\cdots x_n$ has degree $n$).

  A $G$-graded module over a $G$-graded ring, $A$, is just an abelian group $M$ with a decomposition
  $$M = \bigoplus_{g \in G} M_g$$
  such that $A_g M_h \subseteq M_{g + h}$.
  A morphism $\varphi \colon M \to N$ of $G$-graded modules is a morphism of $A$-modules that respects the grading
  $$\varphi(M_g) \subseteq N_g.$$
  
  Analogous to the previous example, we construct a ringoid, $\A$, with $G$ objects.
  The objects of $\A$ are the elements of $G$ and the morphisms are
  $$\A(g,h) = A_{h - g}.$$
  Composition of morphisms is simply the multiplication
  $$\begin{tikzcd}[row sep=tiny]
    \A(g,h) \times \A(h,k) \arrow{r} & \A(g,k)\\
    (a,b) \arrow[mapsto]{r} & ab
  \end{tikzcd}$$
  since
  $$\A(g,h)\A(h,k) = A_{h - g}A_{k - h} \subseteq A_{k - g} = \A(g,k).$$

  An additive functor $\M \colon \A \to \Ab$ is just the data of $G$-many abelian groups, $M_g := \M(g)$, equipped a structure morphism
  $$\M(g,h) \colon \A(g,h) = A_{h - g} \to \Ab(M_g, M_h).$$
  For any two elements $g,h \in G$, this defines an action of $A_g$ on $M_h$ in the following way.
  Consider the morphism of abelian groups
  $$\M(h,g + h) \colon \A(h,g+h) = A_{g} \to \Ab(M_h, M_{g + h}).$$
  For any element $a \in A_g = \A(h,g+h)$ we define the action of $a$ on $M_h$ by the morphism
  $$\M(a) \colon M_h \to M_{g + h}.$$
  This ensures that the abelian group
  $$M := \bigoplus_{g \in G} M_g$$
  is a $G$-graded module over the $G$-graded ring, $A$.

  A morphism, $\eta$, of functors $\M,\N \colon \A \to \Ab$ is just the data of $G$-many morphisms of abelian groups
  $$\eta(g) \colon M_g \to N_g$$
  such that all diagrams
  $$\begin{tikzcd}
    g\arrow{d}{a} & M_g\arrow{d}\arrow{r} & N_g\arrow{d}\\
    h & M_h\arrow{r} & N_h
  \end{tikzcd}$$
  The morphism $\varphi \colon M \to N$ induced from the $\eta(g)$ by the universal property for coproducts is easily seen to be an $A$-linear morphism of abelian groups that respects the grading via the defining commutative diagrams
  $$\begin{tikzcd}
    M_g \arrow{r}\arrow{d}{\eta(g)} & M\arrow{d}{\varphi}\\
    N_g \arrow{r} & N 
  \end{tikzcd}$$
\end{example}

\begin{exercise}[Chain Complexes Are Modules]
  An $\Ab$-enriched category, $\C$, is essentially a category with morphism sets $\C(X,Y)$ abelian groups.
  An $\Ab$-enriched functor, $F \colon \C \to \D$, between two $\Ab$-enriched categories is essentially just a functor for which the morphisms
  $$F(X,Y) \colon \C(X,Y) \to \D(X,Y)$$
  are morphisms of abelian groups.

  Fix your favorite commutative ring, $k$.
  For the two sentences above, replace $\Ab$ now with the category of chain complexs, $\Ch{k}$ and essentially you have the definition of $\Ch{k}$-enriched and $\Ch{k}$-enriched functor.
  The natural transformations require some care, but essentially it boils down to just saying that a degree $n$ natural transformation is a collection of morphisms $\eta(C) \colon F(C) \to G(C)[n]$ for all objects $C$ of the domain.
  The natural transformations will naturally collate themselves into a chain complex, with the collection of all degree $n$ natural transformations being the $k$-module in degree $n$.
  Since $\Ch{k}$-enriched is terrible to write, one usually calls these Differential Graded Categories (or dg-categories) and Differential Graded Functors (or dg-functors).

  Fix two cochain complexes, $C$ and $D$, over $k$.
  One computes the internal hom (read: the cochain complex of homomorphisms between cochain complexes) by forming the double complex
  $$\begin{tikzcd}
  & & \vdots & \vdots & \vdots\\
  h_{D^{y+1}}\left(C\right) & \cdots \arrow{r} & h_{D^{y+1}}\left(C^{x+1}\right)\arrow{u}\arrow{r} & h_{D^{y+1}}\left(C^x\right)\arrow{u}\arrow{r} & h_{D^{y+1}}\left(C^{x-1}\right) \arrow{u}\arrow{r} & \cdots\\
  h_{D^{y}}\left(C\right) & \cdots \arrow{r} & h_{D^{y}}\left(C^{x+1}\right)\arrow{u}\arrow{r} & h_{D^{y}}\left(C^x\right)\arrow{u}\arrow{r} & h_{D^{y}}\left(C^{x-1}\right) \arrow{u}\arrow{r} & \cdots\\
  h_{D^{y-1}}\left(C\right) & \cdots \arrow{r} & h_{D^{y-1}}\left(C^{x+1}\right)\arrow{u}\arrow{r} & h_{D^{y-1}}\left(C^x\right)\arrow{u}\arrow{r} & h_{D^{y-1}}\left(C^{x-1}\right) \arrow{u}\arrow{r} & \cdots\\
  & & \vdots\arrow{u} & \vdots\arrow{u} & \vdots\arrow{u}\\
  & & h^{C^{x+1}}\left(D\right) & h^{C^x}\left(D\right) & h^{C^{x-1}}\left(D\right)
  \end{tikzcd}$$
  where $h_{D^{y}}(C^x) = \Mod{(k)}(C^x, D^y)$ and $h^{C^x}(D^y) = \Mod{(k)}(D^y,C^x)$ are the morphisms as $k$-modules.
  Generally speaking, this is not a bounded complex, so totalising with respect to the product and coproduct are \emph{not} the same.
  One totalizes this complex with respect to the product to get the hom cohain complex
  $$\CH{k}(C,D)^n = \prod_{x + y = n} \Mod{(k)}(C^x, D^{x+n})$$
  with the differential applied to an element $f \in \CH{k}(C,D)^n$ given by
  $$d_D \circ f + (-1)^{n+1} f \circ d_C.$$
  \begin{remark}
    \noindent
    \begin{enumerate}
    \item
      The specifics of this construction are not entirely important at the moment.
      The idea is just to realize that one can construct such an object.
    \item
      One can perform a similar operation with tensor products, totalizing with the coproduct instead of the product, which is the left adjoint of this Hom total complex (in complete analogy with the usual Tensor-Hom adjunction of modules).
      This makes $\CH{k}$ an example of a symmetric monoidal closed category.
      The tensor product equipped with natural isomorphisms $C \otimes_k D \cong D \otimes_k C$ is the symmetric monoidal structure, and the internal hom (which is the right adjoint of the tensor product) is the closed structure.
      This appears later in \cite{Weibel95}.
    \end{enumerate}
  \end{remark}
  We can take the category $\CH{k}$ to be the category with objects chain complexes and morphisms given by this Hom-total complex, $\CH{k}(C,D)$.
  This is the canonical dg-category.
  Note that the degree zero cocycles of $\CH{k}(C,D)$ are \emph{exactly} the usual morphisms of chain complexes, $\Ch{k}(C,D)$.
  One can construct from any dg-category, $\A$, ancillary categories $Z^0(\A)$ and $H^0(\A)$ which have the same objects as $\A$, but morphisms
  $$Z^0(\A)(X,Y) = Z^0(\A(X,Y))\ \text{and}\ H^0(\A)(X,Y) = H^0(\A(X,Y))$$
  (recall that $\A(X,Y)$ is a \emph{cochain complex}).
  In our current language, $\Ch{k} = Z^0(\CH{k})$, and the category $\mathbf{K}(k)$ from \cite[Exercise 1.4.5]{Weibel95} is just $H^0(\CH{k})$.
  

  Now fix your favorite unital (ungraded) $k$-algebra, $A$; this means a ring, $A$, equipped with a morphism $k \to A$.
  If you choose $k = \Z$, and like to think about commutative things, this just means $A$ is a commutative ring.
  Take your associated ringoid, $\A$, and recognize this as a dg-category by viewing the morphisms $\A(\ast,\ast)$ as the chain complex with $A$ in degree 0 and differentials the zero morphisms.
  Now recognize the full dg-subcategory of $\Fun(\A^\opp,\CH{k})$ (this means all contravariant functors $F \colon \A \to \CH{k}$ such that the structure morphism $A \to \CH{k}(F(\ast), F(\ast))$ is a morphism of complexes) as the dg-category $\CH{A^\opp}$ with objects chain complexes of right $A$-modules.
  Moreover, one recovers the usual category of chain complexes as $\Ch{A^\opp} = Z^0(\CH{A^\opp})$, and the homotopy category $\mathbf{K}(A) = H^0(\CH{A^\opp})$.
  One regards the category $\CH{A}$ as the so-called dg $\A$-modules, $\dgMod{\A}$.

  The seemingly strange choice of defining $\dgMod{\A}$ to be contravariant functors arises from the convenience of having an enriched Yoneda embedding
  $$\begin{tikzcd}[row sep=tiny]
    \A \arrow{r}& \dgMod{\A}\\
    X \arrow[mapsto]{r} & \A(-, X)
  \end{tikzcd}$$
  In the world of commutative rings, this choice is artificial since the rings $A$ and $A^\opp$ are morita equivalent, meaning that the categories of left and right modules are equivalent.

  The truly adventurous reader will attempt to formulate such an equivalence for a graded ring, $A$.
  The generalization is more or less straightforward, but without being a little bit clever, is quite tedious.
  There is an explcit construction in \cite{KNCP} (shameless, I know).
\end{exercise}

As the example of modules over a ring might suggest, the main advantage of functor categories is that they inherit the properties of the codomain in the following sense.

\begin{proposition}\label{prop: functor cats to ab are ab}
  Let $\C$ be any category and let $\A$ be an abelian category.
  The full additive subcategory of the functor category $\Fun(\C,\A)$ is abelian.
\end{proposition}

\begin{proof}
  The main observation here is that all the nice properties of the abelian category are inherited from $\A$ levelwise.

  The $\Ab$-enrichment comes from defining the levelwise addition of natural transformations.
  Given two natural transformations $\eta,\nu \colon F \to G$, they are simply the data of morphisms
  $$\eta(X), \nu(X) \in \A(FX,GX)$$
  indexed by the elements of $\C$ subject to commutativity of naturality squares.
  We define their sum to be
  $$\eta + \nu(X) = \eta(X) + \nu(X) \in \A(FX,GX)$$
  and need only check that the diagrams
  $$\begin{tikzcd}[column sep=large]
    X \arrow{d}{f} & FX \arrow{d}{F(f)}\arrow{r}{\eta(X) + \nu(X)} & GX\arrow{d}{G(f)}\\
    Y & FY \arrow{r}{\eta(Y) + \nu(Y)} & GY
  \end{tikzcd}$$
  But this follows from bilinearity of composition in $\A$
  \begin{eqnarray*}
    G(f) \circ (\eta(X) + \nu(X)) &=& G(f) \circ \eta(X) + G(F) \circ \eta(X)\\
    &=& \eta(Y) \circ F(f) + \nu(Y) \circ F(f)\\
    &=& \left(\eta(Y) + \nu(Y)\right) \circ F(f).
  \end{eqnarray*}

  Next we define kernels and cokernels levelwise.
  Given a natural transformation $\eta \colon F \to G$, for each object $X$ we have a short exact sequence of $\A$
  $$\begin{tikzcd}
    0 \arrow{r} & K(X) \arrow{r}{\op{ker} \eta(X)} & F(X) \arrow{r}{\eta(X)} & G(X) \arrow{r}{\op{coker} \eta(X)} & C(X) \arrow{r} & 0.
  \end{tikzcd}$$
  We define the two functors
  $$\begin{tikzcd}[row sep=tiny]
    K \colon \C \arrow{r}& \A & & C \colon \C \arrow{r} & \A\\
    X \arrow[mapsto]{r}& K(X) & & X \arrow[mapsto]{r} & C(X)
  \end{tikzcd}$$
  then check that the collection of morphisms $(\op{ker} \eta)(X) := \op{ker} \eta(X)$ and $(\op{coker} \eta)(X) := \op{coker} \eta(X)$ form natural transformations.

  We exhibit the kernel and note that the cokernel is formally dual.
  For any morphism $f \in \C(X,Y)$, we have the commutative diagram
  $$\begin{tikzcd}
    K(X) \arrow{r}{\op{ker} \eta(X)}\arrow[dashed]{d}{\exists !K(f)} & F(X) \arrow{r}{\eta(X)}\arrow{d}{F(f)} & G(X)\arrow{d}{G(f)}\\
    K(Y) \arrow{r}{\op{ker} \eta(Y)} & F(Y) \arrow{r}{\eta(Y)} & G(Y)
  \end{tikzcd}$$
  because
  $$\eta(Y) \circ F(f) \circ \op{ker} \eta(X) = G(f) \circ \eta(X) \circ \op{ker} \eta(X) = 0.$$
  Hence $\op{ker} \eta \colon K \to F$ is a natural transformation.

  Finally, we note that the normality of kernels and cokernels is equivalent to the factorization of any natural transformation $\eta \colon F \to G$.
  Since the factorization of a morphism of $\A$ is a decomposition into a coimage morphism followed by an image morphism
  $$\begin{tikzcd}
    A \arrow{rr}{f}\arrow[swap]{rd}{\op{coim} f = \op{coker}(\op{ker}f)} && B\\
    & f(A) \arrow[swap]{ur}{\op{im} f = \op{ker}(\op{coker} f)}
  \end{tikzcd}$$
  the factorization in $\Fun(\C,\A)$ can be defined level-wise
  $$\begin{tikzcd}
    FX \arrow{rr}{\eta(X)} \arrow[swap]{rd}{\op{coim} \eta(X)} && GX\\
    & \eta(X)(FX) \arrow[swap]{ur}{\op{im} \eta(X)}
  \end{tikzcd}$$
  In particular, $\op{ker} \eta$ is the kernel of $\op{coim} \eta$ and $\op{coker} \eta$ is the cokernel of $\op{im} \eta$.
\end{proof}

\begin{definition}
  Let $F \colon \A \to \B$ be a an additive functor between abelian categories.
  We say $F$ is \textbf{left exact} (resp. \textbf{right exact}) if for every short exact sequence
  $$0 \to A \to B \to C \to 0$$
  of $\A$ the sequence
  $$0 \to FA \to FB \to FC\ (\text{resp.}\ FA \to FB \to FC \to 0)$$
  is exact in $\B$.
  If $F$ is both left exact and right exact, then we say that $F$ is \textbf{exact}.
  That is to say, $F$ preserves exact sequences.

  Formally, we say that a contravariant functor $F \colon \A \to \B$ is (left/right) exact if the covariant functor $F \colon \A^\opp \to \B$ is (left/right) exact.
\end{definition}

\begin{lemma}
  Let $\A$ be an abelian category and consider two morphisms
  $$A \overset{f} \to B \overset{g} \to C.$$
  If $g$ is monic, then $\op{ker}(g \circ f) = \op{ker} f$.
  Dually, if $f$ is epic, then $\op{coker}(g \circ f) = \op{coker} g$.
\end{lemma}
\begin{proof}
  It suffices to show that $\op{ker} f \colon K \to A$ satisfies the universal property.
  Towards that end, suppose we have a morphism $\zeta \colon Z \to B$ such $(g \circ f) \circ \zeta = 0$.
  Since $g$ is monic, this implies that $f \circ \zeta = 0$ and so we have a commutative diagram
  $$\begin{tikzcd}
    &Z\arrow{d}{\zeta}\arrow[dashed,swap]{ld}{\exists !h}\\
    K \arrow{r}{\op{ker} f} &A \arrow{r}{g \circ f} & C
  \end{tikzcd}$$
  Therefore $\op{ker} f = \op{ker}(g \circ f)$.
\end{proof}

\begin{exercise}
  Show that the above definitions are equivalent to the following.
  A functor $F \colon \A \to \B$ is left (resp. right) exact if exactness of the sequence
  $$0 \to A \to B \to C\ (\text{resp.}\ A \to B \to C \to 0)$$
  implies exactness of the sequence
  $$0 \to FA \to FB \to FC\ (\text{resp.}\ FA \to FB \to FC \to 0).$$
\end{exercise}

\begin{proof}
  We handle here the equivalence of definitions of left exact, since the equivalence of definitions of right exact are formally dual.

  One should note the two subtly different statements these definitions make.
  The first definition says, precisely, that $F$ preserves kernels of cokernels, while the second says that $F$ preserves kernels of all morphisms.
  Given that every morphism in an abelian category factors as an epimorphism followed by a monomorphism
  $$\begin{tikzcd}
    B \arrow{rr}{g}\arrow[swap]{rd}{\op{coim} g} && C\\
    & g(B) \arrow[swap]{ur}{\op{im} g}
  \end{tikzcd},$$
  and abelian categories are binormal, this suggests that the two notions should be the same in an abelian category.
  Moreover, this factorization should be the key to the forward direction.

  The reverse implication is trivial, since kernels of cokernels are, obviously, kernels.
  Explicitly, given an exact sequence
  $$0 \to A \to B \to C \to 0$$
  the sequence
  $$0 \to A \to B \to C$$
  is also exact, so
  $$0 \to FA \to FB \to FC$$
  is exact.
  
  Assume that $F$ is left exact as in the first definition and consider an exact sequence
  $$0 \to A \overset{f}\to B \overset{g}\to C.$$
  The first definition asserts that
  $$F(f) = F(\op{im}f) = F(\op{ker}(\op{coker}f) = \op{ker}F(\op{coker}f)$$
  while the second definition says that we should have
  $$\op{ker}F(g) = F(\op{ker}g) = F(f).$$
  This implies that the desired result is
  $$\op{ker} F(\op{coker} f) = \op{ker} F(g) = \op{ker}\left( F(\op{im}g) \circ F(\op{coim} g)\right).$$
  We reduce further by observing that $F(\op{im}g)$ is necessarily monic because
  $$F(\op{im}g) = F(\op{ker}(\op{coker}g)) = \op{ker}F(\op{coker} g),$$
  hence
  $$\op{ker} F(g) = \op{ker}\left( F(\op{im}g) \circ F(\op{coim} g)\right) = \op{ker} F(\op{coim} g).$$
  The key observation is now that, since $f = \op{ker} g$,
  $$\op{coker}f = \op{coker}(\op{ker}g) =: \op{coim}g$$
  and the result now follows.
\end{proof}

\begin{remark}
  Alternatively, we could rephrase the latter argument with the following diagram argument.
  We have, by unicity, a commutative diagram
  $$\begin{tikzcd}
    0 \arrow{r} & A \arrow[bend left]{rr}{0}\arrow{r}{f} & B \arrow{r}{\op{coim}(g)}\arrow{d}{g} & B/A \arrow[dashed]{ld}{\exists !\op{im}(g)} \to 0\\
    & & C
  \end{tikzcd}$$
  with exact top row.
  Now assume that we have a morphism $\zeta \colon Z \to F(B)$ such that $F(g) \circ \zeta = 0$.
  We obtain by applying $F$ a commutative diagram of $\B$
  $$\begin{tikzcd}[column sep=huge]
    &&Z\arrow{d}{\zeta}\arrow[dashed,swap]{ld}{\exists !h}\\
    0 \arrow{r} & FA \arrow{r}{F(f)} & FB \arrow{d}{F(g)}\arrow{r}{F(\op{coim}g)} & FC^\prime \arrow{ld}{F(\op{im}g)}\\
    & & FC
  \end{tikzcd}$$
  since the middle row is exact, and, as $F(\op{im}g)$ is monic,
  \begin{eqnarray*}
    F(\op{im}g) \circ F(\op{coim}g) \circ \zeta
    &=& F(\op{im}g \circ \op{coim}g) \circ \zeta\\
    &=& F(g) \circ \zeta\\
    &=& 0
  \end{eqnarray*}
  implies $F(\op{coim}g) \circ \zeta = 0$.
  Therefore $F(f) = \op{ker} F(g)$, as desired.
\end{remark}

\begin{proposition}
  Let $\A$ be an abelian category.
  For every object $A$ of $\A$, the covariant functor
  $$\begin{tikzcd}[row sep=tiny]
    h^A \colon \A \arrow{r} & \Ab\\
    B \arrow[mapsto]{r} & \A(A,B)
  \end{tikzcd}$$
  is left exact.
\end{proposition}
\begin{proof}
  Fix an object $A$ of $\A$.
  For any exact sequence
  $$0 \to X \overset{f}\to Y \overset{g}\to Z \to 0$$
  consider the sequence
  $$\begin{tikzcd}
    \A(A,X) \arrow{r}{h^A(f)} & \A(A,Y) \arrow{r}{h^A(g)} & \A(A,Z)
  \end{tikzcd}$$
  It's clear that for any morphism $A \to X$ such that the composition
  $A \to X \overset{f}\to Y$ is the zero morphism, then the morphism $A \to X$ is necessarily zero by the definition of monic.
  Hence $h^A(f)$ is monic.

  It's similarly clear that
  $$h^A(g) \circ h^A(f) = h^A(g \circ f) = 0$$
  so it suffices to show that $h^A(f)$ is the kernel of $h^A(g)$.
  Let $h \in \A(A,Y)$ be a morphism for which $h^A(g)(h) = 0$.
  Since $f = \ker{g}$ we have a commutative diagram
  $$\begin{tikzcd}
    & A\arrow{d}{h}\arrow[dashed,swap]{ld}{\exists !k}\\
    X \arrow{r}{f} & Y \arrow{r}{g} & Z
  \end{tikzcd}$$
  and so we see that $h^A(f)(k) = f \circ k = h$.
\end{proof}

\begin{corollary}\label{cor: reps are lex}
  Let $\A$ be an abelian category.
  For every object $A$ of $\A$, the contravariant functor
  $$\begin{tikzcd}[row sep=tiny]
    h_A \colon \A \arrow{r} & \Ab\\
    B \arrow[mapsto]{r} & \A(B,A)
  \end{tikzcd}$$
  is left exact.
\end{corollary}
\begin{proof}
  Apply the previous Proposition to $\A^\opp$.
\end{proof}

\subsection{The Yoneda Embedding}
For any locally small category, $\C$, there is a functor
$$\begin{tikzcd}[row sep=tiny]
  h \colon \C \arrow{r} & \Fun(\C^\opp, \op{{\bf Sets}})\\
  X \arrow[mapsto]{r} & h_X
\end{tikzcd}$$
called the Yoneda embedding.
The objects in the essential image of the Yoneda embedding are called \textbf{representable functors}.
When $\A$ is an abelian category, we can modify this functor to have codomain $\widehat{\A}$, the full additive subcategory of $\Fun(\A,\Ab)$.
Here, the word embedding means that the Yoneda functor is fully faithful.
That is, for every pair of objects $A$,$B$ of $\A$, there is an isomorphism of abelian groups
$$\A(A,B) \cong \widehat{\A}(h_B, h_A).$$
A morphism $f \colon A \to B$ is sent to the collection of morphisms
$$\begin{tikzcd}[row sep=tiny]
  h^X(f) \colon \A(X,B) \arrow{r} & \A(X,A)\\
  g \arrow[mapsto]{r} & f \circ g
\end{tikzcd}$$
which is easily verified to be natural by way of the diagram
$$\begin{tikzcd}
  Y\arrow{d}{g} & \A(X,B) \arrow{d}{h^B(g)}\arrow{r}{h^X(f)} & \A(X,A)\arrow{d}{h^A(g)}\\
  X & \A(Y,B) \arrow{r}{h^Y(f)} & \A(Y,A)
\end{tikzcd}$$
and the fact that the composition is associative.

The isomorphism follows from the more general fact.
\begin{proposition}[Yoneda Lemma]
  Let $\A$ be a locally small abelian category.
  For any functor $F \colon \A^\opp \to \Ab$ there is an isomorphism of abelian groups
  $$\widehat{\A}(h_A, F) \cong F(A).$$

  In particular,
  $$\widehat{\A}(h_A, h_B) \cong h_B(A) = \A(A,B)$$
\end{proposition}

\begin{proof}
  The obvious morphism is defined by
  $$\begin{tikzcd}[row sep=tiny]
    \widehat{\A}(h_A, F) \arrow{r} & F(A)\\
    \eta \arrow[mapsto]{r} & \eta(A)(\id_A)
  \end{tikzcd}$$
  One constructs a morphism in the other direction as follows.
  Given an element $p \in F(A)$, define for each object $X$ of $\A$ the morphism
  $$\begin{tikzcd}[row sep=tiny]
    p(X) \colon \A(X,A) \arrow{r} & F(X)\\
    f \arrow[mapsto]{r} & F(f)(p)
  \end{tikzcd}$$
  The naturality square
  $$\begin{tikzcd}
    Y \arrow{d}{g} & \A(X,A) \arrow{d}{h_A(g)}\arrow{r}{p(X)} & F(X)\arrow{d}{F(g)}\\
    X & \A(Y,A) \arrow{r}{p(Y)} & F(Y)
  \end{tikzcd}$$
  commutes since for any $f \colon X \to A$ we have
  $$F(g) \circ p(X)(f) = F(g) \circ F(f)(p) = F(f \circ g)(p) = F(h_A(g)(f))(p) = p(Y) \circ h_A(g)(f).$$

  Finally, one verifies that these compose to the respective identities essentially by chasing the identity on $A$ through the diagram
  $$\begin{tikzcd}
    X\arrow{d}{f} & \A(A,A) \arrow{d}{h_A(f)}\arrow{r}{\eta(A)}& F(A)\arrow{d}{F(f)}\\
    A & \A(X,A) \arrow{r}{\eta(X)} & F(X)
  \end{tikzcd}$$
  for arbitrary $\eta \colon h_A \to F$ to see that if $p = \eta(A)(\id_A) \in F(A)$, then
  $$p(X)(f) = F(f)(p) = \eta(X)(f)$$
\end{proof}

A nice fact about the Yoneda embedding is that it suffices to study functors, which often times are nicer than the objects, in the following sense.

\begin{proposition}
  Let $F \colon \C \to \D$ be a fully faithful functor.
  An element $f \in \C(X,Y)$ is an isomorphism if and only if $F(f) \in \D(FX,FY)$ is an isomorphism.
  Any such functor is said to \textbf{reflect isomorphisms}.
\end{proposition}

\begin{proof}
  The forward direction is clear from the definition of functor, so assume that $F(f)$ is an isomorphism with inverse $g \in \D(FY, FX)$.
  Since $F$ is full, there exists some $h$ such that $g = F(h)$.
  Since $F$ is faithful we see that $h = f^{-1}$ by
  $$F(\id_X) = \id_{F(X)} =  F(f) \circ F(h) = F(f \circ h)$$
  and
  $$F(\id_Y) = \id_{F(Y)} =  F(h) \circ F(f) = F(h \circ f).$$
\end{proof}
\begin{proposition}
  For a locally small abelian category, $\A$, the Yoneda embedding is left exact.
  Moreover, the Yoneda functor reflects exactness.
\end{proposition}

\begin{proof}
  The first statement is essentially just combining the fact that kernels are computed levelwise and Corollary~\ref{cor: reps are lex}.
  Given a short exact sequence of $\A$
  $$0 \to A \overset{f}\to B \overset{g}\to C \to 0$$
  the morphisms 
  $$h_A \overset{h(f)} \to h_B \overset{h(g)}\to h_C$$
  are levelwise the exact sequences
  $$\begin{tikzcd}
    0 \arrow{r} & \A(X,A) \arrow{r}{h^X(f)} & \A(X,B) \arrow{r}{h^X(g)} & \A(X,C).
  \end{tikzcd}$$
  This establishes that $h_A \overset{h(f)}\to h_B$ is the kernel of $h(g)$.

  For the second statement, assume that
  $$h_A \overset{\eta} \to h_B \overset{\nu}\to h_C$$
  is an exact sequence of $\widehat{\A}$.
  Under the isomorphism
  $$\widehat{\A}(h_X, h_Y) \cong \A(X,Y)$$
  we see that by defining $f = \eta(A)(\id_A)$ and $g = \nu(B)(\id_B)$ we have $\eta = h(f)$ and $\nu h(g)$.
  The sequence
  $$A \to B \to C$$
  is indeed a complex since
  $$g \circ f = h^A(g) \circ h^A(f)(\id_A) = \eta(A) \circ \nu(A)(\id_A) = 0.$$

  We have the inclusion
  $$\begin{tikzcd}
    A \arrow{r}{f}\arrow{d}[swap]{\op{coim}f} & B \arrow{r}{\op{coim} g} & g(B) \arrow{r}{\op{im}g} & C\\
    f(A) \arrow{ur}{\op{im} f}\arrow[dashed]{r}{\exists !\alpha} & K \arrow[swap]{u}{\op{ker} g}
  \end{tikzcd}$$
  induced by the universal property of $\op{ker} g = \op{ker}(\op{coim} g)$.
  Namely, $\op{coim} g \circ \op{im} f = 0$ since $\op{im} g$ is monic, $\op{coim} f$ is epic, and
  $$0 = g \circ f = \op{im}g \circ \op{coim} g \circ \op{im} f \circ \op{coim} f.$$
  Using the exactness of the original sequence, we have a morphism $\beta \in \A(K,A)$ by
  $$\begin{tikzcd}[row sep=tiny]
    \A(K,A) \arrow{r}{h^K(f)} & \A(K,B) \arrow{r}{h^K(g)} & \A(K,C)\\
    \exists \beta \arrow[mapsto]{r} & \op{ker} g \arrow[mapsto]{r} & 0
  \end{tikzcd}$$
  We show that $\alpha^{-1} = \op{coim} f \circ \beta$.

  We see that
  $$\op{ker} g \circ \id_K
  = \op{ker} g
  = \op{im} f \circ \op{coim} f \circ \beta
  = \op{ker} g \circ \alpha \circ \op{coim} f \circ \beta$$
  implies, because $\op{ker} g$ is monic, that
  $$\id_K = \alpha \circ (\op{coim} f \circ \beta).$$
  In the other direction we have
  $$\op{im} f \circ \op{coim} f \circ \beta \circ \alpha = f \circ \beta \circ \alpha = \op{ker} g \circ \alpha = \op{im} f \circ \id_{f(A)}$$
  which implies that $\id_{f(A)} = \op{coim} f \circ \beta \circ \alpha$ because $\op{im} f$ is monic.
  Therefore $\alpha$ is an isomorphism and the sequence
  $$A \overset{f}\to B \overset{g}\to C$$
  is exact.
\end{proof}

\subsection{Weakly Effaceable Functors}
For the original reference, see Grothendieck's landmark Tohoku paper, \cite{Tohoku}.

\begin{definition}
  We say that a functor $W \colon \A^\opp \to \Ab$ is \textbf{weakly effaceable} if for every object $X$ of $\A$ and every element $x \in W(X)$, there exists an epimorphism $P \overset{f}\to X \to 0$ of $\A$ such that $W(f)(x) = 0$.
\end{definition}

Consider the failure of the exactness of the Yoneda embedding.
For a given exact sequence
$$0 \to A \overset{f}\to B \overset{g}\to C \to 0$$
we can define for each object $X$ the object $W(X)$ by the exact sequence
$$\begin{tikzcd}
  0 \arrow{r} \A(X,A) \arrow{r}{h^X(f)} & \A(X,B) \arrow{r}{h^X(g)} & \A(X,C) \arrow{r}{p(X)} & W(X) \arrow{r} & 0 
\end{tikzcd}$$
with $p(X) = \op{coker} h^X(g)$.
It's clear from the universal nature of $p$ and $W(X)$ that this assignment is functorial and $p$ is a natural transformation.

Given an object $x \in W(X)$, we can always choose a representative $h \in \A(X,C)$ and construct the pullback in $\A$
$$\begin{tikzcd}
0 \arrow{r} &  K \arrow{r}{\op{ker} g^\prime}\arrow[dashed]{d}{\exists !\op{iso}} & X \times_C B \arrow{d}{h^\prime}\arrow{r}{g^\prime} & X \arrow{d}{h}\arrow{r} & 0\\
0 \arrow{r} & A \arrow{r}{f} & B \arrow{r}{g} & C \arrow{r} & 0
\end{tikzcd}$$
with exact rows.
We obtain by the naturality of $p$ a commutative diagram
$$\begin{tikzcd}
  \A(X,C) \arrow{r}{p(X)}\arrow{d}{h_C(g^\prime)} & W(X)\arrow{d}{W(g^\prime)}\\
  \A(X \times_C B, C) \arrow{r}{p(X \times_C B)} & W(X \times_C B)
\end{tikzcd}$$
and see that $W$ is effaceable by
\begin{eqnarray*}
  W(g^\prime)(x) &=& W(g^\prime) \circ p(X)(h)\\
  &=& p(X \times_C B) \circ h_C(g^\prime)(h)\\
  &=& p(X \times_C B)(h \circ g^\prime)\\
  &=& p(X \times_C B)(g \circ h^\prime)\\
  &=& p(X \times_C B)\circ h^{X \times_C B}(g)(h)\\
  &=& 0
\end{eqnarray*}

\begin{remark}
  In general, this weakly effaceable functor is \emph{not} trivial.
  As an easy example, take a PID, $A$, $T \to T^\prime$ a non-zero morphism of torsion $A$-modules, and
  $$0 \to F_1 \to F_2 \to T^\prime \to 0$$
  a free resolution.
  The sequence
  $$0 \to \Mod{A}(T,F_1) \to \Mod{A}(T,F_2) \to \Mod{A}(T,T^\prime) \to 0$$
  will be exact if and only if we have a lift of every diagram
  $$\begin{tikzcd}
    & T\arrow{d}\arrow[dashed,swap]{ld}{\exists}\\
    F_2 \arrow{r} & T^\prime \arrow{r} & 0
  \end{tikzcd}$$
  However, this is impossible because torsion is left orthogonal to free; that is for any free $A$-module, $F$, $\Mod{A}(F, T) \cong 0$.

  Concretely, take $A = Z$, and the resolution
  $$0 \to \Z \overset{2}\to \Z \overset{p}\to \Z/2\Z \to 0.$$
  The sequence
  $$0\to \Ab(\Z/6\Z,\Z/2\Z) \to 0$$
  is not exact because of the canonical quotient $\Z/6\Z \to \Z/2\Z.$
\end{remark}

We can now make a somewhat unusual definition.
\begin{definition}
  An object $P$ of $\A$ for which \emph{every} weakly effaceable functor, $W$, constructed as above satisfies $W(P) = 0$ are \textbf{projective objects}, which correspond precisely to \textbf{injective objects} of the opposite category, $\A^\opp$.
\end{definition}

This is a somewhat perverse definition; the normal way to define a projective object of $\A$ is as an object $P$ for which the functor $\A(P, -)$ is exact, and the injective objects of $\A$ are those $Q$ for which the functor $\A(-, Q)$ is exact.

\subsection{Injective Envelopes and Grothendieck Categories: An Interlude}
We will have use for specific types of injective objects, that are in some sense maximal.
The existence of such objects is nontrivial and, frankly, out of scope for these notes.  As such, we simply define the objects and loosely quote a result about Grothendieck categories.

We start with a couple of definitions.

\begin{definition}
  An object $G$ of a category $\C$ is called a generator if for any two morphisms $f \neq g \in \C(X,Y)$, there exists a morphism $h \in \C(G,X)$ such that the diagram
  $$\begin{tikzcd}
    & G \arrow{rd}{h}\arrow{ld}{h}\\
    X \arrow{d}{f} & & X\arrow{d}{g}\\
    Y \arrow{rr}{\id_Y} && Y
  \end{tikzcd}$$
  does not commute.
\end{definition}

\begin{definition}
  An abelian category $\mathcal{A}$ is called a \textbf{Grothendieck category} if
  \begin{itemize}
  \item
    $\A$ has a generator,
  \item
    $\A$ has all coproducts, and
  \item
    filtered colimits (it's likely you know these by the unfortunate name 'direct limit') are exact.
  \end{itemize}
\end{definition}

The canonical Grothendieck categories are module categories.

\begin{definition}
  Let $\A$ be an abelian category.
  \begin{enumerate}
  \item
    An \textbf{essential monomorphism} is a monomorphism, $e \colon X \to E$, such that for every non-zero monomorphism, $m \colon Y \to E$, the pullback
    $$\begin{tikzcd}
      X \times_E Y \arrow{r}\arrow{d} & Y\arrow{d}{m}\\
      X \arrow{r}{e} & E
    \end{tikzcd}$$
    is non-zero.
  \item
    An essential monomorphism $e \colon X \to Q$ for $Q$ an injective object of $\A$ is called an \textbf{injective envelope}.
  \end{enumerate}
\end{definition}

\begin{remark}
  In the world of commutative algebra, it seems common to call an essential monomorphism an essential extension.
  One defines such an object, $E$, to be a module containing $M$ such that every submodule $E^\prime$ of $E$ intersects $M$ non-trivially.
  Since modules are just group objects in the category of sets equipped with a ring action, the fiber product over the canonical inclusions $M \to E$, $E^\prime \to E$ above is easily recognizable as $M \times_E E^\prime = M \cap E^\prime$, the fiber product in the category of sets.
  Since module categories are canonically Grothendieck abelian categories, this justifies the extension to arbitrary abelian categories.
  
  Injective hull and maximal essential extensions are also commonly used alternatives for injective envelope.
\end{remark}

\begin{proposition}
  For $\A$ any locally small abelian category, the functor category $\widehat{\A}$ has injective envelopes.
\end{proposition}

\begin{proof}
  This essentially follows from the fact that $\Ab$ is Grothendieck, and hence so is $\widehat{\A}$.  See, for example, \cite[Section 9.6]{KaSch05}.
\end{proof}

\subsection{Localizing Subcategories and Sections of a Quotient}
In this section, we will provide a nice characterization of the full subcategory, $\mathcal{L}$, of left exact functors $\mathcal{\A}$.

\begin{definition}
  A full subcategory, $\mathcal{S}$, of an abelian category, $\A$ is called a Serre (or \'{e}paisse/thick/dense) subcategory if for any short exact sequence
  $$\begin{tikzcd}
    0 \arrow{r} & X^\prime \arrow{r} & X \arrow{r} & X^{\prime\prime} \arrow{r} & 0
  \end{tikzcd}$$
  of $\A$, $X$ is an object of $\mathcal{S}$ if and only if both $X^\prime$ and $X^{\prime\prime}$ are.
\end{definition}

\begin{proposition}
  The full subcategory $\mathcal{W}$ of $\widehat{\A}$ with objects weakly effaceable functors is a Serre subcategory.
\end{proposition}

\begin{proof}
  Let
  $$0 \to F \overset{\eta}\to G \overset{\nu} \to H \to 0$$
  be a short exact sequence of $\widehat{\A}$.

  First assume that $G$ is weakly effaceable.
  Given an object $X$ of $\A$, choose an element $y \in HX$ and let $x \in \nu(X)^{-1}(y)$.
  Choose an epimorphism $P \overset{f}\to X \to 0$ of $\A$ such that $G(f)(x) = 0$.
  Then
  $$H(f)(y) = H(f) \circ \nu(X) (x) = \nu(P) \circ G(f)(x) = 0.$$
  Similarly, choose an element $z \in FX$ and an epimorphism $g \colon P^\prime \to X$ such that
  $$0 = G(g) \circ \eta(X)(z) = \eta(P^\prime) \circ F(g)(z).$$
  Since $\eta$ is monic, $F(g)(z) = 0$.
  Hence $F$ and $H$ are weakly effaceable.

  Conversely, assume that $F$ and $H$ are both weakly effaceable.
  Given an object $X$ of $\A$ and an element $x \in GX$, choose an epimorphism $f \colon P \to X$ such that
  $$H(f) \circ \nu(X)(x) = \nu(P) \circ G(f)(x) = 0$$
  and pull $G(f)(x)$ back to $z \in FP$ along $\eta(P)$.
  Choose an epimorphism $g \colon P^\prime \to P$ such that $F(g)(z) = 0$.
  Then we have
  $$G(g) \circ G(f)(x) = G(g) \circ \eta(P)(z) = \eta(P^\prime) \circ F(g)(z) = 0.$$
  Therefore $G$ is weakly effaceable.
\end{proof}


Serre subcategories of an abelian category could, reasonably, be thought of as something like normal subgroups in the following sense.

\begin{definition}
  Let $\A$ be an abelian category and let $\mathcal{S}$ be a Serre subcategory.
  The quotient category $\A/\mathcal{S}$ is the category with the same objects as $\A$ and morphisms given by the filtered colimit
  $$\A/\mathcal{S}(X,Y) = \colim_{X^\prime, Y^\prime} \A(X^\prime, Y/Y^\prime)$$
  over the collection of all subobjects $X^\prime$ of $X$ such that $X/X^\prime$ is an object of $\mathcal{S}$ and $Y^\prime$ is an $\mathcal{S}$-subobject of $Y$.

  This category is also abelian and there is a canonical additive functor $\pi \colon \A \to \A/\mathcal{S}$.
\end{definition}

\begin{remark}
  This is not necessarily trivial.  For the original reference, see \cite{Ga62}.
\end{remark}

Our eventual goal is to identify the full subcategory, $\mathcal{L}$, of left exact functors as the quotient category of $\widehat{\A}$ by the Serre subcategory $\mathcal{W}$.
The quotient category $\widehat{\A}/\mathcal{W}$ indeed exists, but we want the following nicer characterization.

\begin{proposition}\label{prop: quotient by effaceable is lex}
  Let $\A$ be a locally small abelian category.
  There exists an exact functor
  $$\pi \colon \widehat{\A} \to \mathcal{L}$$
  that is the identity on left exact functors and is zero only on $\mathcal{W}$.

  Moreover, this functor is the left adjoint of the canonical inclusion of $\mathcal{L}$ into $\widehat{\A}$.
  Such a functor is called a \textbf{reflection} and the category $\mathcal{L}$ is called a \textbf{reflective subcategory}.
\end{proposition}

Recall that a pair of functors $L \colon \C \to \D$ and $R \colon \D \to \C$ are said to be an \textbf{adjoint pair} if for all objects $X$ of $\C$ and $Y$ of $\D$ there is an isomorphism
$$\D(LX, Y) \cong \C(X, RY)$$
natural in both slots.

In the language of localization of categories, this Proposition boils down to saying that the Serre subcategory $\mathcal{W}$ is a localizing subcategory and $\widehat{\A}/\mathcal{W}$ is equivalent to $\mathcal{L}$.
More precisely,

\begin{definition}
  A Serre subcategory, $\mathcal{S}$, of an abelian category, $\A$, is called \textbf{localizing} if the canonical projection $\pi \colon \A \to \A/\mathcal{S}$ admits a fully faithful right adjoint, $\omega \colon \A/\mathcal{S} \to \A$.

  The functor $\omega$ is usually called the section functor because it gives rise to a natural isomorphism, $\varepsilon \colon \id_\A \to \omega \circ \pi$, called the unit of adjunction.
  In this sense, $\omega$ truly is a section of $\pi$.
\end{definition}

\begin{remark}
  Every adjunction $L \dashv R \colon \C \to \D$ gives rise to a unit $\varepsilon \colon \id_{\C} \to RL$ (as well as a counit, $\eta \colon LR \to \id_{\D}$), and the condition that $R$ is fully faithful is equivalent to the unit being an isomorphism.
  Dually, the condition that $L$ is fully faithful is equivalent to the counit being an isomorphism.
  Assuming these two results, it's clear that $L$ is an equivalence if and only if $L$ and $R$ are both fully faithful.
\end{remark}

We introduce some language to make the formulation of this result easier.

\begin{definition}
  Let $\A$ be an abelian category and $\mathcal{S}$ a Serre subcategory.
  \begin{enumerate}
  \item
    We say an object $X$ of $\A$ is {\bf $\mathcal{S}$-torsion free} if every monomorphism $S \to X$ with $S$ an object of $\mathcal{S}$ is zero.

    Equivalently, the $\mathcal{S}$-torsion free objects are right orthogonal to $\mathcal{S}$.
  \item
    We say an $\mathcal{S}$-torsion free object, $X$, of $\A$ is {\bf $\mathcal{S}$-closed} if every short exact sequence
    $$0 \to X \to Y \to S \to 0$$
    with $S$ an object of $\mathcal{S}$ splits.
  \end{enumerate}
\end{definition}

\begin{remark}
  Note that one can equivalently define an injective object, $Q$, of an abelian category to be an object that splits any short exact sequence
  $$0 \to Q \to X \to Y \to 0,$$
  so this is very much the statement that the $\mathcal{S}$-closed objects are a certain type of injective.
  In particular, $\mathcal{S}$-torsion free injective objects are \emph{always} $\mathcal{S}$-closed.
\end{remark}

\begin{theorem}
  Let $\A$ be an abelian category and $\mathcal{S}$ a Serre subcategory.
  The following are equivalent.
  \begin{enumerate}
  \item
    The canonical projection $\pi \colon \A \to \A/\mathcal{S}$ admits a right adjoint, and
  \item
    Every object of $\A$ has a maximal $\mathcal{S}$-subobject, and every object with trivial maximal $\mathcal{S}$-subobject has a monomorphism into an $\mathcal{S}$-closed object.
  \end{enumerate}
  In this case, the essential image of the section functor is the full subcategory of $\mathcal{S}$-closed objects.
\end{theorem}

This effectively states that in the case of $\widehat{\A}$ and the Serre subcategory, $\mathcal{W}$, we need only show the following:
\begin{itemize}
\item
  every functor admits a maximal weakly effaceable subfunctor,
\item
  every functor with trivial maximal weakly effaceable subfunctor has a monomorphism into a $\mathcal{W}$-closed object, and
\item
  the $\mathcal{W}$-closed objects are the left exact functors.
\end{itemize}

The first statement is not difficult to prove.
\begin{proposition}
  Every additive functor $F \colon \mathcal{A}^\opp \to \Ab$ has a maximal weakly effaceable subfunctor.
\end{proposition}

\begin{proof}
  Let $X$ be an object of $\A$.
  We will call an element $x \in FX$ effaceable if there exists an epimorphism $f \colon P \to X$ such that $F(f)(x) = 0$.
  Define the set $WX$ to be the subset of effaceable elements of $FX$.
  Note that this is clearly non-empty since it always contains the zero element.
  For any pair of elements $x_1, x_2 \in WX$, we can choose epimorphisms
  $f_1 \colon P_1 \to X$ and $f_2 \colon P_2 \to X$, and it's easy to see that if $f \colon P_1 \times_X P_2 \to X$ is the epimorphism from the pullback to $X$, then
  $$F(f)(x_1 - x_2) = 0.$$
  This establishes that $WX$ is indeed an abelian group.

  To see this is indeed a functorial assignment, take a morphism $\varphi \in \A(Y, X)$.
  For any object $x \in WX$ we can choose an epimorphism $f \colon P \to X$ such that $F(f)(x) = 0$.
  Taking the pullback in $\A$
  $$\begin{tikzcd}
    P \times_X Y\arrow{r}{f^\prime}\arrow{d}{\varphi^\prime} & Y \arrow{r}\arrow{d}{\varphi} & 0\\
    P \arrow{r}{f} & X \arrow{r} & 0
  \end{tikzcd}$$
  we see that
  $$F(f^\prime) \circ F(\varphi)(x) = F(\varphi^\prime) \circ F(f)(x) = 0.$$
  Hence we can simply define the morphism $W(\varphi) = F(\varphi)|_{WX}$.

  The canonical inclusion of $W$ into $F$ is clearly natural, and $W$ is necessarily maximal by construction.
\end{proof}

If the statement we wish to prove is true, then at the very least we must have the following.
\begin{lemma}
  An object $F$ of $\widehat{\A}$ is $\mathcal{W}$-torsion free if and only if $F$ transforms epimorphisms in $\A$ into monomorphisms in $\widehat{\A}$.
  In particular, the objects of $\mathcal{L}$ are $\mathcal{W}$-torsion free.
\end{lemma}
\begin{proof}
  Assume that $M$ transforms epimorphisms of $\A$ into monomorphisms of $\widehat{\A}$.
  For an object $X$ of $\A$ there exists an effaceable element, $x \in MX$.
  By definition there exists an epimorphism $f \colon P \to X$ such that $M(f)(x) = 0$.
  However, by assumption, we have an exact sequence of abelian groups
  $$0 \to MX \overset{M(f)}\to MP$$
  and thus $x = 0$.
  Hence $M$ is $\mathcal{W}$-torsion free.

  Conversely, assume that $F$ is $\mathcal{W}$-torsion free.
  Given an epimorphism $f \colon P \to X$ of $\A$, suppose that for some $x \in F(X)$ that $F(f)(x) = 0$.
  By definition, the element $x$ is effaced, so it follows that $x$ must be zero.
  Therefore $F(f)$ is a monomorphism.
\end{proof}

To prove the remaining statements, we make use of the fact that $\widehat{\A}$ has injective envelopes.  First a nice statement about the injective objects of $\widehat{\A}$.

\begin{lemma}
  If $Q \colon \A^\opp \to \Ab$ is an injective object of $\widehat{\A}$, then $Q$ is right exact.
\end{lemma}

\begin{proof}
  Consider an exact sequence
  $$0 \to A \to B \to C \to 0$$
  of $\A$.
  Then we have an exact sequence
  $$0 \to h_A \to h_B \to h_C$$
  of $\widehat{\A}$.
  Applying $\widehat{\A}(-,Q)$ we obtain by the Yoneda Lemma the commutative diagram
  $$\begin{tikzcd}
    \widehat{\A}(h_C, Q) \arrow{d}[rotate=90,yshift=-.5ex,xshift=-.5ex]{\sim}\arrow{r} & \widehat{\A}(h_B, Q) \arrow{r}\arrow{d}[rotate=90,yshift=-.5ex,xshift=-.5ex]{\sim} & \widehat{\A}(h_A, Q)\arrow{d}[rotate=90,yshift=-.5ex,xshift=-.5ex]{\sim} \arrow{r} & 0\\
    Q(C) \arrow{r} & Q(B) \arrow{r} & Q(A)
  \end{tikzcd}$$
  with exact top row.
  Therefore $Q$ is right exact.
\end{proof}

Next we characterize the injective envelopes of $\mathcal{W}$-torsion free objects.
This characterization will prove important as it will allow us to conclude that $\mathcal{W}$ is indeed a localizing subcategory.

\begin{lemma}
  If $M$ is a $\mathcal{W}$-torsion free object, then any essential extension, $e \colon M \to E$, is also $\mathcal{W}$-torsion free.
  Consequently, the injective envelope of an object of any $\mathcal{W}$-torsion free object is exact.
\end{lemma}

\begin{proof}
  The second statement boils down to noting that if $E$ is injective, then for any short exact sequence of $\A$
  $$0 \to A \to B \to C \to 0$$
  the sequence of abelian groups
  $$EC \to EB \to EA \to 0$$
  is exact.
  The fact that $E$ is torsion free implies $EC \to EB$ is monic and hence
  $$0 \to EC \to EB \to EA \to 0$$
  is exact.
  
  Assume to the contrary that there exists an epimorphism $f \colon P \to X$ for which $E(f) \colon EX \to EP$ is not monic.
  Choose an element $0 \neq x \in EX$ such that $E(f)(x) = 0$.
  Define for each object $Y$ of $\A$ the set
  $$F(Y) = \left\{y \in EY \;\middle\vert\; \exists g \in \A(Y,X), E(f)(x) = y\right\}.$$
  If we have a morphism $h \in \A(Y^\prime, Y)$ then for any element $y \in F(Y)$ we have a morphism $g \in \A(Y,X)$ such that $E(g)(x) = y$, so
  $$E(g \circ h)(x) = E(h) \circ E(g)(x) = E(h)(y)$$
  implies we have a morphism $F(h) = E(f)|_{FY} \colon FY \to FY^\prime$.
  That $FY$ is a subgroup is easy to verify.

  It's clear that this is not the zero functor, since $x \in F(X)$, so the pullback $F \times_E M$ is necessarily non-zero because $e$ is essential.
  In particular, for some $Y$, $F(Y) \times_E M(Y) = F(Y) \cap M(Y) \neq 0$.
  This implies that there exists a non-zero element $y \in M(Y)$ and a morphism $g \colon Y \to X$ such that $E(g)(x) = y$.
  If we take pullback in $\A$
  $$\begin{tikzcd}
    P \times_X Y \arrow{d}{g^\prime}\arrow{r}{f^\prime} & Y\arrow{d}{g} \arrow{r} & 0\\
    P \arrow{r}{f} & X \arrow{r} & 0
  \end{tikzcd}$$
  then we have a commutative diagram in $\Ab$
  $$\begin{tikzcd}
    0 \arrow{r} & M(X) \arrow{r}{M(f)}\arrow{d}{M(g)} & M(P)\arrow{d}{M(g^\prime)}\\
    0 \arrow{r} & M(Y) \arrow{r}{M(f^\prime)} & M(P \times_X Y)
  \end{tikzcd}$$
  We observe that $M(f^\prime)(y) \neq 0$ since $M(f^\prime)$ is monic.
  But by the diagram
  $$\begin{tikzcd}
    E(X) \arrow{r}{E(f)}\arrow{d}{E(g)} & E(P)\arrow{d}{E(g^\prime)}\\
    E(Y) \arrow{r}{E(f^\prime)} & E(P \times_X Y)\\
    M(Y)\arrow{u}{e(Y)}\arrow{r}{M(f^\prime)} & M(P \times_X Y)\arrow{u}{e(P \times_XY)}
  \end{tikzcd}$$
  we have
  \begin{eqnarray*}
    0 
    &=& E(g^\prime) \circ E(f)(x)\\
    &=& E(f^\prime) \circ E(g)(x)\\
    &=& E(f^\prime)(y)\\
    &=& E(f^\prime) \circ e(Y)(y)\\
    &=& e(P \times_X Y) \circ M(f^\prime)(y).
  \end{eqnarray*}
  Since $e(P \times_X Y)$ is, by assumption, a monomorphism, it follows that $M(f^\prime)(y) = 0$, a contradiction.
  Therefore $E$ transforms epimorphisms of $\A$ into monomorphisms.
\end{proof}

\begin{corollary}\label{cor: effaceables are localizing}
  The Serre subcategory $\mathcal{W}$ is a localizing subcategory.
\end{corollary}

\begin{proof}
  For any $\mathcal{W}$-closed object $F$ of $\widehat{\A}$, the injective envelope $e \colon F \to E$ is an essential monomorphism, so $E$ is also $\mathcal{W}$-torsion free.
  Since $E$ is injective it splits every short exact sequence
  $$0 \to E \to F \to W \to 0$$
  and hence is $\mathcal{W}$-closed.
  Moreover, as will prove useful later, we observe that not only is this a monomorphism into an $\A$-closed object, but it is in fact a monomorphism into an exact functor.
\end{proof}

It remains only to show that an object is $\mathcal{W}$-closed if and only if it is left exact.

\begin{proposition}\label{prop: effaceable closed iff lex}
  An object $F$ of $\widehat{\A}$ is $\mathcal{W}$-closed if and only if it is an object of $\mathcal{L}$.
\end{proposition}

\begin{proof}
  Given a $\mathcal{W}$-torsion free object $F$ of $\widehat{\A}$, let $e \colon F \to E$ be its injective envelope, and let $p \colon E \to E/F$ be the cokernel of $e$.
  Note that $E$ is necessarily an exact functor.
  For a short exact sequence of $\A$
  $$0 \to A \overset{f}\to B \overset{g}\to C \to 0$$
  we obtain a commutative diagram
  $$\begin{tikzcd}
    & 0\arrow{d} & 0\arrow{d} & 0\arrow{d}\\
    0 \arrow{r} & F(C) \arrow{r}{F(g)}\arrow{d}{e(C)} & F(B) \arrow{r}{F(f)}\arrow{d}{e(B)} & F(A) \arrow{d}{e(A)}\\
    0 \arrow{r} & E(C)\arrow{d}{p(C)}\arrow{r}{E(g)} & E(B)\arrow{d}{p(B)}\arrow{r}{E(f)} & E(A)\\
    0 \arrow{r} & E/F(C)\arrow{r}{E/F(g)} & E/F(B)
  \end{tikzcd}$$
  with exact columns and exact middle row.
  Since the top row is exact if and only if the bottom row is exact, we reduce to showing that $F$ is $\mathcal{W}$-closed if and only if $E/F$ is $\mathcal{W}$-torsion free.
  
  First assume that $F$ is $\widehat{\A}$-closed and let $e \colon F \to E$ be its injective envelope.
  We have a short exact sequence of $\widehat{\A}$
  $$0 \to F \overset{e}\to E \overset{p}\to E/F \to 0.$$
  If $m \colon W \to E/F$ is a weakly effaceable subfunctor, then we have the pullback diagram
  $$\begin{tikzcd}
    0 \arrow{r} & K \arrow[dashed]{d}{\exists !h}\arrow{r}{\op{ker} p^\prime}& E \times_{E/F} W \arrow{d}{m^\prime}\arrow{r}{p^\prime} & W\arrow{d}{m}\arrow{r} & 0\\
    0 \arrow{r} & F \arrow{r}{e} & E \arrow{r}{p} & E/F \arrow{r} & 0
  \end{tikzcd}$$
  with $h$ an isomorphism.
  Since $F$ is assumed to be $\mathcal{W}$-closed, the top sequence splits $E \times_{E/F} W \cong F \oplus W$, and we see that the canonical inclusion composes with $m^\prime$ to be the zero morphism
  $$\begin{tikzcd}
    W \arrow{rrd}{0}\arrow{r} & F \oplus W \arrow{r}{\sim} & E \times_{E/F} W \arrow{d}{m^\prime}\\
    & & E
  \end{tikzcd}$$
  because $E$ is also $\mathcal{W}$-closed.
  Since $m^\prime$ is monic, the canonical inclusion is necessarily the zero morphism and by the commutative diagram
  $$\begin{tikzcd}
    W \arrow{r}\arrow[swap]{rd}{\id_W} & F \oplus W \arrow{d}\\
    & W
  \end{tikzcd}$$
  we see that $\id_W = 0$ implies $W \cong 0$.
  Hence $E/F$ is $\mathcal{W}$-torsion free.

  Conversely, assume that $E/F$ is $\mathcal{W}$-torsion free.
  Necessarily, $F$ is $\mathcal{W}$-torsion free since it is a subfunctor of the $\mathcal{W}$-closed object $E$.
  Since $E$ is injective we have an extension
  $$\begin{tikzcd}
    0 \arrow{r} & F \arrow{d}{e}\arrow{r}{\eta} & G\arrow[dashed]{ld}{\exists \tau}\\
    & E
  \end{tikzcd}$$
  We have a commutative diagram
  $$\begin{tikzcd}
    0 \arrow{r} & F \arrow{r}{\eta}\arrow{d}{\id_F} & G \arrow{r}{\nu}\arrow{d}{\tau} & W \arrow{r}\arrow[dashed]{d}{\exists !0} & 0\\
    0 \arrow{r} & F \arrow{r}{e} & E \arrow{r}{p} & E/F \arrow{r} & 0
  \end{tikzcd}$$
  since $\nu = \op{coker} \eta$,
  $$(p \circ \tau) \circ \eta = p \circ e = 0,$$
  and $E/F$ is assumed to be $\mathcal{W}$-torsion free.
  This implies that $p \circ \tau = 0$, so we obtain by the universal property of $e = \op{ker} p$ a commutative diagram
  $$\begin{tikzcd}
    & G \arrow[dashed,swap]{ld}{\exists !\gamma}\arrow{d}{\tau}\\
    F \arrow{r}{e} & E \arrow{r}{p} & E/F
  \end{tikzcd}$$
  This gives the desired splitting since $e$ monic and
  $$e \circ \id_F = e = \tau \circ \eta = e \circ \gamma \circ \eta$$
  implies $\id_F = \gamma \circ \eta$.
  Therefore $F$ is $\mathcal{W}$-closed.
  
\end{proof}

\begin{proof}[Proof of Proposition~\ref{prop: quotient by effaceable is lex}]
  By Corollary~\ref{cor: effaceables are localizing}, $\mathcal{W}$ is a localizing subcategory, which means we have a fully faithful section, $\omega \colon \widehat{\A}/\mathcal{W} \to \widehat{\A}$, of the canonical projection, $\pi \colon \widehat{\A} \to \widehat{\A}/\mathcal{W}$.
  The essential image of $\omega$ is the full subcategory of $\mathcal{W}$-closed objects, which by by Proposition~\ref{prop: effaceable closed iff lex}, is precisely the full subcategory, $\mathcal{L}$, of left exact functors.
  Since a fully faithful functor induces an equivalence with its essential image, we conclude that $\widehat{\A}/\mathcal{W}$ is equivalent to $\mathcal{L}$, and the reflection onto $\mathcal{L}$ is precisely the composition of $\pi$ with this equivalence.
\end{proof}

\begin{corollary}
  The Yoneda embedding
  $$h \colon \A \to \widehat{\A}$$
  is exact and fully faithful.
\end{corollary}
\begin{proof}
  We have already shown that $h$ is fully faithful, has essential image in $\mathcal{L}$, and is left exact.
  We simply observe that we have shown above that for any exact sequence
  $$0 \to A \to B \to C \to 0$$
  of $\A$, we have an exact sequence of $\widehat{\A}$
  $$0 \to h_A \to h_B \to h_C \to W \to 0$$
  with $W$ a weakly effaceable functor.
  Since $\pi \colon \widehat{\A} \to \mathcal{L}$ is exact and takes weakly effaceable functors to 0, we obtain a short exact sequence
  $$0 \to h_A \to h_B \to h_C \to 0$$
  of $\mathcal{L}$.
\end{proof}
\subsection{The Embedding}

We sketch the embedding,
\begin{theorem}[Mitchell]
  Let $\A$ be a small abelian category.
  Then there exists a ring, $A$, and an exact, fully faithful functor $\A \to \Mod{A^\opp}$, the category of right $A$-modules.
\end{theorem}

\begin{theorem}
  If $\A$ is a Grothendieck category and $\mathcal{S}$ is a localizing subcategory of $\A$, then $\A/\mathcal{S}$ is again a Grothendieck category.
\end{theorem}

The heart of the argument is showing that $\mathcal{L}$, which is a Grothendieck category, admits a projective generator, $P$, which, as the name suggests, is both a generator and a projective object of $P$.
One then applies the following form of a theorem of Gabriel and Mitchell, 
\begin{theorem}
  Let $\A$ be a Grothendieck category with a projective generator, $P$.
  The endomorphisms $A = \A(P,P)$ form a ring, and there is an equivalence of categories between $\Mod{A}$ and $\A$.
\end{theorem}

For $P$ a projective generator of $\mathcal{L}$, $A = \mathcal{L}(P,P)$, the composition of the functors
$$\A \to \mathcal{L} \to \Mod{A}$$
gives the desired embedding.
\section{\cite[Section 2.1]{Weibel95}: Derived Functors}

\subsection{$\delta$-Functors}
\begin{definition}
  A (covariant) \textbf{homological} (resp. \textbf{cohomological}) $\delta$-functor between $\A$ and $\B$ is a collection of additive functors $T_n \colon A \to \B$ (resp. $T^n \colon \A \to \B$) for $0 \leq n$, together with morphisms
  $$\delta_n \colon T_n(C) \to T_{n-1}(A)\quad (\text{resp.}\ \delta^n \colon T^n(C) \to T^{n+1}(A))$$
  defined for each short exact sequence $0 \to A \to B \to C \to 0$ of $\A$.
  Here, we make the convention that $T^n = T_n = 0$ for $n < 0$.
  These two conditions are imposed:
  \begin{enumerate}
  \item
    For each short exact sequence as above, there is a long exact sequence
    $$\cdots \to T_{n+1}(C) \overset{\delta}\to T_n(A) \to T_n(B) \to T_n(C) \overset{\delta}\to T_{n-1}(A) \to \cdots$$
    or, respectively,
    $$\cdots \to T^{n-1}(C) \overset{\delta}\to T^n(A) \to T^n(B) \to T^n(C) \overset{\delta}\to T^{n+1}(A) \to \cdots$$
    In particular, $T_0$ is right exact and $T^0$ is left exact.
  \item
    For each morphism of short exact sequences
    $$\begin{tikzcd}
      0 \arrow{r} & A^\prime \arrow{d}\arrow{r} & B^\prime \arrow{d}\arrow{r} & C^\prime \arrow{d}\arrow{r} & 0\\
      0 \arrow{r} & A \arrow{r} & B \arrow{r} & C \arrow{r} & 0
    \end{tikzcd}$$
    the $\delta$'s give a commutative diagram
    $$\begin{tikzcd}
      T_n(C^\prime) \arrow{r}{\delta}\arrow{d} & T_{n-1}(A^\prime)\arrow{d}\\
      T_n(C) \arrow{r}{\delta} & T_{n-1}(A)
    \end{tikzcd}$$
    or, respectively,
    $$\begin{tikzcd}
      T^n(C^\prime) \arrow{r}{\delta}\arrow{d} & T^{n+1}(A^\prime)\arrow{d}\\
      T^n(C) \arrow{r}{\delta} & T^{n+1}(A)
    \end{tikzcd}$$
  \end{enumerate}
\end{definition}

\begin{example}
  Homology and cohomology give $\delta$-functors from bounded below (co)chain complexes.
\end{example}

\begin{exercise}
  Let $\mathcal{S}$ be the category of short exact sequences of $\A$.
  Show that $\delta_n$ is a natural transformation from the functor sending a short exact
  $$0 \to A \to B \to C \to 0$$
  to $T_n(C)$ to the functor sending the same short exact sequence to $T_{n-1}(C)$.
\end{exercise}
\begin{proof}
  Use the naturality square in part 2 of the definition.
\end{proof}

\begin{example}[$p$-torsion]
  If $p$ is an integer, the functors $T_0(A) = A/pA$ and
  $$T_1(A) =\ _pA \equiv \left\{ a \in A \;\middle\vert\; pa = 0\right\}$$
  fit together to form a homological $\delta$-functor, or a cohomological $\delta$-functor (with $T^0 = T_1$ and $T^1 = T_0$) from $\Ab$ to $\Ab$.
  To see this, apply the Snake Lemma to
  $$\begin{tikzcd}
    0 \arrow{r} & A\arrow{r}\arrow{d}{p} & B \arrow{d}{p}\arrow{r} & C\arrow{d}{p}\arrow{r} & 0\\
    0 \arrow{r} & A \arrow{r} & B \arrow{r} & C \arrow{r} & 0
  \end{tikzcd}$$
  to get the long exact sequence
  $$0 \to\ _pA \to\ _pB \to\ _pC \overset{\delta}\to A/pA \to B/pB \to C/pC \to 0.$$

  One can easily generalize this to an arbitrary ring element $a \in A$ and a cohomological delta functor $\Mod{A}$ to $\Ab$.
  Once we define left derived functors, we will see that we obtain a homological $\delta$-functor, $\op{Tor}$, associated to tensor products.
  In particular, for any short exact sequence of $\Mod{A}$
  $$0 \to M \to N \to P \to 0$$
  we have, in this particular example,
  $$\begin{tikzcd}
    \cdots \arrow{r} & \op{Tor}^A_2(A/aA, P) \arrow{r} & \op{Tor}^A_1(A/aA, M) \arrow[out=-30,in=150]{lld}\\
    \op{Tor}^A_1(A/aA, N) \arrow{r} & \op{Tor}^A_1(A/aA,P) \arrow{r} & N/aN\arrow[out=-30,in=150]{lld}\\
    \arrow{r} M/Am & P/aP \arrow{r} & 0
  \end{tikzcd}$$
  
  where
  $$M/aM \cong M \otimes_A A/a.$$

  If $a$ is a non-zero divisor, then $\op{Tor}^A_1(A/aA, M) =\ _aM$ and $\op{Tor}^A_n(A/a, M) = 0$ for $2 \leq n$.
  In general, $\ _aA \neq 0$, while $A$ being a flat $A$-module implies $\op{Tor}_1^A(A/a, A) = 0$, so they aren't the same.

  $\op{Tor}_1^A(M, A/a)$ is the quotient of $\ _aM$ by the submodule ($\ _aA)M$ generated by
  $$\left\{bm \;\middle\vert\; ab = 0, b \in A, m \in M\right\}.$$
  These funtors are universal in the following sense.
\end{example}

\begin{definition}
  A morphism $S \to T$ of $\delta$-functors is a system of natural transformations $S_n \to T_n$ (resp. $S^n \to T^n$) that commute with $\delta$.
  This is fancy language for the assertion that there is a commutative ladder diagram connecting the long exact sequences for $S$ and $T$ associated to any short exact sequence in $\A$.

  A homological $\delta$-functor is \textbf{universal} if, given any other $\delta$-functor $S$ and a natural transformation $f_0 \colon S_0 \to T_0$, there exists a unique morphism $\{f_n \colon S_n \to T_n\}$ of $\delta$-functors that extends $f_0$.

  A cohomological $\delta$-functor, $T$, is universal if, given any other $\delta$-functor $S$ and $f^0 \colon T^0 \to S^0$, there exists a unique morphism $T \to S$ of $\delta$-functors extending $f^0$.
\end{definition}

\begin{example}
  Homology and cohomology are the canonical universal delta functors.
  In fact, this will be used to show that left and right derived functors are universal, since they're defined using (co)homology.
\end{example}

\begin{exercise}
  If $F \colon \A \to \B$ is an exact functor, show that $T_0 = F$ and $T_n = 0$ for $n \neq 0$ defines a universal $\delta$-functor (of both homological and cohomological type).
\end{exercise}
\begin{proof}
  For any homological $\delta$-functor $S$ with a morphism $f_0 \colon S_0 \to F$, define the $f_n$ to be the zero morphism.
  This is clearly the unique morphism since for any short exact sequence
  $$0 \to A \to B \to C \to 0$$
  the long exact sequence of $F$ is just
  $$\cdots \to 0 \to 0 \to FA \to FB \to FC \to 0.$$

  The cohomological case is formally dual.
\end{proof}

%%%%%%%%%%%%%%%%%%%%%%%%%%%%%%%%%%%%%%%%%%%%%%%%%%

\bibliographystyle{alpha}
\bibliography{bibliography}
\end{document}
