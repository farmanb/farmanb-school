\documentclass[reqno, 12pt]{amsart}
\usepackage{style}

\title[Seminar]{Homological Algebra Seminar}
\author{Blake Farman}
\date{October 4, 2017}

\begin{document}
\maketitle

\section{\cite[Section 3.5]{Weibel95}: Derived Functors of the Inverse Limit}

Recall that a category, $\C$, is said to be \textbf{complete} if for all small categories $I$ and all functors $F \colon I \to \C$, the limit, $\lim_I F$, exists in $\C$.
\begin{exercise}
  An abelian category, $\A$, is complete if and only if
  $\prod_{i \in I} A_i$
  exists in $\A$.

  That is, for an abelian category, replacing 'every small category $I$' by 'every small \textit{discrete} indexing category $I$' in the definition gives an equivalent notion.
\end{exercise}

\begin{remark}
  This is also known as the condition (AB$3^\ast$).
\end{remark}

Let $\A$ be a complete abelian category with enough injectives, $I$ be a small category and consider the functor category $\A^I = \Fun(I,A)$.
By \cite[Example 2.3.13]{Weibel95} this is an abelian category with enough injectives.

\begin{definition}
  A \textbf{filtered category} $I$ is a category for which
  \begin{itemize}
  \item
    $I$ is non-empty,
  \item
    For any two objects $i,j$ of $I$ there exists an object $k$ of $I$ and morphisms
    $$\begin{tikzcd}
      i \arrow{r} & k & j \arrow{l}
    \end{tikzcd}$$
  \item
    For any two parallel morphisms
    $$\begin{tikzcd}
      i \arrow[shift left=.5ex]{r}{f} \arrow[swap,shift left=-.5ex]{r}{g} & j
    \end{tikzcd}$$
    there exists a morphism $h \colon j \to k$ such that the diagram
    $$\begin{tikzcd}
      i \arrow{r}{f}\arrow{d}{g}& j\arrow{d}{h}\\
        j\arrow{r}{h} & k
    \end{tikzcd}$$
    commutes.
  \end{itemize}
  A \textbf{cofiltered category} $I$ is a category for which $I^\opp$ is filtered.
\end{definition}

Say we have a totally ordered set $I$, and a collection of objects $\{A_i\}_I$ with morphisms $A_i \to A_j$ for $j \leq i$ such that all possible morphisms
$$\begin{tikzcd}
  A_i \arrow{r}\arrow{rd} & A_j \arrow{d}\\
  & A_k
\end{tikzcd}$$
commute.
This gives an \textbf{inverse system} of objects
$$\begin{tikzcd}
  \cdots \arrow{r} & A_{i} \arrow{r} & A_{j} \arrow{r} & \cdots
\end{tikzcd}$$
The limit of the inverse system, often called the \textbf{inverse limit} and denoted by $A = \varprojlim_I A_i$, is a universal cone
$$\begin{tikzcd}
  & A \arrow{d}\arrow{rd}\arrow{ld}\arrow{rrd}\\
  \cdots \arrow{r} & A_{i} \arrow{r} & A_{j} \arrow{r} & \cdots
\end{tikzcd}$$
meaning that $A$ is the final object amongs all such cones.
It's easy to verify that such an $I$ is an example of a cofiltered category, and $A$ is a cofiltered limit; meaning that it is the limit of a functor from a cofiltered category.

\begin{example}
  Let $I$ be the category with objects $\mathbb{N}$ and morphisms
  $$I(m,n) = \left\{
  \begin{array}{ll}
    \{\ast\} & n \leq m,\\
    \emptyset & \text{else}
  \end{array}
  \right.$$
  Fix a rational prime, $p$.
  In the category of rings, we have the inverse system
  $$\begin{tikzcd}
    \cdots \arrow{r} & \Z/p^n\Z \arrow{r} & \Z/p^{n-1}\Z \arrow{r} & \cdots \arrow{r} & \Z/p\Z
  \end{tikzcd}$$
  with the morphisms
  $$\begin{tikzcd}
    p^{n}\Z \arrow{r}\arrow[bend left]{rrr}{0} & p^{n-1}\Z \arrow{r} & \Z \arrow{d}\arrow{r} & \Z/p^{n-1}\Z\\
    & & \Z/p^{n}\Z \arrow[dashed]{ur}{\exists !}
  \end{tikzcd}$$
\end{example}
\bibliographystyle{alpha}
\bibliography{bibliography}
\end{document}
