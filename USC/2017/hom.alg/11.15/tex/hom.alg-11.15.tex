\documentclass[reqno, 12pt]{amsart}
\usepackage{style}

\title[Seminar]{Homological Algebra Seminar}
\author{Blake Farman}
\date{October 4, 2017}

\begin{document}
\maketitle

\section{\cite[Section 3.5]{Weibel95}: Derived Functors of the Inverse Limit}

Recall that a category, $\C$, is said to be \textbf{complete} if for all small categories $\I$ and all functors $F \colon \I \to \C$, the limit, $\lim_\I F$, exists in $\C$.
\begin{exercise}
  An abelian category, $\A$, is complete if and only if
  $\prod_{i \in \I} A_i$
  exists in $\A$.

  That is, for an abelian category, replacing 'every small category $\I$' by 'every small \textit{discrete} indexing category $\I$' in the definition gives an equivalent notion.
\end{exercise}

\begin{remark}
  This is also known as the condition (AB$3^\ast$).
\end{remark}

Let $\A$ be a complete abelian category, let $\I$ be a small category and consider the functor category $\A^\I = \Fun(\I,A)$.

\begin{proposition}
  The functor
  $$\lim_\I \colon \Fun(\I, \A) \to \A$$
  admits a left adjoint
  $$\Delta \colon \A \to \Fun(\I, \A),$$
  called the \textbf{diagonal functor}.
  As a result, the functor $\lim_\I$ is left exact.
  
  The functor $\Delta$ is defined as follows.
  For each object $A$ of $\A$, the functor $\Delta(A) \colon \I \to \A$ is defined to be $\Delta(A)(i) = A$ on objects and $\Delta(A)(\varphi) = \id_A$ on morphisms $\varphi \in \I(i,j)$.
  To each morphism $f \in \A(A,B)$ we assign the natural transformation $\Delta(f)(i) = f$.
\end{proposition}

\begin{proof}
  We provide the isomorphism
  $$\Fun(\I,\A)(\Delta(A), F) \cong \A(A, \lim_\I F)$$
  and note that naturality is clear from the universal property of limits and the construction.

  Fix a functor $F \colon \I \to \A$ and an object $A$ of $\A$.
  For ease of notation, let $L = \lim_\I F$.
  First we note that $L$ comes equipped with morphisms $\lambda_i \colon L \to F(i)$ and  we have a natural transformation $\lambda \colon \Delta(L) \to F$
  $$\begin{tikzcd}
    i \arrow{d}{\varphi} & L \arrow{r}{\lambda(i) = \lambda_i} \arrow{d}{\id_{L F}} & F(i)\arrow{d}{F(\varphi)}\\
    j & L \arrow{r}{\lambda(j) = \lambda_j} & F(j)
  \end{tikzcd}$$

  We construct the morphism
  $$\Phi \colon \A(A, L) \to \Fun(\I, \A)(\Delta(A), F)$$
  as follows.
  Given a morphism $f \in \A(A, L)$ we then have the horizontal composition of natural transformations
  $$\begin{tikzcd}
    i \arrow{d}{\varphi} & A \arrow{r}{f}\arrow{d}{\id_A} & L \arrow{r}{\lambda_i}\arrow{d}{\id_{L}} & F(i)\arrow{d}{F(\varphi)}\\
    j & A \arrow{r}{f} & L \arrow{r}{\lambda_j} & F(j)
  \end{tikzcd}$$
  since
  $$F(\varphi) \circ \lambda_i \circ f = \lambda_j \circ f.$$
  Define
  $$\Phi(f)(i) = (\lambda \circ \Delta)(i) = \lambda(i) \circ \Delta(f)(i) = \lambda_i \circ f.$$

  Next we construct a morphism
  $$\Psi \colon \Fun(\I,\A)(\Delta(A), F) \to \A(A, L)$$
  and show that this is the inverse of $\Phi$.
  Giving a morphism $\eta \colon \Delta(A) \to F$ is equivalent to giving a collection of morphisms $\eta(i) \colon \Delta(A)(i) = A \to F(i)$ for each $i$ subject to the naturality squares in $\A$
  $$\begin{tikzcd}
    i \arrow{d}{\varphi} & A \arrow{d}{\id_A}\arrow{r}{\eta(i)} & F(i)\arrow{d}{F(\varphi)}\\
    j & A \arrow{r}{\eta(j)} & F(j)
  \end{tikzcd}$$
  which is just to say that the object $A$ equipped with the morphisms $A \overset{\eta(i)}\to F(i)$ is a cone over the inverse system defined by $F$.
  By the universal property of limits, there exists a unique morphism $f \in \A(A,L)$ such that for all morphisms $\varphi \in \I(i,j)$ the diagram
  $$\begin{tikzcd}[row sep=large,column sep=huge]
    & A\arrow[swap]{ldd}{\eta(i)}\arrow{rdd}{\eta(j)}\arrow[dashed]{d}{\exists !f}\\
    & L\arrow[swap]{ld}{\lambda_i}\arrow{rd}{\lambda_j}\\
    F(i) \arrow{rr}{F(\varphi)} & & F(j)
  \end{tikzcd}$$
  Define $\Psi(\eta) = f$ and note that by construction we have 
  $$\Phi \circ \Psi(\eta)(i) = f \circ \lambda_i = \eta(i).$$
  Now for an arbitrary $f \in \A(A,L)$ we have
  $$\Psi \circ \Phi(f) = \Psi(\lambda \circ \Delta(f))$$
  is the unique morphism making all diagrams
  $$\begin{tikzcd}[row sep=large,column sep=huge]
    & A\arrow[swap]{ldd}{f \circ \lambda_i}\arrow{rdd}{f \circ \lambda_j}\arrow[dashed]{d}{\exists !}\\
    & L\arrow[swap]{ld}{\lambda_i}\arrow{rd}{\lambda_j}\\
    F(i) \arrow{rr}{F(\varphi)} & & F(j)
  \end{tikzcd}$$
  commute, which implies by unicity that $f = \Psi \circ \Phi(f)$.
\end{proof}

By \cite[Example 2.3.13]{Weibel95} $\Fun(\I,\A)$ is an abelian category with enough injectives, so the right derived functors of $\lim_\I$ make sense.

\bibliographystyle{alpha}
\bibliography{bibliography}
\end{document}
