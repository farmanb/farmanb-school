\documentclass[10pt]{amsart}
\usepackage{amsmath,amsthm,amssymb,amsfonts,wasysym,mymath}
\openup 5pt
\author{Blake Farman}
\title{Math-331:\\Homework 4}
\date{February 25, 2010}\pdfpagewidth 8.5in
\usepackage[margin=1in]{geometry}
\pdfpageheight 11in
\begin{document}
\maketitle

\newtheorem{thm}{}

\begin{thm}
  \label{Ex1}
  Prove that if $f$ is meromorphic on an open set $U \subseteq \C$ with only a finite number of poles, then $f = g/h$ where $g$ and $h$ are analytic on $U$. 
\begin{proof}
  Let $S$ be the set of singularities of $f$.  
  Each $s \in S$ is a pole, hence we have by our Theorem from class that $f$ has a convergent Laurent series expansion for some deleted neighborhood of $s$,
  \begin{equation}
    \label{1.1}
    f(z) = \frac{c_{-n}}{(z-s)^n} + \frac{c_{-n+1}}{(z-s)^{n-1}} + \ldots + c_0 + c_1(z-s) + \ldots\,.
  \end{equation}
  Multiplying \eqref{1.1} by $(z-s)^n$, which is entire, we obtain
  $$(z-s)^nf(z) = c_{-n} + c_{-n+1}(z-s) + \ldots + c_0(z-s)^n + c_1(z-s)^{n+1} + \ldots\,.$$
  We observe that $(z-s)^nf(z)$ now converges not just on the deleted neighborhood, but at $s$ as well; indeed it takes on the value $c_{-n}$ at $z = s$.
  Hence $(z-s)^nf(z)$ is analytic at $s$.
  
  Let $s_1, s_2, \ldots, s_k$ be an enumeration of $S$ and define the function 
  \begin{equation}
    \label{1.2}
    g(z) = f(z)\cdot\prod_{i=1}^k(z-s_i)^{n_i}, \quad  n_i = -\ord_{s_i}{f}.
  \end{equation}
  If we rewrite \eqref{1.2} as $g(z) = (z-s_i)^{n_i}f(z){\cdot}\prod_{j=1,j \not = i}^k(z-s_j)^{n_j}$, then it is clear by the argument above that $g$ is analytic at each $s_i$.
  Since $f$ was assumed to be meromorphic on $U$, we have shown that $g$ is analytic on $U$.
  Moreover, if we take $h(z) = \prod_{i=1}^k(z-s_i)^{n_i}$, then $f = g/h$ is the ratio of functions analytic on $U$, as was to be shown.
\end{proof}
\end{thm}

\begin{thm}
  \label{Ex2}
  Let $$f(z) = \sum\limits_{n=1}^\infty \dfrac{1}{(z+n)^2}.$$
Prove that $f$ is meromorphic on $\C$ and determine its poles and their orders.
\end{thm}

\begin{thm}
  \label{Ex3}
  A set $S$ is called {\it star shaped} if there is a point $z_0 \in S$ such that
the line segment between $z_0$ and any point in $S$ is contained in $S$.
Prove that a star shaped set is simply connected.
\end{thm}

\begin{thm}
  \label{Ex4}
  Let $\gamma_0, \gamma_1$ and $\delta_0,\delta_1$ be four closed curves in $U$,
all of which have the same initial point $z_0$ (for some parameterizations).  
Assume that $\gamma_0$ and $\delta_0$ are
homotopic in $U$ with homotopy given by $\Gamma_0$;
and assume that $\gamma_1$ and $\delta_1$ are homotopic in $U$ with homotopy
given by $\Gamma_1$.
Prove that the product curves $\gamma_0 \gamma_1$ and $\delta_0 \delta_1$ are also homotopic
in $U$ by exhibiting an explicit homotopy in terms of $\Gamma_0$ and $\Gamma_1$
(and showing it is a homotopy).
\end{thm}

\begin{thm}
  \label{Ex5}
  Let $\gamma$ be a closed curve in $U$ with initial point $z_0$, and let
$\gamma^{-}$ denote its reverse curve.
Prove that $\gamma \gamma^{-}$ is null homotopic
in $U$ (exhibit and verify an explicit homotopy).
\end{thm}

\end{document}
