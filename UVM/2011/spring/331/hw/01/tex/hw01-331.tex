\documentclass[10pt]{amsart}
\usepackage{graphicx,enumerate,amsmath,amsthm,amssymb}
\openup 5pt
\author{Blake Farman}
\title{Math-333: Homework 3\\}
\date{Friday, September 24, 2010}\pdfpagewidth 8.5in
\usepackage[margin=1in]{geometry}
\pdfpageheight 11in
\begin{document}

\maketitle
\renewcommand{\qedsymbol}{\(\blacksquare\)}


\newcommand{\Z}{\mathbb{Z}}
\newcommand{\R}{\mathbb{R}}
\newcommand{\Q}{\mathbb{Q}}
\newcommand{\C}{\mathbb{C}}

\renewcommand{\phi}{\varphi}
\newenvironment{alphaenum}{
  \begin{enumerate}
    \renewcommand{\theenumi}{(\alph{enumi})}
    \renewcommand{\labelenumi}{\theenumi}
  }
  {\end{enumerate}}
\newtheorem{thm}{}

\begin{thm}
  \label{Ex1}
  Verify the Cauchy-Riemann equations for $f(z) = z^2 + z^3$.
  \begin{proof}
    Let $z = x+iy$, so then $f$ can be rewritten as 
    $$f(z) = u(x,y) + iv(x,y) = x^3 + x^2 - y^2(3x+1) + i(y(2x+3x^2) - y^3).$$
    Then taking the necessary partials, we get
    \begin{align*}
      \begin{split}
        u_x &= 3x^2 + 2x - 3y^2 \quad  v_x = 2y + 6ay\\
        u_y &= -(2y + 6xy)  \quad \quad v_y = 3x^2 + 2x - 3y^2
      \end{split}
    \end{align*}
    Therefore $u_x = v_y$ and $u_y = -v_x$ as desired.
  \end{proof}
\end{thm}

\begin{thm}
  \label{Ex2}
  Let $f(z) = u(x,y) + iv(x,y)$ where $z = x+iy$ and $u,v$ are the real and imaginary parts of $f$.
  Suppose $f$ satisfies the Cuachy-Riemann equations at the point $z = x_0+iy_0.$
  Verify the Cauchy-Riemann equations for the function $f^2$ at $x_0 + iy_0$.
  \begin{proof}
    Using the component functions of $f$, rewrite $f^2$ as 
    \begin{equation}
      \label{Ex1:1}
      f^2(z) = u^2(x,y) - v^2(x,y) + i2u(x,y)v(x,y).
    \end{equation}
    Let $U(x,y) = u^2(x,y) - v^2(x,y)$ and $V(x,y) = 2u(x,y)v(x,y)$ so that \eqref{Ex1:1} reduces to $f^2(z) = U(x,y) + iV(x,y)$.
    To see that $f^2(z)$ satisfies the Cauchy-Riemann equations, take the appropriate partial derivatives to obtain
    \begin{align*}
      \begin{split}
        U_x(x_0,y_0) &= 2u(x_0,y_0)u_x(x_0,y_0) - 2v(x_0,y_0)v_x(x_0,y_0),\\
        U_y(x_0,y_0) &= 2u(x_0,y_0)u_y(x_0,y_0) - 2v(x_0,y_0)v_y(x_0,y_0),\\
        V_x(x_0,y_0) &= 2u(x_0,y_0)v_x(x_0,y_0) - 2u_x(x_0,y_0)v(x_0,y_0), \,\text{and}\\
        V_y(x_0,y_0) &= 2u(x_0,y_0)v_y(x_0,y_0) + 2u_y(x_0,y_0)v(x_0,y_0).
      \end{split}
    \end{align*}
    By hypothesis, we know that $f$ satisfies the Cuachy-Riemann equations for every $z \in \C$, so by substituting the Cauchy-Riemann equations for $f$ into the equations for $V_x$ and $V_y$ we obtain
    \begin{align*}
      \begin{split}
        -V_x(x_0,y_0) &= 2u(x_0,y_0)u_y(x_0,y_0) - 2v_y(x_0,y_0)v(x_0,y_0)\\
        &= U_y(x_0,y_0),
    \end{split}
    \end{align*}
    and
    \begin{align*}
      \begin{split}
        V_y(x_0,y_0) &= 2u(x_0,y_0)u_x(x_0,y_0) - 2v_x(x_0,y_0)v(x_0,y_0)\\
        &= U_x(x_0,y_0),
    \end{split}
    \end{align*}
    Therefore $f^2$ satisfies the Cauchy-Riemann equations, as desired.
  \end{proof}
\end{thm}

\begin{thm}
  \label{Ex3}
  Prove that if $f:\C \rightarrow \C$ is entire, then so is the function $F(z) = \overline{f(\overline{z})}$.
\end{thm}

\begin{thm}
  \label{Ex4}
  Assume $f$ is holomorphic on an open disk, $D$, and is real valued on $D$.  
  Prove that $f$ is constant on $D$.
  \begin{proof}
    Let $f= u + iv$.  
    Since $f$ is real-valued on $D$, we have that $v$ is identically the zero function and so must $v_x$ and $v_y$ be.
    Now, because $f$ is holomorphic, $f$ satisfies the Cauchy-Riemann equations and we have $u_x = u_y = 0$.

    To see that $f$ is constant on $D$ it remains only to show that $u$ is a constant function.
    Take $z_0 = (x_0,y_0)$ to be the center of D and let $z = (x,y)$ be an arbitrary point on the interior of D and consider the partials.
    By the Mean-Value Theorem from Real Analysis and the convexity of $D$, there exists $c$ between $x_0$ and $x$ such that 
    \begin{equation}
      \label{Ex4:1}
      u(x,y_0) - u(x_0,y_0) = (x - x_0)u_x(d,y_0) = 0
    \end{equation}
    Similarly, there exists a point $d$ between $y_0$ and $y$ such that 
    \begin{equation}
      \label{Ex4:2}
      u(x,y) - u(x,y_0) = (y - y_0)u_y(x_0,c) = 0.
    \end{equation}
    Hence by \eqref{Ex4:1} and \eqref{Ex4:2} we have $u(x,y_0) = u(x_0,y_0)$ and  $u(x,y) = u(x,y_0)$, thus $u(x,y) = u(x_0,y_0)$.
    Therefore, since the choice of $z$ was arbitrary, $u$ is constant on $D$, as desired.
  \end{proof}
\end{thm}

\begin{thm}
  \label{Ex5}
  Show that $f(z) = |z|$ is not holomorphic at any point.
\end{thm}

\begin{thm}
  \label{Ex6}
  Show that complex conjugation is the only continuous non-identity automorphism of $\C$.
\end{thm}
\end{document}
