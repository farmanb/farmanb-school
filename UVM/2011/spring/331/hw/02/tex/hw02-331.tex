\documentclass[10pt]{amsart}
\usepackage{graphicx,enumerate,amsmath,amsthm,amssymb}
\openup 5pt
\author{Blake Farman}
\title{Math-351: Homework 2\\}
\date{Friday, February 4, 2011}\pdfpagewidth 8.5in
\pdfpageheight 11in
\begin{document}

\maketitle

\renewcommand{\qedsymbol}{\(\blacksquare\)}
\newcommand{\C}{\mathbb{C}}
\newcommand{\real}[1]{\operatorname{Re}}

\newtheorem*{1}{1}
\begin{1}
  Determine the radius of convergence for the following power series:
  
  \item{(d)}
    \[\sum_{n=2}^{\infty} (\log^2(n)) z^n\]
  \item{(h)}
    \[\sum_{n=1}^{\infty} \frac{(n!)^3}{(3n)!} z^n\]
  
  \begin{proof}
    \item{(d)}
      Using the ratio test, \[r = \lim_{n\rightarrow\infty}\left(\frac{\log(n+1)}{\log(n)}\right)^2.\]
      By an application of L'H\^opital's Rule, 
      \begin{align*}
        \begin{split}
          \lim_{n\rightarrow\infty} \frac{\log(n+1)}{\log(n)} &= \lim_{n\rightarrow\infty} \frac{n}{n+1}\\
          & = 1.
        \end{split}
      \end{align*}
      Hence by the Algebraic Limit Laws
      \begin{align*}
        \begin{split}
          \lim_{n\rightarrow\infty} \left(\frac{\log(n+1)}{\log(n)}\right)^2 &= 1^2\\
          & = 1.
        \end{split}
      \end{align*}
      Therefore the radius of convergence is \(R = \frac{1}{r} = 1\).
      
    \item{(h)}
      Using the ratio test once again, \(\lim_{n\rightarrow\infty}\frac{a_{n+1}}{a_n}\) reduces with some minor algebra to 
      \begin{align*}
        r &= \lim_{n\rightarrow\infty} \frac{(n+1)!^3(3n)!}{(3n+3)!(n!)^3}\\
        &= \frac{1}{3}\lim_{n\rightarrow\infty}\frac{n^2+2n+1}{9n^2+9n+2}\\
        &= \frac{1}{27}.
      \end{align*}
      Therefore the radius of convergence is given by \(R = \frac{1}{r} = 27\).
    \end{proof}
  \end{1}

  \newtheorem*{2}{2}
  \begin{2}
    Find the power series expansion of \[f(z) = \frac{4+3z}{(z+1)(z+2)^2}\] expanded about zero, and find its radius of convergence.
    \begin{proof}
      Using the 'apart' function on Mathematica, \(f\) can be rewritten by the method of partial fraction decomposition as \[f(z) = \frac{1}{z+1} - \frac{z}{(2+z)^2}.\]
      Then by using \[\frac{1}{1-u} = \sum_{n=0}^{\infty} z^n\] the series for the first term, \[\frac{1}{1+z} = \sum_{n=0}^{\infty} (-1)^nz^n,\] is obtained by using \(u = -z\) and substituting accordingly.
      The series for the second term, \[\frac{z}{(z+2)^2} = \sum_{n=1}^{\infty} \frac{(-1)^{n}n}{2^{n}}z^n,\] is obtained by making the substitution \(u = \frac{-z}{2}\) and using the observation that 
      \[\frac{2}{(2+z)^2} = \frac{1}{2}\left(\frac{1}{(1 - (\frac{-z}{2})^2}\right) = \frac{d}{dz}\left(\frac{1}{1-u}\right).\]
      Combining these series together, the result after minor algebraic rearrangement is \[f(z) = 1 + \sum_{n=1}^{\infty}\frac{2^n-n}{2^n}z^n.\]
      Finally, by the ratio test \[r = \lim_{n\rightarrow\infty} \frac{2^{n+1}-n-1}{2^{n+1}}\frac{2^n}{2^n - n} = \lim_{n\rightarrow\infty} \frac{2^{n+1} - n - 1}{2^{n+1} - n}.\]
      Applying L'H\^opital's Rule we have \(r = 1\).
      Therefore the radius of convergence is given by \(R = \frac{1}{r} = 1\).
    \end{proof}
  \end{2}

  \newtheorem*{3}{3}
  \begin{3}
    Expand \[f(z) = \frac{2z+3}{z+1}\] in powers of \(z-1\) and determine the radius of convergence of this power series (the radius of the disc of convergence with center \(z=1\)). 
    \begin{proof}
      First rewrite \(f\) as \[f(x) = \frac{2(z+1) + 1}{z+1} = 2 + \frac{1}{z+1}.\] 
      Then, to expand \(f\), rearrange the sum above as \[f(x) = 2 + \frac{1}{2}\left(\frac{1}{1 - (\frac{-(z-1)}{2})}\right).\] 
      Now expand the rightmost term as a series to obtain \[f(x) = 2 + \sum_{n=0}^{\infty}\frac{(-1)^n}{2^n}(z-1)^n.\]
      Finally, calculate the radius of convergence using the ratio test to get \[r = \lim_{n\rightarrow\infty}\frac{2^n}{2^{n+1}} = \frac{1}{2}.\]
      Therefore the radius of convergence is given by \(R = \frac{1}{r} = 2.\)
    \end{proof}
  \end{3}

  \newtheorem*{4}{4}
  \begin{4}
    For what values of \(z\) does \[\sum\limits_{n=0}^{\infty} \left (\frac{z}{z+1} \right )^n\] converge?
    \begin{proof}
      First observe the series can be rewritten as \[\frac{1}{1 - \frac{z}{z+1}} = \sum_{n=0}^{\infty}\left(\frac{z}{z+1}\right)^n,\] which converges for \(\left|z\right| < 1\).
      Squaring both sides of the inequality, we have \[\frac{z\overline{z}}{\overline{z+1}} = \frac{x^2 + y^2}{(x+1)^2 + y^2} < 1.\]
      Now solving the inequality above using routine algebra we obtain the inequality \[x > \frac{-1}{2}.\]
      Therefore the series converges for \(z \in \C\) with \(\real{Re}(z) > \frac{-1}{2}\).
    \end{proof}
  \end{4}

  \newtheorem*{5}{5}
  \begin{5}
    Suppose \(f\) is analytic on the disc \(D\) and for each \(a \in D\) the power series of \(f\) expanded about \(a\) has at least one coefficient equal to zero.  
    Prove that \(f\) is a polynomial on \(D\).  
  \end{5}
\end{document}

