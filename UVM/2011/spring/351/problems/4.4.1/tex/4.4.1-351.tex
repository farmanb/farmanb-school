\documentclass[10pt]{amsart}
\usepackage{amsmath,amsthm,amssymb,amsfonts}
\openup 5pt
\author{Blake Farman}
\title{Math-351:\\4.4.1}
\date{February 16, 2010}\pdfpagewidth 8.5in
\usepackage[margin=1in]{geometry}
\pdfpageheight 11in
\begin{document}
\maketitle

\newcommand{\Z}{\mathbb{Z}}
\newcommand{\R}{\mathbb{R}}
\newcommand{\Q}{\mathbb{Q}}

\renewcommand{\qedsymbol}{\(\blacksquare\)}
\newcommand{\znz}[1]{\Z / #1\Z}
\newcommand{\mznz}[1]{(\Z / #1\Z)^*}

\renewcommand{\phi}{\varphi}
\newenvironment{alphaenum}{
  \begin{enumerate}
    \renewcommand{\theenumi}{(\alph{enumi})}
    \renewcommand{\labelenumi}{\theenumi}
  }
  {\end{enumerate}}

\newcommand{\quadeq}[3]{\frac{-(#2) \pm \sqrt{(#2)^2 - 4(#1)(#3)}}{2(#3)}}
% \newcommand{quadeq}[3]{}

\newtheorem{thm}{}

\newcommand{\A}{\mathfrak{a}}
\begin{thm}
  Let $\A$ be a non-zero ideal of $\mathcal{O}_K$.  Show that $\mathfrak{a} \cap \Z \not = \left\{ 0 \right\}$.
  \begin{proof}
    Let $0 \not =\alpha \in \A$ be given and let $m_{\alpha}(x) = x^n + a_{n-1}x^{n-1} + \ldots + a_1x + a_0$ be the minimal, monic polynomial with integer coefficients such that $\alpha$ is a root.
    Then we have 
    \begin{equation}
      \label{1}
      \alpha^n + a_{n-1}\alpha^{n-1} + \ldots + a_1\alpha + a_0 = 0.
    \end{equation}
    Now observe that if $a_0 = 0$, then 
    $$\alpha(\alpha^{n-1} + a_{n-1}\alpha^{n-2} \ldots + a_2\alpha + a_1) = 0.$$
    Since $\alpha$ was assumed to be non-zero, it follows that $\alpha^{n-1} + a_{n-1}\alpha^{n-2} \ldots + a_2\alpha + a_1 = 0$, which contradicts the minimality of $m_{\alpha}$.
    Hence it suffices to assume $a_0 \not = 0$ and thus we can rearrange \eqref{1} to obtain
    \begin{equation}
      \label{2}
      -a_0 = {\alpha}^{n} + a_{n-1}{\alpha}^{n-1} + \ldots + a_1\alpha.
    \end{equation}
    Now observe that since the coefficients are elements of the ring $\mathcal{O}_K$ and $\mathfrak{a}$ is an ideal, the right-hand side of \eqref{2} is an element of $\mathfrak{a}$.
    Therefore $\mathfrak{a} \cap \Z \not = \left\{0\right\}$, as desired.
  \end{proof}
\end{thm}

\end{document}
