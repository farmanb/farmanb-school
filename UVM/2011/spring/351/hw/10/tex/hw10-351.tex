\documentclass[10pt]{amsart}
\usepackage{amsmath,amsthm,amssymb,amsfonts,mymath,amscd}
\openup 5pt
\author{Blake Farman}
\title{Math-351:\\Homework 10}
\date{April 18, 2011}
\usepackage[margin=1in]{geometry}
\pdfpagewidth 8.5in
\pdfpageheight 11in
\begin{document}
\maketitle

\newtheorem{thm}{}

\newcommand{\OO}{\mathcal{O}}
\renewcommand{\epsilon}{\varepsilon}
\begin{thm}
  \label{Ex1}
  \begin{alphaenum}
  \item
    Show that, for any real quadratic field $K = \Q(\sqrt{D})$, where $D$ is a positive square-free integer, $U_K \cong \Z/2\Z \times \Z$.
    That is, there is a fundamental unit $\epsilon \in U_K$ such that $U_K = \left\{ \pm \epsilon^k \mid\, k\in\Z\right\}$.
    Conclude that the equation $x^2 - Dy^2 = 1$ has infinitely many integer solutions for $D \equiv 2,3 \pmod{4}$ and that the equation $x^2 - Dy^2 = 4$ has infinitely many integer solutions for $D \equiv 1 \pmod{4}$.
  \item
    Let $D \equiv 2,3 \pmod{4}$.
    Let $b$ be the smallest positive integer such that one of $Db^2 \pm 1$ is a square, say $a^2$, $a > 0$.
    Then $a + b \sqrt{D}$ is a unit.
    Show that it is the fundamental unit.
    Using this algorithm, determine the fundamental units of $\Q(\sqrt{2})$, $\Q(\sqrt{3})$.
  \item
    Devise a similar algorithm to compute the fundamental unit in $\Q(\sqrt{D})$ for $D \equiv 1 \pmod{4}$.
    Determine the fundamental unit of $\Q(\sqrt{5})$.
  \end{alphaenum}
  
  \begin{proof}
    \begin{alphaenum}
    \item
      Note that $K$ is a real degree 2 extension of $\Q$ by assumption, hence the only roots of unity are $\pm 1$ and $r = 1$.
      It follows immediately from Dirichlet's Theorem that $U_K \cong \Z/2\Z \times \Z$.
      Moreover, we have $U_K = \left\{\pm \epsilon^k \mid\, k \in \Z\right\}$.
      We observe that $N\alpha = \pm1$ holds for any $\alpha \in U_K$ and by the multiplicativity of the norm we have $N\alpha^2 = 1$.
      
      If $D \equiv 2,3 \pmod{4}$, then we can use the ring structure of $\OO_K$ to write $\alpha^2 = a + b\sqrt{D}$ for some $a,b \in \Z$.
      Hence it follows from the observation above that $a^2 - b^2D = 1$
      Similarly, if $D \equiv 1 \pmod{4}$ we have $\alpha^2 = a + b\omega$ for some $a,b \in \Z$, where $\omega = (1 + \sqrt{D})/2$ and 
      $$4N\alpha^2 = 4a^2 + 4ab + b^2 - b^2D = (2a + b)^2 - b^2D = 4.$$
      
      In both cases $\alpha = \epsilon^k$ for some $k \in \Z$, hence we have the units whose coefficients satisfy the desired equation given by
      $$\{\alpha^2 \mid\, \alpha \in U_K\} = \{\epsilon^{2k} \mid\, k \in \Z\}.$$
      Therefore the equation $x^2 - Dy^2 = 1$ has infinitely many integer solutions for $D \equiv 2,3 \pmod{4}$ and that the equation $x^2 - Dy^2 = 4$ has infinitely many integer solutions for $D \equiv 1 \pmod{4}$.
    \item
      Let $\epsilon$ be the fundamental unit.
      Since $\pm\epsilon$ and $\pm\epsilon^{-1}$ are all fundamental units, it suffices to assume $\epsilon = \alpha + \beta\sqrt{D}$ for $0< \alpha,\beta \in \Z$.
      Assume to the contrary that for some $k > 1$ we have $\epsilon^k = a + b\sqrt{D}$.
      Since $\alpha,\beta > 0$, it must be the case that $\alpha < a$ and $\beta < b$.
      But then $\alpha^2 - \beta^2D = \pm 1$ contradicts the minimality of $a$ and $b$.
      Therefore $\epsilon = a + b\sqrt{D}$.
      
      The fundamental units of $\Q(\sqrt{2})$ and $\Q(\sqrt{3})$ are $1 + \sqrt{2}$ and $2 + \sqrt{3}$, respectively.
    \item
      By the same argument, mutatis mutundis, the minimal solution to $b^2D \pm 4 = a^2$ yields the fundamental unit.
      In $\Q(\sqrt{5})$, that unit is $\omega = (1 + \sqrt{5})/2$
    \end{alphaenum}
  \end{proof}
\end{thm}

\begin{thm}
  \newcommand{\vol}{\operatorname{vol}}
  \label{Ex2}
  Let $\Gamma = \Z v_1 + \ldots \Z v_n$ be a lattice in $\R^n$ and let $\Delta = \vol(\Gamma) = \det((v_1\, v_2\, \ldots v_n)^{T})$.
  If $X$ is a convex, symmetric, bounded domain in $\R^n$ with $\vol(X) > 2^n\Delta$, then $X$ contains a non-zero point of $\Gamma$.

  \begin{proof}
    Let $x$ be an arbitrary point of $X$.
    Since $X$ is convex and symmetric, the line
    $$\ell(t) = x(2t - 1),\, t \in [0,1]$$
    is contained in $X$ and passes through the origin.
    Define for each $i \leq n$ the sets 
    $A_i = \left\{r \in \R^{\geq 0} \mid \, rv_i \in X\right\}$
    and let $a_i = \sup A_i$ and construct from the the compact sets
    $B_i = \left\{ra_i \mid\, r\in\R, |r| \leq a_i\right\}$
    a closed $n$-box
    $$B = \prod_{i=1}^n B_i.$$
    The $v_i$ form a basis of $\R^n$ by the definition of a lattice, so it is now clear that $B$ contains $X$.
    Moreover, by assumption $X$ is an open set and hence must lie on the interior of $B$.
    
    Now assume to the contrary that $X$ does not contain a non-zero point of $\Gamma$ and observe then $a_i \leq 1$ necessarily holds for each $i \leq n$.
    However, we now have the inequality
    $$2^n\Delta < \vol(X) < \vol(B) = \det
    \begin{pmatrix}
      2a_1 & & \text{\huge{0}}\\
      & \ddots &\\
      \text{\huge{0}} & & 2a_n
    \end{pmatrix}
    \det
    \begin{pmatrix}
      v_1\\
      \vdots\\
      v_n
    \end{pmatrix}
    \leq 2^n\Delta.
    $$
    This is a contradiction.
    Therefore $X$ contains a non-zero point of $\Gamma$.
  \end{proof}
\end{thm}

\end{document}
