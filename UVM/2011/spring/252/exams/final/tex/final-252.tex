\documentclass[10pt]{amsart}
\usepackage{amsmath,amsthm,amssymb,amsfonts,mymath}
\openup 5pt
\author{Blake Farman}
\title{Math-252:\\Final Exam}
\date{May 6, 2011}
\pdfpagewidth 8.5in
\pdfpageheight 11in
\usepackage[margin=1in]{geometry}
\begin{document}
\maketitle

\newtheorem{thm}{}

\begin{thm}
  \label{Ex1}
  Determine up to similarity all $3 \times 3$ matrices $A$ with coefficients in the finite field $\F_7$ with $A^3 = I$.
  In particular, determine whether such a matrix can be diagonalized over $\F_7$
  
  \begin{proof}
    If $A^3 = I$, then the minimal polynomial of $A$, $m_A$, must divide $x^3 - 1 = (x-1)(x^2 + x + 1)$, which splits completely in $\F_7$ into the linear factors $(x-1)(x-2)(x-4)$.
    It then follows from Corollary 25 of Section 12.3 that such a matrix can be diagonalized, as the minimal polynomial does not have any repeated roots.
    The possible choices for the minimal polynomial are $x-1, x-2, (x-1)(x-2), (x-4), (x-1)(x-2), (x-1)(x-4), (x-2)(x-4) \text{ and } (x-1)(x-2)(x-4)$.
    The associated elementary divisors are
    \begin{enumerate}
    \item
      $(x-1),(x-1),(x-1)$
    \item
      $(x-2),(x-2),(x-2)$
    \item
      $(x-4),(x-4),(x-4)$
    \item
      $(x-1),(x-1),(x-2)$
    \item
      $(x-1),(x-2),(x-2)$
    \item
      $(x-1),(x-1),(x-4)$
    \item
      $(x-1),(x-4),(x-4)$
    \item
      $(x-2),(x-2),(x-4)$
    \item
      $(x-2),(x-4),(x-4)$
    \item
      $(x-1),(x-2),(x-4)$
    \end{enumerate}
    The associated Jordan canonical forms are then $I, 2I, 4I,$ 
  
    \begin{enumerate}
      \item
        $I$
      \item
        $2I$
      \item
        $4I$
      \item
        $
        \begin{pmatrix}
          1 & 0 & 0\\
          0 & 1 & 0\\
          0 & 0 & 2
        \end{pmatrix} 
        $
      \item
        $\begin{pmatrix}
          1 & 0 & 0\\
          0 & 2 & 0\\
          0 & 0 & 2
        \end{pmatrix}$
      \item
        $\begin{pmatrix}
          1 & 0 & 0\\
          0 & 1 & 0\\
          0 & 0 & 4
        \end{pmatrix}$
      \item
        $\begin{pmatrix}
          1 & 0 & 0\\
          0 & 4 & 0\\
          0 & 0 & 4
        \end{pmatrix}$
      \item
        $\begin{pmatrix}
          2 & 0 & 0\\
          0 & 2 & 0\\
          0 & 0 & 4
        \end{pmatrix}
        $
      \item
        $\begin{pmatrix}
          2 & 0 & 0\\
          0 & 4 & 0\\
          0 & 0 & 4
        \end{pmatrix}$
      \item
        $\begin{pmatrix}
          1 & 0 & 0\\
          0 & 2 & 0\\
          0 & 0 & 4
        \end{pmatrix}$
      \end{enumerate}            
  \end{proof}
\end{thm}

\begin{thm}
  \label{Ex2}
  Determine whether the regular $15420$-gon can be constructed by straightedge and compass.
  \begin{proof}
    We observe that the prime factorization of 15420 is $2^2 \cdot 3 \cdot 5 \cdot 257$, where 3, 5, and 257 are all Fermat primes.
    By Proposition 29 of Section 14.5, the regular 15420-gon can be constructed by straightedge and compass.
  \end{proof}
\end{thm}

\begin{thm}
  \label{Ex3}
  Let $\alpha$ be a root of the polynomial $x^4 + 4$.
  Compute $[\Q(\alpha):\Q]$.
  
  \begin{proof}
    We observe that $x^4 + 4 = (x^2 - 2x + 2)(x^2 + 2x + 2)$, which are both irreducible by Eisenstein's criterion applied for the prime 2.
    Hence any root of $x^4 + 4$ must have a degree 2 minimal polynomial.
    Therefore $[\Q(\alpha):\Q] = 2$
  \end{proof}
\end{thm}

\begin{thm}
  \label{Ex4}
  Give an example of fields $F_1, F_2, F_3$ with $\Q \subset F_1 \subset F_2 \subset F_3$, $[F_3 : \Q] = 8$ and each field is Galois over all its subfields with the exception that $F_2$ is not Galois over $\Q$.

  \begin{proof}
    Let $F_1 = \Q\left(\sqrt{3}\right)$, $F_2 = \Q\left(\sqrt{1 + \sqrt{3}}\right)$, $F_3 = \Q\left(\sqrt{1 + \sqrt{3}},\sqrt{1 - \sqrt{3}}\right)$.
    From Homework 9, Exercise 5, the fields $ F_1 \subset F_2 \subset F_3$ satisfy these conditions.
    The fields $F_1$ and $F_3$ are both Galois extensions of $\Q$ with $\Gal(F_3/\Q) \cong D_8$.
    The subgroup corresponding to $F_2$ is isomorphic to the subgroup generated by $sr^2$, which is not normal in $D_8$.
    Hence $F_2$ not Galois over $\Q$ by the Fundamental Theorem of Galois Theory.
    
  \end{proof}
\end{thm}

\begin{thm}
  \label{Ex5}
  Suppose $F = \Q\left(i, \sqrt{3}, \sqrt{5}, \sqrt{7}\right)$.
  Prove that $F$ does not contain the element $\sqrt[3]{2}$.
  
  \begin{proof}
    Observe $F_0 = \Q \subset F_1 = \Q\left(i\right) \subset F_2 = F_1\left(\sqrt{3}\right) \subset F_3 = F_2\left(\sqrt{5}\right) \subset F = F_3\left(\sqrt{7}\right)$ is a root extension with $[F_{i+1}:F_i] = 2$.
    Hence $[F:\Q] = 8$.
    Suppose $\sqrt[3]{2}$ were contained in $F$.  
    Then $K = \Q\left(\sqrt[3]{2}\right)$ would be a subfield of $F$.
    However, $[K:\Q] = 3$ does not divide $[F:\Q]$.
    Therefore $K \not \subset F$.
  \end{proof}
\end{thm}

\begin{thm}
  \label{Ex6}
  \newcommand{\Tr}{\operatorname{Tr}}
  Suppose $K/F$ is a Galois extension with Galois group $G$.
  For any $\alpha \in K$ define the trace of $\alpha$ from $K$ to $F$ to be $$\Tr_{K/F}(\alpha) = \sum_{\sigma \in G} \sigma(\alpha),$$
  i.e. the sum of all the Galois conjugates of $\alpha$.
  Prove that $\Tr_{K/F}(\alpha)$ is an element of $F$.
  
  \begin{proof}
    Let $\tau \in G$ be given and consider $$\tau(\Tr_{K/F}(\alpha)) = \tau\left(\sum_{\sigma \in G}\sigma(\alpha)\right) = \sum_{\sigma \in G} \tau\sigma(\alpha).$$
    Since $\tau$ is an element of $G$ and the sum ranges over all $\sigma \in G$, we obtain a reordering of the original sum.
    Hence $$\tau\left(\sum_{\sigma\in G} \sigma(\alpha)\right) = \Tr_{K/F}(\alpha).$$
    Seeing as the choice of $\tau$ was arbitrary, we have that $\Tr_{K/F}(\alpha)$ is fixed by $G$ and thus is an element of $F$.
  \end{proof}
\end{thm}

\begin{thm}
  \label{Ex7}
  \begin{alphaenum}
  \item
    Determine the splitting field $K$ of the polynomial $(x^3 - 2)(x^2 - 3)$ over $\Q$.
    In particular, determine the degree $[K:\Q]$.
  \item
    Explain why $K$ is Galois over $\Q$ and determine the Galois group $\Gal(K/\Q)$ up to isomorphism.
  \item
    Determine the number of subfields $E$ of $K$ with $[E:\Q] = 6$.
  \end{alphaenum}

  \begin{proof}
    \begin{alphaenum}
    \item
      The splitting field for $x^3 - 2$ is the field $Q\left(\sqrt[3]{2},\rho\right)$, where $\rho$ is a primitive third root of unity satisfying $\overline{\rho} + \rho + 1 = 0$.
      Moreover, $\im{\rho} = \sqrt{3}/2$, so by the identity $2\im{\rho} = \sqrt{3} = \rho - \overline{\rho}$, we have that the splitting field for $x^3 - 2$ contains $\Q\left(\sqrt{3}\right)$, the splitting field for $x^2 - 3$.
      Therefore $K = \Q\left(\sqrt[3]{2},\rho\right)$, an extension of degree 6.
    \item
      $K$ the splitting field for a separable polynomial and hence is Galois over $\Q$ by Theorem 13 of Section 14.2.
      The isomorphism type of the Galois group is $S_3$, as was determined in class.
    \item
      $K$ is the only subfield of $K$ with degree 6.
    \end{alphaenum}
  \end{proof}
\end{thm}

\begin{thm}
  \label{Ex8}
  \begin{enumerate}
  \item
    Suppose the field $L$ is a subfield of the complex numbers $\C$ that is Galois over $\Q$ with Galois group (isomorphic to) the quaternion group $Q_8$ of order 8.
    Prove that $L$ contains a subfield $K$ with $[L:K] = 2$ such that $K = \Q(\sqrt{a},\sqrt{b})$, $a,b \in \Q$.
  \item
    Prove that $a$ and $b$ must both be positive.
  \end{enumerate}
  
  \begin{proof}
    \begin{enumerate}
    \item
      Let $G$ = $\Gal(L/\Q)$ and let $K$ be the field fixed by the subgroup of $G$ isomorphic to $\langle-1\rangle$, so that $[L:K] = 2$ by the Fundamental Theorem of Galois Theory.
      Then there are three subfields, $F_1, F_2$, and $F_3$, fixed by the subgroups isomorphic to $\langle i \rangle$, $\langle j \rangle$, $\langle k \rangle$, each of degree 2 over $\Q$ with $[K:F_i] = 2$ for each $i$.
      Since $\langle -1 \rangle$ is normal in $Q_8$, $K$ is Galois over $\Q$ with $\Gal(K/\Q) \cong V$, the Klein 4-group.
      Therefore by Homework 4, Exercise 4 (b), $K$ is a biquadratic extension of $\Q$.
    \item
      Now suppose to the contrary that $K = \Q(\sqrt{a}, \sqrt{b})$ were not a real field.
      Let $\phi = \phi\vert_L$ be complex conjugation restricted to $L$ and observe that $\phi \in \Gal(L/\Q)$ would be the only non-trivial automorphism of order 2.
      However, $K$ was supposed not to be a real field.
      Hence it is not fixed by $\langle\phi\rangle \cong \langle -1 \rangle$, which contradicts the result of (a).
      Therefore both $a$ and $b$ must be positive.
    \end{enumerate}
  \end{proof}
\end{thm}

\begin{thm}
  \label{Ex9}
  Suppose the prime $p$ does not divide the positive integer $n$.
  \begin{alphaenum}
  \item
    Prove that the finite field $\F_{p^m}$ contains a primitive $n^{\text{th}}$ root of unity if and only if $n$ divides $p^m - 1$.
  \item
    Explain why $(a)$ implies that the minimal polynomial over $\F_p$ for a primitive $n^{\text{th}}$ root of unity is of degree $f$, where $f$ is the smallest power of $p$ with $p^f \equiv 1 \pmod n$.
  \item
    Explain why the polynomial $f(x) = x^6 + x^5 + x^4 + x^3 + x^2 + x + 1$ is irreducible in $\Z[x]$ and is the product of 2 irreducible cubics in $\F_{67}$.
    Determine the first prime $p$ such that $f(x)$ is also irreducible modulo $p$.
  \end{alphaenum}
  \begin{proof}
    Let $K = \F_{p^m}$.
    \begin{alphaenum}
    \item
        By Lagrange's Theorem, the multiplicative group $G = (K)^{\times}$ contains an element of order $n$ if and only if $n$ divides $|G| = p^m - 1$.
        Therefore there exists an element $\zeta \in K$ such that $\zeta^n = 1$ if and only if $n$ divides $p^m - 1$.
      \item
        By part (a), for $\zeta_n$ a primitive $n^{\text{th}}$ root of unity, $\zeta_n \in K$ if and only if $n \equiv 1 \pmod p^m$.
        Consider the intermediate extension $\F_p(\zeta_n)$ of degree $f$ over $\F_p$.  Say $k = [K:F_p(\zeta_n)]$ so that $fk = m$ and thus $p^{fk} = p^m$ is the size of $K$.
        It now follows that $\zeta_n$ is an element of $K$ if and only if $k = 1$ and $p^{f} \equiv 1 \pmod n$.
        The minimality of such an $f$ follows directly from the minimality of $m_{\zeta_n,\F_p}$.
      \item
        The polynomial $f(x)$ is the seventh cyclotomic polynomial, which we have shown is irreducible over $\Z$.
        The smallest value of $f$ with $67^f \equiv 4^f \equiv 1 \pmod 7$ is $f = 3$, so the minimal polynomial of the seventh root of unity has degree 3.
        Hence $f$ must split into two irreducible cubics.
        The smallest prime for which this polynomial is irreducible is 3 as 6 is the smallest power of 3 such that $3^f \equiv 1 \pmod 7.$
      \end{alphaenum}
  \end{proof}
\end{thm}

\begin{thm}
  What grade do you feel you should receive for this class?
\end{thm}
I'd like to think I've earned an A.  Although, I have been a bit of a bonehead at times, so perhaps that's being a bit optimistic.\\


I certify that I have neither given nor received any assistance on this exam.

\vspace{.75in}

Blake Farman
\end{document}
