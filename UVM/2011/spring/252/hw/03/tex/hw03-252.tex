\documentclass[10pt]{amsart}
\usepackage{amsmath,amsthm,amssymb,amsfonts}
\openup 5pt
\author{Blake Farman}
\title{Math-252:\\Homework 3}
\date{February 16, 2010}\pdfpagewidth 8.5in
\usepackage[margin=1in]{geometry}
\pdfpageheight 11in
\begin{document}
\maketitle

\newcommand{\Z}{\mathbb{Z}}
\newcommand{\R}{\mathbb{R}}
\newcommand{\Q}{\mathbb{Q}}

\renewcommand{\qedsymbol}{\(\blacksquare\)}
\newcommand{\znz}[1]{\Z / #1\Z}
\newcommand{\mznz}[1]{(\Z / #1\Z)^*}

\renewcommand{\phi}{\varphi}
\newenvironment{alphaenum}{
  \begin{enumerate}
    \renewcommand{\theenumi}{(\alph{enumi})}
    \renewcommand{\labelenumi}{\theenumi}
  }
  {\end{enumerate}}

\newcommand{\quadeq}[3]{\frac{-(#2) \pm \sqrt{(#2)^2 - 4(#1)(#3)}}{2(#3)}}
\newcommand{\F}{\mathbb{F}}
\newtheorem{thm}{}

\begin{thm}
  \label{Ex1}
  Determine whether the following polynomials are irreducible in the rings indicated.  For those that are reducible, determine their factorization into irreducibles.
  \begin{alphaenum}
  \item 
    $x^2 + x + 1$ in $\F_2[x].$
  \item
    $x^3 + x + 1$ in $\F_3[x]$.
  \item
    $x^4 + 1$ in $\F_5[x].$
  \item
    $x^4 + 10x^2 + 1$ in $\Z[x]$.
  \end{alphaenum}
  
  \begin{proof}
    \begin{alphaenum}
    \item
      By Proposition 10, it suffices to check that neither 0 nor 1 are roots of the polynomial.  
      Therefore, $x^2 + x + 1$ is irreducible in $\F_2[x]$.
    \item
      In $\F_3[x]$, it is easy to check by hand that $x^3 + x + 1$ factors as $(x-1)(x^2 + x + 2).$
    \item
      In $\F_5[x]$, it is easy to check by hand that $x^4+1$ factors as $(x^2 + 2)(x^2-2)$.
    \item
      Observe first that because $x^4,x^2 \geq 0$, there are no integral solutions to $x^4 + 10x^2 + 1 = 0$.
      Hence there are neither linear nor cubic factors, so it remains only to show that there are no quadratic factors.
      Suppose $x^4 + 10x^2 + 1$ factors into two quadratics, 
      \begin{align*}
        \begin{split}
          x^4 + 10x^2 + 1 &= (x^2 + ax + b)(x^2 + cx + d)\\ &= x^4 + x^3(c+a)+x^2(b+d+ac) +x(ad+bc)+bd.
        \end{split}
      \end{align*}
      Then we have the following equations:  $c+a = 0,\, d+b+ac = 10,\, ad+bc=0$ and $bd = 1$.
      It follows from the latter equation that $b = d = \pm 1$.
      Now, if we take  $c = -a$, then $\pm 2 - 10 = a^2$.
      However, this leads to a contradiction; namely, $a^2 = \pm 2 - 10$ does not have a solution in the integers.
      Therefore $x^4 + 10x^2 + 1$ is irreducible in $\Z[x]$.
    \end{alphaenum}
  \end{proof}
\end{thm}

\begin{thm}
  \label{Ex2}
  Prove that the following polynomials are irreducible in $\Z[x]$:
  \begin{alphaenum}
  \item
    $x^4 - 4x^3 + 6$
  \item
    $x^6 + 30x^5 - 15x^3 + 6x - 120$
  \item
    $x^4 + 4x^3 + 6x^2 +2x + 1$
  \item
    $\frac{(x+2)^p - 2^p}{x}$, where $p$ is an odd prime.
  \end{alphaenum}
  \begin{proof}
    \begin{alphaenum}
    \item
      This polynomial is irreducible by Eisenstein's Criterion applied for the prime 2.
    \item
      This polynomial is irreducible by Eisenstein's Criterion applied for the prime 3.
    \item
      Substituting $x-1$ for $x$ and expanding, we have $$(x-1)^4 + 4(x-1)^3 + 6(x-1)^2 + 2(x-1)+1 = x^4 + 4x + 2,$$ which is irreducible by Eisenstein's Criterion applied for the prime 2.  
      Since any factorization of $x^4 + 4x^3 + 6x^2 +2x + 1$ would imply a factorization of $x^4 + 4x + 2$ by replacing $x$ by $x-1$ in each of the factors, $x^4 + 4x^3 + 6x^2 +2x + 1$ must be irreducible.
    \item
      Using the binomial theorem to expand $(x+2)^p$ we have 
      \begin{align*}
        \begin{split}
          \frac{(x+2)^p - 2^p}{x} &= \frac{1}{x}\left[\sum_{k=0}^{p-1}{p \choose k}x^{p-k}2^k + 2^p - 2^p\right]\\
          &= \sum_{k=0}^{p-2}{p \choose k}x^{p-k-1}2^p + p2^{p-1}.
        \end{split}
      \end{align*}
      The resulting polynomial is monic, with each coefficient divisble by $p$ and the constant term not divisible by $p^2$.
      Therefore, by Eisenstein's Criterion, $\frac{(x+2)^p - 2^p}{x}$ is irreducible when $p$ is an odd prime.
    \end{alphaenum}
    
  \end{proof}
\end{thm}

\begin{thm}
  \label{Ex3}
  Show that the polynomial $(x-1)(x-2)\ldots(x-n)-1$ is irreducible over $\Z$ for all $n \geq 1$.
  \begin{proof}
    Let $p(x) = (x-1)(x-2)\ldots(x-n)-1$.
    By Gauss' Lemma, it suffices to assume that if $p$ is reducible, then it factors over $\Z[x]$.
    Suppose $p = \alpha\cdot\beta$ for some $\alpha, \beta \in \Z[x]$ and note that $\deg(p) = \deg(\alpha) + \deg(\beta) = n.$
    Observe that for $1 \leq i \leq n$, we have $(\alpha\cdot\beta)(i) = -1$ and thus $\alpha(i) = \pm 1$ and $\beta(i) = -\alpha(i)$.
    If we now consider the polynomial $\alpha + \beta$ of degree at most $n$, then it follows from the observation above that for $1 \leq i \leq n$, $$\alpha(i)+\beta(i) = \alpha(i) - \alpha(i) = 0.$$
    These roots correspond, by Proposition 19, to $n$ linear factors and thus $\alpha + \beta$ has degree $n$.  
    Hence one of $\alpha$ or $\beta$ must have degree $n$.
    Moreover, since $\deg(\alpha) + \deg(\beta) = n$, the other must have degree 0.
    Indeed, because $p$ is monic, one of $\alpha$ or $\beta$ must be one of the two units, $1$ or $-1$.
    Therefore $p$ is irreducible.
    
    
  \end{proof}
\end{thm}

\begin{thm}
  \label{Ex4}
  Prove that the polynomial $x^2 - \sqrt{2}$ is irreducible over $\Z[\sqrt{2}]$.
  \begin{proof}
    Since $\sqrt{2}$ is irreducible in $\Z[\sqrt{2}]$, $x^2 - \sqrt{2}$ must also be irreducible as any root would provide a factorization of $\sqrt{2}$.
  \end{proof}
\end{thm}

\begin{thm}
  \label{Ex5}
  Factor each of the two polynomials: $x^8 -1$ and $x^6 - 1$ into irreducibles over each of the following rings:
    (a) $\Z$, (b) $\Z/2\Z$, (c) $\Z/3\Z$.
  \begin{proof}
    \begin{alphaenum}
    \item
      In $\Z$, the factorizations are 
      \begin{align*}
        \begin{split}
          x^8 - 1 &= (x^4 + 1)(x^2 + 1)(x+1)(x-1), \,\text{and}\\
          x^6 - 1 &= (x^2+x+1)(x^2-x+1)(x+1)(x-1).
        \end{split}
      \end{align*}
    \item
      In $\Z/2\Z$, the factorizations are 
      \begin{align*}
        x^8 - 1 = (x+1)^8, \quad \text{and} \quad  x^6 - 1 = (x + 1)^2(x^2 + x + 1)^2.
      \end{align*}
    \item
      In $\Z/3\Z$, the factorizations are 
      \begin{align*}
        \begin{split}
          x^8 - 1 &= (x^x+x+2)(x^2+2x+2)(x^2+1)(x+1)(x-1), \,\text{and}\\
          x^6 - 1 &= (x+1)^3(x+2)^3.
        \end{split}
      \end{align*}
    \end{alphaenum}
  \end{proof}
\end{thm}

\end{document}
