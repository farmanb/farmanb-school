\documentclass[10pt]{amsart}
\usepackage{amsmath,amsthm,amssymb,amsfonts,mymath}
\openup 5pt
\author{Blake Farman}
\title{Math-252:\\Homework 6}
\date{March 23, 2011}\pdfpagewidth 8.5in
\usepackage[margin=1in]{geometry}
\pdfpageheight 11in
\begin{document}
\maketitle

\newtheorem{thm}{}

\begin{thm}
  \label{Ex1}
  Prove that if $\lambda_1, \ldots \lambda_n$ are the eigenvalues of the $n \times n$ matrix $A$ then $\lambda_1^k, \ldots, \lambda_n^k$ are the eigenvalues of $A^k$ for any $k \geq 0$.
    \begin{proof}
    By assumption, $c_A(x) = (x - \lambda_1)\ldots(x - \lambda_n)$ and thus $m_A(x)$, which must divide $c_a(x)$, does not have repeated roots.
    Hence by Corollary 25, $A$ is diagonalizable and the Jordan canonical form is 
    $$
    \begin{pmatrix}
      \lambda_1  & & \text{\huge{0}}\\
       & \ddots & \\ 
      \text{\huge{0}} & & \lambda_n
    \end{pmatrix}.
    $$
    Therefore the Jordan canonical form of $A^k$ is 
    $$
    \begin{pmatrix}
      \lambda_1^k  & & \text{\huge{0}}\\
       & \ddots & \\ 
      \text{\huge{0}} & & \lambda_n^k
    \end{pmatrix}
    $$
    and the eigenvalues are $\lambda_1^k, \ldots, \lambda_n^k$.
  \end{proof}
\end{thm}

\begin{thm}
  \label{Ex2}
  Compute the Jordan canonical form for the matrix
  $$
  A =  \begin{pmatrix}
    1 & 0 & 0\\
    0 & 0 & -2\\
    0 & 1 & 3
  \end{pmatrix}.
  $$
  \begin{proof}
    The characteristic polynomial of $A$ is $c_A(x) = (x-1)^2(x-2)$ and the minimal polynomial is $m_A(x) = (x-1)(x-2)$, which does not have repeated roots.
    Therefore it follows from Corollary 25 that the Jordan canonical form for $A$ is 
    $$
    \begin{pmatrix}
      1 & 0 & 0\\
      0 & 1 & 0\\
      0 & 0 & 2
    \end{pmatrix}.
    $$
  \end{proof}
\end{thm}

\begin{thm}
  \label{Ex3}
  Determine which of the following matrices are similar:
  $$
  A = \begin{pmatrix}
    -1 & 4 & -4\\
    2 & -1 & 3\\
    0 & -4 & 3
  \end{pmatrix}
  \quad
  B = \begin{pmatrix}
    -3 & -4 & 0\\
    2 & 3 & 0\\
    8 & 8 & 1
  \end{pmatrix}
  \quad
  C = \begin{pmatrix}
    -3 & 2 & -4\\
    2 & 1 & 0\\
    3 & -1 & 3
  \end{pmatrix}
  \quad
  D = \begin{pmatrix}
    -1 & 4 & -4\\
    0 & -3 & 2\\
    0 & -4 & 3
  \end{pmatrix}.
  $$
  \begin{proof}
    The characteristic polynomials of the matrices are $c_A(x) = c_B(x) = c_C(x) = c_D(x) = (x+1)(x-1)^2$.
    The minimal polynomials are $m_A(x) = m_C(x) = m_D(x) = (x+1)(x-1)^2$ and $m_B(x) = (x+1)(x-1)$.
    The Jordan canonical forms for the similar matrices $A, C$ and $D$ are 
    $$
    \begin{pmatrix}
      -1 & 0 & 0\\
      0 & 1 & 1\\
      0 & 0 & 1\\
    \end{pmatrix}
    $$
    and the Jordan canonical form for $B$ is 
    $$
    \begin{pmatrix}
      -1 & 0 & 0\\
      0 & 1 & 0\\
      0 & 0 & 1
    \end{pmatrix}.
    $$
  \end{proof}
\end{thm}

\begin{thm}
  \label{Ex4}
  Prove that the matrices 
  $$
  A = \begin{pmatrix}
    -8 & -10 & -1\\
    7 & 9 & 1\\
    3 & 2 & 0
  \end{pmatrix}
  \quad
  B = \begin{pmatrix}
    -3 & 2 & -4\\
    4 & -1 & 4\\
    4 & -2 & 5
  \end{pmatrix}
  $$
  both have $(x-1)^2(x+1)$ as characteristic polynomial but that one can be diagonalized and the other cannot.
  Determine the Jordan canonical form for both matrices.
  \begin{proof}
    Computing $\det(xI - A)$ and $\det(xI - B)$ then factoring, we obtain the characteristic polynomials $c_A(x) = c_B(x) = (x-1)^2(x+1)$.
    The minimal polynomial of $A$ is $m_A(x) = c_A(x)$ and the minimal polynomial for $B$ is $m_B(x) = (x-1)(x+1)$.
    By Corollary 25, $B$ is diagonalizable, but $A$ is not.  
    Hence the matrices are not similar.
    The Jordan canonical forms for $A$ and $B$ are given, in that order, by
    $$
    \begin{pmatrix}
      -1 & 0 & 0\\
      0 & 1 & 1\\
      0 & 0 & -1
    \end{pmatrix}
    \quad \text{and} \quad
    \begin{pmatrix}
      1 & 0 & 0\\
      0 & 1 & 0\\
      0 & 0 & -1
    \end{pmatrix}.
    $$
  \end{proof}
\end{thm}

\begin{thm}
  \label{Ex5}
  Verify that the characteristic polynomial of 
  $$
  A = \begin{pmatrix}
    1 & 0 & 0 & 0\\
    0 & 1 & 0 & 0\\
    -2 & -2 & 0 & 1\\
    -2 & 0 & -1 & -2
  \end{pmatrix}
  $$
  is a product of linear factors over $\Q$.  Determine the rational and Jordan canonical forms for $A$ over $\Q$.
  \begin{proof}
    Working out the characteristic polynomial of $A$, we have $c_A(x) = (x-1)^2(x+1)^2$.
    Then it is easy enough to check that $m_a(x) = (x-1)(x+1)^2 = x^3 + x^2 - x - 1$.
    The invariant factors are $(x-1)(x+1)^2, x-1$ and the elementary divisors are $x-1, x-1, (x+1)^2$.
    The rational and Jordan canonical forms are given, in that order, by
    $$
    \begin{pmatrix}
      1 & 0 & 0 & 0\\
      0 & 0 & 0 & 1\\
      0 & 1 & 0 & 1\\
      0 & 0 & 1 & -1
    \end{pmatrix}
    \quad \text{and} \quad
    \begin{pmatrix}
      1 & 0 & 0 & 0\\
      0 & 1 & 0 & 0\\
      0 & 0 & -1 & 1\\
      0 & 0 & 0 & -1
    \end{pmatrix}.
    $$
  \end{proof}
\end{thm}

\begin{thm}
  \label{Ex6}
  Determine all possible Jordan canonical forms for a linear transofmration with characteristic polynomial $(x-2)^3(x-3)^2$.
  \begin{proof}
    We begin by determining the possible elementary divisors, which are as follows
    \begin{enumerate}
    \item
      $(x-2)^3, (x-3)^2$
    \item
      $(x-2)^3, (x-3), (x-3)$
    \item
      $(x-2)^2, (x-2), (x-3)^2$
    \item
      $(x-2)^2, (x-2), (x-3), (x-3)$
    \item
      $(x-2), (x-2), (x-2), (x-3)^2$
    \item
      $(x-2), (x-2), (x-2), (x-3), (x-3)$.
    \end{enumerate}
    The Jordan canonical forms for each are as follows
    \begin{enumerate}
    \item
      $
      \begin{pmatrix}
        2 & 1 & 0 & 0 & 0\\
        0 & 2 & 1 & 0 & 0\\
        0 & 0 & 2 & 0 & 0\\
        0 & 0 & 0 & 3 & 1\\
        0 & 0 & 0 & 0 & 3
      \end{pmatrix}
      $
    \item
      $
      \begin{pmatrix}
        2 & 1 & 0 & 0 & 0\\
        0 & 2 & 1 & 0 & 0\\
        0 & 0 & 2 & 0 & 0\\
        0 & 0 & 0 & 3 & 0\\
        0 & 0 & 0 & 0 & 3
      \end{pmatrix}
      $
    \item
      $
      \begin{pmatrix}
        2 & 1 & 0 & 0 & 0\\
        0 & 2 & 0 & 0 & 0\\
        0 & 0 & 2 & 0 & 0\\
        0 & 0 & 0 & 3 & 1\\
        0 & 0 & 0 & 0 & 3
      \end{pmatrix}
      $
    \item
      $
      \begin{pmatrix}
        2 & 1 & 0 & 0 & 0\\
        0 & 2 & 0 & 0 & 0\\
        0 & 0 & 2 & 0 & 0\\
        0 & 0 & 0 & 3 & 0\\
        0 & 0 & 0 & 0 & 3
      \end{pmatrix}
      $
    \item
      $
      \begin{pmatrix}
        2 & 0 & 0 & 0 & 0\\
        0 & 2 & 0 & 0 & 0\\
        0 & 0 & 2 & 0 & 0\\
        0 & 0 & 0 & 3 & 1\\
        0 & 0 & 0 & 0 & 3
      \end{pmatrix}
      $
    \item
      $
      \begin{pmatrix}
        2 & 0 & 0 & 0 & 0\\
        0 & 2 & 0 & 0 & 0\\
        0 & 0 & 2 & 0 & 0\\
        0 & 0 & 0 & 3 & 0\\
        0 & 0 & 0 & 0 & 3
      \end{pmatrix}.
      $
    \end{enumerate}
  \end{proof}
\end{thm}

\begin{thm}
  \label{Ex7}
  Show that if $A^2 = A$ then $A$ is similar to a diagonal matrix which has only 0's and 1's along the diagonal.
  \begin{proof}
    If $A^2 = A$, then $A^2 - A = 0$ and so $m_A(x) | x(x-1)$.
    Then $m_A(x)$ does not have any repeated roots and hence $A$ is diagonalizable by Corollary 25.
    Moreover, the only eigenvalues of $A$ are 0 and 1.
    Therefore $A$ must have 0's and 1's on the diagonal.
  \end{proof}
\end{thm}

\begin{thm}
  \label{Ex8}
  Prove that an $n \times n$ matrix $A$ with entries from $\C$ satisfying $A^3 = A$ can be diagonalized.
  Is the same statement true over {\it any} field $F$?
  \begin{proof}
    If $A^3 = A$, then $A^3 - A = 0$ and thus $m_A(x) | x^3 - x = x(x+1)(x-1)$.
    Then $m_A(x)$ does not have any repeated roots (over $\C$) and $A$ is diagonalizable by Corollary 25.

    This result does not hold for an arbitrary field, $F$.
    In particular, consider $\mathbb{F}_2$.
    Then $x^3 - x = x(x+1)^2$ and $m_A(x)$ may have repeated roots.
  \end{proof}
\end{thm}

\end{document}
