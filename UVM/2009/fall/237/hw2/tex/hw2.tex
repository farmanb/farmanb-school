\documentclass[12pt]{article}
\usepackage{amssymb}
\author{Blake Farman}
\title{Math-237: Homework 2}

\begin{document}
\maketitle
\newpage
\section*{3.1}
\subsection*{2}
Newton
\begin{eqnarray*}
P_n(x) & = & f[x_1] + f[x_1, x_2](x-1) + f[x_1, x_2, x_3](x-x_1)(x-x_2)\\
& + & f[x_1, x_2, x_3, x_4](x-x_1)(x-x_2)(x-x_3) + \ldots\\
& + & f[x_1, \ldots, x_n](x-x_1)\ldots(x-x_n)
\end{eqnarray*}

\begin{eqnarray*}
  f[x_1] & = & f(x_1)\\
  f[x_1, x_2] & = & \frac{f(x_2) - f(x_1)}{x_2 - x_1}\\
  f[x_1, x_2, x_3] & = & \frac{\frac{f(x_3) - f(x_2)}{x_3 - x_2} - f[x_1, x_2]}{x_3 - x_1}\\
  f[x_1, x_2, x_3] & = & \frac{f[x_3, x_4] - f[x_2, x_3]}{x_1-x_2}\\
  & = & \frac{\frac{\frac{f(x_4) - f(x_3)}{x_4 - x_3} - \frac{f(x_3) - f(x_2)}{x_3 - x_2}}{x_4-x_2} - f[x_1, x_2, x_3]}{x_4 - x_1}\\
\end{eqnarray*}

Lagrange

\[
L_k(x) = \frac{(x-x_1)\ldots(x-x_{k-1})(x-x_{k+1})\ldots(x-x_n)}
   {
     (x_k-x_1)
     \ldots
     (x_k-x_{k-1})
     (x_k-x_{k+1})
     \ldots
     (x_k-x_n)
   }\\
\]
\[P_{n-1}(x) = \sum_{i=1}^{n}y_iL_k(x)\]

\subsubsection*{a)}
\[(0,1), (2,3), (3,0)\]
\begin{eqnarray*}
  P_2(x) & = & f[x_1] + f[x_1, x_2](x-1) + f[x_1, x_2, x_3](x-x_1)(x-x_2)\\ 
  & = & 1 + \frac{2 - 1}{2}(x) + \frac{\frac{3-3}{2-0} - \frac{1}{2}}{3-0}(x)(x - 2)\\
  & = & 1 + \frac{1}{2}(x) - \frac{1}{2}(x)(x-2)
\end{eqnarray*}

\begin{eqnarray*}
  P_2(x) & = & \frac{(x-2)(x-3)}{(-2)(-3)} + \frac{-3(x)(x-3)}{(2)(2-3)}\\
  & = & \frac{(x-2)(x-3)}{6} - \frac{3(x)(x-3)}{2}\\
\end{eqnarray*}
\subsubsection*{b)}
\[(-1,0), (2,1), (3,1), (5,2)\]
\begin{eqnarray*}
  P_3(x) & = & f[x_1] + f[x_1, x_2](x-1) + f[x_1, x_2, x_3](x-x_1)(x-x_2) \\
  & + &f[x_1, x_2, x_3, x_4](x-x_1)(x-x_2)(x-x_3)\\
  & = & 0 + 
  \frac{1-0}{2 - (-1)}(x-(-1)) + 
  \frac{
    \frac{1 - 1}{3 - 2} -
    \frac{1}{3}}{3 - (-1)}(x-(-1))(x-2) + 
  \frac{\frac{\frac{2 - 1}{5-3} - 
      \frac{1 - 1}{3 - 2}}{5-2} +
    \frac{1}{12}}{5-(-1)}\\
  & = & \frac{1}{3}(x+1) - \frac{1}{12}(x+1)(x-2) + \frac{\frac{2}{12} - \frac{1}{12}}{6}(x+1)(x-2)(x-3)\\
  & = & \frac{1}{3}(x+1) - \frac{1}{12}(x+1)(x-2) + \frac{1}{6}(\frac{3}{12})(x+1)(x-2)(x-3)\\
  & = & \frac{1}{3}(x+1) - \frac{1}{12}(x+1)(x-2) + \frac{1}{24}(x+1)(x-2)(x-3)\\
\end{eqnarray*}

\begin{eqnarray*}
  P_3(x) & = & \frac{(x+1)(x-3)(x-5)}{3(-1)(-3)} + \frac{(x+1)(x-2)(x-5)}{4(1)(-2)} + \frac{2(x+1)(x-2)(x-3)}{6(3)(2)}\\
  & = & \frac{(x+1)(x-3)(x-5)}{9} - \frac{(x+1)(x-2)(x-5)}{8} + \frac{(x+1)(x-2)(x-3)}{18}\\
\end{eqnarray*}
\subsubsection*{c)}
\[(0, -2), (2,1), (4,4)\]
\begin{eqnarray*}
  P_2(x) & = & f[x_1] + f[x_1, x_2](x-1) + f[x_1, x_2, x_3](x-x_1)(x-x_2)\\
  & = & -2 + \frac{3}{2}(x - 0) + \frac{\frac{4-1}{4-2} - \frac{3}{2}}{6}(x-0)(x-2)\\
  & = & -2 + \frac{3}{2}(x)\\
\end{eqnarray*}

\subsection*{6}
(1,1), (2,3), (3,3), (4,4)

\begin{eqnarray*}
  P_3(x) & = & 1 + \frac{3-1}{2-1}(x-1) + \frac{\frac{3-3}{3-1} - 2}{3-1}(x-1)(x-2) + \frac{\frac{\frac{4-3}{4-3} - \frac{3-3}{3-2}}{4-2} + 1}{4-1}(x-1)(x-2)(x-3)\\
  & = & 1 + 2(x-1) - (x-1)(x-2) + \frac{\frac{1}{2} + 1}{3}(x-1)(x-2)(x-3)\\
  & = & 1 + 2(x-1) - (x-1)(x-2) + \frac{1}{2}(x-1)(x-2)(x-3)\\
  P_5(x) & = &  P_3(x) + c_1O(x^4) + c_2O(x^5), c_2 \neq 0
\end{eqnarray*}

\((x-1), (x-2), (x-3), (x-4)\) are all \(O(x)\)


\(\Rightarrow (x-1)(x-2)(x-3)(x-4)\) is \( O(x^4)\) and \((x-1)^2(x-2)(x-3)(x-4)\) is \(O(x^5)\). Thus,
\[P_5(x) = P_3(x) + c_1(x-1)(x-2)(x-3)(x-4) + c_2(x-1)^2(x-2)(x-3)(x-4), c_2 \neq 0\]

\subsection*{12}
Let \(P_4(x)\) and \(P_3(x)\) be fourth- and third-degree polynomials, respectively, such that
\begin{eqnarray*}
P_4(x) & = & a_0x^4 + a_1x^3 + a_2x^2 + a_3x + a_4\\
P_3(x) & = & b_0x^3 + b_1x^2 + b_2x + b_3\\
\end{eqnarray*}
where \[a_0 \neq 0, b_0 \neq 0\]\\

Then, to say that these two polynomials intersect is to say
\[P_4(x) = P_3(x)\]  
Or, equivalently, \[P_4(x) - P_3(x) = 0\]  From the definitions of \(P_4\) and \(P_3\), 
\[P_4(x) - P_3(x) = a_0x^4 + a_1x^3 + a_2x^2 + a_3x + a_4 - (b_0x^3 + b_1x^2 + b_2x + b_3)\]
where 
\[a_0 \neq 0, b_0 \neq 0\]
Rearranging terms yields
\[P_4(x) - P_3(x) = a_0x^4 + (a_1 - b_0)x^3 + (a_2 - b_1)x^2 + (a_3 - b_2)x + (a_4- b_3)\]
Since \(a_0 \neq 0\), \(P_4(x) - P_3(x)\) is a fourth-degree polynomial.  
It follows that since \(P_4(x) - P_3(x)\) is a fourth-degree polynomial, then it can have, at most, four roots.
Equivalently, \(P_4(x)\) and \(P_3(x)\) can intersect at, at most, four points.  
Therefore, it is not possible for a fourth-degree polynomial and a third-degree polynomial to intersect at five points.

\subsection*{14}
\subsubsection*{a)}
\[(40,75), (50,63), (60,54)\]
\begin{eqnarray*}
  P_2(x) & = & 75 + \frac{63-75}{50-40}(x-40) + \frac{\frac{54-63}{60-50} + \frac{12}{10}}{60-40}(x-40)(x-50)\\
  & = & 75 + \frac{-12}{10}(x-40) + \frac{\frac{-9}{10} + \frac{12}{10}}{20}(x-40)(x-50)\\
  & = & 75 + \frac{-12}{10}(x-40) + \frac{3}{10}\frac{1}{20}(x-40)(x-50)\\
  P_2(70) & = & 75 + \frac{-12}{10}(70-40) + \frac{\frac{3}{10}}{20}(70-40)(70-50)\\
  & = & 75 + \frac{-12}{10}(30) + \frac{3}{10}\frac{1}{20}(30)(20)\\
  & = & 75 -12(3) + (3)(3)\\
  & = & 75 -36 + 9\\
  & = & 48\\
\end{eqnarray*}

\subsubsection*{b)}
\[(25,95), (40,75), (50,63), (60,54)\]

\begin{eqnarray*}
  f[x_1] & = & 95\\
  f[x_1, x_2]  & = & \frac{75 - 95}{40-25}\\
  & = & \frac{-20}{15}\\
  & = & \frac{-4}{3}\\
  f[x_1, x_2, x_3] & = & \frac{\frac{63-75}{50-40} + \frac{4}{3}}{50-25}\\
  & = & (\frac{-12}{10} + \frac{4}{3})\frac{1}{25}\\
  & = & \frac{-(3^2)(2^2) + 2^3(5)}{5(2)(3)}(\frac{1}{5(5)})\\
  & = & \frac{-(3^2)(2) + (2^2)(5)}{(5)(3)}(\frac{1}{5(5)})\\
  & = & \frac{-18 + 20}{5^3(3)}\\
  f[x_1, x_2, x_3, x_4] & = & ((\frac{54-63}{60-50} - \frac{63-75}{50-40})(\frac{1}{60-40}) - \frac{2}{5^3(3)})(\frac{1}{60-25})\\
  & = & ((\frac{-9}{10} - \frac{-12}{10})(\frac{1}{20}) - \frac{2}{5^3(3)})(\frac{1}{35})\\
  & = & ((\frac{-3}{10}(\frac{1}{20}) - \frac{2}{5^3}{3})(\frac{1}{35})\\
  & = & (\frac{3}{20(10)} - \frac{2}{5^3(3)})(\frac{1}{7(5)})\\
  & = & \frac{3^2(5^3) - 2^4(5^2)}{2^3(5^6)(3)(7)}\\
  & = & \frac{3^2(5) - 2^4}{2^3(5^4)(3)(7)}\\
  P_3(x) & = & 95 - \frac{4}{3}(x-25) + \frac{-18 + 20}{5^3(3)}(x-25)(x-40)\\
  & + &\frac{3^2(5) - 2^4}{2^3(5^4)(3)(7)}(x-25)(x-40)(x-50)\\
  P_3(70) & = & 95 - \frac{4}{3}(70-25) + \frac{-18 + 20}{5^3(3)}(70-25)(70-40)\\
  & + &\frac{3^2(5) - 2^4}{2^3(5^4)(3)(7)}(70-25)(70-40)(70-50)\\
  & = & 95 - \frac{2^2}{3}(3^2(5)) + \frac{2}{5^3(3)}(3^2(5))(3(5)(2))\\
  & + &\frac{3^2(5) - 2^4}{2^3(5^4)(3)(7)}(3^2(5))(3(5)(2))(2^2(5))\\
  & = & 19(5) - \frac{2^2}{3}(3^2)(5) + \frac{2}{5^3(3)}(3^3)(5^2)(2)\\
  & + &\frac{3^2(5) - 2^4}{2^3(5^4)(3)(7)}(3^3)(5^3)(2^3)\\
  & = & 19(5) - (2^2)(3)(5) + \frac{2^2(3^2)}{5} + \frac{29}{(5)(7)}(3^2)\\
  & = & \frac{19(5^2)(7) - (2^2)(3)(5^2)(7) + (2^2)(3^2)(7) + 29(3^2)}{5(7)}\\
  & = & \frac{1738}{35} \approx 49.6571\\
\end{eqnarray*}

\section*{3.2}
\subsection*{2}
\subsubsection*{a)}
\[(1,0), (2,ln(2)), (4, ln(4))\]
\begin{eqnarray*}
  P_2(x) & = & 0 + \frac{ln(2) -0}{2-1}(x-1) + \frac{\frac{ln(4) - ln(2)}{4-2} - ln(2)}{4-1}(x-1)(x-2)\\
  & = & ln(2)(x-1) + \frac{\frac{log(\frac{4}{2})}{2} - ln(2)}{3}(x-1)(x-2)\\
  & = & ln(2)(x-1) + \frac{\frac{log(2)}{2} - ln(2)}{3}(x-1)(x-2)\\
  & = & ln(2)(x-1) - \frac{log(2)}{6}(x-1)(x-2)\\
  & = & ln(2)(x-1)(1 - \frac{(x-2)}{6})\\
\end{eqnarray*}
\subsubsection*{b)}
\begin{eqnarray*}
  P_2(3) & = & ln(2)(3-1)(1 - \frac{(3-2)}{6})\\
  & = & ln(2)(2)(1 - \frac{(1)}{6})\\
  & = & ln(2)(2)(\frac{5}{6})\\
  & = & ln(2)(\frac{5}{3})\\
\end{eqnarray*}
\subsubsection*{c)}
\begin{eqnarray*}
  f(x) & = & \ln(x)\\
  |f(x) - P_2(x)| & = & \left|\frac{(x-1)(x-2)(x-4)f^{'''}(c)}{3!}\right|\\
  f^{'}(x) & = & \frac{1}{x}\\
  f^{''}(x) & = & \frac{-1}{x^2}\\\\
  f^{'''}(x) & = & \frac{2}{x^3}\\\\
  |f(x) - P_2(x)| & \leq & \left|\frac{(x-1)(x-2)(x-4)f^{'''}(1)}{3!}\right|\\
  & = & \left|\frac{(x-1)(x-2)(x-3)(2)}{3!}\right|\\
  & = & \left|\frac{(x-1)(x-2)(x-3)}{3}\right|\\
  |f(3) - P_2(3)| & \leq & \left|\frac{(3-1)(3-2)(3-4)}{3}\right|\\
  & = & \frac{2}{3}\\
\end{eqnarray*}
\subsubsection*{d)}
\begin{eqnarray*}
  |\frac{5}{3}\ln(2) - \ln(3)| & = & |\ln(\frac{2^{\frac{5}{3}}}{3})|\\
  |2/3 - \ln(\frac{2^{\frac{5}{3}}}{3})| & \approx & .061\\
\end{eqnarray*}
\subsection*{4}
\(x = 2k, k \leq 5, k \in \mathbb{N}\)\\
\(n = 6\)
\begin{eqnarray*}
  f(x) & = & \frac{1}{x+5} = (x+5)^{-1}\\
  f^{'}(x) & = & \frac{1}{x+5} = -(x+5)^{-2}\\
  f^{''}(x) & = & \frac{1}{x+5} = 2!(x+5)^{-3}\\
  f^{'''}(x) & = & \frac{1}{x+5} = -3!(x+5)^{-4}\\
  f^{(iv)}(x) & = & \frac{1}{x+5} = 4!(x+5)^{-5}\\
  f^{(v)}(x) & = & \frac{1}{x+5} = -5!(x+5)^{-6}\\
  f^{(vi)}(x) & = & \frac{1}{x+5} = 6!(x+5)^{-7}\\
  |f(x) - P(x)| & = & |\frac{(x-x_1)(x-x_2)(x-x_3)(x-x_4)(x-x_5)(x-x_6)f^{vi}(c)}{6!}|\\
\end{eqnarray*}

\(x_i \in \mathbb{N} \Rightarrow x-x_i < x, \forall x_i\)\\


\begin{eqnarray*}
  \Rightarrow |f(x) - P(x)| & \leq & \left|\frac{(x^6)f^{(vi)}(c)}{6!}\right|\\
  & \leq & \left|\frac{x^6(6!)}{(6!)5^{7}}\right|\\
  & = & \left|\frac{x^6}{5^{7}}\right|\\
\end{eqnarray*}

\(f^{vi}(c)\) maximized at \(c = 0\)\\
\begin{eqnarray*}
  \Rightarrow |f(1) - P(1)| & \leq & \left|\frac{(x^6)6!}{6!(5^7)}\right|\\
  & = & \left|\frac{x^6}{5^{7}}\right|\\
\end{eqnarray*}
\subsubsection*{a)}
\(x = 1\)
\begin{eqnarray*}
  |f(1) - P(1)| & \leq & \left|\frac{1}{5^{7}}\right|\\
  & = & \frac{1}{5^7}\\
\end{eqnarray*}
\subsubsection*{b)}
\(x = 5\)
\begin{eqnarray*}
  |f(1) - P(1)| & \leq & \left|\frac{5^6}{5^{7}}\right|\\
  & = & \frac{1}{5}\\
\end{eqnarray*}

\section*{3.4}
\subsection*{4}
\begin{displaymath}
  S(x) = \left\{
  \begin{array}{lr}
    4 + k_1x + 2x^2 - \frac{1}{6}x^3 \\
    1 - \frac{4}{3}(x-1) + k_2(x-1)^2 - \frac{1}{6}(x-1)^3\\
    1 + k_3(x-2) + (x-2)^2 - \frac{1}{6}(x-2)^3\\
    \end{array}
  \right.
\end{displaymath}

\(S_i(x_{i+1}) = y_{i+1}\), \(i \in [1,3]\)
\begin{eqnarray*}
  &\Rightarrow & S_1(1) = 1,\\
  & & S_2(2) = 1\\
\end{eqnarray*}

\begin{eqnarray*}
  S_1(1) & = & 4 + k_1 + 2 - \frac{1}{6}\\
  & = & k_1 + 6 - \frac{1}{6}\\
  & = & k_1 + \frac{36}{6} - \frac{1}{6}\\
  & = & k_1 + \frac{35}{6}\\
  \frac{6}{6} & = & k_1 + \frac{35}{6}\\
  k_1 & = & \frac{-29}{6}\\
  S_2(2) & = & 1 -4 \frac{4}{3} + k_2 - \frac{1}{6}\\
  & = & 1 - \frac{8+1}{6} + k_2\\
  1 & = & 1 - \frac{8+1}{6} + k_2\\
  k_2 & = & \frac{3}{2}\\
\end{eqnarray*}

\begin{eqnarray*}
  S_2^{'}(2) & = & S_3^{'}(2)\\
  S_2^{'}(x) & = & \frac{-4}{3} + 2k_2(x-1) - \frac{1}{2}(x-1)^2\\
  S_3^{'}(x) & = & k_3 + 2(x-2) - \frac{1}{2}(x-2)^2\\
  S_2^{'}(2) & = & \frac{-4}{3} + 3 - \frac{1}{2}\\
  & = & \frac{-8 + 18 - 3}{6}\\
  & = & \frac{7}{6}\\
  S_3^{'}(2) & = & k_3\\
  \Rightarrow k_3 & = & \frac{7}{6}\\
\end{eqnarray*}

\begin{displaymath}
  k = \left[
    \begin{array}{lr}
      \frac{-29}{6}\\
      \frac{3}{2}\\
      \frac{7}{6}\\
    \end{array}
    \right]
\end{displaymath}

\begin{eqnarray*}
  S_1^{'}(x) & = & k_1 + 4x - \frac{1}{2}x^2\\
  S_1^{''}(x) & = & 4 - x\\
  S_1^{''}(0) & \neq & 0\\
\end{eqnarray*}
\(\Rightarrow\) not a natural spline.\\

\begin{eqnarray*}
  S_1^{'''}(x) & = & -1\\
  S_3^{''}(x) & = & 2 - (x-2)\\
  S_3^{'''}(x) & = & - 1\\
\end{eqnarray*}

\(\Rightarrow\) not-a-knot.\\
\subsection*{8}
\subsubsection*{a)}
\((0,1), (2,3), (3,2)\)\\

\[
\begin{array}{ll}
  a_1 = 1, & a_2 = 3\\
  \delta_1 = 2, &  \delta_2 = 1\\
  \Delta_1 = 2, & \Delta_2 = -1\\
\end{array}
\]

\[
\left[
\begin{array}{ccc}
  1 & 0 & 0\\
  \delta_1 & 2(\delta_1 + \delta_2) & \delta_2\\
  0 & 0 & 1\\
\end{array}
\right]
\left[
\begin{array}{c}
  c_1\\
  c_2\\
  c_3\\
\end{array}
\right]
=
\left[
\begin{array}{c}
  0\\
  3(\frac{\Delta_2}{\delta_2} - \frac{\Delta_1}{\delta_1})\\
  0\\
\end{array}
\right]
\]

\[
\left[
\begin{array}{ccc}
  1 & 0 & 0\\
  2 & 2(2 + 1) & 1\\
  0 & 0 & 1\\
\end{array}
\right]
\left[
\begin{array}{c}
  c_1\\
  c_2\\
  c_3\\
\end{array}
\right]
=
\left[
\begin{array}{c}
  0\\
  3(\frac{-1}{1} - \frac{2}{2})\\
  0\\
\end{array}
\right]
=
\left[
\begin{array}{c}
  0\\
  -6\\
  0\\
\end{array}
\right]
\]
\(\Rightarrow c_1 = c_3 = 0\), \(6c_2 = -6\)\\
\(\Rightarrow c_2 = -1\)\\

\[
\begin{array}{cc}
  d_1 = \frac{c_2 - c_1}{3(\delta_1)}, & b_1 = \frac{\Delta_1}{\delta_1} - (c_1)(\delta_1) - (d_1)(\delta_1)^2\\
  d_2 = \frac{c_3 - c_2}{3(\delta_2)}, & b_2 = \frac{\Delta_2}{\delta_2} - (c_2)(\delta_2) - (d_2)(\delta_2)^2\\
\end{array}
\]

\[
\begin{array}{cc}
  d_1 = \frac{-1 - 0}{3(2)}, & b_1 = \frac{2}{2} - (0)(2) - (d_1)(2)^2\\
  d_2 = \frac{0 - -1}{3(1)}, & b_2 = \frac{-1}{1} - (-1)(1) - (d_2)1^2\\
\end{array}
\]

\[
\Rightarrow
\begin{array}{cc}
  d_1 = \frac{-1}{6}, & b_1 = 1 - \frac{-2}{3} = \frac{5}{3}\\
  d_2 = \frac{1}{3}, & b_2 = - \frac{1}{3}\\
\end{array}
\]

\begin{displaymath}
  S(x) = \left\{
  \begin{array}{c}
    1 + \frac{5}{3}(x) - \frac{1}{6}(x)^3\\
    3 - \frac{1}{3}(x-2) -(x-2)^2 + \frac{1}{3}(x-2)^3\\
    \end{array}
  \right.
\end{displaymath}

\subsubsection*{b)}
\((0,0),(1,1),(2,6)\)\\
\[
\begin{array}{ll}
  a_1 = 0, & a_2 = 1\\
  \delta_1 = 1, &  \delta_2 = 1\\
  \Delta_1 = 1, & \Delta_2 = 5\\
\end{array}
\]

\[
\left[
\begin{array}{ccc}
  1 & 0 & 0\\
  1 & 4 & 1\\
  0 & 0 & 1\\
\end{array}
\right]
\left[
\begin{array}{c}
  0\\
  3\\
  0\\
\end{array}
\right]
=
\left[
\begin{array}{c}
  0\\
  12\\
  0\\
\end{array}
\right]
\]

\[
\begin{array}{cc}
  d_1 = 1, & b_1 =  0\\
  d_2 = -1, & b_2 = 3\\
\end{array}
\]

\begin{displaymath}
  S(x) = \left\{
  \begin{array}{c}
    x^3\\
    1 + 3(x-1) + 3(x-1)^2  - (x-1)^3\\
    \end{array}
  \right.
\end{displaymath}
\end{document}
